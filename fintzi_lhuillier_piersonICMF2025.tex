\documentclass[12pt,a4paper]{article}
\usepackage[T1]{fontenc}
\usepackage{times}
\usepackage{graphicx}
\usepackage{xcolor}
\usepackage{fancyhdr}
\usepackage{natbib}
\usepackage{amssymb}
\usepackage{amsmath}
\usepackage{amsthm}
\usepackage{mathrsfs}
\usepackage{empheq}
\usepackage{bm}
\newcommand{\size}{0.22\textwidth}
\newcommand{\avg}[1]{\left<#1\right>}
\renewcommand{\avg}[1]{\left<#1\right>}
\newcommand{\Exp}[1]{\overline{\overline{#1}}}
\newcommand{\davg}[1]{\left<#1\right>_d}
\newcommand{\cavg}[1]{\left<#1\right>_c}
\newcommand{\pavg}[1]{\avg{\delta_\alpha #1}}
% \newcommand{\pnavg}[1]{n\left<#1\right>_p}

\newcommand{\avgcond}[1]{\left<#1\right>}
\renewcommand{\avgcond}[1]{\overline{#1}}
\newcommand{\condavg}[2]{\overline{#1}^{#2}}
\newcommand{\ravg}[1]{\avgcond{#1}^\textbf{r}}
\newcommand{\Tavg}[1]{\avgcond{#1}^T}
\newcommand{\Xavg}[1]{\avgcond{#1}^X}
\newcommand{\TXavg}[1]{\Tavg{\Xavg{#1}}}
\newcommand{\kavg}[1]{\avgcond{#1}^k}
\newcommand{\Iavg}[1]{\avgcond{#1}^I}
\newcommand{\pnnavg}[1]{\avgcond{#1}^{p}}
\newcommand{\pnavg}[1]{n_p\pnnavg{#1}}
\newcommand{\oneavg}[1]{\avgcond{#1}^1}
\newcommand{\twoavg}[1]{\avgcond{#1}^2}
\newcommand{\smallavg}[2]{\avgcond{#1}^{#2}}
\newcommand{\sym}[1]{\text{Sym}\left[#1\right]}

\newcommand{\nstavg}[1]{\overline{#1}^{nst}}
\newcommand{\nstrelavg}[1]{\overline{#1}^{nst}_{rel}}
\newcommand{\mavg}[1]{\left<#1\right>_m}
\newcommand{\gavg}[2][\gamma]{\left<#2\right>_{#1}}
\newcommand{\partials}[1]{\partial_{i_1}\partial_{i_2}\ldots\partial{i_{#1}}}
\newcommand{\partialp}[2]{ \prod_{m=#1}^{#2} \partial_{i_m}}
\newcommand{\hatpartialp}[2]{ \prod_{m=#1}^{#2} \hat{\partial}_{j_m}}
\newcommand{\hatpartialpi}[2]{ \prod_{m=#1}^{#2} \hat{\partial}_{i_m}}
\newcommand{\pri}[2]{ \prod_{m=#1}^{#2} r_{i_m}}
\newcommand{\prj}[2]{ \prod_{m=#1}^{#2} r_{j_m}}

\newcommand{\grad}{\mathbf{\nabla}}
\renewcommand{\div}{\mathbf{\nabla}\cdot}
\newcommand{\gradI}{\mathbf{\nabla}_{||}}
\newcommand{\divI}{\mathbf{\nabla}_{||}\cdot}

\newcommand{\ddt}{\frac{d}{dt}}
% \renewcommand{\ddt}{d_t}
\newcommand{\pddt}{\frac{\partial}{\partial t}}
\newcommand{\Dt}{D_t}
\newcommand{\pddx}{\frac{\partial}{\partial \textbf{x}}}
\newcommand{\pddr}{\frac{\partial}{\partial \textbf{r}}}
\newcommand{\pddy}{\frac{\partial}{\partial \textbf{y}}}
\newcommand{\pddw}{\frac{\partial}{\partial \textbf{w}}}
\renewcommand{\pddx}{{\partial_\textbf{x}}}
\renewcommand{\pddr}{{\partial_\textbf{r}}}
\renewcommand{\pddy}{{\partial_\textbf{y}}}
\renewcommand{\pddw}{{\partial_\textbf{w}}}
\newcommand{\pdda}{\partial_a}
\renewcommand{\pddt}{\partial_t}
\newcommand{\norm}[1]{\hat{#1}}
\newcommand{\Jump}[1]{\llbracket #1 \rrbracket \cdot \textbf{n}_k }
\renewcommand{\Jump}[1]{\sum_{k=d,f} \left[#1\right] \cdot \textbf{n}_k }

\newcommand{\intO}[1]{\int_{V_\alpha} #1 dV}
% \renewcommand{\intO}[1]{ ( #1 )_{\Omega}}
\newcommand{\intS}[1]{\oint_{S_\alpha} #1 dS}
% \renewcommand{\intS}[1]{ ( #1 )_{\Sigma}}
\newcommand{\pOavg}[1]{\pavg{\intO{#1}}}
\newcommand{\pSavg}[1]{\pavg{\intS{#1}}}
\newcommand{\pMavg}[1]{\mathscr{M}\!\!\left[#1\right]}
\newcommand{\pMOavg}[1]{\mathscr{M}_\Omega\!\!\left[#1\right]}
\newcommand{\pMSavg}[1]{\mathscr{M}_\Sigma\!\!\left[#1\right]}
\newcommand{\CC}{\mathscr{C}}
\newcommand{\PP}{\mathscr{P}}
\newcommand{\FF}{\mathscr{F}}

%%% Utiliser pour les commentaires
\newcommand{\JL}[1]{\color{red}#1\color{black}}
\newcommand{\DL}[1]{\color{green}#1\color{black}}
\newcommand{\tb}[1]{\color{blue}#1\color{black}}
% \renewcommand{\alpha}{}
% \renewcommand{\tb}[1]{}

\renewcommand{\size}[1]{0.3\textwidth}
\newcommand{\expo}[2][n]{\frac{(-1)^#1}{#1!} \partialp{1}{#1} \pavg{\int_{\Omega_\alpha} \pri{1}{#1}#2 d\Omega}}
\newcommand{\expoU}[2][n]{\frac{(-1)^#1}{#1!} \partialp{1}{#1} \pavg{\textbf{u}_\alpha\int_{\Omega_\alpha} \pri{1}{#1}#2 d\Omega}}
\newcommand{\expoS}[2][n]{\frac{(-1)^#1}{#1!} \partialp{1}{#1} \pavg{\int_{\Gamma_\alpha} \pri{1}{#1}#2 d\Sigma}}

% \newcommand{\numref}[1]{\ref{#1}}
% \renewcommand{\ref}[1]{\autoref{#1}}

% FORMAT OF PAGE
\addtolength{\textheight}{4.5cm}
\addtolength{\topmargin}{-1.5cm}
\addtolength{\footskip}{0cm}
\addtolength{\textwidth}{5cm}
\addtolength{\evensidemargin}{-2.5cm}
\addtolength{\oddsidemargin}{-2.5cm}

\definecolor{mygray}{gray}{0.6}
\renewcommand\refname{\textbf{\large References}}


\begin{document}

\pagestyle{fancy}
\fancyhf{}

\lhead{\textcolor{mygray}{12th International Conference on Multiphase flow}}
\rhead{\textcolor{mygray}{ICMF 2025, Toulouse, France, May 12-16, 2025}}
\lfoot{}
\cfoot{}
\rfoot{}

% TITLE IN CAPITAL LETTERS
\begin{center}
%{\large {\bf On the Pseudo Turbulent Kinetic Energy (PTKE) equation for disperse multiphase flow made of arbitrary fluid droplets}}
{\large {\bf Pseudo Turbulent Kinetic Energy (PTKE) equation for disperse multiphase flow: closures in the Stokes regime}}

\vspace{10pt}

% AUTHORS

\underline{Nicolas Fintzi}$^1$, Jean-Lou Pierson$^1$\\
% AFFILIATIONS
{\it
$^1$IFP Energies Nouvelles, Solaize, 69360, France,jean-lou.pierson@ifpen.fr\\
%$^2$Affiliation of second author: University, School, Department, Institute, City, Country, E-mail\\
%$^3$Affiliation of third author: University, School, Department, Institute, City, Country, E-mail\\
}
\end{center}

\vspace{10pt}
\noindent{\bf {\large Abstract}}:\\
    %Buoyancy-driven droplet flows are encountered in many chemical engineering processes such as gravity separators and liquid-liquid extractors. 
    The usual engineering practice to model such chemical engineering processes is to use the averaged Navier-Stokes equations, that is: mass and momentum transport equations for the dispersed and continuous phase. 
    These equations are often completed by a \textit{pseudo turbulent kinetic energy} (PTKE) equation for the continuous phase. 
    % The PTKE is defined as the average of the dot product of the local velocity fluctuations within the continuous phase : namely, $\phi_f k_f =\frac{1}{2}\avg{\chi_f \textbf{u}_f'\cdot \textbf{u}_f'}$, where $\chi_f$ is the continuous phase indicator function and $\phi_f$ the continuous phase volume fraction. 
    Numerous authors derived PTKE equations using what we call a \textit{two-fluid} formulation \citep{kataoka1986local}. 
    However, none of these authors took advantage of the topology of the dispersed phase, which becomes problematic for the modeling of the exchange terms of the PTKE equaiton. 
    Therefore, inspired by the Lagrangian-based modeling introduced by \citep{buyevich1979flow, lhuillier1992volume}, we propose to take advantage of the dispersed nature of the inclusions and derive a PTKE equation considering Newtonian fluid particles, we call it the \textit{hybrid formulation} of the PTKE equation. 
    This new formulation enables us to better understand the physical significance of the interphase exchange terms. 
    Specifically, we identify the source term appearing in the PTKE equation that is due uniquely to the consideration of the droplet's internal motions.
    Then it is demonstrated that the consideration of hill vortex motions within the droplets drastically impacts the total kinetic energy exchange between the droplets and continuous phases.
    This conclusion as well as the general \textit{hybrid formulation} of the PTKE equation constitutes the major result of this work, as it extends our current understanding of the PTKE, and includes seemingly non-negligible exchange terms that are related to the fluid nature of the particles.
    \vspace{10pt}

\noindent{{\bf Keywords}}: PTKE equations, hybrid model, Averaged equations, dispersed multiphase flows

\vspace{10pt}
\noindent{\bf {\large Hybrid formulation of the PTKE equaiton}}:

Let us consider a dispersed two-phases flow made of two immiscible fluids of different densities and viscosities separated by an interface with constant surface tension.
In this context, the PTKE equations can be derived within a classic \textit{two-fluid} formulation and reads as, 
\begin{align}
    \pddt (\phi_f\rho_fk_f)  
    + \div (
        \phi_f\rho_fk_f\textbf{u}_f
        + \textbf{q}_{eq}^\text{k} 
        )
    &= 
    - \avg{\chi_f\bm{\sigma}_f^0 : \grad \textbf{u}_f^0}
    - \bm{\sigma}_f^\text{eq} : \grad \textbf{u}_f
    + \avg{\delta_\Gamma \textbf{u}_f' \cdot \bm{\sigma}_f^0 \cdot \textbf{n}_f},
    \label{eq:PTKE}
    \\
    \bm{\sigma}_k^\text{eq}
    &= 
     \rho_k\avg{\chi_k \textbf{u}_k'\textbf{u}_k'}
      - \phi_k \bm{\sigma}_k \nonumber
\end{align}
where we introduced: $\avg{\ldots}$ as an ensemble average procedure; $\chi_f$ as the fluid phase indicator function; $k_f$ as the PTKE; the tensor $\bm\sigma_f^0$ and $\bm\sigma_f$ as the local and averaged fluid phase stress; the vector $\textbf{u}_f^0$ and $\textbf{u}_f$ represent the local and averaged continuous phase velocities; $\delta_\Gamma$ is the interface indicator function; $\textbf{u}_f' = \textbf{u}_f^0 - \textbf{u}_f$ is the fluctuating velocity; and $\textbf{q}_{eq}^\text{k}$ represents the effective heat flux. 
In light of \eqref{eq:PTKE} we observe that $k_f$ results from an equilibrium between what we call the equivalent or macroscopic dissipation term $\bm{\sigma}_f^\text{eq} : \grad \textbf{u}_f$, the average of the local continuous phase dissipation $\avg{\chi_f\bm{\sigma}_f^0 : \grad \textbf{u}_f^0}$, and the interphase energy transfer term: $\avg{\delta_\Gamma \textbf{u}_f' \cdot \bm{\sigma}_f^0 \cdot \textbf{n}_f}$. 
It is important to highlight that under this form the latter term is not easily interpretable, since it involves the local velocity at every point on the interface. 

% \textit{two-fluid} model fails to adequately distinguish between the two phases, as evidenced by the \textit{symmetry} of the exchange term $\avg{\delta_\Gamma \textbf{u}_f' \cdot \bm{\sigma}_f^0 \cdot \textbf{n}_f}$ which has the same form for the dispersed phase PTKE equation. 
% This symmetry does not hold physically because the dispersed phase possesses a distinct topological nature compared to the continuous phase. 
% In a dispersed two-phase flow system, the closure terms are typically expressed as functions of the Lagrangian properties of the particles. 
% In contrast, the current system of equations yields continuously averaged quantities, which are not directly connected to the Lagrangian properties.

Upon noticing that the interface-exchange term present in \eqref{eq:PTKE} can be expanded in a Taylor series around the center of mass of the particles, we arrive at the major result of this work, which is the \textit{hybrid} formulation of PTKE equation for arbitrarily shaped fluid particles, namely
\begin{align}
    \pddt (\phi_f\rho_fk_f)  
    + \div (
        \phi_f\rho_fk_f\textbf{u}_f
        + \textbf{q}_f^\text{k-h} 
        )
    = 
    (\textbf{u}_f - \textbf{u}_p)\cdot \pSavg{{\bm{\sigma}_f^0 \cdot \textbf{n}}} 
    - \pavg{ \textbf{u}_\alpha' \cdot \intS{  \bm{\sigma}_f^0 \cdot \textbf{n}}}\nonumber\\
    - \pavg{ \intS{\textbf{w}^0 \cdot \bm{\sigma}_f^0 \cdot \textbf{n}}},
    - \avg{\chi_f\bm{\sigma}_f^0 : \grad \textbf{u}_f^0}
    - \bm{\sigma}_f^\text{eq} : \grad \textbf{u}_f
    \label{eq:PTKE2}
\end{align}
where we have noted that $\textbf{u}_d^0 = \textbf{u}_p + \textbf{u}_\alpha' +\textbf{w}_d^0$, with $\textbf{u}_p$ the average of the particle center of mass velocity, $\textbf{u}_\alpha' = \textbf{u}_\alpha - \textbf{u}_p$ the center of mass velocity of the particle $\alpha$ relative to $\textbf{u}_p$, and $\textbf{w}_\alpha = \textbf{u}_d^0 - \textbf{u}_\alpha$ the velocity within the particle volume and surface relative to its center of mass velocity. 
Under this form the contribution of the kinetic energy exchange is explicit. 
The first term on the right-hand side of \eqref{eq:PTKE2} represents the work done by the mean particle-phase motion with the mean drag force.
The second term represents the covariance of the velocity of the particles with their respective drag forces.
The last term represents the work made by the local force traction on the particle surface with the velocity at the surface of the particles $\textbf{w}_d^0$.

While the first two terms on the right-hands side of \eqref{eq:PTKE2} have already been identified in several studies, the first term of the second line appears to be new when expressed in such generality.
We claim that the latter term results from particle internal motions, which can be either due to the particles rotation, or what is of interest here, the droplet's fluid internal circulation. 

\vspace{10pt}

\noindent{\bf {\large Impact of the droplets internal motion on the exchange terms  }}:

Let us consider spherical droplets of radius $a$ and viscosity $\mu_f \lambda$, and let us assume that the particle internal motions are determined uniquely by the continuous and particle phase mean relative motions, that is: the mean relative velocity $\textbf{u}_f - \textbf{u}_p$, and mean gradient of the velocity $\textbf{E}_f = \grad \textbf{u}_f+ (\grad \textbf{u}_f)^\dagger$ at the particle position. 
In this situation, and based on the singularity solution of isolated droplets in Stokes flow, we can express $\textbf{w}_d^0$ as a function of $\textbf{u}_f$, $\textbf{u}_p$ and $\textbf{E}_f$ and show that, 
\begin{align}
    \pSavg{\textbf{w}_2^0 \cdot \bm{\sigma}_1^0\cdot\textbf{n}_2}
    &\approx
    \frac{1}{\lambda+1}(\textbf{u}_{f} - \textbf{u}_p) \cdot \left[
        -\frac{1}{2}\pSavg{ \bm{\sigma}_1^0\cdot\textbf{n}_2}
        % + \pavg{\textbf{u}_{\alpha}' \cdot \intS{ \bm{\sigma}_1^0\cdot\textbf{n}_2} }
        % + \frac{1}{2a^2}
        % \pSavg{\textbf{rr}\cdot \bm{\sigma}_1^0\cdot\textbf{n}_2}
    \right]\nonumber
    % \\
    &+ \frac{1}{\lambda+1} \left[
        -\frac{1}{2}
        \pavg{\textbf{u}_\alpha' \cdot  \intS{\bm{\sigma}_1^0\cdot\textbf{n}_2}}
        % + \pavg{\textbf{u}_{\alpha}' \cdot \intS{ \bm{\sigma}_1^0\cdot\textbf{n}_2} }
        % + \frac{1}{2a^2}
        % \pavg{\textbf{u}_\alpha' \cdot \intS{\textbf{rr}\cdot \bm{\sigma}_1^0\cdot\textbf{n}_2}}
    \right] \nonumber
    \\
    % \left[
    %     \textbf{u}_{p f} \cdot
    %     +
        % \pavg{\textbf{u}_{\alpha}' \cdot \intS{\textbf{rr}\cdot \bm{\sigma}_1^0\cdot\textbf{n}_2}}
    % \right]
    &+ \frac{1}{\lambda + 1} \textbf{E}_{f} : \left[ 
     \pSavg{\textbf{r} \bm{\sigma}_1^0\cdot\textbf{n}_2}
    %  -\frac{1}{a^2} 
    %  \pSavg{ \textbf{rrr} \cdot \bm{\sigma}_1^0\cdot\textbf{n}_2}
    \right]
    \label{eq:energy_term}
\end{align} 
Note how the terms present in this equation shear similarities with the first two terms on the RHS of \eqref{eq:PTKE2}. 
In fact by injecting \eqref{eq:energy_term} in \eqref{eq:PTKE2} we observe that the coefficient in front of the drag force velocity terms, which is originally $1$, changes to $\frac{\lambda +\frac{1}{2}}{\lambda+1}$, and that a new term proportional to $\textbf{E}_f$ appears. 
Consequently, the consideration of hill's vortex ends up adding a coefficient in front of the exchange term which varies from $1$ (for $\lambda = \infty$) to $1/2$ for ($\lambda = 0$), respectively.  
Additionally, the last term of \eqref{eq:energy_term} implies that even neutrally buoyant droplets can generate PTKE due to their simple presence in a linear flow, and that the PTKE production is proportional to $\textbf{E}_f$ and a quantity related to the first moment of the hydrodynamic forces.  

Anyhow, the consideration of the internal motion of particles have a very significant impact regarding the magnitude of the pseudo turbulent exchange terms in \eqref{eq:PTKE2}, especially when one is considering bubbly flow ($\lambda = 0$). 
The physical explanation of the decrease of the coefficient in front of the exchange terms for bubbles can be due to the facts that the fluid slip on the bubbles or droplet's surface induce less work exchange than if the fluid followed exactly the particle's surface as it is the case for solid particles. 


Overall, based on solid theoretical ground, we demonstrated that it is necessary to well take in account the fluid nature of the particle phase in the PTKE equation to model accurately the exchange term.

 

\bibliographystyle{abbrv}
\bibliography{Bib/bib_bulles.bib}
\end{document}


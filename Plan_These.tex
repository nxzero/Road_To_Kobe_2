\documentclass[12pt]{book}

\usepackage{graphicx}
\usepackage{tikz}
\usepackage{amssymb}
\usepackage{amsmath}
\usepackage{amsthm}
\usepackage{mathrsfs}
\usepackage{empheq}
% \usepackage{mdframed}
\usepackage{bm}
\usepackage[
  colorlinks=true
]{hyperref}
\hypersetup{
    colorlinks=true,
    linkcolor=blue!50!black,
    filecolor=blue!50!black,
    citecolor = green!50!black,
    urlcolor=cyan,
}
\usepackage{bm}
\usepackage{natbib}
\bibliographystyle{apalike}

\usepackage{url}
\usepackage{authblk}
\renewcommand\Affilfont{\itshape\small}

\usepackage{stmaryrd}
\AtBeginDocument{\renewcommand{\ref}[1]{\autoref{#1}}}

\usepackage[
    a4paper,
    left=20mm,
    right=20mm,
    top=20mm
]{geometry}
%%%%%%%%%%%%%%%%%%%%%%%%%%%%%%%%%%%%%%%%%%%%%%%%%%%%%%%%%%%%%%%%%%%%%%%%%%%%%%%
\newcommand{\size}{0.22\textwidth}
\newcommand{\avg}[1]{\left<#1\right>}
\renewcommand{\avg}[1]{\left<#1\right>}
\newcommand{\condavg}[1]{\left<#1 | \mathscr{C}_1\right>}
\newcommand{\Exp}[1]{\overline{\overline{#1}}}
\newcommand{\davg}[1]{\left<#1\right>_d}
\newcommand{\cavg}[1]{\left<#1\right>_c}
\newcommand{\Iavg}[1]{\left<#1\right>_I}
\newcommand{\pavg}[1]{\avg{\delta_\alpha #1}}
% \newcommand{\pnavg}[1]{n\left<#1\right>_p}

\newcommand{\avgcond}[1]{\left<#1\right>}
\renewcommand{\avgcond}[1]{\overline{#1}}
\newcommand{\kavg}[1]{\avgcond{#1}^k}
\newcommand{\pnnavg}[1]{\avgcond{#1}^{p}}
\newcommand{\pnavg}[1]{n_p\pnnavg{#1}}
\newcommand{\oneavg}[1]{\avgcond{#1}^1}
\newcommand{\smallavg}[2]{\avgcond{#1}^{#2}}
\newcommand{\sym}[1]{\left(#1\right)^{\text{Sym}}}

\newcommand{\nstavg}[1]{\overline{#1}_{nst}}
\newcommand{\nstrelavg}[1]{\overline{#1}_{nst}^{rel}}
\newcommand{\mavg}[1]{\left<#1\right>_m}
\newcommand{\gavg}[2][\gamma]{\left<#2\right>_{#1}}
\newcommand{\partials}[1]{\partial_{i_1}\partial_{i_2}\ldots\partial{i_{#1}}}
\newcommand{\partialp}[2]{ \prod_{m=#1}^{#2} \partial_{i_m}}
\newcommand{\hatpartialp}[2]{ \prod_{m=#1}^{#2} \hat{\partial}_{j_m}}
\newcommand{\hatpartialpi}[2]{ \prod_{m=#1}^{#2} \hat{\partial}_{i_m}}
\newcommand{\pri}[2]{ \prod_{m=#1}^{#2} r_{i_m}}
\newcommand{\prj}[2]{ \prod_{m=#1}^{#2} r_{j_m}}
\newcommand{\nablab}{\mathbf{\nabla}}
\newcommand{\nablabh}{\nablab}
\newcommand{\nablabhI}{\nablab_{||}}
\newcommand{\ddt}{\frac{d}{d t}}
\newcommand{\pddt}{\frac{\partial}{\partial t}}
\renewcommand{\pddt}{\partial_t}
\newcommand{\norm}[1]{\hat{#1}}
\newcommand{\Jump}[1]{\llbracket #1 \rrbracket \cdot \textbf{n} }

%%% Utiliser pour les commentaires
\newcommand{\JL}[1]{\color{red}#1\color{black}}
\newcommand{\DL}[1]{\color{green}#1\color{black}}
\newcommand{\tb}[1]{\color{blue}#1\color{black}}
% \renewcommand{\alpha}{}
\renewcommand{\JL}[1]{}
% \renewcommand{\tb}[1]{}

\renewcommand{\size}[1]{0.3\textwidth}
\newcommand{\expo}[2][n]{\frac{(-1)^#1}{#1!} \partialp{1}{#1} \pavg{\int_{\Omega_\alpha} \pri{1}{#1}#2 d\Omega}}
\newcommand{\expoU}[2][n]{\frac{(-1)^#1}{#1!} \partialp{1}{#1} \pavg{\textbf{u}_\alpha\int_{\Omega_\alpha} \pri{1}{#1}#2 d\Omega}}
\newcommand{\expoS}[2][n]{\frac{(-1)^#1}{#1!} \partialp{1}{#1} \pavg{\int_{\Sigma_\alpha} \pri{1}{#1}#2 d\Sigma}}

% \newcommand{\numref}[1]{\ref{#1}}
\renewcommand{\ref}[1]{\autoref{#1}}

%%%%%%%%%%%%%%%%%%%%%%%%%%%%%%% Title & Author %%%%%%%%%%%%%%%%%%%%%%%%%%%%%%%%


%\title{The hybrid model for arbitrary dispersed multiphase flows with surface properties}
\title{Thesis}

\author[1,2]{Nicolas Fintzi}
\affil[1]{IFP Energies Nouvelles, Rond-point de l’changeur de Solaize, 69360 Solaize}
\affil[2]{Sorbonne Université, Institut Jean le Rond ∂’Alembert, 4 place Jussieu, 75252 PARIS CEDEX 05, France}

\usepackage{pgfgantt}

\begin{document}

\maketitle
\part{Outline of the manuscript}
\section{Chap. Introduction}

\section{Chap. The hybride model for fluid particles}
\section{Chap. Meso-scale model for dispersed multiphase flow}
\section{Chap. Direct numerical simulation. (validation and set-up)}
\section{Chap. Mono-disperse water/oil emulsion}
\section{Chap. Maybe Ploy-disperse flows ? }
\section{Chap. Maybe bubbly flows ?}
\section{Chap. Conclusion}

\newpage
\part{Articles}
\setcounter{section}{0}
\section{Averaged equations for disperse two-phase flow made of fluid particles}
\begin{itemize}
    \item Objective : derive a hybrid model for two phase flow made of fluid particles. 
    \item Discussion : Identify the main closure terms and give hint for the DNS calculations. 
\end{itemize}
\section{Inter-particle scale averaged equations for disperse two-phase flow.}
\begin{itemize}
    \item Objective : From the Lagrangian equation of the 1$^{st}$ Article and kinetic theories, we demonstrate how to derive averaged equations to describe relative kinematics   and dynamic between the nearest pair.  
    (Also possible to derive equation for the Eulerian  phase including particles)
    \item Discussion : 
    \begin{itemize}
        \item How the averaged relative velocity, distance and force evolve through an interaction time (Theoritical expression in simplified cases). 
        \item Which quantities Facilitate the cluster arising ? 
        \item Give a theoretical expression based on these equations for the averaged mean part of the particle-fluid-particle stress tensor. 
    \end{itemize}
\end{itemize}
\section{Inertial stress and interphase force in oil-water emulsion}
\begin{itemize}
    \item Objective : We provide empirical closure for the interphase force and averaged stress within a water-oil emulsion. 
    (Can be extended to vapor drop in nuclear engineering since only $\mu_r$ changes). 
    \item Discussion : 
    \begin{itemize}
        \item Comparason between : Stresslet, particle-fluid-particle stress and granular temperature. $\rightarrow$ it is the same order of magnitude at moderate inertial regime, thus they all must be important. 
    \end{itemize}
\end{itemize}

\subsection*{Outline :}
\begin{enumerate}
    \item Introduction
    \item Numerical methodology
    \begin{enumerate}
        \item Problem statement 
        \item Simulation set up
        \item The no-coalescence algorithm
    \end{enumerate}
    \item Preliminary tests / Validation
    \begin{enumerate}
        \item fixed array of bubbles 
        \begin{itemize}
            \item Comparison with \citet{esmaeeli2005direct}
        \end{itemize}
        \item Free array of droplets
        \begin{itemize}
            \item Mesh independence studies 
            \item Space convergence (Appendix)
            \item Statistical convergence
        \end{itemize}
    \end{enumerate}
    \item Inter-phase drag force
    \begin{itemize}
        \item Definition of the interphase drag  
        \item $Re$ Vs. $Ga$ \& $\phi$ 
        \item $F_y$ Vs. $Ga$ \& $\phi$
        \item How is it different from solid spheres or bubbles ?
        \item Include $\mu_r =1$ compare to vapor drops. 
    \end{itemize}
    \item The multifactorial Stress
    \begin{enumerate}
        \item Fluid velocity fluctuation and Granular temperature
        \item First moment  ? 
        \item Particle-fluid-particle stress ?
    \end{enumerate}
    \item Conclusion and discussion
    \begin{itemize}
        \item Discussion on the numerical method, how it allowed us to carry out massive DNS
        \item We provided empirical closure in the limit of mono-disperse suspension
        \item We have shown how the fluctuation beaved
        \item Perspective : Poly-disperse suspension
    \end{itemize}
\end{enumerate}


\section{Kinetic and dynamical description of binary interactions in oil-water emulsion}
\begin{itemize}
    \item Objective :  Describe relative kinematics   and dynamic properties between the nearest pair (also nearest eulerian fields)
    \item Discussion :
    \begin{itemize}
        \item Mean time interactions
        \item Relative velocity direction 
        \item Visual unedrstanding of the PFP stress
    \end{itemize}
    \item Perspective :
    \begin{itemize}
        \item This approche could help us to derive more accurate models for coalesce and colision kernels blabalbla.. .
    \end{itemize}
\end{itemize}

\subsection*{Outline :}
\begin{enumerate}
    \item Introduction
    \item Theoretical background 
    \begin{enumerate}
        \item Introduce nearest particle statistics and averaging processes
        \item Transport equation for the nearest probability pair
        \item Derivaiton of relative properties 
        \item Ensemble averaged equations. 
    \end{enumerate}
    \item Direct numerical simulations
    \begin{itemize}
        \item Simulation set-up
        \item Refer to the preceding paper for validation
    \end{itemize}
    \item Statistics of fluid-particle interaction.   
    \begin{enumerate}
        \item Mean shape of the drops. 
        \item Mean velocity field inside and around the drop 
        \item Mean acceleration fields. 
    \end{enumerate}
    \item Statistics of particle-fluid-particle interaction.   
    \begin{enumerate}
        \item Relative space description of the interactions 
        \item Age description of the interaction
        \item Ensemble average of relative properties
        \begin{itemize}
            \item  link ww with u'u'
            \item gives PFP stress and its contribution
            \item give rb 
        \end{itemize}
    \end{enumerate}
    \item damping or coalescence models. 
    \item Conclusion and discussion.
    \begin{itemize}
        \item We presented an accurate description of relative particles' interaction through the fluid. 
        \item We gave quantitative results (PFP stress and other). 
    \end{itemize}
\end{enumerate}

\section{Particle-fluid-particle stress in inertial bubbly flows}
\begin{itemize}
    \item Objective :  we provide quantitative results regarding the PFP stress 
\end{itemize}




\newpage
\part{Timeline}
\begin{figure}[h!]
    \centering
    \begin{ganttchart}[
        % hgrid,
        vgrid,
        time slot format=isodate-yearmonth,
        time slot unit=month
        ]{2022-01}{2023-04}
        \gantttitlecalendar{year, month} \\
        \ganttbar[progress=100]{Rapport bib$(+)$}{2022-07}{2022-08}\\
        \ganttbar[progress=100]{Mid thesis report$(+)$}{2022-11}{2023-04}\\
        \ganttbar[progress=90]{Simulation setup}{2022-01}{2022-12}\\
        \ganttgroup[progress=100]{$0^{st}$ Set of DNS (2D)}{2022-05}{2022-08}\\
        \ganttbar[progress=100]{\underline{Article stage} $(**)$}{2022-01}{2022-03}\\
        % \ganttgroup{$1^{st}$ Set of DNS}{2022-12}{2022-12}
    \ganttgroup[progress=100]{$1^{st}$ Set of DNS : Emulsion}{2023-01}{2023-02}\\

        % \ganttbar{Task 1}{2022}{2023}
        \ganttlink[link type=f-f]{elem3}{elem0}
    \end{ganttchart}
    \caption{Already accomplished task of the year 2022, $(+)$ Represents the \textit{must do} tasks, $(*)$ Optional task, $(**)$ preferred optional tasks \ldots}
\end{figure}

\begin{figure}
    \centering

\begin{ganttchart}[
    % hgrid,
    vgrid,
    today=2023-09,
    time slot format=isodate-yearmonth,
    time slot unit=month
    ]{2023-01}{2024-12}
    \gantttitlecalendar{year, month} \\
    \ganttgroup[progress=100]{$1^{st}$ Set of DNS : Emulsion}{2023-01}{2023-02}\\
    \ganttbar[progress =100]{Mid-thesis report $(+)$}{2023-01}{2023-04}\\
    \ganttmilestone{ICMF - kobe $(+)$}{2023-04}\\
    \ganttmilestone{Mid-thesis $(+)$}{2023-05}\\
    \ganttbar[progress = 70]{\underline{Article 1} $(*)$}{2023-06}{2023-10}\\
    \ganttbar[progress = 20]{\underline{Article 3} $(*)$}{2023-08}{2023-12}\\
    \ganttbar[progress = 10]{\underline{Article 4} $(*)$}{2024-01}{2024-04}\\
    \ganttbar[progress = 5]{\underline{Article 2} $(***)$}{2024-01}{2024-05}\\
    % \ganttbar{Rapport bib}{2022-07}{2022-10}\\
    % \ganttbar{Simulation DNS}{2022-01}{2022-12}
    \ganttmilestone{Conference ??? $(+)$}{2024-04}\\
    \ganttbar[progress = 30]{Thesis report $(+)$}{2024-07}{2024-11}\\
    \ganttmilestone{Soutenance these ? $(+)$}{2024-12}\\
    \ganttgroup[progress=1]{$2^{nd}$ DNS : Bi-disperse}{2023-10}{2023-10}\\
    \ganttgroup[progress=1]{$3^{nd}$ DNS : Bubbles}{2023-11}{2023-12}\\
    \ganttbar[progress = 0]{short  \underline{Article 5}  $(**)$}{2024-05}{2024-06}\\
    % \ganttbar{Task 1}{2022}{2023}
    % \ganttlink{elem0}{elem2}
    \ganttlink{elem0}{elem5}
    \ganttlink{elem0}{elem6}
    \ganttlink[link type=f-f]{elem4}{elem5}
    \ganttlink[link type=s-s]{elem7}{elem6}
    \ganttlink[link type=f-f]{elem5}{elem6}
    \ganttlink[link type=f-s]{elem9}{elem10}
    \ganttlink{elem12}{elem13}
    \ganttlink{elem11}{elem6}
\end{ganttchart}
\caption{Ambitious planning for the rest of the thesis. $(+)$ Represents the \textit{must do} tasks, $(*)$ $1^{st}$ Optional task, $(**)$ second rank optional tasks\ldots}
\end{figure}


Less greedy  : 
\begin{figure}
\begin{ganttchart}[
    % hgrid,
    vgrid,
    today=2023-08,
    time slot format=isodate-yearmonth,
    time slot unit=month
    ]{2023-01}{2024-12}
    \gantttitlecalendar{year, month} \\
    \ganttgroup{$1^{st}$ Set of DNS}{2023-01}{2023-02}\\
    \ganttbar[progress =100]{Mid-thesis report}{2023-01}{2023-04}\\
    \ganttmilestone{ICMF - kobe}{2023-04}\\
    \ganttmilestone{Mid-thesis}{2023-05}\\
    % \ganttbar[progress = 80]{Article 1}{2023-05}{2023-08}\\
    \ganttbar[progress = 30]{Article 2}{2023-08}{2023-11}\\
    \ganttbar[progress = 20]{Article 4}{2023-12}{2024-03}\\
    % \ganttbar{Rapport bib}{2022-07}{2022-10}\\
    % \ganttbar{Simulation DNS}{2022-01}{2022-12}
    \ganttmilestone{Conference ??? }{2024-04}\\
    \ganttbar[progress = 30]{Thesis report}{2024-06}{2024-12}\\
    \ganttmilestone{Soutenance these ?}{2025-01}
    % \ganttbar{Task 1}{2022}{2023}
    \ganttlink{elem0}{elem3}
    \ganttlink{elem0}{elem4}
    \ganttlink{elem0}{elem5}
    \ganttlink{elem5}{elem6}
    \ganttlink[link type=f-s]{elem7}{elem8}
\end{ganttchart}
\caption{Lighter planning for the rest of the thesis. $(+)$ Represents the \textit{must do} tasks, $(*)$ Optional task, $(**)$ preferred optional tasks}
\end{figure}

\bibliography{Bib/bib_bulles.bib}

\end{document}
\section{Discussion and conclusion}

In this chapter, we have considered the modeling of dilute emulsion of slightly deformable droplets. 
We summarize these advancements in three main points which are as follows: 
\begin{enumerate}
    \item We have shown in the first place how, from the first moment of momentum and second moment of mass equation we derive tensor equations for the droplet mean shear rate and deformation tensor. 
In addition to the classic equation of deformation, already obtained in \citet{goddard1967nonlinear},  and the angular momentum equation, it has been shown that the symmetric part of the moment of momentum equation corresponds to the second oscillatory mode equation of the droplets. 
More specifically, we derived this equation while staying general on the droplet's internal motion, making the forcing term of this oscillatory equation explicit in terms of the local properties of the droplets and carrier fluid. 
Although not closed these forcing terms exhibit more physical sense than the usually undefined forcing term used in the classic second-order oscillatory model.
For example, in \citet{riviere2021sub} they use such a forcing term, but it is not directly related to the fluid and droplet locals properties.
With the present formalism we might be able to express explicitly what are the forcing terms of this equation, although not closed. 

\item 
In this chapter we have also proposed of solving for the eigenvalues of the deformation and rate of deformation and angular rotation vector of the droplet, instead of directly solving for the corresponding quantities in the laboratory reference frame. 
We have demonstrated that we reduce drastically the number of equations due to the limited degrees of freedom considered for the deformation of the drops. 
These Lagrangian equations have been averaged. 
We then discovered the counterpart of solving for the eigenvalues of the mentioned tensors; 
Indeed, at the coarse-grained level, we solve for the average of these eigenvalues. 
However, we now need new closure to determine the average of the corresponding tensor in the laboratory reference frame.  
This calls into question the efficiency of this local basis formulation since the expression of some of these tensors, especially the angular rotation vector is indispensable in some situations. 

% \item Finally, we propose an analysis of the impact of the phases relative motion on the particle deformation and carrier fluid phase effective stress. 
% In the first place, we could demonstrate an alternative derivation of the droplet's deformation generated by relative translation at finite Reynolds number, as in the study of \citet{taylor1964deformation}. 
% Our methodology is based on the averaged particle equations derived in the previous section and explicit closures are derived using an extended Reciprocal Theorem suited for spherical droplets. 
% With the aid of DNS we could validate the theoretical result in the low but finite inertial regime. 
% Lastly, we discuss the impact of particle translation on the suspension stress. 
% It is shown that the exchange terms responsible for the droplets' deformation, i.e. the \textit{Stresslet} term, were also responsible for an additional source in the continuous phase stress. 
% Notably, we could show that the \textit{Stresslet} term usually neglected in this context is not null and is function of the particle-carrier phase relative velocity square and $\phi Re$.  
% When considering only relative translation this term is in competition with the \textit{Reynolds stress} term, which is shown to be comparable to the \textit{Stresslet} term. 
% As the Reynolds stress term is shown to be $\sim \phi^{2/3}$ while the \textit{Stresslet} is $\sim \phi$ we might expect the latter negligible in the dilute regime but dominant in the dense regime. 
% At $\phi = 0.05$ we state that the contribution of the \textit{Stresslet} is about 10\% of the \textit{Reynolds stress} contribution. 
% The \textit{Stresslet} term might eventually be more important than the \textit{Reynolds stress} term if the droplets possess a higher density ratio. 
\end{enumerate}
In \ref{chap:closure-disperse}, we demonstrate how this set of equations can be used to theoretically recover the deformation of a droplet through the evolution equation of $\bm{\chi}_p$. 
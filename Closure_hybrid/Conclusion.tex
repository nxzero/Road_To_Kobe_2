\section{Discussion and conclusion}

In this chapter many advances have been maid regarding the modeling of dilute emulsion of slightly deformable droplets. 
We summary these progesses in three main points which are as follows: 
\begin{enumerate}
    \item We have shown in the first place how, from the first moment of momentum and second moment of mass equation we could derive tensor equations for the particle mean shear rate and deformation tensor. 
In addition to the classic equation of deformation, already obtained in \citet{goddard1967nonlinear},  and the angular momentum equation, it has been shown that the symmetric part of the moment of momentum equation corresponds to the second oscillatory mode equation of the droplets. 
More specifically, we derived this equation while staying general on the droplets internal motion, making the forcing term of this oscillatory equation explicit in terms of the local properties of the particles and carrier fluid. 
Although not closed these forcing terms exhibit a lots more physical sense than the usually undefined  forcing term used in the classic second order oscillatory model.
For example in \citet{riviere2021sub} they  use such a forcing term, but it is not directly related to the fluid and particle locals properties.
However, with our formalism we were able to express explicitly what are the forcing terms of this equation, although not closed. 

\item 
The second advancement of this chapter is the proposition of solving for the eigenvalues of the deformation and rate of deformation and angular rotation vector of the particle, instead of directly solving for the corresponding quantities in the laboratory basis. 
In this way we demonstrate that we reduce drastically the number of equations due to the limited degrees of freedom considered for the deformation of the drops. 
These Lagrangian equations are then averaged. 
We then discover the counterpart of solving for the eigenvalues of the mentioned tensors; 
Indeed, at the cross-grained level we solve for the average of these eigenvalues. 
However, we now need new closure to determine the average of the corresponding tensor in the laboratory reference frame.  
This calls into question the efficiency of these local basis formulation since the expression of some of these tensor, especially the angular rotation vector is being indispensable in some situation. 

\item Finally,  in the motivation of improving the modeling 1D Navier-Stokes averaged equations we end this work on an analysis of the phases relative motion on the particle deformation and carrier fluid phase stress. 
In the first place, we could demonstrate an alternative derivation of the droplet's deformation generated by relative translation at finite Reynolds number, as in the study of \citet{taylor1964deformation}. 
Our methodology is based on the averaged particle-equations derived in the previous section and explicit closure are derived using a extended Reciprocal Theorem suited for spherical droplets. 
Lastly, we discuss the impact of particles translation on the suspension stress. 
It is shown that the exchange terms responsible for the droplets' deformation, i.e. the \textit{Stresslet} term, were also responsible for an additional source in the continuous phase stress. 
Notably, we could show that the \textit{Stresslet} terms neglected in this context were in fact not null and function of the particle-carrier phase relative velocity square and $\phi Re$.  
When considering only relative translation this term is in competition with the \textit{Reynolds stress} term, which is shown to be comparable to the \textit{Stresslet} term. 
The \textit{Stresslet} term might eventually be more important than the \textit{Reynolds stress} term if the droplets posses higher density ratio. 
\end{enumerate}

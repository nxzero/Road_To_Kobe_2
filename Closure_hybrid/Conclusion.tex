In this work we provided a well-developed hybrid model for dispersed two phase flow made of fluid particles.
To model the continuous phase we derived \ref{eq:avg_dt_chi_f} with a phase-averaging approach following the usual method introduced by \citet{drew1983mathematical}.
The first contribution of the work is the derivation of the dispersed phase equation. 
Indeed, it is derived in the most general way based on the volume and surface governing equations valid at the local scale, namely \ref{eq:dt_f_I} and \ref{eq:dt_f_k}. 
It is done considering fluid particles with surface properties. 
Then the first-order and higher moments equations are derived.
This allows us to describe higher degree of freedom of the particle-phase, such as the particle shape or moment of momentum. 

After averaging the equations that govern the dispersed phase using two distinct frameworks, namely, the phase-average and the particle-average, we demonstrated the equivalence between both formalism. 
In light of \ref{eq:scheme_equivalence} we reached the major conclusion of this work, i.e.  the phase-averaged equation for the dispersed phase, is a series expansion of the particle-averaged equations.  
As assumed by \citet{zhang1997momentum} it is possible to describe a particle with an arbitrary order of accuracy by deriving the moments equations of a particle. 
In this work we provided a general form of these equations together with a clear physical explanation. 
Acknowledging this fact we concluded that  any physical phenomenon could be modeled with the moments equations since they constitute the phase-averaged equation, i.e. \ref{eq:avg_dt_chi_f} with $k =2$. 
Overall, the main advancement of this model is its ability to incorporate the effects of the particles' surface and volume properties, and provide a framework for deriving particle-average equations tailored to any type of problem.
In addition, to the first order conservation laws we also derived the higher order moments equations. 

To illustrate our point all along the derivation of the hybrid model we treat the case of the momentum conservation equation. 
Especially, it is shown that the surface tension play no role on the linear and angular conservation of momentum equation, but it does  affect so-called stretching of momentum conservation equation. 
In general the surface diffusive fluxes are not involved in the linear moment, \ref{eq:avg_dt_dq_alpha_tot}, but act as a source term in the first order moment balance equations \ref{eq:avg_dt_dQ_alpha_tot}. 
Then, to give a more physical insight on these moments equations we end this work by proposing some examples. 
Then, we propose a new case were we use the first moment equation of the surfactant distribution to derive a transport equation of the center of mass of surfactant along the surface of the particle. 

The main draw back of this work is that inter particles interactions are not included naturally in these models. 
It would be interested in a future studies to show how pair-particles interaction can be unpacked from the phase-averaged equations of the dispersed phase. 

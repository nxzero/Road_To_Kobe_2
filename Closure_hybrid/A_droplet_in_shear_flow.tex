


\section{A translating spherical droplet in stokes}
\label{ap:Translating_sphere}
In this appendix we expose the velocity fields solution for the stokes flow past a spherical drop. 
In a second step, we compute the form of the moment of momentum closure mentioned in \ref{sec:Lagrangian} in terms of the fluid properties and the drop fluid relative velocity.

We consider a drop of radius $a$ translating with the velocity $\textbf{u}_\alpha$ in a stokes flow with undisturbed velocity $\textbf{u}_1$.
The relative velocity between the droplet and the fluid is defined as $\textbf{u}_{\alpha 1}= \textbf{u}_\alpha - \textbf{u}_1(\textbf{x}_\alpha)$ where $\textbf{u}_1(\textbf{x}_\alpha)$ is the undisturbed averaged fluid velocity at the position $\textbf{x}_\alpha$. 
In these conditions, the local fluid phase velocity $\textbf{u}_1^0$, the local particle interior velocity $\textbf{u}_2^1$ and the local stress fields $\bm\sigma_2^0$ within the particle phase, might be written as\citet{pozrikidis1992boundary}\footnote{The solution of this problem is derived in the above cited book p .207.  However, the author made a slight mistakes on the constant $\textbf{c}$ which we corrected here. }, 
\begin{align*}
    u_{1,i}^0
    = \left(\frac{\delta_{ik}}{r} + \frac{r_ir_k}{r^3}\right)  g_k
    + \left(-\frac{\delta_{ik}}{r^3} + \frac{3r_ir_k}{r^5}\right)  d_k\\
    u_{2,i}^0
    = c_i
    + \left(2 r^2 \delta_{ik} - r_ir_k\right) e_k\\
    % e_{2,ik}^0
    % = \mu(
    %     3 \delta_{ij} r_k 
    %     + 3 \delta_{kj} r_i
    %     -2 r_j \delta_{ki}
    % )e_j 
    \sigma_{2,ik}^0
    = \mu 3(
        - 4 \delta_{ik} r_j
        + \delta_{ij} r_k
        + r_i \delta_{kj}
    )e_j 
\end{align*}
with, 
\begin{align*}
    &\textbf{g} = a\frac{1}{4}\left(\frac{3\lambda + 2}{\lambda +1}\right) \textbf{u}_{\alpha 1},
    &\textbf{d} = -a^3\frac{1}{4}\left(\frac{\lambda}{\lambda +1}\right) \textbf{u}_{\alpha 1},\\
    &\textbf{c} = \frac{1}{2}\left(\frac{2\lambda + 3}{\lambda +1}\right) \textbf{u}_{\alpha 1},
    &\textbf{e} = -\frac{1}{a^2}\frac{1}{2}\left(\frac{1}{\lambda +1}\right)  \textbf{u}_{\alpha 1}.\\
\end{align*}

Now that we have an explicit solution for the internal flow and stress one can compute the terms appearing in the energy and moment of momentum balance equations. 
The integral involving the particle internal motion in the equation exposed in \ref{sec:Lagrangian} are evaluated exhaustively and yield, 
\begin{equation*}
    \intO{\rho_2(\textbf{w}_2^0)_i (\textbf{w}_2^0)_j}
    = \frac{m_\alpha}{140 (\lambda +1)^2}
    (7 (\textbf{u}_{\alpha 1})_i(\textbf{u}_{\alpha 1})_j + (\textbf{u}_{\alpha 1}\cdot \textbf{u}_{\alpha 1})\delta_{ij})
\end{equation*} 
\begin{equation*}
    \intO{\rho_2(\textbf{w}_2^0)_k (\textbf{w}_2^0)_k}
    = \frac{m_\alpha}{14 (\lambda +1)^2}
     (\textbf{u}_{\alpha 1})_k(\textbf{u}_{\alpha 1})_k
\end{equation*} 
\begin{equation*}
    \intO{\rho_2(\textbf{w}_2^0)_i (\textbf{w}_2^0)_j}^\text{dev}
    = \frac{m_\alpha}{20 (\lambda +1)^2}
    ((\textbf{u}_{\alpha 1})_i(\textbf{u}_{\alpha 1})_j + (\textbf{u}_{\alpha 1}\cdot \textbf{u}_{\alpha 1})\delta_{ij})
\end{equation*} 
\begin{equation*}
    \intO{\bm\sigma_2^0}
    = 0 
\end{equation*}
\begin{equation*}
    \intO{\bm{\sigma}_2^0:\grad \textbf{u}_2^0}
    = 2\mu_2 \intO{\textbf{e}_2^0: \textbf{e}_2^0 }
    = 
    \frac{6 \mu_2 v_p}{a^2(1+\lambda)^2}
    (\textbf{u}_{\alpha 1}\cdot \textbf{u}_{\alpha 1})
\end{equation*}
\begin{align*}
    \intO{\textbf{r}\textbf{u}_2^0}
    = 0 
\end{align*}
\begin{equation*}
    \intS{\bm\sigma_1^0\cdot \textbf{n}_2}
    = - \frac{3v_\alpha\mu_1}{2 a^2} 
    \left(\frac{3\lambda+2}{\lambda+1}\right) 
    \textbf{u}_{\alpha 1}
\end{equation*}
\begin{equation*}
    \intS{\textbf{r}\bm\sigma_2^0\cdot \textbf{n}_2}
    = 0 
\end{equation*}
\begin{equation*}
    \intS{(\bm\sigma_2^0\cdot \textbf{n}_2)_i r_kr_l}
    = \frac{3\mu_1\phi_2}{10}\left(\frac{5\lambda+2}{\lambda+1}\right)u_{1 \alpha,i}\delta_{kl}
    + \frac{3\mu_1\phi_2}{5}\left(\frac{1}{\lambda+1}\right)(u_{1 \alpha,k}\delta_{il}+u_{1 \alpha,l}\delta_{ki})
\end{equation*}
\begin{equation*}
    \intO{\mu(\textbf{e}_2^0)_{ik} r_l} =
    \frac{\phi_2\mu_1}{10}\left(\frac{1}{\lambda+1}\right)
    \left(
        2\delta_{ik}u_{1 \alpha , l}
        -3\delta_{kl}u_{1 \alpha , i}
        -3\delta_{il}u_{1 \alpha , k}
    \right)
\end{equation*}
We can notice that now all these quantities are determined by the relative velocity.  
While the hill vortex is a solution valid in potential and stokes flow, the solution for the exterior flow is limited to stoke flow. 
Consequently, the zero, first and second moment of surface force are limited to stokes flow. 

\section{A translating oblate spheroidal droplet in stokes}

In this appendix we expose the velocity fields solution for the stokes flow past a spherical drop. 
In a second step, we compute the form of the moment of momentum closure mentioned in \ref{sec:Lagrangian} in terms of the fluid properties and the drop fluid relative velocity.

We consider a drop of radius $a$ translating with the velocity $\textbf{u}_\alpha$ in a stokes flow with undisturbed velocity $\textbf{u}_1$.
The relative velocity between the droplet and the fluid is defined as $\textbf{u}_{\alpha 1}= \textbf{u}_\alpha - \textbf{u}_1(\textbf{x}_\alpha)$ where $\textbf{u}_1(\textbf{x}_\alpha)$ is the undisturbed averaged fluid velocity at the position $\textbf{x}_\alpha$. 
In these conditions, the local fluid phase velocity $\textbf{u}_1^0$, the local particle interior velocity $\textbf{u}_2^1$ and the local stress fields $\bm\sigma_2^0$ within the particle phase, might be written as linear combinaison of spherical harmonique proportional to $\textbf{u}_{1\alpha} = U$ thus
\begin{align*}
    u_{1,i}^0
    = \left(\frac{\delta_{ik}}{r} + \frac{r_ir_k}{r^3}\right)  g U_k
    + \left(-\frac{\delta_{ik}}{r^3} + \frac{3r_ir_k}{r^5}\right)  d U_k\\
    p_1^0 = 
    \mu g U_j 2\frac{r_j}{r^3}\\
    u_{2,i}^0
    = c U_i
    + \left(2 r^2 \delta_{ik} - r_ir_k\right) e U_k\\
    p_2^0 
    = 10\mu e U_j r_j
\end{align*}
The stress is express as, 
\begin{align*}
    \sigma_{2,ij}^0 
    = - p_2^0 \delta_{ij}
    +\mu_2 (\partial_i u_{2,j}^0 + \partial_j u_{2,i}^0)
\end{align*}

\section{A droplet in shear flow}

We consider an infinite linear flow $\textbf{u}^\infty = \bm\Gamma\cdot \textbf{r} =(\bm\Omega + \textbf{E})\cdot \textbf{r} $ such that the undisturbed flow around the sphere is $\textbf{u}^\infty = \bm\Gamma\cdot\textbf{r}$.
The disturbance flow in this case might be determined following the method outlined in \citep{leal2007advanced}, we obtain  :
\begin{align*}
    \textbf{u}^0_1
    = \bm\Gamma\cdot\textbf{r}
    -\frac{\lambda}{(\lambda + 1)r^5} \textbf{E}\cdot\textbf{r}
    - \left(\frac{5\lambda +2}{2(\lambda +1 )r^5} - \frac{5\lambda}{2(\lambda+1)r^7}\right) \textbf{r}(\textbf{E}:\textbf{rr})\\
    p_1^0 
    = -\frac{(5\lambda+2)}{(\lambda+1)r^5}\textbf{E}:\textbf{rr}\\
% \end{align*}
% \begin{align*}
    \textbf{u}^0_2
    = \bm\Omega\cdot\textbf{r}
    + \frac{5r^2- 3}{2(\lambda + 1)} 
    \textbf{E}\cdot\textbf{r}
    -\frac{1}{\lambda+1} \textbf{r}(\textbf{E}:\textbf{rr})\\
    p_2^0 
    = \frac{21\lambda}{2(\lambda+1)}
    \textbf{E}:\textbf{rr}
\end{align*}

With that we are able to compute the following integrals,
\begin{equation*}
    \intO{\rho_2(\textbf{w}_2^0)_i (\textbf{w}_2^0)_j}
    = \frac{4\pi}{15}\bm\Omega\cdot\bm\Omega
    + \frac{4\pi}{945(\lambda+1)^2}
    (2\textbf{E}:\textbf{EI} + 15 \textbf{E}\cdot \textbf{E})
\end{equation*}
\begin{equation*}
    \intO{\rho_2(\textbf{w}_2^0)_k (\textbf{w}_2^0)_k}
    = \frac{4\pi}{15}\bm\Omega:\bm\Omega
    + \frac{4\pi}{45(\lambda+1)^2}
    \textbf{E}:\textbf{E} 
\end{equation*} 
\begin{equation*}
    \intO{\bm\sigma_2^0}
    = \frac{8\pi}{5(\lambda+1)}
    \textbf{E}
\end{equation*}
\begin{equation*}
    \intO{\textbf{e}_2^0}
    = \frac{8\pi}{3(\lambda+1)}
    \textbf{E}
\end{equation*}
\begin{equation*}
    \intO{\bm{\sigma}_2^0:\grad \textbf{u}_2^0}
    = 2\mu_2 \intO{\textbf{e}_2^0: \textbf{e}_2^0 }
    = 
    \frac{4\pi}{(\lambda+1)^2}\textbf{E}:\textbf{E}
\end{equation*}
\begin{align*}
    \intO{\textbf{r}\textbf{u}_2^0}
    = 0 
\end{align*}
\begin{equation*}
    \intS{\bm\sigma_1^0\cdot \textbf{n}_2}
    = 0
\end{equation*}
\begin{equation*}
    \intS{\textbf{r}\bm\sigma_1^0\cdot \textbf{n}_2}
    = \frac{4\pi}{5}\frac{5\lambda+2}{\lambda+1}\textbf{E} 
\end{equation*}
\begin{equation*}
    \intS{(\bm\sigma_2^0\cdot \textbf{n}_2)_i r_kr_l}
    = 0
\end{equation*}
\begin{equation*}
    \intO{\mu(\textbf{e}_2^0)_{ik} r_l} =
    0
\end{equation*}
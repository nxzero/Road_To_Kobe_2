In this section we propose to study two example. 
This objective is to demonstrate how to derive specific application from our general hybrid model. 
Secondly, we wish to point out in wish situation one needs or might not need the more than the zeroth order equation of the particle phase.

% \subsection{Solid axis symmetric particle suspension}

% Let's start by the second moment of mass averaged equation.
As a first example let examine the case of a mono-disperse axissymmetric suspension of solid particles, such as ellipsoid or cylinders.
Let the vector $\textbf{p}$ denote the unit vector representing the orientation of the particle along its main axis of inertia. 
Then, the second moment of mass can be written as $\mathcal{M}_{\alpha,ij} =  p_ip_j M_\alpha^{||} +  \delta_{ij} M_\alpha^\bot$, where $M_{\bot}$ and $M_{||}$ represent the coefficients corresponding to the principal directions of the particles' mass distribution.
It is well-established that at least the drag force exhibits a significant dependence on the orientation of the particle \citep{kim2013microhydrodynamics}.
Therefore, despite being a second-order moment, the particle field $\mathcal{M}_p$ is indispensable to close the mono-disperse axissymmetric suspension problem.
We will see that to determine the orientation of the particles one need to know about its angular velocity, which give the use to the equation for $\mathcal{P}_p$, equally.  


% \subsubsection{Single particle equations}

We first focus on the one particle dynamic before deriving the averaged equations. 
As stated above we assume the velocity of the dispersed phase to be of the form : $u_{2,i}^0(\textbf{x}_\alpha + \textbf{r}) = u_{i}^\alpha + \epsilon_{ijk} {\omega}_{j}^\alpha {r}_k$, with $\omega_{a}^\alpha$ the angular velocity of the particle $\alpha$ and $\epsilon$ the Levi-Cita symbol.
Injecting this definition of the velocity in \ref{eq:mu_def}, \ref{eq:S_def} and \ref{eq:E_def} one can re-write the moment of momentum skew-symmetric and symmetric part and the internal energy equation, of the particle as,
\begin{align}
    % \label{eq:S_def}
    2\mathcal{S}_{\alpha,ij}
    % = \Omega_{\alpha,jk} \mathcal{I}_{\alpha,ki} + \Omega_{\alpha,ik} \mathcal{I}_{\alpha,kj}
    = - \epsilon_{jak} \omega_a^\alpha \mathcal{I}_{ki}^\alpha
      - \epsilon_{ibk} \omega_b^\alpha \mathcal{I}_{kj}^\alpha
    % = 
    % M_\alpha^{||}  \left(
    %     \epsilon_{ikl} \omega_k 
    %     p_lp_j 
    %     +  \epsilon_{jkl} \omega_k 
    %     p_lp_i 
    % \right)
    \\
    % \label{eq:mu_def}
    \mu_{\alpha,i}
    % = 
    % \omega_{\alpha,j}(\mathcal{M}_{\alpha,kk} \delta_{\alpha,ij} - \mathcal{M}_{\alpha,ij})=
    % \omega_j \left[
    %     \delta_{ji} (M_\alpha^{||} + 2M_\alpha^\bot)
    %     - p_jp_i M_\alpha^{||} 
    % \right]
    =  \mathcal{I}_{ij}^\alpha \omega^\alpha_j\\
    2W_\alpha 
    = \omega^\alpha_k\omega^\alpha_l \mathcal{I}_{kl}^\alpha
\end{align}
Under this form note that the angular momentum is the product of the angular velocity $\omega_{\alpha,i}$ times $\mathcal{I}_{ik}^\alpha = (\mathcal{M}_{kk}^\alpha \delta_{ij} - \mathcal{M}_{ij}^\alpha)$ which is consistent with the definition of the moment of inertia used in solid mechanics.
In fact, it has been found  more practical to express the following equation with $\mathcal{I}_\alpha$ rather than with $\mathcal{M}_\alpha$. 
Thus, in this section we express all closure and equation with respect to $\mathcal{I}_\alpha$
Now we make use of the preceding definition to rewrite the first moment of mass, angular momentum and internal energy equation, i.e. \ref{eq:dt_M_alpha}, \ref{eq:dt_P_alpha} and \ref{eq:dt_w2_alpha}, gives directly, 
\begin{align}
    \label{eq:dt_I_alpha}
    \ddt \mathcal{I}_{ij}
    = - \epsilon_{jak} \omega_a^\alpha \mathcal{I}_{ki}^\alpha + 
    - \epsilon_{ibk}   \omega_b^\alpha \mathcal{I}_{kj}^\alpha\\
    \label{eq:dt_Iomega_alpha}
    \ddt (\mathcal{I}_{ij}^\alpha \omega_j^\alpha)
    = t^\alpha_i\\
    \label{eq:dt_Iomegaomega_alpha}
    \frac{1}{2}\ddt (\omega^\alpha_k\omega^\alpha_l \mathcal{I}_{kl}^\alpha)
    = \omega_k t^\alpha_k. 
\end{align}
As, can be seen here the dependence of the inertia matrices with time induce a significant complication to the problem. 
Firstly we must keep track of the evolution of the inertia matrices which add one supplementary equation. 
Secondly, the angular momentum and angular kinetic energy of the particle is not function solely of the angular velocity but also on its own inertia matrices. 
Which will add difficulty to unpair to angular velocity from the inertia matrix in the averaged equation. 

% \tb{maybe add the heat issipation}
% It will be useful in the following derivation to have an equation for $\omega_i^\alpha$, manipulating the equation of angular momentum with the use of the equation of evolution of the inertia matrix we obtain, 
% \begin{equation}
%     \ddt (\omega_i^\alpha)
%     = (\mathcal{I}^\alpha)_{ik}^{-1}(  
%         t^\alpha_k
%     -  \mathcal{S}_{kl}^\alpha\omega_l^\alpha
%     )
% \end{equation}
% Taking the dot product of this equation with $\omega_j^\alpha$ and adding its transpose in the indices $ij$ gives, 
% \begin{equation}
%     \ddt (\omega_i^\alpha\omega_j^\alpha)
%     = 
%     \omega_j^\alpha
%     (\mathcal{I}^\alpha)_{ik}^{-1}(  
%         t^\alpha_k
%     -  \mathcal{S}_{kl}^\alpha\omega_l^\alpha
%     )
%     + \omega_i^\alpha
%     (\mathcal{I}^\alpha)_{jk}^{-1}(  
%         t^\alpha_k
%     -  \mathcal{S}_{kl}^\alpha\omega_l^\alpha
%     )
% \end{equation}

% Also, note that the stress within the particle is supposed to be unknown. 
% Indeed, we considered rigid body motion inside the particles.
% However, it is still possible to evaluate the stress using the stretch of momentum equation which in this case yields, 
% \begin{equation*}
%     \intO{(\sigma_2^0)_{ij}}
%     = 
%     - \ddt \mathcal{S}^\alpha_{ij}
%     + \epsilon_{iab}\epsilon_{ibd}\omega_{b}^\alpha\omega_{d}^\alpha(
%         \frac{1}{2}\mathcal{I}_{kk}\delta_{ac}
%         - \mathcal{I}^\alpha_{ac} )
%     + S^\alpha_{ij}.
%     \\
% \end{equation*}
% Therefore, the stress can be computed conditionally on the knowledge of the angular velocity and acceleration of the particle, which can be determinate by solving the previous equations. 
% It is then possible to validate the hypothesis of deformable particle  upon the knowledge of the material body law. 

% Finally, in the perspective of deriving an equation for the covariance of the translational motion and the orientation we multiply the equation of $\mathcal{I}_{ij}^\alpha$ by $u_k^\alpha$ and the equation of $u_k^\alpha$ by $\mathcal{I}_{ij}^\alpha$, which gives us get the following equation,
% \begin{equation*}
%     \ddt (\mathcal{I}_{ij}^\alpha u_k^\alpha)
%     =
%     u_k^\alpha
%     \ddt \mathcal{I}_{ij}^\alpha
%     + \mathcal{I}_{ij}^\alpha \ddt u_k^\alpha
%     = u_k^\alpha \mathcal{S}_{ij}^\alpha
%     +  \mathcal{I}_{ij}^\alpha g_k
%     +  \mathcal{I}_{ij}^\alpha f_k^\alpha/m^\alpha
% \end{equation*}

% \subsubsection{Averaged equation for the fluid phase}

% It is somewhat simpler to first modify the fluid phase equations. 
% Indeed, among the 4 equations to be solved only the terms involving the internal velocity will be explicitly modified.  
% These terms appear in \ref{eq:dt_hybrid_k1}, under the form $\pSavg{{\textbf{rw}_2^0\times\bm{\sigma}_1^0 \cdot \textbf{n}_2}}$, where we substitute $\textbf{w}$ for the expression of a solid particle velocity field. 
% Substituting these terms into the pseudo turbulent energy equation gives, 
% \begin{multline*}
%     \pddt (\phi_1\rho_1k_1)  
%     + \div (
%         \phi_1\rho_1k_1\textbf{u}_1
%         + \textbf{q}_1^\text{k} 
%         )
%     = 
%     - \avg{\chi_1\bm{\sigma}_1^0 : \grad \textbf{u}_1^0}
%     - \bm{\sigma}_1^\text{eq} : \grad \textbf{u}_1
%     + (\textbf{u}_1 - \textbf{u}_p)
%     \cdot \pSavg{{\bm{\sigma}_1^0 \cdot \textbf{n}_2}}\\
%     - \pavg{ \textbf{u}^{\alpha'} \cdot \intS{  \bm{\sigma}_1^0 \cdot \textbf{n}_2}}
%     - \bm{\omega}_{p} \cdot  \pavg{\intS{\textbf{r}\times\bm{\sigma}_1^0 \cdot \textbf{n}_2}}
%     - \pavg{\bm{\omega}^{'}_\alpha  \cdot \intS{\textbf{r} \times \bm{\sigma}_1^0 \cdot \textbf{n}_2}}
%     % \pSavg{\textbf{r}\textbf{w}^0_2 \cdot \textbf{r}\bm{\sigma}_1^0 \cdot \textbf{n}_2}_l
% \end{multline*}
% with,
% \begin{multline*}
%     \textbf{q}_1^\text{k}
%     = \rho_1 \avg{\chi_1 \textbf{u}_1' k_1} 
%     - \avg{\chi_1 \textbf{u}_1' \cdot \bm{\sigma}_1^0}
%     + (\textbf{u}_1 - \textbf{u}_p)\cdot
%     \pSavg{{\textbf{r}\bm{\sigma}_1^0 \cdot \textbf{n}_2}}
%     \\
%     - \pavg{ \textbf{u}_\alpha' \cdot \intS{ \textbf{r} \bm{\sigma}_1^0 \cdot \textbf{n}_2}}
%     - \epsilon_{ijk} \omega_{p,j} \pavg{ \intS{\textbf{rr}\bm{\sigma}_1^0 \cdot \textbf{n}_2}}_{kli}
%     - \epsilon_{ijk}\pavg{\omega^{\alpha'}_{j}  \intS{\textbf{rr}\bm{\sigma}_1^0 \cdot \textbf{n}_2}}_{kli}
% \end{multline*}
% Now, we clearly see appear the mean and fluctuating work due to the rotational motion of the particles. 
% Which are the product of the angular velocities times the torque on the surface of the particles. 
% Similar comments can be made for the flux term were we see appear the work of the second moments of the surface traction with the angular velocity. 

% By the assumption of solid particle motion we already reformulated the closure terms of the form $\pSavg{{\textbf{rw}_2^0\times\bm{\sigma}_1^0 \cdot \textbf{n}_2}}$ surface traction tensor which were already present in the problem. 
% Nevertheless, we still now need to find additional closure related to the angular velocity covariance with the torque. 

% \begin{align*}
%     \pSavg{\textbf{w}^0_2\cdot \bm{\sigma}_1^0 \cdot \textbf{n}_2}
%     =
%     \epsilon_{ijk}\omega_{p,j}  \pavg{\intS{\textbf{r}\bm{\sigma}_1^0 \cdot \textbf{n}_2}_{ki}}
%     + \epsilon_{ijk}\pavg{{\omega}^{\alpha'}_{j}  \intS{\textbf{r}\bm{\sigma}_1^0 \cdot \textbf{n}_2}_{ki}}\\
%     \pSavg{\textbf{r}\textbf{w}^0_2 \cdot \textbf{r}\bm{\sigma}_1^0 \cdot \textbf{n}_2}_l
%     =
%     \epsilon_{ijk} \omega_{p,j} \pavg{ \intS{\textbf{rr}\bm{\sigma}_1^0 \cdot \textbf{n}_2}_{kli}}
%     + \epsilon_{ijk}\pavg{\omega^{\alpha'}_{j}  \intS{\textbf{rr}\bm{\sigma}_1^0 \cdot \textbf{n}_2}_{kli}}
% \end{align*}


\subsubsection{Orientation equation in stoke regime.}

Averaging the second-order moment of mass yields, $n_p \mathcal{M}_p =  \textbf{A}_p (M_p^{||} - M_p^\bot) +  \textbf{I} M_p^\bot$ where $\textbf{A}_p$ is the orientation tensor defined as $\textbf{A}_p = \avg{\delta_\alpha\textbf{pp}}$.
Then, let's express the internal motion of a solid particle by : $\textbf{u}_2(\textbf{x}_\alpha) = \textbf{u}_\alpha + \textbf{r}\times \omega_\alpha$ where $\omega_\alpha$ represents the angular velocity of the particle.
It follows the expression of the stretching of momentum : $\mathcal{S}_\alpha = (M_\alpha^{||} - M_\alpha^\bot) \left(
    \omega_\alpha \times
    \textbf{pp}
    + \textbf{pp} \times \omega_\alpha
\right)$. 
Subsequently,  we can easily derive the transport equation for $\textbf{A}$ by averaging \ref{eq:dt_M_alpha} and using the previous expressions for $\mathcal{M}_\alpha$ and $\mathcal{S}_\alpha$.
The resulting equation is given by~:
\begin{equation}
    \pddt (n_p\textbf{A})
    + \div (
        n_p\textbf{u}_p\textbf{A}_p
        + \mathbf{\Sigma}
        )
    =
    \pavg{\textbf{pp} \times \omega_\alpha}
    + \pavg{\omega_\alpha \times \textbf{pp}},
    % + \pnavg{\textbf{pp}' \times \omega_\alpha'}
    % +\pnavg{\omega_\alpha' \times \textbf{pp}'}
    \label{eq:avg_dt_M_alpha}
\end{equation}
where $\mathbf{\Sigma} = \pavg{\textbf{u}'_\alpha(\textbf{pp})'}$ is the covariance term between the fluctuation of the velocity and the orientation tensor.
We recall that the definition of the fluctuation notation is provided by the expression from \ref{eq:def_fluctu}.
At this stage we need to find closure for both terms on the RHS of \ref{eq:avg_dt_M_alpha}. 
Therefore, we assume torque free rigid particle in to Stokes flow, where we can utilize Jeffery's equation \citep{guazzelli2011}.
It reads,
\begin{equation}
    \omega_\alpha \times \textbf{p}
    = \mathbf{\Omega}\cdot\textbf{p}
    + \beta\left(
        \textbf{E}\cdot \textbf{p}
        - \textbf{E} : \textbf{ppp}
    \right),
    \label{eq:jefferey}
\end{equation}
with $\textbf{E}$ and $\mathbf{\Omega}$ being the symmetric and antisymmetric parts of the bulk velocity gradient, respectively, such that $\grad\avg{\textbf{u}}=\textbf{E}+\mathbf{\Omega}$.
The coefficient $\beta$  is a constant related to the aspect ratio of the particle.
Finally, by substituting the RHS terms of \ref{eq:avg_dt_M_alpha}, by using \ref{eq:jefferey}, we arrive at the closed form of the second moment of mass equation~:
\begin{equation}
    \pddt \textbf{A}
    + \div (
        \pnnavg{\textbf{u}_\alpha}\textbf{A}
    )
    =
    \mathbf{\Omega} \cdot \textbf{A}
    - \textbf{A} \cdot \mathbf{\Omega}
    + \beta\left[
        \textbf{E} \cdot \textbf{A}
        -\textbf{A} \cdot \textbf{E}
        - \textbf{E} : \mathbb{A}
    \right]
    - \div \mathbf{\Sigma}
    \label{eq:hybrid_avg_dt_pp}
\end{equation}
where the fourth-order tensor $\mathbb{A}$, is defined as $\mathbb{A} = \pavg{\textbf{pppp}}$.
In this expression we removed the fluctuation terms, but they must appear and they are surly not negligible. 
In \citet{wang2008objective} they derive \ref{eq:hybrid_avg_dt_pp} by the means of kinetic theory, based on \ref{eq:jefferey} and the fact that $\ddt \textbf{p} = \omega_\alpha \times \textbf{p}$ (Equation (3) of their article).
Their equation is similar to \ref{eq:hybrid_avg_dt_pp} except that their employ a phenomenological closure for the term, $\div \mathbf{\Sigma}$, which account for particles interactions \tb{je pense que c'est ca mais pas sur}.
Anyhow, we showed how it is possible to derive the orientation tensor conservation equation, commonly used in fiber field theory, from the second-order moments of mass's equation. 
However, \ref{eq:jefferey} is valid in low inertial regime only. 
Therefore, it is indispensable to consider a more general framework when working with inertial particles. 

\subsubsection{Averaged equation for the particle phase}

Now let's turn our attention to the particle phase equations in inertial form. 
We first introduce the average of the inertia matrix, angular momentum and internal energy as,
\begin{align}
    \label{eq:Sp_def}
    2(\mathcal{S}_{p})_{ij}
    = - \epsilon_{jak} \omega_{p,a} \mathcal{I}_{p,ki}
    - \epsilon_{ibk}   \omega_{p,b} \mathcal{I}_{p,kj}
    - \epsilon_{jak} k_{p,kia}^{\mathcal{I}\omega} 
    - \epsilon_{ibk} k_{p,kjb}^{\mathcal{I}\omega}
    \\
    \label{eq:mup_def}
    \mu_{p,i}
    = \mathcal{I}_{p,ij} \omega_{p,j}
    + k_{p,ijj}^{\mathcal{I}\omega}
    \\
    \label{eq:Wp_deformations}
    2 W_p 
    = 
    {\omega}_{p,i}{\omega}_{p,j} \mathcal{I}_{p,ij}
    % \omega_{p,i}\mu_{p,i}
    + \mathcal{I}_{p,ij} {k}^{ww}_{p,ij}
    + 2 {\omega}_{p,i} {k}_{p,ijj}^{\mathcal{I}\omega}
    + k^{Iww}_p
\end{align}
where we introduced the following fluctuating quantities, 
\begin{align*}
    n_p(\textbf{k}_p^{\mathcal{I}\omega})_{ijk}
    = 
    \pavg{\mathcal{I}^{\alpha'}_{ij} \omega^{\alpha'}_k}
    && n_p (k_p^{\omega\omega})_{ij}
    = 
    \pavg{\omega^{\alpha'}_i \omega^{\alpha'}_j}
    && n_p{k}_p^{\mathcal{I}\omega\omega}
    = 
    \pavg{\mathcal{I}^{\alpha'}_{ij} \omega^{\alpha'}_i\omega^{\alpha'}_j}
\end{align*}
Due to the covariance terms, the problem which contained only two unknown at the particle scale, i.e. $\mathcal{I}_\alpha$ and $\bm{\omega}_\alpha$, now contains 5 unknown on the macroscopic scale, i.e. $\mathcal{I}_p$, $\bm{\omega}_p$, $\textbf{k}^{\mathcal{I}\omega}_p$,$\textbf{k}^{ww}_p$ and $\textbf{k}_p^{\mathcal{I}\omega\omega}$
We therefore, need to derive five equation to compute properly the angular momentum and internal fluctuating energy of the particle phase. 


Taking the particle average of \ref{eq:dt_I_alpha},\ref{eq:dt_Iomega_alpha} and \ref{eq:dt_Iomegaomega_alpha} yield the particle average equations for the averaged inertia matrix the angular momentum and the internal kinetic energy, 
\begin{align}
    \label{eq:dt_avg_Ip}
    \pddt (n_p\mathcal{I}_{p,ij})
    + \div (n_p\mathcal{I}_{p,ij} u_{p,k}
    + \mathcal{I}^\text{flux}_{p,ijk})
    = n_p \mathcal{S}_{p,ij},\\ 
    \label{eq:dt_avg_mup}
    \pddt (
    %     n_p \mathcal{I}_{p,ij} \omega_{p,j}
    % + n_p k^{\mathcal{I}\omega}_{p,ijj}
    n_p \mu_{p,i}
    )
    + \div (
        n_p \mu_{p,i}u_{p,k}
    %     n_p \mathcal{I}_{p,ij} \omega_{p,j} u_{p,k}
    % + k^{\mathcal{I}\omega}_{p,ijj} u_{p,k}
    +  \omega_{p,j} \mathcal{I}^\text{flux}_{p,ijk}
    + \mu_{p,ik}^\text{flux}
    )
    = n_p t_{p,i},\\
    \pddt (n_p W_p)
    + \div  (
        n_p W_p u_{p,k}
        + \omega_{p,i} \omega_{p,j}
        \mathcal{I}_{p,ij}^\text{flux}/2
        + \omega_{p,i}\mu_{p,ik}^\text{flux}
    + W_{p,k}^\text{flux}
    )
    = 
    n_p \omega_{p,k} t_{p,k}
    +  \pavg{\omega_k^{\alpha'} t^\alpha_k},
\end{align}
respectively. 
Also, within these averaged equations the correlation of the translational velocity and each of these quantities  appear under the form $(\ldots)^\text{flux}$ have the following expression, 
\begin{align*}
    n_p \mathcal{I}_{p,ijk}^\text{flux}
    = n_p k_{p,ijk}^{\mathcal{I}u}
    = 
    \pavg{\mathcal{I}^{\alpha'}_{ij} u_{k}^{\alpha'}}
    &&
    n_p \mu_{p,ik}^\text{flux}
    = 
    \pavg{\mathcal{I}^{\alpha'}_{ij} \omega^{\alpha'}_j u_{k}^{\alpha'}} 
    + \mathcal{I}_{p,ij} \pavg{u_k^{\alpha'}\omega_j^{\alpha'}}\\
    && n_p W_{p,k}^\text{flux}
    = \frac{1}{2}\mathcal{I}_{p,ij}
    \pavg{
        \omega_j^{\alpha'}
        \omega_i^{\alpha'}
        u_k^{\alpha'}
    }
    + \frac{1}{2}\pavg{
        \mathcal{I}_{ij}^{\alpha'}
        \omega_j^{\alpha'}
        \omega_i^{\alpha'}
        u_k^{\alpha'}
    }.
\end{align*}
At this stage all the fluctuating terms $k^{\ldots}$ can be considered as closure term or as terms to be solved thanks to a secondary transport equations in the same spirit as the pseudo turbulent equation for $k_p$.
Upon substituting $\mathcal{I}_p$ with the orientation vector $\textbf{pp}$ the reader can note that \ref{eq:dt_avg_Ip} reveal being the equation for the orientation tensor $\pavg{\textbf{pp}}$ \citep{wang2008objective} which is widely used in the fiber suspension study. 
In the context of stokesian suspension the term appearing in the RHS of \ref{eq:dt_avg_Ip} is directly closed through the use of Jeffery's equation \citet{guazzelli2011}.
It leads to a closed equation for the orientation tensor in terms of the derivative of the mean flow field. 
In a more general context, to close the inertia tensor equation and compute the total angular momentum, one needs to obtain an equation for $k^{\mathcal{I}\omega}_{p,ijj}$. 
This is done by first taking the dot product of $\bm{\omega}_{p}$ with \ref{eq:dt_avg_Ip} on the index $j$, then subtracting this expression to \ref{eq:dt_avg_mup}.
This gives rise to an equation for the mean angular velocity and mean fluctuating part of the angular momentum, namely, 
\begin{align}
    \label{eq:dt_avg_kIp}
    \pddt (n_p\mathcal{I}_{p,ij}\omega_{p,j})
    + \div (n_p\mathcal{I}_{p,ij}\omega_{p,j} u_{p,k}
    + \omega_{p,j}\mathcal{I}^\text{flux}_{p,ijk})
    &= 
    n_p \mathcal{S}_{p,ij}\omega_{p,j}
    + \mathcal{I}_{p,ijk}^\text{flux}\grad \omega_{p,j}
    + \mathcal{I}_{p,ij}\pavg{\dot{\omega}^\alpha_j}
    \\
    \label{eq:dt_avg_kIp}
    \mathcal{I}_{p,ij}\left[
        \pddt (n_p\omega_{p,j})
        + \div (n_p\omega_{p,j} u_{p,k})
    \right]
    &= 
    \omega_{p,j}\div\mathcal{I}_{p,ijk}^\text{flux} 
    + \mathcal{I}_{p,ij}\pavg{\dot{\omega}^\alpha_j},
    \\
    \pddt (n_p k^{\mathcal{I}\omega}_{p,ijj})
    + \div (k^{\mathcal{I}\omega}_{p,ijj} u_{p,k}
    + \mu_{p,ik}^\text{flux}
    )
    &= 
    - n_p \mathcal{S}_{p,ij}\omega_{p,j}
    - \mathcal{I}_{p,ijk}^\text{flux}\grad \omega_{p,j}
    - \mathcal{I}_{p,ij}\pavg{\dot{\omega}^\alpha_j}
    + n_p t_{p,i},
\end{align}
respectively. 
The third term in the right hands side of this equation and the mean torque term eventually cancel out leaving only with fluctuating quantities. 
% Now let's turn our attention to the energy parts, 
% \begin{align*}
%     \pddt (n_p\mathcal{I}_{p,ij}\omega_{p,j}\omega_{p,i})
%     + \div (n_p\mathcal{I}_{p,ij}\omega_{p,j}\omega_{p,i} u_{p,k}
%     + \omega_{p,j}\omega_{p,i}\mathcal{I}^\text{flux}_{p,ijk})
%     = 
%     n_p \mathcal{S}_{p,ij}\omega_{p,j}\omega_{p,i}\\
%     + \mathcal{I}_{p,ijk}^\text{flux}\grad (\omega_{p,j}\omega_{p,i})
%     + \mathcal{I}_{p,ij}(
%         \omega_{p,i}\pavg{\dot{\omega}^\alpha_j}
%         +  \omega_{p,j}\pavg{\dot{\omega}^\alpha_i}
%     )
%     \\
%     \pddt (n_p \omega_{p,i}k^{\mathcal{I}\omega}_{p,ijj})
%     + \div (\omega_{p,i}k^{\mathcal{I}\omega}_{p,ijj} u_{p,k}
%     + \omega_{p,i}\mu_{p,ik}^\text{flux}
%     )
%     = 
%     - \omega_{p,i} n_p \mathcal{S}_{p,ij}\omega_{p,j}
%     - \omega_{p,i} \mathcal{I}_{p,ijk}^\text{flux}\grad \omega_{p,j}\\
%     - \omega_{p,i} \mathcal{I}_{p,ij}\pavg{\dot{\omega}^\alpha_j}
%     + \omega_{p,i} n_p t_{p,i}
%     + k^{\mathcal{I}\omega}_{p,ijj} \pavg{\dot{\omega}^\alpha_i}
%     + \mu_{p,ik}^\text{flux} \grad\omega_{p,i}
% \end{align*}
Subtracting from the energy equation half of \ref{eq:dt_avg_Ip} times $\omega_{p,i}\omega_{p,j}$, and \ref{eq:dt_avg_kIp} time $\omega_{p,i}$ gives us the last equation for the angular velocity covariance terms, namely, 
\begin{multline}
    \pddt (n_p (\mathcal{I}_{p,ij} k^{\omega\omega}_{p,ij}+k^{\mathcal{I}\omega\omega}_{p}))
    + \div  (
        n_p (\mathcal{I}_{p,ij} k^{\omega\omega}_{p,ij}+k^{\mathcal{I}\omega\omega}_{p}) u_{p,k}
    + W_{p,k}^\text{flux}
    )
    % = 
    % n_p \omega_{p,k} t_{p,k}
    % + \pavg{\omega_k^{\alpha'} t^\alpha_k} \\
    % + \omega_{p,i} n_p \mathcal{S}_{p,ij}\omega_{p,j}
    % + \omega_{p,i} \mathcal{I}_{p,ijk}^\text{flux}\grad \omega_{p,j}
    % + \omega_{p,i} \mathcal{I}_{p,ij}\pavg{\dot{\omega}^\alpha_j}
    % - \omega_{p,i} n_p t_{p,i}
    % - k^{\mathcal{I}\omega}_{p,ijj} \pavg{\dot{\omega}^\alpha_i}
    % - \mu_{p,ik}^\text{flux} \grad\omega_{p,i}\\
    % - \frac{1}{2} n_p \mathcal{S}_{p,ij}\omega_{p,j}\omega_{p,i}
    % - \frac{1}{2} \mathcal{I}_{p,ijk}^\text{flux}\grad (\omega_{p,j}\omega_{p,i})
    % - \frac{1}{2} \mathcal{I}_{p,ij}(
    %     \omega_{p,i}\pavg{\dot{\omega}^\alpha_j}
    %     +  \omega_{p,j}\pavg{\dot{\omega}^\alpha_i}
    % )\\
    = \\
    \pavg{\omega_k^{\alpha'} t^\alpha_k} 
    + \frac{n_p}{2}\omega_{p,i} \omega_{p,j} \mathcal{S}_{p,ij}
    % + \omega_{p,i} \mathcal{I}_{p,ijk}^\text{flux}\grad \omega_{p,j}
    % + \omega_{p,i} \mathcal{I}_{p,ij}\pavg{\dot{\omega}^\alpha_j}
    % - \omega_{p,i} n_p t_{p,i}
    - k^{\mathcal{I}\omega}_{p,ijj} \pavg{\dot{\omega}^\alpha_i}
    - \mu_{p,ik}^\text{flux} \grad\omega_{p,i}
    % - \frac{1}{2} n_p \mathcal{S}_{p,ij}\omega_{p,j}\omega_{p,i}
    % - \frac{1}{2} \mathcal{I}_{p,ijk}^\text{flux}\grad (\omega_{p,j}\omega_{p,i})
    % - \frac{1}{2} \mathcal{I}_{p,ij}(
    %     \omega_{p,i}\pavg{\dot{\omega}^\alpha_j}
    %     +  \omega_{p,j}\pavg{\dot{\omega}^\alpha_i}
    % )
    \label{eq:dt_avg_kpww}
\end{multline}
However, we can observe that the exchange term appearing in the fluid phase turbulent equation is also present here. 
Meaning that the correlation of the torque with the angular velocity makes the energy transfer between the fluid turbulent energy $k_1$ and the particles fluctuating motion $\mathcal{I}_{p,ij} k^{\omega\omega}_{p,ij}+k^{\mathcal{I}\omega\omega}_{p}$. 
It is now complicated to see how to isolate $k^{\omega\omega}_{p,ij}$ from $k^{\mathcal{I}\omega\omega}_{p}$ into two distinct equation ,however it is really needed ? 
In the specific case of solid spherical particles all the quantities involving the inertial matrix contribution vanish due to the frame independence.
Upon noticing that, compare \ref{eq:dt_avg_kpww} with Equation (9.28) of \citet[Chapter 9]{rao2008introduction}. 
Their equation, derived with the help of kinetic theory remain consistent with \ref{eq:dt_avg_kpww} however in the form of the hybrid model we clearly see which exchange terms communicate with which equations. 


% In order to isolate these terms we must derive the averaged equation for $(\omega\omega)$. 
% \begin{multline*}
%     \pddt (n_p \omega_{p,i}\omega_{p,j} 
%     +n_p k^{\omega\omega}_{p,ij})
%     + \div (n_p \omega_{p,i}\omega_{p,j}u_{p,k} 
%     +n_p k^{\omega\omega}_{p,ij}u_{p,k}
%     + \omega^\text{flux}_{p,ijk})
%     = \\
%     \pavg{\omega_j^\alpha
%     (\mathcal{I}^\alpha)_{ik}^{-1}(  
%         t^\alpha_k
%     -  \mathcal{S}_{kl}^\alpha\omega_l^\alpha
%     )}
%     + \pavg{\omega_i^\alpha
%     (\mathcal{I}^\alpha)_{jk}^{-1}(  
%         t^\alpha_k
%     -  \mathcal{S}_{kl}^\alpha\omega_l^\alpha
%     )}
% \end{multline*}
% Which upon multiplying by $\mathcal{I}_{p,ij}$ gives the last equations, 
% \begin{multline*}
%     \pddt (n_p \omega_{p,i}\omega_{p,j} \mathcal{I}_{p,ij}
%     +n_p k^{\omega\omega}_{p,ij}\mathcal{I}_{p,ij}
%     )
%     + \div (n_p \omega_{p,i}\omega_{p,j}u_{p,k}  \mathcal{I}_{p,ij}
%     +n_p k^{\omega\omega}_{p,ij}u_{p,k} \mathcal{I}_{p,ij}
%     + \omega^\text{flux}_{p,ijk} \mathcal{I}_{p,ij}
%     )
%     = \\
%     +\omega_{p,ijk}^\text{flux}\grad \mathcal{I}_{p,ij}
%     + 2 \mathcal{I}_{p,ij} 
%     \pavg{\ddt(\omega_i^\alpha\omega_j^\alpha)}
    % \\
    % \mathcal{I}_{p,ij}\pavg{\omega_j^\alpha
    % (\mathcal{I}^\alpha)_{ik}^{-1}(  
    %     t^\alpha_k
    % -  \mathcal{S}_{kl}^\alpha\omega_l^\alpha
    % )}
    % + \mathcal{I}_{p,ij}\pavg{\omega_i^\alpha
    % (\mathcal{I}^\alpha)_{jk}^{-1}(  
    %     t^\alpha_k
    % -  \mathcal{S}_{kl}^\alpha\omega_l^\alpha
    % )}
% \end{multline*}
% Now by subtracting from this equation the equation of the mean angular momentum gives us, 
% \begin{multline*}
%     \pddt (
%     n_p k^{\omega\omega}_{p,ij}\mathcal{I}_{p,ij}
%     )
%     + \div (
%     n_p k^{\omega\omega}_{p,ij}u_{p,k} \mathcal{I}_{p,ij}
%     + \omega^\text{flux}_{p,ijk} \mathcal{I}_{p,ij}
%     - \omega_{p,i}\omega_{p,j}\mathcal{I}_{p,ijk}^\text{flux}
%     )
%     = 
%     \omega_{p,ijk}^\text{flux}\grad \mathcal{I}_{p,ij}
%     + 2 \mathcal{I}_{p,ij} 
%     \pavg{\ddt(\omega_i^\alpha\omega_j^\alpha)}
%     \\
%     - n_p \mathcal{S}_{p,ij}\omega_{p,j}\omega_{p,i}
%     - \mathcal{I}_{p,ijk}^\text{flux}\grad (\omega_{p,j}\omega_{p,i})
%     - \mathcal{I}_{p,ij}(
%         \omega_{p,i}\pavg{\dot{\omega}^\alpha_j}
%         +  \omega_{p,j}\pavg{\dot{\omega}^\alpha_i}
%     )
% \end{multline*}
% which finally gives us the equation for the variance of the angular velocity. 

% Let's now derive a transport equation for the correlation tensor $\mathcal{I}^\text{flux}_{ijk}$. 
% \begin{equation*}
%     \pddt(n_p \mathcal{I}_{p,ij} u_{p,k} + \mathcal{I}^\text{flux}_{p,ijk})
%     + \div(\mathcal{I}_{p,ij} u_{p,k} u_{p,l}
%     + \mathcal{I}^\text{flux}_{p,ijk} u_{p,l}
%     + 
%     )
%     = 
%        \pavg{u_k^\alpha \mathcal{S}_{ij}^\alpha}
%     +  \pavg{\mathcal{I}_{ij}^\alpha }g_k
%     +  \pavg{\mathcal{I}_{ij}^\alpha f_k^\alpha/m^\alpha}
% \end{equation*}


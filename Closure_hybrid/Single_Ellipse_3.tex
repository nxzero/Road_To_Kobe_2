
\section*{Re-derivation but more easy}
All these went way to complicated. 
So here is a more consize derivation that does keep track of the physics. 
We describe the geometry of the particle with the second moment of mass tensor, 
\begin{equation*}
    M_{\alpha,ij} 
    =\frac{m_\alpha a^2}{5} M_{\alpha,ij}^*
    = \frac{m_\alpha a^2}{5}\left[
        \left(
            \frac{a_1}{a}
        \right)^2 p_i p_j
        + 
        \left(
            \frac{a_1}{a}
        \right)^2(\delta_{ij}-  p_i p_j)
    \right]
\end{equation*}
This tensor has several notable properties, the volume conservation : $a_1 a_2^2 = a^3$ which means that $\textbf{M}_\alpha^*$ has two eigenvalues linked through $M^1 = (M^2)^{-2}$ or $M^2 = (M^1)^(-1/2)$. 
At small deformation, i.e. $M_1 - 1\ll 1$ we therefore have $M^1 = 1 - (M^1-1)/2$.
This ultimately means that at small deformation the trace is a constant namely $M_{ij} \delta_{ij} = 3+\mathcal{O}((M^1 -1)^2)$. 




Let's now reformulate the integral of the problem namely, 
\begin{align}
    \intO{(\textbf{rw}_ 2^0 )_{ij}+ (\textbf{w}_2^0 \textbf{r})_{ij}} 
    = \textbf{M}_{\alpha,ik} \cdot \bm\Gamma_{\alpha,jk}
        +  \bm\Gamma_{\alpha,ik} \cdot \textbf{M}_{\alpha,jk}
    \\
    \intO{\rho_2 \textbf{w}_{2,i}^0\textbf{w}_{2,j}^0}
    = \bm\Gamma_{\alpha,jl}\bm\Gamma_{\alpha,ik} \textbf{M}_{\alpha,kl}  
        +\intO{\rho_2 \textbf{w}_{2,i}^s\textbf{w}_{2,j}^s}
    \\
    \intO{\bm\sigma_{2,ij}^0}
    =
    2 \mu_2 v_\alpha \textbf{E}_{\alpha,ij}
    - \intO{p_2^0} \textbf{I}_{ij}
    + \mu_2 \intS{(\textbf{n}_i \textbf{w}_{2,j}^s + \textbf{n}_j \textbf{w}_{2,i}^s)}
    \\
    \intS{\bm\sigma_{I,ij}^0}
    = \frac{\gamma v_\alpha }{a} \left[
        2\textbf{I}_{ij} 
        - \frac{4  }{5} (\textbf{M}_{\alpha,ij}^* - \textbf{I}_{\alpha,ij})
    \right]
    +\mathcal(O)(\textbf{C}\cdot \textbf{C})\\
    s_\alpha 
    = 4\pi a^2 (1+\frac{\textbf{C}:\textbf{C}}{15})
\end{align}
for the surface tension term is made of two terms, one corresponding to Laplace pressure and the other to the deviatoric stress. 




The equation for the orientation, second moment of mass torque, symmetric moment of momentum and trace of the moment of momentum reads, 
\begin{align*}
    \ddt \textbf{pp}_{\alpha,ij}
    = \textbf{pp}_{\alpha,ik} \cdot \bm\Omega_{\alpha,jk}
    +  \bm\Omega_{\alpha,ik} \cdot \textbf{pp}_{\alpha,jk}\\
    \ddt \textbf{M}_{\alpha,ij}
    = \textbf{M}_{\alpha,ik} \cdot \bm\Gamma_{\alpha,jk}
    +  \bm\Gamma_{\alpha,ik} \cdot \textbf{M}_{\alpha,jk}\\
    \ddt (\textbf{I}_{\alpha,ik}\bm\omega_{\alpha,k} )
    = 
    \intS{(\textbf{r}\times\bm\sigma_1^0\cdot \textbf{n})_i} \\
    \frac{1}{2}\ddt^2 \textbf{M}_{\alpha,ij}
    -  \bm\Gamma_{\alpha,jl}\bm\Gamma_{\alpha,ik} \textbf{M}_{\alpha,kl}  
    + 2 \mu_2 v_\alpha \textbf{E}_{\alpha,ij}
    + \frac{\gamma v_\alpha }{a} \left[
    2\textbf{I}_{ij} 
    - \frac{4 \gamma v_\alpha }{5 a} (\textbf{M}_{\alpha,ij}^* - \textbf{I}_{\alpha,ij})
    \right]\\
    = 
    \frac{1}{2}\intS{(\textbf{r}\bm\sigma_1^0 + \bm\sigma_1^0\textbf{r})\cdot \textbf{n}} 
    + \intO{\rho_2 \textbf{w}_{2,i}^s\textbf{w}_{2,j}^s}
    + \intO{p_2^0} \textbf{I}_{ij}
    - \mu_2 \intS{(\textbf{n}_i \textbf{w}_{2,j}^s + \textbf{n}_j \textbf{w}_{2,i}^s)}\\
    % \frac{1}{2}\ddt^2 \textbf{M}_{\alpha,mm}
    -  \bm\Gamma_{\alpha,ml}\bm\Gamma_{\alpha,mk} \textbf{M}_{\alpha,kl}  
    + \frac{\gamma v_\alpha }{a} 
    % \left[
    2\textbf{I}_{mm} 
    % - \frac{4 }{5 } (\textbf{M}_{\alpha,mm}^* - \textbf{I}_{\alpha,mm})
    % \right]
    % \\
    = 
    \intS{\textbf{r}_m\cdot\bm\sigma_{1,mk}^0\cdot \textbf{n}_k} 
    + \intO{\rho_2 \textbf{w}_{2,m}^s\cdot \textbf{w}_{2,m}^s}
    + \intO{p_2^0} \textbf{I}_{mm}
\end{align*}

As the derivative of $M_{ij}$ in these equations actually takes in account the change of orientation of the particle it might be useful to re-derive these equations in the eigenbasis of the matrix $M_{ij}$

To the first order in deformation the deviatoric part reads, 
\begin{align*}
    \frac{1}{2}\ddt^2 \textbf{M}_{\alpha,ij}
    -   \textbf{M}_{\alpha,kl} 
    (\bm\Gamma_{\alpha,jl}\bm\Gamma_{\alpha,ik}  
    - \frac{1}{3}
    \bm\Gamma_{\alpha,ml}\bm\Gamma_{\alpha,mk}  
    \textbf{I}_{ij}
    )
    + 2 \mu_2 v_\alpha \textbf{E}_{\alpha,ij}
    - \frac{\gamma v_\alpha }{a} 
    \frac{4  }{5} (\textbf{M}_{\alpha,ij}- \textbf{I}_{ij})
    \\
    = 
    \frac{1}{2}\intS{(\textbf{r}\bm\sigma_1^0 + \bm\sigma_1^0\textbf{r} - \frac{2}{3}\textbf{r}\cdot \bm\sigma_1^0 \textbf{I})\cdot \textbf{n}} 
    + \intO{\rho_2 (\textbf{w}_{2,i}^s\textbf{w}_{2,j}^s - \frac{1}{3}\textbf{w}_{2,m}^s\textbf{w}_{2,m}^s \textbf{I}_{ij}) }
    - \mu_2 \intS{(\textbf{n}_i \textbf{w}_{2,j}^s + \textbf{n}_j \textbf{w}_{2,i}^s)}\\
\end{align*}
It can be somewhat useful to extract the proportionality coefficient of each terms. 
Denotining dimensionless qte by a $*$ we have, 
\begin{align*}
    \frac{\rho_2 a^2}{5 \tau^2}\left[\frac{1}{2}\ddt^2 \textbf{M}_{\alpha,ij}^*
    -  \bm\Gamma_{\alpha,jl}^*\bm\Gamma_{\alpha,ik}^* \textbf{M}_{\alpha,kl}^*
    \right]  
    + \frac{2 \mu_2  }{\tau} \textbf{E}_{\alpha,ij}^*
    + \frac{\gamma  }{a} \left[
    2\textbf{I}_{ij} 
    - \frac{4 }{5} (\textbf{M}_{\alpha,ij}^* - \textbf{I}_{\alpha,ij})
    \right]\\
    = 
    \frac{1}{2}\intS{(\textbf{r}\bm\sigma_1^0 + \bm\sigma_1^0\textbf{r})\cdot \textbf{n}} 
    + \intO{\rho_2 \textbf{w}_{2,i}^s\textbf{w}_{2,j}^s}
    + \intO{p_2^0} \textbf{I}_{ij}
    - \mu_2 \intS{(\textbf{n}_i \textbf{w}_{2,j}^s + \textbf{n}_j \textbf{w}_{2,i}^s)}\\
\end{align*}

The forcing terms are really problem dependent. 
Thus, at this stage it cannot be scaled in terms of timescale. 

In order to be more consis, we may write :
\begin{align*}
    \textbf{F}_{ij}(\textbf{w}^s,\bm\sigma_1^0)
    = 
    \frac{1}{2}\intS{(\textbf{r}\bm\sigma_1^0 + \bm\sigma_1^0\textbf{r} - \frac{2}{3}\textbf{r}\cdot \bm\sigma_1^0 \textbf{I})\cdot \textbf{n}} 
    + \intO{\rho_2 (\textbf{w}_{2,i}^s\textbf{w}_{2,j}^s - \frac{1}{3}\textbf{w}_{2,m}^s\textbf{w}_{2,m}^s \textbf{I}_{ij}) }
    - \mu_2 \intS{(\textbf{n}_i \textbf{w}_{2,j}^s + \textbf{n}_j \textbf{w}_{2,i}^s)}\\
\end{align*} 
Indicating that the forcing term is related to the exterior contribution and the external stresses. 

\begin{align*}
    \ddt^2 \textbf{M}_{\alpha,ij}^*
    -2  \textbf{M}_{\alpha,kl}^* 
    (\bm\Gamma_{\alpha,jl}^*\bm\Gamma_{\alpha,ik}^*  
    - \frac{1}{3}
    \bm\Gamma_{\alpha,ml}^*\bm\Gamma_{\alpha,mk}^*  
    \textbf{I}_{ij}
    )
    + \frac{10 \mu_2 \tau}{ \rho_2 a^2} 2\textbf{E}_{\alpha,ij}^*
    - 8 \frac{\tau^2 \gamma  }{\rho_2 a^3} 
     (\textbf{M}_{\alpha,ij}^*- \textbf{I}_{ij})\\
    = 
    \frac{\tau^2 10}{a^2 m_\alpha}
    \left[\textbf{F}_{\sigma_1, ij}
    + \textbf{F}_{ww, ij}
    + \textbf{F}_{\sigma_2, ij}\right]
\end{align*}

All these unclose term are really problem dependent, for now all we now is that at low Reynolds number we are in the viscous regime, besides this external flow that drives these scales. 
Meaning that, 
\begin{align*}
    \intO{\rho_2 \textbf{w}_{2,i}^0\textbf{w}_{2,j}^0}
    \sim \frac{m_\alpha a^2}{\tau_u^2} \textbf{F}_{ww}^*
    \\
    \mu_2 \intS{(\textbf{n}_i \textbf{w}_{2,j}^s + \textbf{n}_j \textbf{w}_{2,i}^s)}
    \sim \frac{v_\alpha \mu_2}{\tau_u} \textbf{F}_{e}^*
    \\
    \intS{\bm\sigma_1^0 \textbf{rn}}
    \sim 
    \frac{v_\alpha \mu_1}{\tau_u} \textbf{F}_{\sigma}^*
\end{align*} 

\begin{align*}
    \ddt^2 \textbf{M}_{\alpha,ij}^*
    -2  \textbf{M}_{\alpha,kl}^* 
    (\bm\Gamma_{\alpha,jl}^*\bm\Gamma_{\alpha,ik}^*  
    - \frac{1}{3}
    \bm\Gamma_{\alpha,ml}^*\bm\Gamma_{\alpha,mk}^*  
    \textbf{I}_{ij}
    )
    + \frac{10 \mu_2 \tau}{ \rho_2 a^2} 2\textbf{E}_{\alpha,ij}^*
    - 8 \frac{\tau^2 \gamma  }{\rho_2 a^3} 
     (\textbf{M}_{\alpha,ij}^*- \textbf{I}_{ij})\\
    = 
    \frac{\tau^2 10 \mu_1 }{a^2 \rho_2 \tau_u}
    \textbf{F}_{\sigma_1, ij}
    + \frac{\tau^2 10}{\tau_u^2} \textbf{F}_{ww, ij}
    + \frac{\tau^2 10 \mu_1 }{a^2 \rho_2 \tau_u} \textbf{F}_{\sigma_2, ij}
\end{align*}


\subsubsection*{Re-derivation but with dimensionless scaling}

\begin{align}
    \intO{(\textbf{rw}_ 2^0 )_{ij}+ (\textbf{w}_2^0 \textbf{r})_{ij}} 
    = \textbf{M}_{\alpha,ik} \cdot \bm\Gamma_{\alpha,jk}
        +  \bm\Gamma_{\alpha,ik} \cdot \textbf{M}_{\alpha,jk}
    \\
    \intO{\rho_2 \textbf{w}_{2,i}^0\textbf{w}_{2,j}^0}
    = \bm\Gamma_{\alpha,jl}\bm\Gamma_{\alpha,ik} \textbf{M}_{\alpha,kl}  
        +\intO{\rho_2 \textbf{w}_{2,i}^s\textbf{w}_{2,j}^s}
    \\
    \intO{\bm\sigma_{2,ij}^0}
    =
    2 \mu_2 v_\alpha \textbf{E}_{\alpha,ij}
    - \intO{p_2^0} \textbf{I}_{ij}
    + \mu_2 \intS{(\textbf{n}_i \textbf{w}_{2,j}^s + \textbf{n}_j \textbf{w}_{2,i}^s)}
    \\
    \intS{\bm\sigma_{I,ij}^0}
    = \frac{\gamma v_\alpha }{a} \left[
        2\textbf{I}_{ij} 
        - \frac{4  }{5} (\textbf{M}_{\alpha,ij}^* - \textbf{I}_{\alpha,ij})
    \right]
    +\mathcal(O)(\textbf{C}\cdot \textbf{C})\\
    s_\alpha 
    = 4\pi a^2 (1+\frac{\textbf{C}:\textbf{C}}{15})
\end{align}

Then, 
\begin{align*}
    \frac{\rho_2 a^2}{5}\left[
        \frac{1}{\tau^2}\frac{1}{2}\ddt^2 \textbf{M}_{\alpha,ij}
    -   \frac{1}{\tau^2}\bm\Gamma_{\alpha,jl}\bm\Gamma_{\alpha,ik} \textbf{M}_{\alpha,kl}  
    - \frac{1}{\tau_u^2} \textbf{F}_{ww}^*
    \right]\\
    + \mu_2  \left[
        \frac{1}{\tau} 2\textbf{E}_{\alpha,ij}
    + \frac{1}{\tau_u} \textbf{F}_\sigma^*
    \right]\\
    + \frac{\gamma  }{a} \left[
    2\textbf{I}_{ij} 
    - \frac{4  }{5} (\textbf{M}_{\alpha,ij}^* - \textbf{I}_{\alpha,ij})
    \right]\\
    = 
    \frac{1}{2}
    \frac{ \mu_1}{\tau_u} 
    \textbf{F}_{\sigma_1}^*
    % \intS{(\textbf{r}\bm\sigma_1^0 + \bm\sigma_1^0\textbf{r})\cdot \textbf{n}} 
    % + \intO{p_2^0} \textbf{I}_{ij}
\end{align*}

which then, 
\begin{align*}
    \frac{\rho_2}{\rho_1}\frac{\rho_1 a^2}{5\mu_1\tau_u}\left[
        \frac{\tau_u^2}{\tau^2}\frac{1}{2}\ddt^2 \textbf{M}_{\alpha,ij}
    -   \frac{\tau_u^2}{\tau^2}\bm\Gamma_{\alpha,jl}\bm\Gamma_{\alpha,ik} \textbf{M}_{\alpha,kl}  
    - \textbf{F}_{ww}^*
    \right]\\
    + \frac{\mu_2}{\mu_1}  \left[
        \frac{\tau_u}{\tau} 2\textbf{E}_{\alpha,ij}
    +  \textbf{F}_\sigma^*
    \right]\\
    + \frac{\gamma \tau_u }{a\mu_1} \left[
    2\textbf{I}_{ij} 
    - \frac{4  }{5} (\textbf{M}_{\alpha,ij}^* - \textbf{I}_{\alpha,ij})
    \right]\\
    = 
    \frac{1}{2}
    \textbf{F}_{\sigma_1}^*
    % \intS{(\textbf{r}\bm\sigma_1^0 + \bm\sigma_1^0\textbf{r})\cdot \textbf{n}} 
    % + \intO{p_2^0} \textbf{I}_{ij}
\end{align*}

In terms of dimensionless groups
\begin{align*}
    \zeta Re\left[
        \beta^2 \frac{1}{2}\ddt^2 \textbf{M}_{\alpha,ij}
    -   \beta^2 \bm\Gamma_{\alpha,jl}\bm\Gamma_{\alpha,ik} \textbf{M}_{\alpha,kl}  
    - \textbf{F}_{ww}^*
    \right]\\
    + \lambda  \left[
        \beta 2\textbf{E}_{\alpha,ij}
    +  \textbf{F}_\sigma^*
    \right]\\
    + \frac{1}{Ca} \left[
    2\textbf{I}_{ij} 
    - \frac{4  }{5} (\textbf{M}_{\alpha,ij}^* - \textbf{I}_{\alpha,ij})
    \right]\\
    = 
    \frac{1}{2}
    \textbf{F}_{\sigma_1}^*
    % \intS{(\textbf{r}\bm\sigma_1^0 + \bm\sigma_1^0\textbf{r})\cdot \textbf{n}} 
    % + \intO{p_2^0} \textbf{I}_{ij}
\end{align*}


\subsubsection*{Slow flows wheer beta equal 0 }

\textbf{Steady state equillibrium }
\begin{align*}
    - \zeta Re \textbf{F}_{ww}^*
    + \lambda  \textbf{F}_\sigma^*
    + \frac{1}{Ca} \left[
    2\textbf{I}_{ij} 
    - \frac{4  }{5} (\textbf{M}_{\alpha,ij}^* - \textbf{I}_{\alpha,ij})
    \right]
    = 
    \frac{1}{2}
    \textbf{F}_{\sigma_1}^*
    % \intS{(\textbf{r}\bm\sigma_1^0 + \bm\sigma_1^0\textbf{r})\cdot \textbf{n}} 
    % + \intO{p_2^0} \textbf{I}_{ij}
\end{align*}

In the last part i want to be able to neglect all the components on the left

if the reynolds number is indeed negligible we have 

\begin{align*}
    + \lambda  \textbf{F}_\sigma^*
    + \frac{1}{Ca} \left[
    2\textbf{I}_{ij} 
    - \frac{4  }{5} (\textbf{M}_{\alpha,ij}^* - \textbf{I}_{\alpha,ij})
    \right]
    = 
    \frac{1}{2}
    \textbf{F}_{\sigma_1}^*
    % \intS{(\textbf{r}\bm\sigma_1^0 + \bm\sigma_1^0\textbf{r})\cdot \textbf{n}} 
    % + \intO{p_2^0} \textbf{I}_{ij}
\end{align*}
which give the equillibrium of surface tension internal / external stresses
This must be respected for almost all steady state equilibrium presented in appendix
It is clear that if either $\lambda = 0$ or that we are looking for the fluid phase stress then the int on the left vanish. 
And we are left with an equillibrium between surface tension and stresslet

\tb{In this regime one might recognize Laplace law}

\textbf{Drop in air }
\begin{align*}
    \zeta Re\left[
         \frac{1}{2}\ddt^2 \textbf{M}_{\alpha,ij}
    -    \bm\Gamma_{\alpha,jl}\bm\Gamma_{\alpha,ik} \textbf{M}_{\alpha,kl}  
    \right]
    +   
        \frac{\lambda}{\beta} 2\textbf{E}_{\alpha,ij}
    + \frac{1}{Ca\beta^2} \left[
    2\textbf{I}_{ij} 
    - \frac{4  }{5} (\textbf{M}_{\alpha,ij}^* - \textbf{I}_{\alpha,ij})
    \right]\\
    = 
    0 
\end{align*}
Indeed, in this case we still consier that the surface tension is comparable with the ratio of time scale of course. 
Besides, 
\tb{In this regime one might recognize Lamb's equation }



\subsubsection*{The local basis equation}

Any second order tensor $\textbf{A}$ can be express in the vector $\{\textbf{p}^1,\textbf{p}^2,\textbf{p}^3\}$ which form an orthonormal basis. 
Indeed, 
\begin{equation}
    A_{ij}
    = 
    A^{ab}
    p_i^a
    p_j^b
\end{equation}
with, $A^{ab}$ the components of \textbf{A} in the basis $\{\textbf{p}^1,\textbf{p}^2,\textbf{p}^3\}$. 
These components can be obtained through the operation, 
\begin{equation*}
    A_{ij} 
    p_i^a
    p_j^b
    = 
    A^{cd}
    p_i^c
    p_j^d
    p_i^a
    p_j^b
    = 
    A^{ab}
\end{equation*}
From there it is easy to understand that to obtain the evolution equation in the eigenbasis one has to multiply the previous set of equation with $p_i^ap_j^b$. 

Some properties of the change of basis must be stated. 
If, 
\begin{equation}
    \Omega_{ij} = \frac{1}{2} [A_{ij}-A_ji]
\end{equation}
Then, 
\begin{equation}
    \Omega^{ab} = \frac{1}{2} [A^{ba}-A^{ab}]
\end{equation}
Since, 
\begin{equation}
    \Omega_{ij} = \frac{1}{2} [A^{ab} p^a_i p^b_j-A^{ba} p^b_j p^a_i]
    =  \frac{1}{2} [A^{ab} - A_{ba} ]p^b_j p^a_i
    =  \Omega^{ab} p^b_j p^a_i
\end{equation}
where $\bm\Omega$ is a skew symmetric tensor. 


Since all the $\textbf{p}^a$ rotate according to the same angular velocity we can write, $\ddt p_i^a =\Omega_{ik} p_k^a$, and more generally, 
\begin{equation*}
    \ddt(p_i^ap_j^b)
    = 
    \Omega_{ik} p_k^ap_j^b
    + \Omega_{jk} p_i^ap_k^b
\end{equation*}
with this equation one is able to reformulate any equation of the system above.
To get the second derivative of this tensor we can directly carry out the operation,  
\begin{align*}
    \ddt\ddt(p_i^ap_j^b)
    = 
    \ddt (\Omega_{ik} p_k^ap_j^b)
    + \ddt (\Omega_{jk} p_i^ap_k^b)\\
    = 
    \dot{\Omega}_{ik} p_k^ap_j^b
    + \dot{\Omega}_{jk} p_i^ap_k^b
    + (
        \Omega_{kl} p_l^ap_j^b
    + \Omega_{jl} p_k^ap_l^b
    )\Omega_{ik}
    + (\Omega_{il} p_l^ap_k^b
    + \Omega_{kl} p_i^ap_l^b)\Omega_{jk}\\
    = 
    \dot{\Omega}_{ik} p_k^ap_j^b
    + \dot{\Omega}_{jk} p_i^ap_k^b
    + \Omega_{ik}\Omega_{kl} p_l^ap_j^b
    + \Omega_{ik}\Omega_{jl} p_k^ap_l^b
    + \Omega_{jk}\Omega_{il} p_l^ap_k^b
    + \Omega_{jk}\Omega_{kl} p_i^ap_l^b\\
\end{align*}

Multiplying the equation for \textbf{M} by $\textbf{PP}$ reads, 
\begin{equation*}
    \ddt M^{ab}
    = 
    M^{ac} E^{bc} 
    + E^{ac} M^{bc}
\end{equation*}

The stresslet equation can be written by the same operation but first notice that, 
\begin{align}
    p_i^ap_j^b\ddt^2 M_{ij}
    = 
    \ddt( p_i^ap_j^b \ddt M_{ij})
    - \ddt (p_i^ap_j^b )\ddt M_{ij}\\
    = 
    \ddt( \ddt (p_i^ap_j^b M_{ij}))
    - \ddt(   M_{ij} \ddt  p_i^ap_j^b )
    - \ddt (p_i^ap_j^b )\ddt M_{ij}\\
    = 
    \ddt^2 M^{ab}
    - 2 \ddt (p_i^ap_j^b) \ddt M_{ij}
    - M_{ij} \ddt^2 (p_i^ap_j^b)
    \\
\end{align}
Also, 

The second term reads, 
\begin{align*}
    M_{ij} \ddt^2 (p_i^ap_j^b)
    = M^{cb} \dot{\Omega}^{ca}
    + M^{ac} \dot{\Omega}^{cb}
    + M^{cd} \Omega^{ca}\Omega^{db}
    + M^{cd} \Omega^{ca}\Omega^{db}
    + M^{cb} \Omega^{cd}\Omega^{da} 
    + M^{ac} \Omega^{cd}\Omega^{db} \\
\end{align*}
Must figure out if $\textbf{M}\cdot\bm\Omega$ vanish in the local basis. 
Also, 
\begin{align*}
    \ddt (p_i^ap_j^b) \ddt M_{ij}
    = 
    (\Omega_{ik} p_k^ap_j^b
    + \Omega_{jk} p_i^ap_k^b)
    (M_{il} \Gamma_{jl}
    +  \Gamma_{il} M_{jl})\\
    = 
    + M_{cd}\Omega_{ca} \Gamma_{bd}   
    + M_{cd}\Omega_{db} \Gamma_{ac}   
    + M_{bd}\Omega_{ca} \Gamma_{cd}  
    + M_{ad}\Omega_{cb}  \Gamma_{cd} 
\end{align*}
Some terms of this expression might eventually cancel out with the previous one.
But before let's compute the second inertial term, 
\begin{align*}
    p_i^a p_j^b M_{kl}( \Gamma_{jl}\Gamma_{ik} 
    - \frac{1}{3}
    \Gamma_{ml}\Gamma_{mk}  
    {I}_{ij})
    = 
    M^{cd} \Gamma^{bd}\Gamma^{ac} 
    - \frac{1}{3}
    M^{cd}\Gamma^{ed}\Gamma^{ec}  
    \delta^{ab}
\end{align*}
Now we need a way to simplify these equations. 
This is done by noticing that $M^{ab}$ and $E^{ab}$ are actually diagonal tensors. 

The sum of these terms yields, 
\begin{align*}
    - \frac{1}{2}(
    M^{cb} \dot{\Omega}^{ca}
    + M^{ac} \dot{\Omega}^{cb}
    + M^{cd} \Omega^{ca}\Omega^{db}
    + M^{cd} \Omega^{ca}\Omega^{db}
    + M^{cb} \Omega^{cd}\Omega^{da} 
    + M^{ac} \Omega^{cd}\Omega^{db}
)\\
    - M^{cd}\Omega^{ca} \Gamma^{bd}   
    - M^{cd}\Omega^{db} \Gamma^{ac}   
    - M^{bd}\Omega^{ca} \Gamma^{cd}  
    - M^{ad}\Omega^{cb}  \Gamma^{cd} 
    - M^{cd}( \Gamma^{bd}\Gamma^{ac} 
    - \frac{1}{3}
    \Gamma^{ed}\Gamma^{ec}  
    \delta^{ab})\\
    = 
    -\frac{1}{2}(
    M^{cb} \dot{\Omega}^{ca}
    + M^{ac} \dot{\Omega}^{cb}
    + M^{ac} \Omega^{dc} \Omega^{db}  
    + M^{bd} \Omega^{cd}  \Omega^{ca}
    + 2 M^{ac} E^{dc} \Omega^{db}  
    + 2 M^{bd} E^{cd} \Omega^{ca}
    )\\
    - M^{cd} E^{ac} E^{bd} 
    + \frac{1}{3} M^{cd}
    \Gamma^{ed}\Gamma^{ec}  
    \delta^{ab}
\end{align*}

Applying these changes the stress let equation reads, 
\begin{align*}
    \frac{1}{2}\ddt^2 M^{ab}
    -\frac{1}{2}(
        M^{cb} \dot{\Omega}^{ca}
        + M^{ac} \dot{\Omega}^{cb}
        + M^{ac} \Omega^{dc} \Omega^{db}  
        + M^{bd} \Omega^{cd}  \Omega^{ca}
        + 2 M^{ac} E^{dc} \Omega^{db}  
        + 2 M^{bd} E^{cd} \Omega^{ca}
        )\\
        - M^{cd} E^{ac} E^{bd} 
        - \frac{1}{3} M^{cd}
        \Gamma^{ed}\Gamma^{ec}  
        \delta^{ab}
    + 2 \mu_2 v_\alpha E^{ab}
    - \frac{\gamma v_\alpha }{a} 
    \frac{4  }{5} (M^{ab} - \delta^{ab})
    = \textbf{F}^{ab}(\textbf{w}^s,\bm\sigma_1^0)
\end{align*}
The terms that remains when $a=b$ are, 
\begin{align*}
    \frac{1}{2}\ddt^2 M^{ab}
    - M^{ac} \Omega^{dc} \Omega^{db}  
    - M^{cd} E^{ac} E^{bd} 
    + \frac{1}{3} M^{cd}
    \Gamma^{ed}\Gamma^{ec}  
    \delta^{ab}
    + 2 \mu_2 v_\alpha E^{ab}
    - \frac{\gamma v_\alpha }{a} 
    \frac{4  }{5} (M^{ab} - \delta^{ab})
    = \textbf{F}^{ab}(\textbf{w}^s,\bm\sigma_1^0)
\end{align*}
with the mass equation, 
\begin{align*}
    \ddt M^{ab}
    = 2 E^{ab}M^{ab}
\end{align*}
The terms that does not remain when $a\neq b$ are, 
\begin{align*}
    -\frac{1}{2}(
        M^{cb} \dot{\Omega}^{ca}
        + M^{ac} \dot{\Omega}^{cb}
        + M^{ac} \Omega^{dc} \Omega^{db}  
        + M^{bd} \Omega^{cd}  \Omega^{ca}
        )
    = \textbf{F}^{ab}(\textbf{w}^s,\bm\sigma_1^0)
\end{align*}
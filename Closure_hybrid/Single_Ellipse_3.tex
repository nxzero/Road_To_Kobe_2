
% \section{Re-derivation but more easy}
% \label{ap:local_basis_eq}

% All these went way to complicated. 
% So here is a more consize derivation that does keep track of the physics. 
% We describe the geometry of the particle with the second moment of mass tensor, 
% \begin{equation*}
%     M_{\alpha,ij} 
%     =\frac{m_\alpha a^2}{5} M_{\alpha,ij}^*
%     = \frac{m_\alpha a^2}{5}\left[
%         \left(
%             \frac{a_1}{a}
%         \right)^2 p_i p_j
%         + 
%         \left(
%             \frac{a_1}{a}
%         \right)^2(\delta_{ij}-  p_i p_j)
%     \right]
% \end{equation*}
% This tensor has several notable properties, the volume conservation : $a_1 a_2^2 = a^3$ which means that $\textbf{M}_\alpha^*$ has two eigenvalues linked through $M^1 = (M^2)^{-2}$ or $M^2 = (M^1)^(-1/2)$. 
% At small deformation, i.e. $M_1 - 1\ll 1$ we therefore have $M^1 = 1 - (M^1-1)/2$.
% This ultimately means that at small deformation the trace is a constant namely $M_{ij} \delta_{ij} = 3+\mathcal{O}((M^1 -1)^2)$. 




% Let's now reformulate the integral of the problem namely, 
% \begin{align}
%     \intO{(\textbf{rw}_ 2^0 )_{ij}+ (\textbf{w}_2^0 \textbf{r})_{ij}} 
%     = \textbf{M}_{\alpha,ik} \cdot \bm\Gamma_{\alpha,jk}
%         +  \bm\Gamma_{\alpha,ik} \cdot \textbf{M}_{\alpha,jk}
%     \\
%     \intO{\rho_2 \textbf{w}_{2,i}^0\textbf{w}_{2,j}^0}
%     = \bm\Gamma_{\alpha,jl}\bm\Gamma_{\alpha,ik} \textbf{M}_{\alpha,kl}  
%         +\intO{\rho_2 \textbf{w}_{2,i}^s\textbf{w}_{2,j}^s}
%     \\
%     \intO{\bm\sigma_{2,ij}^0}
%     =
%     2 \mu_2 v_\alpha \textbf{E}_{\alpha,ij}
%     - \intO{p_2^0} \textbf{I}_{ij}
%     + \mu_2 \intS{(\textbf{n}_i \textbf{w}_{2,j}^s + \textbf{n}_j \textbf{w}_{2,i}^s)}
%     \\
%     \intS{\bm\sigma_{I,ij}^0}
%     = \frac{\gamma v_\alpha }{a} \left[
%         2\textbf{I}_{ij} 
%         - \frac{4  }{5} (\textbf{M}_{\alpha,ij}^* - \textbf{I}_{\alpha,ij})
%     \right]
%     +\mathcal(O)(\textbf{C}\cdot \textbf{C})\\
%     s_\alpha 
%     = 4\pi a^2 (1+\frac{\textbf{C}:\textbf{C}}{15})
% \end{align}
% for the surface tension term is made of two terms, one corresponding to Laplace pressure and the other to the deviatoric stress. 




% The equation for the orientation, second moment of mass torque, symmetric moment of momentum and trace of the moment of momentum reads, 
% \begin{align*}
%     \ddt \textbf{pp}_{\alpha,ij}
%     = \textbf{pp}_{\alpha,ik} \cdot \bm\Omega_{\alpha,jk}
%     +  \bm\Omega_{\alpha,ik} \cdot \textbf{pp}_{\alpha,jk}\\
%     \ddt \textbf{M}_{\alpha,ij}
%     = \textbf{M}_{\alpha,ik} \cdot \bm\Gamma_{\alpha,jk}
%     +  \bm\Gamma_{\alpha,ik} \cdot \textbf{M}_{\alpha,jk}\\
%     \ddt (\textbf{I}_{\alpha,ik}\bm\omega_{\alpha,k} )
%     = 
%     \intS{(\textbf{r}\times\bm\sigma_1^0\cdot \textbf{n})_i} \\
%     \frac{1}{2}\ddt^2 \textbf{M}_{\alpha,ij}
%     -  \bm\Gamma_{\alpha,jl}\bm\Gamma_{\alpha,ik} \textbf{M}_{\alpha,kl}  
%     + 2 \mu_2 v_\alpha \textbf{E}_{\alpha,ij}
%     + \frac{\gamma v_\alpha }{a} \left[
%     2\textbf{I}_{ij} 
%     - \frac{4 \gamma v_\alpha }{5 a} (\textbf{M}_{\alpha,ij}^* - \textbf{I}_{\alpha,ij})
%     \right]\\
%     = 
%     \frac{1}{2}\intS{(\textbf{r}\bm\sigma_1^0 + \bm\sigma_1^0\textbf{r})\cdot \textbf{n}} 
%     + \intO{\rho_2 \textbf{w}_{2,i}^s\textbf{w}_{2,j}^s}
%     + \intO{p_2^0} \textbf{I}_{ij}
%     - \mu_2 \intS{(\textbf{n}_i \textbf{w}_{2,j}^s + \textbf{n}_j \textbf{w}_{2,i}^s)}\\
%     % \frac{1}{2}\ddt^2 \textbf{M}_{\alpha,mm}
%     -  \bm\Gamma_{\alpha,ml}\bm\Gamma_{\alpha,mk} \textbf{M}_{\alpha,kl}  
%     + \frac{\gamma v_\alpha }{a} 
%     % \left[
%     2\textbf{I}_{mm} 
%     % - \frac{4 }{5 } (\textbf{M}_{\alpha,mm}^* - \textbf{I}_{\alpha,mm})
%     % \right]
%     % \\
%     = 
%     \intS{\textbf{r}_m\cdot\bm\sigma_{1,mk}^0\cdot \textbf{n}_k} 
%     + \intO{\rho_2 \textbf{w}_{2,m}^s\cdot \textbf{w}_{2,m}^s}
%     + \intO{p_2^0} \textbf{I}_{mm}
% \end{align*}

% As the derivative of $M_{ij}$ in these equations actually takes in account the change of orientation of the particle it might be useful to re-derive these equations in the eigenbasis of the matrix $M_{ij}$

% To the first order in deformation the deviatoric part reads, 
% \begin{align*}
%     \frac{1}{2}\ddt^2 \textbf{M}_{\alpha,ij}
%     -   \textbf{M}_{\alpha,kl} 
%     (\bm\Gamma_{\alpha,jl}\bm\Gamma_{\alpha,ik}  
%     - \frac{1}{3}
%     \bm\Gamma_{\alpha,ml}\bm\Gamma_{\alpha,mk}  
%     \textbf{I}_{ij}
%     )
%     + 2 \mu_2 v_\alpha \textbf{E}_{\alpha,ij}
%     - \frac{\gamma v_\alpha }{a} 
%     \frac{4  }{5} (\textbf{M}_{\alpha,ij}- \textbf{I}_{ij})
%     \\
%     = 
%     \frac{1}{2}\intS{(\textbf{r}\bm\sigma_1^0 + \bm\sigma_1^0\textbf{r} - \frac{2}{3}\textbf{r}\cdot \bm\sigma_1^0 \textbf{I})\cdot \textbf{n}} 
%     + \intO{\rho_2 (\textbf{w}_{2,i}^s\textbf{w}_{2,j}^s - \frac{1}{3}\textbf{w}_{2,m}^s\textbf{w}_{2,m}^s \textbf{I}_{ij}) }
%     - \mu_2 \intS{(\textbf{n}_i \textbf{w}_{2,j}^s + \textbf{n}_j \textbf{w}_{2,i}^s)}\\
% \end{align*}
% It can be somewhat useful to extract the proportionality coefficient of each terms. 
% Denotining dimensionless qte by a $*$ we have, 
% \begin{align*}
%     \frac{\rho_2 a^2}{5 \tau^2}\left[\frac{1}{2}\ddt^2 \textbf{M}_{\alpha,ij}^*
%     -  \bm\Gamma_{\alpha,jl}^*\bm\Gamma_{\alpha,ik}^* \textbf{M}_{\alpha,kl}^*
%     \right]  
%     + \frac{2 \mu_2  }{\tau} \textbf{E}_{\alpha,ij}^*
%     + \frac{\gamma  }{a} \left[
%     2\textbf{I}_{ij} 
%     - \frac{4 }{5} (\textbf{M}_{\alpha,ij}^* - \textbf{I}_{\alpha,ij})
%     \right]\\
%     = 
%     \frac{1}{2}\intS{(\textbf{r}\bm\sigma_1^0 + \bm\sigma_1^0\textbf{r})\cdot \textbf{n}} 
%     + \intO{\rho_2 \textbf{w}_{2,i}^s\textbf{w}_{2,j}^s}
%     + \intO{p_2^0} \textbf{I}_{ij}
%     - \mu_2 \intS{(\textbf{n}_i \textbf{w}_{2,j}^s + \textbf{n}_j \textbf{w}_{2,i}^s)}\\
% \end{align*}

% The forcing terms are really problem dependent. 
% Thus, at this stage it cannot be scaled in terms of timescale. 

% In order to be more consis, we may write :
% \begin{align*}
%     \textbf{F}_{ij}(\textbf{w}^s,\bm\sigma_1^0)
%     = 
%     \frac{1}{2}\intS{(\textbf{r}\bm\sigma_1^0 + \bm\sigma_1^0\textbf{r} - \frac{2}{3}\textbf{r}\cdot \bm\sigma_1^0 \textbf{I})\cdot \textbf{n}} 
%     + \intO{\rho_2 (\textbf{w}_{2,i}^s\textbf{w}_{2,j}^s - \frac{1}{3}\textbf{w}_{2,m}^s\textbf{w}_{2,m}^s \textbf{I}_{ij}) }
%     - \mu_2 \intS{(\textbf{n}_i \textbf{w}_{2,j}^s + \textbf{n}_j \textbf{w}_{2,i}^s)}\\
% \end{align*} 
% Indicating that the forcing term is related to the exterior contribution and the external stresses. 

% \begin{align*}
%     \ddt^2 \textbf{M}_{\alpha,ij}^*
%     -2  \textbf{M}_{\alpha,kl}^* 
%     (\bm\Gamma_{\alpha,jl}^*\bm\Gamma_{\alpha,ik}^*  
%     - \frac{1}{3}
%     \bm\Gamma_{\alpha,ml}^*\bm\Gamma_{\alpha,mk}^*  
%     \textbf{I}_{ij}
%     )
%     + \frac{10 \mu_2 \tau}{ \rho_2 a^2} 2\textbf{E}_{\alpha,ij}^*
%     - 8 \frac{\tau^2 \gamma  }{\rho_2 a^3} 
%      (\textbf{M}_{\alpha,ij}^*- \textbf{I}_{ij})\\
%     = 
%     \frac{\tau^2 10}{a^2 m_\alpha}
%     \left[\textbf{F}_{\sigma_1, ij}
%     + \textbf{F}_{ww, ij}
%     + \textbf{F}_{\sigma_2, ij}\right]
% \end{align*}

% All these unclose term are really problem dependent, for now all we now is that at low Reynolds number we are in the viscous regime, besides this external flow that drives these scales. 
% Meaning that, 
% \begin{align*}
%     \intO{\rho_2 \textbf{w}_{2,i}^0\textbf{w}_{2,j}^0}
%     \sim \frac{m_\alpha a^2}{\tau_u^2} \textbf{F}_{ww}^*
%     \\
%     \mu_2 \intS{(\textbf{n}_i \textbf{w}_{2,j}^s + \textbf{n}_j \textbf{w}_{2,i}^s)}
%     \sim \frac{v_\alpha \mu_2}{\tau_u} \textbf{F}_{e}^*
%     \\
%     \intS{\bm\sigma_1^0 \textbf{rn}}
%     \sim 
%     \frac{v_\alpha \mu_1}{\tau_u} \textbf{F}_{\sigma}^*
% \end{align*} 

% \begin{align*}
%     \ddt^2 \textbf{M}_{\alpha,ij}^*
%     -2  \textbf{M}_{\alpha,kl}^* 
%     (\bm\Gamma_{\alpha,jl}^*\bm\Gamma_{\alpha,ik}^*  
%     - \frac{1}{3}
%     \bm\Gamma_{\alpha,ml}^*\bm\Gamma_{\alpha,mk}^*  
%     \textbf{I}_{ij}
%     )
%     + \frac{10 \mu_2 \tau}{ \rho_2 a^2} 2\textbf{E}_{\alpha,ij}^*
%     - 8 \frac{\tau^2 \gamma  }{\rho_2 a^3} 
%      (\textbf{M}_{\alpha,ij}^*- \textbf{I}_{ij})\\
%     = 
%     \frac{\tau^2 10 \mu_1 }{a^2 \rho_2 \tau_u}
%     \textbf{F}_{\sigma_1, ij}
%     + \frac{\tau^2 10}{\tau_u^2} \textbf{F}_{ww, ij}
%     + \frac{\tau^2 10 \mu_1 }{a^2 \rho_2 \tau_u} \textbf{F}_{\sigma_2, ij}
% \end{align*}


% \subsubsection*{Re-derivation but with dimensionless scaling}

% \begin{align}
%     \intO{(\textbf{rw}_ 2^0 )_{ij}+ (\textbf{w}_2^0 \textbf{r})_{ij}} 
%     = \textbf{M}_{\alpha,ik} \cdot \bm\Gamma_{\alpha,jk}
%         +  \bm\Gamma_{\alpha,ik} \cdot \textbf{M}_{\alpha,jk}
%     \\
%     \intO{\rho_2 \textbf{w}_{2,i}^0\textbf{w}_{2,j}^0}
%     = \bm\Gamma_{\alpha,jl}\bm\Gamma_{\alpha,ik} \textbf{M}_{\alpha,kl}  
%         +\intO{\rho_2 \textbf{w}_{2,i}^s\textbf{w}_{2,j}^s}
%     \\
%     \intO{\bm\sigma_{2,ij}^0}
%     =
%     2 \mu_2 v_\alpha \textbf{E}_{\alpha,ij}
%     - \intO{p_2^0} \textbf{I}_{ij}
%     + \mu_2 \intS{(\textbf{n}_i \textbf{w}_{2,j}^s + \textbf{n}_j \textbf{w}_{2,i}^s)}
%     \\
%     \intS{\bm\sigma_{I,ij}^0}
%     = \frac{\gamma v_\alpha }{a} \left[
%         2\textbf{I}_{ij} 
%         - \frac{4  }{5} (\textbf{M}_{\alpha,ij}^* - \textbf{I}_{\alpha,ij})
%     \right]
%     +\mathcal(O)(\textbf{C}\cdot \textbf{C})\\
%     s_\alpha 
%     = 4\pi a^2 (1+\frac{\textbf{C}:\textbf{C}}{15})
% \end{align}

% Then, 
% \begin{align*}
%     \frac{\rho_2 a^2}{5}\left[
%         \frac{1}{\tau^2}\frac{1}{2}\ddt^2 \textbf{M}_{\alpha,ij}
%     -   \frac{1}{\tau^2}\bm\Gamma_{\alpha,jl}\bm\Gamma_{\alpha,ik} \textbf{M}_{\alpha,kl}  
%     - \frac{1}{\tau_u^2} \textbf{F}_{ww}^*
%     \right]\\
%     + \mu_2  \left[
%         \frac{1}{\tau} 2\textbf{E}_{\alpha,ij}
%     + \frac{1}{\tau_u} \textbf{F}_\sigma^*
%     \right]\\
%     + \frac{\gamma  }{a} \left[
%     2\textbf{I}_{ij} 
%     - \frac{4  }{5} (\textbf{M}_{\alpha,ij}^* - \textbf{I}_{\alpha,ij})
%     \right]\\
%     = 
%     \frac{1}{2}
%     \frac{ \mu_1}{\tau_u} 
%     \textbf{F}_{\sigma_1}^*
%     % \intS{(\textbf{r}\bm\sigma_1^0 + \bm\sigma_1^0\textbf{r})\cdot \textbf{n}} 
%     % + \intO{p_2^0} \textbf{I}_{ij}
% \end{align*}

% which then, 
% \begin{align*}
%     \frac{\rho_2}{\rho_1}\frac{\rho_1 a^2}{5\mu_1\tau_u}\left[
%         \frac{\tau_u^2}{\tau^2}\frac{1}{2}\ddt^2 \textbf{M}_{\alpha,ij}
%     -   \frac{\tau_u^2}{\tau^2}\bm\Gamma_{\alpha,jl}\bm\Gamma_{\alpha,ik} \textbf{M}_{\alpha,kl}  
%     - \textbf{F}_{ww}^*
%     \right]\\
%     + \frac{\mu_2}{\mu_1}  \left[
%         \frac{\tau_u}{\tau} 2\textbf{E}_{\alpha,ij}
%     +  \textbf{F}_\sigma^*
%     \right]\\
%     + \frac{\gamma \tau_u }{a\mu_1} \left[
%     2\textbf{I}_{ij} 
%     - \frac{4  }{5} (\textbf{M}_{\alpha,ij}^* - \textbf{I}_{\alpha,ij})
%     \right]\\
%     = 
%     \frac{1}{2}
%     \textbf{F}_{\sigma_1}^*
%     % \intS{(\textbf{r}\bm\sigma_1^0 + \bm\sigma_1^0\textbf{r})\cdot \textbf{n}} 
%     % + \intO{p_2^0} \textbf{I}_{ij}
% \end{align*}

% In terms of dimensionless groups
% \begin{align*}
%     \zeta Re\left[
%         \beta^2 \frac{1}{2}\ddt^2 \textbf{M}_{\alpha,ij}
%     -   \beta^2 \bm\Gamma_{\alpha,jl}\bm\Gamma_{\alpha,ik} \textbf{M}_{\alpha,kl}  
%     - \textbf{F}_{ww}^*
%     \right]\\
%     + \lambda  \left[
%         \beta 2\textbf{E}_{\alpha,ij}
%     +  \textbf{F}_\sigma^*
%     \right]\\
%     + \frac{1}{Ca} \left[
%     2\textbf{I}_{ij} 
%     - \frac{4  }{5} (\textbf{M}_{\alpha,ij}^* - \textbf{I}_{\alpha,ij})
%     \right]\\
%     = 
%     \frac{1}{2}
%     \textbf{F}_{\sigma_1}^*
%     % \intS{(\textbf{r}\bm\sigma_1^0 + \bm\sigma_1^0\textbf{r})\cdot \textbf{n}} 
%     % + \intO{p_2^0} \textbf{I}_{ij}
% \end{align*}


% \subsubsection*{Slow flows wheer beta equal 0 }

% \textbf{Steady state equillibrium }
% \begin{align*}
%     - \zeta Re \textbf{F}_{ww}^*
%     + \lambda  \textbf{F}_\sigma^*
%     + \frac{1}{Ca} \left[
%     2\textbf{I}_{ij} 
%     - \frac{4  }{5} (\textbf{M}_{\alpha,ij}^* - \textbf{I}_{\alpha,ij})
%     \right]
%     = 
%     \frac{1}{2}
%     \textbf{F}_{\sigma_1}^*
%     % \intS{(\textbf{r}\bm\sigma_1^0 + \bm\sigma_1^0\textbf{r})\cdot \textbf{n}} 
%     % + \intO{p_2^0} \textbf{I}_{ij}
% \end{align*}

% In the last part i want to be able to neglect all the components on the left

% if the reynolds number is indeed negligible we have 

% \begin{align*}
%     + \lambda  \textbf{F}_\sigma^*
%     + \frac{1}{Ca} \left[
%     2\textbf{I}_{ij} 
%     - \frac{4  }{5} (\textbf{M}_{\alpha,ij}^* - \textbf{I}_{\alpha,ij})
%     \right]
%     = 
%     \frac{1}{2}
%     \textbf{F}_{\sigma_1}^*
%     % \intS{(\textbf{r}\bm\sigma_1^0 + \bm\sigma_1^0\textbf{r})\cdot \textbf{n}} 
%     % + \intO{p_2^0} \textbf{I}_{ij}
% \end{align*}
% which give the equillibrium of surface tension internal / external stresses
% This must be respected for almost all steady state equilibrium presented in appendix
% It is clear that if either $\lambda = 0$ or that we are looking for the fluid phase stress then the int on the left vanish. 
% And we are left with an equillibrium between surface tension and stresslet

% \tb{In this regime one might recognize Laplace law}

% \textbf{Drop in air }
% \begin{align*}
%     \zeta Re\left[
%          \frac{1}{2}\ddt^2 \textbf{M}_{\alpha,ij}
%     -    \bm\Gamma_{\alpha,jl}\bm\Gamma_{\alpha,ik} \textbf{M}_{\alpha,kl}  
%     \right]
%     +   
%         \frac{\lambda}{\beta} 2\textbf{E}_{\alpha,ij}
%     + \frac{1}{Ca\beta^2} \left[
%     2\textbf{I}_{ij} 
%     - \frac{4  }{5} (\textbf{M}_{\alpha,ij}^* - \textbf{I}_{\alpha,ij})
%     \right]\\
%     = 
%     0 
% \end{align*}
% Indeed, in this case we still consier that the surface tension is comparable with the ratio of time scale of course. 
% Besides, 
% \tb{In this regime one might recognize Lamb's equation }



\section{The local basis equations}
\label{ap:local_basis_eq}

In this appendix we give details about the derivation of the angular momentum and rate of strain equations in the local basis of the particle. 

Firstly, we consider that any second order tensor $\textbf{A}$ can be expressed in the basis formed by the unit vectors $\{\textbf{p}^0,\textbf{p}^1,\textbf{p}^2\}$, which form an orthonormal basis. 
The first vector $\textbf{p}^0$ corresponds to the vector of the particle orientation. 
The vector $\textbf{p}^1$ and $\textbf{p}^2$ are defined such that $\textbf{p}^1 \times \textbf{p} =\textbf{p}^1 \times \textbf{p}^2=\textbf{p}^2 \times \textbf{p} =0$. 
Since, the vectors $\{\textbf{p}^0,\textbf{p}^1,\textbf{p}^2\}$ form a linearly independent family, any arbitrary second order tensor  \textbf{A} can be written, 
\begin{equation}
    A_{ij}
    = 
    A^{ab}
    p_i^a
    p_j^b.
\end{equation}
The indices in superscript are used to denote the local basis tensors  components and operations, such that $A^{ab}$ corresponds to the components of \textbf{A} in the basis $\{\textbf{p}^0,\textbf{p}^1,\textbf{p}^2\}$. 
Additionally, we also used the Einstein summation convention for the indices in superscript. 
Note that the components $A^{ab}$ can be obtained from \textbf{A} applying the double contracted product,  
\begin{equation*}
    A_{ij} 
    p_i^a
    p_j^b
    = 
    A^{cd}
    p_i^c
    p_j^d
    p_i^a
    p_j^b
    = 
    A^{ab}
\end{equation*}
From there it is easy to understand that to obtain the evolution equation in the eigenbasis one has to multiply the previous set of equation with $p_i^ap_j^b$. 

Let us recall some tensor properties. 
Note that if $\bm{\Omega}$ is defined as a skew-symmetric tensor, such that, 
\begin{equation}
    \Omega_{ij} = \frac{1}{2} [A_{ij}-A_{ji}],
\end{equation}
then, the tensor $\bm\Omega$ written in the local basis will also be skew-symmetric, thus we have
\begin{equation}
    \Omega^{ab} = \frac{1}{2} [A^{ba}-A^{ab}]. 
\end{equation}
Indeed, using the expression of $A_{ij}$ in the local basis one may show the relation, 
\begin{equation}
    \Omega_{ij} = \frac{1}{2} [A^{ab} p^a_i p^b_j-A^{ba} p^b_j p^a_i]
    =  \frac{1}{2} [A^{ab} - A^{ba} ]p^b_j p^a_i
    =  \Omega^{ab} p^b_j p^a_i. 
\end{equation}
Therefore, by identification we deduce the former equation. 
Since the eigenvectors $\textbf{p}^0$, $\textbf{p}^1$ and  $\textbf{p}^2$ rotate according to the same angular velocity tensor $\bm\Omega$, we can write $\ddt p_i^a =\Omega_{ik} p_k^a$ for $a =0,1,2$. 
We deduce the evolution equation for the dyadic $p_i^ap_j^b$, namely,
\begin{equation*}
    \ddt(p_i^ap_j^b)
    = 
    \Omega_{ik} p_k^ap_j^b
    + \Omega_{jk} p_i^ap_k^b.
    \label{eq:orientation_pp}
\end{equation*}
This equation is the key tool that we use to reformulate \ref{eq:dt_M2}, \ref{eq:dt_S2} and \ref{eq:dt_mu2} in the particle local basis. 

\paragraph*{Local equation of the deformation:}
As described in the main text we multiply \ref{eq:dt_M2} by $p_i^ap_j^b$ and directly obtain the equaiton, 
\begin{equation*}
    \ddt M^{ab}
    = 
    M^{ac} E^{bc} 
    + E^{ac} M^{bc}. 
\end{equation*}
Note that the tensor $M^{ab}$ and $E^{ab}$ are diagonal in this basis meaning that we have, 
\begin{equation*}
    \ddt M^I
    = 
    2 M^I E^I
\end{equation*}
where $M^I$ represents the $I^{th}$ eigenvalue of the tensor $M_{ij}$. 

\paragraph*{The local equation of the rate of strain:}
We multiply, \ref{eq:dev} by $p_i^ap_j^b$, then the first term on the left-hand side of \ref{eq:dev} then reads, 
\begin{align}
    p_i^ap_j^b\ddt^2 M_{ij}
    = 
    % \ddt( p_i^ap_j^b \ddt M_{ij})
    % - \ddt (p_i^ap_j^b )\ddt M_{ij}\\
    % = 
    % \ddt( \ddt (p_i^ap_j^b M_{ij}))
    % - \ddt(   M_{ij} \ddt  p_i^ap_j^b )
    % - \ddt (p_i^ap_j^b )\ddt M_{ij}\\
    % = 
    \ddt^2 M^{ab}
    - 2 \ddt (p_i^ap_j^b) \ddt M_{ij}
    - M_{ij} \ddt^2 (p_i^ap_j^b)
    \label{eq:pp_dt2_M}
\end{align}
To get the second order derivative of $p_i^ap_j^b$ we can directly carry out the derivation, it gives, 
\begin{align*}
    \ddt\ddt(p_i^ap_j^b)
    % = 
    % \ddt (\Omega_{ik} p_k^ap_j^b)
    % + \ddt (\Omega_{jk} p_i^ap_k^b)\\
    % = 
    % \dot{\Omega}_{ik} p_k^ap_j^b
    % + \dot{\Omega}_{jk} p_i^ap_k^b
    % + (
    %     \Omega_{kl} p_l^ap_j^b
    % + \Omega_{jl} p_k^ap_l^b
    % )\Omega_{ik}
    % + (\Omega_{il} p_l^ap_k^b
    % + \Omega_{kl} p_i^ap_l^b)\Omega_{jk}\\
    = 
    \dot{\Omega}_{ik} p_k^ap_j^b
    + \dot{\Omega}_{jk} p_i^ap_k^b
    + \Omega_{ik}\Omega_{kl} p_l^ap_j^b
    + \Omega_{ik}\Omega_{jl} p_k^ap_l^b
    + \Omega_{jk}\Omega_{il} p_l^ap_k^b
    + \Omega_{jk}\Omega_{kl} p_i^ap_l^b,
\end{align*}
where we introduced the angular acceleration tensor $\dot{\Omega}_{ik} = \ddt \Omega_{ik}$. 
The third term on the right-hand side of \ref{eq:pp_dt2_M} reads, 
\begin{align*}
    M_{ij} \ddt^2 (p_i^ap_j^b)
    = M^{cb} \dot{\Omega}^{ca}
    + M^{ac} \dot{\Omega}^{cb}
    + M^{cd} \Omega^{ca}\Omega^{db}
    + M^{cd} \Omega^{ca}\Omega^{db}
    + M^{cb} \Omega^{cd}\Omega^{da} 
    + M^{ac} \Omega^{cd}\Omega^{db} 
\end{align*}
The second third term on the right-hand side of \ref{eq:pp_dt2_M} reads, 
\begin{align*}
    \ddt (p_i^ap_j^b) \ddt M_{ij}
    % = 
    % (\Omega_{ik} p_k^ap_j^b
    % + \Omega_{jk} p_i^ap_k^b)
    % (M_{il} \Gamma_{jl}
    % +  \Gamma_{il} M_{jl})\\
    = 
    + M_{cd}\Omega_{ca} \Gamma_{bd}   
    + M_{cd}\Omega_{db} \Gamma_{ac}   
    + M_{bd}\Omega_{ca} \Gamma_{cd}  
    + M_{ad}\Omega_{cb}  \Gamma_{cd} 
\end{align*}
The second term of \ref{eq:dev} multiplied by $p_i^ap_j^b$, reads, 
\begin{align*}
    p_i^a p_j^b M_{kl}( \Gamma_{jl}\Gamma_{ik} 
    - \frac{1}{3}
    \Gamma_{ml}\Gamma_{mk}  
    \delta_{ij})
    = 
    M^{cd} \Gamma^{bd}\Gamma^{ac} 
    - \frac{1}{3}
    M^{cd}\Gamma^{ed}\Gamma^{ec}  
    \delta^{ab}
\end{align*}
The of the first and second term of \ref{eq:dev} may be written in the local basis as, 
\begin{align*}
    - \frac{1}{2}(
    M^{cb} \dot{\Omega}^{ca}
    + M^{ac} \dot{\Omega}^{cb}
    + M^{cd} \Omega^{ca}\Omega^{db}
    + M^{cd} \Omega^{ca}\Omega^{db}
    + M^{cb} \Omega^{cd}\Omega^{da} 
    + M^{ac} \Omega^{cd}\Omega^{db}
)\\
    - M^{cd}\Omega^{ca} \Gamma^{bd}   
    - M^{cd}\Omega^{db} \Gamma^{ac}   
    - M^{bd}\Omega^{ca} \Gamma^{cd}  
    - M^{ad}\Omega^{cb}  \Gamma^{cd} 
    - M^{cd}( \Gamma^{bd}\Gamma^{ac} 
    - \frac{1}{3}
    \Gamma^{ed}\Gamma^{ec}  
    \delta^{ab})\\
    = 
    -\frac{1}{2}(
    M^{cb} \dot{\Omega}^{ca}
    + M^{ac} \dot{\Omega}^{cb}
    + M^{ac} \Omega^{dc} \Omega^{db}  
    + M^{bd} \Omega^{cd}  \Omega^{ca}
    + 2 M^{ac} E^{dc} \Omega^{db}  
    + 2 M^{bd} E^{cd} \Omega^{ca}
    )\\
    - M^{cd} E^{ac} E^{bd} 
    + \frac{1}{3} M^{cd}
    \Gamma^{ed}\Gamma^{ec}  
    \delta^{ab}
\end{align*}
Considering this results, we finally re-write \ref{eq:dev} in the particle eigenbasis of the particle it yields,
\begin{align*}
    \frac{1}{2}\ddt^2 M^{ab}
    -\frac{1}{2}(
        M^{cb} \dot{\Omega}^{ca}
        + M^{ac} \dot{\Omega}^{cb}
        + M^{ac} \Omega^{dc} \Omega^{db}  
        + M^{bd} \Omega^{cd}  \Omega^{ca}
        + 2 M^{ac} E^{dc} \Omega^{db}  
        + 2 M^{bd} E^{cd} \Omega^{ca}
        )\\
        - M^{cd} E^{ac} E^{bd} 
        - \frac{1}{3} M^{cd}
        \Gamma^{ed}\Gamma^{ec}  
        \delta^{ab}
    + 2 \mu_2 v_\alpha E^{ab}
    - \frac{\gamma v_\alpha }{a} 
    \frac{4  }{5} \chi^{ab}
    = F^{ab}. 
\end{align*}
This equation can be further simplified noticing that $M^{ab} = E^{ab} = 0$ for $a\neq b$, in which case we write,
\begin{align*}
    \frac{1}{2}\ddt^2 M^{ab}
    - M^{ac} \Omega^{dc} \Omega^{db}  
    - M^{cd} E^{ac} E^{bd} 
    + \frac{1}{3} M^{cd}
    \Gamma^{ed}\Gamma^{ec}  
    \delta^{ab}
    + 2 \mu_2 v_\alpha E^{ab}
    - \frac{\gamma v_\alpha }{a} 
    \frac{4  }{5} \chi^{ab}
    = F^{ab}
\end{align*}

Interestingly, when $a\neq b$ we obtain this relation, 
\begin{align*}
    -\frac{1}{2}(
        M^{cb} \dot{\Omega}^{ca}
        + M^{ac} \dot{\Omega}^{cb}
        + M^{ac} \Omega^{dc} \Omega^{db}  
        + M^{bd} \Omega^{cd}  \Omega^{ca}
        )
    = \textbf{F}^{ab}
\end{align*}
which might be considered as a constraint on the particle rotation. 

\paragraph{Local basis angular momentum equation:}
We now demonstrate how to derive the angular momentum equation in the particle eigenbasis. 
The angular momentum initially reads, 
\begin{equation*}
    \ddt (\textbf{I}_{\alpha,ik}\bm\omega_{\alpha,k} )
    = 
    \intS{(\textbf{r}\times\bm\sigma_f^0\cdot \textbf{n})_i} . 
    \label{eq:dt_mu_demo}
\end{equation*}
Firstly, the terms on the left-hand side can be written in the local basis as, 
\begin{equation*}
    I_{ik}\omega_{k}
    = 
    I^{ab} p_i^a p_k^b \omega^c p^c_k
    = 
    I^{ab}   \omega^b p_i^a. 
\end{equation*} 
Therefore, we multiply \ref{eq:dt_mu_demo} by $\textbf{p}_i^a$ which eventually leads us to the expression, 
\begin{equation*}
    \ddt (I^{ab}   \omega^b)
    = 
    + I^{ab}\omega^b p_i^a  \ddt p_i^a
    + \intS{(\textbf{r}\times\bm\sigma_f^0\cdot \textbf{n})_i} p_i^a. 
\end{equation*}
Which is the local angular momentum equation in the local reference frame of the particle, with $\omega^b$ the particle's rate of rotation in the eigenbasis. 
Nevertheless, we may simplify this equation noticing that $\ddt p_i^a = \epsilon_{ijk} \omega_j p_k^a = \epsilon_{ijk} \omega^c p_j^c p_k^a$, which gives us, 
\begin{equation*}
    \ddt (I^{ab}   \omega^b)
    = 
    I^{ab}\omega^b  \omega^c \epsilon_{ijk} p_i^a p_j^c p_k^a
    + \intS{(\textbf{r}\times\bm\sigma_f^0\cdot \textbf{n})_i} p_i^a
    = 
    % I^{ab}\omega^b  \omega^c \epsilon_{jki} p_i^a p_j^c p_k^a
    p_i^a\intS{(\textbf{r}\times\bm\sigma_f^0\cdot \textbf{n})_i}, 
\end{equation*}
where we could pass from the first to the second equality by noticing that $\epsilon_{jki} p_i^a p_j^c p_k^a = (\textbf{p}\times \textbf{p}) \cdot \textbf{p} = 0$ since the vector product of any unit tensor with itself is identically null. 
Since $I^{ab} = 0$ for $a \neq b$, we deduce that this vector equation can be reduced to these three equations for the principal values of the rotation vector, namely,
\begin{align*}
    \ddt (I^0   \omega^0)
    = 
    % I^{ab}\omega^b  \omega^c \epsilon_{jki} p_i^a p_j^c p_k^a
    p_i^0\intS{(\textbf{r}\times\bm\sigma_f^0\cdot \textbf{n})_i} \\
    \ddt (I^1   \omega^1)
    = 
    % I^{ab}\omega^b  \omega^c \epsilon_{jki} p_i^a p_j^c p_k^a
    p_i^1\intS{(\textbf{r}\times\bm\sigma_f^0\cdot \textbf{n})_i} \\
    \ddt (I^2   \omega^2)
    = 
    % I^{ab}\omega^b  \omega^c \epsilon_{jki} p_i^a p_j^c p_k^a
    p_i^2\intS{(\textbf{r}\times\bm\sigma_f^0\cdot \textbf{n})_i} 
\end{align*}
Interestingly, for solid particles, the $I^i$ are constant scalar, meaning that it is possible to simply remove these of the derivative operators. 
In our case we may reformulate the inertia tensor as, 
\begin{equation*}
    I^{ab}_\alpha
    = 
    M^{jj}_\alpha \delta^{ab}
    - M^{ab}_\alpha
    = 
    \frac{5}{m_\alpha a^2}[(\chi^{jj} + 2) \delta^{ab}
    - \chi^{ab}]
    = 
    \frac{5}{m_\alpha a^2}[2\delta^{ab} - \chi^{ab}]
    + \mathcal{O}(\chi_I^2)
\end{equation*}
where we have considered only small deformation. 
In conclusion the $I^{th}$ component of the rotation vector can be obtained using the angular momentum equation expressed in the particle eigenbasis, namely, 
\begin{align*}
    \ddt\omega^I 
    = 
    % I^{ab}\omega^b  \omega^c \epsilon_{jki} p_i^a p_j^c p_k^a
    \frac{5}{m_\alpha a^2 (2 - \chi_I)}
    \left(
    \omega^I E^I 
    +
    p_i^0\intS{(\textbf{r}\times\bm\sigma_f^0\cdot \textbf{n})_i} 
    \right). 
\end{align*}
Note that for solid particle the first term on the right-hand side vanish, as expected. 
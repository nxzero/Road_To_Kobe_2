
% \subsection{Phase average and particle averaged equations}
% Correspondingly, we take the average of the particle fields equations by using \ref{eq:avg} on \ref{eq:dt_dq_alpha_tot} and \ref{eq:dt_dQ_alpha_tot}, which gives, 
In the objective of obtaining coarse-grained level equations for the dispersed phase, one must apply an ensemble average operator on \ref{eq:avg_dt_dq_alpha_tot} and \ref{eq:avg_dt_dQ_alpha_tot} which is made possible since the particle fields $\delta_\alpha \ldots$ are now defined over the whole space $\Omega$ thanks to the Dirac delta functions $\delta_\alpha$.  
These equations yields,
\begin{align}
    \pddt \avg{\delta_\alpha  q_\alpha^\text{tot}}
    + \div \avg{\delta_\alpha\textbf{u}_\alpha q_\alpha^\text{tot}}
    &= \pOavg{ s_2^0 }
    + \pSavg{ s_I^0 }
    + \pSavg{ \left[\mathbf{\Phi}_1^0 + f_1^0 (\textbf{u}_I^0-\textbf{u}_1^0) \right] \cdot \textbf{n}_2 ,}
    \label{eq:avg_dt_dq_alpha_tot}\\
    \pddt \avg{\delta_\alpha \mathcal{Q}_\alpha^\text{tot}}
    + \div \avg{\delta_\alpha\textbf{u}_\alpha\mathcal{Q}_\alpha^\text{tot}}
    &=\pOavg{ \left(
        \textbf{r} s_2^0         
        + f_2^0  \textbf{w}_2^0 
        - \mathbf{\Phi}_2^0
    \right) }
    + \pSavg{ \left(
        \textbf{r}s_I^0
        + f_I^0 \textbf{w}_I^0
        - \mathbf{\Phi}_{I||}^0
    \right) }\nonumber\\
    &+ \pSavg{ \textbf{r} \left[
        \mathbf{\Phi}_1^0
        + f_1^0 (\textbf{u}_I^0-\textbf{u}_1^0)
    \right]\cdot \textbf{n}_2  }.
    \label{eq:avg_dt_dQ_alpha_tot}
\end{align}
In \ref{ap:Moments_equations} the derivation of the higher moment particle-averaged equations is provided. 
In this study,\ref{eq:avg_dt_chi_f} and \ref{eq:avg_dt_delta_f} are refereed to as the phase-averaged equations, while \ref{eq:avg_dt_dq_alpha_tot} and \ref{eq:avg_dt_dQ_alpha_tot} are denoted as the particle-averaged equation. 
In these expressions we kept a general notation yet. 
But note that, we can note the particle phase averaged quantity by,
\begin{equation}
     n_p q_p(\textbf{x},t) = \avg{\delta_\alpha q_\alpha}
     \label{eq:p_avg}
\end{equation}
where, $n_p(\textbf{x},t) = \avg{\delta_\alpha}$ is the probable number of finding a particle center of mass at $\textbf{x}$
and $q_p$ is the conditional average of $q_\alpha$ conditionally on the presence of a particle at \textbf{x}. 
Additionally, note that it is possible to define the fluctuating parts of a property by, 
\begin{equation}
    q_\alpha' = q_\alpha - q_p
    % \;\;\;\;\;\;\text{and}
    % \;\;\;\;\;\;
\end{equation}
such that the particle average of a product can be rewritten, $\pavg{q_\alpha\textbf{u}_\alpha} = n_p q_p \textbf{u}_p + \pavg{q_\alpha' \textbf{u}_\alpha'}$. 
These notations will find their use in \ref{sec:averaged_eq}, but for now we keep the formulation rather generic as in \ref{eq:avg_dt_dQ_alpha_tot} and \ref{eq:avg_dt_dq_alpha_tot}

As remarked by \citet{jackson1997locally} for the angular momentum equations of solid spherical particles, and here in a more general case :\ref{eq:avg_dt_chi_f} and \ref{eq:avg_dt_dq_alpha_tot} may seem to be not coupled with the higher order moments equations, i.e. \ref{eq:avg_dt_dQ_alpha_tot}. 
Indeed, the first order moment $\mathcal{Q}_p^\text{tot}$ do not appear explicitly in either \ref{eq:avg_dt_chi_f} or \ref{eq:avg_dt_dq_alpha_tot}.
However, the exchange and source terms 
appearing in both \ref{eq:avg_dt_chi_f} and \ref{eq:avg_dt_dQ_alpha_tot} might depend on the higher order moments of the particles.
Indeed, in the momentum equation of the continuous phase, i.e. the ensemble average of \ref{eq:dt_rhou_k}, the exchange term corresponds to the averaged drag force, namely $\pSavg{\bm{\sigma}_1^0\cdot \textbf{n}_2}$. 
It is clear that $\pSavg{\bm{\sigma}_1^0\cdot \textbf{n}_2}$ has a strong dependency with $\textbf{u}_p$,$\mathcal{P}_p$ and $\mathcal{M}_p$ since the drag force is a function of both the particle's kinematics   and its shape. 
Therefore, \ref{eq:avg_dt_dQ_alpha_tot} is linked to \ref{eq:avg_dt_chi_f} and \ref{eq:avg_dt_dq_alpha_tot} solely through the dependence of the exchange and source terms with the properties of the particles, e.g. $q_p,\mathcal{Q}_p$ and possibly the higher order moments. 
This reasoning applies for the energy equation and all kinds of conservation equation. 
Ultimately, the significance of the higher moments equations can be evaluated based on the dependency of the closure terms present in \ref{eq:avg_dt_chi_f} with the properties of the particle : $q_p, \mathcal{Q}_p$ and potentially the higher-order moments of the particles. 


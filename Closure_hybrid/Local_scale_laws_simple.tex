\section{Local scale equations}

In this section, we present a Lagrangian-based model capable of describing the orientation and deformation of a single particle. 
\label{sec:local_eq_ellipse}

\subsection{Local scale equations}

We consider a system consisting of two phases, separated by a sharp interface $\Gamma(t)$ which evolves over time. 
Each phase subdomain is denoted $\Omega_f(t)$ and $\Omega_d(t)$ for the continuous phase ($f$) and the dispersed phase ($d$) respectively. 
The entire domain, denoted as $\Omega$, is defined as the union of $\Omega_f$, $\Omega_2$, and $\Gamma$.
% To track the position of the phase indexed $k$ and the interfaces, we introduce the phase indicator function and the interface indicator function, 
% \begin{align}
%     \chi_k(\textbf{x},t) =  \left\{
%       \begin{tabular}{cc}
%         $1 \;\text{if} \;\textbf{x} \in \Omega_k(t)$\\
%         $0 \;\text{if} \;\textbf{x} \notin \Omega_k(t)$
%       \end{tabular}
%       \right.
%       \text{for $k = 1,2$},
%     %   \label{eq:PIF}
%     && \delta_I(\textbf{x},t) =  \left\{
%       \begin{tabular}{cc}
%         $1 \;\text{if} \;\textbf{x} \in \Gamma(t)$\\
%         $0 \;\text{if} \;\textbf{x} \notin \Gamma(t)$
%       \end{tabular}
%       \right.,
%     %   \label{eq:PIF}
% \end{align}
% respectively. 

In this chapter, we consider the following hypothesis for the properties of both phases:  
(1) We consider that the only governing equations are the mass and momentum conservation equations for Newtonian fluids without mass transfer. 
(2) We consider a constant density and viscosity in each phase. 
(3) The interfaces are described completely by the constant surface tension coefficient $\gamma$.
(4) the only body force considered is the gravity acceleration $\textbf{g}$. 
% \subsubsection{Inside the volumes}
Within phase $k$, we note $\rho_k$ the density and $\textbf{u}_k^0$ the local velocity.
Consequently, all over the domain $\Omega_k(t)$ the mass and momentum obey the Navier-Stokes equations :
\begin{align}
    % \label{eq:dt_rho}
    % \pddt \rho_k  
    \div 
        % \rho_k
        \textbf{u}_k^0
    &= 
    0\\
    % \label{eq:dt_rhou_k}
    \pddt (\rho_k\textbf{u}_k^0)  
    + \div (
        \rho_k\textbf{u}_k^0\textbf{u}_k^0
        - \bm{\sigma}_k^0 
    )
    &= 
    \rho_k \textbf{g}\\
    % \label{eq:dt_rhoE_k}
    % \pddt (\rho_kE_k^0)  
    % + \div (
    %     \rho_kE_k^0\textbf{u}_k^0
    %     + \textbf{q}_k^0
    %     - \textbf{u}_k^0 \cdot \bm{\sigma}_k^0 
    %     )
    % &= 
    % \textbf{u}_k^0 \cdot \textbf{g}  \rho_k
\end{align} 
where, $\bm{\sigma}_k^0 = - p_k^0 \bm\delta + 2\mu_k \textbf{e}_k^0$ with $\textbf{e}_k^0 = \grad \textbf{u}_k^0 + (\grad \textbf{u}_k^0)^\dagger$ the shear rate and $p_k^0$ the local pressure of phase $k$. 
On the interfaces we recall that the mass and momentum conservation equations are,
\begin{align}
    % \label{eq:dt_rho_I}
    \textbf{u}_I = \textbf{u}_k
    &=0, \\
    \Jump{\bm{\sigma}_k^0} 
    &=
    \divI[\gamma(\bm\delta - \textbf{nn})]
    =
    - \gamma\textbf{n} (\div \textbf{n}).
    % + \gradI\sigma 
    % \label{eq:surface_tension}\\
    % \label{eq:dt_rhoI_uI2}
    % \Jump{\textbf{u}_k^0 \cdot \bm{\sigma}_k^0}
    % &=
    %  \gamma\kappa\textbf{n}\cdot \textbf{u}_{I}^0\\
    % \label{eq:dt_rhoIe_I}
    % \Jump{ \textbf{q}_k^0}
    % &= 
    %  0
\end{align}


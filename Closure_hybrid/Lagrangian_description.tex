
\subsection{Generic first order description of the particles}

Let us introduce Lagrangian fundamental properties of a single particle. 
We define the mass, position of center of mass, momentum, second moment of mass, and moment of momentum (symmetric and skew-symmetric part)  of the particle $\alpha$ such as, 
\begin{align}
    m_\alpha
    &= \intO{ \rho_d  },\\
    \textbf{x}_\alpha
    &= \frac{1}{m_\alpha }\intO{ \rho_d \textbf{x} },\\
    \textbf{p}_\alpha 
    &= \intO{ \rho_d \textbf{u}_d^0 },\\
    % & m_\alpha E_\alpha 
    % &= \intO{ \rho_d [e_d^0 + (u_d^0)^2/2] },\\
    \textbf{M}_\alpha 
    &= \intO{ \rho_d \textbf{rr} }, \\
    \textbf{S}_\alpha 
    &= \frac{1}{2}\intO{ \rho_d (\textbf{r}\textbf{u}_d^0+\textbf{u}_d^0\textbf{r}) },\\
    \bm\mu_\alpha 
    &= \intO{ \rho_d \textbf{r}\times\textbf{u}_d^0 },
    \label{eq:position_and_momentum_def}
\end{align}
respectively. 
We recall that $\textbf{r} = \textbf{x} - \textbf{x}_\alpha(t)$ and that $\textbf{u}_\alpha = \textbf{p}_\alpha /m_\alpha$ is the definition of the center of mass velocity in the absence of mass transfer. 
Following the assumption made in the preceding subsection and the generalized derivation proposed in \ref{chap:daniel2} we easily show that each of these quantities follow the conservation equations, namely,
\begin{align}
    \label{eq:dt_m_alpha}
    \ddt m_\alpha
    &= 
    0\\
    \ddt {\textbf{x}_\alpha}
    &=\textbf{u}_\alpha. 
    \label{eq:dt_x_alpha}\\
    \label{eq:dt_p_alpha}
    \ddt \textbf{p}_\alpha
    &= 
    m_\alpha\textbf{g}
    +  \intS{\bm{\sigma}_f^0 \cdot \textbf{n}_d}\\
    \ddt {\textbf{M}_\alpha}
    &=2\textbf{S}_\alpha. 
    \label{eq:dt_M_alpha}
    \\
    \ddt {\bm\mu_\alpha}
    &=
     \intS{ \textbf{r}\times(\bm{\sigma}_f^0\cdot \textbf{n}_d)} 
    \label{eq:dt_mu_alpha}\\
    \ddt {\textbf{S}_\alpha}
    &= \intO{ \left(
        \rho_d  \textbf{w}_d^0 \textbf{w}_d^0 
        - \bm{\sigma}_d^0
    \right) }
    - \intS{ 
        \gamma (\bm\delta - \textbf{nn})
    }
    + \frac{1}{2}\intS{ (\textbf{r}\bm{\sigma}_f^0+\bm{\sigma}_f^0\textbf{r})\cdot \textbf{n}_d} 
    \label{eq:dt_S_alpha}
    % \label{eq:dt_E_alpha}
    % \ddt E_\alpha^\text{tot}
    % &= 
    % m_\alpha \textbf{u}_\alpha \cdot \textbf{g}
    % +\textbf{u}_\alpha \cdot \intS{\bm{\sigma}_f^0 \cdot \textbf{n}_d}
    % +\intS{\textbf{w}_f^0 \cdot \bm{\sigma}_f^0 \cdot  \textbf{n}_d} 
    % - \intS{\textbf{q}_f^0 \cdot \textbf{n}_d} 
\end{align}
This set of equation can be complemented with an equation for $M_\alpha = \frac{1}{3}\intO{\rho_d \textbf{r}\cdot \textbf{r}}$ defined as the trace of $\textbf{M}_\alpha$. 
It reads, 
\begin{equation}
    \frac{3}{2}\frac{d^2 M_\alpha}{dt^2}
    - \intO{ \rho_d \textbf{w}_d^0 \cdot \textbf{w}_d^0}
    = 
    - \intO{\bm\sigma_d^0:\bm\delta} 
    % - \frac{1}{3}\intS{p_f^0 \textbf{r}\cdot \textbf{n}}
    - \gamma 2 \intS{}
    % - \frac{1}{3}\intS{p_f^0 \textbf{r}\cdot \textbf{n}}
    + \intS{\textbf{r}\cdot\bm\sigma_f^0\cdot \textbf{n}}.
    \label{eq:dt_D_alpha}
\end{equation}
At this stage, the properties and evolution equations of the particles remain general and can be adapted to various types of problems. 
This general framework involves the integral of local quantities, such as the integral of $\textbf{w}_d^0$, $\bm\sigma_d^0$ \ldots which are unknown. 
Therefore, these integral terms can be considered as closure terms since they are not expressed in terms of Lagrangian unknown, what we called particle fundamental properties. 
For example, for solid particles we may simplify most of the integrals terms in these equations by substituting the particle internal velocity, $\textbf{w}_d^0$ with $ \bm\omega \times \textbf{r}$. 
This solid body motion constitutes the most simple closure of this hybrid model.  
Therefore, in the case of deformable particle we must reduce the degrees of freedom of the particle shape and simplify the corresponding equations.
In the next section we adopt restrictive hypotheses regarding the particle shape and internal kinematic which allows us to partially close these equations. 
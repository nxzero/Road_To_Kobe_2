
\subsection{Generic first order description of the particles}

Let us introduce the fundamental Lagrangian properties of a single particle. 
We define the mass, position of the center of mass, momentum, second moment of mass, and moment of momentum, including both the symmetric and skew-symmetric parts, of particle $\alpha$ such as, 
\begin{align}
    m_\alpha
    &= \intO{ \rho_d  },\\
    \textbf{x}_\alpha
    &= \frac{1}{m_\alpha }\intO{ \rho_d \textbf{x} },\\
    \textbf{p}_\alpha 
    &= \intO{ \rho_d \textbf{u}_d^0 },\\
    % & m_\alpha E_\alpha 
    % &= \intO{ \rho_d [e_d^0 + (u_d^0)^2/2] },\\
    \textbf{M}_\alpha 
    &= \intO{ \rho_d \textbf{rr} }, \\
    \textbf{S}_\alpha 
    &= \frac{1}{2}\intO{ \rho_d (\textbf{r}\textbf{u}_d^0+\textbf{u}_d^0\textbf{r}) },\\
    \bm\mu_\alpha 
    &= \intO{ \rho_d \textbf{r}\times\textbf{u}_d^0 },
    \label{eq:position_and_momentum_def}
\end{align}
respectively. 
We recall that $\textbf{r} = \textbf{x} - \textbf{x}_\alpha$ and that $\textbf{u}_\alpha = \textbf{p}_\alpha /m_\alpha$ is the definition of the center of mass velocity in the absence of mass transfer. 
Following the assumptions made in the preceding subsection and the generalized derivation proposed in \ref{chap:daniel2} it can be readily shown that each of these quantities satisfies the following conservation equations:
\begin{align}
    \label{eq:dt_m_alpha}
    \ddt m_\alpha
    &= 
    0\\
    \ddt {\textbf{x}_\alpha}
    &=\textbf{u}_\alpha. 
    \label{eq:dt_x_alpha}\\
    \label{eq:dt_p_alpha}
    \ddt \textbf{p}_\alpha
    &= 
    m_\alpha\textbf{g}
    +  \intS{\bm{\sigma}_f^0 \cdot \textbf{n}_d}\\
    \ddt {\textbf{M}_\alpha}
    &=2\textbf{S}_\alpha. 
    \label{eq:dt_M_alpha}
    \\
    \ddt {\textbf{S}_\alpha}
    &= \intO{ \left(
        \rho_d  \textbf{w}_d^0 \textbf{w}_d^0 
        - \bm{\sigma}_d^0
    \right) }
    - \intS{ 
        \gamma (\bm\delta - \textbf{nn})
    }
    + \frac{1}{2}\intS{ (\textbf{r}\bm{\sigma}_f^0+\bm{\sigma}_f^0\textbf{r})\cdot \textbf{n}_d} 
    \label{eq:dt_S_alpha}\\
    \ddt {\bm\mu_\alpha}
    &=
     \intS{ \textbf{r}\times(\bm{\sigma}_f^0\cdot \textbf{n}_d)} 
    \label{eq:dt_mu_alpha}
    % \label{eq:dt_E_alpha}
    % \ddt E_\alpha^\text{tot}
    % &= 
    % m_\alpha \textbf{u}_\alpha \cdot \textbf{g}
    % +\textbf{u}_\alpha \cdot \intS{\bm{\sigma}_f^0 \cdot \textbf{n}_d}
    % +\intS{\textbf{w}_f^0 \cdot \bm{\sigma}_f^0 \cdot  \textbf{n}_d} 
    % - \intS{\textbf{q}_f^0 \cdot \textbf{n}_d} 
\end{align}
This set of equations can be complemented with an equation for $M_\alpha = \frac{1}{3}\intO{\rho_d \textbf{r}\cdot \textbf{r}}$ which is obtained by taking the double contracted product of \ref{eq:dt_S_alpha} with $\bm\delta$,
it reads, 
\begin{equation}
    \frac{3}{2}\frac{d^2 M_\alpha}{dt^2}
    - \intO{ \rho_d \textbf{w}_d^0 \cdot \textbf{w}_d^0}
    = 
    - \intO{\bm\sigma_d^0:\bm\delta} 
    % - \frac{1}{3}\intS{p_f^0 \textbf{r}\cdot \textbf{n}}
    - \gamma 2 \intS{}
    % - \frac{1}{3}\intS{p_f^0 \textbf{r}\cdot \textbf{n}}
    + \intS{\textbf{r}\cdot\bm\sigma_f^0\cdot \textbf{n}}.
    \label{eq:dt_D_alpha}
\end{equation}
At this stage, the properties and evolution equations of the particles remain general and can be adapted to various types of problems. 
This general framework involves integrals of local quantities, such as $\textbf{w}_d^0$, $\bm\sigma_d^0$ \ldots and others, which are currently unknown. 
Therefore, these integral terms can be considered as closure terms since they are not yet expressed in terms of the Lagrangian unknowns, referred to as the particle's fundamental properties.
For example, in the case of solid particles, we can simplify most of these integral terms by substituting the particle's internal velocity, $\textbf{w}_d^0$ with $ \bm\omega \times \textbf{r}$. 
This solid body motion represents the simplest closure for these equations.  
Thus, for deformable fluid particles, we must find a way to reduce the degrees of freedom related to the particle shape and internal motion and develop a method to simplify the corresponding equations by finding closures for $\bm\sigma_d^0$ and $\textbf{w}_d^0$.
Consequently, in the next section, we adopt restrictive hypotheses regarding the particle shape and internal kinematics  , which will allow us to partially close these equations.
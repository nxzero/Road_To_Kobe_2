
\section{Arbitrary order moments equation}
\label{ap:Moments_equations}
In this appendix we extend the Lagrangian conservation laws to an arbitrary order moment equation. 
Let's define the arbitrary moment of the Eulerian field $f$, by, 
\begin{equation*}
    \mathcal{Q}_{i_1\ldots i_n}
    = \int_{\Omega_\alpha} 
    \pri{1}{n} f d\Omega
\end{equation*}
Then by using the Reynolds transport theorem we can equally show that :
\begin{multline*}
    \ddt {\mathcal{Q}_{i_1\ldots i_n}}
    =\int_{\Omega_\alpha} \left[ \partial_t \left(\pri{1}{n}f\right) 
    + \partial_k \left(u_k \pri{1}{n}f\right) \right]d\Omega\\
    +\int_{\Sigma_\alpha} \pri{1}{n} f \left(u^I_k - u_k\right) n_k d\Sigma.
\end{multline*}
Using the product rule on the first integral derivatives yields, 
\begin{multline*}
    \ddt {\mathcal{Q}_{i_1\ldots i_n}}
    =\int_{\Omega_\alpha} f \left[ \partial_t \left(\pri{1}{n}\right) 
    + u_k \partial_k \left( \pri{1}{n}\right) \right]d\Omega\\
    +\int_{\Omega_\alpha} \pri{1}{n} \left[ \partial_t \left(f\right) 
    +  \partial_k \left(u_k f \right) \right]d\Omega\\
    +\int_{\Sigma_\alpha} \pri{1}{n} f \left(u^I_k - u_k\right) n_k d\Sigma.
\end{multline*}
Proceeding with a similar manner than for the first moment conservation equation we can equally show that, 
\begin{multline*}
    \ddt {\mathcal{Q}_{i_1\ldots i_n}}
    = \sum_{e=1}^{n} \int_{\Omega_\alpha} f  \prod^{n}_{\substack{ m=1 \\   m \neq e}} r_{i_m} w_{i_e}d\Omega
    +\int_{\Omega_\alpha} \pri{1}{n} \div\mathbf{\Phi} d\Omega\\
    + \int_{\Omega_\alpha} \pri{1}{n} \textbf{S} d\Omega
    +\int_{\Sigma_\alpha} \pri{1}{n} f \left(u^I_k - u_k\right) n_k d\Sigma.
\end{multline*}
The second term of this equation can be written,
\begin{align*}
    \int_{\Omega_\alpha} \pri{1}{n} \div\mathbf{\Phi} d\Omega
    &= \int_{\Sigma_\alpha} \div \left(\pri{1}{n} \mathbf{\Phi} \right)d\Omega
    - \int_{\Omega_\alpha} \mathbf{\Phi} \cdot \grad \left(\pri{1}{n} \right)d\Omega\\
    &= \int_{\Sigma_\alpha} \pri{1}{n} \mathbf{\Phi} \cdot \textbf{n}d\Sigma
    -\sum_{e=1}^{n} \int_{\Omega_\alpha} \mathbf{\Phi}  \prod^{n}_{\substack{ m=1 \\m \neq e}} r_{i_m}  d\Omega
\end{align*}
Including this relation into the former equation yields, 
\begin{multline}
    \ddt {\mathcal{Q}_{i_1\ldots i_n}}
    = \sum_{e=1}^{n} \int_{\Omega_\alpha} \prod^{n}_{\substack{ m=1 \\   m \neq e}} r_{i_m} (w_{i_e}f  - \mathbf{\Phi}_{i_e})d\Omega
    +\int_{\Sigma_\alpha} \pri{1}{n} \mathbf{\Phi} \cdot \textbf{n}d\Sigma\\
    + \int_{\Omega_\alpha} \pri{1}{n} \textbf{S} d\Omega
    +\int_{\Sigma_\alpha} \pri{1}{n} f \left(u^I_k - u_k\right) n_k d\Sigma.
    \label{eq:dt_Q_n}
\end{multline}
For scalar quantity this expression is purely symmetric since it involve only product of the $r_{i_n}$ with scalar, and sum over all vector indexed by $i_e$
Let focus on the case were $f$ is a vector indexed $k$ we have, 
\begin{multline}
    \ddt {\mathcal{Q}_{i_0 i_1\ldots i_n }}
    = \sum_{e=1}^{n} \int_{\Omega_\alpha} \prod^{n}_{\substack{ m=1 \\   m \neq e}} r_{i_m} (w_{i_e}f_{i_0}  - \mathbf{\Phi}_{i_0 i_e})d\Omega
    +\int_{\Sigma_\alpha} \pri{1}{n} (\mathbf{\Phi} \cdot \textbf{n})_{i_0}d\Sigma
    + \int_{\Omega_\alpha} \pri{1}{n} \textbf{S}_{i_0} d\Omega
\end{multline}
The symmetric part of $\mathcal{Q}_{i_0 i_1\ldots i_n}$ is, 
\begin{equation*}
    \mathcal{Q}_{(i_0 i_1\ldots i_p \ldots i_n )}
= \frac{1}{n+1}
\sum_{p=0}^{n} \mathcal{Q}_{i_p (i_1\ldots i_0\ldots i_n)}
\end{equation*}
where the parenthesis indicates the symmetric index, and it must be understood that this is permutation of the indices.  
Therefore, the fully symmetric part of the preceding momentum balance can be obtained by summing every permutation of the index $k$ with all other index and dividing by $n$, namely,
\begin{multline}
    \ddt {\mathcal{Q}_{(i_0 i_1\ldots i_n) }}
    = \frac{1}{n+1}
    \sum_{p=0}^{n}
    \sum_{\substack{ e=0 \\   e \neq i_p}}^{n} \int_{\Omega_\alpha} 
    \prod^{n}_{\substack{ m=0 \\   m \neq e}} r_{i_m} (w_{i_e}f_{i_p}  - \mathbf{\Phi}_{i_p i_e})d\Omega\\
    +\frac{1}{n+1}
    \sum_{p=0}^{n}
    \int_{\Sigma_\alpha} \prod^{n}_{\substack{ m=0 \\   m \neq i_p}} r_{i_m}
    (\mathbf{\Phi} \cdot \textbf{n})_{i_p}d\Sigma
    + \int_{\Omega_\alpha} 
    \prod^{n}_{\substack{ m=0 \\   m \neq i_p}} r_{i_m}
    \textbf{S}_{i_p} d\Omega
\end{multline}
It appears that this equation is the fully symmetric parts of the moments equations. 
The skew symmetric parts will be written, 
\begin{multline}
    \frac{d}{dt} (
    \mathcal{Q}_{i_0 i_1\ldots i_n} 
    - \mathcal{Q}_{(i_0 i_1\ldots i_n) }
    )
    = 
    \sum_{e=1}^{n} \int_{\Omega_\alpha} \prod^{n}_{\substack{ m=1 \\   m \neq e}} r_{i_m} (w_{i_e}f_{i_0}  - \mathbf{\Phi}_{i_0 i_e})d\Omega
    -
    \frac{1}{n+1}
    \sum_{p=0}^{n}
    \sum_{\substack{ e=0 \\   e \neq i_p}}^{n} \int_{\Omega_\alpha} 
    \prod^{n}_{\substack{ m=0 \\   m \neq e}} r_{i_m} (w_{i_e}f_{i_p}  - \mathbf{\Phi}_{i_p i_e})d\Omega\\
    +\int_{\Sigma_\alpha} \pri{1}{n} (\mathbf{\Phi} \cdot \textbf{n})_{i_0}d\Sigma
    -
    \frac{1}{n+1}
    \sum_{p=0}^{n}
    \int_{\Sigma_\alpha} \prod^{n}_{\substack{ m=0 \\   m \neq i_p}} r_{i_m}
    (\mathbf{\Phi} \cdot \textbf{n})_{i_p}d\Sigma
    + \int_{\Omega_\alpha} \pri{1}{n} \textbf{S}_{i_0} d\Omega
    -
    \int_{\Omega_\alpha} 
    \prod^{n}_{\substack{ m=0 \\   m \neq i_p}} r_{i_m}
    \textbf{S}_{i_p} d\Omega
\end{multline}
It is known that the non-convective fluxes vanish at the order one of this equation. 
We would like to make appear this property explicitly. 
\begin{multline*}
    \sum_{e=1}^{n} \int_{\Omega_\alpha} \prod^{n}_{\substack{ m=1 \\   m \neq e}} r_{i_m} \mathbf{\Phi}_{i_0 i_e} d\Omega
    -
    \frac{1}{n+1}
    \sum_{p=0}^{n}
    \sum_{\substack{ e=0 \\   e \neq i_p}}^{n} \int_{\Omega_\alpha} 
    \prod^{n}_{\substack{ m=0 \\   m \neq e}} r_{i_m}  \mathbf{\Phi}_{i_p i_e}d\Omega\\
    =
    \sum_{e=1}^{n} \int_{\Omega_\alpha} \prod^{n}_{\substack{ m=1 \\   m \neq e}} r_{i_m} \mathbf{\Phi}_{i_0 i_e}d\Omega
    -
    \frac{1}{n+1}
    \sum_{p=0}^{n}
    \sum_{\substack{ e=0 \\   e \neq i_p}}^{n} \int_{\Omega_\alpha} 
    \prod^{n}_{\substack{ m=0 \\   m \neq e}} r_{i_m}  \mathbf{\Phi}_{i_p i_e}d\Omega\\
    =
    \frac{- 1}{n+1}
    \sum_{p=1}^{n}
    \sum_{\substack{ e=1 \\   e \neq i_p}}^{n} \int_{\Omega_\alpha} 
    \prod^{n}_{\substack{ m=1 \\   m \neq e}} r_{i_m}  \mathbf{\Phi}_{i_p i_e}d\Omega
\end{multline*}
Which makes a non-vanishing parts for the integral of the stress. 
Instead, we rather derive the moments' equation antisymmetric in the indices $i_e$ $i_0$ by subtracting the permuted equation
\begin{multline}
    \ddt{ \mathcal{Q}_{i_0 i_1\ldots i_n }}
    = \sum_{e=1}^{n} \int_{\Omega_\alpha} \prod^{n}_{\substack{ m=1 \\   m \neq e}} r_{i_m} (w_{i_e}f_{i_0}  - \mathbf{\Phi}_{i_0 i_e})d\Omega
    +\int_{\Sigma_\alpha} \pri{1}{n} (\mathbf{\Phi} \cdot \textbf{n})_{i_0}d\Sigma
    + \int_{\Omega_\alpha} \pri{1}{n} \textbf{S}_{i_0} d\Omega
\end{multline}

As an example we give the two first order moments for particles without mass transfer: 
If $n=1$ : 
\begin{equation}
    \ddt{ \mathcal{Q}_{i_1}}
    = \int_{\Omega_\alpha} (w_{i_1}f  - \mathbf{\Phi}_{i_1})d\Omega
    +\int_{\Sigma_\alpha} r_{i_1}\mathbf{\Phi} \cdot \textbf{n}d\Sigma
    + \int_{\Omega_\alpha}r_{i_1} \textbf{S} d\Omega
\end{equation}
and for $n=2$ : 
\begin{multline}
    \label{eq:moment_n2}
    \ddt {\mathcal{Q}_{i_1 i_2}}
    = 
    \int_{\Omega_\alpha} r_{i_2} (w_{i_1}f  - \mathbf{\Phi}_{i_1})d\Omega
    +\int_{\Omega_\alpha} r_{i_1} (w_{i_2}f  - \mathbf{\Phi}_{i_2})d\Omega
    +\int_{\Sigma_\alpha}  r_{i_1}r_{i_2} \mathbf{\Phi} \cdot \textbf{n}d\Sigma\\
    + \int_{\Omega_\alpha} r_{i_1}r_{i_2}  \textbf{S} d\Omega
\end{multline}
For the momentum equation we obtain : 
\begin{equation}
    \ddt{ \mathcal{P}_{ij}}
    = \int_{\Omega_\alpha} (w_{i}w_j \rho_2  - \bm{\sigma}_{ij})d\Omega
    +\int_{\Sigma_\alpha} r_{i} \sigma_{jk} \cdot n_k d\Sigma
    + \int_{\Omega_\alpha}r_{i} \rho_d g_j d\Omega
\end{equation}
\begin{multline*}
    \ddt{ \mathcal{P}_{i j k}}
    = 
    \int_{\Omega_\alpha} r_{j} (w_{i} w_k\rho_2 - \sigma_{ik})d\Omega
    +\int_{\Omega_\alpha} r_{i} (w_{j} w_k\rho_2 - \sigma_{jk})d\Omega
    +\int_{\Sigma_\alpha}  r_{i}r_{j} \sigma_{kl} n_l d\Sigma\\
    + \int_{\Omega_\alpha} r_{i}r_{j}  \rho_2 g_k d\Omega
\end{multline*}


Then it is possible from this equation to carry out a particle-average, which directly yield the $n^{th}$ order moment equation : 
\begin{multline*}
    \pddt \pavg{\mathcal{Q}_{i_1\ldots i_n}^\alpha}
    + \partial_k  \pavg{\mathcal{Q}_{i_1\ldots i_n}^\alpha u_k^\alpha}
    = \sum_{e=1}^{n} \pavg{\int_{\Omega_\alpha} \prod^{n}_{\substack{ m=1 \\   m \neq e}} r_{i_m} (w_{i_e}f  - \mathbf{\Phi}_{i_e})d\Omega}\\
    + \pavg{\int_{\Omega_\alpha} \pri{1}{n} \textbf{S} d\Omega}
    + \pavg{\int_{\Omega_\alpha} \pri{1}{n} \left[
            \mathbf{\Phi}_k
            + f_k
            \left(
                \textbf{u}_I
                - \textbf{u}_k
            \right)
        \right]
        \cdot \textbf{n}_kd\Omega}
\end{multline*}

\section{Arbitrary order equivalence}
\label{sec:demo}
In this appendix we provide a general proof of \ref{eq:scheme_equivalence} between particle-averaged and phase-avergaed equation for the dispersed phase. 
Let's begin by re-writing the phase averaged equation
\begin{equation}
        \pddt \avg{\chi_k f_k}
        = \div \avg{\chi_k \mathbf{\Phi}_k - \chi_k f_k \textbf{u}_k}
        + \avg{\chi_k \textbf{S}_k}
        + \avg{\delta_I\left[
            \mathbf{\Phi}_k
            + f_k
            \left(
                \textbf{u}_I
                - \textbf{u}_k
            \right)
        \right]
        \cdot \textbf{n}_k} 
\end{equation}
expanding each term using the relations \ref{eq:f_exp} directly gives,
\begin{align*}
        0 &=
        - \pddt \expo{f_k} \\
        &+\div \expo{(\mathbf{\Phi}_k - f_k \textbf{u}_k)}\\
        &+ \expo{ \textbf{S}_k}\\
        &+ \expoS{\left[
            \mathbf{\Phi}_k
            + f_k
            \left(
                \textbf{u}_I
                - \textbf{u}_k
            \right)
        \right]
        \cdot \textbf{n}_k} \\
\end{align*}
The third term can be reformulated using the decomposition : $\textbf{u}_2 = \textbf{u}_\alpha + \textbf{w}_\alpha$, which gives,
\begin{multline}
    \expo{f_k \textbf{u}_k}\\
    =     \expoU{f_k }\\
    +     \expo{f_k \textbf{w}_k}
\end{multline}
Injecting this formulation in the former equation yields,
\begin{align}
    & \pddt \expo{f_k} \\
    &+ \div \expoU{f_k}\\
    &= \div \expo{(\mathbf{\Phi}_k - f_k \textbf{w}_k)}\\
    &+ \expo{ \textbf{S}_k}\\
    &+ \expoS{\left[
        \mathbf{\Phi}_k
        + f_k
        \left(
            \textbf{u}_I
            - \textbf{u}_k
        \right)
    \right]
    \cdot \textbf{n}_k} \\
\end{align}
Finlay we can factor out the gradient operator which gives, 
\begin{multline*}
    0 = \frac{(-1)^n}{n!}
    \partialp{1}{n}
    \left[
        - \partial_t
        \pavg{\int_{\Omega_\alpha} \pri{1}{n}f_k d\Omega}
        - \div \pavg{\textbf{u}_\alpha \int_{\Omega_\alpha} \pri{1}{n}f_k d\Omega}
    \right.\\\left.
        +n\pavg{\int_{\Omega_\alpha} \pri{1}{n-1} (\mathbf{\Phi}_k - f_k \textbf{w}_k) d\Omega}
        +\pavg{\int_{\Omega_\alpha} \pri{1}{n} \textbf{S}_k d\Omega}
        \right.\\\left.
        +\pavg{\int_{\Omega_\alpha} \pri{1}{n} \left[
            \mathbf{\Phi}_k
            + f_k
            \left(
                \textbf{u}_I
                - \textbf{u}_k
            \right)
        \right]
        \cdot \textbf{n}_kd\Omega}
    \right]
\end{multline*}
In the square brackets we recognize the moment equaiton of order $n$ which prooves \ref{eq:scheme_equivalence}.
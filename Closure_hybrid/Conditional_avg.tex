\section{Conditional average fields definition}

\subsection{Lagrangian fields}
The ensemble average of an arbitrary local lagrangian field $\textbf{q}_\alpha\delta_\alpha(\textbf{x},\CC,t)$ might be computed as,
\begin{equation*}
    \textbf{q}_pn_p(\textbf{x},t)
    = \int
    \sum_{\alpha=0}^N
    \delta(\textbf{x} - \textbf{x}_\alpha(\FF,t))
     \textbf{q}_\alpha(\CC,t) 
     d\PP. 
\end{equation*}
The pair probability density is defined by, 
\begin{equation*}
    P_2(\textbf{x},\textbf{y},t)
    = \int
    \sum_{\alpha}^N
    \delta(\textbf{x} - \textbf{x}_\alpha(\FF,t))
    \sum_{\beta\neq\alpha}^N
    \delta(\textbf{y} - \textbf{x}_\beta(\CC,t))
     d\PP. 
\end{equation*}
And the average of conditional particle pair property $\textbf{q}_{\alpha,\beta}(\CC,t)$ might be defined by, 
\begin{equation*}
    \textbf{q}_{p,2}(\textbf{x},\textbf{y},t) P_2(\textbf{x},\textbf{y},t)
    = \int
    \sum_{\alpha}^N
    \delta(\textbf{x} - \textbf{x}_\alpha(\FF,t))
    \sum_{\beta\neq\alpha}^N
    \delta(\textbf{y} - \textbf{x}_\beta(\CC,t))
    \textbf{q}_{\alpha,\beta}(\CC,t) 
     d\PP. 
\end{equation*}
It follows from basic probabilistic principles that, $P_2(\textbf{x},\textbf{y},t) = P_2(\textbf{x}|\textbf{y},t)P_1(\textbf{y})$. 
Assuming molecular chaos and dilute regime one can stipulate that $P_2(\textbf{x},\textbf{y},t) = P_2(\textbf{x},t)P_1(\textbf{y},t)$. 
Assuming the latter we can define a new relation for the particle averaged by integrating the former relation, namely, 
\begin{equation*}
    \textbf{q}_pn_p(\textbf{x},t)
    =
    \int_{\mathbb{R}^3}
    \textbf{q}_{p,2}(\textbf{x},\textbf{y},t) P_2(\textbf{x},\textbf{y},t)
    d\textbf{y}
    \approx
    P(\textbf{x},t)
    \int_{\mathbb{R}^3}
    \textbf{q}_{p,2}(\textbf{x},\textbf{y},t) P(\textbf{y},t)
    d\textbf{y}
\end{equation*}


\subsubsection*{\citet{batchelor1972sedimentation} and \citet{zhang1994ensemble} ensemble average statistics}

The probability density function $P(\CC^N)$ can be defined with the Dirac delta function as, 
\begin{equation*}
    P(\CC^N,t)
    =
    P(\textbf{x}_1,\ldots,\textbf{x}_N)
    = \int 
    \prod_{i=1}^{N}\sum_{\alpha}^{N}\delta(\textbf{x}_i - \textbf{x}_\alpha(\FF,t))
    d\PP,
\end{equation*}
The norm of this PDF can then be obtained by integrating over all $d\CC^N$ by noticing that we have by definition $\int d\PP = 1$, 
\begin{align*}
    \int_{\mathbb{R}^N} P(\CC^N,t) d\CC^N
    &= 
    \int 
    \int_{\mathbb{R}^N}
    \prod_{i=1}^{N}
    \sum_{\alpha}^{N}
    \delta(\textbf{x}_i - \textbf{x}_\alpha(\FF,t))
    d\CC^N
    d\PP,\\
    &= 
    \int 
    \prod_{i=1}^{N}\left[
    \sum_{\alpha}^{N}
    \int_{\mathbb{R}^3}
    \delta(\textbf{x}_i - \textbf{x}_\alpha(\FF,t))
    d\textbf{x}_i\right]
    d\PP,
\end{align*}
Then, $\int \delta(\textbf{x}_i - \textbf{x}_\alpha(\FF,t)) d\textbf{x}_i= 1$ by definition of the Dirac delta function. 
Thus, we have got 
\begin{align*}
    \int_{\mathbb{R}^N} P(\CC^N,t) d\CC^N
    &= 
    N^N
\end{align*}
It is however possible to exclude some non-physical event such has : a single particle cannot be at $\textbf{x}_1$ and $\textbf{x}_2$ at the same $\CC,t$. 
In which case all the product $\delta(\textbf{x}_i - \textbf{x}_\alpha)\delta(\textbf{x}_j - \textbf{x}_\alpha)\ldots  = 0$

which goes along with the fields' configuration average of a local quantity $\textbf{f}^0(\CC,x,t)$, namely, 
\begin{equation*}
    \textbf{f}(\textbf{x},t;\CC^N) P(\CC^N)
    = \int 
    \prod_{i=1}^{N}\sum_{\alpha}^{N}\delta(\textbf{x}_i - \textbf{x}_\alpha(\FF,t)) \textbf{f}^0(\textbf{x},t;\CC)
    d\PP,
\end{equation*}
for a Lagrangian quantity it is exactly the same but first multiply the Lagrangian quantity by its own Dirac delata function thus we average $\delta(\textbf{x}-\textbf{x}_\alpha(\FF,t)) \textbf{f}_\alpha$ instead, which yields, 
\begin{equation*}
    \textbf{f}_p(\textbf{x},t;\CC^N) P(\textbf{x},\CC^N)
    = \int 
    \prod_{i=1}^{N}\sum_{\alpha}^{N}\delta(\textbf{x}_i - \textbf{x}_\alpha(\FF,t)) 
    \delta(\textbf{x} - \textbf{x}_\alpha(\FF,t)) 
    \textbf{f}_\alpha(t,\CC)
    d\PP,
\end{equation*}
The difference is in the presence of the $\textbf{x}$ in the particle PDF function. 
That will make appear the number density function in the end. 
Notice that while $\CC^N$ represent the Eulerian fields of the position of the $N$ particle, the $\CC$ represent the complete configurations.
Besides $\alpha$ is the index of the $N$ lagrangian particles' posiiton and $i$ the index of the $N$ eulerian phase space.
Beside $P(\CC)$  might be set up as non-time dependent while $P(\CC^N,t)$ must be time dependent through the definition above contrary to what stated \citet{zhang1994ensemble}. 

Note that from the transport equation of each Dirac delta function, i.e. 
\begin{equation*}
    \pddt \delta(\textbf{x}_i - \textbf{x}_\alpha(\FF,t))
    + \textbf{u}_\alpha(\CC,t)\cdot\partial_{\textbf{x}_i}\delta(\textbf{x}_i-\textbf{x}_\alpha(\FF,t))
    =0 
\end{equation*}
it is easy to derive a Louville-like equation for the full PDF of $\CC^N$ by adding and multiply each Dirac delta function. 
Namely, 
\begin{equation*}
    \pddt \left[
        \prod_{i=1}^{N}\sum_{\alpha}^{N}\delta(\textbf{x}_i - \textbf{x}_\alpha(\FF,t)) 
    \right]
    + 
    \sum_{j=1}^N \partial_{\textbf{x}_j}\cdot\left[
        \prod_{i=1}^{N}\sum_{\alpha}^{N}\delta(\textbf{x}_i - \textbf{x}_\alpha(\FF,t)) \textbf{u}_\alpha(t,\CC) 
    \right]
    =0 
\end{equation*}
Integrating this equation for each flow configuration $d\PP$ leads us to, 
\begin{equation*}
    \pddt P(\CC^N)
    + 
    \sum_{j=1}^N \partial_{\textbf{x}_j} \cdot
    (\textbf{u}_p(\CC^N,t) P(\CC^N) )
    =0 
\end{equation*}


In \citet{batchelor1972sedimentation} they define the ensemble average by, 
\begin{equation*}
    \textbf{f}(\textbf{x},t)
    = \frac{1}{N^N} 
    \int_{\mathbb{R}^{3N}}
    \textbf{f}(\textbf{x},t;\CC^N)
    P(\CC^N,t)
    d\CC^N
    % = \int_{\mathbb{R}^{3N}}
    % \int 
    % \prod_{i=1}^{N}\sum_{\alpha}^{N}\delta(\textbf{x}_i - \textbf{x}_\alpha(\FF,t)) \textbf{f}^0(\textbf{x},t;\CC)
    % d\PP
    % d\CC^N
\end{equation*} 
This ultimately implies the norm, 
\begin{equation*}
    \int 
    P(\CC^N)
    d\CC^N
    = 1
\end{equation*}
which we consider true by definition of $P(\CC)$. 
One of the usefulness of this average is that we can derive an equation for the one particle averaged. 
\begin{equation*}
    \textbf{f}(\textbf{x},t)
    = \int_{\mathbb{R}^{3N}}
    \textbf{f}(\textbf{x},t;\CC^N)
    P(\CC^N,t)
    d\CC^N
    = 
    \int_{\mathbb{R}^{3}}
    \int_{\mathbb{R}^{3(N-1)}}
    \textbf{f}(\textbf{x},t;\CC^N)
    P(\CC^{N-1},t;\CC^1)
    d\CC^{N-1}
    P(\CC^1,t)
    d\CC^1
    % = \int_{\mathbb{R}^{3N}}
    % \int 
    % \prod_{i=1}^{N}\sum_{\alpha}^{N}\delta(\textbf{x}_i - \textbf{x}_\alpha(\FF,t)) \textbf{f}^0(\textbf{x},t;\CC)
    % d\PP
    % d\CC^N
\end{equation*} 
The terms within the second integration can in fact be regarded as the conditional average, 
\begin{equation*}
    \textbf{f}(\textbf{x},t;\CC^1)
    = 
    \int_{\mathbb{R}^{3(N-1)}}
    \textbf{f}(\textbf{x},t;\CC^N)
    P(\CC^{N-1},t;\CC^1)
    d\CC^{N-1}
\end{equation*}
Since it has been averaged over all other particles' configuration. 
Thus, we finally obtain the relation, 
\begin{equation*}
    \textbf{f}(\textbf{x},t)
    =
    \int_{\mathbb{R}^{3}}
    \textbf{f}(\textbf{x},t;\CC^1)P(\CC^1,t)
    d\CC^1
\end{equation*} 
Which reduce the ensemble average to the knowledge of the only one particle PDF 
$P(\CC^1,t)$ however the counterpart is teh apparition of the one particle conditionally averaged quantity $\textbf{f}(\textbf{x},t;\CC^1)$. 
Notice that under these notations $\CC^1 = \textbf{x}^1$. 
In terms of dirac delata funciton the latter quantity might be written as, 
\begin{equation*}
    \textbf{f}(\textbf{x},t;\CC^1)P(\CC^1,t)
    = \int_{\mathbb{R}^{3(N-1)}}
    \int 
    \prod_{i=1}^{N}\sum_{\alpha}^{N}\delta(\textbf{x}_i - \textbf{x}_\alpha(\FF,t)) \textbf{f}^0(\textbf{x},t;\FF)
    d\PP
    d\CC^{N-1}
    \label{eq:conditional_def}
\end{equation*}

So to carry out the derivation for the conditional averaged equation one must carry an operator such as it is defined above. 
$\mathscr{X}^N$

If the function $\textbf{f}(\textbf{x},t,\FF)$ turns out to be the PIF. 
Then for solid spherical particles $\textbf{f}(\textbf{x},t;\CC^N)$ contain as much as information as $\textbf{f}(\textbf{x},t,\FF)$ and can be described by a sum of Heaviside functions, 
\begin{equation}
    \chi_2(\textbf{x},t,\FF)
    = \sum_\alpha
    H(a - |\textbf{x} -\textbf{x}_\alpha(\FF,t)|)
\end{equation}
Therefore the ensemble average of $\chi_2$ can be expressed as, 
\begin{equation}
    \phi_2(\textbf{x},t)
    = \int \chi_2(\textbf{x},t;\FF) d\PP
\end{equation}
Or equally, as, 
\begin{equation}
    \phi_2(\textbf{x},t)
    =\frac{1}{N^N} \int_{\mathbb{R}^{3N}} \phi_2[\textbf{x},t;\CC^N] P(\CC^N) d\CC^N
\end{equation}
with, 
\begin{equation*}
    \phi_2[\textbf{x},t;\CC^N] P(\textbf{x},\CC^N)
    = 
    \int 
    \prod_{i=1}^{N}\sum_{\alpha}^{N}\delta(\textbf{x}_i - \textbf{x}_\alpha(\FF,t)) 
    \chi_2(\textbf{x},t,\FF)
    d\PP,
\end{equation*}

Then it is possible to reduce this statistical description to the single use of one particle. 
To do so we use the relation introduced latter, 
\begin{equation}
    \phi_2[\textbf{x},t]
    =\frac{1}{N} 
    \int_{\mathbb{R}^3} 
    \phi_2[\textbf{x},t;\CC^1]
    P(\CC^1)
    d\CC^1
\end{equation}
With, 
\begin{equation}
    \phi_2[\textbf{x},t;\CC^1]P(\CC^1)
    =     
    \frac{1}{N^{N-1}}
    \int_{\mathbb{R}^{3(N-1)}} 
    \phi_2[\textbf{x},t;\CC^N] P(\CC^{N-1}; \CC^1) 
    d\CC^{N-1}
\end{equation}

Injecting the expression of the PIF into the average procedure yields, 
\begin{multline*}
    \phi_2^N[\textbf{x},t;\CC^N] P(\textbf{x},\CC^N)
    = 
    \int 
    \prod_{i=1}^{N}
    \sum_{\alpha}^{N}\delta(\textbf{x}_i - \textbf{x}_\alpha(\FF,t)) 
    \sum_\beta
    H(a - |\textbf{x} -\textbf{x}_\beta(\FF,t)|)
    d\PP,\\
    \approx 
    \sum_i
    H(a - |\textbf{x} -\textbf{x}_i|)
\end{multline*}
In \citet{lundgren1972slow} they define directly, 
\begin{equation}
    \phi_2^N[\textbf{x},t;\CC^N]
    = \sum_i 
    H(a-|\textbf{x} - \textbf{x}_i|)
\end{equation}
This would imply that, 
\begin{equation*}
    H(a-|\textbf{x} - \textbf{x}_i|) P(\CC^N)
    = \int 
\prod_{i=1}^{N}
\sum_{\alpha}^{N}\delta(\textbf{x}_i - \textbf{x}_\alpha(\FF,t)) 
\sum_\beta
H(a - |\textbf{x} -\textbf{x}_\beta(\FF,t)|)
d\PP
\end{equation*}

It can then be re-averaged to yields the one particle average, 
\begin{equation}
    \phi_2[\textbf{x},t;\CC^1]P(\CC^1)
    =     
    \sum_\beta
    \frac{1}{N^{N-1}}
    \int_{\mathbb{R}^{3(N-1)}} 
    \int 
    \prod_{i=1}^{N}\sum_{\alpha}^{N}\delta(\textbf{x}_i - \textbf{x}_\alpha(\FF,t)) 
    H(a - |\textbf{x} -\textbf{x}_\beta(\FF,t)|)
    d\PP
    d\CC^{N-1}
\end{equation}
The domain of integration $\mathbb{R}^{3(N-1)}$ might be reduced due to the presence of the Heaviside function and the delta function. 
First, the delta function limit the integral to all point for which $\textbf{x}_i = \textbf{x}_\alpha(\FF,t)$ so that in the second sum the argument $\textbf{x}_\beta(\FF,t)$ can be replaced by $\textbf{x}_i$ since the crossed sum $\sum_\alpha\sum_\beta$ has good chance to reduce to $\sum_\alpha$ since it cancel out when $\alpha\neq\beta$. 
In fact it is easier to notice directly that the PIF is not a function of $\textbf{x}_2\ldots\textbf{x}_{N}$ thus it can be factor out of the integration, as well as the delta function of $\textbf{x}_1$ which end up with the basic definition, 
\begin{align*}
    \phi_2^1[\textbf{x},t;\CC^1]P(\CC^1)
    &=     
    \int 
    \sum_{\alpha}^{N}\delta(\textbf{x}_1 - \textbf{x}_\alpha(\FF,t)) 
    \sum_\beta
    H(a - |\textbf{x} -\textbf{x}_\beta(\FF,t)|)
    d\PP\\
    &\approx
    P(\CC^1) 
     H(a - |\textbf{x} - \textbf{x}_1|)
\end{align*}
We can extend this definition to ,
\begin{align*}
    f_2^1\phi_2^1[\textbf{x},t;\CC^1]P(\CC^1)
    &=     
    \int 
    \sum_{\alpha}^{N}\delta(\textbf{x}_1 - \textbf{x}_\alpha(\FF,t)) 
    f_2^0(\textbf{x},t;\FF)
    \sum_\beta 
    H(a - |\textbf{x} -\textbf{x}_\beta(\FF,t)|) 
    d\PP\\
    &
    \approx
    f^1_2[\textbf{x},t;\textbf{x}^1]
    P(\CC^1)  H(a - |\textbf{x} - \textbf{x}_j|)
\end{align*}
which is a direct consequence of the first formula. 

Finally, the ensemble average can be written,  
\begin{equation}
    f_2\phi_2[\textbf{x},t]
    =\frac{1}{N} 
    \int_{\mathbb{R}^3} 
    f^1_2\phi_2^1[\textbf{x},t;\CC^1]
    P(\CC^1)
    d\CC^1
    \approx
    \int_{|\textbf{x} - \textbf{x}^1| < a} 
    P(\CC^1)
    f_2^1[\textbf{x},t;\textbf{x}^1]
    d\CC^1
\end{equation}
The last approximation is done by considering that particles have the same contribution to the PDF. 

Therefore, the terms in the conditionally averaged equations reads, 
% \begin{equation*}
%     \grad_{\textbf{x}_1}\cdot \avg{\textbf{u}_\beta\delta_\beta f^0}
%     = \grad_{\textbf{x}_1}\cdot 
%     \int 
%     \sum_\beta \delta(\textbf{x}^1 - \textbf{x}_\alpha(\FF,t)) 
%     \textbf{u}_\beta(\FF,t)  f^0(\textbf{x},t;\FF) 
%     d\PP
% \end{equation*}
\begin{equation*}
    \div  \avg{\delta_\alpha \textbf f }
    = \div
    \int 
    \sum_\alpha \delta(\textbf{x}^1 - \textbf{x}_\alpha(\FF,t)) 
      \textbf f (\textbf{x},t;\FF) 
    d\PP
\end{equation*}
If \textbf{f} is a phase quantity we have, 
\begin{align*}
    \div  \avg{\delta_\alpha \textbf f \chi_2 }
    = \div
    \int 
    \sum_\alpha \delta(\textbf{x}^1 - \textbf{x}_\alpha(\FF,t)) 
      \textbf f \chi_2  (\textbf{x},t;\FF) 
    d\PP\\
    \approx \div
    \int_{|\textbf{x} - \textbf{x}^1| < a} 
    P(\CC^1)
    f_2^1[\textbf{x},t;\textbf{x}^1]
    d\CC^1
\end{align*}

\subsubsection*{The second definition of the PDF}
To exclude all product of the same Lagrangian particle such that pairs distribution concerns actual pairs of particles we must define the PDF as, 
\begin{equation*}
    P(\CC^N,t)
    = \int 
    \sum_{\alpha_1}^{N}\delta(\textbf{x}_1 - \textbf{x}_{\alpha_1}(\CC,t))
    \sum_{\alpha_2\neq\alpha_1}^{N}\delta(\textbf{x}_2 - \textbf{x}_{\alpha_2}(\CC,t))
    \ldots
    % \sum_{\alpha_N\neq\alpha_1\ldots\alpha_{N-1}}^{N}
    \delta(\textbf{x}_N - \textbf{x}_{\alpha_N}(\CC,t))
    d\PP,
\end{equation*}
This equation might be re-written, 
\begin{equation*}
    P(\CC^N,t)
    = \int 
    \prod_i^N
    \sum_{\alpha_i}^{N}\delta(\textbf{x}_i - \textbf{x}_{\alpha_i}(\FF,t))
    \prod_j^{i-1}\left[1 -  \delta(\alpha_i-\alpha_j)\right]
    d\PP,
\end{equation*}
Or alternatively, 
\begin{equation*}
    P(\CC^N,t)
    = \int 
    \prod_{i=1}^N
    \sum_{\alpha_i = i }^{N}\delta(\textbf{x}_i - \textbf{x}_{\alpha_i}(\FF,t))
    d\PP,
\end{equation*}

The terms in the product is equivalent to the term before except that it is equal to one only for $\prod_j^i  [1 - \delta(i-j)] \neq 0$ which happen if the particle $i\neq i$. 
This definition clearly gives a norm equal to $N!$. 
Thus, one might recover the definition of \citet{zhang1994averaged} by noticing that
\begin{equation*}
    P(\CC^N,t)
    = \frac{1}{N!}\int 
    \prod_i^N
    \sum_{\alpha_i}^{N}\delta(\textbf{x}_i - \textbf{x}_{\alpha_i}(\FF,t))
    \prod_j^{i-1}\left[1 -  \delta(\alpha_i-\alpha_j)\right]
    d\PP,
\end{equation*}
and 
\begin{equation*}
    P(\CC^{K},t)
    = \frac{1}{K!}\iint 
    \prod_i^N
    \sum_{\alpha_i}^{N}\delta(\textbf{x}_i - \textbf{x}_{\alpha_i}(\FF,t))
    \prod_j^{i-1}\left[1 -  \delta(\alpha_i-\alpha_j)\right]
    d\PP d\CC^{N - K},
\end{equation*}
we recover the previous def. 

Now it is interesting to investigate in what these statistical PDf are related to the more straightforward approach use by the new Zhang articles. 
\begin{equation}
    P(\textbf{x}_1,\textbf{x}_2)
    = \int 
    \sum_{\alpha_1}^N\delta(\textbf{x}_1 - \textbf{x}_{\alpha_1}(\CC,t))
    \sum_{\alpha_2\neq \alpha_1}^N\delta(\textbf{x}_2 - \textbf{x}_{\alpha_2}(\CC,t))
    d\PP
\end{equation}
This pdf has a norm of 
\begin{align*}
    \int_{\mathbb{R}^6} P(\textbf{x}_1,\textbf{x}_2) d\CC^2
    &= \int 
    \int_{\mathbb{R}^3}
    \sum_{\alpha_1}\delta(\textbf{x}_1 - \textbf{x}_{\alpha_1}(\CC,t))
    \int_{\mathbb{R}^3}
    \sum_{\alpha_2\neq \alpha_1}\delta(\textbf{x}_2 - \textbf{x}_{\alpha_2}(\CC,t))
    d\textbf{x}_2 d \textbf{x}_1 
    d\PP\\
    &= N(N-1)
\end{align*}
which makes sense. $P(\textbf{x}_1)$ is the number density so that $\int_{\mathbb{R}^3} P(\textbf{x}_1) d\textbf{x}_1$ is the number of particles. 
We now that,
\begin{equation*}
    \textbf{f}(\textbf{x,t})
    = \int_{\mathbb{R}}
    \textbf{f}(x,t;\CC^1) 
    P(\CC^1,t)
    d\CC^1
\end{equation*}
thus we might want to solve an equation for the conditional quantities, $\textbf{f}(x,t;\CC^1)$. 

For succinctness, we introduce the notation, 
\begin{equation*}
    N! P(\CC^N,t)
    =\int 
    \Pi(\CC^N, \CC,t)
    d\PP,
\end{equation*}
which represent the product of all delta function in a physical way. 

\subsubsection*{Conditional equaitons}

Following \citet{hinch1977averaged} consider the arbitrary single fluid formulation of the generic conservation law, 
\begin{equation*}
    \pddt f^0
    + \div 
    (f^0\textbf{u}^0
    - \bm\Phi^0)
    = s^0
\end{equation*}
We first multiply this equation by the Dirac delta function product 
\begin{multline*}
    \pddt \left[\Pi(\CC^N,\CC,t)f^0\right]
    + \div \left[\Pi(\CC^N,\CC,t)
    (f^0\textbf{u}^0
    - \bm\Phi^0)
    \right] \\
    +\sum_{j=1}^N \partial_{\textbf{x}_j}\cdot\left[
        \Pi(\CC^N,\CC,t) 
        \textbf{u}_\alpha(\CC,t)
        \textbf{f}^0(\textbf{x},\CC,t)
    \right]
    = \Pi(\CC^N,\CC,t) s^0
\end{multline*}
Then, we integrate on all flow configuration $d\PP$ yielding the $N^{th}$ particle conditional averaged equation, namely,
\begin{multline*}
    \pddt [f(\textbf{x},t;\CC^N)P(\CC^N,t)]
    + \div [
    (f^0\textbf{u}^0)(\textbf{x},t;\CC^N)P(\CC^N,t)
    - \bm\Phi(\textbf{x},t;\CC^N)P(\CC^N,t)
    ] \\
    +\sum_{j=1}^N \partial_{\textbf{x}_j}\cdot[
        (\textbf{u}_\alpha f^0)(\textbf{x},t;\CC^N)P(\CC^N)
    ]
    = s(\textbf{x},t,\CC^N) P(\CC^N)
\end{multline*}
This is the $N$ particle conditional averaged equation of conservation for the continuous local quantity $f^0$. 
It is now interesting to perform an average of the other particle configuration, the $N-1$ particles configuration, to be exact. 
Thus, let's apply the operation $\int \ldots d\CC^{N-1}$ on the previous equaiton, which gives,
\begin{multline*}
    \pddt [f(\textbf{x},t;\CC^1)P(\CC^1,t)]
    + \div [
    (f^0\textbf{u}^0)(\textbf{x},t;\CC^1)P(\CC^1,t)
    - \bm\Phi(\textbf{x},t;\CC^1)P(\CC^1,t)
    ] \\
    + \partial_{\textbf{x}_1}\cdot[
        (\textbf{u}_\alpha f^0)(\textbf{x},t;\CC^1)P(\CC^1)
    ]
    = s(\textbf{x},t,\CC^1) P(\CC^1)
\end{multline*} 
where the terms in $\partial_{\textbf{x}_j}$ vanished by the use of the divergence theorem and noticing that $\lim_{\textbf{x}_j\to\infty} P(\CC^N) =0$ for any $j$. 

We now derive the conditional average from the more straightforward approach (i.e. we multiply the equation by $\sum_\alpha^N \delta(\textbf{x}_1 - \textbf{x}_\alpha(\FF,t))$) which gives, 
\begin{equation*}
    \pddt (f^0\delta_\alpha)
    + \div 
    [f^0\textbf{u}^0\delta_\alpha
    - \bm\Phi^0\delta_\alpha]
    + \partial_{\textbf{x}_1}(\textbf{u}_\alpha f^0 \delta_\alpha)
    = s^0\delta_\alpha
\end{equation*}
And then we integrate over all $d\PP$, namely, 
\begin{multline*}
    \pddt (f(\textbf{x},t;\CC^1)P(\CC^1,t))
    + \div 
    [(f^0\textbf{u}^0)(\textbf{x},t;\CC^1)P(\CC^1,t)
    - \bm\Phi(\textbf{x},t;\CC^1)P(\CC^1,t)]\\
    + \partial_{\textbf{x}_1}((\textbf{u}_\alpha f^0)(\textbf{x},t;\CC^1)P(\CC^1,t))
    = s(\textbf{x},t;\CC^1)P(\CC^1,t)
\end{multline*}
This equation doesn't tells us much, actually. 
The relevant equation is in fact the two-fluid formulation without mass transfer; because we are looking for the conditional interphase term. 
\begin{equation*}
    \pddt (\chi_k f_k^0 \delta_{1\alpha})
    + \div (
        \chi_k f_k^0 \textbf{u}_k^0 \delta_{1\alpha}
        - \chi_k \mathbf{\Phi}_k^0  \delta_{1\alpha}
        )
    + \partial_{\textbf{x}_1}\cdot (\textbf{u}_\alpha f^0_k\chi_k\delta_\alpha)
    = 
    \chi_k s_k^0 \delta_{1\alpha}
    + \delta_I
         \mathbf{\Phi}_k^0
    \cdot \textbf{n}_k  \delta_{1\alpha},
\end{equation*}
Summing on all $\alpha$ and applying the ensemble average $\int\ldots d\PP$ gives, 
\begin{equation*}
    \pddt \avg{\chi_k f_k^0 \delta_{\alpha}}
    + \div 
        \avg{\chi_k f_k^0 \textbf{u}_k^0 \delta_{\alpha}
        - \chi_k \mathbf{\Phi}_k^0  \delta_{\alpha}}
    + \partial_{\textbf{x}_1}\cdot \avg{\textbf{u}_\alpha f^0_k\chi_k\delta_\alpha}
    = 
    \avg{\chi_k s_k^0 \delta_{\alpha}}
    + \avg{\delta_I
         \mathbf{\Phi}_k^0
    \cdot \textbf{n}_k  \delta_{\alpha}},
\end{equation*}
The conditional mass equation then reads as,  
\begin{equation*}
    \pddt \avg{\chi_k \rho_k \delta_{\alpha}}
    + \div 
        \avg{\chi_k \rho_k \textbf{u}_k^0 \delta_{\alpha}}
    + \partial_{\textbf{x}_1}\cdot \avg{\textbf{u}_\alpha \rho_k \chi_k\delta_\alpha}
    = 
    0,
\end{equation*}
\begin{equation*}
    \pddt \avg{\chi_k \textbf{u}_k^0 \delta_{\alpha}}
    + \div 
        \avg{\chi_k \textbf{u}_k^0 \textbf{u}_k^0 \delta_{\alpha}
        - \chi_k \bm\sigma_k^0  \delta_{\alpha}}
    + \partial_{\textbf{x}_1}\cdot \avg{\textbf{u}_\alpha f^0_k\chi_k\delta_\alpha}
    = 
    \avg{\chi_k s_k^0 \delta_{\alpha}}
    + \avg{\delta_I
         \mathbf{\Phi}_k^0
    \cdot \textbf{n}_k  \delta_{\alpha}},
\end{equation*}
or, 
\begin{equation*}
    \pddt [\phi_k^1(\textbf{x},\textbf{x}^1,t)]
    + \div[\phi_k^1(\textbf{x},\textbf{x}^1,t) \textbf{u}_k^1(\textbf{x},t;\textbf{x}^1)]
    + \partial_{\textbf{x}_1}\cdot \avg{\textbf{u}_p\phi_k^1}
    = 
    0
\end{equation*}



\subsubsection{DEcomposition of the Reynolds stressss}


Equally, it is possible to go one step further considering the two particle average, 
\begin{align*}
    f[\textbf{x},t]
    &= \int_{\mathbb{R}^3}
    f^2[\textbf{x},t;\CC^2]
    P(\CC^2)
    \frac{d\CC^2}{N(N-1)}
\end{align*}
with, 
\begin{equation*}
    f^1[\textbf{x},t;\CC^2] P(\CC^2)
    % = \avg{\delta_\alpha \delta_\beta f^0}[\textbf{x},t;\CC^2] 
    = 
    \int 
    \sum_\alpha^N \delta(\textbf{x} - \textbf{x}_\alpha(\FF,t))
    \sum_{\beta\neq\alpha}^N \delta(\textbf{x}^1 - \textbf{x}_\beta(\FF,t))
    f^0[\textbf{x},t;\FF]
    d\PP
\end{equation*}
with $P(\CC^2)$ the pair probability density function. 

For the Reynolds stress it is trickier since it is a product of velocity. 
Indeed, we have,
\begin{align*}
    \avg{\chi_1 \textbf{u}_1'\textbf{u}_1'}[\textbf{x},t]
    =
    \int \avg{\chi_k \delta_\alpha  \textbf{u}_1'\textbf{u}_1'}[\textbf{x},t;\textbf{x}^1]
    \frac{d\textbf{x}^1}{N}
\end{align*}
Then, how to relate the quadratic quantity : $\avg{\chi_k \delta_\alpha  \textbf{u}_1'\textbf{u}_1'}$ to the conditional velocity averaged fields $\textbf{u}^1_1[\textbf{x},t;\textbf{x}^1]$  that we might be able to compute ? 
For that we may define fluctuating conditional quantity such that : 
\begin{align*}
    \textbf{u}_1^{1'}[\textbf{x},t,\FF]
    &= 
    \textbf{u}_1^0[\textbf{x},t,\FF]
    - \textbf{u}_1^1[\textbf{x},t,\textbf{x}^1]\\
    &= 
    \textbf{u}_1^0[\textbf{x},t,\FF]
    - \textbf{u}_1[\textbf{x},t]
    + \textbf{u}_1[\textbf{x},t]
    - \textbf{u}_1^1[\textbf{x},t,\textbf{x}^1]\\
    &= 
    \textbf{u}_1'[\textbf{x},t,\FF]
    - (\textbf{u}_1^1-\textbf{u}_1)[\textbf{x},t,\textbf{x}^1]
\end{align*}
Or that, 
\begin{equation*}
    (\textbf{u}_1^1-\textbf{u}_1)[\textbf{x},t,\textbf{x}^1]
    = 
    \textbf{u}_1'[\textbf{x},t,\FF]
     -\textbf{u}_1^{1'}[\textbf{x},t,\FF]
\end{equation*}
where the left hands side term vanish under the operation, $\avg{\delta_\alpha \chi_1 \textbf{u}_1^{1'}} = 0$.
It represents the velocity that changes with respect to the conditionally averaged velocity. 
Which means that the Reynolds stress might in fact be written, 
\begin{align*}
    \avg{\chi_1 \textbf{u}_1'\textbf{u}_1'}[\textbf{x},t]
    + \textbf{u}_1\textbf{u}_1
    \phi_1[\textbf{x},t]
    =
    \int_{\mathbb{R}^3}
    \avg{\delta_\alpha 
    \textbf{u}_1^{1'}
    \textbf{u}_1^{1'}
    \chi_k} \frac{d\textbf{x}^1}{N}
    +
    \int_{\mathbb{R}^3} 
    \textbf{u}_1^1\textbf{u}_1^1
    \phi_1^1
    \frac{d\textbf{x}^1}{N}
\end{align*}
In the dilute regime it is known that for a droplet in stokes flow in translation it is not computable. 
Indeed, the velocity fields decay as $r^{-1}$ it is therefore not integrable. 
But know it is clear that the second contribution must correct that. 

To solve for the conditional averaged fields $\textbf{u}^1_k$, $p^1_k$ we must derive the conditional averaged mass and momentum equations. 
To do so one needs to apply the operator $\avg{\delta_\beta \chi_k \ldots}$ defined in\ref{eq:avg_cond_1} on the local scale equations, which gives for the mass and momentum single fluid formulation equation,
\begin{equation*}
    \pddt  0
\end{equation*}
\begin{equation*}
    \pddt (\rho \textbf{u})
\end{equation*}
\begin{equation*}
    \pddt (\rho_1\phi_k^1 )
    + \div 
        (\rho_k\phi_k^1 \textbf{u}_k^1)
    + \nabla_{\textbf{x}_k}\cdot \avg{\rho_k \textbf{u}_\beta \chi_k\delta_\beta}
    = 0
\end{equation*}
\begin{equation*}
    \pddt (\rho_k\phi_k^1 \textbf{u}_k^1)
    + \div [
        \rho_k\phi_k^1 \textbf{u}_k^1\textbf{u}_k^1
        - \phi_k^1\bm\sigma_k^1
        + \bm\sigma^{1,Re}_k
    ]
    + \nabla_{\textbf{x}_k}\cdot \avg{\rho_k \textbf{u}_\beta \textbf{u}_k^0 \chi_k\delta_\beta}
    = 
    \phi_k^1 \rho_k \textbf{g}
    + \avg{\delta_{\beta} \delta_I
         \bm\sigma_k^0
    \cdot \textbf{n}_k  },
\end{equation*}
These equations are exact 
The last term of this expression might be written, 
\begin{equation*}
    \avg{\delta_{\alpha} \delta_I
         \bm\sigma_k^0
    \cdot \textbf{n}_k  }
    = 
    \avg{\delta_{\beta} 
    \delta_\alpha
    \intS{\bm\sigma_k^0
   \cdot \textbf{n}_k }}
    - \div \avg{\delta_{\beta} 
    \delta_\alpha
    \intS{\textbf{r}\bm\sigma_k^0
   \cdot \textbf{n}_k }}
   + \ldots
\end{equation*}
where the first term $\avg{\delta_{\beta} 
\delta_\alpha
\intS{\bm\sigma_k^0
\cdot \textbf{n}_k }}[\textbf{x},t;\textbf{x}^1]$ is the mean particle force at $\textbf{x}$ knowing there is a particle $\beta$ in $\textbf{x}^1$. 
Note that this time $\alpha$ and $\beta$ can represent same indices, and it is in fact normal since $\avg{\delta_{\beta} 
\delta_\alpha
\intS{\bm\sigma_k^0
\cdot \textbf{n}_k }}[\textbf{x},t;\textbf{x}]$
must be equals to the drag for $\textbf{x}=\textbf{x}_1$. 
The last term on the left-hands side represent the convection term on $\textbf{x}_1$ along the averaged particle velocity at $\textbf{x}_1$. 

These equations are exact, and one can recover the original averaged equations by integrating these over $\int_{\mathbb{R}^3}\ldots d\textbf{x}_1$. 
Nevertheless, to solve this problem one need made some hypothesis to brings us back to a solvable problem. 

\tb{
    We must show how particle average is related to something like 
    \begin{equation*}
        \int_V P(x) s(s|x^1)
    \end{equation*}
    In such case the divergence transform into first moments etc. 
    We would like to be in a stokes problem of course. solving for the relative motion between the drops and fluid...
    }
\begin{equation}
    \pddt f^0[\textbf{x},t;\FF]
    + \div(f^0\textbf{u}^0[\textbf{x},t;\FF]
    - \bm\Phi[\textbf{x},t;\FF])
    = s^0[\textbf{x},t;\FF]
\end{equation}
If i take this equation time minus the particle transport equation, 
\begin{equation*}
    \pddt (\delta_\beta  q_\beta^\text{tot}[\textbf{x}_1,t;\FF])
    + \grad_{\textbf{x}_1} \cdot (\delta_\beta\textbf{u}_\beta q_\beta^\text{tot}[\textbf{x}_1,t;\FF])
    = \delta_\beta\intO{s_2^0}[\textbf{x}_1,t;\FF]
    + \delta_\beta\intS{\mathbf{\Phi}_1^0 \cdot \textbf{n}_2}[\textbf{x}_1,t;\FF]
\end{equation*}
If we want to solve the NS equaiton knowing a particle is present at $\textbf{x}_1$ for the relative property $f^0 -  q_\beta/v_\alpha$ we first need to multiply the one fluid form with $\delta_\beta$,  
\begin{equation}
    \pddt ((f^0 - q_\beta^\text{tot}/v_\alpha)\delta_\beta  )
    + \div(f^0\textbf{u}^0\delta_\beta
    - \bm\Phi^0 \delta_\beta)
    + \partial_{\textbf{x}_1}\cdot(\textbf{u}_\beta \delta_\beta (f^0 - q_\alpha/v_\alpha))
    = (s^0 - \intO{s_2^0}/v_\alpha)\delta_\beta 
    - \delta_\beta\intS{\mathbf{\Phi}_1^0 \cdot \textbf{n}_2}/v_\alpha
\end{equation}
applying to the mass and momentum equation, 
\begin{equation}
    \pddt ((\rho  - \rho_2)\delta_\beta  )
    + \div(\rho \textbf{u}^0\delta_\beta
    - \bm\sigma^0 \delta_\beta)
    + \partial_{\textbf{x}_1}\cdot(\textbf{u}_\beta \delta_\beta (\rho - \rho_2))
    = 0
\end{equation}
\begin{equation}
    \pddt ((\rho \textbf{u}^0 - \textbf{u}_\beta\rho_2)\delta_\beta  )
    + \div(\rho \textbf{u}^0\textbf{u}^0\delta_\beta
    - \bm\sigma^0 \delta_\beta)
    + \partial_{\textbf{x}_1}\cdot(\textbf{u}_\beta \delta_\beta (\rho \textbf{u}^0 - \textbf{u}_\beta\rho_2))
    = (\rho - \rho_2)\textbf{g}\delta_\beta 
    + \delta_\beta \intS{\bm\sigma_1^0 \cdot \textbf{n}_2}/v_\alpha
\end{equation}
These equation clearly indicate the problem to solve to determine the drag force term, present on the RHS of the equation. 
Namely, the resultant of the drag force is the force intensity which balance the relative inertial forces described on the left-hand side of the relative momentum equation.
In this way the closure problem is linked to a single point of force problem. 
If we assume $U\sim (\rho \textbf{u}^0 - \textbf{u}_\beta\rho_2)$ the velocity scale between both phases, it is easy to show that 

Another way is to consider the particle conditionally average phase equation for both phase, 
\begin{equation*}
    \pddt (\chi_1 f_1^0 \delta_{\beta})
    + \div (
        \chi_1 f_1^0 \textbf{u}_1^0 \delta_{\beta}
        - \chi_1 \mathbf{\Phi}_1^0  \delta_{\beta}
        )
    + \grad_{\textbf{x}_1}\cdot (\textbf{u}_\beta f^0_1\chi_1\delta_\beta)
    = 
    \chi_1 s_1^0 \delta_{\beta}
    - 
    \delta_I
    \mathbf{\Phi}_1^0
    \cdot \textbf{n}_2  \delta_{\beta},
\end{equation*}
\begin{equation*}
    \pddt (\chi_2 f_2^0 \delta_{\beta})
    + \div (
        \chi_2 f_2^0 \textbf{u}_2^0 \delta_{\beta}
        - \chi_2 \mathbf{\Phi}_2^0  \delta_{\beta}
        )
    + \grad_{\textbf{x}_1}\cdot (\textbf{u}_\beta f^0_2\chi_2\delta_\beta)
    = 
    \chi_2 s_2^0 \delta_{\beta}
    + 
    \delta_I
    \mathbf{\Phi}_2^0
    \cdot \textbf{n}_2  \delta_{\beta},
\end{equation*}
Upon subtracting the first equation with the second one obtain, 
\begin{multline*}
    \pddt ((\chi_1 f_1^0 -\chi_2 f_2^0) \delta_{\beta})
    + \div (
        (\chi_1 f_1^0\textbf{u}_1^0  -\chi_2 f_2^0\textbf{u}_2^0 ) \delta_{\beta}
        - (\chi_1 \mathbf{\Phi}_1^0-\chi_2 \mathbf{\Phi}_2^0)  \delta_{\beta}
        )\\
    + \grad_{\textbf{x}_1}\cdot (\textbf{u}_\beta (f^0_1\chi_1 - f^0_1\chi_2)\delta_\alpha)
    = 
    (\chi_1 s_1^0 - \chi_2 s_2^0) \delta_{\beta}
    - 2
    \delta_I
    \mathbf{\Phi}_1^0
    \cdot \textbf{n}_2  \delta_{\beta},
\end{multline*}
It gives for the mass and momentum, 
\begin{multline*}
    \pddt ((\chi_1 \textbf{u}_1^0\rho_1 -\chi_2 \textbf{u}_2^0 \rho_2) \delta_{\beta})
    + \div (
        (\chi_1 \textbf{u}_1^0 \textbf{u}_1^0\rho_1 
        - \chi_2 \textbf{u}_2^0 \textbf{u}_2^0\rho_2) \delta_{\beta}
        - (\chi_2\bm{\sigma}_1^0  -\chi_2\bm{\sigma}_2^0  )\delta_{\beta}
        )
    + \grad_{\textbf{x}_1}\cdot (\textbf{u}_\beta (f^0_1\chi_1 - f^0_1\chi_2)\delta_\alpha)
    = \\
    (\chi_1 s_1^0 - \chi_2 s_2^0) \delta_{\beta}
    - 2
    \delta_I
    \bm{\sigma}_1^0
    \cdot \textbf{n}_2  \delta_{\beta},
\end{multline*}



\subsection*{The closure problem}

In this section we seek for the closure of the momentum balance defined by \ref{eq:dt_hybrid_rhou_1}.
The exhaustive list of terms that must be closed is, 
\begin{align}
    \pSavg{\bm{\sigma}_1^0\cdot \textbf{n}_2}[\textbf{x},t],\\
    \pSavg{\textbf{rr}\bm{\sigma}_1^0\cdot \textbf{n}_2}[\textbf{x},t],\\
    \pOavg{\textbf{e}_2^0}[\textbf{x},t],\\
    \pOavg{\textbf{re}_2^0}[\textbf{x},t],\\
    \rho_1 \avg{\chi_1 \textbf{u}_1'\textbf{u}_1'}[\textbf{x},t].
    \label{eq:closures}
\end{align}
These are all ensemble averaged quantities. 
To introduce the closure problem we follow the procedure initiated by \citet{hinch1977averaged} and generalized in \citet[Appendix A]{zhang1994ensemble}. 
To do so, we must introduce the one particle conditional averaged definition. 
For example, we may write the average of an arbitrary local quantity $f^0(\textbf{x},t;\FF)$ such as, 
\begin{equation}
    f[\textbf{x},t]
    = \int_{\mathbb{R}^3}
    f^1[\textbf{x},t;\textbf{x}^1]
    P(\textbf{x}^1) d\textbf{x}^1
    \label{eq:avg_cond_1}
\end{equation}
with, 
\begin{equation*}
    f^1[\textbf{x},t;\textbf{x}^1] P(\textbf{x}^1)
    =
    \frac{1}{N} 
    \avg{\delta_\alpha f^0}[\textbf{x},t;\textbf{x}^1]
    = 
    \frac{1}{N}
    \int 
    \sum_\alpha^N \delta(\textbf{x}^1 - \textbf{x}_\alpha(\FF,t))
    f^0[\textbf{x},t;\FF]
    d\PP
\end{equation*}
Where the superscript $^1$ on $f^1 = \avg{\delta_\alpha f^0}[\textbf{x},t;\textbf{x}^1]$ indicate that $f^1$ is the value of $f^1$ at \textbf{x} conditionally on the presence of one particle center of mass at $\textbf{x}^1$ averaged over all the flow realization $\FF$. 

Our original objective was to seek closure for the term of the form $\pSavg{\ldots}$, $\pOavg{\ldots}$ and $\avg{\chi_1 \ldots}$.
Under this form it is difficult to make any physical interpretation and hypotheses, therefore to bring the problem into a more convenient form one might seek an expression of these closures, in terms of the one particle conditionally averaged quantities.
Indeed, as explained by \ref{eq:avg_cond_1} the space average of this sub averaged quantities will give us directly, the ensemble averaged quantity. 
The fluid phase quantity might be expressed directly through the use of $\ref{eq:avg_cond_1}$. 

At this point it is instructive to re demonstrate some formulas regarding conditional average. 
To aim will be to link ensemble average with surface or volume integrals of conditionally averaged quantities. 
To begin with we will consider only spherical particle of radius $a$ to demonstrate how to recover some relations already derived in the literature. 
In this particular case the phase indicator function can be written, $\chi_2[\textbf{x},t;\FF] = \sum_\alpha H(a - |\textbf{x} - \textbf{x}_\alpha|)$
We first consider the continuous phase average of an arbitrary quantity $f_2^0[\textbf{x},t;\FF]$. 
It yields, 
\begin{align}
    \avg{f_2^0 \chi_2}[\textbf{x},t;\FF]
    &= \int f_2^0 \sum_\alpha  H(a - |\textbf{x} - \textbf{x}_\alpha|) d\PP\\
    &= \int _{\mathbb{R}^3} \int f_2^0 \sum_\alpha  \delta(\textbf{x}^1 - \textbf{x}_\alpha(t;\FF)) H(a - |\textbf{x} - \textbf{x}^1|) d\PP d\textbf{x}^1\\
    &= \int _{|\textbf{x} - \textbf{x}^1| < a} \avg{ \delta_\alpha f_2^0}[\textbf{x},t;\textbf{x}^1] d\textbf{x}^1
\end{align} 
Therefore, in this case we integrate over all position pf the conditionally averaged particle. 
Regarding the particles averaged quantity we must perform a sightly different operation since they are in principle already particle conditionally averaged quantities.
To introduce the procedure we first consider spherical particle of radius $a$. 
\begin{align}
    \pSavg{\bm\sigma_1^0\cdot\textbf{n}_2}[\textbf{x}^1,t]
    &= \int \sum_{\alpha=1}^N \delta(\textbf{x}^1-\textbf{x}_\alpha(t; \FF))
    \int_{\Sigma_\alpha(t; \FF)}
    \bm\sigma_1^0\cdot\textbf{n}_2(\textbf{x},t;\FF)
    d\Sigma(\textbf{x}) d\PP\\
    &= 
    \int_{\mathbb{R}^3}
    \int
     \sum_{\alpha=1}^N \delta(\textbf{x}^1-\textbf{x}_\alpha(t; \FF))
    \delta(|\textbf{x} - \textbf{x}_{\alpha}(t;\FF)|-a)
    (\bm\sigma_1^0\cdot\textbf{n}_2)[\textbf{x},t;\FF]
    d\PP
    d\textbf{x}\\
    % &= 
    % \int_{\mathbb{R}^3}
    % \int
    %  \sum_{\alpha=1}^N \delta(\textbf{x}^1-\textbf{x}_\alpha(t; \FF))
    % \delta(|\textbf{x} - \textbf{x}_{\alpha}(t;\FF)|-a)
    % (\bm\sigma_1^0\cdot\textbf{n}_2)[\textbf{x},t;\FF]
    % d\PP
    % d\textbf{x}\\
    &=
    \int_{|\textbf{x}-\textbf{x}^1|=a}
    \avg{\delta_\alpha  \bm\sigma_1^0\cdot \textbf{n}_2}
    [\textbf{x},t;\textbf{x}^1]
    d\textbf{x}\\
    &=
    \int_{|\textbf{x}-\textbf{x}^1|=a}
    \bm\sigma_1^1\cdot \textbf{n}_2
    [\textbf{x},t;\textbf{x}^1]
    P(\textbf{x}^1)
    d\textbf{x}
    \label{eq:conditional_sphere}
\end{align}
To go from the second to third equality we considered that since $\delta(|\textbf{x} - \textbf{x}_{\alpha}(t;\FF)|-a)$ does not depend on other particles parameters, it could be switched in the integration to $\delta(|\textbf{x} - \textbf{x}^1|-a)$ which afterward reduced the domain of integration. 
As it will be presented this kind of simplification does not arise when the particles have different shape and parameters.  
The ensemble averaged term $\avg{\delta_\alpha \bm\sigma_2^0\cdot \textbf{n}_2}
[\textbf{x},t;\textbf{x}^1]$, is clearly the ensemble average of the surface traction force $(\bm\sigma_2^0\cdot \textbf{n}_2)[\textbf{x},t;\FF]$ evaluated at $\textbf{x}$ knowing there is a particle at $\textbf{x}^1$.
This formulation agree with the definitions given by \citet{hinch1977averaged} and \citet{zhang1994averaged}. 
Through this manipulation we reformulated an ensemble averaged term to an integral over the surface of a particle. 

As mentioned in \citet{hinch1977averaged}, if the particle surface or volume is particle dependent relation like \ref{eq:conditional_sphere} is not sufficient. 
In the continuity of the previous example we now consider spheroidal particles described by the Lagrangian conformation tensor $\textbf{C}_\alpha(t;\FF)$. 
Now, the geometry of the particles is described in space by their position $\textbf{x}_\alpha(t,\FF)$
and shape tensor, $\textbf{C}_\alpha(t,\FF)$, besides we recall that the equation for the surface of the particle $\alpha$ is described by the equation $\textbf{rr}:\textbf{C}^{-1}_\alpha - a^2 =0$. 
The internal coordinate of the properties $\textbf{C}_\alpha$ will be noted $\textbf{C}$ in the phase space, similarly as $\textbf{x}$ is liked to $\textbf{x}_\alpha$ in the Eularian frame.  
Using these properties, the ensemble average of the surface traction term might be reformulated as, 
\begin{align}
    &\pSavg{\bm\sigma_1^0\cdot\textbf{n}_2}[\textbf{x}^1,t]
    = \int \sum_{\alpha=1}^N \delta(\textbf{x}^1-\textbf{x}_\alpha(t; \FF))
    \int_{\Sigma_\alpha(t; \FF)}
    \bm\sigma_1^0\cdot\textbf{n}_2(\textbf{x},t;\FF)
    d\Sigma(\textbf{x}) d\PP\nonumber\\
    &= 
    \int_{\mathbb{R}^3}
    \int
     \sum_{\alpha=1}^N \delta(\textbf{x}^1-\textbf{x}_\alpha(t; \FF))
    \delta(\textbf{rr}: \textbf{C}_\alpha^{-1}(t,\FF) -a^2)
    (\bm\sigma_1^0\cdot\textbf{n}_2)[\textbf{x},t;\FF]
    d\PP
    d\textbf{x}\nonumber\\
    &= 
    \int_{\mathbb{R}^9}
    \int_{\mathbb{R}^3}
    \int
     \sum_{\alpha=1}^N 
     \delta(\textbf{x}^1-\textbf{x}_\alpha(t; \FF))
     \delta(\textbf{C}^1-\textbf{C}_\alpha(t; \FF))
     \delta(\textbf{rr}: (\textbf{C}^1)^{-1} -a^2)
    (\bm\sigma_1^0\cdot\textbf{n}_2)[\textbf{x},t;\FF]
    d\PP
    d\textbf{x}
    d\textbf{C}^1
    \nonumber\\
    % &= 
    % \int_{\mathbb{R}^3}
    % \int
    %  \sum_{\alpha=1}^N \delta(\textbf{x}^1-\textbf{x}_\alpha(t; \FF))
    % \delta(|\textbf{x} - \textbf{x}_{\alpha}(t;\FF)|-a)
    % (\bm\sigma_1^0\cdot\textbf{n}_2)[\textbf{x},t;\FF]
    % d\PP
    % d\textbf{x}\\
    &=
    \int_{\mathbb{R}^9}
    \int_{\textbf{rr}: (\textbf{C}^1)^{-1} = a^2}
    \avg{\delta_\alpha \delta_\textbf{C}  \bm\sigma_1^0\cdot \textbf{n}_2}
    [\textbf{x},t;\textbf{x}^1,\textbf{C}^1]
    d\textbf{x}d\textbf{C}^1
    \label{eq:conditional_spheroid}
\end{align}
where $\textbf{r} = \textbf{x} - \textbf{x}_{\alpha}(t;\FF)$ or $\textbf{x} - \textbf{x}^1$ due to the presence of $\delta(\textbf{x} - \textbf{x}^1)$. 
On the second line $\delta(\textbf{rr}: \textbf{C}_\alpha^{-1}(t,\FF) -a^2)$ permits us to extend the particle dependent domain of integration to $\mathbb{R}^3$. 
Finally, we introduced the notation, $\delta_\textbf{C} = \delta(\textbf{C}^1 - \textbf{C}_\alpha(t,\FF))$. 
We therefore reduced our ensemble average to a space average over the surface of a spheroid, and an integral over all $\textbf{C}$ meaning over all shape possible.
However, we now need to close the term $\avg{\delta_\alpha \delta_\textbf{C} \bm\sigma_2^0\cdot \textbf{n}_2}
[\textbf{x},t;\textbf{x}^1,\textbf{C}^1]$, which can be written alternatively $\avg{\delta_\alpha \delta_\textbf{C} \bm\sigma_2^0\cdot \textbf{n}_2}
[\textbf{x},t;\textbf{x}^1] = \avg{\bm\sigma_2^0\cdot \textbf{n}_2}^{1\textbf{C}}
[\textbf{x},t,\textbf{x}^1,\textbf{C}^1]$.
The first term can be thought as the averaged surface force evaluated at $\textbf{x}$ on the surface of a particle centered at $\textbf{x}^1$ with its shape described by $\textbf{C}$.
The second term is the probability density function $P_1(\textbf{x}^1,\textbf{C}^1)$ such that $P_1(\textbf{x}^1,\textbf{C}^1)d\textbf{x}d\textbf{C}$ si the probable number of particle center of mass at $\textbf{x}^1$ having a conformation tensor $\textbf{C}$. 

All the closure exposed above can therefore be expressed as volume or surface integral of the conditionally averaged quantities.
This means that instead of seeking for expression for the terms \ref{eq:closures}, one might look for the conditionally averaged fields, $\bm\sigma^1_1$, $\textbf{u}^1$, $p^1$ \ldots from which it is possible to reconstruct the ensemble averaged quantity through \ref{eq:avg_cond_1},\ref{eq:conditional_spheroid} and other relations. 
Indeed, notice that $\avg{\delta_\alpha \bm\sigma_1^0\cdot \textbf{n}_2}
= \avg{\delta_\alpha  (-p_2^0 \textbf{I} + \mu_1 (\grad \textbf{u}_2^0+\grad \textbf{u}_2^0))\cdot \textbf{n}_2}
= - n_p p_1^1 \textbf{I} + n_p \mu_1 (\grad \textbf{u}_1^1 + \grad \textbf{u}_1^1)\cdot \textbf{n}_2$ since the Dirac delta function is function of $\textbf{x}^1$ and not \textbf{x}. 
Consequently, from the conditional velocity and pressure fields of the fluid phase one might compute the stress. 

To find these conditional fields in a rigorous manner it is common to reduce the problem to a single particle conditional averaged equations \citep{hinch1977averaged}. 
To derive such an equation we multiply the single fluid formulation of the local conservation law \ref{eq:dt_f} by $\delta(\textbf{x}_1 -\textbf{x}_\beta)$ and apply the ensemble average operator, which yields
\begin{equation}
    \pddt (f^1n_p)
    + \div(
        \avg{f^0\textbf{u}^0\delta_\beta}
    - \bm\Phi^1n_p)
    + \grad_1 \cdot
        \avg{\textbf{u}_\beta\delta_\beta f^0}
    = s^1n_p
\end{equation}
for a generic conservation law. 
The mass and momentum conservation law thus read, 
\begin{equation}
    \pddt (\rho^1n_p)
    + \div \avg{\rho^0\textbf{u}^0\delta_\beta}
    + \grad_1 \cdot
    \avg{\textbf{u}_\beta\delta_\beta \rho^0}
    = 0
\end{equation}
\begin{equation}
    \pddt \avg{\delta_\beta \rho^0 \textbf{u}^0}
    + \div(
        \avg{\delta_\beta \rho^0 \textbf{u}^0 \textbf{u}^0 }
    - \avg{\delta_\beta\bm\sigma^0})
    + \grad_1 \cdot
        \avg{\textbf{u}_\beta\delta_\beta \rho^0 \textbf{u}^0}
    = \textbf{g} n_p \rho^1
\end{equation}
At the local scale the flow is not steady, every particle move, however at the averaged scale the flow might be considered steady since it is an average. 
Similarly, at the particle scale for $\textbf{x} - \textbf{x}^1 = \mathcal{O}(a)$ we consider that the inertial effect are negligible. 
Therefore, the conditional averaged momentum equations are reduced to, 
\begin{equation}
    \div \avg{\rho^0\textbf{u}^0\delta_\beta}
    = 0
\end{equation}
\begin{equation}
    \div(
    \avg{\delta_\beta\bm\sigma^0})
    + \textbf{g} n_p \rho^1
    = 0 
\end{equation}



In the two-fluid formulation, 
\begin{equation}
    \pddt (n_p \phi_k^1)
    +  \div (n_p\phi_k^1\textbf{u}_k^1)
    +  \grad_1 \cdot
    (n_p \textbf{u}_p\phi_k^1)
    = 0
\end{equation}
\begin{equation}
    \pddt \avg{\delta_\beta \rho_k \chi_k \textbf{u}_k^0 }
    + \div(
        \avg{\delta_\beta \rho_k \chi_k \textbf{u}_k^0 \textbf{u}^0_k  }
    - \avg{\delta_\beta\chi_k\bm\sigma^0_k})
    + \grad_1 \cdot
        \avg{\textbf{u}_\beta\delta_\beta \rho_k \chi_k \textbf{u}_k^0 \textbf{u}^0}
    = n_p  \rho_k \phi_k^1 \textbf{g} 
    + \avg{\delta_\beta\delta_I \bm\sigma_k^0 \cdot \textbf{n}_k}
\end{equation}
Notice that the number density is function of \textbf{y}. 
The sources term can be written, 
\begin{equation*}
    \avg{\delta_I\delta_\beta \bm\sigma_k^0 \cdot \textbf{n}_2}[\textbf{x},t;\textbf{x}^1]
    = \int_{|\textbf{x}-\textbf{x}^2| = a}
    \avg{\delta_\alpha\delta_\beta \bm\sigma_k^0 \cdot \textbf{n}_2}[\textbf{x},t;\textbf{x}^1,\textbf{x}^2] d\textbf{x}^2
\end{equation*}
Therefore solving for the one particle conditionally averaged Navier-stokes equation, needs a closure of the form of two-particles conditionally average. 
This process might be carried forever. 
\subsubsection*{Stokes equaiton}
The conditional phase average of the carrier fluid stress, $\avg{\delta_\beta\chi_2 \bm\sigma_2^0}$ might be written,
\begin{align*}
    \avg{\delta_\beta\chi_1 \bm\sigma_1^0}
    &= 
    \avg{\delta_\beta (-p_1^0 + \mu_1 \textbf{e}^0 )}
    - \avg{\delta_\beta\chi_2 \bm\sigma_1^0}\\
    &= 
    - n_p p_1^1 \textbf{I}
    + n_p \mu_1 (\grad \textbf{u}_1^1 + \grad \textbf{u}_1^1)
    - \avg{\delta_\beta\chi_2 \bm\sigma_1^0}[\textbf{x},t;\textbf{x}^1]\\
\end{align*}
where the second term is the contribution from the particles to the conditional stress. 
Note that it appear under the divergence sign in the Stokes, equation consequently it can be reformulated as\citet{hinch1977averaged}, 
\begin{equation*}
    \div \avg{\delta_\beta\chi_2 \bm\sigma_1^0}[\textbf{x},t;\textbf{x}^1]
    =\div\int_{|\textbf{x}-\textbf{x}^2| < a}
    \avg{\delta_\alpha\delta_\beta \bm\sigma_1^0}[\textbf{x},t;\textbf{x}^1,\textbf{x}^2] d\textbf{x}^2
\end{equation*}
noticing that, 
\begin{equation*}
    \avg{\delta_\alpha\delta_\beta \bm\sigma_1^0}[\textbf{x},t;\textbf{x}^1,\textbf{x}^2]
    = \int_{|\textbf{x}'-\textbf{x}^2|<a}
    \avg{\delta_\alpha\delta_\beta \bm\sigma_1^0}[\textbf{x}',t;\textbf{x}^1,\textbf{x}^2]
    \delta(\textbf{x}' - \textbf{x})
    d\textbf{x}'
\end{equation*}
we can re-write the first equality as, 
\begin{equation*}
    \div \avg{\delta_\beta\chi_2 \bm\sigma_1^0}[\textbf{x},t;\textbf{x}^1]
    =\div
    \int_{|\textbf{x}'-\textbf{x}^2|<a}
    \int_{|\textbf{x}'-\textbf{x}^2|<a}
    \avg{\delta_\alpha\delta_\beta \bm\sigma_1^0}[\textbf{x}',t;\textbf{x}^1,\textbf{x}^2]
    \delta(\textbf{x}' - \textbf{x})
    d\textbf{x}'
    d\textbf{x}^2
\end{equation*}
The stokes equation in a steady state non-inertial situation reads, 
\begin{equation}
    \div (\phi_k^1\textbf{u}_k^1)
    = 0
\end{equation}
\begin{equation}
    - n_p \grad (p_1^1)
    + \mu_1n_p \div (\grad \textbf{u}_1^1 + \grad \textbf{u}_1^1)
    + n_p  \rho_k \phi_k^1 \textbf{g} 
    + \avg{\delta_I\delta_\beta \bm\sigma_k^0 \cdot \textbf{n}_k}
    = 0 
\end{equation}
with the boundary at large distance from a particle, 
\begin{align*}
    \lim_{|\textbf{x}^1 - \textbf{x}| \to \infty} \phi_1^k[\textbf{x},t;\textbf{x}^1] = \phi_k[\textbf{x},t]\\
    \lim_{|\textbf{x}^1 - \textbf{x}| \to \infty} \textbf{u}_k^1[\textbf{x},t;\textbf{x}^1] = \textbf{u}_k[\textbf{x},t]\\
    \lim_{|\textbf{x}^1 - \textbf{x}| \to \infty} p_k^1[\textbf{x},t;\textbf{x}^1] = p_k[\textbf{x},t]
\end{align*}
The source terms $\avg{\delta_\beta\delta_\alpha \intS{\bm\sigma_k^0 \cdot \textbf{n}_2}}[\textbf{x},t;\textbf{x}^1]$ might be reformulated using the latter manipulation to yields, 
\begin{equation*}
    \div \avg{\delta_\beta\delta_\alpha \intS{\bm\sigma_k^0 \cdot \textbf{n}_2}}
    = 
    \int_{|\textbf{x}-\textbf{x}^2| = a}
    \avg{\delta_\alpha \delta_\beta \bm\sigma_1^0\cdot\textbf{n}_2}
    [\textbf{x}^2,t;\textbf{x},\textbf{x}^1]
    d\textbf{x}^2
\end{equation*}
which is the averaged value of the surface force $\bm\sigma_1^0\cdot\textbf{n}_2$ evaluated in $\textbf{x}^2$, where $\textbf{x}^2$ is a point of the surface of a particle located at $\textbf{x}$ knowing another particle is at $\textbf{x}^1$.
This term is therefore proportional to $n_p^2$.  
\begin{equation*}
    \avg{\delta_\beta\delta_\alpha \intS{\textbf{r}\bm\sigma_k^0 \cdot \textbf{n}_2}}
    = 
    \int_{|\textbf{x}-\textbf{x}^2| = a}
    \avg{\delta_\alpha \delta_\beta  \textbf{r}\bm\sigma_1^0\cdot\textbf{n}_2}
    [\textbf{x}^2,t;\textbf{x},\textbf{x}^1]
    d\textbf{x}^2
\end{equation*}








As mentioned in \citet{hinch1977averaged}, if the particle surface or volume is particle dependent relation like \ref{eq:conditional_sphere} is not sufficient. 
In the continuity of the previous example we now consider spheroidal particles described by the Lagrangian conformation tensor $\textbf{C}_\alpha(t;\FF)$. 
Now, the geometry of the particles is described in space by their position $\textbf{x}_\alpha(t,\FF)$
and shape tensor, $\textbf{C}_\alpha(t,\FF)$, besides we recall that the equation for the surface of the particle $\alpha$ is described by the equation $\textbf{rr}:\textbf{C}^{-1}_\alpha - a^2 =0$. 
The internal coordinate of the properties $\textbf{C}_\alpha$ will be noted $\textbf{C}$ in the phase space, similarly as $\textbf{x}$ is liked to $\textbf{x}_\alpha$ in the Eularian frame.  
Using these properties, the ensemble average of the surface traction term might be reformulated as, 
\begin{align}
    \pSavg{\bm\sigma_1^0\cdot\textbf{n}_2}[\textbf{x}^1,t]
    &=
    \int_{\mathbb{R}^9}
    \int_{\textbf{rr}: (\textbf{C}^1)^{-1} = a^2}
    \avg{\delta_\alpha \delta_\textbf{C}  \bm\sigma_1^0\cdot \textbf{n}_2}
    [\textbf{x},t;\textbf{x}^1,\textbf{C}^1]
    d\textbf{x}d\textbf{C}^1
    \label{eq:conditional_spheroid}
\end{align}
where $\textbf{r} = \textbf{x} - \textbf{x}_{\alpha}(t;\FF)$ or $\textbf{x} - \textbf{x}^1$ due to the presence of $\delta(\textbf{x} - \textbf{x}^1)$. 
On the second line $\delta(\textbf{rr}: \textbf{C}_\alpha^{-1}(t,\FF) -a^2)$ permits us to extend the particle dependent domain of integration to $\mathbb{R}^3$. 
Finally, we introduced the notation, $\delta_\textbf{C} = \delta(\textbf{C}^1 - \textbf{C}_\alpha(t,\FF))$. 
We therefore reduced our ensemble average to a space average over the surface of a spheroid, and an integral over all $\textbf{C}$ meaning over all shape possible.
However, we now need to close the term $\avg{\delta_\alpha \delta_\textbf{C} \bm\sigma_2^0\cdot \textbf{n}_2}
[\textbf{x},t;\textbf{x}^1,\textbf{C}^1]$, which can be written alternatively $\avg{\delta_\alpha \delta_\textbf{C} \bm\sigma_2^0\cdot \textbf{n}_2}
[\textbf{x},t;\textbf{x}^1] = \avg{\bm\sigma_2^0\cdot \textbf{n}_2}^{1\textbf{C}}
[\textbf{x},t,\textbf{x}^1,\textbf{C}^1]$.
The first term can be thought as the averaged surface force evaluated at $\textbf{x}$ on the surface of a particle centered at $\textbf{x}^1$ with its shape described by $\textbf{C}$.
The second term is the probability density function $P_1(\textbf{x}^1,\textbf{C}^1)$ such that $P_1(\textbf{x}^1,\textbf{C}^1)d\textbf{x}d\textbf{C}$ si the probable number of particle center of mass at $\textbf{x}^1$ having a conformation tensor $\textbf{C}$. 




The conditional phase average of the carrier fluid stress, $\avg{\delta_\beta\chi_2 \bm\sigma_2^0}$ might be written,
\begin{align*}
    \avg{\delta_\beta\chi_1 \bm\sigma_1^0}
    &= 
    \avg{\delta_\beta (-p_1^0 + \mu_1 \textbf{e}^0 )}
    - \avg{\delta_\beta\chi_2 \bm\sigma_1^0}\\
    &= 
    - n_p p_1^1 \textbf{I}
    + n_p \mu_1 (\grad \textbf{u}_1^1 + \grad \textbf{u}_1^1)
    - \avg{\delta_\beta\chi_2 \bm\sigma_1^0}[\textbf{x},t;\textbf{x}^1]\\
\end{align*}
where the second term is the contribution from the particles to the conditional stress. 
Note that it appear under the divergence sign in the Stokes, equation consequently it can be reformulated as\citet{hinch1977averaged}, 
\begin{equation*}
    \div \avg{\delta_\beta\chi_2 \bm\sigma_1^0}[\textbf{x},t;\textbf{x}^1]
    =\div\int_{|\textbf{x}-\textbf{x}^2| < a}
    \avg{\delta_\alpha\delta_\beta \bm\sigma_1^0}[\textbf{x},t;\textbf{x}^1,\textbf{x}^2] d\textbf{x}^2
\end{equation*}
we can re-write this equality as, 
\begin{align*}
    \div \avg{\delta_\beta\chi_2 \bm\sigma_1^0}[\textbf{x},t;\textbf{x}^1]
    =
    \int_{|\textbf{x} -\textbf{x}^2|<a}
    \int_{|\textbf{x}'-\textbf{x}^2|<a}
    \avg{\delta_\alpha\delta_\beta \bm\sigma_1^0}[\textbf{x}',t;\textbf{x}^1,\textbf{x}^2]
    \cdot \grad \delta(\textbf{x}' - \textbf{x})
    d\textbf{x}'
    d\textbf{x}^2\\
    =
    \int_{|\textbf{x}-\textbf{x}^2|<a}
    \int_{|\textbf{x}'-\textbf{x}^2|=a}
    \avg{\delta_\alpha\delta_\beta \bm\sigma_1^0}[\textbf{x}',t;\textbf{x}^1,\textbf{x}^2]
    \cdot  \textbf{n}(\textbf{x}') \delta(\textbf{x}' - \textbf{x})
    d\textbf{x}'
    d\textbf{x}^2\\
\end{align*}
which is the volume integral of the stresslet around the point \textbf{x}. 
The stokes equation in a steady state non-inertial situation reads, 
\begin{equation}
    \div (\phi_k^1\textbf{u}_k^1)
    = 0
\end{equation}
\begin{equation}
    - n_p \grad (p_1^1)
    + \mu_1n_p \div (\grad \textbf{u}_1^1 + \grad \textbf{u}_1^1)
    + n_p  \rho_k \phi_k^1 \textbf{g} 
    + \avg{\delta_I\delta_\beta \bm\sigma_k^0 \cdot \textbf{n}_k}
    = 0 
\end{equation}
with the boundary at large distance from a particle, 
\begin{align*}
    \lim_{|\textbf{x}^1 - \textbf{x}| \to \infty} \phi_1^k[\textbf{x},t;\textbf{x}^1] = \phi_k[\textbf{x},t]\\
    \lim_{|\textbf{x}^1 - \textbf{x}| \to \infty} \textbf{u}_k^1[\textbf{x},t;\textbf{x}^1] = \textbf{u}_k[\textbf{x},t]\\
    \lim_{|\textbf{x}^1 - \textbf{x}| \to \infty} p_k^1[\textbf{x},t;\textbf{x}^1] = p_k[\textbf{x},t]
\end{align*}
The source terms $\avg{\delta_\beta\delta_\alpha \intS{\bm\sigma_k^0 \cdot \textbf{n}_2}}[\textbf{x},t;\textbf{x}^1]$ might be reformulated using the latter manipulation to yields, 
\begin{equation*}
    \div \avg{\delta_\beta\delta_\alpha \intS{\bm\sigma_k^0 \cdot \textbf{n}_2}}
    = 
    \int_{|\textbf{x}-\textbf{x}^2| = a}
    \avg{\delta_\alpha \delta_\beta \bm\sigma_1^0\cdot\textbf{n}_2}
    [\textbf{x}^2,t;\textbf{x},\textbf{x}^1]
    d\textbf{x}^2
\end{equation*}
which is the averaged value of the surface force $\bm\sigma_1^0\cdot\textbf{n}_2$ evaluated in $\textbf{x}^2$, where $\textbf{x}^2$ is a point of the surface of a particle located at $\textbf{x}$ knowing another particle is at $\textbf{x}^1$.
This term is therefore proportional to $n_p^2$.  
\begin{equation*}
    \avg{\delta_\beta\delta_\alpha \intS{\textbf{r}\bm\sigma_k^0 \cdot \textbf{n}_2}}
    = 
    \int_{|\textbf{x}-\textbf{x}^2| = a}
    \avg{\delta_\alpha \delta_\beta  \textbf{r}\bm\sigma_1^0\cdot\textbf{n}_2}
    [\textbf{x}^2,t;\textbf{x},\textbf{x}^1]
    d\textbf{x}^2
\end{equation*}



\section*{Particle property conditional average. This is the only good way to fully close the problem. }

Let $\CC_1$ be the configuration of a particle, with its Lagrangian counterpart $\CC_\alpha$. 
The configuration is decomposed into $\CC_1 = (\textbf{x}_1, \bm\Lambda_1)$, equally $\CC_\alpha = (\textbf{x}_\alpha,\bm\Lambda_\alpha)$.

In fact, the only conditionally fields that we are able to close is the $f^1$ defined such that, 
\begin{equation}
    f[\textbf{x},t]
    = \int_{\mathbb{R}^3}
    f^1[\textbf{x},t;\CC^1]
    P(\CC^1) d\CC^1
    \label{eq:avg_cond_C1}
\end{equation}
with, 
\begin{equation*}
    f^1[\textbf{x},t;\CC^1] P(\CC^1)
    =
    \frac{1}{N} 
    \avg{\delta_\alpha f^0}[\textbf{x},t;\CC^1]
    = 
    \frac{1}{N}
    \int 
    \sum_\alpha^N \delta(\CC^1 - \CC_\alpha(\FF,t))
    f^0[\textbf{x},t;\FF]
    d\PP
\end{equation*}
where $\CC^1$ is the vector which contain all the particle properties needed to compute the drag in a certain scenario. 
Note that $\CC^1$ contain only the particles properties not those of his neighbors since it would require another definition for the average. 
To prove the first equality one has to show that, 
\begin{align*}
    f[\textbf{x},t]
    = \frac{1}{N}\sum_\alpha^N\int_{\CC} \delta(\CC_1 - \CC_\alpha) d\CC_1
    \int
    f^0[\textbf{x},t;\FF]
    d\PP\\
    = 
    \int_{\CC} 
    \int
    \frac{1}{N}\sum_\alpha^N
    \delta(\CC_1 - \CC_\alpha) 
    f^0[\textbf{x},t;\FF]
    d\PP
    d\CC_1
    \\
    = 
    \int_{\CC} 
    f^1[\textbf{x},t;\CC_1]
    d\CC_1
\end{align*}
Regarding the particle averaged properties we can also derive, for spherical particle we have shown that,
\begin{align}
    \pSavg{\bm\sigma_1^0\cdot\textbf{n}_2}[\textbf{x}^1,t]
    = 
    \int_{\mathbb{R}^3}
    \int
     \sum_{\alpha=1}^N \delta(\textbf{x}^1-\textbf{x}_\alpha)
    \delta(|\textbf{x} - \textbf{x}_{\alpha}|-a)
    (\bm\sigma_1^0\cdot\textbf{n}_2)[\textbf{x},t;\FF]
    d\PP
    d\textbf{x}\\
    = 
    \int_{\mathbb{R}^3}
    \int
     \sum_{\alpha=1}^N \delta(\textbf{x}^1-\textbf{x}_\alpha)
    \delta(|\textbf{x} - \textbf{x}_{\alpha}|-a)
    (\bm\sigma_1^0\cdot\textbf{n}_2)[\textbf{x},t;\FF]
    d\PP
    d\textbf{x}\\
\end{align}
In the classic way this terms will be non-null only for $\textbf{x}_\alpha = \textbf{x}^1 = \textbf{x} + a$ therefore the domain of integration might be changed, but let add a condition. 
Noticing that $\int_{\bm\Lambda} \delta(\bm\Lambda_1 - \bm\Lambda_\alpha) d\bm\Lambda_1 = 1$ because at some point $\bm\Lambda_\alpha=\bm\Lambda_1$, therefore, 
\begin{align*}
    \pSavg{\bm\sigma_1^0\cdot\textbf{n}_2}[\textbf{x}^1,t]
    = 
    \int_{\mathbb{R}^3,\bm\Lambda}
    \int
    \sum_{\alpha=1}^N 
    \delta(\bm\Lambda_1 - \Lambda_\alpha) 
    \delta(\textbf{x}^1-\textbf{x}_\alpha)
    \delta(|\textbf{x} - \textbf{x}_{\alpha}|-a)
    (\bm\sigma_1^0\cdot\textbf{n}_2)[\textbf{x},t;\FF]
    d\PP
    d\Lambda_1
    d\textbf{x}\\
\end{align*}
The term in the integral is the local stress force evaluated at $\textbf{x},t$ knowing a particle is at $\textbf{x}^1$ with property $\bm\Lambda_1$, with the last condition that the point \textbf{x} is at a distance $a$ from $\textbf{x}_\alpha$. 
\tb{The problem is that we do not have the $1/N$ pre factor, and also the surface condition might include a factor as well  ???} 
Thus, we must integrate on all point \textbf{x} at the surface of the particles conditionally on $\CC_1$, yielding,   
\begin{align*}
    \pSavg{\bm\sigma_1^0\cdot\textbf{n}_2}[\textbf{x}^1,t]
    = 
    \int_{|\textbf{x}-\textbf{x}^1| = a,\bm\Lambda}
    \avg{\delta_1\bm\sigma_1^0\cdot\textbf{n}_2}[\textbf{x},t;\CC_1]
    d\Lambda_1
    d\textbf{x}\\
    = 
    \int_{|\textbf{x}-\textbf{x}^1| = a,\bm\Lambda}
    \bm\sigma_1^1[\textbf{x},t;\CC_1]\cdot\textbf{n}_2P(\CC_1)
    d\Lambda_1
    d\textbf{x}\\
\end{align*}

That said it is easy to generalize \ref{eq:dt_delta_alpha} to, 
\begin{equation*}
    \pddt \delta(\CC^1 - \CC_\alpha(\FF,t))
    + \frac{d \CC_\alpha}{dt}
    \frac{\partial }{\partial \CC^1} \delta(\CC^1 - \CC_\alpha(\FF,t))
    = 0 
\end{equation*}
The particle properties could include acceleration shape and so on.
Let's focus on stokes inertialess partcile. 
Assuming the only internal properties required to compute the drag force is the center of mass position $\textbf{x}^1$ and velocity $\textbf{u}^1$ of the particle, then we have $\delta(\CC^1 - \CC_\alpha(\FF,t)) =\delta(\textbf{x}^1 - \textbf{x}_\alpha(\FF,t)) \delta(\textbf{u}^1 - \textbf{u}_\alpha(\FF,t)) = \delta_1 $ and it's transport equation reduce to, 
\begin{equation*}
    \pddt \delta_1
    + 
    \textbf{u}_\alpha \cdot \grad_{\textbf{x}_1} \delta_1
    + \textbf{f}_\alpha \cdot \grad_{\textbf{u}_1} \delta_1
    = 0 
\end{equation*}
where $\textbf{f}_\alpha = \frac{1}{m_\alpha}\int_{\Sigma_\alpha} \bm\sigma_1^0 \cdot \textbf{n} d\Sigma + \textbf{g}$.  
It is clear that this equation shear some similitude with Louville's equation. 

Until now we stayed pretty general. 
From now on we consider spherical solid particle.
It is good to recall that the term we are trying to formulate is the ensemble average  drag force term. 
This term can be formulated as, 
\begin{align}
    \pSavg{\bm\sigma_1^0\cdot\textbf{n}_2}[\textbf{x}^1,t]
    &=
    \int_{|\textbf{x}-\textbf{x}^1|=a}
    \avg{\delta_1  \bm\sigma_1^0\cdot \textbf{n}_2}
    [\textbf{x},t;\textbf{x}^1,\textbf{u}^1]
    d\textbf{x}
    d\textbf{u}^1
\end{align}
The stress, might be expressed following a Newtonian law, besides, we consider that the particle contribution to this stress is null such that we discard possible contact forces. 
Additionally, notice that the normal to the particle $\alpha$ is entirely determined by the position of the latter particle, therefore, $\textbf{n}_2 = f(\textbf{x},\textbf{x}^1)$ and can be taken out of the ensemble average operator. 
In this situation the fluid phase external stress might be written, 
\begin{equation*}
    \avg{\delta_1  \bm\sigma_1^0\cdot \textbf{n}_2}
    = 
    - \avg{\delta_1  p_1^0 }\textbf{n}_2^1
    + \mu_1 \avg{\delta_1  (\grad \textbf{u}_1^0 + \grad \textbf{u}_1^0)\cdot }\textbf{n}_2^1
    = 
    -   p_1^1 \textbf{n}_2^1
    + \mu_1  (\grad \textbf{u}_1^1 + \grad \textbf{u}_1^1) \cdot \textbf{n}_2^1. 
\end{equation*}
Without any further assumption we demonstrated that the conditional ensemble average operator applies directly through the velocity fields. 
Therefore, upon knowing $\textbf{u}_1^1$ and $p_1^1$ one can determine the ensemble drag force on the dispersed phase. 







To determine these variables we now proceed to the derivation of the conditionally averaged Navier-Stokes equation. 
In the two-fluid formulation we obtain in the most general case, 
\begin{equation}
    \pddt \avg{\delta_1 \chi_k \rho_k}
    +  \div \avg{\delta_1 \chi_k \rho_k \textbf{u}_k^0}
    +  \pddx_1 \cdot
    \avg{\delta_1 \textbf{u}_\alpha \chi_k \rho_k}
    +  \pddu_1 \cdot
    \avg{\delta_1  \textbf{f}_\alpha \chi_k \rho_k}
    = 0
\end{equation}
\begin{multline}
    \pddt \avg{\delta_1 \rho_k \chi_k \textbf{u}_k^0 }
    + \div(
        \avg{\delta_1 \rho_k \chi_k \textbf{u}_k^0 \textbf{u}^0_k  }
    - \avg{\delta_1\chi_k\bm\sigma^0_k})\\
    + \pddx_1 \cdot
        \avg{\textbf{u}_\beta\delta_1 \rho_k \chi_k \textbf{u}_k^0 }
    + \pddu_1 \cdot
        \avg{\textbf{f}_\beta\delta_1 \rho_k \chi_k \textbf{u}_k^0 }
    = n_p  \rho_k \phi_k^1 \textbf{g} 
    + \avg{\delta_1\delta_I \bm\sigma_k^0 \cdot \textbf{n}_k}
\end{multline}
As it is an unusual situation let's describe each of these terms. 

% Neglecting the fluctuating terms we get, 
% \begin{equation}
%     \pddt ( \phi_k^1 P_1)
%     +  \div (\phi_k^1\textbf{u}_k^1 P_1)
%     +  \pddx_1 \cdot
%     ( \textbf{u}^1\phi_k^1 P_1)
%     +  \pddu_1 \cdot
%     ( \textbf{f}^1() \phi_k^1 P_1)
%     = 0
% \end{equation}
% \begin{multline}
%     \pddt (\rho_k \phi_k^1 \textbf{u}_k^1 P_1 )
%     + \div(
%         \rho_k \phi_k^1 \textbf{u}_k^1\textbf{u}_k^1 P_1 
%     -  \phi_k^1 \bm\sigma_k^1 P_1 )\\
%     + \pddx_1 \cdot
%         (\textbf{u}^1 \rho_k \phi_k^1 \textbf{u}_k^1 P_1 )
%     + \pddu_1 \cdot
%         ( \textbf{f}^1 \rho_k \phi_k^1 \textbf{u}_k^1 P_1 )
%     = 
%     \rho_k \phi_k^1 \textbf{g} P_1 
%     + \avg{\delta_1\delta_I \bm\sigma_k^0 \cdot \textbf{n}_k}
% \end{multline}


The boundary condition for the velocity $\textbf{u}_k^1[\textbf{x},t;\textbf{x}^1,\textbf{u}^1]$ and the conditionally averaged volume fraction $\phi_k^1[\textbf{x},t;\textbf{x}^1,\textbf{u}^1]$ are then obvious, 
\begin{align*}
    \lim_{|\textbf{x}^1 - \textbf{x}| \to \infty} \phi^1_k[\textbf{x},t;\textbf{x}^1,\textbf{u}^1] =\phi_k[\textbf{x},t]
    \approx \phi_k[\textbf{x}_1,t] + (\textbf{x} - \textbf{x}_1)\cdot \grad \phi_k[\textbf{x}_1,t] + (\textbf{x} - \textbf{x}_1)^2: \grad^2 \phi_k[\textbf{x}_1,t] + \ldots\\
    \lim_{|\textbf{x}^1 - \textbf{x}| \to \infty} \textbf{u}_k^1[\textbf{x},t;\textbf{x}^1,\textbf{u}^1] = \textbf{u}_k[\textbf{x},t]
    \approx \textbf{u}_k[\textbf{x}_1,t] + (\textbf{x} - \textbf{x}_1)\cdot \grad \textbf{u}_k[\textbf{x}_1,t] + (\textbf{x} - \textbf{x}_1)^2: \grad^2 \textbf{u}_k[\textbf{x}_1,t] + \ldots 
\end{align*}
And close to the particle it must satisfy continuity. 
If we assume a non-roating particles then, 
\begin{align*}
    \textbf{u}_k^1[\textbf{x},t;\textbf{x}^1,\textbf{u}^1]
    = \textbf{u}^1
    \text{ at } \textbf{r} = a\\
    \phi_2^1 [\textbf{x},t;\textbf{x}^1,\textbf{u}^1] = 1 
    \text{ at } 
    r < a
\end{align*}
Note that hard sphere doesn't penetrate which gives us an additional condition for the volume fraction, 
\begin{equation*}
    \phi_2^1 [\textbf{x},t;\textbf{x}^1,\textbf{u}^1] = 0 
    \text{ at } 
    a< r < 2a
\end{equation*}
Then for $r > 2a$ the concentration fields of particle is considered unknown. 
\begin{remark}
    In the dilute case we consider that $\phi_1^1[\textbf{x},t;\textbf{x}^1,\textbf{u}^1] = 0$ for $r >a$. 
\end{remark}
\begin{remark}
    In the non-dilute, but completely random, homogeneous case 
    $\phi_1^1[\textbf{x},t;\textbf{x}^1,\textbf{u}^1] = \phi_1[\textbf{x}^1,t]$ for $r > 2a$
\end{remark}
\begin{remark}
    In the non-homogeneous but random case $\phi_1^1[\textbf{x},t;\textbf{x}^1,\textbf{u}^1] = \phi_1[\textbf{x}^1,t]+(\textbf{x}-\textbf{x}_1)\cdot\grad \phi_1[\textbf{x}_1,t]$ for $r > 2a$
\end{remark}
Also, if the particle is rotating it must be included in the conditional average. 
In conclusion, it is hard to estimate the phase space gradient terms $\pddx\ldots$ and the other $\pddu$. 
From this system it is not impossible that the final drag force end up to be function of $\grad\phi_k$ which is related to the migration force. 
These terms will be treated afterward, for now we focus on the formulation of the momentum exchange terms and drag force terms. 
The momentum exchange terms might be written, 
\begin{equation*}
    \avg{\delta_1 \delta_I
         \bm\sigma_k^0
    \cdot \textbf{n}_k  }[\textbf{x},t;\CC_1]
    = 
    \avg{\delta_1 
    \delta_\alpha
    \intS{\bm\sigma_k^0
   \cdot \textbf{n}_k }}[\textbf{x},t;\CC_1]
    - \div \avg{\delta_1 
    \delta_\alpha
    \intS{\textbf{r}\bm\sigma_k^0
   \cdot \textbf{n}_k }}[\textbf{x},t;\CC_1]
   + \ldots
\end{equation*}
Following the previous relation we might write, 
\begin{align}
    \pSavg{\delta_1\bm\sigma_1^0\cdot\textbf{n}_2}[\textbf{x},t;\CC_1]
    &=
    \int_{|\textbf{x}-\textbf{x}^2|=a}
    \pavg{\delta_1  \bm\sigma_1^0\cdot \textbf{n}_2}
    [\textbf{x}^2,t;\textbf{x},\textbf{x}^1,\textbf{u}^1]
    d\textbf{x}^2
    d\textbf{x}^1
    d\textbf{u}^1\\
    \pSavg{\delta_1\textbf{r}\bm\sigma_1^0\cdot\textbf{n}_2}[\textbf{x},t;\CC_1]
    &=
    \int_{|\textbf{x}-\textbf{x}^2|=a}
    \pavg{\delta_1  \textbf{r}\bm\sigma_1^0\cdot \textbf{n}_2}
    [\textbf{x}^2,t;\textbf{x},\textbf{x}^1,\textbf{u}^1]
    d\textbf{x}^2
    d\textbf{x}^1
    d\textbf{u}^1
\end{align}
Again these integrals require the conditionally averaged stress which can be written in terms of conditionally averaged velocity, but this time, conditioned on the presence of two particles.
Indeed, $\pavg{\delta_1  \bm\sigma_1^0\cdot \textbf{n}_2} =\bm\sigma_1^{1,2}\cdot \textbf{n}_2 =- p^{1,2}_1 \textbf{n}_2^{2} + \mu_1 (\grad \textbf{u}_1^{1,2}+\grad \textbf{u}_1^{1,2})$, therefore to express this term one must obtain these two-particle averaged velocity fields, which can be expressed using the two-particle averaged Navier-Stokes equation. 

Regarding the one-particle averaged fluid phase stress, it might be express in such a way, 
\begin{align*}
    \avg{\delta_1 \chi_1 \bm\sigma_1^0}
    = 
    \avg{\delta_1 \bm\sigma_1^0}
    - \avg{\delta_1 \chi_2 \bm\sigma_1^0}
    =
    - p_1^1 \textbf{I}
    +\mu_1  (\grad \textbf{u}_1^1+\grad \textbf{u}_1^1)
    - \avg{\delta_1 \chi_2 \bm\sigma_1^0}
\end{align*}
\tb{not exactly that but ok}
The volume integral of the particle interior stress might be expressed as a surface integral using the definition of the stress. 
Under these circumstances we might write the bulk stress of the one-particle conditionally averaged equation as, 
\begin{equation*}
    \bm\sigma_1^{1,eq}
    = \avg{\chi_1 \delta_1 \textbf{u}_1^{1'}\textbf{u}_1^{1'}}
    - p_1^1 \textbf{I}
    +\mu_1  (\grad \textbf{u}_1^1+\grad \textbf{u}_1^1)
    + \pSavg{\delta_1\textbf{r}\bm\sigma_1^0\cdot\textbf{n}_2}
    - \avg{\delta_1 \delta_\alpha \bm\sigma_1^0}
\end{equation*}
where we neglected the non-homogeneous terms. 
In short, the bulk stress present in the one-particle averaged momentum equation is the contribution from the stress let on particles minus the internal equivalent fluid stress evaluated at $\textbf{x}$ knowing a particle is present at $\textbf{x}^1$.
As witnessed by the presence of two Dirac delta function these terms are of order $\phi^2$, which at first order in $\phi$ is negligible.  
Regarding the fluctuating terms it represent the standard deviation of the one-particle averaged velocity which is also negligible. 

However, we still cannot neglect the rest of the terms these have different signification. 
Indeed, at $\mathcal{O}(\phi_1)$ the one-particle momentum equations readsd as, 
\begin{equation}
    \pddt (P_1\phi_1^1\rho_1)
    +  \div (P_1\phi_1^1\rho_1\textbf{u}_1^1)
    +  \pddx_1 \cdot
    (P_1\phi_1^1\rho_1 \textbf{u}_{1p})
    +  \pddu_1 \cdot
    (P_1\phi_1^1\rho_1 \textbf{f}_{1p})
    = 0
\end{equation}
\begin{multline}
    \pddt(P_1\phi_1^1 \textbf{u}_1^1)
    + \div(
        \phi_1^1 P_1 \textbf{u}_1^1\textbf{u}_1^1
        +\avg{\chi_1 \delta_1 \textbf{u}_1^{1'}\textbf{u}_1^{1'}}
        + p_1^1 \textbf{I}
        -\mu_1  (\grad \textbf{u}_1^1+\grad \textbf{u}_1^1)
    )\\
    + \pddx_1 \cdot
        (P_1\phi_1^1 \textbf{u}_1^1 \textbf{u}_{p,1})
    + \pddu_1 \cdot
        (P_1\phi_1^1 \textbf{u}_1^1 \textbf{f}_{p,1})
    = \avg{\delta_1 \rho_k \chi_k \textbf{g} }
    % + \avg{\delta_1\delta_I \bm\sigma_k^0 \cdot \textbf{n}_k}
\end{multline}
The $\pddx_1$ and $\pddu_1$ terms correspond to the flux of particles entering the phase space $\textbf{x}_1$ and $\textbf{u}_1$. 
So it is the fluid momentum carried by the velocity of particles entering in this phase space. 
The function $P_1(\textbf{x}_1,\textbf{u}_1)$ is a number density-like probability density function. 
It is the probable number of particle in the phase space $(\textbf{x}_1,\textbf{u}_1)$. 
This PDF increase because of two things, the particles flux through $\textbf{x}_1$ and $\textbf{u}_1$ witnessed by the divergence operators. 
We want to solve for the variable $P_1[\textbf{x}_1,\textbf{u}_1]\phi_1^1[\textbf{x},t;\textbf{x}_1,\textbf{u}_1]$. 
It is clear that we need an equation for $P_1$ first. 
It is given by averaging the transport equation of $\delta_1$ which gives, 
\begin{equation*}
    \pddt P_1
    + 
     \pddx_1\cdot(\textbf{u}_{p,1}  P_1)
    +\pddu_1\cdot(\textbf{f}_{p,1}  P_1)
    = 0 
\end{equation*}
Therefore, the p.d.f can be taken out the div in the last equation yielding, 
\begin{equation}
    \pddt (\phi_1^1\rho_1)
    +    \div (\phi_1^1\rho_1\textbf{u}_1^1)
    +   \textbf{u}_{1p} \cdot \pddx_1  (\phi_1^1\rho_1 )
    +   \textbf{f}_{1p} \cdot \pddu_1  (\phi_1^1\rho_1 )
    = 0
\end{equation}
\begin{multline}
    \pddt(P_1\phi_1^1 \textbf{u}_1^1)
    + \div(
        \phi_1^1 P_1 \textbf{u}_1^1\textbf{u}_1^1
        +\avg{\chi_1 \delta_1 \textbf{u}_1^{1'}\textbf{u}_1^{1'}}
        + p_1^1 \textbf{I}
        -\mu_1  (\grad \textbf{u}_1^1+\grad \textbf{u}_1^1)
    )\\
    + \pddx_1 \cdot
        (P_1\phi_1^1 \textbf{u}_1^1 \textbf{u}_{p,1})
    + \pddu_1 \cdot
        (P_1\phi_1^1 \textbf{u}_1^1 \textbf{f}_{p,1})
    = \avg{\delta_1 \rho_k \chi_k \textbf{g} }
    % + \avg{\delta_1\delta_I \bm\sigma_k^0 \cdot \textbf{n}_k}
\end{multline}
If we neglect the particle flux terms linked to the change in time of the momentum due to the change of $P_1$, then the particle volume fraction in the fluid appear as a gradient in front of the momentum which is logical. 
Maybe all of this study must be carried again. 
But thsi is for an other time. 
\subsection{Continuous phase}
\begin{align}
    \label{eq:dt_&vg_rho}
    \pddt (\phi_k \rho_k)  
    + \div (
        \phi_k \rho_k\textbf{u}_k
    )
    &= 
    0\\
    \label{eq:dt_&vg_rhou_k}
    \pddt (\phi_k \rho_k\textbf{u}_k)  
    + \div (
        \phi_k \rho_k\textbf{u}_k\textbf{u}_k
        - \bm{\sigma}_k^\text{eq}
    )
    &= 
    \phi_k \rho_k (\textbf{g} + \div \bm{\sigma}_k )
    +  \avg{\delta_I \bm{\sigma}_k' \cdot \textbf{n}_k}\\
    \label{eq:dt_&vg_rhoE_k}
    \pddt (\phi_k\rho_kE_k)  
    + \div (
        \phi_k\rho_kE_k\textbf{u}_k
        + \bm{q}_k^\text{eq}
        - \textbf{u}_k \cdot \bm{\sigma}_k^\text{eq}
        % - \textbf{u}_k^0 \cdot \bm{\sigma}_k^0 
        % + \textbf{q}_k^0
        )
    &= 
    \phi_k \rho_k[\textbf{u}_k \cdot \textbf{g} 
    + \div (\textbf{u}_k\cdot \bm{\sigma}_k - \textbf{q}_k)]\\
    &+ \avg{\delta_I (\textbf{u}_k' \cdot \bm{\sigma}_k^0 + \textbf{u}_k \cdot \bm{\sigma}_k' - \textbf{q}_k')\cdot \textbf{n}_k}
\end{align} 
\begin{align*}
    &\bm{\sigma}_k^\text{eq}
    = \phi_k\rho_k (
        % \bm{\sigma}_k%- n_p \textbf{M}_p
        - \kavg{\textbf{u}_k'\textbf{u}_k'})  
    &\textbf{q}_k^\text{eq}
    =\textbf{q}_k^\text{e} 
    +\textbf{q}_k^\text{k}  \\
    &\textbf{q}_k^\text{e}
    = \phi_k\rho_k \kavg{\textbf{u}_k' e_k'} 
    % + \phi_k\textbf{q}_k 
    &\textbf{q}_k^\text{k}
    = \phi_k\rho_k \kavg{\textbf{u}_k' k_k} 
    - \phi_k\kavg{\textbf{u}_k' \cdot \bm{\sigma}_k^0}
\end{align*}


Both transfer term can be reformulated using an expansion within the particle volume and inter particles distances. 
First of all, 
\begin{align*}
    \avg{\delta_I \bm{\sigma}_k' \cdot \textbf{n}_k}
    = n_p \textbf{f}_p - \div (n_p \mathcal{F}_p)
\end{align*}
Additionally $n_p \textbf{f}_p$ can be further decomposed using the nearest particle statistic decomposition, 
\begin{align*}
    \avg{\delta_I \bm{\sigma}_k' \cdot \textbf{n}_k}
    = n_p \textbf{f}_\text{pm} + \div (n_p \mathcal{F}_\text{pfp} - n_p \mathcal{F}_p)
\end{align*}
with the definition,
\begin{align*}
    n_p \mathcal{F}_p 
    = \pavg{\int_{\Sigma_\alpha}  \textbf{r} \bm{\sigma}_k' \cdot \textbf{n}_k d\Sigma }
    &&
    n_p \mathcal{F}_\text{pfp} 
    = \int \textbf{r} \pavg{ \delta_\beta  h_{\alpha\beta}\int_{\Sigma_\alpha} \bm{\sigma}_k' \cdot \textbf{n}_k d\Sigma } d\textbf{r}
\end{align*} 

using the same formalism we can show equally that, 
\begin{align*}
    \avg{\delta_I (\textbf{u}_k' \cdot \bm{\sigma}_k^0 + \textbf{u}_k \cdot \bm{\sigma}_k' - \textbf{q}_k')\cdot \textbf{n}_k}
    &= n_p \textbf{c}_\text{pm}^\text{tot} 
    + \div (n_p \mathcal{C}_\text{pfp}^\text{tot} - n_p \mathcal{C}_p^\text{tot})\\
    &=
    n_p \textbf{c}_\text{pm} 
    + n_p \textbf{e}_\text{pm} 
    + n_p \textbf{u}_k \cdot \textbf{f}_\text{pm}\\
    &+ \div [n_p \mathcal{C}_\text{pfp} - n_p \mathcal{C}_p + n_p \mathcal{E}_\text{pfp} - n_p \mathcal{E}_p + n_p \textbf{u}_k \cdot (\mathcal{F}_\text{pfp} - \mathcal{F}_p) ]
\end{align*}
with the following definitions
\begin{align*}
    n_p \mathcal{C}_p^\text{tot}
    &= \pavg{\int_{\Sigma_\alpha}  \textbf{r} (\textbf{u}_k' \cdot \bm{\sigma}_k^0 + \textbf{u}_k \cdot \bm{\sigma}_k' - \textbf{q}_k')\cdot \textbf{n}_k d\Sigma }\\
    &= n_p \mathcal{C}_p 
    + n_p \mathcal{E}_p 
    + n_p \textbf{u}_k \cdot \mathcal{F}_p 
    \\
    n_p \mathcal{C}_\text{pfp}^\text{tot} 
    &= \int \textbf{r} \pavg{ \delta_\beta  h_{\alpha\beta}\int_{\Sigma_\alpha} (\textbf{u}_k' \cdot \bm{\sigma}_k^0 + \textbf{u}_k \cdot \bm{\sigma}_k' - \textbf{q}_k')\cdot \textbf{n}_k d\Sigma } d\textbf{r} \\
    &= n_p \mathcal{C}_\text{pfp} 
    +n_p \mathcal{E}_\text{pfp} 
    + n_p \textbf{u}_k \cdot \mathcal{F}_\text{pfp} \\
\end{align*} 
But, we could have shown that,
\begin{align*}
    \avg{\delta_I (\textbf{u}_k' \cdot \bm{\sigma}_k^0 + \textbf{u}_k \cdot \bm{\sigma}_k' - \textbf{q}_k')\cdot \textbf{n}_k}
    &= 
    \avg{\delta_I (\textbf{u}_k' \cdot \bm{\sigma}_k^0 - \textbf{q}_k')\cdot \textbf{n}_k}
    + \avg{\delta_I \textbf{u}_k \cdot \bm{\sigma}_k'\cdot \textbf{n}_k}\\
    &=
    n_p \textbf{c}_\text{pm} 
    + \div [n_p \mathcal{C}_\text{pfp} - n_p \mathcal{C}_p]
    + \textbf{u}_k \cdot 
    [n_p \textbf{f}_\text{pm} + \div (n_p \mathcal{F}_\text{pfp} - n_p \mathcal{F}_p)]\\
\end{align*}
Consequently, we must deduce that, 
\begin{equation*}
    + \textbf{u}_k \cdot \div (n_p \mathcal{F}_\text{pfp} - n_p \mathcal{F}_p)
    = 
    + \div [ n_p \textbf{u}_k \cdot (\mathcal{F}_\text{pfp} - \mathcal{F}_p) ]
\end{equation*}
Meaning, $(n_p \mathcal{F}_\text{pfp} - n_p \mathcal{F}_p)\cdot \div \textbf{u}_k = 0$
Using the generic formulation \ref{eq:hybrid_avg_dt_chif} and the local expression of the mass, momentum and total energy expression, i.e. : \ref{eq:dt_rho},\ref{eq:dt_rhou_1} and \ref{eq:dt_rhoE_1} we easily find the averaged form of these equations as, 

\begin{align}
    \pddt (\phi_k \rho_k)  
    + \div (
        \phi_k \rho_k\textbf{u}_k
    )
    &= 
    0\\
    \pddt (\phi_k \rho_k\textbf{u}_k)  
    + \div (
        \phi_k \rho_k\textbf{u}_k\textbf{u}_k
        - \bm{\sigma}_k^\text{eq}
    )
    &= 
    \phi_k  (\rho_k\textbf{g} + \div \bm{\sigma}_k )
    - n_p \textbf{f}_\text{pm}\\
    \pddt (\phi_k\rho_kE_k)  
    + \div (
        \phi_k\rho_kE_k\textbf{u}_k
        + \bm{q}_k^\text{eq}
        - \textbf{u}_k \cdot \bm{\sigma}_k^\text{eq}
        % - \textbf{u}_k^0 \cdot \bm{\sigma}_k^0 
        % + \textbf{q}_k^0
        )
    &= 
    \phi_k \textbf{u}_k\cdot (\rho_k\textbf{g} 
    + \div \bm{\sigma}_k) 
    + \phi_k \bm{\sigma}_k \div\textbf{u}_k
    - \div\textbf{q}_k\\
    & - n_p (\textbf{c}_\text{pm}
    + \textbf{e}_\text{pm})
    - n_p \textbf{u}_k \cdot \textbf{f}_\text{pm}
\end{align} 
\todo{Is this formulation usefull for the NRJ equaiton ? }
where we have defined, 
\begin{align*}
    &\bm{\sigma}_1^\text{eq}
    = - \phi_1\rho_1 
        % \bm{\sigma}_1%- n_p \textbf{M}_p
        \kavg{\textbf{u}_1'\textbf{u}_1'}
        - n_p \mathcal{F}_\text{pfp} + n_p \mathcal{F}_p
    &\textbf{q}_1^\text{eq}
    =\textbf{q}_1^\text{e} 
    +\textbf{q}_1^\text{k}  
    \\
    &\textbf{q}_1^\text{e}
    = \phi_1\rho_1 \kavg{\textbf{u}_1' e_1'} 
    + n_p \mathcal{E}_\text{pfp} 
    - n_p \mathcal{E}_p 
    &\textbf{q}_1^\text{k}
    = \phi_1\rho_1 \kavg{\textbf{u}_1' k_1} 
    - \phi_1\kavg{\textbf{u}_1' \cdot \bm{\sigma}_1^0} 
    + n_p \mathcal{C}_\text{pfp} 
    - n_p \mathcal{C}_p 
\end{align*}
Note that the phase averaged energy equation can be further decompose following, 
\begin{align*}
    E_1 = e_1 + k_1 + u_1^2/2
\end{align*}
where $K_1$ is the pseudo-turbulent kinetic energy defined such as, $\phi_1 k_1 = \avg{\chi_1 (u_1')^2/2}$. 
The Macroscopic kinetic energy equation can be obtain by taking the dot product with $\textbf{u}_1$. 
\begin{align}
    \pddt (\phi_1 \rho_1u_1^2/2)  
    + \div (
        \phi_1 \rho_1\textbf{u}_1u_1^2/2
        - \textbf{u}_1 \cdot \bm{\sigma}_1^\text{eq}
    )
    &= 
    - \bm{\sigma}_1^\text{eq} : \grad \textbf{u}_1
    + \phi_k \textbf{u}_1 \cdot 
    (\textbf{g} \rho_k
    + \div \bm{\sigma}_1
    )
    -  n_p \textbf{u}_1\cdot \textbf{f}_\text{pm}\\
    \pddt (\phi_1\rho_1k_1)  
    + \div (
        \phi_1\rho_1k_1\textbf{u}_1
        + \textbf{q}_1^\text{k} 
        )
    &= 
    - \phi_1 \textbf{d}_1 
    + \bm{\sigma}_1^\text{eq}  : \grad \textbf{u}_1
    + \phi_k \bm{\sigma}_k \div\textbf{u}_k
    -n_p  \textbf{c}_\text{pm}\\
    \pddt (\phi_1\rho_1e_1)  
    + \div (
        \phi_1 \rho_1e_1\textbf{u}_1
        +
        \textbf{q}_1^\text{e} 
        )
    &= 
    \phi_1 (\textbf{d}_1 
    - \div \textbf{q}_k)
    - n_p \textbf{e}_\text{pm}
\end{align}
where $\textbf{d}_1 = \oneavg{\bm{\sigma}_1^0 : \grad \textbf{u}_1^0}$ is the averaged local dissipation tensor. 


\subsection{The dispersed phase equations}

Regarding the dispersed phase, we found the mass, momentum and total energy balance equations, 
\begin{align*}
    \pddt \left(n_p m_p\right)
    + \div \left(n_pm_p\textbf{u}_p
    \right)
    = 
    0\\
    \pddt \left(n_p m_p \textbf{u}_p\right)
    + \div \left(n_p
    m_p \textbf{u}_p \textbf{u}_p 
    - \bm{\sigma}_p^\text{eq}
    \right)
    = 
    n_p v_p  (\div \bm{\sigma}_1 
    + \rho_2 \textbf{g})
    + n_p \textbf{f}_{pm},\\
    \pddt(m_p E_p^\text{tot})
    + \div(m_p E_p^\text{tot} \textbf{u}_p 
    + \textbf{q}_p^\text{eq} - \textbf{u}_p \cdot \bm{\sigma}_p^\text{eq})
    = 
    n_pv_p \div (
        \textbf{u}_1 \cdot \bm{\sigma}_1 - \textbf{q}_1
    )
    + \pnavg{\int_{\Omega_\alpha} \textbf{u}_2^0 \cdot \textbf{g} \rho_2 d\Omega}
    + n_p \textbf{c}_\text{pm}
    % + n_p \textbf{e}_\text{pm}
    % + n_p \textbf{u}_1 \textbf{f}_\text{pm}
\end{align*}
where we have defined, 
\begin{align*}
    \bm{\sigma}_p^\text{eq}
    = - \pnavg{\textbf{u}_\alpha'\textbf{u}_\alpha'}
    + n_p \mathcal{F}_\text{pfp}
\end{align*}



The averaged particle energy $n_p E_p$ can be split into five components,
\begin{equation*}
    n_p m_p E_p(t) 
    = m_p n_p e_p 
    + \pnavg{\int_{\Omega_\alpha(t)} \rho_2  (w_2^0)^2/2 d\Omega}
    + m_p n_p k_p
    + m_p n_p (u_p)^2/2
    + n_p s_p \gamma
    % + \textbf{u}_\alpha \cdot \int_{\Omega_\alpha(t)} \rho_2  \textbf{w}_2^0 d\Omega
\end{equation*}
where $k_p = \pavg{(u_\alpha')^2/2}$.
one equation for each is riquiered 
Using the mass balance and the momentum balance dotted with $\textbf{u}_p$ we obtain the particle kinetic energy balance, 
\begin{equation*}
    \pddt \left(n_p m_p u_p^2/ 2\right)
    + \div \left(n_p
    m_p u_p^2/ 2 \textbf{u}_p 
    - \textbf{u}_p \cdot \bm{\sigma}_p^\text{eq}
    \right)
    = 
    - (\bm{\sigma}_p^\text{eq} + n_p v_p \bm{\sigma}_1) :\grad \textbf{u}_p
    +  n_p v_p  (\div (\textbf{u}_p \cdot \bm{\sigma}_1 )
    + \rho_2 \textbf{g})
    + n_p \textbf{u}_p \cdot \textbf{f}_{pm},\\
\end{equation*}

Additionally, we can add an equation for the first moment of momentum, 
\begin{multline}
    \pddt \left(n_p \mathcal{P}_p\right)
    + \div \left(
        n_p \textbf{u}_p \mathcal{P}_p
    + \Sigma_p^\text{eq}
    \right)
    =
    n_p v_p \bm{\sigma}_1 
    + n_p \mathcal{F}_p\\
    +\pnavg{\int_{\Omega_\alpha} \left(
        \rho_2 \textbf{w}_2^0  \textbf{w}_2^0 
        - \bm{\sigma}_2^0
        \right) d\Omega}
        - \gamma  \pnavg{\int_{\Sigma_\alpha} \textbf{I}_{||} d\Sigma},
\end{multline}

\subsection*{The drag force term}

The drag force term is easily closed by numerical method and some theoretical developments in the limiting case. 
Let now study the stokes 

\subsection*{Stress tensor for the continuous phase }
Regarding the fluid stress it can be reformulated considering Newtonian fluid,
\begin{equation}
    \phi_1 \bm{\sigma}_1 
    = - \phi_1 p_1 \textbf{I}
    + \mu_1 \phi_1 \textbf{e}_1
\end{equation}
with $\textbf{e}_1$ being the averaged shear rate. 
The first model is then, 
\begin{align*}
    \phi_1 \textbf{e}_1
    = \phi_1 (\nabla \textbf{u}_1+ (\grad \textbf{u}_1)^T)
    + \avg{[(\textbf{u}_1^0 - \textbf{u}_1)  \textbf{n}_1 +  \textbf{n}_1(\textbf{u}_1^0 - \textbf{u}_1 )]\delta_I}
\end{align*}
In \citet[chap 9]{ishii1975thermo} they assume,
\begin{equation}
    \avg{[(\textbf{u}_1^0 - \textbf{u}_1)  \textbf{n}_1 +  \textbf{n}_1(\textbf{u}_1^0 - \textbf{u}_1 )]\delta_I}\\
    = 
    (\textbf{u}_2 - \textbf{u}_1)  \grad \phi_1 +  \grad \phi_1(\textbf{u}_2 - \textbf{u}_1 )\\
\end{equation}
But I didn't find out where the derivation came from. 
Alternatively we can say that, 
\begin{align*}
    \phi_1 \textbf{e}_1
    = \nabla \textbf{u}+ (\grad \textbf{u})^T
    - \avg{\chi_2 (\grad\textbf{u}_2^0 + \grad(\textbf{u}_2^0 )^T)}
    = \textbf{e}
    - \phi_2 \textbf{e}_2
\end{align*}

More generally the stress within a suspension can be written,
\begin{align*}
    \bm{\sigma}_1 \phi_1
    &=- \phi_1 p_1 \textbf{I}
    + \mu_1 \textbf{e}
    - \lambda \phi_2 \bm{\tau}_2\\
    \bm{\sigma}_1 
    &= - \left(p_1 + \frac{\lambda \phi_2}{\phi_1} p_2\right) \textbf{I}
    + \frac{\mu_1}{\phi_1} \textbf{e}
    - \frac{\lambda \phi_2}{\phi_1} \bm{\sigma}_2\\
    \bm{\sigma}
    &= - \phi_1 p_1  \textbf{I}
    + \mu_1 \textbf{e}
    + \bm{\sigma}_2 \phi_2 
    +\phi_I \bm{\sigma}_I 
    - \lambda \phi_2 \bm{\tau}_2
\end{align*}
We can reformulate the last expression in the usual way using the first moment of momentum eq, 
\begin{equation}
    -  \dot{\mathcal{P}_p}
    +  \mathscr{S}_p^*
    +  \mathscr{L}_p
    + \frac{1}{3}(\bm{\sigma}_1^0 \cdot \textbf{n}_2 \cdot \textbf{r})_p^\Sigma \textbf{I}
    + n_p (\rho_2 \textbf{w}_2^0  \textbf{w}_2^0 )^\Omega
    =   (\bm{\sigma}_2^0)^\Omega
    + (\bm{\sigma}_I)^\Sigma,
\end{equation}
Or in stokes condition, 
\begin{equation}
    n_p \mathscr{S}_p^*
+ n_p \mathscr{L}_p
+ n_p\frac{1}{3}(\bm{\sigma}_1^0 \cdot \textbf{n}_2 \cdot \textbf{r})_p^\Sigma \textbf{I}
    = n_p \left(
        \bm{\sigma}_2^0
    \right)_p^\Omega
    +n_p (\bm{\sigma}_I)^\Sigma_p
\end{equation}
where we defined, 
\begin{align*}
    \mathscr{S}_p^* =\frac{1}{2} \pnavg{\int_{\Sigma_\alpha} \left(
        \textbf{r} \bm{\sigma}_1^0 \cdot \textbf{n}_2
        +  \bm{\sigma}_1^0 \cdot \textbf{n}_2\textbf{r}
        -
          \frac{2}{3}(\bm{\sigma}_1^0 \cdot \textbf{n}_2 \cdot \textbf{r})\textbf{I}
        \right)  d\Sigma}\\
    \mathscr{L}_p =\frac{1}{2} \pnavg{\int_{\Sigma_\alpha} \left(
        \textbf{r} \bm{\sigma}_1^0 \cdot \textbf{n}_2
        - \bm{\sigma}_1^0 \cdot \textbf{n}_2\textbf{r}
        \right) d\Sigma}
\end{align*}
Thus in homogeneous suspension without inertia we have, 
\begin{align*}
    \bm{\sigma}
    &= [- \phi_1 p_1 
    + n_p (\bm{\sigma}_1^0 \cdot \textbf{n}_2 \cdot \textbf{r})^\Sigma_p] \textbf{I}
    + \mu_1 \textbf{e}
    + n_p \mathscr{S}
    + n_p \mathscr{L}
\end{align*}
where the stress let is defined as $\mathscr{S} = \mathscr{S}_p^* - \lambda \phi_2 \bm{\tau}_2$

This, is definitely not trivial but if one wish to compute the first moment dynamical contribution to the suspension the formulas is given by 
\begin{align*}
    n_p \mathscr{S}_p
+ n_p \mathscr{L}_p
+ n_p\frac{1}{3}(\bm{\sigma}_1^0 \cdot \textbf{n}_2 \cdot \textbf{r})_p^\Sigma \textbf{I}
    &= 
    n_p \left(
        \bm{\sigma}_2^0
    \right)_p^\Omega
    +n_p (\bm{\sigma}_I)^\Sigma_p
    - \lambda \phi_2 \bm{\tau}_2\\
    &= 
    - n_p \left(
        p_2^0
    \right)_p^\Omega \textbf{I}
    +n_p (\bm{\sigma}_I)^\Sigma_p
    + n_p (1 - \lambda)\left(
        \mu_2 \textbf{e}_2^0
    \right)_p^\Omega 
\end{align*}
Let take the trace times $\frac{1}{3}$ of this equation, 
\begin{align*}
    \frac{1}{3} n_p(\bm{\sigma}_1^0 \cdot \textbf{n}_2 \cdot \textbf{r})_p^\Sigma 
    = 
    - n_p \left(
        p_2^0
    \right)_p^\Omega 
    +n_p \frac{1}{3}(\bm{\sigma}_I)^\Sigma_p : \textbf{I}
\end{align*}
Now let's substitute this equation into the former one, 
\begin{align*}
    n_p \mathscr{S}_p
+ n_p \mathscr{L}_p
=
    +n_p (\bm{\sigma}_I - \frac{1}{3}(\bm{\sigma}_I : \textbf{I})\text{I})^\Sigma_p
    + n_p (1 - \lambda)\left(
        \mu_2 \textbf{e}_2^0
    \right)_p^\Omega 
\end{align*}

If the particle is spherical, and that we remove the isotropic part on both sides of the equation we obtain 

In the spherical particle case the symmetric part reads, 
\begin{align*}
    n_p \mathscr{S}_p
    &= 
    + n_p (1 - \lambda)\left(
        \mu_2 \textbf{e}_2^0
    \right)_p^\Omega 
\end{align*}
which is false. 

\subsubsection{A translating sphere}
In the dilute stokes regime the disturbance velocity around a droplet can be written, 
\begin{align*}
    u_i^\text{Ext}(\textbf{r})
    = \left(\frac{\delta_{ik}}{r} + \frac{r_ir_k}{r^3}\right)  g_k
    + \left(-\frac{\delta_{ik}}{r^3} + \frac{3r_ir_k}{r^5}\right)  d_k\\
    u_i^\text{In}(\textbf{r})
    = c_i
    + \left(2 r^2 \delta_{ik} - r_ir_k\right) d_k\\
    e_{ik}
    = \mu(
        3 \delta_{ij} r_k 
        + 3 \delta_{kj} r_i
        -2 r_j \delta_{ki}
    )e_j 
\end{align*}
Applying the non deformation at the interface and other shear free condition we find the constant to be, 
\begin{align*}
    &\textbf{g} = \frac{1}{4}\left(\frac{3\lambda + 3}{\lambda +1}\right) a \textbf{U}
    &\textbf{d} = -\frac{1}{4}\left(\frac{\lambda}{\lambda +1}\right) a \textbf{U}\\
    &\textbf{c} = \frac{1}{2}\left(\frac{3\lambda + 3}{\lambda +1}\right) \textbf{U}
    &\textbf{e} = -\frac{1}{2}\left(\frac{\lambda}{\lambda +1}\right) \frac{1}{a^2} \textbf{U}\\
\end{align*}
Let consider isolated particles immersed in a viscous flow in the case $\textbf{u}_1' = \textbf{u}^{Ext}$.

The averaged internal shear rate $\phi_2 \textbf{e}_2$ can be thus estimated through the integral, 
\begin{align*}
    \avg{\chi_2 (\textbf{e}_2^0)_{ik}}
    &= \pavg{\int_{\Omega} \mu(
        3 \delta_{ij} r_k 
        + 3 \delta_{kj} r_i
        -2 r_j \delta_{ki}
    )e_j d\Omega}
    = 0\\
    &+ \div \pavg{\int_{\Omega} \textbf{r}\mu(
        3 \delta_{ij} r_k 
        + 3 \delta_{kj} r_i
        -2 r_j \delta_{ki}
    )e_j d\Omega}
\end{align*}
\subsubsection{A drop in shear flow}

The functional form of the internal velocity fields for a droplet immersed in a shear flow is,
\begin{align*}
    u_i^\text{Ext}(\textbf{r})
    = \left(\frac{\delta_{ij} r_l - \delta_{il} r_j - \delta_{jl} r_i}{r^3} 
    + \frac{r_ir_jr_l}{r^5}\right)  d_{jl}\\
    + \left(-3 \frac{\delta_{ij} r_l + \delta_{il} r_j + \delta_{jl} r_i}{r^5} 
    + 15\frac{r_ir_jr_l}{r^7}\right)  p_{jl}\\
    u_i^\text{In}(\textbf{r})
    = \left(- 4 \delta_{ij} r_l  + \delta_{il} r_j + \delta_{jl} r_i\right) f_{jl}\\
    e_{ik}
    = \mu [
        \partial_i \left(- 4 \delta_{kj} r_l  + \delta_{kl} r_j + \delta_{jl} r_k\right) f_{jl} + \partial_k \left(- 4 \delta_{ij} r_l  + \delta_{il} r_j + \delta_{jl} r_i\right) f_{jl} 
    ]\\
    = \mu [
        \left(- 4 \delta_{kj} \delta_{li}  + \delta_{kl} \delta_{ji} + \delta_{jl} \delta_{ki}\right) f_{jl} + \left(- 4 \delta_{ij} \delta_{lk}  + \delta_{il} \delta_{kj} + \delta_{jl} \delta_{ki}\right) f_{jl} 
    ]\\
    = \mu [
        \left(- 4 f_{ki}  + f_{ik} + f_{jj}\delta_{ki}\right)  + \left(- 4 f_{ik}  + f_{ki} + f_{jj} \delta_{ki}\right)  
    ]\\
    = \mu [
        \left(- 3 f_{ki}  - 3 f_{ik} +2 f_{jj}\delta_{ki}\right) 
    ]
\end{align*}
Here i know that, 
\begin{align*}
    \textbf{d}
    = - \frac{1}{6} \left(
        \frac{2+5\lambda}{1+\lambda}a^3 \textbf{E}
    \right)
    &&
    \textbf{p}
    = - \frac{1}{6} \left(
        \frac{\lambda(2+5\lambda)}{(1+\lambda)(5\lambda+2)}a^3 \textbf{E}
    \right)
\end{align*}
Integrating this functional over the volume of a droplet yields, 
\begin{equation}
    \avg{\chi_2 (\textbf{e}_2^0)_{ik}}
    = \pavg{\int_{\Omega} 
        \mu [
        \left(- 3 f_{ki}  - 3 f_{ik} +2 f_{jj}\delta_{ki}\right) 
    ] d\Omega}
    = n_pv_p(-3 (f_{ki}+ f_{ik}) + 2 f_{jj} \delta_{ki})
\end{equation}
The only remaining thing is to do determine the form of $f_{ik}$. 


\subsection*{Continuous phase fluctuation term}

The Reynolds stress $\oneavg{\textbf{u}_1'\textbf{u}_1'}$ can be described in the limit of dilute non interaction particles by the wake. 
Therefore, by direct integration of $\textbf{u}^\text{Ext}$ we should be able to find a first correction of the velocity correlation. 
In a homogeneous flow of isolated particle ensemble average is equivalent to volume average thus, 
\begin{align*}
    \oneavg{\textbf{u}_1' \textbf{u}_1' }
    = \int u_i^\text{Ext} u_j^\text{Ext} d\textbf{r}\\
    = \int [\left(\frac{\delta_{ik}}{r} + \frac{r_ir_k}{r^3}\right)  g_k
    + \left(-\frac{\delta_{ik}}{r^3} + \frac{3r_ir_k}{r^5}\right)  d_k]
    [ \left(\frac{\delta_{jl}}{r} + \frac{r_jr_l}{r^3}\right)  g_l
    + \left(-\frac{\delta_{jl}}{r^3} + \frac{3r_jr_l}{r^5}\right)  d_l] d\textbf{r}\\
    = \int [\left(\frac{\delta_{ik}}{r} + \frac{r_ir_k}{r^3}\right)
    \left(\frac{\delta_{jl}}{r} + \frac{r_jr_l}{r^3}\right)  g_l  g_k
    + \left(\frac{\delta_{ik}}{r} + \frac{r_ir_k}{r^3}\right) \left(-\frac{\delta_{jl}}{r^3} + \frac{3r_jr_l}{r^5}\right)  d_l g_k\\
    + \left(-\frac{\delta_{ik}}{r^3} + \frac{3r_ir_k}{r^5}\right) \left(\frac{\delta_{jl}}{r} + \frac{r_jr_l}{r^3}\right)  g_l d_k
    + \left(-\frac{\delta_{ik}}{r^3} + \frac{3r_ir_k}{r^5}\right) \left(-\frac{\delta_{jl}}{r^3} + \frac{3r_jr_l}{r^5}\right)  d_l d_k]
    d\textbf{r}\\
\end{align*}
Since all terms seems identical let's focus on the first one it reads, 
\begin{equation*}
    \int \left(\frac{\delta_{ik}}{r} + \frac{r_ir_k}{r^3}\right)
    \left(\frac{\delta_{jl}}{r} + \frac{r_jr_l}{r^3}\right)  g_l  g_kd\textbf{r}
    = 
    g_l  g_k
    \int \left(
    \frac{\delta_{ik}\delta_{jl}}{r^2} 
    + \frac{\delta_{ik} r_jr_l}{r^4} 
    + \frac{\delta_{jl} r_ir_k}{r^4} 
    + \frac{r_ir_kr_jr_l}{r^6} 
    \right)d\textbf{r}
\end{equation*}
In spherical coordinate $d\textbf{r} = r^2 \sin\theta dr d\theta d\phi$. 
Therefore, the first integration reads, 
\begin{equation*}
    g_l  g_k\delta_{ik}\delta_{jl}
    \int_0^{2\pi} 
    \int_0^{\pi} 
    \int_1^{\infty} 
    \frac{1}{r^2} 
    r^2 \sin\theta dr d\theta d\phi
    = g_l  g_k\delta_{ik}\delta_{jl}
    4\pi 
    \int_1^\infty dr
\end{equation*}
The second terms can be re-written using the Gauss divergence theorem, 
\begin{equation}    
g_l  g_k
\int 
\frac{\delta_{ik} r_jr_l}{r^4} 
d\textbf{r}
\end{equation}
This integral diverges thus it is not possibly feasible to compute such thing,
however : 
\begin{multline*}
    \avg{\chi_k \textbf{u}'_k\textbf{u}'_k}(\textbf{x},t)
    + \phi_k \textbf{u}_k\textbf{u}_k
    = \\
    \underbrace{\int (\nstavg{\chi_k \textbf{u}^0_k}  \nstavg{\chi_k \textbf{u}^0_k} / (\nstavg{\chi_k})  P_{nst}(\textbf{x},t,\textbf{r}) d\textbf{r} }_\text{PWFs}
    +\underbrace{\int \nstavg{\chi_k \textbf{v}_k^0\textbf{v}_k^0}  P_{nst}(\textbf{x},t,\textbf{r}) d\textbf{r}}_\text{WIA}
    \label{eq:def_uu}
\end{multline*}
where, $\textbf{v}_k^0  = \textbf{u}_k^0 - \nstavg{\chi_k \textbf{u}^0_k} / \nstavg{\chi_k}$ 
for a dillute random distribution, $P_\text{nst}^\text{th}(\textbf{y}|\textbf{x}) = n_p e^{-4 \pi n_p (r^3 - a^3)/3}$,
Now the first integral reads, 
\begin{equation}
    g_lg_k \delta_{ik} \delta_{jl} 
    4\pi    
    \int_a^{\infty} 
    \sin\theta e^{-4 \pi n_p (r^3 - a^3)/3} dr 
    = 
\end{equation}
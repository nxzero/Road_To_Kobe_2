Where $\Omega_\alpha$ is the domain occupied by the particle $\alpha$ (see \ref{fig:Scheme}). 
Subsequently, we define the velocity of the particle's center of mass, denoted as $\textbf{u}_\alpha$ which is given by $\textbf{u}_\alpha = \ddt{ \textbf{x}_\alpha}$. 
The derivation of $\ddt {\textbf{x}_\alpha}$ is straightforward but requires some algebra which are detailed in \ref{ap:velocity_definition}. 
The final expression reads,
\begin{equation}
    \textbf{u}_\alpha(t) = \frac{1}{m_\alpha(t)} \left(
        \textbf{p}_\alpha(t)
        +  \int_{\Sigma_\alpha(t)} \rho_2 \textbf{r} (\textbf{u}_I^0 - \textbf{u}_2^0)\cdot \textbf{n}_2 d\Sigma
        \right),
        \label{eq:dt_y_alpha}
\end{equation}
where $\textbf{r}(\textbf{x},t) = \textbf{x} - \textbf{x}_\alpha(t)$. 
In Equation \ref{eq:dt_y_alpha}, it can be observed that the first component of the velocity represents the linear momentum divided by the mass of the particle. 

\subsection{Conservation laws}
Following the same assumption as in \ref{sec:local_eq}, i.e. we consider no mass transfer and weightless interfaces, the Lagrangian  mass, momentum and energy equations for a single particle can be derived using the generic form \ref{eq:dt_q_alpha_tot} and reads as, 
\begin{align}
    \label{eq:dt_m_alpha}
    \ddt m_\alpha
    &= 
    0\\
    \ddt {\textbf{M}_\alpha}
    &=2\textbf{S}_\alpha. 
    \label{eq:dt_M_alpha}
    \\
    \label{eq:dt_p_alpha}
    \ddt \textbf{p}_\alpha
    &= 
    m_\alpha\textbf{g}
    +  \intS{\bm{\sigma}_1^0 \cdot \textbf{n}_2}\\
    \ddt {\textbf{P}_\alpha}
    &= \intO{ \left(
        \rho_2  \textbf{w}_2^0 \textbf{w}_2^0 
        - \bm{\sigma}_2^0
    \right) }
    - \intS{ 
        \gamma \textbf{I}_{||}
    }
    + \intS{ \textbf{r}\bm{\sigma}_1^0\cdot \textbf{n}_2} 
    \label{eq:dt_P_alpha}\\
    \label{eq:dt_E_alpha}
    \ddt E_\alpha^\text{tot}
    &= 
    m_\alpha \textbf{u}_\alpha \cdot \textbf{g}
    +\textbf{u}_\alpha \cdot \intS{\bm{\sigma}_1^0 \cdot \textbf{n}_2}
    +\intS{\textbf{w}_1^0 \cdot \bm{\sigma}_1^0 \cdot  \textbf{n}_2} 
    - \intS{\textbf{q}_1^0 \cdot \textbf{n}_2} \\
\end{align}
where  $\intS{  \bm{\sigma}_1^0 \cdot \textbf{n}_2 }$ is the resultants of the hydrodynamic force and $\intS{ \textbf{q}_1^0 \cdot \textbf{n}_2 }$ is the resultants of the surface heat flux. 
The second term on the right hands side of the energy equation is the work produced by the mean force and the translational motion of the droplets, while $\intS{\textbf{w}_1^0 \cdot \bm{\sigma}_1^0 \cdot  \textbf{n}_2}$ is the work produced by the local forces and local motion of the fluid at the surface of the particle.
Since we integrated the energy over the particle's volume and its surface, we explicitly made appear the surface energy $\gamma s_\alpha$ within the derivative operator. 
Note that these equations does not explicitly account for inter-particle interactions. 
However, it is possible to include manually such forces by noticing that the surface external stress flux $\bm{\sigma}_1^0$ is the sum of hydrodynamic and particles-particles interaction forces, regardless it is pure contact forces from direct contact or a force mediated through the carrier fluid.
From this consideration it is possible to split every term involving the stress $\bm{\sigma}_1^0$ into two terms representing these contributions. 
Same comments can be made for the heat flux $\textbf{q}_1^0$. 
Although this distinction is important, for purpose of clearly we will stay general, and we will keep the fluxes $\bm{\sigma}_1^0$ and $\textbf{q}_1^0$ as such. 
We now discuss the second order moment of mass and first order moment of momentum conservation equations. 
In the following examples, we consider the same hypothesis as in thep previous section. 
Following \ref{eq:Q_n_definition} we define the second-order moment of mass and the first-order moment of momentum as respectively,
\begin{equation}
    \mathcal{M}_\alpha 
    = \intO{ \rho_2 \textbf{r} \textbf{r} }
    \;\;\;\text{and}\;\;\;
    \mathcal{P}_\alpha 
    = \intO{ \rho_2 \textbf{r} \textbf{u}_2^0 }.
    \label{eq:first_moment_of_momentum_def}
\end{equation}
Note that $\mathcal{M}_\alpha$ is analogous to the inertia tensor $\mathcal{I}_\alpha$ in solid mechanics, and they are related through the expression, $\mathcal{I}_\alpha = \text{tr}(\mathcal{M}_\alpha)\textbf{I} - \mathcal{M}_\alpha$.
At constant density the tensor $\mathcal{M}_\alpha$ describes the second moment of the volume distribution around the particle's center of mass.
In order to provide a clearer physical interpretation to the moment of momentum tensor, we decompose $\mathcal{P}_\alpha$ into two distinct part, namely,
$\mathcal{P}_\alpha = \mathcal{S}_\alpha+\mathcal{T}_\alpha$ where $\mathcal{S}_\alpha$ represents the symmetric part and $\mathcal{T}_\alpha$ is the antisymmetric part of $\mathcal{P}_\alpha$.
The tensors $\mathcal{S}_\alpha$ and $\mathcal{T}_\alpha$ correspond respectively to the stretching and angular momentum of the particle $\alpha$. 
The tensor $\mathcal{S}_\alpha$ quantifies how fast and in which direction the particle get elongated or flattened, in other worlds it represents the rate of deformation experienced by the particle.
The tensor $\mathcal{T}_\alpha$ is related to the angular momentum of the particle. 
In this study we use the pseudo vector $\bm{\mu}_\alpha = \intO{ \rho_2 \textbf{r} \times \textbf{u}_2^0 }$ to express this quantity. 
Indeed, both  $\mathcal{T}_\alpha$ and $\bm{\mu}_\alpha$ represent the angular momentum and are related through $(\bm{\mu}_\alpha)_i = \epsilon_{ijk} (\mathcal{P}_\alpha)_{jk}= \epsilon_{ijk} (\mathcal{T}_\alpha)_{jk}$, where $\epsilon$ is the third order alternating unit tensor. 
Lastly, we also introduce the scalar $\mathcal{D}_\alpha = \text{tr}(\mathcal{P}_\alpha) = \frac{1}{3}\int0{ \rho_2 \textbf{r} \cdot \textbf{u}_2^0 }.$, which quantifies the rate at which the particle is being compressed.

Injecting, $f_2 = \rho_2$ in the second-order moment equation derived in \ref{ap:Moments_equations} we obtain :
\begin{equation}
    \ddt {\mathcal{M}_\alpha}=2\mathcal{S}_\alpha. 
    \label{eq:dt_M_alpha}
\end{equation}
which is the general form of the second moment of mass conservation equation. 
From \ref{eq:dt_M_alpha} we deduce that the evolution of the distribution of mass of a particle is solely motivated by the stretching of momentum, denoted by $\mathcal{S}_\alpha$. 
This implies that the angular momentum (not to be confused with the angular velocity) plays no-role in the evolution of the second moment of the mass distribution. 
Note that if the particle has a constant $\mathcal{M}_\alpha$ under change of reference frame, such as for spherical particles where we can write $\mathcal{M}_\alpha= \frac{a^2 m_\alpha}{5} \textbf{I}$, then the stretching of momentum is null $\mathcal{S}_\alpha=0$.
This argument has no restriction on the internal particle motion. 
Additionally, applying the trace operator on both sides of \ref{eq:dt_M_alpha}, yields the interesting relation : $\ddt {\text{tr}(\mathcal{M}_\alpha)}=2\mathcal{D}_\alpha$.
Therefore, we can state that $\text{tr}(\mathcal{M}_\alpha) = \lambda^\alpha_1(t)+\lambda^\alpha_2(t)+\lambda^\alpha_3(t)$, with $\lambda_i^\alpha$ for $i=1,2,3$, being the eigenvalues of $\mathcal{M}_\alpha$.
For unreformable particles it is evident that the eigenvalues are not function of time, therefore $\ddt{ \text{tr}(\mathcal{M}_\alpha)}=0$.  
Consequently, $\mathcal{D}_\alpha$ has the notable property of being null whenever the particle shape remain constant, irrespective of the orientation.
The third invariant of this tensor can be shown to be related to the volume of the particle. 

Now, that we described the shape of the particle through its with the symmetric part of the moment of momentum we might need an equation for the moment of momentum. 
This equation is derived injecting $\mathcal{Q}_\alpha = \mathcal{P}_\alpha$ in \ref{eq:dt_Q_alpha_tot}, it reads, 
\begin{equation}
    \ddt {\mathcal{P}_\alpha}
    = \intO{ \left(
        \rho_2  \textbf{w}_2^0 \textbf{w}_2^0 
        - \bm{\sigma}_2^0
    \right) }
    - \intS{ 
        \gamma \textbf{I}_{||}
    }
    + \intS{ \textbf{r}\bm{\sigma}_1^0\cdot \textbf{n}_2} 
    \label{eq:dt_P_alpha}
\end{equation}
The last term on the right hands side of \ref{eq:dt_P_alpha} represents the first hydrodynamic moment of the force traction on the particle surface.
It is commonly  decomposed into a symmetric and an antisymmetric part defined as, 
\begin{align}
    \label{eq:M_decomposition}
    \mathscr{S}_{\alpha,ij}^*
    &= \frac{1}{2}  \intS{ \left[
        r_i(\sigma_{1,jk}^0 n_k)
        + (\sigma_{1,ik}^0 n_k)r_j
        \right]}
    %     - \frac{\delta_{ij}}{3}\int_{\Sigma_\alpha} \left[
    %         r_l(T_{lk}n_k)
    % \right]d\Sigma
    \\
    \mathscr{L}_{\alpha,ij}
    &= \frac{1}{2}  \intS{ \left[
        r_i(\sigma_{1,jk}^0 n_k)
        - (\sigma_{1,ik}^0 n_k)r_j
    \right]}, \nonumber
\end{align}
respectively. 
It will be shown in \ref{sec:averaged_eq} that $\mathscr{S}_\alpha$ is related to a quantity called the stresslet. 
We introduce the torque vector as $\textbf{t}_\alpha = \intS{ \textbf{r} \times (\bm{\sigma}_1\cdot \textbf{n}_2) }$ which is related to the skew symmetric part of the first moments $t_{\alpha,i} = \epsilon_{ikj} \mathscr{L}_{\alpha,jk}$. 
Each of the other terms appearing in \ref{eq:dt_P_alpha} is discussed in further detail in the following.
 

The conservation equation of the angular momentum $\bm{\mu}_\alpha$ is obtained by taking the double contracted product of \ref{eq:dt_P_alpha} with $\epsilon$, which gives the simple expression :
\begin{equation}
    \ddt\bm{\mu}_\alpha
    =  
    % \textbf{t}_\alpha.
    \intS{ \textbf{r} \times \bm{\sigma}_1^0\cdot \textbf{n}_2 }
    \label{eq:dt_mu_alpha}
\end{equation}
Note that every terms on the RHS of \ref{eq:dt_P_alpha} vanish due to their symmetric nature apart from the first hydrodynamic moment $\mathcal{M}_\alpha$.
Particularly, the surface tension terms do not appear in the angular momentum balance, which is consistent with the findings of \citet{hesla1993note}. 
As a consequence, the surface tension has no effect on the angular momentum regardless of the particle's shape. 
In the literature it is common to include the torque due to inter-particular interactions in the angular momentum balance, as it is done in \citet{jackson1997locally} and \citet{zhang1997momentum}.
Therefore, we remind the reader that $\bm{\sigma}_0^1$ contain interaction forces thus $\textbf{t}_\alpha$ includes particles-particles interactions.


Taking the symmetric part of \ref{eq:dt_P_alpha}, yield an equation for the stretching of momentum, which can be written as,
\begin{equation}    
    \ddt{\mathcal{S}_\alpha}
    =  \intO{
        \rho_2\textbf{w}_2^0 \textbf{w}_2^0
        - \bm{\sigma}_2^0}
        - \sigma\intS{\textbf{I}-\textbf{nn}}
        + \frac{1}{2}\intS{(\textbf{r}\bm\sigma_2^0+ \bm\sigma_2^0\textbf{r})\cdot \textbf{n}}
    \label{eq:dt_S_alpha}
\end{equation}
\tb{introduce the second order derivative here ? }
One might immediately recognize that this equation is in facts an extension to Batchelor’s famous result, 
\begin{equation*}
    \intO{\bm{\sigma}_2^0}
    + \intO{\bm{\sigma}_I^0}
    = \frac{1}{2}\intS{(\textbf{r}\bm\sigma_2^0+ \bm\sigma_2^0\textbf{r})\cdot \textbf{n}}
\end{equation*}
% \tb{it is also an extension to dolata recent results for teh first and second moment equation }
which has been used widely in stokes flow theory to express the unknown internal stress within solid particles in terms of surface integral, i.e. the stress let $\intS{(\textbf{r}\bm\sigma_2^0+ \bm\sigma_2^0\textbf{r})\cdot \textbf{n}}$.
This relation is the main tools used to express the bulk stress of a suspension, it eventually leads to the computation of the famous Einstein equivalent viscosity upon having an analytical formula for $\intS{(\textbf{r}\bm\sigma_2^0+ \bm\sigma_2^0\textbf{r})\cdot \textbf{n}}$. 
Therefore, the significant aspect of \ref{eq:dt_S_alpha} is that it can be interpreted as a generalized equation for the integrated stress tensor within the volume of the particle.
This will become particularly relevant when determining the total stress of an inertial suspension as it will be mentioned in \ref{sec:averaged_eq}.
On the right hands side of \ref{eq:dt_S_alpha} we can identify several terms: 
the internal kinetic energy $\intO{\rho_2\textbf{w}_2^0\textbf{w}_2^0 }$; 
the integral of the particle internal stress $\intO{ \bm{\sigma}_2^0
 }$; 
the integral of the surface stress $\intS{ \sigma (\textbf{I}- \textbf{nn}) }$; 
and the stresslet tensor, $\intS{(\textbf{r}\bm\sigma_2^0+ \bm\sigma_2^0\textbf{r})\cdot \textbf{n}}$ introduced earlier.
Based on \ref{eq:dt_M_alpha} we can infer that the evolution of $\mathcal{M}_\alpha$ is driven by the internal kinetic energy and the stresslet.
However, it is being counteracted by surface tension forces and internal stresses which tend to oppose the deformation of the particle. 
Therefore, if the surface tension forces play no role in the linear and angular momentum equation, it does impact the stretching of momentum $\mathcal{S}_\alpha$.
As a consequence, the surface tension force impact the hydrodynamic behavior of a particle solely through its action on $\mathcal{S}_\alpha$, which is related to the shape of a particle through \ref{eq:dt_M_alpha}.
As remarked by \citet{batchelor1970stress}, since the surface tension force oppose the deformation of a particle, it can be understood as an elastic force. 
Which, as it will be shown in \ref{sec:averaged_eq} has a role on the bulk stress of the suspension. 
Additionally, note that \ref{eq:dt_S_alpha} can be seen as a formula to reformulate the integral of the internal stress $\pOavg{\bm{\sigma}}$.
Equally, in \ref{ap:moment_derivative} we show how to derive the higher order moment of momentum equations, which can also be viewed as formulas for the higher moments of the internal particle stress. 
It is interesting to mention that in a recent study of \citet{dolata2021faxen} they use energy method and recover the first two moments of momentum equations hidden into another but equivalent form, valid in the stokes flow regime. 


Lastly, by taking the trace of \ref{eq:dt_Q_alpha_tot}, directly yields the scalar equation :
\begin{equation}
    \ddt {\mathcal{D}_\alpha}
    = \intO{ \left(
        \rho_2 \textbf{w}_2^0 \cdot \textbf{w}_2^0
        - \bm{\sigma}_2^0 : \textbf{I}
        \right) }
        - 2 s_\alpha \gamma
        + \text{tr}(\textbf{M}_\alpha)
    \label{eq:dt_D_alpha}
\end{equation}
which correspond to the isotropic work balance within the particle's volume and surface. 
As a matter of fact, the rate of compression of a particle, denoted by the scalar $\mathcal{D}_\alpha$ evolves according to : 
the internal kinetic energy, $\intO{\rho_2 \textbf{w}_2^0 \cdot \textbf{w}_2^0 }$;
the trace of the integral of the hydrodynamic stresses, $\intO{ \text{tr}(\bm{\sigma}_2^0)}$; 
the surface energy $\intS{ \gamma }$; 
and the trace of the hydrodynamic first moment, $\text{tr}(\textbf{M}_\alpha)$.
To provide a concrete insight of the physical implication of the above equation, 
% we consider the example of spherical bubbles with time dependent radius $a_\alpha(t)$ and show that from the scalar moment of momentum equation one can recover the Rayleigh-Lamb-Plesset equation. 
% Indeed, in this situation, the internal velocity can be expressed as, $\textbf{w}_2^0 = \frac{d a_\alpha(t)}{dt} \frac{\textbf{r}}{a_\alpha(t)}$, which makes the scalar moment of momentum equation as, 
% \begin{equation*}
%     \frac{3}{5}\rho_2 a_\alpha(t)\frac{d^2 a_\alpha(t)}{dt^2}
%     = \intO{(\bm{\sigma}_2^0)_{kk}}
%     - a_\alpha(t)\intS{\textbf{n}\cdot \bm{\sigma}_2^0 \cdot \textbf{n}}
%     - 2 \gamma s_\alpha
% \end{equation*}
% Upon making use of the constitutive law $\bm{\sigma}_k^0 = -p_k \textbf{I} + \mu_k (\grad \textbf{u}_k^0 + (\grad \textbf{u}_k^0)^T) + \zeta_k \div \textbf{u}$ which we will consider true in both phases except that for the carrier fluid $\zeta_1=0$, one obtain, 
\begin{equation*}
    (\rho_1 + \frac{1}{5}\rho_2)a_\alpha\frac{d^2 a_\alpha}{dt^2}
    + \frac{3}{2}\rho_1\left(\frac{d a_\alpha}{dt}\right)^2
    + (4\mu_1 + 3\zeta_2) \frac{1}{a_\alpha}\frac{d a_\alpha}{dt}
    = \smallavg{p_1}{\Sigma_\alpha} - \smallavg{\sigma}{\Sigma_\alpha} - \frac{2\gamma}{a_\alpha}
\end{equation*}
where,  $\smallavg{p_1}{\Sigma_\alpha}$ and  $\smallavg{\sigma}{\Sigma_\alpha}$ are the surface-averaged external pressure and surface tension coefficient respectively, and $\smallavg{p_2}{\Omega_\alpha}$ represent the volume-averaged internal pressures.
We indeed recovered the Rayleigh-Lamb-Plesset equation. 
\tb{Re do the derivation}
 we examine a single spherical fluid particle of radius $a$, immersed in a steady flow such that $\textbf{u}^0 = 0$ on $\Omega$. 
In this situation, the stress tensor can be written as $\bm{\sigma}_k = \textbf{I} p_k^0$ for $k = 1, 2$ where $p_k^0$ is the local pressure in the phase $k$. 
Therefore, applying these considerations to \ref{eq:dt_D_alpha} yields the relation, 
\begin{equation*}
    \smallavg{p_2^0}{\Omega_\alpha} 
    - \smallavg{p_1^0}{\Sigma_\alpha}
    =
    \frac{2}{a} s_\alpha \gamma
    \label{eq:Laplace_law}
\end{equation*}
Under this form it is evident that \ref{eq:Laplace_law} represent the well-known Laplace's Law. 
Additionally, in light of \ref{eq:dt_M_alpha}, the scalar moment of momentum equation can be interpreted as an equilibrium equation for the particle internal mass distribution, or moment of inertia, since $\ddt{\text{tr}(\mathcal{M}_\alpha)} = 2 \mathcal{D}_\alpha$. 
From this argument and \ref{eq:dt_D_alpha}, one is able to derive the \textit{Rayleigh-Plesset} equation by considering compressible spherical particles with a non-constant particles radius $a_\alpha(t)$ and assuming an internal velocity written as, $\textbf{w}^0_2 = \frac{d a_\alpha(t)}{dt}  \frac{\textbf{r}}{a_\alpha(t)}$. 
% A demonstration of this derivation can be found in the class of \tb{CITER LE COURS DE Lhuillier}. 
By the mean of kinetic theory \citet{zhang1994averaged} derived the \textit{Rayleigh-Plesset} equation under an equivalent but averaged form.
What we demonstrated is that the scalar moment of momentum balance, i.e. \ref{eq:dt_D_alpha} quantify any isotropic dynamical related to a particle. 

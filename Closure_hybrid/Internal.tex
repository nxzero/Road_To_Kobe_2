
\subsection{Influence of the droplets internal motion on the pseudo turbulent energy equations}

The Hill vortex shape that the internal velocity fields adopts in the stokes flow regime turns out to be valid for potential flow too. 
It is therefore reasonable to assume that the drops internal velocity fields can be described by a Hill vortex solution in the transient regime as long as the particle remain spherical. 
In the following we neglect other type of motion as our objective is to show the importance of such internal motion in the pseudo turbulent energy equations. 
The relative inner motion inside a spherical drop translating inside a carrier fluid can be written \ref{ap:Translating_sphere},
\begin{equation}
    \textbf{w}_2^0(\textbf{x}_\alpha + \textbf{r}) = 
    \frac{1}{2}\frac{1}{1+\lambda} \textbf{u}_{\alpha f}\cdot \left[
        (2\lambda+3)\textbf{I}
        - 2 \textbf{r}\cdot \textbf{r} \textbf{I}/a^2
        + \textbf{rr}/a^2
    \right] - \textbf{u}_{\alpha f}\\
    \label{eq:wa_def}
\end{equation}
with $\textbf{u}_{\alpha f} = \textbf{u}_\alpha - \textbf{u}_1$, where we evaluated the averaged fluid velocity at the location of the instantaneous position of the particle center of mass. 

Before, 
From the analytical expression \ref{ap:Translating_sphere}, the averaged internal kinetic energy and rate of dissipation of the particle phase can be directly computed, and yields, 
\begin{align*}
    n_p W_p 
    % =
    % \pOavg{\rho_2 \textbf{w}_2^0 \cdot \textbf{w}_2^0/2} 
    % =\frac{n_pm_p}{14 \left(\lambda +1\right)^2} \left[
    %     \frac{1}{2}(\textbf{u}_p - \textbf{u}_1)\cdot (\textbf{u}_p - \textbf{u}_1)
    %     +k_p
    % \right]
    =\frac{\phi_2\rho_2}{14 \left(\lambda +1\right)^2} 
    \left[
        \textbf{u}_{p f}\cdot \textbf{u}_{p f}/2+k_p
    \right],\\
    \pOavg{\bm{\sigma}_2^0:\grad \textbf{u}_2^0}
    = \frac{6 \mu_2 \phi_2}{a^2(1+\lambda)^2}
    [\textbf{u}_{p f}\cdot \textbf{u}_{p f} /2  
    +  k_p],
\end{align*}
It is interesting to remark that the inner energy of the droplet, and the heat transfer term both are proportional to $|\textbf{u}_{p f}|^2 /2 +  k_p$.
Making the total kinetic energy of the dispersed phase entirely determined the particle momentum and $k_p$.  
This, term is the contribution of the hill's vortex flows inside the particles to the generation of heat in the particle phase. 
It is reasonable to say that collisions and rotation of the particles might have an impact on the droplets internal motion and thus on the dissipation rate and internal energy. 
In fact particles collision will affect every exchange terms derived here. 
Nevertheless, due to the complexity of the problem the dilute limit is still assumed. 

Now let's turn our attention to the fluid phase pseudo turbulent equation. 
In  \ref{eq:dt_hybrid_k1} the exchange terms $\pSavg{\textbf{w}_2^0 \cdot \bm{\sigma}_1^0\cdot\textbf{n}_2}$ and  $\pSavg{\textbf{rw}_2^0 \cdot \bm{\sigma}_1^0\cdot\textbf{n}_2}$ involve explicitly the inner velocity of the drops.
Under the assumption of \ref{eq:wa_def} we are able to reformulate these terms keeping the external stress considered unknown.
% Therefore, we only suppose the form of the relative internal velocity evaluated at the surface points which yields : $\textbf{w}_2^0|_\Sigma =
% \frac{1}{2}\frac{1}{1+\lambda} \textbf{u}_{\alpha f}\cdot \left[
%     (2\lambda+1)\textbf{I}
%     + \textbf{n}_2\textbf{n}_2
% \right] - \textbf{u}_{\alpha f}$. 
% Upon substituting this expression in the exchange term one obtain,
\begin{multline*}
    \pSavg{\textbf{w}_2^0 \cdot \bm{\sigma}_1^0\cdot\textbf{n}_2}
    =  
    \left(\frac{2\lambda + 1}{2\lambda+2} - 1\right)\left[
        \textbf{u}_{p f}\cdot\pSavg{ \bm{\sigma}_1^0\cdot\textbf{n}_2}
        + \pavg{\textbf{u}_{\alpha}' \cdot \intS{ \bm{\sigma}_1^0\cdot\textbf{n}_2} }
    \right]
    + \\
    + \frac{1}{a^2}\frac{1}{2\lambda+2}\left[
        \textbf{u}_{p f} \cdot\pSavg{\textbf{n}_2\textbf{n}_2\cdot \bm{\sigma}_1^0\cdot\textbf{n}_2}
        +
        \pavg{\textbf{u}_{\alpha}' \cdot \intS{\textbf{n}_2\textbf{n}_2\cdot \bm{\sigma}_1^0\cdot\textbf{n}_2}}
    \right]
\end{multline*}
\tb{verifier le 2nd moment en stokes}
% Note that for solid particles all the terms completely vanish.
% This come from the fact that we did not consider rotational motion as in the previous example. 
Injecting the expression of $\textbf{w}_2^0$ in to the pseudo turbulent energy equation, gives,
\begin{multline}
    \label{eq:dt_hybrid_k1_new}
    \pddt (\phi_1\rho_1k_1)  
    + \div (
        \phi_1\rho_1k_1\textbf{u}_1
        + \textbf{q}_1^\text{k} 
        )
    = 
    - \avg{\chi_1\bm{\sigma}_1^0 : \grad \textbf{u}_1^0}
    - \bm{\sigma}_1^\text{eq} : \grad \textbf{u}_1\\
    - \frac{2\lambda +1}{2\lambda +2} 
    \left[
        \textbf{u}_{pf}
        \cdot \pSavg{\bm{\sigma}_1^0 \cdot \textbf{n}_2}
        + \pavg{ \textbf{u}_\alpha' \cdot \intS{  \bm{\sigma}_1^0 \cdot \textbf{n}_2}}
    \right]\\
    - \frac{1}{2\lambda+2}\left[
        \textbf{u}_{p f} \cdot\pSavg{\textbf{n}_2\textbf{n}_2\cdot \bm{\sigma}_1^0\cdot\textbf{n}_2}
        +
        \pavg{\textbf{u}_{\alpha}' \cdot \intS{\textbf{n}_2\textbf{n}_2\cdot \bm{\sigma}_1^0\cdot\textbf{n}_2}}
    \right]
\end{multline}
\begin{multline*}
    \textbf{q}_1^\text{k}
    = \rho_1 \avg{\chi_1 \textbf{u}_1' k_1} 
    - \avg{\chi_1 \textbf{u}_1' \cdot \bm{\sigma}_1^0}
    - \frac{2\lambda +1}{2\lambda +2} 
    \left[
        \textbf{u}_{pf}
        \cdot \pSavg{\textbf{r}\bm{\sigma}_1^0 \cdot \textbf{n}_2}
        + \pavg{ \textbf{u}_\alpha' \cdot \intS{ \textbf{r} \bm{\sigma}_1^0 \cdot \textbf{n}_2}}
    \right]\\
    - \frac{1}{2\lambda+2}\left[
        \textbf{u}_{p f} \cdot\pSavg{\textbf{n}_2\textbf{n}_2\cdot \textbf{r}\bm{\sigma}_1^0\cdot\textbf{n}_2}
        +
        \pavg{\textbf{u}_{\alpha}' \cdot \intS{\textbf{n}_2\textbf{n}_2\cdot \textbf{r}\bm{\sigma}_1^0\cdot\textbf{n}_2}}
    \right]
\end{multline*}
If we compare this equation to the more general expression from \ref{eq:dt_hybrid_k1} we can see that the consideration of hill's vortexes add the coefficient $\frac{\lambda +\frac{1}{2}}{\lambda+1}$ in front of the drag force velocity correlation terms, and includes two terms related to the second higher moments surface traction. 
Besides, the first moments of surface traction forces appearing in the diffusive equivalent flux $\textbf{q}_1^k$ are also subject to these comments.  
By scaling arguments, it is fair to say that the term  $\textbf{u}_{pf} \cdot \pSavg{\textbf{r}\bm{\sigma}_1^0 \cdot \textbf{n}_2}$ is the greatest among all the terms on the RHS of \ref{eq:dt_hybrid_k1_new}. 
Consequently, the consideration of hill's vortex end up to add a coefficient in front of this exchange term which varies from $1$ to $1/2$ for respectively, solid particles and bubbles.  

As a matter of fact the hill's vortex have a very significant impact regarding the magnitude of the pseudo turbulent exchange terms when one is considering bubbly flow. 
The physical explanation of the decrease of the coefficient in front of the exchange terms for bubbles, can be due to the facts that the fluid slip on the bubbles or droplet's surface induce less work exchange than if the fluid followed the particle's surface as it is the case for solid particles. 

If one wish to pursue in a more general manner he might consider the effect the particle internal velocity under a mean shearing motion. 
Indeed, as suggested by the form of the internal velocity fields derived in \ref{ap:Translating_sphere} for droplet in linear shear flow we might expect that, 
$\pSavg{\textbf{w}_2^0 \cdot \bm{\sigma}_1^0\cdot\textbf{n}_2} \sim (\grad \textbf{u}_1 + (\grad \textbf{u}_1)^T) \cdot \pSavg{ \textbf{r} \bm{\sigma}_1^0\cdot\textbf{n}_2}$. 
In another context, if we consider rigid particle we might find the more common results, $\pSavg{\textbf{w}_2^0 \cdot \bm{\sigma}_1^0\cdot\textbf{n}_2} \sim \cdot \pavg{\bm\Omega_\alpha\cdot \intS{ \textbf{r} \bm{\sigma}_1^0\cdot\textbf{n}_2}}$. 
Once again the closure of this term is very problem dependent but is of a great importance and probably not negligible. 
This is what make the pseudo turbulent equation pretty hard. 
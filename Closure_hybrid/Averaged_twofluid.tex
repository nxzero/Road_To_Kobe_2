
\subsection{Ensemble average}
In this study we use the statistical approach to derive the averaged equations of conservation. 
Specifically, we use the method described in \citet{zhang2021ensemble} which extended the ensemble average definition of \citet{batchelor1972sedimentation}. 
In the following we recall some properties of the ensemble average operator. 

Let, $P(\FF)$ be the probability density function that describe the probability of finding the flow in the configuration $\FF$, were $\FF = (\lambda_1,\lambda_2,\lambda_3,\ldots)$ is a set of all the parameters describing the initial flow configuration of both phases.
% \footnote{We assume that the flow can be described by a finite number of parameters related to both phase.}. 
Then, we define $d\mathscr{P} = P(\FF)d\FF$ as the probable number of flows in the incremental region of the particles' phase space $d\FF$ around $\FF$. 
It follows from this definition, that the ensemble average of an arbitrary local property $f^0(\textbf{x},t;\FF)$ defined on the whole space $\Omega$, is,
\begin{equation}
    f(\textbf{x},t)
    = \avg{f^0}(\textbf{x},t)
    =\int f^0(\textbf{x},t;\FF) d\mathscr{P}. 
    \label{eq:avg}
\end{equation}  
Note that we dropped the super script $^0$ on $f$ to indicate that this an averaged quantity. 
% This definition can be applied to Lagrangian properties as well by using the previous formulation, namely $\pavg{q_\alpha}(\textbf{x},t) = \int \delta(\textbf{x} - \textbf{x}_\alpha) q_\alpha(\FF) d\mathscr{P}.$
It is interesting to mention some mathematical properties of the ensemble average operators. 
For two arbitrary Eulerian fields $f$ and $h$ we have,
\begin{align}
    &\avg{f^0+h^0} = f+h, 
    &\avg{\avg{f^0}h^0} = gh, \nonumber \\
    &\avg{\pddt f^0} 
    = \pddt f, 
    &\avg{\grad f^0}
    = \grad f. 
    \label{eq:avg_properties}
\end{align}
The two first relations are called the Reynolds' rules, the $3^{th}$ one is the Leibniz' 
rule and the last one, the Gauss' rule \citep{drew1983mathematical}.
Additionally, for any phase quantity defined in $\Omega_k$ we introduce the definition, 
\begin{equation}
    \phi_k f_k (\textbf{x},t) = \avg{\chi_k f_k^0}
    \label{eq:1_avg}
\end{equation}
where, $\phi_k(\textbf{x},t) = \avg{\chi_k}$ is the volume fraction of the phase $k$
and $f_k(\textbf{x},t)$ the average of the field $f_k^0$ conditioned on the presence of the phase $k$ at $\textbf{x}$ and time $t$.
Equally, for interfaces quantities we have 
\begin{equation}
    \phi_I f_I (\textbf{x},t) = \avg{\delta_I f_I^0}
\end{equation}
with $\phi_I = \avg{\delta_I}$ the interfacial indicator function and $f_I$ the average of $f^0(\textbf{x},t)$ conditioned on the presence of an interface at $\textbf{x}$ and time $t$. 
The fluctuation of a phase-averaged quantity around its mean is defined by,
\begin{equation}
    % q_\alpha' = q_\alpha - q_p
    % \;\;\;\;\;\;\text{and}
    % \;\;\;\;\;\;
    f_k' = f_k^0 - f_k.
    \label{eq:def_fluctu}
\end{equation}
These definitions lead to the following properties, $\avg{\chi_k f'_k} = 0$. 
For example, the product $\avg{\chi_k f^0_kg^0_k}$ can be decomposed as $\avg{\chi_k f_k^0g_k^0}=\phi_k f_kg_k + \avg{\chi_k f'_kg'_k}$. 
This decomposition will play a crucial role in the upcoming section. 
The ensemble average method is one among others, such as the volume average method\citep{jackson1997locally} or the time average
used in thermodynamic fields \citep{ishii2010thermo}.
All of these averaging technics remains equivalent \citep{jackson1997locally,zhang1997momentum}. 

\subsection{The averaged conservation equations}
Now we proceed to the derivation of the averaged equations by first introducing the general formulation of the two-fluid averaged equations, then we expose the mass, momentum and energy equation for the volumetric quantities, and finally we present the averaged jump condition. 

\subsubsection{Generic formulation}
Applying the ensemble average (\ref{eq:avg}) on \ref{eq:dt_chi_k_f_k} and \ref{eq:dt_delta_I_f_I} and considering the properties from \ref{eq:avg_properties} yields the general form of the averaged equations of multiphase flows, namely,
\begin{align}
    \pddt \avg{\chi_k f_k^0}
    +\div \avg{\chi_k f_k^0 \textbf{u}_k^0 - \chi_k \mathbf{\Phi}_k^0}
    &= 
    \avg{\chi_k s_k^0}
    + \avg{\delta_I\left[
        \mathbf{\Phi}_k^0
        + f_k^0
        \left(
            \textbf{u}_I^0
            - \textbf{u}_k^0
        \right)
    \right]
    \cdot \textbf{n}_k} ,
    \label{eq:avg_dt_chi_f}\\
    \pddt \avg{\delta_If_I^0}
    +\div \avg{\delta_I f_I^0 \textbf{u}_I^0-\delta_I \mathbf{\Phi}_{I||}^0 }
    &= 
    \avg{\delta_Is_I^0} 
    - \avg{\delta_I \Jump{
    f_k^0 (\textbf{u}_I^0 - \textbf{u}_k^0)
    + \mathbf{\Phi}_k^0
    } }.
    \label{eq:avg_dt_delta_f}
\end{align}
Together, \ref{eq:avg_dt_chi_f} for $k=1,2$  and \ref{eq:avg_dt_delta_f}, form the \textit{two-fluid} formulation of averaged multiphase flows problem. 
One can derive the so-called averaged \textit{single-fluid} formulation by applying the average on \ref{eq:dt_f} or summing the above equations, in both cases it gives, 
\begin{equation}
    \pddt f
    + \div (f \textbf{u} + \avg{f'\textbf{u}'} - \mathbf{\Phi})
    = 
    s.
    \label{eq:avg_dt_f}
\end{equation}
With these definitions it must be understood that $f = \sum_k \phi_k f_k + f_I \phi_I$, $\bm\Phi = \sum_k \phi_k \bm\Phi_k + \bm\Phi_I \phi_I$ and so on. 


\subsubsection{Volume equation}
Using the generic formulation \ref{eq:avg_dt_chi_f} and the local expression of the mass, momentum and total energy equation, i.e. : \ref{eq:dt_rho},\ref{eq:dt_rhou_k} and \ref{eq:dt_rhoE_k} we easily find the averaged form of the mass, momentum and total energy equation.
They read, 
\begin{align}
    \label{eq:dt_avg_rho}
    \pddt (\phi_k \rho_k)  
    + \div (
        \phi_k \rho_k\textbf{u}_k
    )
    &= 
    0\\
    \label{eq:dt_avg_rhou_k}
    \pddt (\phi_k \rho_k\textbf{u}_k)  
    + \div (
        \phi_k \rho_k\textbf{u}_k\textbf{u}_k
        + \bm{\sigma}_k^\text{eq}
    )
    &= 
    \phi_k \rho_k \textbf{g} 
    +  \avg{\delta_I \bm{\sigma}_k^0 \cdot \textbf{n}_k}\\
    \label{eq:dt_avg_rhoE_k}
    \pddt (\phi_k\rho_kE_k)  
    + \div (
        \phi_k\rho_kE_k\textbf{u}_k
        + \bm{q}_k^\text{eq}
        + \textbf{u}_k \cdot \bm{\sigma}_k^\text{eq}
        % - \textbf{u}_k^0 \cdot \bm{\sigma}_k^0 
        % + \textbf{q}_k^0
        )
    &= 
    \phi_k \rho_k\textbf{u}_k \cdot \textbf{g} 
    + \avg{\delta_I (\textbf{u}_k^0 \cdot \bm{\sigma}_k^0 - \textbf{q}_k^0)\cdot \textbf{n}_k}
\end{align} 
with the equivalent stress and heat flux are defined as, 
\begin{align*}
    &\bm{\sigma}_k^\text{eq}
    = 
     \rho_k\avg{\chi_k \textbf{u}_k'\textbf{u}_k'}
      - \phi_k \bm{\sigma}_k,%- n_p \textbf{M}_p
    &\textbf{q}_k^\text{eq}
    =\textbf{q}_k^\text{e} +\textbf{q}_k^\text{k},  \\
    &\textbf{q}_k^\text{e}
    = \rho_k \avg{\chi_k\textbf{u}_k' e_k'} 
    + \phi_k\textbf{q}_k,
    &\textbf{q}_k^\text{k}
    = \rho_k \avg{\chi_k \textbf{u}_k' k_k} 
    - \avg{\chi_k \textbf{u}_k' \cdot \bm{\sigma}_k^0}.
\end{align*}
The main differences between these equations and their microscale counterpart, is that : (1) we have introduced a pre factor $\phi_k$ in front of most of the terms
(2) an additional stress is present, that is the covariance between the quantity to be conserved and the velocity. 
(3) A new source term appear on the RHS accounting for the exchange across the phases. 

The terms $\avg{\chi_k \textbf{u}_k'\textbf{u}_k'}$ will be referred as the Reynolds stress or pseudo turbulent stress. 
It has a fundamental importance in the multiphase flow problem and as we see now its trace is related to the mean kinetic energy. 
Indeed, the phase averaged total energy can be further decompose into three energy components, that is,  
\begin{align}
    E_k = e_k + k_k + u_k^2/2
    \label{eq:E_def}
\end{align}
where $k_1$ is the pseudo-turbulent kinetic energy defined as, $\phi_k k_k = \frac{1}{2}\avg{\chi_k \textbf{u}_k'\cdot \textbf{u}_k'}$. 
Each of the components of the total energy represent the averaged kinetic energy at different scales. 
From molecular kinetic energy which is $e_k$, to the macroscopic scale with the kinematics   energy $u_k^2$, and in between we find the pseudo turbulent energy $k_k$. 
To fully describe the averaged total energy one must add a supplementary equation either for $k_k$ or $e_k$. 
In fact an equation for each of these components can be derived using the averaged momentum and total energy equation presented above. 
The kinetic energy, pseudo turbulent and internal averaged equations read, 
\begin{align}
    \label{eq:dt_avg_uk2}
    &\pddt (\phi_k \rho_ku_k^2/2)  
    + \div (
        \phi_k \rho_k\textbf{u}_ku_k^2/2
        + \textbf{u}_k \cdot \bm{\sigma}_k^\text{eq}
    )
    = 
    \bm{\sigma}_k^\text{eq} : \grad \textbf{u}_k
    + \phi_k \rho_k \textbf{u}_k\cdot \textbf{g} 
    +  \textbf{u}_k\cdot \avg{\delta_I \bm{\sigma}_k^0 \cdot \textbf{n}_k},\\
    \label{eq:dt_avg_kk}
    &\pddt (\phi_k\rho_kk_k)  
    + \div (
        \phi_k\rho_kk_k\textbf{u}_k
        + \textbf{q}_k^\text{k} 
        )
    = 
    - \avg{\chi_k\bm{\sigma}_k^0 : \grad \textbf{u}_k^0}
    - \bm{\sigma}_k^\text{eq} : \grad \textbf{u}_k
    + \avg{\delta_I \textbf{u}_k' \cdot \bm{\sigma}_k^0 \cdot \textbf{n}_k},\\
    \label{eq:dt_avg_ek}
    &\pddt (\phi_k\rho_ke_k)  
    + \div (
        \phi_k \rho_ke_k\textbf{u}_k
        +
        \textbf{q}_k^\text{e} 
        )
    = 
    \avg{\chi_k\bm{\sigma}_k^0 : \grad \textbf{u}_k^0}
    - \avg{\delta_I \textbf{q}_k^0 \cdot \textbf{n}_k},
\end{align}
respectively. 
This derivation is in agreement with \citet{morel2015mathematical}. 
Under this form the energy transfer across scale is clear. 
Indeed, the term $\bm{\sigma}_k^\text{eq} : \grad \textbf{u}_k$ act as a sink term in \ref{eq:dt_avg_uk2} and a source term in \ref{eq:dt_avg_kk}, while the averaged diffusive terms $\avg{\chi_k\bm{\sigma}_k^0 : \grad \textbf{u}_k^0}$ is a sink in \ref{eq:dt_avg_kk} and a source in \ref{eq:dt_avg_ek}. 
To determine the total energy only two of the four energy equations must be solved. 
In practice, it is useful to solve one equation for $k_1$ since it is related to the Reynolds stress appearing in the momentum equation through $\bm{\sigma}^\text{eq}_k$, with $\avg{\chi_k \textbf{u}_k^0 \textbf{u}_k^0}: \textbf{I} = 2 k_k$ and another for $e_1$ while $u_k^2$ is determinate with the averaged momentum equation. 


\subsubsection{Averaged surface equations}
In light of \ref{eq:avg_dt_chi_f} and \ref{eq:avg_dt_delta_f} the volume averaged equations must be completed by an averaged surface transport equation.  
To this purpose we multiply \ref{eq:dt_rho_I}, \ref{eq:surface_tension}, \ref{eq:dt_rhoIe_I} and \ref{eq:dt_rhoI_uI2} by $\delta_I$ and apply the average operator.
By considering the topological equations it eventually gives us the mass, momentum, and total energy averaged conservation equation, 
\begin{align}
    \label{eq:dt_avg_rho_I}
    \avg{\delta_I \Jump{\rho_k (\textbf{u}_I - \textbf{u}_k^0)}}
    = 0,\\
    \label{eq:dt_avg_uI}
    \avg{\delta_I \Jump{\bm{\sigma}^0_k}}
    = - \div \avg{\delta_I (\textbf{I} - \textbf{nn}) \gamma},\\
    \label{eq:dt_avg_gamma}
    \avg{\delta_I \Jump{\textbf{u}_k^0 \cdot \bm{\sigma}^0_k - \textbf{q}^0_k}}
    = - \pddt \avg{\delta_I \gamma}
    - \div \avg{\delta_I \gamma (\textbf{u}_{I}^0 \cdot \textbf{n})\textbf{n} },
\end{align}
respectively. 
The first equation represents the condition of no mean mass transfer between phase.
The second equation represents the contribution of the surface tension stresses to the bulk momentum equation.
And the last equation represents the amount of energy stoked in the surface. 

\subsection{Some comments on the two-fluid model}

At the coarse-grained level, the two-fluid model is made of four main equations
(one mass, one momentum and two energy equations) for each of the two phases
(or the two components) of the mixture. 
These equations are completed with the four averaged jump condition.  
To be operational that 8-equations
model must be presented in closed form, i.e. it must involve not more (and
not less) than eight unknowns. 
We will not insist on that closure issue and
instead will focus on some weaknesses of the two-fluid model concerning the
description of the dispersed phase. 
At first, it must be noted that the two-fluid model does not really distinguish between the two phases, as witnessed by the
\textit{symmetry} $k = 1$ and $2$ of the equations presented above. That symmetry
is not physically tenable and some questions arise : is the particulate phase
able to sustain a heat flux $\textbf{q}_2$ throughout the suspension without collisions or permanent contacts between the particles ? Is the stress $\bm{\sigma}_2$ and surface tension stress $\bm{\sigma}_I$ really able to have a role in the momentum balance of the particles ? 
One can also wonder why the two-fluid model has no consideration for the angular momentum of the particles ? 
For all these reasons we will develop in the following section a kinetic model
specially devoted to the particulate phase of the mixture.

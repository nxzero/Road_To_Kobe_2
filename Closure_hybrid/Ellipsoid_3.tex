\section{
    Finite inertial effect on the Rheology of deformable droplets' emulsion.
    % phase relative velocity on the rheology of mono-disperse dilute emulsion of deformable droplets
    }

In this section we make use of the \textit{hybrid model} to demonstrate how to derive the minimal set of equation intended to model deformable droplets in a buoyant emulsion. 

Preceding studies have intended to model the rheology of dilute emulsion of neutrally buoyant deformable droplets, in stokes regime \citep{goddard1967nonlinear,lhuillier1987phenomenology,maffettone1998equation}.
More recently, researcher have been trying to include inertial effect in these models \citet{raja2010inertial,mwasame2018macroscopic}. 
However, these studies have been conducted in the objective of determining the equivalent shear stress of a neutrally buoyant suspension, i.e. they study the mean stress $\bm{\sigma}$ in terms of the imposed flow gradient $\textbf{e}$ for a given microstructure. 
Therefore, to the knowledge of the author all preceding study disregards the impact of relative motion on the suspension rheology. 

In this application, we focus on the Rheological behavior of an inertial emulsion of spheroidal deformable rising droplets with a finite phase relative velocity. 
Thus, we seek for an expression of the bulk stress $\bm{\sigma}$, in terms of the mean relative motions $\textbf{u}_{pf}$. 
As the purpose is to point out the effect of the translational motions on the stress, we consider that there is no mean shear within the emulsion, i.e. $\textbf{e} = 0$. 
Nevertheless, we want to be clear on the facts that of mean shear are present its contribution to the stress is certainly non-negligible. 

\subsubsection*{Determination of the shape and strain based on \citet{taylor1964deformation} theoretical results}

\begin{itemize}
    \item In the capillary dominant regime stresslet = shape and this is it 
    \item Iso-viscous low density bubbles (that might not existe) permitt to neglect these
    \item (1) In the prvious section we proposed an equation for shape and rate of strain of the particles
    \item (2) In taylor they provide direct closure for the shape in steady state regime, no need for an equation anymore
    \item (3) considering this relation we can in fact re formulate the  bulk stress in term of this shape direct closure
    \item (4) Also the fluid phase stress can be written as ww -p etc... In any case we now that the functional form of the stress will be proportional to $u u $ 
    \item (5) In inertial regime the stresslet will also be present 
    \item (6) Major  conclusion is that the stesslet is also function of uu and that is far from being negligible. 
\end{itemize}




In \citet{taylor1964deformation} they derive the shape and the drag force of a translating droplet at $\textbf{O}(Re^2)$ in a steady state situation. 
The derivation is done in the low but finite $Re$ and $Ca$. 
The dimensionless radius of the droplets can be express in the polar coordinate reference frame of the drop as, 
\begin{equation*}
    \frac{R_\text{taylor}(\theta)}{a} = 1 - \beta \textit{We} P_2(\cos\theta)
    + \mathcal{O}(We^2)
    \label{eq:taylor_solution}
\end{equation*}
where $Re = |\textbf{u}_{pf}| a /\nu_1$ with $|\textbf{u}_{pf}|$ the phase relative velocity, $P_2(\cos\theta)$ is the Legendre polynomial of degree two and $\kappa$ is a coefficient related to physical parameters given in \citet{taylor1964deformation}.  
This solution need to satisfy $\kappa Re \ll 1$ and $Re \ll 1$ to be valid. 
It is then possible to reformulate $\textbf{C}_\alpha$ in terms of the relative velocity $\textbf{u}_{pf}$, it yields, 
\begin{equation}
    \textbf{C}_\alpha =  \frac{ \rho_1 a \beta}{\gamma} \left[
        -  3
        (\textbf{u}_{\alpha f}\textbf{u}_{\alpha f})
        +   (\textbf{u}_{\alpha f}\cdot\textbf{u}_{\alpha f})\textbf{I}
    \right]
    % +\textbf{u}_{\alpha f}\cdot\textbf{u}_{\alpha f}\left(\frac{ \beta \rho_1 a}{\gamma}\right)^2
    % \left[
    %     \frac{3}{4}
    %     \textbf{u}_{p f}\textbf{u}_{p f}
    %     +
    %     \frac{1}{4}
    %     (\textbf{u}_{p f}\cdot\textbf{u}_{p f})
    %     \textbf{I}
    % \right]
    + \mathcal{O}(\textit{We}^2)
\end{equation}
where we have neglected terms of $\mathcal{O}(\textit{We}^2)$ as at this order the droplet shape isn't spheroidal anymore \citet{taylor1964deformation}. 
We now apply an average procedure on $\textbf{C}_\alpha$, and by noticing that, $\pavg{\textbf{u}_{\alpha f}\textbf{u}_{\alpha f}} = \textbf{u}_{pf}\textbf{u}_{pf} + \pavg{\textbf{u}_\alpha'\textbf{u}_\alpha'}$ we obtain the following expression 
for the averaged shape of the droplets, 
\begin{equation*}
    \textbf{C}_p  =  \frac{ \rho_1 a \beta}{\gamma} \left[
        -  3
        (\textbf{u}_{p f}\textbf{u}_{p f} + \avg{\textbf{u}_\alpha'\textbf{u}_\alpha'})
        +   (\textbf{u}_{p f}\cdot\textbf{u}_{p f}+2k_p)\textbf{I}
    \right]
    + \mathcal{O}(\textit{We}^2)
\end{equation*}
Under these quite restrictive assumption we have shown that the shape of the particle $\textbf{C}_p$ can be expressed with the relative phase velocity and particles fluctuating terms. 

Now that we have determined the mean deformation tensor within our suspension we use the second moment of mass equation to determine the strain rate and angular velocities of the particles. 
Averaging \ref{eq:dt_Cs} directly gives, 
\begin{equation}
    n_p 
    (\bm\Gamma_{p,ij}
    +  \bm\Gamma_{p,ji})
    + \textbf{C}_{p,ik}^* \cdot \bm\Gamma_{p,kj}
    + \bm\Gamma_{p,ki}^* \cdot \textbf{C}_{p,jk},
    = 
    % \pavg{\ddt \textbf{C}_{p,ij}^*}
    \pddt (n_p \textbf{C}_{p,ij}^*)
    + \div(n_p \textbf{u}_p \textbf{C}_{p,ij}^*)
\end{equation}
The fluctuations terms between the particles shape, $\textbf{C}_\alpha^*$ and the particle velocity $\textbf{u}_\alpha$, will in fact be fluctuations terms of third order in the particle velocity distribution.
Indeed, using \ref{C_def_with_u} eventually leads to $\pavg{\textbf{C}'_\alpha \textbf{u}_\alpha'} \sim \pavg{\textbf{u}_\alpha'\textbf{u}_\alpha' \textbf{u}_\alpha'}$. 
Regarding the rate of strain particle's velocity covariance we suppose that it is null as the rate of strain isn't a priori correlated with the latter property. 
Anyhow, at this stage of advancement we must suppose these terms negligible. 

\subsubsection*{Expression of the fluid phase equivalent stress}

We have seen that for spherical particle the relative velocity $\textbf{u}_{fp}$ is involved in the momentum equation through its presence via the drag force term and the second moment of the surface stress only.  
In this case we have shown that the averaged stresslet, is also a function of the relative velocity due to the inertial nature of the suspension. 

In the equivalent stress, the stress let contribution appear as, 

% Using \ref{eq:Stresslet_quasi_steay} 
% Therefore, the stress for the fluid phase can be re written, 
% \begin{align*}
%     \bm{\sigma}^\text{eq}_{1,ik} =
%     \rho_1\avg{\chi_1\textbf{u}_1'\textbf{u}_1'}_{ik} 
%     + p_1 \delta_{ik}
%     - 2 \mu_{Ein}(\phi_2) \textbf{e}\\
%     + \mu_{Rel}(Re) \left[\partial_k (\phi_2\textbf{u}_{fp,i}) + \partial_i (\phi_2\textbf{u}_{fp,k})\right]
%     + \lambda_\text{Rel}(Re)
%         \delta_{ik} \partial_l(\phi_2 u_{fp,l})
% \end{align*}
% where we have assumed that the second moment of the surface traction preserve its functional form but have an additional coefficient proportional to $Re$. 

% \tb{FOR BUBBLE WE CAN GIVES A CLEAR CONCLUSION # the stresslet is directly linked to the pressure + shape}
The expression for equivalent bulk stress might be written in the most general way as, 
\begin{align*}
    \bm{\sigma}^\text{eq}
    = \avg{\textbf{u}'\textbf{u}'}
    - p \textbf{I}
    + \mu_1 \textbf{e}
    + \mu_1 \phi_2 \textbf{e}_2 (\lambda - 1)
    +\phi_I \bm{\sigma}_I 
\end{align*}
The integral of the internal stress is quite complicated. 
If we discard internal stresses we found, 
\begin{align*}
    \bm{\sigma}^\text{eq}
    = \avg{\textbf{u}'\textbf{u}'} 
    - p_1 \textbf{I}
    + \mu_1 \textbf{e}
    + \frac{\phi_2 \gamma}{a} \left[
        2 \textbf{I}_{ij} - \frac{4}{5} \textbf{C}_{p,ij}^*
    \right]
\end{align*}
\tb{add all the terms and say it is neglected }
where the function is the integral of the stress in the expression of \citet{taylor1964deformation}. 
In the quasi steady hypothesis the ratio of viscosity must respect $\lambda \to 0$. 
However, we keep it there since it doesn't bring any simplifications to the expression of the bulk stress. 
The unknown function $f_{\bm\sigma}$ can be determined from the velocity fields solution of \citet{taylor1964deformation} by integrating the stress over the volume of the spheroidal droplet. 
The expression obtained is determined by the relative phase velocities $\textbf{u}_{\alpha f}$, and will add a contribution to the stress due to the steady state internal velocity and pressure flied of the drops. 
Note that it is possible to give a similar expression for carrier fluid phase stress. 
The only difference with the former is that in the carrier fluid phase stress the coefficient $(\lambda - 1)$ becomes $1$. 
% \subsubsection*{The expression of the carrier fluid stress}
% Regarding the fluid phase stressit 
% As before we assume a homogeneous suspension yielding a fluid phase stress similar to \ref{eq:sigma_eq_0} but without the term under the divergence operator. 
% Substituting the first moment of the particle by \ref{eq:dt} 
\begin{multline*}
    \bm{\sigma}^\text{eq}_1 = 
    \rho_1\avg{\chi_1\textbf{u}_1'\textbf{u}_1'} 
    + \phi_1 p_1 \textbf{I} 
    - \mu_1 \textbf{e} 
    + \frac{1}{2}\ddt^2 \textbf{M}_{\alpha,ij}
    -  \bm\Gamma_{\alpha,jl}\bm\Gamma_{\alpha,ik} \textbf{M}_{\alpha,kl}  
    + \mu_2 v_\alpha 2\textbf{E}_{\alpha,ij}\\
    + \frac{\gamma v_\alpha }{a} \left[
    2\textbf{I}_{ij} 
    - \frac{4 }{5} (\textbf{M}_{\alpha,ij}^* - \textbf{I}_{\alpha,ij})
    \right]\\
    - \intO{\rho_2 \textbf{w}_{2,i}^s\textbf{w}_{2,j}^s}
    - \intO{p_2^0} \textbf{I}_{ij}
    + (\mu_2 - \mu_1)\intS{(\textbf{n}_i \textbf{w}_{2,j}^s + \textbf{n}_j \textbf{w}_{2,i}^s)}
    \label{eq:Steady_state_bubble}
\end{multline*} 
% where $f_{\textbf{e}}$ is the integral of the strain rate within the drop from the solution of \citet{taylor1964deformation}. 
As mentioned for small particle deformation and quasi steady state we have,  
\begin{multline*}
    \bm{\sigma}^\text{eq}_1 = 
    \rho_1\avg{\chi_1\textbf{u}_1'\textbf{u}_1'} 
    + \phi_1 p_1 \textbf{I} 
    - \mu_1 \textbf{e} 
    + \frac{\gamma v_\alpha }{a} \left[
    2\textbf{I}_{ij} 
    - \frac{4 }{5} (\textbf{M}_{\alpha,ij}^* - \textbf{I}_{\alpha,ij})
    \right]\\
    - \intO{\rho_2 \textbf{w}_{2,i}^s\textbf{w}_{2,j}^s}
    - \intO{p_2^0} \textbf{I}_{ij}
    + (\mu_2 - \mu_1)\intS{(\textbf{n}_i \textbf{w}_{2,j}^s + \textbf{n}_j \textbf{w}_{2,i}^s)}
\end{multline*} 
Note that in quasi steady state 
\begin{equation*}
    + \frac{\gamma v_\alpha }{a} 
    \left[
    2
    % - \frac{4 }{5 } (\textbf{M}_{\alpha,mm}^* - \textbf{I}_{\alpha,mm})
    \right]
    = 
    \frac{1}{3}\intS{\textbf{r}_m\cdot\bm\sigma_{1,mk}^0\cdot \textbf{n}_k} 
    + \frac{1}{3}\intO{\rho_2 \textbf{w}_{2,m}^s\cdot \textbf{w}_{2,m}^s}
    + \intO{p_2^0} \textbf{I}_{mm}
\end{equation*}
\tb{bulk pressure with laplace on the side why not, the problem is that all closure are related to $p_1$... That is not evident}
In summary the effect of relative motion on the bulk stress appear to be anisotropic due to the imbalance of the stresses at the surface of the droplet caused by inertial effect. 
The bulk stress can be determined by averaging the particle internal and surface stresses in terms of the relative drop velocity. 
The surface stress brings quadratic terms with respect to the relative phase velocity. 
In is in fact similar to the Reynolds stress in potential flows \citet{van1982bubble}. 
The internal stress has the form of a \textit{convected} derivative of the surface velocity plus contribution $f_{\bm\sigma} \sim \textbf{u}_{p 1}$. 

It must be understood that the anisotropic nature of the stress isn't solely due to the deformation of the particles but to the unbalenced surface traction on the surface of the drop which generate a first moment. 
Consequently, inertial spherical particle in translation in a fluid will also have a non-null first moment.  

These expressions remain true under very specific circonstencies.
Clearly, more work is needed on the determination of $f_{\bm\sigma}$, $f_{\textbf{ww}}$ and $\pSavg{\textbf r \bm\sigma_1^0 \textbf{n}_2}$.  
Nevertheless, we shown that a stress is present due to the non-spherical shape of particle which is itself due to the presence of relative velocity between phases in slightly inertial regime. 
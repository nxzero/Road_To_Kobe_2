\section{Averaged system of equations for spheroidal deformable particles}
\label{sec:averaged_eq}


Now, we present the final averaged system of equations describing the averaged shape that the droplets adopt in a dispersed multiphase flow. 
As discussed earlier we decided to describe the droplets based on their orientation $\textbf{p}$ and the aspect ratio $\chi_I$.
Consequently, at the cross-grained level we would like to determine the mean, or averaged orientation of the particles, and the average aspect ratio of the particles.  
To simplify the problem at hand we consider a mono-disperse emulsion, meaning that all droplets have the same volume. 

As demonstrated below the averaged equation for the is similar to the one found in fiber media theory. 
Indeed, we transport a quantity corresponding to the average of the tensor $\textbf{pp}$ which is completed by an equation for the mean angular momentum. 
The equation for the mean aspect ratio, angular momentum, and averaged rate of strain are then derived, in the particle eigenbasis.
It will be shown that the closure problem becomes quite difficult to solve. 

\subsection{Fundamental averaged properties of the dispersed phase} 

As discussed in the preceding section, we reduced the description of the shape and orientation of a particle, to the scalar $E_I$ and $\chi_I$ and the vector $\textbf{p}$. 
Thus, the objective of this section is to present the equations of the average of $\chi_I$ and $E_I$. 
Therefore, let us first present the link between the averaged deformation tensor $\bm\chi_p$ and rate of strain tensor $\textbf{E}_p$ with their respective averaged eigenvalues.
According to the particle average formalism, the latter quantities are defined as,
\begin{align*}
    n_p \bm\chi_p = \pavg{\bm\chi_\alpha}\\
    n_p \textbf{E}_p = \pavg{\textbf{E}_\alpha}.
\end{align*}
Likewise, the average of the eigenvalues of the particle's deformation and rate of strain are defined as, 
\begin{align*}
    n_p \chi_p(\textbf{x},t) = \pavg{\chi_I(\textbf{x},t,\FF)},\\
    n_p E_p(\textbf{x},t) = \pavg{E_I(\textbf{x},t,\FF)},
    % n_p \bm\omega_p(\textbf{x},t) = \pavg{\bm\omega_I(\textbf{x},t,\FF)},
\end{align*}
respectively. 
We recall here that all the local non-averaged quantities are a function of the local position, $\textbf{x}$, time $t$, and flow configurations $\FF$. 
While the averaged quantities are only a function of the position and time. 
From the definition of $\bm\chi_\alpha$ in the limit of small  deformation (see \ref{eq:chi_I_small_def}), we obtain
\begin{equation}
    n_p \bm\chi_p
    = \chi_p
    \frac{1}{2}\left[
        3\textbf{A}_p 
        - \bm\delta
    \right]
    + \frac{3}{2}\avg{(\chi_I - \chi_p) \textbf{pp} }, 
    \label{eq:chi_p_def}
\end{equation}
Thus, the mean orientation tensor $\textbf{A}_p = \pavg{\textbf{pp}}$, and mean principal deformation $\chi_p$ are not sufficient to recover the mean deformation tensor $\bm\chi_p$.
Indeed, as exemplified by \ref{eq:chi_p_def} one needs in addition the covariance term $\avg{(\chi_I - \chi_p) \textbf{pp} }$ which is the correlation between the orientation and deformation. 
The same comments can be made regarding the shear rate of the particles $\textbf{E}_p$ and the average of its eigenvalues $E_p$. 
As will be seen, the explicit expression of $\bm\chi_p$ and $\textbf{E}_p$ are not necessarily needed in the other equations of the system. 
Thus, solving for the particle mean eigenvalue, can be interesting upon having adequate closure terms which are functions of this means eigenvalues instead of the average of the tensor quantities.
% In any case, it must be stated that the mean of the eigenvalues with the mean orientation tensor is surely not equivalent to the whole tensor $\bm\chi_p$. 
% Nevertheless, the gain in computational expenses might make it worth it. 

Regarding the particle angular rotation vector $\bm\omega_\alpha$, we have demonstrated that in the local basis, it can be expressed as $\bm\omega_\alpha  =\sum_{a=1}^3 \omega^a \textbf{p}^a$ where we recall that $\omega^a$ corresponds to the components of the rotation vector on the $a^{th}$ axis of the local basis, i.e. $\textbf{p}^a$.
Additionally, we have seen that solving for the components of $\bm\omega_\alpha$ in the local basis of the particle simplifies the angular momentum equation, see \ref{eq:dt_omega_a}.  
This simplification is particularly effective in the case of solid particles. 
The average of the $\omega^a_p$ are related to $\bm\omega_p$ through the relation, 
\begin{equation}
    n_p \bm\omega_p
    = \pavg{\bm\omega_\alpha}
    =\sum_{a=1}^3\pavg{\omega^a \textbf{p}^a}
    =\sum_{a=1}^3(n_p \omega^a_p   \textbf{p}_p^a
    +\pavg{(\omega^a)' \textbf{p}^a})
    \label{eq:omega_p_local}
\end{equation}
where $n_p \textbf{p}_p^a = \pavg{\textbf{p}^a}$ is the average orientation vector, this term is therefore null when the particle orientation is random.  
$\pavg{(\omega^a)' \textbf{p}^a}$ represents the covariance between rotation express in the local basis of the droplet and its orientation vector. 


\subsection{Orientation equation}

Let us first derive the equation for the mean orientation tensor $\textbf{A}_p = \pavg{\textbf{pp}}$. 
This is easily done by averaging \ref{eq:dt_pp}, the resulting equation is given by~:
\begin{equation}
    \pddt (n_p\textbf{A})
    + \div (
        n_p\textbf{u}_p\textbf{A}_p
        + \mathbf{\Sigma}
        )
    =
    \pavg{\textbf{pp} \times \bm\omega_\alpha}
    + \pavg{\bm\omega_\alpha \times \textbf{pp}},
    % + \pnavg{\textbf{pp}' \times \omega_\alpha'}
    % +\pnavg{\omega_\alpha' \times \textbf{pp}'}
    \label{eq:avg_dt_pp_alpha}
\end{equation}
where $\mathbf{\Sigma} = \pavg{\textbf{u}'_\alpha(\textbf{pp})'}$ is the covariance term between the fluctuation of the velocity and the orientation tensor.

For the purpose of understanding, we now show how this equation is the starting point of the usual equations used in fiber media theory. 
By assuming torque-free rigid particle into Stokes flow, we can use Jeffery's equation \citep{guazzelli2011},
\begin{equation}
    \omega_\alpha \times \textbf{p}
    = \bm{\Omega}_f\cdot\textbf{p}
    + \beta\left(
        \textbf{E}_f\cdot \textbf{p}
        - \textbf{E}_f : \textbf{ppp}
    \right),
    \label{eq:jefferey}
\end{equation}
with $\textbf{E}_f$ and $\bm{\Omega}_f$ being the symmetric and antisymmetric parts of the bulk velocity gradient, respectively, such that $\grad \textbf{u}_f=\textbf{E}_f+\bm{\Omega}_f$.
The coefficient $\beta$  is a constant related to the aspect ratio of the particle.
Finally, by substituting the RHS terms of \ref{eq:avg_dt_pp_alpha}, by using \ref{eq:jefferey}, we arrive at the closed form of the second moment of mass equation:
\begin{equation}
    \pddt \textbf{A}_p
    + \div (
        \textbf{u}_p\textbf{A}_p
        + \mathbf{\Sigma}
    )
    =
    \bm{\Omega}_f \cdot \textbf{A}
    - \textbf{A} \cdot \bm{\Omega}_f
    + \beta\left[
        \textbf{E}_f \cdot \textbf{A}
        -\textbf{A} \cdot \textbf{E}_f
        - \textbf{E}_f : \mathbb{A}
    \right]
    \label{eq:hybrid_avg_dt_pp}
\end{equation}
where the fourth-order tensor $\mathbb{A}_p$, is defined as $\mathbb{A}_p = \pavg{\textbf{pppp}}$.
In this expression, we have removed the fluctuation terms that should appear due to the averaging procedure. 
In \citet{wang2008objective} they derive \ref{eq:hybrid_avg_dt_pp} by the means of kinetic theory, based on \ref{eq:jefferey} and the fact that $\ddt \textbf{p} = \omega_\alpha \times \textbf{p}$ (Equation (3) of their article).
Their equation is similar to \ref{eq:hybrid_avg_dt_pp}.
%  except that they employ a phenomenological closure for the term, $\div \mathbf{\Sigma}$, which accounts for particle interactions.
Thus, we showed how it is possible to derive the orientation tensor conservation equation, commonly used in fiber field theory, from the second-order moments of mass's equation. 
However, \ref{eq:jefferey} and \ref{eq:hybrid_avg_dt_pp} are only valid in the Stokes regime. 


Therefore, it is indispensable to consider a more general framework when working with inertial particles. 
The equation for $\textbf{A}_p$ in this case reads as, 
\begin{equation}
    \pddt (n_p\textbf{A})
    + \div (
        n_p\textbf{u}_p\textbf{A}_p
        + \mathbf{\Sigma}
        )
    =
    n_p \textbf{A}_p \times \bm\omega_p
    + n_p \bm\omega_p \times \textbf{A}_p
    + \pavg{\textbf{pp} \times \bm\omega_\alpha'}
    + \pavg{\bm\omega_\alpha' \times \textbf{pp}},
    \label{eq:dt_Ap}
\end{equation}
where the last two terms on the right-hand side correspond to the covariance between orientation and rotation. 
% The mean particle orientation tensor $\textbf{A}_p$ appearing on the right-hand side of \ref{eq:dt_Ap} is an unknown of the problem.
The vector $\bm\omega_p$ in \ref{eq:dt_Ap} is solved with the angular momentum equation, in opposition to the Stokes regime where an algebraic closure is directly provided for $\bm\omega_p$ (i.e. \ref{eq:jefferey}). 
As suggested by \ref{eq:omega_p_local} and \ref{eq:dt_omega_a}, we may solve for the averaged values expressed in the particles local basis, i.e. for the $\omega^a_p$ instead of the $(\bm\omega_p)_i$.
However, this may not be practical as one need $\bm\omega_p$ in \ref{eq:dt_Ap} which implies the need for the closure terms $\pavg{(\omega^a)' \textbf{p}^a}$.


\subsection{Deformation, rate of strain and angular momentum equations}

In the small deformation regime, the aspect ratio is governed by \ref{eq:small_def}, and the particle rate of strain by  \ref{eq:lamb_like_model}. 
Therefore, the equations for $\chi_p$ and $E_p$ can be derived averaging the latter equations and reads,  
\begin{align}
    \pddt (n_p \chi_p)
    + \div (
        n_p \textbf{u}_p \chi_p 
        + \pavg{\textbf{u}_\alpha' \chi_p'}
    )
    &= 2 n_p E_p,
    \label{eq:avg_chi_I}
    \\
    \pddt (n_p E_p)
    + \div (
        n_p \textbf{u}_p E_p 
        + \pavg{\textbf{u}_\alpha' E_p'}
    )
    &= 
    - \frac{8 \gamma }{a^3\rho_d} n_p \chi_p
    - \frac{10 \mu_d}{a^2\rho_d} n_p E_p
    + \frac{10}{m_\alpha a^2} n_p F^{||}_p,
    \label{eq:avg_E_I}
\end{align}
respectively.
We recall that in principle, $\textbf{u}_p$ and $n_p$ are already known by solving the mass and momentum equation of the particle phase.
Thus, the only closure terms required by these equations is the covariance term $\pavg{\textbf{u}_\alpha' \chi_p'}$, $ \pavg{\textbf{u}_\alpha' E_p'}$ and of course the forcing term project on the principal axis of the particle, namely $\pavg{\textbf{pp} : \textbf{F}} = n_p F_p^{||}$. 
Note that if we had considered the deformation and rate of strain equation in the laboratory reference frame this would represent two tensor equations instead of two scalar ones. 


To complete entirely this system of equations we may add one equation for each component of the angular velocity, $\omega_p^a$. 
This is done by averaging \ref{eq:dt_omega_a}, namely,
\begin{multline}
    \pddt (n_p \omega^a_p)
    + \div (
        n_p \textbf{u}_p \omega^a_p 
        + \pavg{\textbf{u}_\alpha' \omega^a_p}
    )
    = 
    % I^{ab}\omega^b  \omega^c \epsilon_{jki} p_i^a p_j^c p_k^a
    \frac{5}{2 m_\alpha a^2}\left\{
        \omega^a_p E^a_p + \pavg{(\omega^a_\alpha)' E^a_p} \phantom{\intS{}} \right. \\ \left.
        +\pavg{
        \frac{2 \textbf{p}^a}{ (2 - \chi_I)}\cdot 
        \left[\intS{(\textbf{r}\times\bm\sigma_f^0\cdot \textbf{n})} 
        - \ddt\intO{\rho_d(\textbf{r}\times \textbf{v}_d^0)}\right]
        }
    \right\}. 
\end{multline}
The additional closure term related to the averaging is $\pavg{(\omega^a_\alpha)' E^a_p}$.
This represents the covariance between the angular velocity along the axis $\textbf{p}^a$ of the droplet and its rate of deformation along the same axis. 
For example, in cases where deformation is induced by the rotation of droplets due to centrifugal forces, there is a clear correlation between rotation and deformation that happen along the same axis. 
As a result, $\pavg{(\omega^a_\alpha)' E^a_p}$ will not be negligible in such scenarios.
Thus, notice how we drastically reduced the complexity of the equation by expressing it in the local reference frame instead of directly averaging \ref{eq:dt_mu2} which would yield a complicated expression due to the presence of the inertia tensor inside the time derivative.  

Overall, the equations for $E_p$ and $\chi_p$ are relatively straightforward since they are scalar equations. 
If it is possible to derive appropriate closure terms that only involve $\chi_p$ and $E_p$, bypassing the need for $\bm\chi_p$ and $\textbf{E}_p$, then this approach is clearly the most efficient. 
The challenge in this problem lies in the orientation equation.
Indeed, determining $\textbf{A}_p$ requires the vector $\bm\omega_p$ in the laboratory reference frame.
We have demonstrated how to derive the angular momentum equations in the particle's local reference frame. 
While this approach seems to reduce the complexity of the expression, it introduces the necessity for new closure terms to recover $\bm\omega_p$ from the $\omega^a_p$. 
Thus, it may be more efficient to solve for the mean principal values of $\bm\chi_p$ and $\textbf{E}_p$ reducing tensor equations to scalar ones.
However, deriving an equation for the mean values of $\bm\omega_p$ appears less practical, given its explicit necessity in the equation for $\textbf{A}_p$. 


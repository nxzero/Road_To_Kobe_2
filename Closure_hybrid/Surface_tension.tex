
\section{Ellipsoidal surface tension stress}
\label{ap:surface_tension}
\tb{METTRE A JOUR LES FORMULES}
In this section we provide the detailed calculation of te surface tension stress tensor for ellipsoidal particles. 

In indices notation,  
\begin{equation*}
    \sigma_{I,ij}^0 =\gamma\left[
    \delta_{ij} - \frac{ H_{ik} H_{jl} :  r_kr_l}{  H_{ab}  H_{ac} r_br_c} \right]
\end{equation*}
Integrating this over a surface will eventually leads to an elliptic integral due to the elliptical surface. 

If the spheroid is aligned with the principal axes $\textbf{H} = \textbf{e}_1\textbf{e}_1 / a^2(t) + (\textbf{e}_2\textbf{e}_2+ \textbf{e}_3\textbf{e}_3)/b^2(t)$.
Thus, $\textbf{H}\cdot \textbf{r} = H_{ij} r_j = {e}_{1,i}{e}_{1,j}r_j / a^2(t) + ({e}_{2,i}{e}_{2,j} + {e}_{3,i}{e}_{3,j})r_j/b^2(t) $ which gives, the vector, 
$\textbf{H}\cdot \textbf{r} = \textbf{e}_{1} x/ a^2(t) + (\textbf{e}_{2}y + \textbf{e}_{3}z)/b^2(t)  = \textbf{e}_i r_i /a_i^2$
Consequently, the second term of the last equality reads in a main axis as, 
\begin{equation*}
    \frac{ H_{ik} H_{jl} :  r_kr_l}{  H_{ab}  H_{ac} r_br_c} 
    = \textbf{e}_i \textbf{e}_j \frac{ r_i r_j /(a_i a_j)^2 }
    {\frac{z^2}{a^4}+\frac{y^2+x^2}{b^4}}
\end{equation*}
To dealt with this ellipsoid integral we may employ the following change of variable :
\begin{align*}
    x = b \sin \theta \cos \phi\\
    y = b \sin \theta \sin \phi\\
    z = a \cos \theta 
\end{align*}
with $\theta \in [0,\pi]$ and $\phi \in [0,2\pi]$. 
The point on the ellipsoid surface can then be written as $\textbf{v} = [x(\theta,\phi),y(\theta,\phi),z(\theta,\phi)]$. 
Now let's first consider the surface calculation, in our coordinate system we have, 
\begin{equation*}
    \iint_S dS
    = 
    \int_{0}^{2\pi}
    \int_{0}^{\pi}
    \left|\frac{\partial \textbf{v}}{\partial \theta} 
    \times 
    \frac{\partial \textbf{v}}{\partial \phi} \right|
    d\theta
    d\phi
\end{equation*}
The partial derivative reads, 
\begin{align*}
    \frac{\partial \textbf{v}}{\partial \theta}
    = 
    b \cos \theta \cos \phi \textbf{e}_x
    + b \cos \theta \sin \phi\textbf{e}_y
    - a \sin \theta \textbf{e}_z
    \\
    \frac{\partial \textbf{v}}{\partial \phi}
    = 
    - b \sin \theta \sin \phi \textbf{e}_x
    + b \sin \theta \cos \phi\textbf{e}_y
\end{align*}
Taking the norm of the above vector cross product yields the relation : 
$dS = \left|\frac{\partial \textbf{v}}{\partial \theta} 
\times 
\frac{\partial \textbf{v}}{\partial \phi} \right| d\theta d\phi = b \sin\theta (a^2\sin^2\theta+ b^2 \cos^2\theta)^{1/2}d\theta d\phi$. 

Injecting these coordinate into the previous integrand formula for the dyadic normal reads, 
\begin{equation*}
    \frac{ H_{ik} H_{jl} :  r_kr_l}{  H_{ab}  H_{ac} r_br_c} 
    = \textbf{e}_i \textbf{e}_j \frac{ r_i r_j /(a_i a_j)^2 }
    {\frac{1}{a^2}\cos^2 \theta+\frac{1}{b^2}\sin^2 \theta (\cos^2 \phi+ \sin^2 \phi)}
    = \textbf{e}_i \textbf{e}_j \frac{ r_i r_j}
    {\frac{(a_i a_j)^2}{a^2}\cos^2 \theta+\frac{(a_i a_j)^2}{b^2}\sin^2 \theta }
\end{equation*}
Due to symmetry consideration the crossed terms are all null thus the remaining terms to evaluate are just the diagonal terms. 
The first term of the surface stress, that we call $S_1$ is the surface energy, i.e. the surface of the ellipsoid times the surface tension coefficient $\gamma$. 
\begin{equation*}
    (S_1)_{ij} 
    = \delta_{ij} 2\pi 
    \int_0^\pi 
    b \sin\theta (a^2\sin^2\theta+ b^2 \cos^2\theta)^{1/2} d\theta 
    = 2\pi b \left[\frac{a^2}{\sqrt{b^2-a^2}} \text{asinh}\left(\frac{\sqrt{b^2-a^2}}{a}\right)+b\right]
\end{equation*} 
where $\text{asinh}$ is the Hyperbolic Arc Sine function.
Note that the surface can be reformulated as,  
\begin{equation*}
    s_\alpha
    = 2\pi b \left[\frac{a^2}{\sqrt{b^2-a^2}} \text{ln}\left(\frac{b + \sqrt{b^2-a^2}}{a}\right)+b\right]
\end{equation*} 
which is more convenient. 
This expression is consistent with the classic ex pression of an ellipsoid surface. 
Now let's compute the integral $S_{2,zz}$ it reads, 
\begin{align*}
    S_{||}
    = 
    \gamma\int_0^{2\pi}\int_0^{\pi}\left[
        \frac{\cos^2\theta}{\cos^2 \theta+\frac{a^2}{b^2}\sin^2 \theta }
    \right]
    b \sin\theta (a^2\sin^2\theta+ b^2 \cos^2\theta)^{1/2} d\theta d\phi\\
    = 
    \gamma\int_0^{2\pi}\int_0^{\pi}\left[
        \frac{\cos^2\theta}{b^2\cos^2 \theta+a^2\sin^2 \theta }
    \right]
    b^3 \sin\theta (a^2\sin^2\theta+  b^2\cos^2\theta)^{1/2} d\theta d\phi\\
    = 
    \gamma 2\pi \int_0^{\pi}
    b^3 \cos^2\theta\sin\theta (a^2\sin^2+  b^2\cos^2\theta)^{-1/2} d\theta\\
    = \gamma 2\pi  b^3\,\left({{2\,b}\over{2\,b^2-2\,a^2}}-{{a^2\,{\rm asinh}\; \left(
    {{\sqrt{b^2-a^2}}\over{a}}\right)}\over{\left(b^2-a^2\right)^{{{3
    }\over{2}}}}}\right)
\end{align*}
Now let's see for the perpendicular components, namely, 
\begin{align*}
    S_{\bot}
    = 
    b a^2\gamma\int_0^{2\pi}\sin^2\phi d\phi 
    \int_0^{\pi}
        \sin^2\theta
    \sin\theta (a^2\sin^2\theta+ b^2 \cos^2\theta)^{-1/2} d\theta \\
    = 
    b a^2\gamma \pi
    \int_0^{\pi}
    \sin^3\theta (a^2\sin^2\theta+ b^2 \cos^2\theta)^{-1/2} d\theta \\
\end{align*}

The final results is the following, 
The surface tension stress can be written in terms of the two principal axis of the ellipsoid in the laboratory reference frame with, 
\begin{equation*}
    \intS{\bm{\sigma}_I^0}
    = \frac{2}{3} s_\alpha \gamma \textbf{I} + \textbf{pp} (-2 S)  + (\textbf{I} - \textbf{pp}){S}
\end{equation*}
where the first term is the isotropic part consisting into the surface energy and the second and third terms correspond to the components of anisotropy of the surface stress, namely, 
\begin{align*}
    S
    = 
    {{\left(\pi\,a^4\,b-4\,\pi\,a^2\,b^3\right)\,\log \left({{\sqrt{b^2
    -a^2}+b}\over{a}}\right)+\sqrt{b^2-a^2}\,\left(2\,\pi\,b^4+\pi\,a^2
    \,b^2\right)}\over{\sqrt{b^2-a^2}\,\left(3\,b^2-3\,a^2\right)}}
\end{align*}
We recall that this quantity drives the stress jump at the interface. 
As it is expected this stress jump is axis symmetric around the particle main axis. 
Besides, it is maximum at the poles and minimum at the equator of the particle. 
It is then possible to compute the integral of the stress by direct integration in the reference frame of the ellipsoid principal axes. 
The exact result yields, 
\begin{equation*}
    \gamma\pSavg{\textbf{I}-\textbf{nn}}
    = \gamma \left[
        \frac{2}{3} s_\alpha \textbf{I}
        + S (\textbf{I}-\textbf{pp}) -2S\textbf{pp}
        \right]
\end{equation*}
where the first component correspond to the isotropic part of the surface stress, and the second component to the deviatoric part of the surface stress. 
Notice that the deviatoric part of this tensor is function of one unique coefficient, $S$ due to the axis symmetrical nature of the droplets. 
Exact solution can be given in terms of the small deformation parameter $e_\bot = b/r$. 
Then an approximation can be deduced for the $e -1 \ll 1$, it gives,
\begin{align*}
    s_\alpha 
    = 4\pi r^2 \left[\frac{e_\bot^2}{2} + \frac{\ln\left(\sqrt{{e_\bot^6}-1}+{e_\bot^3}\right)}{2e_\bot\sqrt{e_\bot^6-1}}\right]
    = 4 \pi r^2 + \frac{24 v }{5 r} (e_\bot-1)^2\\
    S = \frac{4}{3} \pi r^2 \left[
    \frac{\left( \frac{1}{4} - e_\bot^6\right)  \log{\left( \sqrt{e_\bot^6-1}+{e_\bot^3}\right) } }
    { e_\bot  \left( e_\bot^6- 1\right)^{3/2} }
    +  \frac{e_\bot^2\left( e_\bot^6+  \frac{1}{2}\right)}{2\left( e_\bot^6- 1\right)}  \right]
    \approx 
    \frac{8 v}{5 r}(e_\bot-1) + \frac{12 v }{35r}(e_\bot-1)^2 \ldots
\end{align*}
\tb{METTRE A JOUR LES FORMULES}
Regarding the expression of the surface of the spheroid, it can be noticed that the function within bracket tends to $1$ for $e=1$ leaving us with $s_\alpha = 4\pi r^2$ which is the surface of the sphere. 
Then, it slowly increases when $e$ is either superior or inferior to $1$. 

Equally, the results can be expressed in terms of the deformation parameter $e_{||} = a/r$ in which case the previous results give at the second order in $e_\bot-1$ the following expression, 
\begin{align*}
    s_\alpha 
    \approx 4 \pi r^2 + \frac{6 v }{5 r} (e_{||}-1)^2 \ldots\\
    S 
    \approx 
    - \frac{4 v}{5 r}(e_{||}-1) + \frac{24 v }{35r}(e_{||}-1)^2 \ldots
\end{align*}
Noticing that the strain tensor $\textbf{C} = (e_{||}-1) \textbf{pp} + (e_\bot-1)(\textbf{I}- \textbf{pp})$ one can finally write,
\begin{align*}
    \gamma\pSavg{\textbf{I}-\textbf{nn}}
    = \frac{\gamma v}{r} \left[
        2  + \frac{1 }{15 } (\textbf{C}:\textbf{C})\right] \textbf{I}\\
        + \frac{\gamma v}{r} \left[ \frac{8}{5} \textbf{C}
        + \frac{12}{35}[\textbf{C}\cdot \textbf{C}- 6 (\textbf{C}:\textbf{pp})^2\textbf{pp}]
        \right]
\end{align*}
This gives the surface stress tensor up to the second order terms in accuracy. 
To our knowledge this has never been proposed. 
\tb{check in better details}

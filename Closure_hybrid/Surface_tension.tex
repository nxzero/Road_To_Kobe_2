
\section{Ellipsoidal surface tension stress}
\label{ap:surface_tension}
% \tb{METTRE A JOUR LES FORMULES}
% In this section we provide the detailed calculation of te surface tension stress tensor for ellipsoidal particles. 

% In indices notation,  
% \begin{equation*}
%     \sigma_{I,ij}^0 =\gamma\left[
%     \delta_{ij} - \frac{ H_{ik} H_{jl} :  r_kr_l}{  H_{ab}  H_{ac} r_br_c} \right]
% \end{equation*}
% Integrating this over a surface will eventually leads to an elliptic integral due to the elliptical surface. 

% If the spheroid is aligned with the principal axes $\textbf{H} = \textbf{e}_1\textbf{e}_1 / a^2(t) + (\textbf{e}_2\textbf{e}_2+ \textbf{e}_3\textbf{e}_3)/b^2(t)$.
% Thus, $\textbf{H}\cdot \textbf{r} = H_{ij} r_j = {e}_{1,i}{e}_{1,j}r_j / a^2(t) + ({e}_{2,i}{e}_{2,j} + {e}_{3,i}{e}_{3,j})r_j/b^2(t) $ which gives, the vector, 
% $\textbf{H}\cdot \textbf{r} = \textbf{e}_{1} x/ a^2(t) + (\textbf{e}_{2}y + \textbf{e}_{3}z)/b^2(t)  = \textbf{e}_i r_i /a_i^2$
% Consequently, the second term of the last equality reads in a main axis as, 
% \begin{equation*}
%     \frac{ H_{ik} H_{jl} :  r_kr_l}{  H_{ab}  H_{ac} r_br_c} 
%     = \textbf{e}_i \textbf{e}_j \frac{ r_i r_j /(a_i a_j)^2 }
%     {\frac{z^2}{a^4}+\frac{y^2+x^2}{b^4}}
% \end{equation*}
% To dealt with this ellipsoid integral we may employ the following change of variable :
% \begin{align*}
%     x = b \sin \theta \cos \phi\\
%     y = b \sin \theta \sin \phi\\
%     z = a \cos \theta 
% \end{align*}
% with $\theta \in [0,\pi]$ and $\phi \in [0,2\pi]$. 
% The point on the ellipsoid surface can then be written as $\textbf{v} = [x(\theta,\phi),y(\theta,\phi),z(\theta,\phi)]$. 
% Now let's first consider the surface calculation, in our coordinate system we have, 
% \begin{equation*}
%     \iint_S dS
%     = 
%     \int_{0}^{2\pi}
%     \int_{0}^{\pi}
%     \left|\frac{\partial \textbf{v}}{\partial \theta} 
%     \times 
%     \frac{\partial \textbf{v}}{\partial \phi} \right|
%     d\theta
%     d\phi
% \end{equation*}
% The partial derivative reads, 
% \begin{align*}
%     \frac{\partial \textbf{v}}{\partial \theta}
%     = 
%     b \cos \theta \cos \phi \textbf{e}_x
%     + b \cos \theta \sin \phi\textbf{e}_y
%     - a \sin \theta \textbf{e}_z
%     \\
%     \frac{\partial \textbf{v}}{\partial \phi}
%     = 
%     - b \sin \theta \sin \phi \textbf{e}_x
%     + b \sin \theta \cos \phi\textbf{e}_y
% \end{align*}
% Taking the norm of the above vector cross product yields the relation : 
% $dS = \left|\frac{\partial \textbf{v}}{\partial \theta} 
% \times 
% \frac{\partial \textbf{v}}{\partial \phi} \right| d\theta d\phi = b \sin\theta (a^2\sin^2\theta+ b^2 \cos^2\theta)^{1/2}d\theta d\phi$. 

% Injecting these coordinate into the previous integrand formula for the dyadic normal reads, 
% \begin{equation*}
%     \frac{ H_{ik} H_{jl} :  r_kr_l}{  H_{ab}  H_{ac} r_br_c} 
%     = \textbf{e}_i \textbf{e}_j \frac{ r_i r_j /(a_i a_j)^2 }
%     {\frac{1}{a^2}\cos^2 \theta+\frac{1}{b^2}\sin^2 \theta (\cos^2 \phi+ \sin^2 \phi)}
%     = \textbf{e}_i \textbf{e}_j \frac{ r_i r_j}
%     {\frac{(a_i a_j)^2}{a^2}\cos^2 \theta+\frac{(a_i a_j)^2}{b^2}\sin^2 \theta }
% \end{equation*}
% Due to symmetry consideration the crossed terms are all null thus the remaining terms to evaluate are just the diagonal terms. 
% The first term of the surface stress, that we call $S_1$ is the surface energy, i.e. the surface of the ellipsoid times the surface tension coefficient $\gamma$. 
% \begin{equation*}
%     (S_1)_{ij} 
%     = \delta_{ij} 2\pi 
%     \int_0^\pi 
%     b \sin\theta (a^2\sin^2\theta+ b^2 \cos^2\theta)^{1/2} d\theta 
%     = 2\pi b \left[\frac{a^2}{\sqrt{b^2-a^2}} \text{asinh}\left(\frac{\sqrt{b^2-a^2}}{a}\right)+b\right]
% \end{equation*} 
% where $\text{asinh}$ is the Hyperbolic Arc Sine function.
% Note that the surface can be reformulated as,  
% \begin{equation*}
%     s_\alpha
%     = 2\pi b \left[\frac{a^2}{\sqrt{b^2-a^2}} \text{ln}\left(\frac{b + \sqrt{b^2-a^2}}{a}\right)+b\right]
% \end{equation*} 
% which is more convenient. 
% This expression is consistent with the classic ex pression of an ellipsoid surface. 
% Now let's compute the integral $S_{2,zz}$ it reads, 
% \begin{align*}
%     S_{||}
%     = 
%     \gamma\int_0^{2\pi}\int_0^{\pi}\left[
%         \frac{\cos^2\theta}{\cos^2 \theta+\frac{a^2}{b^2}\sin^2 \theta }
%     \right]
%     b \sin\theta (a^2\sin^2\theta+ b^2 \cos^2\theta)^{1/2} d\theta d\phi\\
%     = 
%     \gamma\int_0^{2\pi}\int_0^{\pi}\left[
%         \frac{\cos^2\theta}{b^2\cos^2 \theta+a^2\sin^2 \theta }
%     \right]
%     b^3 \sin\theta (a^2\sin^2\theta+  b^2\cos^2\theta)^{1/2} d\theta d\phi\\
%     = 
%     \gamma 2\pi \int_0^{\pi}
%     b^3 \cos^2\theta\sin\theta (a^2\sin^2+  b^2\cos^2\theta)^{-1/2} d\theta\\
%     = \gamma 2\pi  b^3\,\left({{2\,b}\over{2\,b^2-2\,a^2}}-{{a^2\,{\rm asinh}\; \left(
%     {{\sqrt{b^2-a^2}}\over{a}}\right)}\over{\left(b^2-a^2\right)^{{{3
%     }\over{2}}}}}\right)
% \end{align*}
% Now let's see for the perpendicular components, namely, 
% \begin{align*}
%     S_{\bot}
%     = 
%     b a^2\gamma\int_0^{2\pi}\sin^2\phi d\phi 
%     \int_0^{\pi}
%         \sin^2\theta
%     \sin\theta (a^2\sin^2\theta+ b^2 \cos^2\theta)^{-1/2} d\theta \\
%     = 
%     b a^2\gamma \pi
%     \int_0^{\pi}
%     \sin^3\theta (a^2\sin^2\theta+ b^2 \cos^2\theta)^{-1/2} d\theta \\
% \end{align*}

% The final results is the following, 
% The surface tension stress can be written in terms of the two principal axis of the ellipsoid in the laboratory reference frame with, 
% \begin{equation*}
%     \intS{\bm{\sigma}_I^0}
%     = \frac{2}{3} s_\alpha \gamma \textbf{I} + \textbf{pp} (-2 S)  + (\textbf{I} - \textbf{pp}){S}
% \end{equation*}
% where the first term is the isotropic part consisting into the surface energy and the second and third terms correspond to the components of anisotropy of the surface stress, namely, 
% \begin{align*}
%     S
%     = 
%     {{\left(\pi\,a^4\,b-4\,\pi\,a^2\,b^3\right)\,\log \left({{\sqrt{b^2
%     -a^2}+b}\over{a}}\right)+\sqrt{b^2-a^2}\,\left(2\,\pi\,b^4+\pi\,a^2
%     \,b^2\right)}\over{\sqrt{b^2-a^2}\,\left(3\,b^2-3\,a^2\right)}}
% \end{align*}
% We recall that this quantity drives the stress jump at the interface. 
% As it is expected this stress jump is axis symmetric around the particle main axis. 
% Besides, it is maximum at the poles and minimum at the equator of the particle. 
% It is then possible to compute the integral of the stress by direct integration in the reference frame of the ellipsoid principal axes. 
% The exact result yields, 
% \begin{equation*}
%     \gamma\pSavg{\textbf{I}-\textbf{nn}}
%     = \gamma \left[
%         \frac{2}{3} s_\alpha \textbf{I}
%         + S (\textbf{I}-\textbf{pp}) -2S\textbf{pp}
%         \right]
% \end{equation*}
% where the first component correspond to the isotropic part of the surface stress, and the second component to the deviatoric part of the surface stress. 
% Notice that the deviatoric part of this tensor is function of one unique coefficient, $S$ due to the axis symmetrical nature of the droplets. 
% Exact solution can be given in terms of the small deformation parameter $e_\bot = b/r$. 
% Then an approximation can be deduced for the $e -1 \ll 1$, it gives,
% \begin{align*}
%     s_\alpha 
%     = 4\pi r^2 \left[\frac{e_\bot^2}{2} + \frac{\ln\left(\sqrt{{e_\bot^6}-1}+{e_\bot^3}\right)}{2e_\bot\sqrt{e_\bot^6-1}}\right]
%     = 4 \pi r^2 + \frac{24 v }{5 r} (e_\bot-1)^2\\
%     S = \frac{4}{3} \pi r^2 \left[
%     \frac{\left( \frac{1}{4} - e_\bot^6\right)  \log{\left( \sqrt{e_\bot^6-1}+{e_\bot^3}\right) } }
%     { e_\bot  \left( e_\bot^6- 1\right)^{3/2} }
%     +  \frac{e_\bot^2\left( e_\bot^6+  \frac{1}{2}\right)}{2\left( e_\bot^6- 1\right)}  \right]
%     \approx 
%     \frac{8 v}{5 r}(e_\bot-1) + \frac{12 v }{35r}(e_\bot-1)^2 \ldots
% \end{align*}
% \tb{METTRE A JOUR LES FORMULES}
% Regarding the expression of the surface of the spheroid, it can be noticed that the function within bracket tends to $1$ for $e=1$ leaving us with $s_\alpha = 4\pi r^2$ which is the surface of the sphere. 
% Then, it slowly increases when $e$ is either superior or inferior to $1$. 

% Equally, the results can be expressed in terms of the deformation parameter $e_{||} = a/r$ in which case the previous results give at the second order in $e_\bot-1$ the following expression, 
% \begin{align*}
%     s_\alpha 
%     \approx 4 \pi r^2 + \frac{6 v }{5 r} (e_{||}-1)^2 \ldots\\
%     S 
%     \approx 
%     - \frac{4 v}{5 r}(e_{||}-1) + \frac{24 v }{35r}(e_{||}-1)^2 \ldots
% \end{align*}
% Noticing that the strain tensor $\textbf{C} = (e_{||}-1) \textbf{pp} + (e_\bot-1)(\textbf{I}- \textbf{pp})$ one can finally write,
% \begin{align*}
%     \gamma\pSavg{\textbf{I}-\textbf{nn}}
%     = \frac{\gamma v}{r} \left[
%         2  + \frac{1 }{15 } (\textbf{C}:\textbf{C})\right] \textbf{I}\\
%         + \frac{\gamma v}{r} \left[ \frac{8}{5} \textbf{C}
%         + \frac{12}{35}[\textbf{C}\cdot \textbf{C}- 6 (\textbf{C}:\textbf{pp})^2\textbf{pp}]
%         \right]
% \end{align*}
% This gives the surface stress tensor up to the second order terms in accuracy. 
% To our knowledge this has never been proposed. 
% \tb{check in better details}

% \subsection*{Same au propre}
This appendix is dedicated to the derivation of the surface stress tesnor appearing in the conservation equation of the particle mean rate of strain $\textbf S_\alpha$. 
The surface tension stress tensor may be written as, 
\begin{equation*}
    \textbf{S} 
    = \intS{\bm\sigma_I^0}
    = \intS{\gamma (\delta - \textbf{nn})}
\end{equation*}
This tensor can be composed into a deviatoric and isotropic part, 
\begin{equation*}
    \textbf{S} 
    = 
    \frac{2}{3}\gamma\intS{}
    +\gamma \intS{ (\frac{1}{3}\bm\delta - \textbf{nn})} 
\end{equation*}
Assuming, that  \textbf{p} is one of the principal axis of the particle we can note, 
\begin{equation*}
    \textbf{S} 
    = 
    \frac{2}{3}\gamma\intS{}
    + S \textbf{pp} - \frac{1}{2}S (\bm\delta - \textbf{pp})
\end{equation*}
where $S$ is the first eingen value of the deviatoric part of  \textbf{S}, namely
\begin{equation*}
    S = \gamma \intS{ (\frac{1}{3}\bm\delta - \textbf{nn})} : \textbf{pp}.
\end{equation*}



If we suppose that our droplets is ellipsoidal then it can be entirely described by a matrix \textbf{H} such that $\textbf{r}\cdot \textbf{H}\cdot \textbf{r}- 1= 0$ represents all the points lying on the surface.
With this definition $\textbf{H} = \sum_{i=0}^{2} \frac{1}{A_i^2} \textbf{p}^i\textbf{p}^i$ where $A_i$ are the semi axis of the particle and $\textbf{p}^i$ its corresponding eigenvectors. 
In this case the normal vector \textbf{n} at the surface can be written, \citep{nadim1996concise},
\begin{equation*}
    \textbf{n} = \frac{\textbf{H}\cdot \textbf{r}}{(\textbf{r}\cdot \textbf{H}\cdot \textbf{H}\cdot \textbf{r})^{1/2}}
\end{equation*}
Thus, 
\begin{equation*}
    (\bm\sigma_I^0)_{ij}
    = \delta_{ij}
    - 
    \frac{H_{ik}H_{jl}r_lr_k}{(H_{ab}H_{ac}r_br_c)}. 
\end{equation*}


Now let us introduce the dimensionless surface stress as, 
\begin{equation*}
    \textbf{S}^* 
    = \textbf{S} / \gamma /a^2
    = \frac{1}{a^2}\intS{(\bm\sigma_I^0)^*}
    = \frac{1}{a^2}\intS{ \left[\delta_{ij} - \frac{H_{ik}^*H_{jl}^*r_l^*r_k^*}{(H_{ab}^*H_{ac}^*r_b^*r_c^*)} \right]}
\end{equation*}
where, $H^* = a^2 H$ and $r^* = r/a$.
For ease of reading we will remove all the $^*$ in the following and introduce the $a_i = A_i/a$ as the dimensionless semi-axis.
Therefore, the objective of this study is to compute the integral, 
\begin{equation*}
    \intS{\left[\delta_{ij} -\frac{H_{ik}H_{jl}r_lr_k}{(H_{ab}H_{ac}r_br_c)}\right] }
\end{equation*}
where the domain of integration is an ellipsoid of mean radius $1$ given by the eq $\textbf{r}\cdot \textbf{H}\cdot \textbf{r} -1 =0$. 
This parametric equation can be written using ellipsoidal coordinate as,
\begin{align*}
    x = b \sin\theta \cos\phi\\
    y = b \sin\theta \sin\phi\\
    x = a \cos\theta 
\end{align*}
where we have assumed that $a_0 = a, a_1=a_2=b$ for $\theta \in [0,\pi]$ and $\phi \in [0,2\pi]$.
Thus, the point on the ellipsoid surface can then be written as the vector $\textbf{v} = [x(\theta,\phi),y(\theta,\phi),z(\theta,\phi)]$. 
Meaning that the infinitesimal portion of surface may be written, 
\begin{equation*}
    d\Gamma = \left|\frac{\partial \textbf{v}}{\partial \theta}\times \frac{\partial \textbf{v}}{\partial \phi}\right| d\theta d\phi
\end{equation*}
\begin{align*}
    \frac{\partial \textbf{v}}{\partial \theta}
    = 
    b \cos \theta \cos \phi \textbf{p}_x
    + b \cos \theta \sin \phi\textbf{p}_y
    - a \sin \theta \textbf{p}_z
    \\
    \frac{\partial \textbf{v}}{\partial \phi}
    = 
    - b \sin \theta \sin \phi \textbf{p}_x
    + b \sin \theta \cos \phi\textbf{p}_y
\end{align*}
Taking the norm of the above vector cross product yields the relation : 
\begin{equation*}
    d\Gamma = \left|\frac{\partial \textbf{v}}{\partial \theta} 
    \times 
    \frac{\partial \textbf{v}}{\partial \phi} \right| d\theta d\phi 
    = b \sin\theta (a^2\sin^2\theta+ b^2 \cos^2\theta)^{1/2}d\theta d\phi. 
\end{equation*}
As it is required for the computation of the surface stress, let us in a first step compute the surface of the ellipsoid.
For oblate spheroid we find
\begin{equation}
    s_\alpha
    = 2\pi 
    \int_0^\pi 
    b \sin\theta (a^2\sin^2\theta+ b^2 \cos^2\theta)^{1/2} d\theta 
    =
    2b\pi \left( b + \frac{a^2}{\sqrt{a^2-b^2}}\text{ArcCos}(b/a)\right). 
    \label{eq:surfcae}
\end{equation} 
For instance, we consider oblate shape only. 

In the eigenbasis basis we can write $\textbf{H}_{ij} =H^{ab}\textbf{p}_i^a \textbf{p}^b_j$ where $H^{ab} = \frac{1}{a_a a_b} \delta^{ab}$, and it is implied Einstein summation on the index $a$ and $b$ only when these indices are note present on the left-hand side of the equation, i.e. on the first equation not on the second one. 
We also can re-write the classic dimensionless Cartesian coordinate as, $\textbf{r}_k = r^a \textbf{p}^a_k$. 
Using these coordinates and the fact that $\textbf{p}^i_k\textbf{p}^j_k = \bm\delta^{ij}$ we may show that, 
\begin{equation*}
    H_{ij}r_j
    = H^{ab} r^c p_i^a p_j^b p_j^c
    = H^{ab} r^b p_i^a
\end{equation*}
thus, 
\begin{equation*}
    H_{ij} r_j  H_{ik} r_k
    = 
    H^{ab} r^b  H^{ad} r^d 
    =
    \sum_i \frac{1}{a_i^4} r^i r^i
\end{equation*}
Thus, using the parametric formulation, 
\begin{equation*}
    H_{ij} r_j  H_{ik} r_k
    =
    \frac{1}{b^2}  \sin^2\theta \cos^2\phi
    +   \frac{1}{b^2} \sin^2\theta \sin^2\phi
    +   \frac{1}{a^2} \cos^2\theta 
    = \frac{1}{b^2}  \sin^2\theta 
    +   \frac{1}{a^2} \cos^2\theta 
\end{equation*}
and since we have, 
\begin{equation*}
    H_{ij} r_j  H_{jk} r_k
    = 
    H^{ab} r^b p_i^a
    H^{cd} r^d p_j^c,
\end{equation*}
the ratio can be written,
\begin{equation*}
    \frac{ H_{ik} H_{jl} :  r_kr_l}{  H_{ab}  H_{ac} r_br_c} 
    = p_i^a p_j^c \frac{H^{ab} r^b H^{cd} r^d}
    { \frac{1}{b^2}  \sin^2\theta 
    +   \frac{1}{a^2} \cos^2\theta }
\end{equation*}
According to what stated earlier we are really just interested in the component in the first principal direction, thus we multiply the preceding expression by $p_i^2p_j^2$ which gives, 
\begin{equation*}
    p_i^2p_j^2 \frac{ H_{ik} H_{jl} :  r_kr_l}{  H_{ab}  H_{ac} r_br_c} 
    =  \frac{H^{2b} r^b H^{2d} r^d}
    { \frac{1}{b^2}  \sin^2\theta 
    +   \frac{1}{a^2} \cos^2\theta }
    =  \frac{ \cos^2\theta}
    { \frac{a^2}{b^2}  \sin^2\theta 
    +    \cos^2\theta }
\end{equation*}
Thus, the scalar value $S$ is given by, 
\begin{align}
    S &= \intS{\left[
        \frac{1}{3} -  
        \frac{ \cos^2\theta}
        { \frac{a^2}{b^2}  \sin^2\theta 
    +    \cos^2\theta }
    \right]
    } 
    =  
    \frac{s_\alpha}{3}
    -
    \intS{
        \frac{ \cos^2\theta}
        { \frac{a^2}{b^2}  \sin^2\theta 
    +    \cos^2\theta }
    } 
    \\
    &= 
    \frac{s_\alpha}{3}
    - 2\pi \int_0^\pi\left[
        \frac{ \cos^2\theta}
        { \frac{a^2}{b^2}  \sin^2\theta 
    +    \cos^2\theta }
    \right]
    b \sin\theta (a^2\sin^2\theta+ b^2 \cos^2\theta)^{1/2}d\theta \\
    &= 
    \frac{s_\alpha}{3}
    - 2\pi 
    b^3 
    \int_0^\pi
         \cos^2\theta
    \sin\theta (a^2\sin^2\theta+ b^2 \cos^2\theta)^{- 1/2}d\theta \\
    % &= 
    % \frac{2}{3}b\pi\left(
    %     b 
    %     + \frac{3b^3}{a^2 - b^2}
    %     + a^2 \frac{\text{ArcCos}[b/a]}{(a^2 - b^2)^{1/2}}
    %     + 3 a^2 b^2 \frac{\text{ArcCosh}[b/a]}{(b^2 - a^2)^{3/2}}
    % \right)
    &= 
    \frac{2}{3}b\pi
    \frac{
            b \sqrt{a^2 - b^2}(a^2 + 2b^2)
            + (a^4  - 4 a^2 b^2)
            \text{ArcCos}(b/a)
    }{
    (a^2 - b^2)^{3/2}
    }
    \label{eq:S_solution}
\end{align}
That expression is derived assuming $a>b$. 



In the objective of finding a formula of the form $\textbf{S} = f(\bm\chi)$ we start to reformulate the above formula in terms of $\chi_I$ and $\chi_{II}$. 
% The problem at hand requires expression this expression in terms of $\chi_I = a^2 -1$ or $\chi_{II} = b^2 - 1 = (\chi_I+1)^{-1/2} -1$, or even $\bm\chi_\alpha$.
Thus, we apply the following substitutions, 
\begin{align*}
    a \to (\chi_I +1)^{1/2}\\
    b \to (\chi_I+1)^{-1/4},
\end{align*}
or in terms of $\chi_{II}$, 
\begin{align*}
    a \to (\chi_{II} +1)^{-1}\\
    b \to (\chi_{II}+1)^{1/2} 
\end{align*}
To obtain an expression in terms of either $\chi_I$ or $\chi_{II}$, 
we also may use the relation $b^2 = a^{-1}$. 
Thus, we reformulate the surface in term of $\chi_I$ and $\chi_{II}$ which gives us two formulas, 
\begin{align*}
    s_\alpha(\chi_I)
    = \frac{2\pi}{\sqrt{1+ \chi_I}}\left\{
        1 + \frac{(1+\chi_I)^{5/4} \text{ArcCosh}[(1+\chi_I)^{-3/4}]}{\sqrt{(1+\chi_I)^{-1/2}-(\chi_I+1)}}
    \right\}\\
    s_\alpha(\chi_{II})
    = 2\pi
    \left\{
        1 +\chi_{II} 
        + \frac{
            \text{ArcCosh}[(1+\chi_{II})^{3/2}]
            }{
                (1+\chi_{II})^{3/2}
                \sqrt{\chi_{II}+1- (1+\chi_{II})^{-2}}
                }
    \right\}
\end{align*}
we may consider only small values of $\chi_I$ and $\chi_{II}$ which gives,
\begin{align*}
    \frac{s_\alpha}{4\pi a_0^2}(\chi_I)
    = 
    1  
    + \frac{1}{10}\chi_I^2 
    + \frac{47}{420}\chi_I^3 
    % + \frac{53}{480}\chi_I^4
    + \mathcal{O}(\chi_I^4)\\ 
    \frac{s_\alpha}{4\pi a_0^2}(\chi_{II})
    = 
    1  
    + \frac{2}{5}\chi_{II}^2 
    + \frac{32}{105}\chi_{II}^3 
    % + \frac{53}{480}\chi_I^4
    + \mathcal{O}(\chi_{II}^4). 
\end{align*}
Notice that the first order term are note present in the expression. 
Thus, at the lowest order $s_\alpha/(4\pi a_0^2) - 1 \sim  \chi_I^2$. 
Additionally, a scalar quantity (here $s_\alpha$) is known to be function of a tensor (here $\bm\chi$) the functional form of $s_\alpha$ must be written as, 
\begin{equation*}
    s_\alpha 
    =C_1 + C_2 \bm\chi:\bm\chi + C_3  (\bm\chi:\bm\chi)^2 \ldots.
\end{equation*}
Where $C_i$ are constant we already know that $C_1 = 4\pi a_0^2$. 
We recall that the tensor $\bm\chi_\alpha$ used in the main text may be written, 
\begin{equation*}
    \bm\chi = \chi_I \textbf{pp} + \chi_{II}(\bm\delta - \textbf{pp})
\end{equation*}
which means that, 
\begin{equation*}
    \bm\chi : \bm\chi 
    = 
    \chi_I^2 
    + 
    2 \chi_{II}^2
    % = 
    % \chi_I^2 
    % + 
    % 2 [(\chi_{I}+1)^{-1/2} -1]^2
\end{equation*}

Therefore, we may re-write the surface as, 
\begin{equation*}
    s_\alpha^* 
    = \frac{1}{3}\left(
        2 s_\alpha^*(\chi_I)
        s_\alpha^*(\chi_{II})
    \right)
    = 
    1 + \frac{1}{15} (\bm\chi_\alpha:\bm\chi_\alpha)
    + \mathcal{O}(|\bm\chi|^3),
\end{equation*}
which finally gives us an expression for the particle surface. 
Note that the higher order term can also be obtained, 
\begin{equation*}
    s_\alpha^* 
    = 
    1 + \frac{1}{15} (\bm\chi:\bm\chi)
    - \frac{32}{315} [\bm\chi : (\bm\delta - \textbf{pp})]^3
    - \frac{47}{630} [\bm\chi : \textbf{pp}]^3
    + \frac{5}{63} [\bm\chi : (\bm\delta - \textbf{pp})]^4
    + \frac{53}{720} [\bm\chi : \textbf{pp}]^4
    \ldots
\end{equation*}
Unfortunatly at this stage we are constrained  to use the vector $\textbf{pp}$ since it is impossible to obtain scalar quantities as $\sim \chi_I$ from combination of $\bm\chi$. 

Now let us  apply the same reasoning but for the computation of $S$. 
By substitutions of $a$ and $b$ by $\chi_I$ and $\chi_{II}$ we obtain,
\begin{align*}
    S / v_\alpha a_0 =  \frac{4}{5} \chi_I - \frac{19}{35}\chi_I^2 +\mathcal{O}(\chi_I^3)\\
    S / v_\alpha a_0 =  - \frac{8}{5} \chi_{II} + \frac{8}{35}\chi_{II}^2+\mathcal{O}(\chi_{II}^3). 
    \label{eq:taylor_exp_S}
\end{align*}
which will be usefull. 



Let us now summary the results obtained here: 

\paragraph{The exact formula: }
The surface stress tensor may be finally written into its exact form: 
\begin{equation*}
    \intS{\bm\sigma_I^0}
    = \frac{2\gamma}{3}s_\alpha 
    + S[\textbf{pp} - \frac{1}{2}(\bm\delta - \textbf{pp})]
    \label{eq:exact_solution}
\end{equation*}
where the exact formula for $s_\alpha$ is given by \ref{eq:surface} and $S$ by \ref{eq:S_solution}.  

\paragraph{The second order formula: }
At second order we may notice that $s_\alpha =  1 + \frac{1}{15}(\bm\chi:\bm\chi)$ and that,
\begin{equation*}
    \bm\chi\cdot\bm\chi
    = 
    \chi_I^2 \textbf{pp}
    + 
    \chi_{II}^2 (\bm\delta - \textbf{pp}),
\end{equation*}
Therefore, using \ref{eq:exact_solution} and the approximations given in \ref{eq:taylor_exp_S} we may write at second order accuracy, 
\begin{align}
    \intS{\bm\sigma_I^0}
    = 
    \gamma \frac{8\pi a_0^2}{3} [
        1
        +
        \frac{1}{15}
        (\bm\chi:\bm\chi)
    ] 
    \bm\delta  
    + 
    \frac{4 v_\alpha \gamma}{5 a_0} \left\{
        \bm\chi
        - \frac{19}{28} [\bm\chi \cdot \bm\chi
        + \frac{15}{19}(\bm\chi:\textbf{pp})\textbf{pp}]
    \right\}
    +\mathcal{O}(|\bm\chi|^3)
    \label{eq:app2_formula}
\end{align}
where the first groups of terms correspond to the isotropic part of the surface stress and the second group to the anisotropic part. 


\paragraph*{The first order approximation:}
Keeping only the first order term in that approximation we obtain the simple results,
\begin{align}
    \intS{\bm\sigma_I^0}
    = 
    \frac{\gamma v_\alpha }{a_0} \left[
        2\bm\delta  
        + 
        \frac{4}{5} 
        \bm\chi_\alpha
        \right]
    +\mathcal{O}(|\bm\chi|^2).
    \label{eq:app1_formula}
\end{align}
If we average over all particle configuration this term reads, 
\begin{align}
    \pSavg{\bm\sigma_I^0}
    = 
    \frac{\gamma \phi_d }{a_0} \left[
        2\bm\delta  
        + 
        \frac{4}{5} 
        \bm\chi_p
        \right]
    +\mathcal{O}(|\bm\chi|^2).
\end{align}
Under that form we notice that we reached the same results as \citet{lhuillier1987phenomenology} if one account for the different definition used for $\bm\chi$. 


In conclusion, we have obtained the surface tress tensor for a spheroidal particle. 
It is shown to exhibit complicated relation with the principal value of deformation $\chi_I$ and $\chi_{II}$.
However, at first order nice tensor relations could be obtained. 


\section{Introduction}
\label{sec:intro_ellipse}
With the aim of developing more sophisticated average models, we discussed in the preceding chapter the potential of modeling droplet shape and rate of deformation. 
This approach could lead to more realistic modeling, as the \textit{Bond} number involved in the processes of interest (see \ref{part:intro}) are not always smaller than one (typically it is not the case for bubbles columns). 

Many studies have been conducted to determine the influence of droplet deformation on the rheology of suspensions 
\citep{goddard1967nonlinear,lhuillier1987phenomenology,rallison1984deformation,Cox1969deformation,stone1994dynamics,wu2002ellipsoidal,mwasame2017macroscopic}.
These studies introduce various methods to model the kinematics   and dynamics of particle deformation within an averaged model framework. 
Specifically, the authors considered the rheology of emulsions with neutrally buoyant droplets in linear flows, meaning they did not account for the possible relative velocity between the dispersed and continuous phases. 
In contrast, this work focuses on the buoyant motion of particles, which induces uniform relative motion between the phases. 
Consequently, the methods proposed in these studies are not directly applicable to our context, although some of their results may be used. 
On the other hand, in the context of the modeling of fiber suspensions, it is common to use averaged equations to describe the orientation of fibers in a suspension, as seen in works like \citet{wang2008objective}. 
Additionally, \citet{curtiss1956kinetic} derived a set of kinetic-theory-like averaged equations describing the mass and momentum conservation of axisymmetric solid particles, including the effect of inertia. 
% When particles are non-spherical, equations for particle orientation are often used.
As we will see, these equations share similarities with the deformable droplet equations and are therefore of interest for our purposes.
Therefore, these studies, which model particle orientation considering relative motion, are still valuable in our context. 
However, these equations do not consider the evolution of particle deformation and the rate of deformation, as they are limited to solid particles, it is therefore necessary to generalize those to the present context.  
% To the authors' knowledge, no other averaged models for deformable particles under sedimentation have been proposed since these studies. 

In this work, we aim to present a more general framework inspired by both the fiber media modeling the community studying deformable droplets. 
We focus on modeling emulsions with deformable fluid inclusions while accounting for the possibility of mean relative motion between phases. 
Specifically, we explore the possibilities to derive a set of averaged equations that describe particle deformation and orientation, as well as the rate of deformation and angular momentum of the particles.
We begin by describing the behavior of a single deformable drop and propose a minimal set of conservation laws to capture its orientation and deformation. 
In the perspective of applying these equations to the scenario of rising homogeneous buoyant suspension of droplets,  we limit this study to small deformations and consider the droplets shape to be spheroidal (and not ellipsoidal). 
This assumption is consistent with the behavior of rising droplets or bubbles under the influence of gravity, which are known to adopt an oblate spheroidal shape when low inertial effects are present \citep{taylor1964deformation}. 


This chapter is organized as follows:
% In \ref{sec:local_eq_ellipse} we introduce the Lagrangian conservation equations, which describe the evolution of particle mass, velocity, momentum, the second moment of mass, and moment of momentum conservation. 
By assuming, a priori, that the particle adopts an oblate spheroidal shape, we demonstrate how the equations of deformation and orientation naturally arise from the first moment of momentum and the second-order moment of mass of the particle.
Notably, it is shown that by expressing the particle properties in the eigenbasis of the particle second moment of mass, the number of equations is drastically reduced.
In \ref{sec:averaged_eq} we present the set of averaged equations for particle orientation and deformation. 
% Finally, in \ref{sec:particle_def} we utilize the averaged equation of rate of deformation to compute the deformation of buoyant droplets in dilute emulsions, as well as the averaged stress within the emulsion. 
% It is shown that the deformation is cased by the \textit{Stresslet} tensor, which also appears in the equivalent continuous phase stress.
% We conclude that the effect of small inertia generates a stress proportional to the square of the relative velocity, which in turns deform the droplets. 


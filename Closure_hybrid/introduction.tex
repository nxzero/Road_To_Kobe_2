\section{Introduction}
\label{sec:intro_ellipse}
Many studies have been conducted to determine the influence of the droplets' deformation on suspension Rheology \citet{goddard1967nonlinear,lhuillier1987phenomenology,maffettone1998equation,raja2010inertial}.
Most of the authors if not all of them considered the Rheology of neutrally buoyant droplets in linear flows. 
In fiber media modeling it is common to use averaged equations describing the averaged orientation of fiber in a suspension, for example see \citep{wang2008objective}. 
However, these equations (1) do not consider the particles' deformation, and (2) they neglect inertial contribution. 
In the study by \citet{curtiss1956kinetic}, a set of averaged equations describing the mass and momentum conservation of axisymmetric solid particles was derived including the effect of inertia. 
To the authors' knowledge, since the cited studies, no other extended model for non-isotropic deformable particles has been proposed. 
In this work, we would like to present a more general framework in which we consider the deformation of droplet's in arbitrary linear flows, including relative translation between phases. 
% We would like to generalize this approach including the effect of inertia, deformation as well as buoyancy forces. 

% Therefore, this chapter aims to propose a minimal set of averaged equations describing the mean particle shape and orientation of the particles. 
Therefore, in this chapter we explore the possibility to derive a set of averaged equations describing particle deformation, orientation and the kinematic counterpart of these properties. 
We first describe the behavior of a single deformable drop, and propose a minimal set of conservation laws to describe its orientation and deformation. 
Specifically, we consider deformable ellipsoidal particles, which might represent rising droplets or bubbles under the action of gravity, which are known to adopt an ellipsoidal shape \citep{taylor1964deformation} at low inertial effect.
After a complete presentation of the closure problem, we decide to focus on a simplifying senario. 
More specifically, we present effect of uniform phase relative motion on the Rheology at finite Reynolds number, and how this is closely related to the particle deformation. 



In organization of this chapter is as follows. 
In \ref{sec:local_eq_ellipse} we recall the general form of the Lagrangian equations governing the continuous and dispersed phases bulk quantities as well as the interfaces properties.
Then, in \ref{sec:Lagrange_ellipse} we introduce the Lagrangian conservation equation, describing the evolution of the particle mass, velocity, momentum, second-moment of mass and moment of momentum conservation. 
Then, by assuming, a priori, that the particle adopt an ellipsoidal shape we demonstrate that the equations of deformation and orientation arise naturally from the first moment of momentum and the second-order moment of mass of the particle.
In \ref{sec:averaged_ellipsoid} we expose the set of averaged equation for the particle orientation and deformation. 
It is shown that by expressing the particle's properties in the eigenbasis of the particle second moment of mass, we reduce drastically the number of equations. 
Lastly, in \ref{sec:particle_def} we make use of the averaged equation of deformation to compute the deformation of buoyant droplets in dilute emulsions. 
It is shown that the deformation is cased by the \textit{Stresslet} tensor, which also appears in the equivalent continuous phase stress.
We conclude that the effect of small inertia generates a stress proportional to the square of the relative velocity, which in turns deform the droplets. 


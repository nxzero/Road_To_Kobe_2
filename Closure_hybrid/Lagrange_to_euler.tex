
\subsection{From Lagrangian to Eulerian fields}

Up to this point, we have described the dispersed phase within a Lagrangian framework.
However, to be consistent with the Eulerian conservation equations used to describe the continuous phase, we need to extend the Lagrangian equations to an Eulerian models. 
In order to achieve this, we introduce the function $\delta_\alpha$, which is defined as follows, 
\begin{align}
    \delta_\alpha(\textbf{x},t) = \delta(\textbf{x}-\textbf{x}_\alpha(t,\FF)).
    \label{eq:delta_alpha}
\end{align}
where $\delta$ is the Dirac delta function.
Note that we explicitly note $\textbf{x}_\alpha(t,\FF)$ since the posiiton of the particle $\alpha$ is function of time and of the flow configuration $\FF$.
In the sens of generalized functions, the partial time derivative, $\pddt \delta_\alpha(\textbf{x},t,\FF) =  \frac{\partial \textbf{x}_\alpha}{\partial t} \cdot \grad_{\textbf{x}_\alpha} \delta_\alpha$ can be re-written into the following expression, 
\begin{equation}
    \pddt \delta_\alpha
    + \div (\textbf{u}_\alpha  \delta_\alpha)
    =0,
    \label{eq:dt_delta_alpha}
\end{equation}
where we used the identity, $\grad_{\textbf{x}_\alpha} \delta_\alpha = -\grad \delta_\alpha$ and the fact that $\textbf{u}_\alpha(t;\FF)$ is not a function of $\textbf{x}$. 
Additionally, it should be noted that \ref{eq:dt_delta_alpha} is not applicable if changes in topology, such as break up or coalescence events, occur.
In such cases it is possible, as it is done in population balance equations, to include a source term on the RHS of \ref{eq:dt_delta_alpha} to account for particle birth or death. 
Multiplying each Lagrangian quantities by $\delta_\alpha$ yields the \textit{particle field} of a quantity $q_\alpha$, denoted as $q_\alpha(t)\delta_\alpha(\textbf{x},t)$, which is defined throughout space and time.
Likewise, for any derivative of Lagrangian quantities, such as $\ddt q_\alpha$, we define its corresponding Eulerian field by Multiplying $\ddt q_\alpha$ with $\delta_\alpha$ and show that :
\begin{equation}
    \delta_\alpha \ddt q_\alpha
    = \pddt (\delta_\alpha q_\alpha)
    + \div (\delta_\alpha q_\alpha \textbf{u}_\alpha)
    \label{eq:dt_delta_alpha_q_alpha}
\end{equation}
where we have utilized the fact that $q_\alpha(t)$ and $\textbf{u}_\alpha(t)$ are solely functions of time, and we made use of \ref{eq:dt_delta_alpha}.
Additionally, let's consider a volume containing $N$ particles.
We can then define the particle-field of a given quantity $q_\alpha$ as the sum of all the independent field, i.e. $\sum_{\alpha=0}^N \delta_\alpha q_\alpha$.
Notice that \ref{eq:dt_delta_alpha_q_alpha} remains valid for a sum of fields since derivative operators are linear.
To simplify the notations, we consider implicitly the summation over all particles included in $\Omega$ whenever a Lagrangian field denoted by $\delta_\alpha (\ldots)$ is present.

Multiplying \ref{eq:dt_q_alpha_tot} and \ref{eq:dt_Q_alpha_tot} by $\delta_\alpha$, summing over all particles, and by considering \ref{eq:dt_delta_alpha_q_alpha}, it is straightforward to show that,
\begin{equation}
    \pddt (\delta_\alpha  q_\alpha^\text{tot})
    + \div (\delta_\alpha\textbf{u}_\alpha q_\alpha^\text{tot})
    = \delta_\alpha\intO{ s_2^0 }
    + \delta_\alpha\intS{ s_I^0 }
    + \delta_\alpha\intS{ \left[\mathbf{\Phi}_1^0 + f_1^0 (\textbf{u}_I^0-\textbf{u}_1^0) \right] \cdot \textbf{n}_2 },
    \label{eq:dt_dq_alpha_tot}
\end{equation}
\begin{multline}
    \pddt (\delta_\alpha  \mathcal{Q}_\alpha^\text{tot})
    + \div (\delta_\alpha\textbf{u}_\alpha \mathcal{Q}_\alpha^\text{tot})
    = \delta_\alpha\intO{ \left(
        \textbf{r} s_2^0         
        + f_2^0  \textbf{w}_2^0 
        - \mathbf{\Phi}_2^0
    \right) }\\
    + \delta_\alpha\intS{ \left(
        \textbf{r}s_I^0
        + f_I^0 \textbf{w}_I^0
        - \mathbf{\Phi}_{||I}^0
    \right) }
    + \delta_\alpha\intS{ \textbf{r} \left[
        \mathbf{\Phi}_1^0
        + f_1^0 (\textbf{u}_I^0-\textbf{u}_1^0)
    \right]\cdot \textbf{n}_2  }.
    \label{eq:dt_dQ_alpha_tot}
\end{multline}
Similar consideration can be applied to the higher order moments equations derived in \ref{ap:moment_derivative}.

At this stage, we obtained two sets of equations that can be used to describe the dispersed phase. 
The first set of equations is the global conservation laws, i.e. \ref{eq:dt_chi_k_f_k} with for $k=2$ and \ref{eq:dt_delta_I_f_I}. 
The other is the particle-fields equations, such as \ref{eq:dt_dq_alpha_tot} and potentially the higher moments equations.
Therefore, some comments are in order regarding the differences and compatibility of these two sets of equations.
Solving \ref{eq:dt_dq_alpha_tot} ideally provides us with a field $\delta_\alpha(q_\alpha+q_{I\alpha})$ which contains the Lagrangian properties $q_\alpha+q_{I\alpha}$.
Thus, it corresponds to the volume and surface integral of $f_2^0$ and $f_I^0$ on $\Omega_\alpha$ and $\Sigma_\alpha$ respectively.
While, in \ref{eq:dt_chi_k_f_k} we solve the equation for the complete field $f_2^0$ defined inside the domains $\Omega_\alpha$.  
Thus, from  \ref{eq:dt_f_k} to \ref{eq:avg_dt_dq_alpha_tot} we lose the detailed description of $f_2^0$ within the particles' domain.
Indeed, with \ref{eq:avg_dt_dq_alpha_tot}, we recover solely the integrated value of $f_2^0$ over the particles' volume and surface. 
Therefore, \ref{eq:dt_dq_alpha_tot} can be though as averaged equations of \ref{eq:dt_chi_k_f_k} and \ref{eq:dt_delta_I_f_I} since we recover only the integrated properties of each particle. 
It is important to understand that in this sense, the passage from \ref{eq:dt_chi_k_f_k} and \ref{eq:dt_delta_I_f_I} to \ref{eq:dt_dq_alpha_tot} is an average operation carried out on the particles' volume and surface.
Likewise, \ref{eq:dt_dQ_alpha_tot} is an equation for the first moment of the distribution of $f_2^0$ and $f_I^0$ within the particle's volume and surface.

% Note that this is different to the usual averaged technics that refer to the ones used to derive the classic averaged models such as in \citet{jackson1997locally} and \citet{zhang1994averaged}.
% which are the subject of the following section. 

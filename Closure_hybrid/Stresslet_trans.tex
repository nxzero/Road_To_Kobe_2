\section{Attempt to compute the stresslet of a dorp in translation}

\paragraph*{The real problem}
We consider the inertial motion of a translating spherical droplet. 
The dimensionless equations of motions for the disturbance field read \citet{chap:daniel1}, 
\begin{align*}
    \div\bm\sigma_k
    = 
    Re [
    \pddt \textbf{u}_k
    + \textbf{u}_k\cdot \grad \textbf{u}_k
    + \textbf{u}_k\cdot \grad \textbf{U}_f
    + \textbf{U}_f\cdot \grad \textbf{u}_k]
    = Re \textbf{f}_k\\
    \div \textbf{u}_k = 0
\end{align*}
where $\textbf{U}$ is the center of mass relative velocity. 
With the boundary condition at the surface of the particle, 
\begin{align*}
    \textbf{n}\cdot \textbf{u} 
    = \textbf{n}\cdot \textbf{U}_f=\textbf{n}\cdot \{
        - \textbf{U}
        - \textbf{r}\cdot \textbf{E}_f
        - \textbf{rr}\cdot \textbf{K}_f/2
    \}\\
    \textbf{u}_d = \textbf{u}_f\\
    \bm\sigma_d\cdot (\bm\delta - \textbf{nn}) = \bm\sigma_f\cdot (\bm\delta - \textbf{nn})
\end{align*}
for simplicity here we assume that, $\textbf{E}_f$ and $\textbf{K}_f =0$. 

We then consider small inertia effect such that,
\begin{align*}
    \textbf{u}_k = \textbf{u}_k^{(0)} + Re \textbf{u}_k^{(1)} + \ldots\\
    \textbf{f}_k = \textbf{f}_k^{(0)} + Re \textbf{f}_k^{(1)} + \ldots\\
    \bm\sigma_k = \bm\sigma_k^{(0)} + Re \bm\sigma_k^{(1)} + \ldots
\end{align*}
which results into the following system of equations, 
\begin{align*}
    \div \bm\sigma^{(0)}_k = 0 
    && 
    \div \textbf{u}_k^{(0)}=0
    \\
    \div \bm\sigma^{(1)}_k = \textbf{f}^{(0)}_k 
    && \div \textbf{u}_k^{(1)} =0 
    \\
\end{align*}
The boundary condition also can be written, 
\begin{align}
    \textbf{n}\cdot \textbf{u} 
    &= \textbf{n}\cdot \textbf{U}_f=\textbf{n}\cdot \{
        - \textbf{U}
        - \textbf{r}\cdot \textbf{E}_f
        - \textbf{rr} : \textbf{K}_f/2
    \}\\
    \textbf{u}_d &= \textbf{u}_f\\
    \bm\sigma_d^{(0)}\cdot (\bm\delta - \textbf{nn}) &= \bm\sigma_f^{(0)}\cdot (\bm\delta - \textbf{nn})
\end{align}
with the boundary conditions, 

\subsubsection*{title}
\paragraph*{The tool solution}

We now introduce what we call the tool problem. 
We now consider the disturbance velocity field $\hat{\textbf{u}}$ produced by a drop immersed in a linear flows and that respect the governing equations, 
\begin{align*}
    \div \hat{\bm\sigma}_k = 0 \\
    \div \hat{\textbf{u}}_k = 0 \\
    \textbf{n}\cdot \hat{\textbf{u}}=  \textbf{n}\cdot \hat{\textbf{U}}_f = \textbf{n}\cdot \textbf{U} + \textbf{r}\cdot \textbf{E}_f \cdot \textbf{n}\\
    \hat{\textbf{u}_f} = \hat{\textbf{u}_d}\\
    \hat{\bm\sigma_d}\cdot (\bm\delta - \textbf{nn}) = \hat{\bm\sigma_f}\cdot (\bm\delta - \textbf{nn})
\end{align*} 
Which solution yields, 
\begin{align*}
    \hat{\textbf{u}}_k = \mathcal{U}_k\cdot \textbf{U}+ \mathcal{E}_k : \textbf{E}_f\\
    \hat{\textbf{e}}_k = \mathcal{U}_k^e\cdot \textbf{U}+ \mathcal{E}_k^e : \textbf{E}_f\\
    \hat{\bm\sigma}_k\cdot \textbf{n}|_{r=1} = \mathcal{U}_k^\sigma\cdot \textbf{U}+ \mathcal{E}_k^\sigma : \textbf{E}_f
\end{align*}

\paragraph*{The reciprocal theorem for stokes flow on the exterior of the drop}


Multiplying the conservation equaiton by the tool solution and vice et versa gives, 
\begin{equation*}
    \hat{\textbf{u}}_k\cdot \div \bm\sigma_k^{(0)}
    = 
    \textbf{u}_k^{(0)}\cdot \div \hat{\bm\sigma}_k
\end{equation*}
which is reformulated as, 
\begin{equation*}
    \hat{\textbf{u}}_k\cdot \div \bm\sigma_k^{(0)}
    = 
    \textbf{u}_k^{(0)}\cdot \div \hat{\bm\sigma}_k
\end{equation*}
Integrating over the domain $V_f$ and noticing that the velocity field decay at infinity, 
\begin{equation*}
    \intS{\hat{\textbf{u}}_f\cdot  \bm\sigma_f^{(0)} \cdot \textbf{n}}
    \intOf{\hat{\textbf{u}}_f\cdot  (\div \bm\sigma_f^{(0)})}
    = 
    \intS{\textbf{u}_f^{(0)}\cdot  \hat{\bm\sigma}_f \cdot \textbf{n}}
    \intOf{\textbf{u}_f^{(0)}\cdot ( \div \hat{\bm\sigma}_f)}
\end{equation*}
Considering the stokes regime and using the identity $\textbf{u}_f = \textbf{u}_f +\textbf{U}_f-\textbf{U}_f$, 
\begin{equation*}
    \intS{\hat{\textbf{U}}_f\cdot  \bm\sigma_f^{(0)} \cdot \textbf{n}     }
    + \intS{(\hat{\textbf{u}}_f - \hat{\textbf{U}}_f)\cdot  \bm\sigma_f^{(0)} \cdot \textbf{n}}
    = 
    \intS{\textbf{U}_f\cdot  \hat{\bm\sigma}_f \cdot \textbf{n}}
    + \intS{(\textbf{u}_f^{(0)} - \textbf{U}_f)\cdot  \hat{\bm\sigma}_f \cdot \textbf{n}}
\end{equation*}
Note that, due to the boundary conditions, $(\textbf{u}_f^{(0)} - \textbf{U}_f)$ is a tangent vector. 
We are therefore able to use a reciprocal theorem-like for the inside of the particles. 

\paragraph*{Reciprocal on the droplet interior}
Since the objectif is to substitute the problematic integral in the previous expression we propose the following relation, 
\begin{equation*}
    (\hat{\textbf{u}}_d - \hat{\textbf{U}}_f)\cdot \div \bm\sigma_d^{(0)}
    = 
    (\textbf{u}_d^{(0)} - \textbf{U}_f)\cdot \div \hat{\bm\sigma}_d
\end{equation*}
Carrying the classical step of the reciprocal theorem we show that, 
\begin{equation*}
    \div [\bm\sigma_d^{(0)}  \cdot (\hat{\textbf{u}}_d - \hat{\textbf{U}}_f)]
    - \bm\sigma_d^{(0)} : \grad (\hat{\textbf{u}}_d - \hat{\textbf{U}}_f)
    = 
    \div [\hat{\bm\sigma}_d \cdot (\textbf{u}_d^{(0)} - \textbf{U}_f)]
    - \hat{\bm\sigma}_d : \grad (\textbf{u}_d^{(0)} - \textbf{U}_f)
\end{equation*}
Since both the undisturbed and disturbance fields are divergence free we may show that, 
\begin{align}
    \bm\sigma_d^{(0)} : \grad (\hat{\textbf{u}}_d - \hat{\textbf{U}}_f)
    = 
    2\textbf{e}_d^{(0)} : \grad (\hat{\textbf{u}}_d - \hat{\textbf{U}}_f)
    = 
    2\mu_d \textbf{e}_d^{(0)} : (\hat{\textbf{e}}_d - \hat{\textbf{E}}_f)\\
    \hat{\bm\sigma}_d : \grad (\textbf{u}_d^{(0)} - \textbf{U}_f)
    = 
    2\hat{\textbf{e}}_d : \grad (\textbf{u}_d^{(0)} - \textbf{U}_f)
    =
    2\mu_d \hat{\textbf{e}}_d : (\textbf{e}_d^{(0)} - \textbf{E}_f - \textbf{K}_f : (\textbf{r} \bm\delta + \bm\delta \textbf{r}))
\end{align}
Where we have used the relation, 
\begin{equation*}
    \grad \textbf{U}_f 
    = 
    \grad (\textbf{U} + \textbf{r}\cdot \textbf{E}_f + \textbf{rr} : \textbf{K}_f )
    =
    \textbf{E}_f 
    + \textbf{K}_{f,ijk} : (r_j \delta_{li} +r_i \delta_{lj})
    =
    \textbf{E}_f 
    +  (r_j \textbf{K}_{f,ljk} + r_i \textbf{K}_{f,ilk})
\end{equation*}
Thus, the expression may be re-written as, 
\begin{equation*}
    \div [\bm\sigma_d^{(0)}  \cdot (\hat{\textbf{u}}_d - \hat{\textbf{U}}_f)]
    + 2\mu_d \textbf{e}_d^{(0)} : (  \hat{\textbf{E}}_f)
    = 
    \div [\hat{\bm\sigma}_d \cdot (\textbf{u}_d^{(0)} - \textbf{U}_f)]
    + 2\mu_d \hat{\textbf{e}}_d : (  \textbf{E}_f + \textbf{K}_f : (\textbf{r} \bm\delta + \bm\delta \textbf{r}))
\end{equation*}
\tb{at this stage we could let $\grad \textbf{U}_f$ under the \textit{brut} form}

Integrating over the particle volume thus gives, 
\begin{equation*}
    - \intS{
         (\hat{\textbf{u}}_d - \hat{\textbf{U}}_f) \cdot \bm\sigma_d^{(0)}  \cdot \textbf{n}
    }
    + \intO{2\mu_d \textbf{e}_d^{(0)} : \hat{\textbf{E}}_f}
    = 
    - \intS{
     (\textbf{u}_d^{(0)} - \textbf{U}_f)\cdot \hat{\bm\sigma}_d \cdot \textbf{n}
    }
    +\intO{ 2\mu_d \hat{\textbf{e}}_d : [ \textbf{E}_f + \textbf{K}_f : (\textbf{r} \bm\delta + \bm\delta \textbf{r})] }
\end{equation*}

\paragraph*{Final form of the Reciprocal theorem: }
Adding the latter expression to the \textit{Reciprocal theorem} on the exterior of the droplet yields, 
\begin{equation*}
    \intS{\hat{\textbf{U}}_f\cdot  \bm\sigma_f^{(0)} \cdot \textbf{n}}
    + \intO{2\mu_d \textbf{e}_d^{(0)} : \hat{\textbf{E}}_f}
    = 
    \intS{\textbf{U}_f\cdot  \hat{\bm\sigma}_f \cdot \textbf{n}}
    +\intO{ 2\mu_d \hat{\textbf{e}}_d : [ \textbf{E}_f + \textbf{K}_f : (\textbf{r} \bm\delta + \bm\delta \textbf{r})] }
\end{equation*}
This, is the expression that will be used to compute the velocity field. 
At this stage we note that one must consider the interior rate of strain as a part of the solution. 

Now we may use the exact form of, $\hat{\textbf{U}}_f$, $\hat{\textbf{u}}_k$ and $\hat{\textbf{e}}_k$, and process by identification and obtain these relations, 
\begin{align*}
    \intS{  \bm\sigma_f^{(0)} \cdot \textbf{n}}
    % + \intO{2\mu_d \textbf{e}_d^{(0)} : \hat{\textbf{E}}_f}
    &= 
    \textbf{U}\cdot \intS{   \mathcal{U}_f^\sigma}
    + \textbf{E}_f : \intS{  \textbf{r} \mathcal{U}_f^\sigma}
    + \textbf{K}_f \vdots \intS{  \textbf{rr} \mathcal{U}_f^\sigma}\\
    &
    +\textbf{E}_f : \intO{ 2\mu_d  \mathcal{U}_d^e } 
    +\intO{ 2\mu_d \mathcal{U}_d^e :   (\textbf{r} \cdot \textbf{K}_f +\textbf{K}_f\cdot \textbf{r}) } 
    \\
    \intS{  \textbf{r}\bm\sigma_f^{(0)} \cdot \textbf{n}}
    + \intO{2\mu_d \textbf{e}_d^{(0)}}
    &= 
    \textbf{U}\cdot \intS{   \mathcal{E}_f^\sigma}
    + \textbf{E}_f : \intS{  \textbf{r} \mathcal{E}_f^\sigma}
    + \textbf{K}_f \vdots \intS{  \textbf{rr} \mathcal{E}_f^\sigma}\\
    &
    +\textbf{E}_f : \intO{ 2\mu_d  \mathcal{E}_d^e } 
    +\intO{ 2\mu_d \mathcal{E}_d^e :   (\textbf{r} \cdot \textbf{K}_f +\textbf{K}_f\cdot \textbf{r}) } 
    \\
    % \intS{\hat{\textbf{U}}_f\cdot  \bm\sigma_f^{(0)} \cdot \textbf{n}}
    % + \intO{2\mu_d \textbf{e}_d^{(0)} : \hat{\textbf{E}}_f}
    % = 
    % \intS{\textbf{U}_f\cdot  \hat{\bm\sigma}_f \cdot \textbf{n}}
    % +\intO{ 2\mu_d \hat{\textbf{e}}_d : [ \textbf{E}_f + \textbf{K}_f : (\textbf{r} \bm\delta + \bm\delta \textbf{r})] } \\
\end{align*}
The first int should gives faxen laws, the second however yields the Stresslet faxen law

If we assume that $\textbf{U}_f = \textbf{U}$ then the Stresslet is identically null in stoke flows. 

\subsection{Inertial components of the stress let}

We applies the reciprocal theorem on the particle inertial equation. 
This, yields, 
\begin{equation*}
    \hat{\textbf{u}}_k\cdot \div \bm\sigma_k^{(1)}
    = 
    \hat{\textbf{u}}_k\cdot \textbf{f}^{(0)}
    + \textbf{u}_k^{(0)}\cdot \div \hat{\bm\sigma}_k
\end{equation*}
which again can be reformulated as, 
\begin{equation*}
    \intS{\hat{\textbf{U}}_f\cdot  \bm\sigma_f^{(1)} \cdot \textbf{n}}
    + \intS{(\hat{\textbf{u}}_f - \hat{\textbf{U}}_f)\cdot  \bm\sigma_f^{(1)} \cdot \textbf{n}}
    + \intOf{\hat{\textbf{u}}_f\cdot  \textbf{f}_f^{(0)}}
    = 
    \intS{\textbf{U}_f\cdot  \hat{\bm\sigma}_f \cdot \textbf{n}}
    + \intS{(\textbf{u}_f^{(1)} - \textbf{U}_f)\cdot  \hat{\bm\sigma}_f \cdot \textbf{n}}
\end{equation*}
On the droplet interior we have,
\begin{equation*}
    \intS{
         (\hat{\textbf{u}}_d - \hat{\textbf{U}}_f) \cdot \bm\sigma_d^{(1)}  \cdot \textbf{n}
    }
    + \intO{2\mu_d \textbf{e}_d^{(1)} :\grad \hat{\textbf{U}}_f}
    + \intO{(\hat{\textbf{u}}_d - \hat{\textbf{U}}_f)\cdot  \textbf{f}_d^{(0)}}
    = 
    \intS{
     (\textbf{u}_d^{(1)} - \textbf{U}_f)\cdot \hat{\bm\sigma}_d \cdot \textbf{n}
    }
    +\intO{ 2\mu_d \hat{\textbf{e}}_d : \grad \textbf{U}_f}
\end{equation*}
Merging both formula gives, 
\begin{equation*}
    \intS{\hat{\textbf{U}}_f\cdot  \bm\sigma_f^{(1)} \cdot \textbf{n}}
    - \intO{2\mu_d \textbf{e}_d^{(1)} :\grad \hat{\textbf{U}}_f}
    = 
    \intS{\textbf{U}_f\cdot  \hat{\bm\sigma}_f \cdot \textbf{n}}
    -\intO{ 2\mu_d \hat{\textbf{e}}_d : \grad \textbf{U}_f}
    - \intOf{\hat{\textbf{u}}_f\cdot  \textbf{f}_f^{(0)}}
    + \intO{(\hat{\textbf{u}}_d - \hat{\textbf{U}}_f)\cdot  \textbf{f}_d^{(0)}}
\end{equation*}


Assuming $\hat{\textbf{U}}_f = \textbf{r}\cdot \hat{\textbf{E}}_f$ and $\textbf{U}_f = \textbf{U}$, 

\begin{equation*}
    \intS{\textbf{r}  \bm\sigma_f^{(1)} \cdot \textbf{n}}
    - \intO{2\mu_d \textbf{e}_d^{(1)}}
    = 
    \textbf{U}\cdot \intS{  \mathcal{E}_f^\sigma }
    % -\intO{ 2\mu_d \hat{\textbf{e}}_d : \grad \textbf{U}_f}
    - \intOf{\mathcal{E}_f \cdot  \textbf{f}_f^{(0)}}
    + \intO{(\hat{\textbf{u}}_d - \hat{\textbf{U}}_f)\cdot  \textbf{f}_d^{(0)}}
\end{equation*}

We may expect the first term on the right-hand side to vanish yielding a
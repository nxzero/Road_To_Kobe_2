
In this section we get more specific and investigate three different applications all of which are derived in the dilute limit neglecting all terms of $\mathcal{O}(\phi_2^2)$ or above.
In the first two application we revisit the momentum and energy equations for spherical droplets.
The first application focus on the classic case of a dilute emulsion in the stokes regime. 
The second application discus the influence of the droplets' internal motions on the energy balance equations. 
In the last application we study the impact of slightly inertial translating droplets, on the suspension rheology. 


\subsection{Closure to the momentum equations for a dilute mono disperse emulsion of spherical droplets}
In this section we revisit the application of  \citet[Appendix A]{zhang1997momentum} by considering the case of spherical buoyant droplets immersed in an arbitrary stokes flow in the dilute limit. 
% After presenting the closure relations for the momentum equation, we observe how the internal motions due to the fluid nature of the particle phase impact the averaged equations. 
Under these hypotheses, the first moment of momentum, second moment of mass and energy equations, can be entirely disregarded since the shape and internal velocity will be entirely determined by the external flow properties. 

\subsubsection*{The closure problem. }


Then the computation for the closures are rather straightforward, we gathered the results in \ref{ap:Translating_sphere} and read, 
\begin{align}
    \pSavg{\bm{\sigma}_1^0\cdot \textbf{n}_2} = 
    - \phi_2 \grad p_1
    + \frac{3\phi_2\mu_1}{2 a^2} 
    \left(\frac{3\lambda+2}{\lambda+1}\right) \textbf{u}_{f p} 
    + \frac{3\phi_2\mu_1}{4} \left(\frac{\lambda}{\lambda+1}\right)\grad^2_0\textbf{u}^\infty\\
    \label{eq:first_mom}
    \pavg{\intS{\textbf{r}\bm{\sigma}_1^0 \cdot \textbf{n}_2} - \intO{\mu_1 \textbf{e}^0_2}} 
    = - \phi_2 p_1 + 
    \frac{\mu_1 \phi_2}{2} \left(\frac{2+5\lambda}{1+\lambda}\right)
    % \left[
    %     1+\frac{\lambda}{2(5\lambda +2)}\grad^2
    %     \right]_0
         \left(\grad \textbf{u}_1+ (\grad \textbf{u}_1)^T\right)
        \\
        \pavg{\intS{(\bm{\sigma}_1^0 \cdot \textbf{n}_2)_ir_kr_l}} =
        - \frac{\rho_1\phi_2 a^2}{5}
        \left(
            \delta_{ik} (\grad p_1)_l
            +\delta_{il} (\grad p_1)_k
            +\delta_{kl} (\grad p_1)_i
        \right)\nonumber\\
        + \frac{3\mu_1\phi_2}{10}\left(\frac{5\lambda+1}{\lambda+1}\right)u_{fp,i}\delta_{kl}
        + \frac{3\mu_1\phi_2}{5}\left(\frac{1}{\lambda+1}\right)(u_{fp,k}\delta_{il}+u_{fp,l}\delta_{ki})
        \nonumber \\
        \pOavg{{\mu(\textbf{e}_2^0)_{ik} r_l}} =
        \frac{\phi_2\mu_1}{10}\left(\frac{1}{\lambda+1}\right)
        \left(
            2\delta_{ik}u_{fp,l}
            -3\delta_{kl}u_{fp,i}
            -3\delta_{il}u_{fp,k}
        \right)
\end{align}
where we introduced the fluid particle relative velocity $\textbf{u}_{fp} = \textbf{u}_1-\textbf{u}_p$. 
We recall that since we have considered no torque on the particle the first moment of the surface force traction is purely symmetric. 
Note that the second moment of momentum is present under the $\partial_k\partial_l$ operator in the moment of momentum equation, meaning that the skew-symmetic part of $\pavg{\intS{(\bm{\sigma}_1^0 \cdot \textbf{n}_2)_ir_kr_l}} $ and $\pSavg{{\mu(\textbf{e}_2^0)_{ik} r_l}}$ vanish in the momentum equation. 
Therefore, the second moment of surface traction force might be written, 
% \begin{equation*}
%     \pSavg{{\mu(\textbf{e}_2^0)_{ik} r_l}} 
%     % = \frac{4\mu \pi a^3}{15}\left(\frac{1}{\lambda+1}\right)
%     % \left(
%     %     2\delta_{ik}U_{l}
%     %     -3\delta_{kl}U_{i}
%     %     -3\delta_{il}U_{k}
%     % \right)
%     % + \frac{4\mu \pi a^3}{15}\left(\frac{1}{\lambda+1}\right)
%     % \left(
%     %     2\delta_{il}U_{k}
%     %     -3\delta_{lk}U_{i}
%     %     -3\delta_{ik}U_{l}
%     % \right)
%     = \frac{\mu_1\phi_2}{5}\left(\frac{1}{\lambda+1}\right)
%     \left(
%         6\delta_{kl}u_{fp,i}
%         + \delta_{ik}u_{fp,l}
%         + \delta_{il}u_{fp,k}
%     \right)
% \end{equation*}
% The difference between both terms that will appear in the equation is then, 
% \begin{multline*}
%     \frac{1}{2}\pavg{\intS{(\bm{\sigma}_1^0 \cdot \textbf{n}_2)_ir_kr_l}} -
%     \pSavg{{\mu(\textbf{e}_2^0)_{ik} r_l}} 
%     = \\
%     \frac{\mu_1\phi_2}{20}\left(\frac{15\lambda-21}{\lambda+1}\right)u_{fp,i}\delta_{kl}
%     +\frac{\mu_1\phi_2}{10}\left(\frac{1}{\lambda+1}\right)
%     \left(
%         \delta_{ik}u_{fp,l}
%         +\delta_{il}u_{fp,k}
%     \right)
% \end{multline*}
% Likewise, note that what matter is the divergence of the stress $\partial_k \sigma_{1,ik}^0$ and not the stress alone, consequently, applying the operator $\partial_l$ (as it is in the momentum equation) gives us, 
\begin{multline*}
    \partial_l[\frac{1}{2}\pavg{\intS{(\bm{\sigma}_1^0 \cdot \textbf{n}_2)_ir_kr_l}} -
    \pSavg{{\mu(\textbf{e}_2^0)_{ik} r_l}} ]
    = \\
    \frac{3 \mu_1 }{20}\left(\frac{5\lambda+7}{\lambda+1}\right)
    [
        \partial_k(\phi_2 u_{fp,i})
        + \partial_i(\phi_2 u_{fp,k})
    ]
    - \frac{\mu_1}{20}\left(\frac{15\lambda+19}{\lambda+1}\right)
    \delta_{ik} \partial_l(\phi_2 u_{fp,l})
\end{multline*}
where we neglected the terms related to the pressure gradient for same reason as \citet{jackson1997locally}. 
% Additionaly, one can note that the conditional average fluid velocity can be re write
% \begin{equation*}
%     \textbf{u}_1 
%     = 
%     \textbf{u}
%     +\phi_1 (
%         \textbf{u}_1
%         -\textbf{u}_2
%     )
%     \approx
%     \textbf{u}
%     +\phi_1 \textbf{u}_{fp}
% \end{equation*} 
% In such way the stress let  can be express as, 
% % \begin{equation}    
% %     \pavg{\intS{\textbf{r}\bm{\sigma}_1^0 \cdot \textbf{n}_2} - \intO{\mu_1 \textbf{e}^0_2}} =
% %     - \phi_2 p^\infty 
% %     + \frac{\mu_1 \phi_2}{2} \left(\frac{2+5\lambda}{1+\lambda}\right)
% %         \textbf{e}_{ik}
% %     + \frac{\mu_1 \phi_2}{2} \left(\frac{2+5\lambda}{1+\lambda}\right)
% %         \left(\partial_i {u}_{pf,j} + \partial_j u_{pf,i}\right)_0
% % \end{equation}
Before exposing the averaged equation it is important to understand that $\phi_2 = n_p v_p + \mathcal{O}(n_p^2)$ consequently, we will adopt this simplification in the following. 
Injecting every of these terms in the averaged momentum equations yields the fluid and particle phase averaged mass, and momentum equation system, 
\begin{align*}
    % \phi_1
    % +\phi_2 
    % = 
    \phi_1
    + n_pv_p + \frac{a^2 v_p}{10}\nabla n_p
    &= 1
    \\
    \pddt \phi_1
    + \div (
        \phi_1\textbf{u}_1
    )
    &= 
    0\\
    \pddt n_p
    + \div (
        n_p\textbf{u}_p
    )
    &= 
    0\\
    \pddt (\phi_1 \rho_1u_{1,i})  
    + \div (
        \phi_1 \rho_1u_{1,i}u_{1,k}
        + {\sigma}_{1,ik}^\text{eq}
    )
    &=  \rho_1 g_i 
    -  \frac{3n_p v_p \mu_1}{2 a^2} 
    \left(\frac{3\lambda+2}{\lambda+1}\right) u_{f p,i} 
    - \frac{3n_p v_p \mu_1}{4} \left(\frac{\lambda}{\lambda+1}\right)\grad^2u_{1,i}\\
    \pddt \left(n_p v_p \rho_2 {u}_{p,i}\right)
    + \partial_{k} \left(n_p v_p \rho_2 {u}_{p,i} {u}_{p,k} 
    + {\sigma}_{p,ik}^\text{eq}
    \right)
    &= 
    n_p v_p (\rho_2 -\rho_1)g_i 
    + \frac{3n_p v_p \mu_1}{2 a^2} 
    \left(\frac{3\lambda+2}{\lambda+1}\right) u_{f p,i} 
    + \frac{3n_p v_p \mu_1}{4} \left(\frac{\lambda}{\lambda+1}\right)\grad^2u_{1,i}
\end{align*}
\tb{put $\phi$ on the right}
the equivalent stress tensor is then, 
% \begin{multline*}
%     - \mu_1 \textbf{e}_{ik} 
%     - \mu_1 \phi_2 \left(\frac{2+5\lambda}{2(1+\lambda)}\right)
%     \left(\partial_k u_{1,i}+\partial_i u_{1,k}\right)
%     = - \mu_1 \textbf{e}_{ik} \left[1 + \phi_2 \left(\frac{2+5\lambda}{2(1+\lambda)}\right)\right]\\
%     - \mu_1 \phi_2 \left(\frac{2+5\lambda}{2(1+\lambda)}\right)
%             \left(\partial_k u_{fp,i}+\partial_i u_{fp,k}\right)    
% \end{multline*}
\begin{align*}
    \bm{\sigma}^\text{eq}_{1,ik} =
    \rho_1\avg{\chi_1\textbf{u}_1'\textbf{u}_1'}_{ik} 
    + p_1 \delta_{ik}
    - 2 \mu_{Ein}(\phi_2) \textbf{e}\\
    + \mu_{Rel} \left[\grad (\phi_2\textbf{u}_{fp}) + ^t\grad (\phi_2\textbf{u}_{fp})\right]
    + \lambda_\text{Rel}
        \delta_{ik} \partial_l(\phi_2 u_{fp,l})\\
        % - \frac{\rho_1 a^2}{10}
        % \left(
        %     \delta_{ik} g_l
        %     +\delta_{il} g_k
        %     +\delta_{kl} g_i
        % \right)
        % \partial_l \phi_1
    \bm\sigma_p^\text{eq}
    = \pavg{m_\alpha\textbf{u}_\alpha'\textbf{u}_\alpha'}
\end{align*}

\tb{maybe think about reputting the $n_p$ which is more rigorous}
\tb{hyperbolicite du system ? }
\tb{talk about non homogeneous terms }
% \begin{align}
%     \bm{\sigma}^\text{eq}_{1,ik} =& 
%     p_1 \delta_{ik}
%     + \rho_1\avg{\chi_1\textbf{u}_1'\textbf{u}_1'}_{ik} 
%     - \mu_1 \textbf{e}_{ik} \left[1 + \frac{\phi_2}{2} \left(\frac{2+5\lambda}{1+\lambda}\right)\right]
%     - \frac{\mu_1 \phi_2}{20} \left(\frac{35\lambda-1}{1+\lambda}\right)
%         \left(\partial_k u_{fp,i}+\partial_i u_{fp,k}\right)\nonumber \\
%         &+ \frac{3 \mu_1}{20}\left(\frac{5\lambda+7}{\lambda+1}\right)
%         [
%             u_{fp,i} \partial_k \phi_2
%             + u_{fp,k} \partial_i \phi_2 
%         ]
%         - \frac{\mu_1}{20}\left(\frac{15\lambda+19}{\lambda+1}\right)
%         \delta_{ik} \partial_l(\phi_2 u_{fp,l})
%         % - \frac{\rho_1 a^2}{10}
%         % \left(
%         %     \delta_{ik} g_l
%         %     +\delta_{il} g_k
%         %     +\delta_{kl} g_i
%         % \right)
%         % \partial_l \phi_1
%     % \bm\sigma_p^\text{eq}
%     % =& \pavg{m_\alpha\textbf{u}_\alpha'\textbf{u}_\alpha'}
% \end{align}
where we have reformulated the first moment of surface force in terms of bulk rate of strain $2\textbf{e} = \grad \textbf{u} +  ^t \grad \textbf{u}$ using the relation, $\textbf{u}_1 \approx \textbf{u} + \phi_2 (\textbf{u}_1 - \textbf{u}_2)$ in \ref{eq:first_mom} and noticing that the contribution of the relative velocity is of order $\mathcal{O}(\phi_2^2)$ and therefore negligible. 
The first component of the bulk stress is a macroscopic Newtonian stress, it is constituted with a pressure term $p_1 \textbf{I}$ independent of $\phi_2$, and an equivalent viscosity terms $\textbf{e}\mu_\text{Ein}(\phi_2)$, with $\mu_\text{Ein}(\phi_2) = \mu_1\left[1 + \frac{\phi_2}{2} \left(\frac{2+5\lambda}{1+\lambda}\right)\right]$ which is linear with the particle volume fraction. 
The second contribution relates the gradient of $\phi_2 \textbf{u}_{fp}$ to the stress via the second viscosity coefficient $\mu_\text{Rel} = \frac{3 \mu_1 }{20}\left(\frac{5\lambda+7}{\lambda+1}\right)$.
This stress is the contribution of either a gradient of particle concentration or strong gradient of particle relative phase velocity gradient. 
The last term is proportional to the divergence of the relative velocity, the coefficient $\lambda_\text{Rel}= - \frac{\mu_1}{20}\left(\frac{15\lambda+19}{\lambda+1}\right)$ can be assimilated to a compression related parameters on the fields $\phi_2\textbf{u}_{fp}$. 
Under this form it is clear that the bulk stress remain symmetric as long as no external torque is applied on the particles. 

One last remark that has been overlooked in the literature. 
The rate of strain \textbf{e} require for the bulk velocity \textbf{u} which is express at the first order in the volume fraction : $\textbf{u} = \textbf{u}_1 \phi_1 + \textbf{u}_pv_p - \div(n_p \mathcal{P}_p)$.
However, in this context, the moment of momentum  posses only a skew symmetric part since $\mathcal{S}_p = 0$. 
Nevertheless, in the momentum equation $\mathcal{P}_{li}$, appear under the operator $\partial_l\partial_i$ which cancel out its contribution. 
Therefore, under these conditions we might write $\partial_l\partial_i \mathcal{P}_{p,li} = \partial_l\partial_i (n_p \mathcal{S}_{p,li}) = 0$. 
Note that for non-spherical particles where $\mathcal{S}_p \neq 0$ have to keep this term. 

The last unclosed terms in these expressions are the Reynolds stress, although it is supposed to be negligible in the low Reynolds number regime it might not always be negligible due to collective motion of particles \citet{zhang1997momentum}. 
One way to close this term theoretically is to follow an approach similar than \citet{van1982bubble} for potential flow in which case we find that $\rho_1\avg{\chi_1\textbf{u}_1'\textbf{u}_1'} \sim \textbf{u}_pf\cdot \textbf{u}_pf$.
Therefore, the contribution of the relative velocity might appear explicitly in the stress. 
Nevertheless, this approach doesn't work in the case of stokes flow due to the problem of non-convergent integrals. 
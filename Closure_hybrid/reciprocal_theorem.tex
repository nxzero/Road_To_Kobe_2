\section{The reciprocal theorem and equaitons for a droplet force and Stresslet}
\label{ap:reciprocal}


In this section we revisit the methodology of \citet{stone2001inertial} and \citet{raja2010inertial}  to compute the Stresslet and forces on a droplet in a general stokes flow at first order in the Reynolds number. 
The improvement or differences with the two cited studies which are similar in a lots of way with what is proposed here are : 
(1) we generalized the work of \citet{stone2001inertial} by considering spherical droplets instead of solid particles. 
(2) In the work of \citet{raja2010inertial} they already considered droplets, however in opposition to their work we focus on the particles relative motions with the continuous phase where they considered neutrally buoyant particles. 
Overall we provide a general method greatly inspired based on the work of \citet{stone2001inertial} to compute the closure terms at the first order in $\phi$ and $Re$ for spherical droplets. 


\subsection{Introduction of the problem}

We consider a dilute emulsion of spherical droplets at low but finite inertial effects. 
In this situation, we have shown that the relevant equations to be solved to compute the closure terms of the averaged equations are the disturbance fields equations, or the conditionally averaged equations. 
For ease of understanding we note here $\textbf{u}_k$ the disturbance fields of the phase $k$ conditionally on the presence of a particle at \textbf{x} with center of mass velocity \textbf{w}.
In this case $\textbf{u}_k[\textbf{y}|\textbf{x},\textbf{w}] + \textbf{U}_f[\textbf{y}]$ represents the velocity field seen by an observer in the laboratory reference frame, and $\textbf{u}_k[\textbf{y}|\textbf{x},\textbf{w}] + \textbf{U}_f[\textbf{y}] - \textbf{w}$ the velocity field seen by an observer in the particle center of mass reference frame.
Note that the averaged velocity field $\textbf{U}_f$ can be re-written in terms of a taylor expansion around the particle center of mass such that, $\textbf{U}_f[\textbf{y}] = \textbf{U}_f[\textbf{x}] + \textbf{r} \cdot \grad \textbf{U}_f + \ldots$, nevertheless for purpose of generality we keep the more general notation $\textbf{U}_f[\textbf{y}]$. 
Thus, in the dilute limit we may write in the dimensionless form the following set of equations (see \ref{chap:daniel2}, \citep{stone2001inertial}) for the fluid phase, 
\begin{align*}
    \div\bm\sigma_f
    = 
    Re [
    \pddt \textbf{u}_f
    + \textbf{u}_f\cdot \grad \textbf{u}_f
    + \textbf{u}_f\cdot \grad (\textbf{U}_f - \textbf{w})
    + (\textbf{U}_f - \textbf{w})\cdot \grad \textbf{u}_f]
    = Re \textbf{f}_f\\
    \div \textbf{u}_f = 0
\end{align*}
and, 
\begin{align*}
    \div\bm\sigma_d
    = 
    \frac{\zeta}{\lambda}Re [
    \pddt \textbf{u}_d
    + \textbf{u}_d\cdot \grad \textbf{u}_d
    + \textbf{u}_d\cdot \grad (\textbf{U}_f - \textbf{w})
    + (\textbf{U}_f - \textbf{w})\cdot \grad \textbf{u}_d]
    = \frac{\zeta}{\lambda}Re \textbf{f}_d\\
    \div \textbf{u}_d = 0
\end{align*}
for the dispersed phase. 
We recall that $Re = \frac{\rho_f d U}{\mu_f}$ is the Reynolds number, $\zeta = \rho_d / \rho_f$ the density ratio and $\lambda = \mu_d / \mu_f$ the viscosity ratio, with $U = |\textbf{U}_f[\textbf{x}] - \textbf{w}|$. 
The boundary conditions valid at the particle surface are, 
\begin{align}
    \textbf{n}\cdot\textbf{u}_f
    &= \{\textbf{w} - \textbf{U}_f[\textbf{y},t]\}\cdot \textbf{n} 
    = \textbf{U}[\textbf{y},t]\cdot \textbf{n} \\
    \textbf{u}_f &= \textbf{u}_d\\
    (\bm\sigma_f\cdot\textbf{n})\cdot (\bm\delta - \textbf{nn})
    &= \lambda (\bm\sigma_d\cdot \textbf{n})(\bm\delta - \textbf{nn}),
    \label{eq:bc_stress_orig}
\end{align}
which are completed by the following boundary condition at infinity,
\begin{align*}
    \lim_{|\textbf{r}|\to\infty }\textbf{u}_f[\textbf{y}|\textbf{x}] = 0. 
\end{align*}
We introduced the relative velocity field $\textbf{U}[\textbf{y},t]$. 
It is clear that solving this problem into all its generality is not feasible theoretically.
Thus, we now perform some simplifying hypothesis. 

\subsection{The low Reynolds number expansion method}

To simplify the problem we now consider small \textit{Reynolds} numbers perturbation methods. 
It is well known that when performing expansion methods one must perform an ``outer'' and an ``inner'' expansion to obtain the correct correction on the drag force at the leading order \citet{proudman1957expansions}. 
However, as shown in \citet{masoud2019reciprocal} the ``outer'' solution provides only an $\mathcal{O}(Re^{3/2})$ on the closure terms of interest in our work.
Therefore, as done in \citet{raja2010inertial} we may completely disregard the ``outer'' solution and only focus on the ``inner'' expansion thereby obtaining a result accurate at $\mathcal{O}(Re)$ with an error of $\mathcal{O}(Re^{3/2})$. 

Anyhow, the disturbance velocity field of phase $k$ may be re-written as an expansion around $Re \approx 0$ namely, 
\begin{equation*}
    \textbf{u}_k = \textbf{u}_k^{(0)} + Re \textbf{u}_k^{(1)} + \ldots
\end{equation*}
with similar expression for the fields $\bm\sigma_k$ and $\textbf{f}_k$. 
Substituting these expansions into the governing equations gives two system of equations, one for the zeroth order terms $\textbf{u}^{(0)}_k$, and another for the first order terms $\textbf{u}^{(1)}_k$. 
Consequently, the governing equations for the continuous phase yield, 
\begin{align}
    \div\bm\sigma_f^{(0)}
    = 0,
    && \div \textbf{u}_f^{(0)} = 0 
    \label{eq:zeroth_order_NS_f}
    \\
    \div\bm\sigma_f^{(1)},
    =  \textbf{f}_f^{(0)},
    && \div \textbf{u}_f^{(1)} = 0.  
    \label{eq:first_order_NS_f}
    \\
    &&\vdots
\end{align}
While in the dispersed phase we have, 
\begin{align}
    \div\bm\sigma_d^{(0)}
    = 0,
    && \div \textbf{u}_d^{(0)} = 0 
    \label{eq:zeroth_order_NS_d}
    \\
    \div\bm\sigma_d^{(1)},
    = \frac{\zeta}{\lambda} \textbf{f}_d^{(0)},
    && \div \textbf{u}_d^{(1)} = 0. 
    \label{eq:first_order_NS_f}
    \\
    &&\vdots
\end{align}
The three verticals dots indicates that one might eventually add an arbitrary number of equations for the $\textbf{u}^{(n)}_k$. 
However, as mentioned above if one do so he would then need to solve for the ``outer'' field  expansion. 
Thus, in the present study only the zeroth and first order equations are relevant. 
The velocity boundary condition can also be written as, 
\begin{align}
    \textbf{n}\cdot\textbf{u}_f^{(0)}
    = \textbf{U}[\textbf{y},t]\cdot \textbf{n} \\
    \textbf{n}\cdot\textbf{u}_f^{(1)}
    = 0
    \label{eq:bc_inertial}
\end{align}
Indeed, by identification we deduce that since $\textbf{u}_f^{(0)}$ must match exactly the undisturbed velocity field at the particle surface the first Reynolds correction projected on the normal of the surface is simply zero.

\subsection{A non-inertial spherical droplet in linear flows.}

The Reciprocal theorem requires the use of a similar known solution of the Navier Stokes equation.
We call that solution the \textit{tool} solution as its only purpose is to compute the \textit{real} solutions of the equations introduced above. 
We note the tool velocity and stress fields of phase $k$ as $\hat{\textbf{u}}_k$ and $\hat{\bm\sigma}_k$. 
Originally, as a tool solution one consider the situation of a fixed particle immersed in a velocity field generated by a point force on its exterior, in which case the Reciprocal theorem end up giving the famous Faxen laws.
Instead, we adopt a method similar to \citet{stone2001inertial} and use as a tool solution, the solution of the problem of a non-inertial spherical droplet immersed in a linear flows. 

In this situation the velocity and stress fields are governed by, 
\begin{align*}
    \div \hat{\bm\sigma}_k = 0 
    && \div \hat{\textbf{u}}_k = 0 
    \label{eq:tool_sol}
\end{align*}
in phase $k$ and, 
\begin{align}    
    \textbf{n}\cdot \hat{\textbf{u}}=  \textbf{n}\cdot(\textbf{w} - \hat{\textbf{U}}_f) = \textbf{n}\cdot \hat{\textbf{U}} + \textbf{r}\cdot \textbf{E}_f \cdot \textbf{n}\\
    \hat{\textbf{u}_f} = \hat{\textbf{u}_d}\\
    (\hat{\bm\sigma_f}\cdot \textbf{n}) \cdot (\bm\delta - \textbf{nn})
    = 
    \lambda (\hat{\bm\sigma_d}\cdot \textbf{n}) \cdot (\bm\delta - \textbf{nn}) \label{eq:bc_stress}
\end{align} 
on the interface. 
Where we recall that $\hat{\textbf{U}} = \textbf{w} - \hat{\textbf{U}}_f$, and that $\hat{\textbf{U}}_f$ is linear with $\textbf{r}$ such that $ \hat{\textbf{U}}_f[\textbf{y}] = \hat{\textbf{U}}_f[\textbf{x}]+\textbf{r}\cdot \hat{\textbf{E}}_f[\textbf{x}]$ in opposition to $\textbf{U}_f$ which were not assumed linear but rather arbitrary. 

This problem has been treated extensively in the literature, \citet{pozrikidis1992boundary,leal2007advanced},  the solutions read, 
\begin{align}
    \hat{\textbf{u}}_k = \mathcal{U}_k\cdot \hat{\textbf{U}}+ \mathcal{E}_k : \textbf{E}_f\\
    \hat{p}_k = \mathcal{U}_k^p\cdot \hat{\textbf{U}}+ \mathcal{E}_k^p : \textbf{E}_f\\
    \hat{\textbf{e}}_k = \mathcal{U}_k^e\cdot \hat{\textbf{U}}+ \mathcal{E}_k^e : \textbf{E}_f\\
    \hat{\bm\sigma}_k = \mathcal{U}_k^\sigma\cdot \hat{\textbf{U}}+ \mathcal{E}_k^\sigma : \textbf{E}_f
    \label{eq:solution_hat}
\end{align}
with, 
\begin{align*}
    (\mathcal{U}_{f})_{ik} &= 
    \delta_{ik}
    + \frac{1}{4}\left(\frac{3\lambda + 2}{\lambda +1}\right)
    \left(\frac{\delta_{ik}}{r} + \frac{r_ir_k}{r^3}\right) 
    + 
    \frac{1}{4}\left(\frac{\lambda}{\lambda +1}\right)
    \left(\frac{\delta_{ik}}{r^3} - \frac{3r_ir_k}{r^5}\right)  \\
    (\mathcal{E}_{f})_{ijk}
    &=
    %  \bm\delta\textbf{r}
    -\frac{\lambda}{(\lambda + 1)r^5} \bm\delta\textbf{r}
    -\left(
        5\lambda +2
        - \frac{5\lambda}{r^2}
        \right) 
    \frac{\textbf{rrr}}{2(\lambda+1)r^5}\\
    (\mathcal{U}_{d})_{ik} &= 
    \frac{1}{2}\left(\frac{2\lambda +3}{\lambda +1}\right)\bm\delta
    -\frac{1}{2} (2r^2 \bm\delta - \textbf{rr})
    \left(\frac{1}{\lambda +1}\right)\\
    (\mathcal{E}_{d})_{ijk}
    &= -\bm\delta \textbf{r}
    + \frac{5r^2 -3}{2(\lambda +1)} \textbf{r}\bm\delta
    - \frac{1}{\lambda+1}\textbf{rrr}\\
    \mathcal{U}_f^p &= 
      \frac{1}{2}\left(\frac{3\lambda + 2}{\lambda +1}\right) \frac{\textbf{r}}{r^3}\\
     \mathcal{U}_d^p &= 
    - 5 \left(\frac{1}{\lambda +1}\right) \textbf{r}\\
    \mathcal{E}_f^p &= - \frac{(5\lambda+2)}{(\lambda+1)r^5}\textbf{rr}\\
    \mathcal{E}_d^p &= \frac{21\lambda}{2(\lambda+1)} \textbf{rr}\\
    (\mathcal{U}_k^e)_{ijk}
    &= 
    \frac{1}{2}(
    \partial_j 
    \mathcal{U}_{f,ik}
    + 
    \partial_i 
    \mathcal{U}_{f,jk}
    )\\
    (\mathcal{E}_k^e)_{ijkl}
    &= 
    \frac{1}{2}(
    \partial_j 
    \mathcal{E}_{f,ikl}
    + 
    \partial_i 
    \mathcal{E}_{f,jkl}
    )\\
    (\mathcal{U}_k)_{ijk}^\sigma  n_j 
    &= 
     - (\mathcal{U}_k^p)_{k}\delta_{ij}
     + 2(\mathcal{U}_k^e)_{ijk}
     = - \frac{2}{\lambda +1}\left(
         \frac{3}{4}\lambda\delta_{ik} 
         + \frac{3}{2} n_in_k
     \right)
     \\
     (\mathcal{E}_k^\sigma)_{ijkl}
     &=
     - (\mathcal{E}_k^p)_{kl}\delta_{ij}
     + 2(\mathcal{E}_k)_{ijkl}
\end{align*}
Particularly, note the linearity of the velocity, pressure and stress fields with the relative velocity $\hat{\textbf{U}}$ and the shear rate $\hat{\textbf{E}}$. 
We can also gives the exact expression, 
% \begin{align*}
%     (\mathcal{U}_k^e)_{ijk}
%     = - \frac{1}{2}\left(\frac{1}{1+\lambda}\right)(
%         \frac{3}{2}\delta_{ik} r_j
%         + \frac{3}{2}\delta_{jk} r_i
%         - r_k \delta_{ij} 
%         )
% \end{align*}
% Or, 
\begin{equation*}
    (\mathcal{U}_d^e)_{ijk} 
    % = 
    % 3(-4\delta_{ik} r_j + \delta_{ij} r_k + \delta_{kj} r_i)
    % + 10 r_j  \delta_{ik}
    = 
    - \frac{1}{2\lambda}\left(\frac{1}{1+\lambda}\right)(-2 \delta_{ik} r_j + 3\delta_{ij} r_k + 3\delta_{kj} r_i) 
\end{equation*}

\subsection{The general form of the reciprocal theorem for stokes flows}

In a first step we consider only the calculation of the closure based on $\textbf{u}^{(0)}_k$ meaning for a spherical drop immersed in an arbitrary stokes flow fields. 

\subsubsection{The continuous phase integral relation}

Let us take the dot product of \ref{eq:zeroth_order_NS_f} with $\hat{\textbf{u}}_f$ and \ref{eq:tool_sol} (with $k = f$) with $\textbf{u}_k^{(0)}$, then subtracting both expressions gives, 
\begin{equation*}
    \hat{\textbf{u}}_f\cdot \div\bm\sigma_f^{(0)}
    =
    \textbf{u}_k^{(0)} \cdot \div \hat{\bm\sigma}_k, 
\end{equation*}
Upon integrating over $\Omega_f$ we obtain the following relation, 
\begin{equation*}
    \intS{\hat{\textbf{u}}_f\cdot  \bm\sigma_f^{(0)} \cdot \textbf{n}}
    % \intOf{\hat{\textbf{u}}_f\cdot  (\div \bm\sigma_f^{(0)})}
    = 
    \intS{\textbf{u}_f^{(0)}\cdot  \hat{\bm\sigma}_f \cdot \textbf{n}}
    % \intOf{\textbf{u}_f^{(0)}\cdot ( \div \hat{\bm\sigma}_f)}
\end{equation*}
For solid particle this relation constitute the basis to compute the stress distribution on the particle, whihc corresponds to the left-hand side term. 
However, note that for droplets the velocity fields $\textbf{u}_f^{(0)}$ is unknown.
Thus, using the identity $\textbf{u}_f = \textbf{u}_f +\textbf{U}_f-\textbf{U}_f$ we reformulate the reciprocal theorem as, 
\begin{equation}
    \intS{\hat{\textbf{U}}\cdot  \bm\sigma_f^{(0)} \cdot \textbf{n}}
    + \intS{(\hat{\textbf{u}}_f - \hat{\textbf{U}})\cdot  \bm\sigma_f^{(0)} \cdot \textbf{n}}
    = 
    \intS{\textbf{U}\cdot  \hat{\bm\sigma}_f \cdot \textbf{n}}
    + \intS{(\textbf{u}_f^{(0)} - \textbf{U})\cdot  \hat{\bm\sigma}_f \cdot \textbf{n}}
    \label{eq:reciprocal_f}
\end{equation}
The first integral on the left-hand side will be the results of this study, the order integral remains to be calculated. 
However, note that the unkown, $\bm\sigma_f^{(0)}$ appears on the  left-hand side of this equation. 
As the solution can not include itself one needs to reformulate this integral. 
In this objective, we note that both term $(\textbf{u}_f^{(0)} - \textbf{U})$ and $(\hat{\textbf{u}}_f - \hat{\textbf{U}})$ are by definition tangent vector to the particle surface. 
Considering the boundary condition \ref{eq:bc_stress} and \ref{eq:bc_stress_orig} way deduce that $(\hat{\textbf{u}}_f - \hat{\textbf{U}})\cdot  \bm\sigma_f^{(0)} = \lambda (\hat{\textbf{u}}_d - \hat{\textbf{U}})\cdot  \bm\sigma_d^{(0)}$ and  $({\textbf{u}}_f^{(0)} - \hat{\textbf{U}})\cdot  \hat{\bm\sigma}_f = \lambda (\hat{\textbf{u}}_d^{(0)} - \hat{\textbf{U}})\cdot  \hat{\bm\sigma}_d$
This introduces the needs for a reciprocal theorem on the droplets internal volume. 

\subsubsection{The dispersed phase integral relation}

We take the dot product of \ref{eq:zeroth_order_NS_d} with $\hat{\textbf{u}}_d - \hat{\textbf{U}}$ and \ref{eq:tool_sol} (with $k=d$) with $\textbf{u}_d^{(0)} -\textbf{U}$, then subtracting both expressions gives, 
\begin{equation*}
    (\hat{\textbf{u}}_d - \hat{\textbf{U}})\cdot \div\bm\sigma_d^{(0)}
    =
    (\textbf{u}_d^{(0)} - \textbf{U}) \cdot \div \hat{\bm\sigma}_d, 
\end{equation*}
which can be reformulated as, 
\begin{equation*}
    \div [\bm\sigma_d^{(0)} \cdot (\hat{\textbf{u}}_d - \hat{\textbf{U}})]
    - \bm\sigma_d^{(0)} : \grad (\hat{\textbf{u}}_d - \hat{\textbf{U}})
    =
    \div [\hat{\bm\sigma}_d \cdot (\textbf{u}_d^{(0)} - \textbf{U})], 
    - \hat{\bm\sigma}_d : \grad ({\textbf{u}}_d^{(0)} - \textbf{U})
\end{equation*}
Noticing that $- \hat{\bm\sigma}_d : \grad ({\textbf{u}}_d^{(0)} - \textbf{U}) + \bm\sigma_d^{(0)} : \grad (\hat{\textbf{u}}_d - \hat{\textbf{U}}) = 2 \hat{\textbf{e}}_d : \grad \textbf{U} - 2 \textbf{e}_d^{(0)} : \grad \hat{\textbf{U}}$, and integrating over the volume of the particle yields, 
\begin{equation}
    \intS{(\hat{\textbf{u}}_d - \hat{\textbf{U}})\cdot \bm\sigma_d^{(0)} \cdot \textbf{n}}
    + 2\intO{\textbf{e}_d^{(0)} : \grad \hat{\textbf{U}}}
    =
    \intS{(\textbf{u}_d^{(0)} - \textbf{U}) \cdot  \hat{\bm\sigma}_d\cdot \textbf{n} }
    + 2\intO{\hat{\textbf{e}}_d : \grad \textbf{U}} 
    \label{eq:reciprocal_d}
\end{equation}
This relation indicates that the of surface forces traction is balanced by the internal shearing motion. 


\subsubsection{First form of the reciprocal theorem for stokes flows}

Making use of the relation \ref{eq:reciprocal_d} into \ref{eq:reciprocal_d} yields the final form of the reciprocal theorem, 
\begin{equation}
    \intS{\hat{\textbf{U}}\cdot  \bm\sigma_f^{(0)} \cdot \textbf{n}}
    - \lambda 2\intO{\textbf{e}_d^{(0)} : \grad \hat{\textbf{U}}}
    = 
    \intS{\textbf{U}\cdot  \hat{\bm\sigma}_f \cdot \textbf{n}}
    -\lambda  2\intO{\hat{\textbf{e}}_d : \grad \textbf{U}}. 
    \label{eq:reciprocal_all}
\end{equation}
As evidenced by the second term on the left-had side, we are only able to compute the force traction minus the internal shear with this method, since $\textbf{e}^{(0)}$ is also part of the solution.
However, it turns out that it is exactly the closure that we are looking for.  


\subsubsection{Another formulation of the reciprocal theorem? }
It will be useful in the following to use a second normalization for this reciprocal theorem. 
The stress condition at the boundary of the particle is written, for the disturbance fields as, 
\begin{equation*}
    (\bm\sigma_f\cdot\textbf{n})\cdot (\bm\delta - \textbf{nn})
    = \lambda (\bm\sigma_d\cdot \textbf{n})(\bm\delta - \textbf{nn})
\end{equation*}
However, note that this boundary condition must also hold for the complete velocity field $\textbf{u}_k - \textbf{U}$, thus we may also write, 
\begin{equation*}
    [\bm\sigma_f
    - (\grad \textbf{U}+\grad \textbf{U})
    ]\cdot \textbf{n}\cdot  (\bm\delta - \textbf{nn})
    = 
    \lambda [\bm\sigma_d\cdot \textbf{n}
    - (\grad \textbf{U}+\grad \textbf{U})
    ]\cdot \textbf{n}\cdot (\bm\delta - \textbf{nn})
\end{equation*}
The same boundary condition holds for the test problem. 
Then we may reformulate the volume integral on the left-hand side of \ref{eq:reciprocal_all} using, 
\begin{align*}
    \intO{ 2\textbf{e}_d^{(0)} : \grad \hat{\textbf{U}} }
    &=\intO{ \grad \textbf{u}_d^{(0)} : (\grad \hat{\textbf{U}} +^\dagger \grad \textbf{U}) }
    =
    \intS{  \textbf{u}_d^{(0)} \cdot (\grad \hat{\textbf{U}} + ^\dagger\grad \hat{\textbf{U}})  \cdot \textbf{n}}\\
    &=
    \intS{  \textbf{U} \cdot (\grad \hat{\textbf{U}} + ^\dagger\grad \hat{\textbf{U}})  \cdot \textbf{n}}
    + \intS{  (\textbf{u}_d^{(0)} - \textbf{U})\cdot (\grad \hat{\textbf{U}} + ^\dagger\grad \hat{\textbf{U}})  \cdot \textbf{n}}
    % -\intO{\textbf{u}_d^{(0)} \cdot \grad^2 \hat{\textbf{U}} }
    \\
\end{align*}
Note that the first integral on the left-hand side is entirely known, since it corresponds to integrals of the undisturbed fields.
One could also write, 
\begin{multline}
    \intO{ 2\hat{\textbf{e}}_d : \grad {\textbf{U}} }
    =\intO{ \grad \hat{\textbf{u}}_d : (\grad {\textbf{U}} +^\dagger \grad \textbf{U}) }\\
    =
    \intS{  \hat{\textbf{u}}_d \cdot (\grad {\textbf{U}} + ^\dagger\grad {\textbf{U}})  \cdot \textbf{n}}
    -\intO{\hat{\textbf{u}}_d \cdot \grad^2 {\textbf{U}} }
\end{multline}
Combining this relation with \ref{eq:reciprocal_d} gives,
\begin{multline}
    \intS{(\hat{\textbf{u}}_d - \hat{\textbf{U}})\cdot \bm\sigma_d^{(0)} \cdot \textbf{n}}
    =
    - \intS{  \textbf{U} \cdot (\grad \hat{\textbf{U}} + ^\dagger\grad \hat{\textbf{U}})  \cdot \textbf{n}}\\
    + \intS{(\textbf{u}_d^{(0)} - \textbf{U}) \cdot  [\hat{\bm\sigma}_d -(\grad \hat{\textbf{U}} + ^\dagger\grad \hat{\textbf{U}})]\cdot \textbf{n} }
    + 2\intO{\hat{\textbf{e}}_d : \grad \textbf{U}} 
\end{multline}
Multiplying this expression by $\lambda$ and using the boundary condition finally gives us the relation,
\begin{multline}
    \intS{(\hat{\textbf{u}}_f - \hat{\textbf{U}})\cdot \bm\sigma_f^{(0)} \cdot \textbf{n}}
    =
    - \lambda\intS{  \textbf{U} \cdot (\grad \hat{\textbf{U}} + ^\dagger\grad \hat{\textbf{U}})  \cdot \textbf{n}}\\
    + \intS{(\textbf{u}_f^{(0)} - \textbf{U}) \cdot  [\hat{\bm\sigma}_f -(\grad \hat{\textbf{U}} + ^\dagger\grad \hat{\textbf{U}})]\cdot \textbf{n} }
    + 2\lambda\intO{\hat{\textbf{e}}_d : \grad \textbf{U}} 
\end{multline}
where only the integral $+ 2\lambda\intO{\hat{\textbf{e}}_d : \grad \textbf{U}} $ appears. 
Thus, making use of this new relation gives, 
\begin{multline}
    \intS{\hat{\textbf{U}}\cdot  \bm\sigma_f^{(0)} \cdot \textbf{n}}
    - \intS{(\textbf{u}_f^{(0)}\textbf{n}+\textbf{nu}_f^{(0)})  : \grad \hat{\textbf{U}} }
    % + \intS{(\hat{\textbf{u}}_f - \hat{\textbf{U}})\cdot  \bm\sigma_f^{(0)} \cdot \textbf{n}}
    = 
    \intS{\textbf{U}\cdot  \hat{\bm\sigma}_f \cdot \textbf{n}}\\
    +(\lambda -1) \intS{  (\textbf{U} \textbf{n}+ \textbf{n} \textbf{U}) : \grad \hat{\textbf{U}}  }
    - 2\lambda\intO{\hat{\textbf{e}}_d : \grad \textbf{U}}
    \label{eq:reciprocal_f2}
\end{multline}
We will see latter than upon making the good choice for $\hat{\textbf{U}}$ the left-hand side term corresponds identically to the stress let quantity. 


\subsubsection{Calculation of the drag force in arbitrary stokes flows:}
To derive an expression for the drag force we simply consider that $\hat{\textbf{U}}$ is uniform, i.e. that $\hat{\textbf{E}}_f = 0$. 
We are allowed to do that as $\hat{\textbf{U}}$ is part of the \textit{tool} problem and is therefore completely arbitrary. 
In this situation, we can express the fields $\hat{\textbf{u}}_k$, $\hat{\bm\sigma}_k$ and $\hat{\textbf{e}}_k$ using \ref{eq:solution_hat} and retaining only the terms proportional to $\hat{\textbf{U}}$. 
Additionally, we recall that the undisturbed fields $\textbf{U}[\textbf{y}] = \textbf{w} - \textbf{U}_f[\textbf{x}] - \textbf{r}\cdot \grad \textbf{U}_f[\textbf{x}] - \textbf{rr}:\grad\grad \textbf{U}_f[\textbf{x}] + \ldots =  \textbf{w} - \textbf{U}_f[\textbf{x}] - \textbf{r}\cdot \textbf{E}_f - \frac{1}{2}\textbf{rr}: \textbf{K}_f + \ldots$.
We deduce form all these remarks and the relation  given by \ref{eq:reciprocal_all} that the zeroth-order drag force on a spherical droplet is given by, 
\begin{multline*}
    \left(
        \intS{\bm\sigma_f^{(0)} \cdot \textbf{n}}
    \right)_m\\
    = 
    U_i  \intS{  (\mathcal{U}_f^\sigma)_{ilm}  n_l }
    % + \partial_k U_i  \intS{ r_k (\mathcal{U}_f^\sigma)_{ilm}  n_l }
    + \frac{1}{2}\partial_k\partial_j U_i \intS{ r_kr_j (\mathcal{U}_f^\sigma)_{ilm}  n_l }
    % + \ldots
    % \\
    % -2\lambda  \partial_j U_i \intO{(\mathcal{U}_d^e)_{ijm} }. 
    -\lambda \partial_k\partial_j U_i  \intO{ r_k (\mathcal{U}_d^e)_{ijm} }. 
    + \ldots
    \label{eq:reciprocal_drag}
\end{multline*}
We obtained a relation that is entirely computable form the explicit expression given by \ref{eq:solution_hat}. 
Indeed, it can be shown using \ref{eq:solution_hat} that, 
\begin{align*}
    \left(
        \intS{\bm\sigma_f^{(0)} \cdot \textbf{n}}
    \right)_m
    = 
    - 2\pi\left(\frac{3\lambda +2 }{\lambda +1}\right)
    U_m  
    - \pi \left(\frac{\lambda +2/5 }{\lambda +1}\right)\partial^2 U_m 
    + \frac{2 \pi}{5} \left(\frac{1}{1+\lambda}\right) 
     \partial^2 U_m 
\end{align*}
It can be shown that the first term on the right-hand side, proportional to the relative velocity $\textbf{U}$ is the Hadamard-Ribczynski contribution. 
The terms proportional to $\textbf{E}_f$ can be shown to be zero and the one proportional to $\textbf{K}_f$ are the Faxen contribution. 
In tensor notation this gives, 
\begin{align*}
    \left(
        \intS{\bm\sigma_f^{(0)} \cdot \textbf{n}}
    \right)_m
    = 
     2\pi\left(\frac{3\lambda +2 }{\lambda +1}\right)
    \textbf{u}_{fp}  
    + \pi \left(\frac{\lambda }{\lambda +1}\right)\partial^2 \textbf{u}_f
    % + \frac{2 \pi}{5} \left(\frac{1}{1+\lambda}\right) 
    %  \partial^2 U_m 
\end{align*}
We thus recovered Faxen law. 
Note that this method, in this context, is less efficient than the method used originally to derive the Faxen relations \citet{kim2013microhydrodynamics}. 
Indeed, in the former method we have no clue about the value of the higher order unless we compute them, while the relations form Faxen explicitly state that the force must have the same functional form as the disturbance fields given by \ref{eq:solution_hat} thereby implying that the higher order terms are zeroth. 

% \subsubsection{Calculation of the stresslet in arbitrary stokes flows:}
% To compute the stresslet term we make use of the second formulation and assume that $\hat{\textbf{U}} = \hat{\textbf{E}}\cdot \textbf{r}$, in which case \ref{eq:reciprocal_f2} yields, 
% \begin{equation}
%     \intS{\textbf{r} \bm\sigma_f^{(0)} \cdot \textbf{n}}
%     - \intS{(\textbf{u}_f^{(0)}\textbf{n}+\textbf{nu}_f^{(0)}) }
%     % + \intS{(\hat{\textbf{u}}_f - \hat{\textbf{U}})\cdot  \bm\sigma_f^{(0)} \cdot \textbf{n}}
%     = 
%     \intS{\textbf{U}\cdot  \mathcal{E}^\sigma_f}
%     +(\lambda -1) \intS{  (\textbf{U} \textbf{n}+ \textbf{n} \textbf{U})  }
%     - 2\lambda\intO{\mathcal{E}_d^e : \grad \textbf{U}}
%     \label{eq:reciprocal_f2}
% \end{equation}
% To verify that our reciprocal theorem is write one can simply set $\textbf{U} = - \textbf{r}\cdot \textbf{E}_f$ and see that 
% \begin{equation}
%     \intS{\textbf{r} \bm\sigma_f^{(0)} \cdot \textbf{n}}
%     - \intS{(\textbf{u}_f^{(0)}\textbf{n}+\textbf{nu}_f^{(0)}) }
%     % + \intS{(\hat{\textbf{u}}_f - \hat{\textbf{U}})\cdot  \bm\sigma_f^{(0)} \cdot \textbf{n}}
%     = 
%     - \textbf{E}_f :\left[ \intS{\textbf{r}  \mathcal{E}^\sigma_f}
%     + (\lambda -1) \intS{  (\textbf{r}\textbf{n}+ \textbf{n} \textbf{r})  }
%     - 2\lambda\intO{\mathcal{E}_d^e }\right]
%     \label{eq:reciprocal_f2}
% \end{equation}

\subsection{Reciprocal theorem to compute the first order inertial correction}

To compute the first order correction in $Re$ we apply the same methodology but on the first order equaitons. 
As before we start by the continuous phase integral relation by taking the dot product of \ref{eq:first_order_NS_f} with $\hat{\textbf{u}}_f$ and \ref{eq:tool_sol} (with $k = f$) with $\textbf{u}_k^{(0)}$, then subtracting both expressions gives, 
\begin{equation*}
    \hat{\textbf{u}}_f\cdot \div\bm\sigma_f^{(1)}
    =
    \textbf{u}_k^{(1)} \cdot \div \hat{\bm\sigma}_k, 
    + \hat{\textbf{u}}_f\cdot  \textbf{f}^{(0)}_f
\end{equation*}
Carrying an integration on the whole continuous phase domain directly gives the relation, 
\begin{equation*}
    \intS{\hat{\textbf{U}}\cdot  \bm\sigma_f^{(1)} \cdot \textbf{n}}
    + \intS{(\hat{\textbf{u}}_f - \hat{\textbf{U}})\cdot  \bm\sigma_f^{(1)} \cdot \textbf{n}}
    = 
    \intS{\textbf{u}_f^{(1)}\cdot  \hat{\bm\sigma}_f \cdot \textbf{n}}
    - \intOf{\hat{\textbf{u}}_f\cdot  \textbf{f}^{(0)}_f}
    % \intOf{\textbf{u}_f^{(0)}\cdot ( \div \hat{\bm\sigma}_f)}
\end{equation*}
From \ref{eq:bc_inertial} we deduce that $\textbf{u}_f^{(1)}$ has only a tangential component allowing us to directly use the boundary condition on the tangential stress in order to substitute the integral on the left-hand side of this expression. 
However, the integral on left-hand side still needs to be decomposed into a normal and tangential components. 

As before we apply the Reciprocal theorem on the interior of the drop,
but this time we multiply \ref{eq:tool_sol} (with $k = f$) with $\textbf{u}_k^{(1)}$ instead of $\textbf{u}_k^{(0)} - \textbf{U}$ since $\textbf{u}_k^{(1)}$ is already a tangent vector to the surface, this yield,
\begin{equation*}
     \div [\bm\sigma_d^{(1)} \cdot (\hat{\textbf{u}}_d - \hat{\textbf{U}})]
     +2 \textbf{e}_d^{(1)} : \grad \hat{\textbf{U}}
    - 
    \frac{\zeta}{\lambda}(\hat{\textbf{u}}_d - \hat{\textbf{U}}) \cdot \textbf{f}_d^{(0)}
    =
     \div [\hat{\bm\sigma}_d \cdot \textbf{u}_d^{(1)}]
\end{equation*}
Integrating on the volume of the particle gives, 
\begin{equation*}
    \intS{ (\hat{\textbf{u}}_d - \hat{\textbf{U}})\cdot \bm\sigma_d^{(1)} \cdot \textbf{n}}
    + 2 \intO{ \textbf{e}_d^{(1)} : \grad \hat{\textbf{U}} }
    - \frac{\zeta}{\lambda} \intO{(\hat{\textbf{u}}_d - \hat{\textbf{U}}) \cdot \textbf{f}_d^{(0)}}
    =
    \intS{
         \textbf{u}_d^{(1)}\cdot\hat{\bm\sigma}_d \cdot \textbf{n}
    }. 
\end{equation*}
Subtracting this equation times $\lambda$ and using the boundary conditions  for the interface stress and velocity finally gives the final form of the Reciprocal theorem, namely, 
\begin{equation*}
    \intS{\hat{\textbf{U}}\cdot  \bm\sigma_f^{(1)} \cdot \textbf{n}}
    -\lambda 2 \intO{ \textbf{e}_d^{(1)} : \grad \hat{\textbf{U}} }
    % + \intS{(\hat{\textbf{u}}_f - \hat{\textbf{U}})\cdot  \bm\sigma_f^{(1)} \cdot \textbf{n}}
    = 
    - \zeta \intO{(\hat{\textbf{u}}_d - \hat{\textbf{U}}) \cdot \textbf{f}_d^{(0)}}
    - \intOf{\hat{\textbf{u}}_f\cdot  \textbf{f}^{(0)}_f}.
    % \intS{\textbf{u}_f^{(1)}\cdot  \hat{\bm\sigma}_f \cdot \textbf{n}}
    % \intOf{\textbf{u}_f^{(0)}\cdot ( \div \hat{\bm\sigma}_f)}
    \label{eq:reciprocal_f_one}
\end{equation*}
As one can see we are constrained to obtain a formula for the interface stress minus the internal shear, these cannot be separated. 

This formula will prove useful in the following however as mentioned since we are also interested in the stress let quantity we may use the relation 
\begin{align*}
    \intO{ 2\textbf{e}_d^{(1)} : \grad \hat{\textbf{U}} }
    &=\intO{ \grad \textbf{u}_d^{(1)} : (\grad \hat{\textbf{U}} +^\dagger \grad \textbf{U}) }
    =
    \intS{  \textbf{u}_d^{(1)} \cdot (\grad \hat{\textbf{U}} + ^\dagger\grad \hat{\textbf{U}})  \cdot \textbf{n}}\\
    % &=
    % \intS{  \textbf{U} \cdot (\grad \hat{\textbf{U}} + ^\dagger\grad \hat{\textbf{U}})  \cdot \textbf{n}}
    % + \intS{  (\textbf{u}_d^{(1)} - \textbf{U})\cdot (\grad \hat{\textbf{U}} + ^\dagger\grad \hat{\textbf{U}})  \cdot \textbf{n}}
    % -\intO{\textbf{u}_d^{(0)} \cdot \grad^2 \hat{\textbf{U}} }
\end{align*}
which means that the reciprocal theroem on the inside of the particle may equally be written as, 
\begin{equation*}
    \intS{ (\hat{\textbf{u}}_d - \hat{\textbf{U}})\cdot \bm\sigma_d^{(1)} \cdot \textbf{n}}
    - \frac{\zeta}{\lambda} \intO{(\hat{\textbf{u}}_d - \hat{\textbf{U}}) \cdot \textbf{f}_d^{(0)}}
    =
    % \intS{  \textbf{u}_d^{(1)} \cdot (\grad \hat{\textbf{U}} + ^\dagger\grad \hat{\textbf{U}})  \cdot \textbf{n}}
    \intS{
         \textbf{u}_d^{(1)}\cdot[\hat{\bm\sigma}_d  - (\grad \hat{\textbf{U}} + ^\dagger\grad \hat{\textbf{U}})]\cdot \textbf{n}
    }
\end{equation*}
which, by considering the boundary condition on $\hat{\bm\sigma}_d  - (\grad \hat{\textbf{U}} + ^\dagger\grad \hat{\textbf{U}})$ gives the final form of the reciprocal theorem, namely, 
\begin{equation*}
    \intS{\hat{\textbf{U}}\cdot  \bm\sigma_f^{(1)} \cdot \textbf{n}}
    - \intS{(\textbf{u}_f^{(1)} \textbf{n} + \textbf{n} \textbf{u}_f^{(1)})\cdot :\grad \hat{\textbf{U}} }
    = 
    - \zeta \intO{(\hat{\textbf{u}}_d - \hat{\textbf{U}}) \cdot \textbf{f}_d^{(0)}}
    - \intOf{\hat{\textbf{u}}_f\cdot  \textbf{f}^{(0)}_f}
    % \intOf{\textbf{u}_f^{(0)}\cdot ( \div \hat{\bm\sigma}_f)}
    \label{eq:reciprocal_f2_one}
\end{equation*}
from this relation and a good choice for $\hat{\textbf{U}}$ one is able to determine the stresslet. 
Note that the right-hand side of \ref{eq:reciprocal_f_one} and \ref{eq:reciprocal_f2_one} is similar. 
Therefore, we must conclude that the first order Reynolds correction,
\begin{equation*}
    \intS{(\textbf{u}_f^{(1)} \textbf{n} + \textbf{n} \textbf{u}_f^{(1)})}
    = \lambda \intS{(\textbf{u}_f^{(1)} \textbf{n} + \textbf{n} \textbf{u}_f^{(1)})}
\end{equation*}
Since $\lambda$ is an arbitrary constant we deduce that, 
\begin{equation*}
    \intS{(\textbf{u}_f^{(1)} \textbf{n} + \textbf{n} \textbf{u}_f^{(1)})}= 0
\end{equation*}

\subsubsection{Computations of the inertial stresslet for a particle in steady state translation}

The objective of this study is to determine the effect of inertial translation of a drop on the deformation.
To do so we have shown that it is necessary to compute the first moments of the hydrodynamic forces as well as the droplets internal inertia and velocity. 
To this end we assume that $\hat{\textbf{U}} = - \textbf{r} \cdot \hat{\textbf{E}}_f$ meaning  that the \textit{tool} far field flow is purely linear, and since we consider only relative translation we set $\textbf{U}[\textbf{y}] = \textbf{U}[\textbf{x}] = \textbf{w} - \textbf{U}_f$. 
Choosing the appropriate velocity field as a tool solution we conclude that, 
\begin{equation*}
    \intS{r_i (\bm\sigma_f^{(1)} \cdot \textbf{n})_j}
    -\lambda 2 \intO{ (\textbf{e}_d^{(1)})_{ij}  }
    % + \intS{(\hat{\textbf{u}}_f - \hat{\textbf{U}})\cdot  \bm\sigma_f^{(1)} \cdot \textbf{n}}
    = 
    + \zeta\intO{(\mathcal{E}_d + \textbf{r}\bm\delta) \cdot \textbf{f}_d^{(0)}}
    + \intOf{\mathcal{E}_f \cdot  \textbf{f}^{(0)}_f}
    % \intS{\textbf{u}_f^{(1)}\cdot  \hat{\bm\sigma}_f \cdot \textbf{n}}
    % \intOf{\textbf{u}_f^{(0)}\cdot ( \div \hat{\bm\sigma}_f)}
\end{equation*}
These integrals requires the calculation of integrals in terms of $\textbf{f}^{(0)}_k$, which are known fields since it is solely function of the $\textbf{u}^{(0)}_k$. 
In the case considered here, meaning considering only particle relative translation we have, 
\begin{align*}
    \intOf{\mathcal{E}_f \cdot  (\textbf{f}^{(0)}_f)}
    &=
    \intOf{\mathcal{E}_f \cdot  \pddt \textbf{u}_f^{(0)}}
    + \intOf{\mathcal{E}_f \cdot (\textbf{u}_f^{(0)}\cdot \grad \textbf{u}_f^{(0)})}\\
    &+ \intOf{\mathcal{E}_f \cdot (\textbf{u}_f^{(0)}\cdot \grad (\textbf{U}_f - \textbf{w}))}
    + \intOf{\mathcal{E}_f \cdot ((\textbf{U}_f - \textbf{w})\cdot \grad \textbf{u}_f^{(0)})}.
\end{align*}
Since $\textbf{U}_f - \textbf{w}$ is a constant vector its gradient vanish. 
Besides, since $ \textbf{u}_f \sim \textbf{U}_f - \textbf{w}$  and that $\textbf{U}_f - \textbf{w}$ is not function of time as well, the partial time derivative of $\textbf{u}_f$ vanish as well. 
Approximately the same remarks can be made regarding the  dispersed phase integral.
This leads us to the relations, 
\begin{align*}
    \intOf{\mathcal{E}_f \cdot  (\textbf{f}^{(0)}_f)}
    &=
    \intOf{(\mathcal{E}_f)_{ijm} \cdot (\textbf{u}_f^{(0)}\cdot \grad \textbf{u}_f^{(0)})_i}\\
    % + \intOf{(\mathcal{E}_f)_{ijm} \cdot (\textbf{u}_f^{(0)}\cdot \grad (\textbf{U}_f^{(0)} - \textbf{w}))}
    &+ \intOf{(\mathcal{E}_f)_{ijm} \cdot ((\textbf{U}_f - \textbf{w})\cdot \grad \textbf{u}_f^{(0)})_i}.\\
    \intO{(\mathcal{E}_d + \textbf{r}\bm\delta)_{ijk} \cdot (\textbf{f}_d^{(0)})_m}
    &=
    \intO{(\mathcal{E}_d + \textbf{r}\bm\delta)_{ijk} \cdot (\textbf{u}_d^{(0)}\cdot \grad \textbf{u}_d^{(0)})_m}\\
    % + \intO{(\mathcal{E}_d + \textbf{r}\bm\delta)_{ijk} \cdot (\textbf{u}_d^{(0)}\cdot \grad (\textbf{U}_d^{(0)} + \textbf{w}))}
    &+ \intO{(\mathcal{E}_d + \textbf{r}\bm\delta)_{ijk} \cdot ((\textbf{U}_f - \textbf{w})\cdot \grad \textbf{u}_d^{(0)})_m}
    =0.
\end{align*}
Where we have used indices notation to get rid of any confusions. 
The integral inside the particle can be shown to vanish. 
The integrals on the fluid domain are entirely convergent and can be computed. 

Since we can use the stokes flow solution for $\textbf{u}^{(0)}_f$ we may write, 
\begin{equation}
    (\textbf{u}^{(0)}_f)_i = (\mathcal{U}_f)_{ij} U_j
\end{equation}
And since $(\textbf{U}_f - \textbf{w}) = - \textbf{U}$ we obtain the following relation, 
\begin{align}
    \intOf{(\mathcal{E}_f)_{ijk} (\textbf{f}^{(0)}_f)_i }
    &= 
    \intOf{(\mathcal{E}_f)_{ijk} \cdot (\textbf{u}_f^{(0)} - \textbf{U})_l \partial_l (\textbf{u}_f^{(0)})_i}\\
    &=
    \intOf{(\mathcal{E}_f)_{ijk} \cdot ((\mathcal{U}_f)_{lm} - \delta_{lm}) \partial_l (\mathcal{U}_f)_{in} }
    U_m U_n
\end{align}

The only traceless symmetric second order tensor that can be formed with a combinaison of $\textbf{U}$ is $\textbf{UU} - (\textbf{U}\cdot \textbf{U})\bm\delta /3$. 
This implies that only one scalar integration is necessary to determine the tensor form of the  stress, this yields,
% \begin{align*}
%     \intS{r_i (\bm\sigma_f^{(1)} \cdot \textbf{n})_j}
%     -\lambda 2 \intO{ (\textbf{e}_d^{(1)})_{ij}  }
%     % + \intS{(\hat{\textbf{u}}_f - \hat{\textbf{U}})\cdot  \bm\sigma_f^{(1)} \cdot \textbf{n}}
%     = 
%     A
%     [
%         \textbf{UU} - \frac{1}{3}(\textbf{U}\cdot \textbf{U})\bm\delta 
%     ]\\
%     A= - \frac{345 \lambda^{3} + 928 \lambda^{2} + 856 \lambda + 272}{320 \left(\lambda + 1\right)^{3}}
%     + \frac{\zeta \left(12 \lambda + 13\right)}{20 \left(\lambda + 1\right)^{2}}
% \end{align*}
\begin{align*}
    \pSavg{r_i (\bm\sigma_f^{(1)} \cdot \textbf{n})_j}
    % -\lambda 2 \intO{ (\textbf{e}_d^{(1)})_{ij}  }
    % + \intS{(\hat{\textbf{u}}_f - \hat{\textbf{U}})\cdot  \bm\sigma_f^{(1)} \cdot \textbf{n}}
    = 
    \phi A
    [
        \textbf{UU} - \frac{1}{3}(\textbf{U}\cdot \textbf{U})\bm\delta 
    ]\\
    A= - \frac{63 \lambda^{3} + 150 \lambda^{2} + 112 \lambda + 28}{80 \left(\lambda + 1\right)^{3}}
\end{align*}
It must be understand from this expression that we only determined the traceless symmetric part of the left-hand side due to the specific choice made for $\hat{\textbf{U}}_f$. 

\subsubsection{Trace of the first moment.}

Following, \citet{stone2001inertial} we now use the test problem of a point source of strength $Q$ located at the center of the sphere, in this case the test velocity  and stress fields, reads, 
\begin{align*}
    \hat{\textbf{u}} = \frac{Q}{4\pi} \frac{\textbf{r}}{r^3}
    && \hat{\bm\sigma} = \mu_f \frac{Q}{2\pi}\left(
        \frac{\bm\delta}{r^3}
        - \frac{3 \textbf{rr}}{r^5}
    \right)
\end{align*}
Using, this solution in \ref{eq:reciprocal_f2_one} gives, 
\begin{align*}
    \intS{ \textbf{r}\cdot  \bm\sigma_f^{(1)} \cdot \textbf{n}}
    = 
    \intS{\textbf{u}_f^{(1)} \cdot  \mu_f \frac{1}{2\pi}\left(
        \bm\delta
        - 3 \textbf{rr}
    \right) \cdot \textbf{n}}
    - \intOf{ \frac{\textbf{r}}{r^3}\cdot  \textbf{f}^{(0)}_f}
    = 
    % \intS{\textbf{u}_f^{(1)}\textsc{ \cdot  \mu_f \frac{1}{2\pi}\left(
    %     \bm\delta
    %     - 3 \textbf{rr}
    % \right) \cdot \textbf{n}}}
    - \intOf{ \frac{\textbf{r}}{r^3}\cdot  \textbf{f}^{(0)}_f}\\
    % \intOf{\textbf{u}_f^{(0)}\cdot ( \div \hat{\bm\sigma}_f)}
\end{align*}
Since $\intS{\textbf{u}^{(1)} \cdot \textbf{n}} =0$ due to the divergence free property of $\textbf{u}^{(0)}$. 
When $\textbf{u}^{(0)} = \textbf{U}\cdot \mathcal{U}_f$ correspond to the disturbance velocity fields of a droplet in stokes flow we obtain, 
\begin{equation*}
    \pSavg{ \textbf{r}\cdot  \bm\sigma_f^{(1)} \cdot \textbf{n}}
    = \phi \frac{3\lambda^2 + 6\lambda + 4}{16(\lambda +1 )^2} (\textbf{u}_{pf}\cdot \textbf{u}_{pf})
\end{equation*}
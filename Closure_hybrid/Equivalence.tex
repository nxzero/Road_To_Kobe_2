
%\subsubsection*{Equivalence between particle and continuous models}
\subsection{Equivalence between particle-averaged and phase-averaged equations}
\label{sec:equivalence}
To model the dispersed phase we can either use \ref{eq:avg_dt_chi_f} with $k=2$, or the particle-average \ref{eq:avg_dt_dq_alpha_tot}, \ref{eq:avg_dt_dQ_alpha_tot} and possibly the higher moments equations. 
As mentioned in \ref{sec:Lagrangian} we notified that \ref{eq:dt_dq_alpha_tot} was already subject to an average over the particles' volume and the surfaces. 
Meaning that \ref{eq:avg_dt_dq_alpha_tot} is the results of two average processes. 
Consequently, it is fair to address the question of the compatibility and differences between both formalism, i.e. between \ref{eq:avg_dt_chi_f} and \ref{eq:avg_dt_dq_alpha_tot}. 

To begin with, it has been demonstrated in various studies \citep{nott2011suspension,jackson1997locally,zhang1994averaged}, that phase-averaged quantities can be expressed as a Taylor series expansion of particle-averaged quantities. 
Indeed, the dispersed phase indicator function $\chi_2(\textbf{x},t)$ can be expressed as a sum of phase indicator function, $\chi_2(\textbf{x},t) = \sum_\alpha\chi_\alpha(\textbf{x},t)$ where $\chi_\alpha =1$ in the particle domain $\Omega_\alpha(t)$ and $0$ otherwise. 
Then, notice that any a dispersed phase quantity can be written as, 
\begin{equation*}
    f^0_2 \chi_2(\textbf{x},t)
    = \sum_\alpha f^0_2 \chi_\alpha(\textbf{x},t) 
    = \sum_\alpha \int_{\mathbb{R}^3} f^0_2 \chi_\alpha(\textbf{x}_\alpha+\textbf{r},t)\delta(\textbf{x}- \textbf{x}_\alpha - \textbf{r}) d\textbf{r} 
\end{equation*}
Which upon using the Taylor expansion of the Dirac delta function in the neighborhood of $\textbf{r}=0$ one obtain :  $\delta(\textbf{x}- \textbf{x}_\alpha - \textbf{r}) = \delta(\textbf{x}- \textbf{x}_\alpha) - \textbf{r}\cdot\grad \delta(\textbf{x} - \textbf{x}_\alpha) + \frac{\textbf{rr}}{2}:\grad\grad\delta(\textbf{x}- \textbf{x}_\alpha)\ldots $.
% Thus, the dispersed phase quantity $(f_2^0\chi_2)$ can be re-written as, 
% \begin{equation*}
%     f^0_2 \chi_2(\textbf{x},t)
%     = \sum_\alpha \left[
%         \intO{f^0_2}
%         - \div\intO{\textbf{r} f^0_2}
%         + \frac{1}{2}\grad\grad :\intO{\textbf{rr} f^0_2}
%         \ldots
%     \right]
% \end{equation*}
% where we recognize the zeroth, first and second order moments of $f_2^0$. 
Applying these considerations to interfaces quantities and averaging over all configurations of the phase space, one obtain a general relation between continuous and particle averaged quantities, namely, 
\begin{align}
    \avg{\chi_2f_2^0} 
    &=  \pavg{q_\alpha}
        - \div  
        \pavg{\mathcal{Q}_\alpha}        
        + \frac{1}{2} \grad\grad : \pavg{\mathcal{Q}_{2\alpha}}
        + \ldots  
        \nonumber\\
    \avg{\delta_I f_I^0} 
    &=  \pavg{q_{I\alpha}}        
        - \div \pavg{\mathcal{Q}_{I\alpha}}
        + \frac{1}{2} \grad\grad : \pavg{\mathcal{Q}_{I\alpha}^{2}}
        + \ldots  
    \label{eq:f_exp}
\end{align}
It must be noted that \ref{eq:f_exp} is the bridge between the continuous phase average formalism and the particle formulation. 
One of the consequences of these relations is that, 
\begin{align}
    \phi_2 \rho_2
    = m_p n_p 
    + \frac{1}{2}\grad^2 : (n_p\mathcal{M}_p)+\ldots\\
    \phi_2 \rho_2 \textbf{u}_2
    = m_p n_p \textbf{u}_p 
    - \div (n_p\mathcal{P}_p)+\ldots
    \label{eq:f_exp_exe}
\end{align}
meaning that $\phi_2\rho_2$ is in fact related to $\mathcal{M}_p$ and the phase averaged velocity $\textbf{u}_2$ contain the first  moment of momentum $\mathcal{P}_p$, which account for rotational and stretching motions of the particles. 

In order to establish the equivalence between both formalism, we follow the strategy of \citep{lhuillier2000bilan,lhuillier2009rheology} by taking the Taylor expansion of each terms in \ref{eq:avg_dt_chi_f} with $k=2$ using the relation \ref{eq:f_exp}. 
Since we made use of the surface transport equations in the particles phase equations : \ref{eq:avg_dt_dq_alpha_tot} and \ref{eq:avg_dt_dQ_alpha_tot}, we also need to consider \ref{eq:avg_dt_delta_f}  to prove equivalence. 
As the resulting expression can become quite cumbersome, we will adopt the following definition. 
Let $\mathbb{C}_2$ represent the phase-averaged equation of conservation (\ref{eq:avg_dt_chi_f} with $k=2$) and $\mathbb{C}_I$ the averaged surface transport equation, namely, 
\begin{align*}
    \mathbb{C}_2
    &=
    - \pddt \avg{\chi_2f_2^0}
    - \div \avg{\chi_2 \mathbf{\Phi}_2^0 - \chi_2f_2^0 \textbf{u}_2^0}
    + \avg{\chi_2 s_2^0}
    + \avg{\delta_I\left[
        \mathbf{\Phi}_2^0
        + f_2^0
        \left(
            \textbf{u}_I^0
            - \textbf{u}_2^0
        \right)
    \right]
    \cdot \textbf{n}_2}.\\
    \mathbb{C}_I
    &= 
    -\pddt \avg{\delta_If_I^0}
    -\div \avg{\delta_I f_I^0 \textbf{u}_I^0-\delta_I \mathbf{\Phi}_{I||}^0 }
    + \avg{\delta_Is_I^0} 
    - \avg{\delta_I \Jump{
    f_k^0 (\textbf{u}_I^0 - \textbf{u}_k^0)
    + \mathbf{\Phi}_k^0
    } }. 
\end{align*}
It must be understood from \ref{eq:avg_dt_chi_f} and \ref{eq:avg_dt_delta_f} that $\mathbb{C}_2=0$ and $\mathbb{C}_I=0$.
Then, by taking the Taylor expansion of each terms of $\mathbb{C}_2+\mathbb{C}_I$ according to \ref{eq:f_exp}, we can equally show that,
\begin{equation}
    \mathbb{C}_2 
    + \mathbb{C}_I 
    = \mathbb{M}_0 - \div \mathbb{M}_1 + \frac{1}{2} \grad\grad : \mathbb{M}_2 \ldots = 0,
    \label{eq:scheme_equivalence}
\end{equation} 
where the expression $\mathbb{M}_0$ and $\mathbb{M}_1$ turn out to be, 
\begin{align*}
    &\mathbb{M}_0
    = 
    - \avg{\delta_\alpha \ddt {q_\alpha^\text{tot}}}
    % -\avg{\delta_\alpha\textbf{u}_\alpha q_\alpha^\text{tot}}
    + \pOavg{ s_2^0 }
    + \pSavg{ s_I^0 }
    + \pSavg{ 
    \left[\mathbf{\Phi}_1^0 
    + f_1^0 (\textbf{u}_I^0-\textbf{u}_1^0) \right] \cdot \textbf{n}_2 },\\
    &\mathbb{M}_1 =
    -  \avg{\delta_\alpha \ddt {\mathcal{Q}_\alpha^\text{tot}}}
    % - \avg{\delta_\alpha\textbf{u}_\alpha \mathcal{Q}_\alpha^\text{tot}}
     + \pOavg{ \left(
        \textbf{r} s_2^0         
        + f_2^0  \textbf{w}_2^0 
        - \mathbf{\Phi}_2^0
    \right) }
    + \pSavg{ \left(
        \textbf{r}s_I^0
        + f_I^0 \textbf{w}_I^0
        - \mathbf{\Phi}_{I||}^0
    \right) }\\
    &+ \pSavg{ \textbf{r} \left[
        \mathbf{\Phi}_1^0
        + f_1^0 (\textbf{u}_I^0-\textbf{u}_1^0)
    \right]\cdot \textbf{n}_2  },
\end{align*}
respectively. 
In the presence of \ref{eq:scheme_equivalence} we reach the major conclusion of this work. 
Indeed, we can observe that $\mathbb{M}_0$, $\mathbb{M}_1$ and $\mathbb{M}_2$ represent the zeroth, first and second order moments equations, respectively. 
In fact, it is shown in \ref{ap:Moments_equations} that the coefficient $\mathbb{M}_n$ in \ref{eq:scheme_equivalence} correspond to the $n^{th}$ order particle-average conservation equation. 
As a matter of fact, the phase average applied to the dispersed phase contains all the particle-averaged moments equations.
In \ref{ap:Moments_equations} we provide the expression for each moment equation $\mathbb{M}_n$ as well as the complete derivation of \ref{eq:scheme_equivalence}. 
In \cite{lhuillier2000bilan} they reached similar conclusion when comparing the area density phase-averaged and particle-averaged equations of conservation for spherical particles. 
Thus, from \ref{eq:scheme_equivalence} it is evident that one can use an arbitrary order of moments equations to reach an arbitrary accurate description of the dispersed phase.

Another approach is to notice that $\mathbb{M}_n=0$ for all $n$. Thus, we can rewrite \ref{eq:scheme_equivalence} such that all moments equations vanish, except $\mathbb{M}_0$, which gives, 
\begin{equation}
    \mathbb{C}_2 = \mathbb{M}_0 = 0.
\end{equation}
This implies that equation \ref{eq:avg_dt_chi_f} with the surface transport equation \ref{eq:avg_dt_delta_f} is rigorously equivalent to \ref{eq:avg_dt_dq_alpha_tot} which has been shown by \cite[Appendix A]{nott2011suspension} for the specific case of the momentum conservation equation of solid spherical particle.
In fact, we generalize the conclusion of \citet[Appendix A]{zhang1997momentum} which stipulate that the particle momentum equation is as legitimate as the phase averaged equation. 
In fact, it is not surprising at all, since the phases and particles averaged equations are all constructed from \ref{eq:dt_f_k}.
However, if one do not consider a proper derivation as it is done in \ref{sec:Lagrangian} it might not be as obvious, even if it should remain.
Nevertheless, it is important to note that this conclusion is not entirely objective since following the same procedure we could show equally that $\mathbb{C}_2  = -\div\mathbb{M}_1=0$ and $\mathbb{C}_2  = \frac{1}{2}\grad\grad:\mathbb{M}_2=0$ and so on for the other higher terms. 
Thus, it is more appropriate to examine the problem from the perspective of \ref{eq:scheme_equivalence}. 
Namely, the particle-averaged equations constitute a system of equations with one equation for each moment, while the phase-averaged equation contains all the terms of the particle-averaged equations within a single equation.
Therefore, the particle-averaged formalism encompasses more information since it provide one equation for each moment. 
This,  gain in information have been possible through the consideration of the topology of the dispersed phase. 

The major consequence of this finding is that it enable us to better understand the role of the particle phase stresses $\phi_2\bm{\sigma}_2$ and $\phi_I\bm{\sigma}_I$, in the particle averaged momentum equation such as it is written in a kinetic-like model. 
Indeed, as shown by the general form \ref{eq:avg_dt_dq_alpha_tot}, $\phi_2\bm{\sigma}_2$ and $\phi_I\bm{\sigma}_I$ will not play a role on the particle averaged momentum equations, i.e. the equation of : $n_p m_p \textbf{u}_p$. 
However, as shown in \ref{eq:dt_avg_uk2}, the phase averaged momentum : $\rho_2 \phi_2 \textbf{u}_2$,  will be subject to the particle surface and internal stresses, since $\phi_2\bm{\sigma}_2$ and $\phi_I\bm{\sigma}_I$ are present in this phase averaged equation.
This is made consistent if one consider that these stresses act as source terms on the higher moments of momentum equations $\mathbb{M}_1$\ldots, which are related to $\rho_2 \phi_2 \textbf{u}_2$ through \ref{eq:f_exp_exe}. 
In brief, the non-convective fluxes $\phi_2\bm{\sigma}_2$ and $\phi_I\bm{\sigma}_I$  have no direct impact on the particle averaged center of mass momentum :$n_pm_p\textbf{u}_p$, regardless of the particles nature and volume fraction. 
The influence of $\phi_2\bm{\sigma}_2$ and $\phi_I\bm{\sigma}_I$ on the particles' averaged momentum, $n_p m_p \textbf{u}_p$ is made through the dependence of the source terms present in \ref{eq:avg_dt_dq_alpha_tot} with the higher moments of the particles, which depend themselves on the non-convective fluxes as suggested by the moments of momentum equations. 
Consequently, it must be understood that the kinetic-like equations are formally exact and apply for any type of particle and particle volume fraction, as long as the closure terms are well modeled. 
Similar remarks can be made regrading the non-convective fluxes of the energy equation $\textbf{q}_2$ and for all other conservation equation. 



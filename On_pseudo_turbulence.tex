\documentclass[11pt]{My_preprint}
\title{
    Theoretical calculation of the droplet induced agitation (or pseudoturbulence) in mono disperse buoyant emulsions for low inertia and dilute regime.
    }

\author[1,2]{Nicolas Fintzi}
% \author[1]{Jean-Lou Pierson}
% \author[2]{Stephane Popinet}
\affil[1]{IFP Energies Nouvelles, Rond-point de l’echangeur de Solaize, 69360 Solaize}
\affil[2]{Sorbonne Universit\'e, Institut Jean le Rond d'Alembert, 4 place Jussieu, 75252 PARIS CEDEX 05, France}
\normalmarginpar


\begin{document}

\maketitle

\begin{abstract}
    In this study we derive an analytical expression for the continuous phase \textit{Reynolds stress} tensor. 
    This derivation is restricted to the dilute regime where the dispersed phase volume fraction labeled $\phi$ is ravishingly small and for particle Reynolds number $Re \ll 1$. 
    In this context we derive an analytical formula for the pseudoturbulence generated by translating bubbles or droplets inside a Newtonian fluid. 
    To by-pass the integral convergence difficulties generated due to the $\mathcal{O}(r^{-1})$ decay of the disturbance velocity field we make use of the Nearest-Particle-Statistical (NPS) as it is introduced in \citet{zhang2023evolution}. 
\end{abstract}

\section{Introduction}

The main result of this work is that the pseudoturbulent or \textit{Reynolds stress } tensor is shown to be equal to,
\begin{equation}
    \frac{\avg{\chi_f (\textbf{u}_f')_y (\textbf{u}_f')_y}}{\phi_f  U^2}
    = 
    \frac{63\lambda^2+84\lambda+28}{60\lambda^2+120\lambda+60} \Gamma\left(1/3\right)  \phi^{2/3}
    - \frac{17\lambda^2+22\lambda+7}{5\lambda^2+10\lambda+5} \phi
\end{equation}
\begin{equation}
    \frac{\avg{\chi_f (\textbf{u}_f')_x (\textbf{u}_f')_x}}{\phi_f  U^2}
    = 
    - \frac{6\lambda^2+6\lambda+1}{20\lambda^2+40\lambda+20} \phi
    +\frac{9\lambda^2+12\lambda+4}{240\lambda^2+480\lambda+240} \Gamma\left(1/3\right) \phi^{2/3}
\end{equation}
at first order in particle volume fraction. 
In tensor notation this reads, 
\begin{multline}
    \avg{\chi_f \textbf{u}_f'\textbf{u}_f'}
    =
    \frac{- (72\lambda^2 + 72\lambda + 12)\phi + 
    \left(9\lambda^2 + 12\lambda + 4\right)\Gamma(1/3) \phi^{2/3}}{240\lambda^2 + 480\lambda + 240}
    (\textbf{U}\cdot \textbf{U}) \textbf{I}\\
    + 
    \frac{- (248\lambda^2 + 328\lambda + 108)\phi + 
    \left(81\lambda^2 + 108\lambda + 36\right)\Gamma(1/3) \phi^{2/3}}{80\lambda^2 + 160\lambda + 80}
    \textbf{UU}\\
\end{multline}
\section{Preliminary definitions}

\subsection{ensemble average}
Let $\FF$ be a flow configuration and $d\PP = P(\FF)d\FF$ the probable number of flow in such a configuration. 
Equally, let $\textbf{u}_f^0(\textbf{x},t,\FF)$ be the local fluid velocity and $\chi_f(\textbf{x},t,\FF)$ the continuous phase indicator function evaluated at $\textbf{x}$ and time $t$ for a flow configuration $\FF$. 
The continuous phase averaged velocity is written as, 
\begin{equation}
    \phi_f \textbf{u}_f(\textbf{x},t)
    = \avg{\chi_f \textbf{u}_f^0} 
    = \int  \chi_f \textbf{u}_f^0(\textbf{x},t,\FF) d\PP
\end{equation}
where $\phi_f$ is the fluid phase volume fraction and $\avg{\ldots}$ denote an ensemble average operator. 
With these notations the \textit{Reynolds stress} tensor for the fluid phase can be written as 
\begin{equation}
    \phi_f \bm\sigma^\text{Re}
    =\avg{\chi_f \textbf{u}_f'\textbf{u}_f'}
    =\avg{\chi_f \textbf{u}_f^0 \textbf{u}_f^0 }
    - \phi_f \textbf{u}_f \textbf{u}_f
\end{equation}
\subsection{Nearest particle average}
In the objective of closing this term we now introduce the \textit{Nearest Particle Statistics} (NPS). 
We introduce the following relation \citet{zhang2021ensemble} :
\begin{equation*}
    \textbf{u}_f [\textbf{x},t]
    = 
    \int_{\mathbb{R}^3}
    \int_{\mathbb{R}^3}
    \textbf{u}_f^\text{nst}[\textbf{x},t, \textbf{r},\textbf{w}]
    P_\text{nst}[\textbf{r},\textbf{w}|\textbf{x},t]
    d\textbf{r}
    d\textbf{w}
\end{equation*}
with 
\begin{equation*}
    \textbf{u}_f^\text{nst}[\textbf{x},t; \textbf{r},\textbf{w}]
    P_\text{nst}[\textbf{r},\textbf{w};\textbf{x},t]
    =
    \frac{1}{\phi_f[\textbf{x},t]} 
    \int 
    \chi_f \textbf{u}_f^0[\textbf{x},t,\FF]
    \sum_i 
    \delta(\textbf{x}+\textbf{r}-\textbf{x}_i[t,\FF])
    \delta(\textbf{w}-\textbf{u}_i[t,\FF])
    h_i[\textbf{x},t,\FF]
    d\PP
\end{equation*}

\subsection{Reynolds stress decomposition}
\begin{equation}
    \bm\sigma^\text{Re}
    = 
    \int_{\mathbb{R}^3}
    \int_{\mathbb{R}^3}
    \textbf{v}_f^\text{nst}
    \textbf{v}_f^\text{nst}
    P_\text{nst}
    d\textbf{r}
    d\textbf{w}
    + 
    \frac{1}{\phi_f}
    \int_{\mathbb{R}^3}
    \int_{\mathbb{R}^3}
    \avg{
        \chi_f
        \textbf{u}_f''
        \textbf{u}_f''
        \sum_i 
        \delta(\textbf{x}+\textbf{r}-\textbf{x}_i)
        \delta(\textbf{w}-\textbf{u}_i)
        h_i
    }
    d\textbf{r}
    d\textbf{w}
\end{equation}
with 
\begin{align*}
    \text{fluctuation of the local value around the ensemble average : }\textbf{u}_f' = \textbf{u}_f^0 - \textbf{u}_f\\
    \text{fluctuation of the local value around the single particle conditional average : }\textbf{u}_f'' = \textbf{u}_f^0 - \textbf{u}_f^\text{nst}\\
    \text{Fluctuation of the single particle conditional average around the ensemble average : }\textbf{v}_f^\text{nst} = \textbf{u}_f^\text{nst} - \textbf{u}_f
\end{align*}
Assuming we neglect the second term we still need to solve for $\textbf{v}^\text{nst}$



Mass and momentum conservation without body forces, 
\begin{align*}
    \pddt \rho^0 + \div(\textbf{u}^0 \rho^0 ) = 0 \\
    \pddt (\textbf{u}^0 \rho^0) + \div(\textbf{u}^0 \textbf{u}^0 \rho^0 - \bm\sigma^0) = 0 \\
\end{align*}

\subsection*{The single particle conditional averaged equations}
Let takes 
$\pddt \delta(\textbf{x}_i(\FF,t) - \textbf{y}) 
= \frac{\partial \textbf{x}_i}{\partial t} \frac{\partial }{\partial \textbf{x}_i} \delta(\textbf{x}_i(\FF,t) - \textbf{y})
= - \textbf{u}_i \cdot \frac{\partial }{\partial \textbf{y}} \delta(\textbf{x}_i - \textbf{y})$
The Dirac delta function follows, This constrains,
\begin{align*}
    \pddt \delta(\textbf{x}_i(t,\FF)  - \textbf{y})
    +\textbf{u}_i(t,\FF)  \cdot \pddy [ \delta(\textbf{x}_i(t,\FF)  - \textbf{y})]
    = 0\\
    \pddt \delta(\textbf{u}_i(t,\FF) -\textbf{w})
    + \pddw \cdot [\textbf{a}_i(t,\FF)  \delta(\textbf{u}_i(t,\FF)  - \textbf{w})]
    = 0\\
    \pddt \chi_f(\textbf{x},t,\FF) 
    + \textbf{u}_\Gamma^0(\textbf{x},t,\FF) \cdot \pddx \chi_f(\textbf{x},t,\FF) = 0 \\
    \pddx \chi_f(\textbf{x},t,\FF) = - \delta_\Gamma \textbf{n}_f(\textbf{x},t,\FF)
\end{align*}
Thus, $\delta_1 =\sum_i \delta(\textbf{x}_i(t,\FF)  - \textbf{y})\delta(\textbf{u}_i(t,\FF) -\textbf{w})$ gives,
\begin{equation*}
    \pddt \delta_1 
    + \pddy\cdot [\textbf{w} \delta_1 ]
    + \pddw\cdot [\textbf{a}_i \delta_1 ]
    =0 
\end{equation*} 
multiplying by 
\begin{align*}
    \pddt (\textbf{u}^0 \rho^0\delta_1) + \div(\textbf{u}^0 \textbf{u}^0 \rho^0 \delta_1 - \bm\sigma^0\delta_1) 
    +\pddy\cdot (\textbf{u}^0 \rho^0  \textbf{w}\delta_1)
    +\pddw\cdot (\textbf{u}^0 \rho^0  \textbf{a}_i\delta_1) = 0 \\
\end{align*}
\begin{align*}
    \pddt \avg{\textbf{u}^0 \rho^0\delta_1} 
    + \div\avg{\textbf{u}^0 \textbf{u}^0 \rho^0 \delta_1 - \bm\sigma^0\delta_1}
    +\pddy\cdot \avg{\textbf{u}^0 \rho^0  \textbf{w}\delta_1}
    +\pddw\cdot \avg{\textbf{u}^0 \rho^0  \textbf{a}_i\delta_1} 
    = 0 \\
\end{align*}
Now let's consider that we neglect the fluctuation terms
\begin{align*}
    \pddt (\textbf{u}^1 \rho^1n_p) + \div(\textbf{u}^1 \textbf{u}^1 \rho^1 n_p - \bm\sigma^1n_p) 
    +\pddy\cdot (\textbf{u}^1 \rho^1  \textbf{w}n_p)
    +\pddw\cdot (\textbf{u}^1 \rho^1  \textbf{a}_in_p) = 0 \\
\end{align*}
The last terms can be re-written 
\begin{equation*}
    +\pddy\cdot (\textbf{u}^1 \rho^1  \textbf{w}n_p)
    = 
    \frac{\partial \textbf{x}}{\partial \textbf{y}}
    \frac{\partial }{\partial \textbf{x}}
    \cdot (\textbf{u}^1 \rho^1  \textbf{w}n_p)
    = 
    \frac{\partial }{\partial \textbf{x}}
    \cdot (\textbf{u}^1 \rho^1  \textbf{w}n_p)
\end{equation*}
\begin{equation*}
    +\pddw\cdot (\textbf{u}^1 \rho^1  \textbf{a}_i n_p)
    = 
    \frac{\partial \textbf{x}}{\partial \textbf{w}}
    \frac{\partial }{\partial \textbf{x}}
    \cdot (\textbf{u}^1 \rho^1  \textbf{a}_i n_p)
    = 
    \frac{\partial }{\partial \textbf{x}}
    \cdot (\textbf{u}^1 \rho^1  \textbf{a}_i n_p)
\end{equation*}
The first line assume $\textbf{x} = \textbf{y}- \textbf{r}$ the second must assume that $\textbf{x} = \textbf{w}-$
Neglecting the acc gives 
\begin{align*}
    \pddt (\textbf{u}^1 \rho^1n_p) + \div(\textbf{u}^1 \textbf{u}^1 \rho^1 n_p - \bm\sigma^1n_p) 
    +\pddx\cdot (\textbf{u}^1 \rho^1  \textbf{w}n_p)
    = 0 \\
\end{align*}
\subsubsection*{title}
\subsection*{The Nearest particle conditional averaged equations}
Let takes 
$\pddt \delta(\textbf{x}_i(\FF,t) - \textbf{y}) 
= \frac{\partial \textbf{x}_i}{\partial t} \frac{\partial }{\partial \textbf{x}_i} \delta(\textbf{x}_i(\FF,t) - \textbf{y})
= - \textbf{u}_i \cdot \frac{\partial }{\partial \textbf{y}} \delta(\textbf{x}_i - \textbf{y})$
The Dirac delta function follows, This constrains,
\begin{align*}
    \pddt \delta(\textbf{x}_i(t,\FF)  - \textbf{y})
    + \pddy \cdot [\textbf{u}_i(t,\FF)  \delta(\textbf{x}_i(t,\FF)  - \textbf{y})]
    = 0\\
    \pddt \delta(\textbf{u}_i(t,\FF) -\textbf{w})
    + \pddw \cdot [\textbf{a}_i(t,\FF)  \delta(\textbf{u}_i(t,\FF)  - \textbf{w})]
    = 0\\
    \pddt \chi_f(\textbf{x},t,\FF) 
    + \textbf{u}_\Gamma^0(\textbf{x},t,\FF) \cdot \pddx \chi_f(\textbf{x},t,\FF) = 0 \\
    \pddx \chi_f(\textbf{x},t,\FF) = - \delta_\Gamma \textbf{n}_f(\textbf{x},t,\FF)
\end{align*}
Note that since $\textbf{y} = \textbf{x} + \textbf{r}$, 
\begin{equation*}
    \frac{\partial }{\partial \textbf{r}}
    = \frac{\partial }{\partial \textbf{x}}
    \frac{\partial (\textbf{y} - \textbf{r})}{\partial \textbf{r}}
    = - \frac{\partial }{\partial \textbf{x}}
    = \frac{\partial }{\partial \textbf{y}}
    \frac{\partial (\textbf{x} + \textbf{r})}{\partial \textbf{r}}
    = \frac{\partial }{\partial \textbf{y}}
\end{equation*}
\begin{equation*}
    \frac{\partial }{\partial \textbf{x}}
    = \frac{\partial }{\partial \textbf{y}}
    \frac{\partial (\textbf{x} + \textbf{r})}{\partial \textbf{x}}
    = \frac{\partial }{\partial \textbf{y}}
    = \frac{\partial }{\partial \textbf{r}}
    \frac{\partial (\textbf{y} - \textbf{x})}{\partial \textbf{x}}
    = - \frac{\partial }{\partial \textbf{r}}
\end{equation*}
\begin{equation*}
    \frac{\partial }{\partial \textbf{y}}
    = \frac{\partial }{\partial \textbf{x}}
    \frac{\partial (\textbf{y} - \textbf{r})}{\partial \textbf{y}}
    = \frac{\partial }{\partial \textbf{x}}
    = \frac{\partial }{\partial \textbf{r}}
    \frac{\partial (\textbf{y} - \textbf{x})}{\partial \textbf{y}}
    = \frac{\partial }{\partial \textbf{r}}
\end{equation*}
So the evolution equation for 
$\delta_\text{nst}(\textbf{x},\textbf{y},\textbf{w},t,\FF) =  \sum_i \delta(\textbf{x}_i -\textbf{y}) \delta(\textbf{u}_i - \textbf{w}) h_i(\textbf{x})$
\begin{align*}
    \pddt \delta_{nst}
    + \pddy \cdot (\textbf{w} \delta_{nst})
    + \pddw \cdot (\textbf{a}_i  \delta_{nst})
    = 
    \delta(\textbf{x}_i -\textbf{y}) \delta(\textbf{u}_i - \textbf{w}) \pddt h_i
\end{align*}

The conditional mass and momentum equations then read,
\begin{align*}
    \pddt (\rho^0 \delta_{nst}) 
    + \pddx \cdot (\textbf{u}^0 \rho^0 \delta_{nst} ) 
    + \pddy \cdot (\textbf{w}   \rho^0 \delta_{nst})
    + \pddw \cdot (\textbf{a}_i \rho^0 \delta_{nst})
    = 
    \rho^0 \delta(\textbf{x}_i -\textbf{y}) \delta(\textbf{u}_i - \textbf{w}) 
    [\pddt h_i 
    +  \textbf{u}^0 \cdot \pddx h_i] 
    \\
    \pddt (\textbf{u}^0 \rho^0 \delta_{nst}) 
    + \pddx \cdot (\textbf{u}^0 \textbf{u}^0 \rho^0 \delta_{nst} - \bm\sigma^0 \delta_{nst}) 
    + \pddy \cdot (\textbf{u}^0\rho^0 \textbf{w}   \delta_{nst}) 
    + \pddw \cdot (\textbf{u}^0\rho^0 \textbf{a}_i \delta_{nst}) \\
    = \rho^0 \textbf{u}^0  
    \delta(\textbf{x}_i -\textbf{y}) \delta(\textbf{u}_i - \textbf{w}) 
    [
        \pddt h_i 
        + \textbf{u}^0\cdot \pddx\delta_{nst}
        ]
    \\
\end{align*}
ensemble average the eq for $\delta_{nst}$ gives, 
\begin{align*}
    \pddt P_{nst}
    + \pddy \cdot (\textbf{w} P_{nst})
    + \pddw \cdot (\textbf{a}_p^\text{nst}  P_{nst})
    = 
    \avg{\delta(\textbf{x}_i -\textbf{y}) \delta(\textbf{u}_i - \textbf{w}) \pddt h_i}
\end{align*}
\begin{align*}
    \pddt (\rho^{nst} P_{nst}) 
    + \pddx \cdot \avg{\textbf{u}^0 \rho^0 \delta_{nst} }
    + \pddy \cdot (\textbf{w}   \rho^{nst} P_{nst})
    + \pddw \cdot \avg{\textbf{a}_i \rho^0 \delta_{nst}}
    = 
    \avg{\rho^0 \delta(\textbf{x}_i -\textbf{y}) \delta(\textbf{u}_i - \textbf{w}) D_t h_i }
    \\
    \pddt \avg{\textbf{u}^0 \rho^0 \delta_{nst}}
    + \pddx \cdot \avg{\textbf{u}^0 \textbf{u}^0 \rho^0 \delta_{nst} - \bm\sigma^0 \delta_{nst}}
    + \pddy \cdot \avg{\textbf{u}^0\rho^0 \textbf{w}   \delta_{nst} }
    + \pddw \cdot \avg{\textbf{u}^0\rho^0 \textbf{a}_i \delta_{nst} }
    = \avg{\rho^0 \textbf{u}^0  
    \delta(\textbf{x}_i -\textbf{y}) \delta(\textbf{u}_i - \textbf{w})D_t h_i }
    \\
\end{align*}

Ensembel avg time $P_{nst}$ gives, 
\begin{align*}
    \pddt (\rho P_{nst}) 
    + \pddx \cdot (\avg{\textbf{u}^0 \rho^0 } P_{nst})
    + \pddy \cdot (\textbf{w}  P_{nst} \rho)
    + \pddw \cdot (\textbf{a}_p^{nst} P_{nst}\rho)
    = \\
    \rho \avg{\delta(\textbf{x}_i -\textbf{y}) \delta(\textbf{u}_i - \textbf{w}) \pddt h_i }
    \\
    \pddt (\avg{\textbf{u}^0 \rho^0 } P_{nst})
    + \pddx \cdot (\avg{\textbf{u}^0 \textbf{u}^0 \rho^0  - \bm\sigma^0 } P_{nst})
    + \pddy \cdot ( \avg{\rho^0 \textbf{u}^0} \textbf{w}   P_{nst} )
    + \pddw \cdot ( \avg{\rho^0 \textbf{u}^0} \textbf{a}_p^{nst} P_{nst} )
    = \\
    \avg{\rho^0 \textbf{u}^0}  
    \avg{\delta(\textbf{x}_i -\textbf{y}) \delta(\textbf{u}_i - \textbf{w}) \pddt h_i }
    \\
\end{align*}


\subsection*{Dilute assumption}
\begin{equation*}
    \avg{\rho^0 \textbf{u}^0 \delta_{nst}(\textbf{y},\textbf{w},t,\FF)} 
    = 
    \avg{\chi_f\rho_f \textbf{u}^0_f(\textbf{x},t,\FF) \delta_{nst}(\textbf{y},\textbf{w},t,\FF)} 
    + \avg{\chi_d\rho_d \textbf{u}^0_d(\textbf{x},t,\FF) \delta_{nst}(\textbf{y},\textbf{w},t,\FF)} 
\end{equation*}
If $|\textbf{x}-\textbf{y}| > a$ the $\chi_d = 0$ and $\chi_f = 1$ since at \textbf{y} it is the nearest
\begin{equation*}
    \avg{\rho^0 \textbf{u}^0 \delta_{nst}(\textbf{y},\textbf{w},t,\FF)} 
    = 
    \phi_f \rho_f \textbf{u}_f^{nst} P_{nst|f}
    + \phi_d \rho_d \textbf{u}_d^{nst} P_{nst|d}
\end{equation*}
\begin{align*}
    \pddt (\rho_f P_{nst}) 
    + \pddx \cdot (\rho_f \textbf{u}_f^{nst} P_{nst})
    + \pddy \cdot (\textbf{w}   \rho_f P_{nst})
    + \pddw \cdot ( \rho_f \textbf{a}_p^{nst} P_{nst})
    = 
    \avg{\rho^0 \delta(\textbf{x}_i -\textbf{y}) \delta(\textbf{u}_i - \textbf{w}) \pddt h_i }
    \\
    \pddt ( \rho_f\textbf{u}_f^{nst} P_{nst})
    + \pddx \cdot ( \rho_f \textbf{u}_f^{nst}\textbf{u}_f^{nst}P_{nst} + \avg{\textbf{u}''\textbf{u}''\delta_{nst}} - \bm\sigma_f^{nst} P_{nst})\\
    + \pddy \cdot (\textbf{u}^{nst}_f \rho_f \textbf{w}   P_{nst} )
    + \pddw \cdot (\textbf{u}^{nst}_f \rho_f \textbf{a}_p^{nst} P_{nst} )
    = \avg{\rho^0 \textbf{u}^0  
    \delta(\textbf{x}_i -\textbf{y}) \delta(\textbf{u}_i - \textbf{w}) \pddt h_i }
    \\
\end{align*}

The difficultly is on the term 
\begin{equation*}
    \pddy \cdot \avg{\textbf{u}^0\rho^0 \textbf{w}   \delta_{nst} }
    = 
    \pddy \cdot \avg{\textbf{u}^0(\textbf{x})\rho^0(\textbf{x}) \textbf{w}   \sum_i \delta(\textbf{x}+ \textbf{r} - \textbf{x}_i)\delta(\textbf{w}- \textbf{u}_i) h_i(\textbf{x}) }
\end{equation*}
\tb{that s all wrong because we $\delta_{nst} = f(\textbf{x})$}


\begin{equation*}
    \frac{\partial }{\partial \textbf{y}}
    = 
    \frac{\partial }{\partial \textbf{x}}
    \frac{\partial (\textbf{y} - \textbf{r} )}{\partial \textbf{y}}
\end{equation*}

\input{On_pseudo_turbulence/One_point_state.tex}

\bibliography{Bib/bib_bulles.bib}
\appendix

\section*{Computation of the derive of $h_i$}

The definitions with $r_i = |\textbf{x}_i - \textbf{x}|$.  
\begin{equation*}
    h_i(\textbf{x},t,\FF)
    = 
    \frac{1}{N(\textbf{x},t,\FF)}
    \prod_j
    H(r_j - r_i)
\end{equation*}
and 
\begin{equation*}
    N(\textbf{x},t\FF)
    = \sum_i \prod_j 
    H(r_j - r_i)
\end{equation*}

The Gradient of this function gives
\begin{equation*}
    \pddt  h_i(\textbf{x},t,\FF)
    = 
    -  \frac{\pddt N(\textbf{x},t,\FF)}{N(\textbf{x},t,\FF)^2}
    \prod_j
    H(r_j - r_i)
    + \frac{1}{N(\textbf{x},t,\FF)}
    \pddt \prod_j
    H(r_j - r_i)
\end{equation*}
\begin{equation*}
    \pddx  h_i(\textbf{x},t,\FF)
    = 
    -  \frac{\pddx N(\textbf{x},t,\FF)}{N(\textbf{x},t,\FF)^2}
    \prod_j
    H(r_j - r_i)
    + \frac{1}{N(\textbf{x},t,\FF)}
    \pddx \prod_j
    H(r_j - r_i)
\end{equation*}
with, 
\begin{equation*}
    \pddt 
    \prod_j
    H(r_j - r_i)
    = 
    \sum_k 
    \delta(r_k - r_i)
    (\textbf{u}_k  \cdot \hat{\textbf{r}}_k - \textbf{u}_i  \cdot \hat{\textbf{r}}_i)
    \prod_{j\neq k}
    H(r_j - r_i)
\end{equation*}
\begin{equation*}
    \pddx
    \prod_j
    H(r_j - r_i)
    = 
    \sum_k 
    \delta(r_k - r_i)
    ( \hat{\textbf{r}}_i -  \hat{\textbf{r}}_k)
    \prod_{j\neq k}
    H(r_j - r_i)
\end{equation*}

\begin{align*}
    \pddt r_k
    = \pddt [(\textbf{x}_k(\FF,t) - \textbf{x})\cdot (\textbf{x}_k(\FF,t) - \textbf{x})]^{1/2}\\
    = 
    2 \textbf{u}_k(\FF,t)  \cdot (\textbf{x}_k(\FF,t) - \textbf{x})
    \frac{1}{2}[(\textbf{x}_k(\FF,t) - \textbf{x})\cdot (\textbf{x}_k(\FF,t) - \textbf{x})]^{- 1/2}
    = 
    \textbf{u}_k  \cdot \hat{\textbf{r}}_k
\end{align*}
\begin{align*}
    \pddx r_k
    = \pddx [(\textbf{x}_k(\FF,t) - \textbf{x})\cdot (\textbf{x}_k(\FF,t) - \textbf{x})]^{1/2}
    = - \hat{\textbf{r}}_k
\end{align*}

The $N$ function derivative gives 
\begin{equation*}
    \pddt 
    N 
    = \pddt 
    \sum_i \prod_j
    H(r_j - r_i)
    = 
    \sum_i 
    \sum_k 
    \delta(r_k - r_i)
    (\textbf{u}_k  \cdot \hat{\textbf{r}}_k - \textbf{u}_i  \cdot \hat{\textbf{r}}_i)
    \prod_{j\neq k}
    H(r_j - r_i) = 0 
\end{equation*}

Thus, we have \citet{zhang2021ensemble} 
\begin{align*}
    \pddt  h_i(\textbf{x},t,\FF)
    = 
    \frac{1}{N(\textbf{x},t,\FF)}
    \sum_k 
    \delta(r_k - r_i)
    (\textbf{u}_k  \cdot \hat{\textbf{r}}_k - \textbf{u}_i  \cdot \hat{\textbf{r}}_i)
    \prod_{j\neq k}
    H(r_j - r_i)
    \\
    \textbf{u}_i \cdot \grad h_i(\textbf{x},t,\FF)
    = 
    \frac{1}{N(\textbf{x},t,\FF)}
    \sum_k 
    \delta(r_k - r_i)
    (\textbf{u}_i  \cdot \hat{\textbf{r}}_i -  \textbf{u}_i  \cdot\hat{\textbf{r}}_k)
    \prod_{j\neq k}
    H(r_j - r_i)
    \\
\end{align*}
Since $\delta(r_k - r_i)$ we have $r_k = r_i$ therefor $H(r_k - r_i) = H(0) = 1 $
\begin{align*}
    \pddt  h_i(\textbf{x},t,\FF)
    = 
    \sum_k 
    \delta(r_k - r_i)
    (\textbf{u}_k  \cdot \hat{\textbf{r}}_k - \textbf{u}_i  \cdot \hat{\textbf{r}}_i)
    h_i
    \\
    \textbf{u}^0 \cdot \grad h_i(\textbf{x},t,\FF)
    = 
    \sum_k 
    \delta(r_k - r_i)
    (\textbf{u}^0  \cdot \hat{\textbf{r}}_i -  \textbf{u}^0  \cdot\hat{\textbf{r}}_k)
    h_i
    \\
\end{align*}
Idea, maybe if i sum on i it cancel ? 
\begin{align*}
    \pddt  h_i(\textbf{x},t,\FF) +\textbf{u}^0 \cdot \grad h_i(\textbf{x},t,\FF)
    &= 
    \sum_k 
    \delta(r_k - r_i)
    (\textbf{u}_k  \cdot \hat{\textbf{r}}_k  - \textbf{u}_i  \cdot \hat{\textbf{r}}_k)
    h_i
    \\
    &= 
    \sum_k 
    \delta(r_k - r_i)
    [(\textbf{u}_k  - \textbf{u}^0) \cdot \hat{\textbf{r}}_k - (\textbf{u}_i - \textbf{u}^0)  \cdot \hat{\textbf{r}}_i]
    h_i
    \\
\end{align*}
This can be treated in a similar way that previously

ensemble average these equations gives 

\end{document}


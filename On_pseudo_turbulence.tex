\documentclass[11pt]{My_preprint}
\title{
    Theoretical calculation of the droplet induced agitation (or pseudoturbulence) in mono disperse buoyant emulsions for low inertia and dilute regime.
    }

\author[1,2]{Nicolas Fintzi}
% \author[1]{Jean-Lou Pierson}
% \author[2]{Stephane Popinet}
\affil[1]{IFP Energies Nouvelles, Rond-point de l’echangeur de Solaize, 69360 Solaize}
\affil[2]{Sorbonne Universit\'e, Institut Jean le Rond d'Alembert, 4 place Jussieu, 75252 PARIS CEDEX 05, France}
\normalmarginpar


\begin{document}

\maketitle

\begin{abstract}
    In this study we derive an analytical expression for the continuous phase \textit{Reynolds stress} tensor. 
    This derivation is restricted to the dilute regime where the dispersed phase volume fraction labeled $\phi$ is ravishingly small and for particle Reynolds number $Re \ll 1$. 
    In this context we derive an analytical formula for the pseudoturbulence generated by translating bubbles or droplets inside a Newtonian fluid. 
    To by-pass the integral convergence difficulties generated due to the $\mathcal{O}(r^{-1})$ decay of the disturbance velocity field we make use of the Nearest-Particle-Statistical (NPS) as it is introduced in \citet{zhang2023evolution}. 
\end{abstract}

\section{Introduction}

The main result of this work is that the pseudoturbulent or \textit{Reynolds stress } tensor is shown to be equal to,
\begin{equation}
    \frac{\avg{\chi_f (\textbf{u}_f')_y (\textbf{u}_f')_y}}{\phi_f  U^2}
    = 
    \frac{63\lambda^2+84\lambda+28}{60\lambda^2+120\lambda+60} \Gamma\left(1/3\right)  \phi^{2/3}
    - \frac{17\lambda^2+22\lambda+7}{5\lambda^2+10\lambda+5} \phi
\end{equation}
\begin{equation}
    \frac{\avg{\chi_f (\textbf{u}_f')_x (\textbf{u}_f')_x}}{\phi_f  U^2}
    = 
    - \frac{6\lambda^2+6\lambda+1}{20\lambda^2+40\lambda+20} \phi
    +\frac{9\lambda^2+12\lambda+4}{240\lambda^2+480\lambda+240} \Gamma\left(1/3\right) \phi^{2/3}
\end{equation}
at first order in particle volume fraction. 
In tensor notation this reads, 
\begin{multline}
    \avg{\chi_f \textbf{u}_f'\textbf{u}_f'}
    =
    \frac{- (72\lambda^2 + 72\lambda + 12)\phi + 
    \left(9\lambda^2 + 12\lambda + 4\right)\Gamma(1/3) \phi^{2/3}}{240\lambda^2 + 480\lambda + 240}
    (\textbf{U}\cdot \textbf{U}) \textbf{I}\\
    + 
    \frac{- (248\lambda^2 + 328\lambda + 108)\phi + 
    \left(81\lambda^2 + 108\lambda + 36\right)\Gamma(1/3) \phi^{2/3}}{80\lambda^2 + 160\lambda + 80}
    \textbf{UU}\\
\end{multline}
\section{Preliminary definitions}

\subsection{ensemble average}
Let $\FF$ be a flow configuration and $d\PP = P(\FF)d\FF$ the probable number of flow in such a configuration. 
Equally, let $\textbf{u}_f^0(\textbf{x},t,\FF)$ be the local fluid velocity and $\chi_f(\textbf{x},t,\FF)$ the continuous phase indicator function evaluated at $\textbf{x}$ and time $t$ for a flow configuration $\FF$. 
The continuous phase averaged velocity is written as, 
\begin{equation}
    \phi_f \textbf{u}_f(\textbf{x},t)
    = \avg{\chi_f \textbf{u}_f^0} 
    = \int  \chi_f \textbf{u}_f^0(\textbf{x},t,\FF) d\PP
\end{equation}
where $\phi_f$ is the fluid phase volume fraction and $\avg{\ldots}$ denote an ensemble average operator. 
With these notations the \textit{Reynolds stress} tensor for the fluid phase can be written as 
\begin{equation}
    \phi_f \bm\sigma^\text{Re}
    =\avg{\chi_f \textbf{u}_f'\textbf{u}_f'}
    =\avg{\chi_f \textbf{u}_f^0 \textbf{u}_f^0 }
    - \phi_f \textbf{u}_f \textbf{u}_f
\end{equation}
\subsection{Nearest particle average}
In the objective of closing this term we now introduce the \textit{Nearest Particle Statistics} (NPS). 
We introduce the following relation \citet{zhang2021ensemble} :
\begin{equation*}
    \textbf{u}_f [\textbf{x},t]
    = 
    \int_{\mathbb{R}^3}
    \int_{\mathbb{R}^3}
    \textbf{u}_f^\text{nst}[\textbf{x},t, \textbf{r},\textbf{w}]
    P_\text{nst}[\textbf{r},\textbf{w}|\textbf{x},t]
    d\textbf{r}
    d\textbf{w}
\end{equation*}
with 
\begin{equation*}
    \textbf{u}_f^\text{nst}[\textbf{x},t; \textbf{r},\textbf{w}]
    P_\text{nst}[\textbf{r},\textbf{w};\textbf{x},t]
    =
    \frac{1}{\phi_f[\textbf{x},t]} 
    \int 
    \chi_f \textbf{u}_f^0[\textbf{x},t,\FF]
    \sum_i 
    \delta(\textbf{x}+\textbf{r}-\textbf{x}_i[t,\FF])
    \delta(\textbf{w}-\textbf{u}_i[t,\FF])
    h_i[\textbf{x},t,\FF]
    d\PP
\end{equation*}

\subsection{Reynolds stress decomposition}
\begin{equation}
    \bm\sigma^\text{Re}
    = 
    \int_{\mathbb{R}^3}
    \int_{\mathbb{R}^3}
    \textbf{v}_f^\text{nst}
    \textbf{v}_f^\text{nst}
    P_\text{nst}
    d\textbf{r}
    d\textbf{w}
    + 
    \frac{1}{\phi_f}
    \int_{\mathbb{R}^3}
    \int_{\mathbb{R}^3}
    \avg{
        \chi_f
        \textbf{u}_f''
        \textbf{u}_f''
        \sum_i 
        \delta(\textbf{x}+\textbf{r}-\textbf{x}_i)
        \delta(\textbf{w}-\textbf{u}_i)
        h_i
    }
    d\textbf{r}
    d\textbf{w}
\end{equation}
with 
\begin{align*}
    \text{fluctuation of the local value around the ensemble average : }\textbf{u}_f' = \textbf{u}_f^0 - \textbf{u}_f\\
    \text{fluctuation of the local value around the single particle conditional average : }\textbf{u}_f'' = \textbf{u}_f^0 - \textbf{u}_f^\text{nst}\\
    \text{Fluctuation of the single particle conditional average around the ensemble average : }\textbf{v}_f^\text{nst} = \textbf{u}_f^\text{nst} - \textbf{u}_f
\end{align*}
Assuming we neglect the second term we still need to solve for $\textbf{v}^\text{nst}$



Mass and momentum conservation without body forces, 
\begin{align*}
    \pddt \rho^0 + \div(\textbf{u}^0 \rho^0 ) = 0 \\
    \pddt (\textbf{u}^0 \rho^0) + \div(\textbf{u}^0 \textbf{u}^0 \rho^0 - \bm\sigma^0) = 0 \\
\end{align*}

\subsection*{The single particle conditional averaged equations}
Let takes 
$\pddt \delta(\textbf{x}_i(\FF,t) - \textbf{y}) 
= \frac{\partial \textbf{x}_i}{\partial t} \frac{\partial }{\partial \textbf{x}_i} \delta(\textbf{x}_i(\FF,t) - \textbf{y})
= - \textbf{u}_i \cdot \frac{\partial }{\partial \textbf{y}} \delta(\textbf{x}_i - \textbf{y})$
The Dirac delta function follows, This constrains,
\begin{align*}
    \pddt \delta(\textbf{x}_i(t,\FF)  - \textbf{y})
    +\textbf{u}_i(t,\FF)  \cdot \pddy [ \delta(\textbf{x}_i(t,\FF)  - \textbf{y})]
    = 0\\
    \pddt \delta(\textbf{u}_i(t,\FF) -\textbf{w})
    + \pddw \cdot [\textbf{a}_i(t,\FF)  \delta(\textbf{u}_i(t,\FF)  - \textbf{w})]
    = 0\\
    \pddt \chi_f(\textbf{x},t,\FF) 
    + \textbf{u}_\Gamma^0(\textbf{x},t,\FF) \cdot \pddx \chi_f(\textbf{x},t,\FF) = 0 \\
    \pddx \chi_f(\textbf{x},t,\FF) = - \delta_\Gamma \textbf{n}_f(\textbf{x},t,\FF)
\end{align*}
Thus, $\delta_1 =\sum_i \delta(\textbf{x}_i(t,\FF)  - \textbf{y})\delta(\textbf{u}_i(t,\FF) -\textbf{w})$ gives,
\begin{equation*}
    \pddt \delta_1 
    + \pddy\cdot [\textbf{w} \delta_1 ]
    + \pddw\cdot [\textbf{a}_i \delta_1 ]
    =0 
\end{equation*} 
multiplying by 
\begin{align*}
    \pddt (\textbf{u}^0 \rho^0\delta_1) + \div(\textbf{u}^0 \textbf{u}^0 \rho^0 \delta_1 - \bm\sigma^0\delta_1) 
    +\pddy\cdot (\textbf{u}^0 \rho^0  \textbf{w}\delta_1)
    +\pddw\cdot (\textbf{u}^0 \rho^0  \textbf{a}_i\delta_1) = 0 \\
\end{align*}
\begin{align*}
    \pddt \avg{\textbf{u}^0 \rho^0\delta_1} 
    + \div\avg{\textbf{u}^0 \textbf{u}^0 \rho^0 \delta_1 - \bm\sigma^0\delta_1}
    +\pddy\cdot \avg{\textbf{u}^0 \rho^0  \textbf{w}\delta_1}
    +\pddw\cdot \avg{\textbf{u}^0 \rho^0  \textbf{a}_i\delta_1} 
    = 0 \\
\end{align*}
Now let's consider that we neglect the fluctuation terms
\begin{align*}
    \pddt (\textbf{u}^1 \rho^1n_p) + \div(\textbf{u}^1 \textbf{u}^1 \rho^1 n_p - \bm\sigma^1n_p) 
    +\pddy\cdot (\textbf{u}^1 \rho^1  \textbf{w}n_p)
    +\pddw\cdot (\textbf{u}^1 \rho^1  \textbf{a}_in_p) = 0 \\
\end{align*}
The last terms can be re-written 
\begin{equation*}
    +\pddy\cdot (\textbf{u}^1 \rho^1  \textbf{w}n_p)
    = 
    \frac{\partial \textbf{x}}{\partial \textbf{y}}
    \frac{\partial }{\partial \textbf{x}}
    \cdot (\textbf{u}^1 \rho^1  \textbf{w}n_p)
    = 
    \frac{\partial }{\partial \textbf{x}}
    \cdot (\textbf{u}^1 \rho^1  \textbf{w}n_p)
\end{equation*}
\begin{equation*}
    +\pddw\cdot (\textbf{u}^1 \rho^1  \textbf{a}_i n_p)
    = 
    \frac{\partial \textbf{x}}{\partial \textbf{w}}
    \frac{\partial }{\partial \textbf{x}}
    \cdot (\textbf{u}^1 \rho^1  \textbf{a}_i n_p)
    = 
    \frac{\partial }{\partial \textbf{x}}
    \cdot (\textbf{u}^1 \rho^1  \textbf{a}_i n_p)
\end{equation*}
The first line assume $\textbf{x} = \textbf{y}- \textbf{r}$ the second must assume that $\textbf{x} = \textbf{w}-$
Neglecting the acc gives 
\begin{align*}
    \pddt (\textbf{u}^1 \rho^1n_p) + \div(\textbf{u}^1 \textbf{u}^1 \rho^1 n_p - \bm\sigma^1n_p) 
    +\pddx\cdot (\textbf{u}^1 \rho^1  \textbf{w}n_p)
    = 0 \\
\end{align*}
\subsubsection*{title}
\subsection*{The Nearest particle conditional averaged equations}
Let takes 
$\pddt \delta(\textbf{x}_i(\FF,t) - \textbf{y}) 
= \frac{\partial \textbf{x}_i}{\partial t} \frac{\partial }{\partial \textbf{x}_i} \delta(\textbf{x}_i(\FF,t) - \textbf{y})
= - \textbf{u}_i \cdot \frac{\partial }{\partial \textbf{y}} \delta(\textbf{x}_i - \textbf{y})$
The Dirac delta function follows, This constrains,
\begin{align*}
    \pddt \delta(\textbf{x}_i(t,\FF)  - \textbf{y})
    + \pddy \cdot [\textbf{u}_i(t,\FF)  \delta(\textbf{x}_i(t,\FF)  - \textbf{y})]
    = 0\\
    \pddt \delta(\textbf{u}_i(t,\FF) -\textbf{w})
    + \pddw \cdot [\textbf{a}_i(t,\FF)  \delta(\textbf{u}_i(t,\FF)  - \textbf{w})]
    = 0\\
    \pddt \chi_f(\textbf{x},t,\FF) 
    + \textbf{u}_\Gamma^0(\textbf{x},t,\FF) \cdot \pddx \chi_f(\textbf{x},t,\FF) = 0 \\
    \pddx \chi_f(\textbf{x},t,\FF) = - \delta_\Gamma \textbf{n}_f(\textbf{x},t,\FF)
\end{align*}
Note that since $\textbf{y} = \textbf{x} + \textbf{r}$, 
\begin{equation*}
    \frac{\partial }{\partial \textbf{r}}
    = \frac{\partial }{\partial \textbf{x}}
    \frac{\partial (\textbf{y} - \textbf{r})}{\partial \textbf{r}}
    = - \frac{\partial }{\partial \textbf{x}}
    = \frac{\partial }{\partial \textbf{y}}
    \frac{\partial (\textbf{x} + \textbf{r})}{\partial \textbf{r}}
    = \frac{\partial }{\partial \textbf{y}}
\end{equation*}
\begin{equation*}
    \frac{\partial }{\partial \textbf{x}}
    = \frac{\partial }{\partial \textbf{y}}
    \frac{\partial (\textbf{x} + \textbf{r})}{\partial \textbf{x}}
    = \frac{\partial }{\partial \textbf{y}}
    = \frac{\partial }{\partial \textbf{r}}
    \frac{\partial (\textbf{y} - \textbf{x})}{\partial \textbf{x}}
    = - \frac{\partial }{\partial \textbf{r}}
\end{equation*}
\begin{equation*}
    \frac{\partial }{\partial \textbf{y}}
    = \frac{\partial }{\partial \textbf{x}}
    \frac{\partial (\textbf{y} - \textbf{r})}{\partial \textbf{y}}
    = \frac{\partial }{\partial \textbf{x}}
    = \frac{\partial }{\partial \textbf{r}}
    \frac{\partial (\textbf{y} - \textbf{x})}{\partial \textbf{y}}
    = \frac{\partial }{\partial \textbf{r}}
\end{equation*}
So the evolution equation for 
$\delta_\text{nst}(\textbf{x},\textbf{y},\textbf{w},t,\FF) =  \sum_i \delta(\textbf{x}_i -\textbf{y}) \delta(\textbf{u}_i - \textbf{w}) h_i(\textbf{x})$
\begin{align*}
    \pddt \delta_{nst}
    + \pddy \cdot (\textbf{w} \delta_{nst})
    + \pddw \cdot (\textbf{a}_i  \delta_{nst})
    = 
    \delta(\textbf{x}_i -\textbf{y}) \delta(\textbf{u}_i - \textbf{w}) \pddt h_i
\end{align*}

The conditional mass and momentum equations then read,
\begin{align*}
    \pddt (\rho^0 \delta_{nst}) 
    + \pddx \cdot (\textbf{u}^0 \rho^0 \delta_{nst} ) 
    + \pddy \cdot (\textbf{w}   \rho^0 \delta_{nst})
    + \pddw \cdot (\textbf{a}_i \rho^0 \delta_{nst})
    = 
    \rho^0 \delta(\textbf{x}_i -\textbf{y}) \delta(\textbf{u}_i - \textbf{w}) 
    [\pddt h_i 
    +  \textbf{u}^0 \cdot \pddx h_i] 
    \\
    \pddt (\textbf{u}^0 \rho^0 \delta_{nst}) 
    + \pddx \cdot (\textbf{u}^0 \textbf{u}^0 \rho^0 \delta_{nst} - \bm\sigma^0 \delta_{nst}) 
    + \pddy \cdot (\textbf{u}^0\rho^0 \textbf{w}   \delta_{nst}) 
    + \pddw \cdot (\textbf{u}^0\rho^0 \textbf{a}_i \delta_{nst}) \\
    = \rho^0 \textbf{u}^0  
    \delta(\textbf{x}_i -\textbf{y}) \delta(\textbf{u}_i - \textbf{w}) 
    [
        \pddt h_i 
        + \textbf{u}^0\cdot \pddx\delta_{nst}
        ]
    \\
\end{align*}
ensemble average the eq for $\delta_{nst}$ gives, 
\begin{align*}
    \pddt P_{nst}
    + \pddy \cdot (\textbf{w} P_{nst})
    + \pddw \cdot (\textbf{a}_p^\text{nst}  P_{nst})
    = 
    \avg{\delta(\textbf{x}_i -\textbf{y}) \delta(\textbf{u}_i - \textbf{w}) \pddt h_i}
\end{align*}
\begin{align*}
    \pddt (\rho^{nst} P_{nst}) 
    + \pddx \cdot \avg{\textbf{u}^0 \rho^0 \delta_{nst} }
    + \pddy \cdot (\textbf{w}   \rho^{nst} P_{nst})
    + \pddw \cdot \avg{\textbf{a}_i \rho^0 \delta_{nst}}
    = 
    \avg{\rho^0 \delta(\textbf{x}_i -\textbf{y}) \delta(\textbf{u}_i - \textbf{w}) D_t h_i }
    \\
    \pddt \avg{\textbf{u}^0 \rho^0 \delta_{nst}}
    + \pddx \cdot \avg{\textbf{u}^0 \textbf{u}^0 \rho^0 \delta_{nst} - \bm\sigma^0 \delta_{nst}}
    + \pddy \cdot \avg{\textbf{u}^0\rho^0 \textbf{w}   \delta_{nst} }
    + \pddw \cdot \avg{\textbf{u}^0\rho^0 \textbf{a}_i \delta_{nst} }
    = \avg{\rho^0 \textbf{u}^0  
    \delta(\textbf{x}_i -\textbf{y}) \delta(\textbf{u}_i - \textbf{w})D_t h_i }
    \\
\end{align*}

Ensembel avg time $P_{nst}$ gives, 
\begin{align*}
    \pddt (\rho P_{nst}) 
    + \pddx \cdot (\avg{\textbf{u}^0 \rho^0 } P_{nst})
    + \pddy \cdot (\textbf{w}  P_{nst} \rho)
    + \pddw \cdot (\textbf{a}_p^{nst} P_{nst}\rho)
    = \\
    \rho \avg{\delta(\textbf{x}_i -\textbf{y}) \delta(\textbf{u}_i - \textbf{w}) \pddt h_i }
    \\
    \pddt (\avg{\textbf{u}^0 \rho^0 } P_{nst})
    + \pddx \cdot (\avg{\textbf{u}^0 \textbf{u}^0 \rho^0  - \bm\sigma^0 } P_{nst})
    + \pddy \cdot ( \avg{\rho^0 \textbf{u}^0} \textbf{w}   P_{nst} )
    + \pddw \cdot ( \avg{\rho^0 \textbf{u}^0} \textbf{a}_p^{nst} P_{nst} )
    = \\
    \avg{\rho^0 \textbf{u}^0}  
    \avg{\delta(\textbf{x}_i -\textbf{y}) \delta(\textbf{u}_i - \textbf{w}) \pddt h_i }
    \\
\end{align*}


\subsection*{Dilute assumption}
\begin{equation*}
    \avg{\rho^0 \textbf{u}^0 \delta_{nst}(\textbf{y},\textbf{w},t,\FF)} 
    = 
    \avg{\chi_f\rho_f \textbf{u}^0_f(\textbf{x},t,\FF) \delta_{nst}(\textbf{y},\textbf{w},t,\FF)} 
    + \avg{\chi_d\rho_d \textbf{u}^0_d(\textbf{x},t,\FF) \delta_{nst}(\textbf{y},\textbf{w},t,\FF)} 
\end{equation*}
If $|\textbf{x}-\textbf{y}| > a$ the $\chi_d = 0$ and $\chi_f = 1$ since at \textbf{y} it is the nearest
\begin{equation*}
    \avg{\rho^0 \textbf{u}^0 \delta_{nst}(\textbf{y},\textbf{w},t,\FF)} 
    = 
    \phi_f \rho_f \textbf{u}_f^{nst} P_{nst|f}
    + \phi_d \rho_d \textbf{u}_d^{nst} P_{nst|d}
\end{equation*}
\begin{align*}
    \pddt (\rho_f P_{nst}) 
    + \pddx \cdot (\rho_f \textbf{u}_f^{nst} P_{nst})
    + \pddy \cdot (\textbf{w}   \rho_f P_{nst})
    + \pddw \cdot ( \rho_f \textbf{a}_p^{nst} P_{nst})
    = 
    \avg{\rho^0 \delta(\textbf{x}_i -\textbf{y}) \delta(\textbf{u}_i - \textbf{w}) \pddt h_i }
    \\
    \pddt ( \rho_f\textbf{u}_f^{nst} P_{nst})
    + \pddx \cdot ( \rho_f \textbf{u}_f^{nst}\textbf{u}_f^{nst}P_{nst} + \avg{\textbf{u}''\textbf{u}''\delta_{nst}} - \bm\sigma_f^{nst} P_{nst})\\
    + \pddy \cdot (\textbf{u}^{nst}_f \rho_f \textbf{w}   P_{nst} )
    + \pddw \cdot (\textbf{u}^{nst}_f \rho_f \textbf{a}_p^{nst} P_{nst} )
    = \avg{\rho^0 \textbf{u}^0  
    \delta(\textbf{x}_i -\textbf{y}) \delta(\textbf{u}_i - \textbf{w}) \pddt h_i }
    \\
\end{align*}

The difficultly is on the term 
\begin{equation*}
    \pddy \cdot \avg{\textbf{u}^0\rho^0 \textbf{w}   \delta_{nst} }
    = 
    \pddy \cdot \avg{\textbf{u}^0(\textbf{x})\rho^0(\textbf{x}) \textbf{w}   \sum_i \delta(\textbf{x}+ \textbf{r} - \textbf{x}_i)\delta(\textbf{w}- \textbf{u}_i) h_i(\textbf{x}) }
\end{equation*}
\tb{that s all wrong because we $\delta_{nst} = f(\textbf{x})$}


\begin{equation*}
    \frac{\partial }{\partial \textbf{y}}
    = 
    \frac{\partial }{\partial \textbf{x}}
    \frac{\partial (\textbf{y} - \textbf{r} )}{\partial \textbf{y}}
\end{equation*}


\section{Single-particle conditional average}

Before diving into the NPS strategy adopted in this work we present the original strategy that \citet{van1982bubble} used in his work. 

\subsection{Basic Definitions}
Let $\FF$ be a flow configuration and $d\PP = P(\FF)d\FF$ the probable number of flow in such a configuration. 
Equally, let $\textbf{u}_f^0(\textbf{x},t,\FF)$ be the local fluid velocity and $\chi_f(\textbf{x},t,\FF)$ the continuous phase indicator function evaluated at $\textbf{x}$ and time $t$ for a flow configuration $\FF$. 
The continuous phase averaged velocity is written as, 
\begin{equation}
    \phi_f \textbf{u}_f(\textbf{x},t)
    = \avg{\chi_f \textbf{u}_f^0} 
    = \int  \chi_f \textbf{u}_f^0(\textbf{x},t,\FF) d\PP
\end{equation}
where $\phi_f$ is the fluid phase volume fraction and $\avg{\ldots}$ denote an ensemble average operator. 
With these notations the \textit{Reynolds stress} tensor for the fluid phase can be written as 
\begin{equation}
    \bm\sigma^\text{Re}
    =\avg{\chi_f \textbf{u}_f'\textbf{u}_f'}
    =\avg{\chi_f \textbf{u}_f^0 \textbf{u}_f^0 }
    - \phi_f \textbf{u}_f \textbf{u}_f
\end{equation}

In order to be able to close this term one must perform conditional average. 
To introduce this we first notice the relation 
\begin{equation}
    \int_{\mathbb{R}^3}
    \sum_i^N \delta(\textbf{r}-\textbf{x}_i(t,\FF))
    d\textbf{r}
    = N 
\end{equation}
where $\textbf{x}_i$ is the Lagrangian position vector of the particle $i$. 
For purpose of generality let $\Lambda$ and $\Lambda_i(t,\FF)$ be a list of Eulerian and Lagrangian properties solely related to the particles among which the position vector is present. 
In this case we may write the more general expression, 
\begin{equation}
    \int
    \sum_i^N \delta(\bm\Lambda-\bm\Lambda_i(t,\FF))
    d\bm\Lambda
    = N 
\end{equation}
Based on this relation we can show that, 
\begin{equation}
    \phi_f \textbf{u}_f(\textbf{x},t)
    = 
    \int 
    \textbf{u}^1_f 
    P_{1f}(\textbf{x},t,\Lambda)
    d\Lambda
\end{equation}
where $\textbf{u}^1_f(\textbf{x},t,\bm\Lambda)$ is the averaged velocity of the continuous phase at \textbf{x} and $t$ conditionally on the single particle state $\bm\Lambda$. 

The \textit{Reynolds stress} tensor may be expressed in terms of conditionally average velocity fields, 
\begin{equation}
    \avg{\chi_f\textbf{u}_f^0 \textbf{u}_f^0}
    = 
    \int \textbf{u}_f^1\textbf{u}_f^1 P_{1f} d\bm\Lambda
    + 
    \int \avg{\delta_1 \chi_f\textbf{u}_f'' \textbf{u}_f''} d\bm\Lambda
\end{equation}
where we have defined $\textbf{u}_f''(\textbf{x},t,\FF) = \textbf{u}^0_f(\textbf{x},t,\FF) -\textbf{u}^1_f(\textbf{x},t,\bm\Lambda)$. 
Subtracting $\phi_f \textbf{u}_f\textbf{u}_f$ yields the formula for the \textit{Reynolds Stress}, namely 
\begin{equation}
    \avg{\chi_f\textbf{u}_f' \textbf{u}_f'}
    = 
    \int \textbf{v}_f^1\textbf{v}_f^1 P_{1f} d\bm\Lambda
    + 
    \int \avg{\delta_1 \chi_f\textbf{u}_f'' \textbf{u}_f''} d\bm\Lambda
    \label{eq:Re}
\end{equation}
To summarize we have introduced the following definition, 
\begin{align*}
    \text{fluctuation of the local value around the ensemble average : }\textbf{u}_f' = \textbf{u}_f^0 - \textbf{u}_f\\
    \text{fluctuation of the local value around the single particle conditional average : }\textbf{u}_f'' = \textbf{u}_f^0 - \textbf{u}_f^1\\
    \text{Fluctuation of the single particle conditional average around the ensemble average : }\textbf{v}_f^1 = \textbf{u}_f^1 - \textbf{u}_f
\end{align*}
Based on these definitions we state that we can separate the \textit{Reynolds stress} into two distinct contribution :  (1) the agitation generated due to the averaged wakes around the particles; (2) all other source of fluctuations such as those generated through particles interactions and the single phase turbulence. 

\subsection{Conditional averaged Navier-Stokes equation}

The first term of \ref{eq:Re} require the velocity field $\textbf{v}_f^1$. 
It is obtained by solving what called by \citet{hinch1977averaged} the single particle conditional averaged Navier-Stokes equation. 

In our notation the Navier-Stokes equations for an incompressible fluid reads, 
\begin{align}
    \div \textbf{u}_k^0  = 0 \\
    \pddt (\rho_k \textbf{u}_k^0 )
    + \div(\rho_k \textbf{u}_k^0  \textbf{u}_k^0 ) 
    = - \grad p_k^0 
    + \rho_k \textbf{g}
    + \mu_k \grad^2 \textbf{u}_k^0
\end{align}
where $k$ is either referring to the dispersed or continuous phase. 
From our previous work we showed that the two fluid formulation for the Navier-Stokes equation is rather 
\begin{align}
    \pddt (\chi_k \rho_k )
    + \div (
        \chi_k \rho_k  \textbf{u}_k^0
        )
        = 
        0 \\
        \pddt (\chi_k \rho_k \textbf{u}_k^0)
        + \div (
        \chi_k \rho_k \textbf{u}_k^0 \textbf{u}_k^0
        - \chi_k \bm\sigma_k^0 
        )
    = 
    % \chi_k \rho_k \textbf{g}
    + \delta_I
         \bm\sigma_k^0
    \cdot \textbf{n}_k\\
        \div (\delta_I \bm\sigma_{I||}^0 )
        = 
        - \sum_k
        \delta_I
        \bm\sigma_k^0
   \cdot \textbf{n}_k
\end{align}
To derive the single-particle conditionally averaged Navier-Stokes we must multiply these equations by $\delta^1$. 
For this purpose notice that, 
\begin{equation}
    \pddt \delta^1+\grad_\Lambda\cdot(\dot{\bm\Lambda}_i \delta^1 ) = 0 
\end{equation} 
Using this expression leads us to 
\begin{align}
    \pddt (\chi_k \delta_1 \rho_k )
    + \div (
        \chi_k  \delta_1 \rho_k  \textbf{u}_k^0
        )
    +  \grad_\Lambda \cdot (\chi_k \rho_k \dot{\bm\Lambda}_i \delta^1)
    = 
    0 \\
    \pddt (\chi_k \rho_k \delta_1 \textbf{u}_k^0)
    + \div (
        \chi_k\delta_1 \rho_k \textbf{u}_k^0 \textbf{u}_k^0
        - \chi_k\delta_1 \bm\sigma_k^0 
        )
    +  \grad_\Lambda \cdot ( \rho_k \textbf{u}_k^0 \chi_k \dot{\bm\Lambda}_i \delta^1)
    = 
    % \chi_k \rho_k \textbf{g}
    + \delta_I\delta_1
         \bm\sigma_k^0
    \cdot \textbf{n}_k\\
        \div (\delta_I \bm\sigma_{I||}^0 \delta_1 )
        = 
        - \sum_k\delta_1
        \delta_I
        \bm\sigma_k^0
   \cdot \textbf{n}_k
\end{align}

Now we apply the ensemble average operator, which gives independently 
\begin{align}
    \avg{\chi_k\delta_1 \rho_k \textbf{u}_k^0 \textbf{u}_k^0}
    = 
    \textbf{u}_k^1 \textbf{u}_k^1 P_{1f}
    + \avg{\chi_k\delta_1 \rho_k \textbf{u}_k'' \textbf{u}_k''}\\
    \avg{\rho_k \textbf{u}_k^0 \chi_k \dot{\bm\Lambda}_i \delta^1}
    = 
    \textbf{u}_k^1 \avg{\rho_k  \chi_k \dot{\bm\Lambda}_i \delta^1}
    + \avg{\rho_k \textbf{u}_k'' \chi_k \dot{\bm\Lambda}_i \delta^1}
\end{align}

To help underscoring note that if $\Lambda_i = (\textbf{x}_i,\textbf{u}_i)$ and $\bm\Lambda = (\textbf{x}+\textbf{r},\textbf{c})$ we would have, $\avg{\rho_k  \chi_k \textbf{u}_i \delta^1}$ explicitly this reads, 
\begin{equation}
    \avg{\rho_k  \chi_k \textbf{u}_i \delta^1}
    = \int 
    \rho_f \chi_k(\textbf{x},t,\FF)
    \sum_i^N \textbf{u}_i(\FF,t) 
    \delta(\textbf{x}+\textbf{r}-\textbf{x}_i)
    \delta(\textbf{c}-\textbf{u}_i)
    d\PP
    = 
    \rho_k \textbf{c}
    P_{1f}(\textbf{x},t,\textbf{r},\textbf{c})
\end{equation}
and also 
\begin{equation}
    \avg{\rho_k  \chi_k \textbf{a}_i \delta^1}
    = \int 
    \rho_f \chi_k(\textbf{x},t,\FF)
    \sum_i^N \textbf{a}_i(\FF,t) 
    \delta(\textbf{x}+\textbf{r}-\textbf{x}_i)
    \delta(\textbf{u}-\textbf{u}_i)
    d\PP
    = 
    \rho_k \textbf{a}_p^1(\textbf{x},t,\textbf{r},\textbf{c})
    P_{1f}(\textbf{x},t,\textbf{r},\textbf{c})
\end{equation}
This is the acceleration of the particles center of mass conditioned on the position \textbf{r} with velocity \textbf{u} with fluid at \textbf{x} at time $t$. 
Generally it can be state that, $\textbf{u}_k^1 \avg{\rho_k  \chi_k \dot{\bm\Lambda}_i \delta^1} = \rho_k \textbf{u}_k^1 \dot{\Lambda}_p^1 P_{1f}$

Neglecting this term the equations above yields, 
\begin{align}
    \pddt ( P_{1f} \rho_k )
    + \div (
        \rho_k  \textbf{u}_k^1 P_{1f}
        )
    +  \grad_\Lambda \cdot ( \rho_k \dot{\bm\Lambda}_p^1 P_{1f})
    = 
    0 \\
    \pddt (\rho_k \textbf{u}_k^1 P_{1f})
    + \div (
         \rho_k \textbf{u}_k^1 \textbf{u}_k^1 P_{1f}
        - \bm\sigma_k^{1-eq}
        )
    +  \grad_\Lambda \cdot ( \rho_k \textbf{u}_k^1 \dot{\bm\Lambda}_p^1 P_{1f} + \textbf{R})
    = 
    % \chi_k \rho_k \textbf{g}
    + \avg{\delta_I\delta_1
         \bm\sigma_k^0
    \cdot \textbf{n}_k}\\
        \div \avg{\delta_I \bm\sigma_{I||}^0 \delta_1 }
        = 
        - \sum_k\avg{\delta_1
        \delta_I
        \bm\sigma_k^0
   \cdot \textbf{n}_k}
\end{align}
with, 
\begin{align}
    \bm\sigma_k^{1-eq}
    =
    \avg{\chi_f \delta_1 \rho_k \textbf{u}''_k\textbf{u}''_k}
    - \bm\sigma_k^1 P_{1f}\\
    \textbf{R}
    = 
    \avg{\rho_k \textbf{u}_k'' \chi_k \dot{\bm\Lambda}_i \delta^1}
\end{align}
Thus, these equations still needs closure terms. 
At this point it is convenient to neglect all of them otherwise the problem is never solvable. 

We first notice that, 
\begin{multline}
    \pddt (\rho_k \textbf{u}_k^1 P_{1f})
    + \div (
         \rho_k \textbf{u}_k^1 \textbf{u}_k^1 P_{1f})
    = \rho_k P_{1f} \left[\pddt \textbf{u}_k^1 
        +\textbf{u}_k^1 \cdot \grad \textbf{u}_k^1 
    \right]
    + \textbf{u}_k^1 \left[
        \pddt (P_{1f}\rho_k)
        +\div(\rho_k \textbf{u}_k^1 P_{1f})
    \right]\\
    = \rho_k P_{1f} \left[\pddt \textbf{u}_k^1 
        +\textbf{u}_k^1 \cdot \grad \textbf{u}_k^1 
    \right]
    - \textbf{u}_k^1 
        \grad_\Lambda \cdot ( \rho_k \dot{\bm\Lambda}_p^1 P_{1f})
    \\
\end{multline}
It can also be express (and it might be better) : 
\begin{multline}
    \avg{\delta_1 \chi_f \bm\sigma_f^1}
    = 
    - p_f^1 P_{1f} \textbf{I}
    + \mu_f P_1 (\grad \textbf{u}^1 + ^\dagger \grad \textbf{u}^1)
    - \avg{\delta_1\chi_d \mu_f (\grad \textbf{u}_d^0 + ^\dagger \grad \textbf{u}_d^0)}\\
\end{multline}
The second term is the particle contribution and the first $\textbf{u}^1  =  \phi_f^1 \textbf{u}_f^1 + \phi_d^1 \textbf{u}_d^1 $ and $P_1 = \avg{\delta_1}$ is the probability of finding a particle in the state $\bm\Lambda$. 
Let re-write the momentum equation conserving the advective term on the left while gathering the stresses on  the right, 
\begin{multline}
    \rho_k P_{1f} \left[\pddt \textbf{u}_k^1 
        +\textbf{u}_k^1 \cdot \grad \textbf{u}_k^1 
    \right]
    +  \rho_k \dot{\bm\Lambda}_p^1 P_{1f} \cdot \grad_\Lambda \textbf{u}_k^1 
    =  - \grad (p_f^1 P_{1f})
    + \mu_f P_1 \grad^2 \textbf{u}_f^1
    \\
    - (\grad,\grad_\Lambda) \cdot (\avg{\chi_f \delta_1 \rho_k \textbf{u}''_k\textbf{u}''_k} + \avg{\delta_1 \chi_d \mu_f \textbf{e}_d^0},\textbf{R})
    + \avg{\delta_I\delta_1
         \bm\sigma_k^0
    \cdot \textbf{n}_k}\\
\end{multline}
This, equation is the full Conditioned Navier-Stokes equation. 
It must be completed with some boundary condition at infinity. 
Note that $P_{1f} = \phi_f^1(\textbf{x},t|\bm\Lambda) P_1(\bm\Lambda)$.
Namely, we have, 
\begin{align}
    \lim_{r \to \infty}  \textbf{u}_k^1(\textbf{x},t,\bm\Lambda)
    &= \textbf{u}_k(\textbf{x},t)
    \approx \textbf{u}_k(\textbf{x}+\textbf{r},t)
    + \textbf{r} \cdot \grad \textbf{u}_k(\textbf{x}+\textbf{r},t) 
    + \ldots\\
    \lim_{r \to \infty}  p_k^1(\textbf{x},t,\bm\Lambda)
    &= p_k(\textbf{x},t)
    \approx p_k(\textbf{x}+\textbf{r},t)
    + \textbf{r} \cdot \grad p_k(\textbf{x}+\textbf{r},t) 
    + \ldots\\
    \lim_{r \to \infty}  \phi_f^1(\textbf{x},t,\bm\Lambda)
    &= \phi_f^1(\textbf{x},t)
    \approx \phi_f(\textbf{x}+\textbf{r},t)
    + \textbf{r} \cdot \grad \phi_f(\textbf{x}+\textbf{r},t) 
    + \ldots\\
    \lim_{r \to \infty}  \bm\sigma_f^1(\textbf{x},t,\bm\Lambda)
    &= \bm\sigma_f^1(\textbf{x},t)
\end{align}
Based on this information we define, $\textbf{u}_f^{1d} = \textbf{u}_f^1 - \textbf{u}_f$, $p_f^{1d} = p_f^1 - p_f$ and $\phi_1^{1d} = \phi_f^1 - \phi_f$ as the disturbance fields caused by the particle. 
Instead of solving directly the CNSE we prefer to solve for the disturbance field. 
Thus, we substitute these formulas into the CANS which gives, 
% \begin{multline}
%     \rho_k P_{1f} \left[\pddt \textbf{v}_k^1 
%         +\textbf{v}_k^1 \cdot \grad \textbf{v}_k^1 
%     \right]
%     +  \rho_k \dot{\bm\Lambda}_p^1 P_{1f} \cdot \grad_\Lambda \textbf{v}_k^1 
%     =  - \grad (q_f^1 P_{1f})
%     + \mu_f P_1 \grad^2 \textbf{v}^1
%     \\
%     - \grad (p_f P_{1f})
%     + \mu_f P_1 \grad^2 \textbf{u}
%     - \rho_k P_{1f} \left[\pddt \textbf{u}_k
%         +\textbf{u}_k \cdot \grad \textbf{u}_k 
%         +\textbf{u}_k \cdot \grad \textbf{v}_k^1 
%         +\textbf{v}_k^1 \cdot \grad \textbf{u}_k 
%     \right]
%     +  \rho_k \dot{\bm\Lambda}_p^1 P_{1f} \cdot \grad_\Lambda \textbf{u}_k 
%     \\
%     - (\grad,\grad_\Lambda) \cdot (\avg{\chi_f \delta_1 \rho_k \textbf{u}''_k\textbf{u}''_k} + \avg{\delta_1 \chi_d \mu_f \textbf{e}_d^0},\textbf{R})
%     + \avg{\delta_I\delta_1
%          \bm\sigma_k^0
%     \cdot \textbf{n}_k}\\
% \end{multline}
\begin{equation}
    \pddt ( \phi_k^{1d} P_1  \rho_k )
    + \div (\rho_k  \textbf{u}_k^{1d} \phi_k^{1d} P_1 )
    +  \grad_\Lambda \cdot ( \rho_k \dot{\bm\Lambda}_p^1 \phi_k^{1d} P_1 )
    = 
    - \div (\rho_k  \textbf{u}_k \phi_k^{1d} P_1 )
    - \div (\rho_k  \textbf{u}_k^{1d} \phi_k P_1 )
    % - \div (\rho_k  \textbf{u}_k \phi_k P_1 )
    % - \pddt ( \phi_k  \rho_k P_1 )
    % -  \grad_\Lambda \cdot ( \rho_k \dot{\bm\Lambda}_p^1 \phi_k P_1)
\end{equation}
\begin{multline}
    \pddt (\rho_k \textbf{u}_k^{1d}  P_{1f})
    + \div (
         \rho_k \textbf{u}_k^{1d} \textbf{u}_k^{1d}  P_{1f}
        - \bm\sigma_k^{1d} P_{1f}
        )
    +  \grad_\Lambda \cdot ( \rho_k \textbf{u}_k^{1d} \dot{\bm\Lambda}_p^1  P_{1f} )\\
    = 
    - \pddt (\rho_k \textbf{u}_k  P_{1f})
    - \div (
        \rho_k \textbf{u}_k \textbf{u}_k  P_{1f}
        + \rho_k \textbf{u}_k \textbf{u}_k^{1d}  P_{1f}
        + \rho_k \textbf{u}_k^{1d} \textbf{u}_k  P_{1f}
        - \bm\sigma_k P_{1f}
        )
    +  \grad_\Lambda \cdot ( \rho_k \textbf{u}_k \dot{\bm\Lambda}_p^1  P_{1f} )\\
    - (\grad,\grad_\Lambda)\cdot(\avg{\chi_f \delta_1 \rho_k \textbf{u}''_k\textbf{u}''_k},\avg{\rho_k \textbf{u}_k'' \chi_k \dot{\bm\Lambda}_i \delta^1})
    % \chi_k \rho_k \textbf{g}
    + \avg{\delta_I\delta_1
         \bm\sigma_k^0
    \cdot \textbf{n}_k}\\ 
\end{multline}
Expanding the fluid phase volume fraction yields, 


To simplify this equation notice that the averaged velocity field $\textbf{u}_k$ satisfy, 
\begin{align}
    \pddt (\phi_k \rho_k)  
    + \div (
        \phi_k \rho_k\textbf{u}_k
    )
    = 
    0\\
    \pddt (\phi_k \rho_k\textbf{u}_k)  
    + \div (
        \phi_k \rho_k\textbf{u}_k\textbf{u}_k
        - \phi_f\bm\sigma_f
    )
    = 
    - \div\avg{\chi_f\textbf{u}_f' \textbf{u}_f'}
    + \avg{\delta_I \bm{\sigma}_k^0 \cdot \textbf{n}_k}
\end{align}
a conservation equation for $P_{1f}$ is just the one for $P_1$ and $\phi_1^{1d}$
The conservation equation for $P_1$ reads, 
\begin{equation}
    \pddt P_1 +\grad_\Lambda\cdot(\dot{\bm\Lambda}_p^1 P_1) = 0 
\end{equation} 
Therefore, 
\begin{align*}
    \pddt (\phi_k \rho_k P_1 )  
    + \div (\phi_k \rho_k\textbf{u}_k P_1)
    + \grad_\Lambda (\phi_k \rho_k\dot{\bm\Lambda}_p^1 P_1)
    = 
    0\\
    \pddt (P_1 \phi_k \rho_k\textbf{u}_k)  
    + \div (
        P_1 \phi_k \rho_k\textbf{u}_k\textbf{u}_k
        - P_1 \phi_f\bm\sigma_f
    )
    + \grad_\Lambda\cdot(\phi_k \rho_k \textbf{u}_k \dot{\bm\Lambda}_p^1 P_1)
    = 
    - P_1 \div\avg{\chi_f\textbf{u}_f' \textbf{u}_f'}
    + P_1 \avg{\delta_I \bm{\sigma}_k^0 \cdot \textbf{n}_k}
\end{align*}
Using this equation into teh CNSE yields, 
\begin{multline}
    \pddt (\rho_k \textbf{u}_k^{1d}  \phi_f^{1d} P_1)
    + \div (
     \rho_k \textbf{u}_k^{1d} \textbf{u}_k^{1d}  \phi_f^{1d} P_1
    - \bm\sigma_k^{1d} \phi_f^{1d} P_1
    )
    +  \grad_\Lambda \cdot ( \rho_k \textbf{u}_k^{1d} \dot{\bm\Lambda}_p^1  \phi_f^1 P_1 )\\
    = 
    - \pddt (\rho_k \textbf{u}_k^{1d}  \phi_f P_1)
    - \div (
    \rho_k \textbf{u}_k^{1d} \textbf{u}_k^{1d}  \phi_f P_1
    - \bm\sigma_k^{1d} \phi_f P_1
    )
    \\
    - \pddt (\rho_k \textbf{u}_k  \phi_f^{1d} P_1)
    - \div (
        \rho_k \textbf{u}_k \textbf{u}_k  \phi_f^{1d} P_1
        + \rho_k \textbf{u}_k \textbf{u}_k^{1d}  \phi_f^1 P_1
        + \rho_k \textbf{u}_k^{1d} \textbf{u}_k  \phi_f^1 P_1
        - \bm\sigma_k \phi_f^{1d} P_1
        )
    +  \grad_\Lambda \cdot ( \rho_k \textbf{u}_k \dot{\bm\Lambda}_p^1 \phi_f^{1d} P_{1} )\\
    - (\grad,\grad_\Lambda)\cdot(\avg{\chi_f \delta_1 \rho_k \textbf{u}''_k\textbf{u}''_k} 
    - P_1 \avg{\chi_f \textbf{u}_f'\textbf{u}_f'},\avg{\rho_k \textbf{u}_k'' \chi_k \dot{\bm\Lambda}_i \delta^1})
    % \chi_k \rho_k \textbf{g}
    + \avg{\delta_I\delta_1
         \bm\sigma_k^0
    \cdot \textbf{n}_k}\\ 
\end{multline}
which is way too complicated. 
The fluctuating term can be express to, 
\begin{align}
    \avg{\chi_f \delta_1 \rho_k \textbf{u}''_k\textbf{u}''_k}
    = 
    \avg{\chi_f \delta_1 \rho_k \textbf{u}_k^0 \textbf{u}_k^0 }
    - P_1 \phi_f^1 \textbf{u}_f^1\textbf{u}_f^1 \\
    P_1 \avg{\chi_f \rho_k \textbf{u}_k'\textbf{u}_k'}
    = 
    P_1 \avg{\chi_f \rho_k \textbf{u}_k^0 \textbf{u}_k^0 }
    - P_1 \phi_f \textbf{u}_f\textbf{u}_f 
\end{align}
therefore the previous tensor reads, 
\begin{equation}
    \avg{\chi_f \delta_1 \rho_k \textbf{u}_k^0 \textbf{u}_k^0 
    - P_1 \chi_f \rho_k \textbf{u}_k^0 \textbf{u}_k^0}
    - P_1 \phi_f^1 \textbf{u}_f^1\textbf{u}_f^1
    + P_1 \phi_f \textbf{u}_f\textbf{u}_f 
\end{equation}

Note on the conditioned volume fraction in the dilute regime :
\begin{itemize}
    \item $\phi_f^1$ is one outside the particle 
    \item This is inconsistent with the boundary condition $\lim_{r\to\infty} \phi_f^1 = \phi_f$. 
\end{itemize}
In the uniform regime and outside the particle $\phi_f^1 = \phi_f$ also, $\phi_f^{1d} =  0$. 
In this sense we can re-write the above eq. 

\subsection{Dilute approximation}

In the dilute approximation and at $|\textbf{r}| > a$ the CANS reads,

\begin{equation*}
    \pddt ( P_1  \rho_f )
    + \div (
        \rho_f  \textbf{u}_f^1 P_1 
        )
    +  \grad_\Lambda \cdot ( \rho_f \dot{\bm\Lambda}_p^1 P_1 )
    = 
    0 
\end{equation*}
\begin{multline}
    \pddt (\rho_f \textbf{u}_f^{1d}  P_1)
    + \div (
         \rho_f \textbf{u}_f^{1d} \textbf{u}_f^{1d}  P_1
        - \bm\sigma_f^{1d} P_1
        )
    +  \grad_\Lambda \cdot ( \rho_f \textbf{u}_f^{1d} \dot{\bm\Lambda}_p^1  P_1 )\\
    = 
    - \pddt (\rho_f \textbf{u}_f  P_1)
    - \div (
        \rho_f \textbf{u}_f \textbf{u}_f  P_1
        + \rho_f \textbf{u}_f \textbf{u}_f^{1d}  P_1
        + \rho_f \textbf{u}_f^{1d} \textbf{u}_f  P_1
        - \bm\sigma_f P_1
        )
\end{multline}
The far fields equation in the limit $\phi_f \sim 1$ is, 
\begin{align*}
    \pddt (\rho_f P_1 )  
    + \div (\rho_f\textbf{u}_f P_1)
    + \grad_\Lambda (\rho_f\dot{\bm\Lambda}_p^1 P_1)
    = 
    0\\
    \pddt (P_1 \rho_f\textbf{u}_f)  
    + \div (
        P_1 \rho_f\textbf{u}_f\textbf{u}_f
        - P_1 \bm\sigma_f
    )
    + \grad_\Lambda\cdot(\rho_f \textbf{u}_f \dot{\bm\Lambda}_p^1 P_1)
    = 0
    % - P_1 \div\avg{\chi_f\textbf{u}_f' \textbf{u}_f'}
    % + P_1 \avg{\delta_I \bm{\sigma}_f^0 \cdot \textbf{n}_f}
\end{align*}
where we have assumed that the fluctuating terms where proportional to $\phi^2$ at the lowest order. 
If $\avg{\chi_f \textbf{u}_f' \textbf{u}_f'} \sim \phi $ which is negligible in this case. 
This is in fact just the NS equations. 

Using this approximation in the previous equation yields, 
\begin{align*}
    \div (
        \rho_f  \textbf{u}_f^{1d} P_1 
        )
    = 
    0 \\
    \pddt (\rho_f \textbf{u}_f^{1d}  P_1)
    + \div (
         \rho_f \textbf{u}_f^{1d} \textbf{u}_f^{1d}  P_1
        - \bm\sigma_f^{1d} P_1
        )
    +  \grad_\Lambda \cdot ( \rho_f \textbf{u}_f^{1d} \dot{\bm\Lambda}_p^1  P_1 )\\
    = 
    - \div (
        % \rho_f \textbf{u}_f \textbf{u}_f  P_1
         \rho_f \textbf{u}_f \textbf{u}_f^{1d}  P_1
        + \rho_f \textbf{u}_f^{1d} \textbf{u}_f  P_1
        % - \bm\sigma_f P_1
        )
\end{align*}
Or in conservative form, 
\begin{align*}
    \rho_f \Dt \textbf{u}_f^{1d}  
    +  \rho_f  \dot{\bm\Lambda}_p^1  P_1   \cdot \grad_\Lambda \textbf{u}_f^{1d} 
    + \rho_f \textbf{u}_f  P_1 \grad \textbf{u}_f^{1d}
    +  \rho_f \textbf{u}_f^{1d} P_1 \grad\textbf{u}_f
    = 
    \div ( \bm\sigma_f^{1d} P_1)
\end{align*}
Now we can assume that $\bm\dot{\Gamma} =(\textbf{x}+ \textbf{r},\textbf{c})$ such that the momentum eq reduce to 
\begin{align*}
    \pddt (\rho_f \textbf{u}_f^{1d}  P_1)
    + \div (
         \rho_f \textbf{u}_f^{1d} \textbf{u}_f^{1d}  P_1
        - \bm\sigma_f^{1d} P_1
        )
    +  \grad_\textbf{r} \cdot ( \rho_f \textbf{u}_f^{1d} \textbf{c}  P_1(\textbf{r},\textbf{c}) )
    +  \grad_\textbf{c} \cdot ( \rho_f \textbf{u}_f^{1d} \textbf{a}_p^1  P_1 )\\
    = 
    - \div (
        % \rho_f \textbf{u}_f \textbf{u}_f  P_1
         \rho_f \textbf{u}_f \textbf{u}_f^{1d}  P_1
        + \rho_f \textbf{u}_f^{1d} \textbf{u}_f  P_1
        % - \bm\sigma_f P_1
        )
\end{align*}

\subsection{Dimensionless approximation}
Let $\textbf{u}^{1d}_f = \textbf{u}^{1d*}_f U$ and $\pddt = \pddt^* \tau$, $\grad = \grad^* l$ and $\bm\sigma^{1d}_f = \bm\sigma^{1d*}_f \mu_f U/l$ and $\textbf{u}_f = \textbf{u}_f^* U^\infty$.
$\textbf{c} = \textbf{c}U^\infty_p$ and $\textbf{a}_p^1 = U^\infty_p / \tau$. 
Let $\grad_\textbf{r} = \grad_\textbf{r}^* l$ 

\begin{align*}
    \frac{\rho_f l^2 }{\tau \mu_f}\pddt^*( \textbf{u}_f^{1d*}  P_1)
    + \frac{\rho_f U l}{\mu_f }\grad^* \cdot (\textbf{u}_f^{1d*} \textbf{u}_f^{1d*}  P_1) 
    -  \div (\bm\sigma_f^{1d*} P_1)
    + \frac{l \rho_f U^\infty_p}{\mu_f } \grad_\textbf{r}^* \cdot (  \textbf{u}_f^{1d*} \textbf{c}^*  P_1(\textbf{r},\textbf{c}) )
    +  \frac{\rho_f l^2}{\mu_f \tau U^\infty_p} \grad_\textbf{c}^* \cdot (  \textbf{u}_f^{1d*} \textbf{a}_p^{1*}  P_1 )\\
    = 
    -  \frac{U^\infty\rho_f l}{\mu_f } \div ( \textbf{u}_f \textbf{u}_f^{1d}  P_1
        +  \textbf{u}_f^{1d} \textbf{u}_f  P_1)
\end{align*}
Assuming that $l/\tau = U$ as it is often the case we have, 
\begin{align*}
    Re \pddt^*( \textbf{u}_f^{1d*}  P_1)
    + Re \grad^* \cdot (\textbf{u}_f^{1d*} \textbf{u}_f^{1d*}  P_1) 
    -  \div (\bm\sigma_f^{1d*} P_1)
    + Re_p^\infty \grad_\textbf{r}^* \cdot (  \textbf{u}_f^{1d*} \textbf{c}^*  P_1(\textbf{r},\textbf{c}) )
    +  \frac{\rho_f l^2}{\mu_f \tau U^\infty_p} \grad_\textbf{c}^* \cdot (  \textbf{u}_f^{1d*} \textbf{a}_p^{1*}  P_1 )\\
    = 
    - Re^\infty \div ( \textbf{u}_f \textbf{u}_f^{1d}  P_1
        +  \textbf{u}_f^{1d} \textbf{u}_f  P_1)
\end{align*}
It is hard to say if my advective terms vanish or not. 
we may have $\grad_\textbf{r}^* \cdot = - \div$ 
It must be checked, but in all case the advective term remain. 
It seems that we find back Stone eq in conservative form. 
In the conservative form we have 
\begin{align*}
    \rho_f \Dt  \textbf{u}_f^{1d*}  
    +  \rho_f  \textbf{c}  P_1   \cdot \pddr \textbf{u}_f^{1d} 
    +  \rho_f  \textbf{a}_p^1  P_1   \cdot \partial_\textbf{c} \textbf{u}_f^{1d} 
    + \rho_f \textbf{u}_f  P_1 \grad \textbf{u}_f^{1d}
    +  \rho_f \textbf{u}_f^{1d} P_1 \grad\textbf{u}_f
    = 
    \div ( \bm\sigma_f^{1d} P_1)
\end{align*}
In Dimensionless form 
\begin{align*}
    Re \Dt^*  \textbf{u}_f^{1d*}  
    + Re_p  \textbf{c}  P_1   \cdot \pddr \textbf{u}_f^{1d} 
    +  \frac{\rho_f U l^2 }{\mu_f \tau U_p}  \textbf{a}_p^1  P_1   \cdot \partial_\textbf{c} \textbf{u}_f^{1d} 
    + Re^\infty \textbf{u}_f  P_1 \grad \textbf{u}_f^{1d}
    + Re^\infty \textbf{u}_f^{1d} P_1 \grad\textbf{u}_f
    = 
    \div ( \bm\sigma_f^{1d} P_1)
\end{align*}
Due to the \textbf{x} \textbf{r} symmetry we have 
\begin{align*}
    Re \Dt^*  \textbf{u}_f^{1d*}  
    +  \frac{\rho_f U l^2 }{\mu_f \tau U_p}  \textbf{a}_p^1  P_1   \cdot \partial_\textbf{c} \textbf{u}_f^{1d} 
    + Re ( \textbf{u}_f -   \textbf{c}  ) P_1 \grad \textbf{u}_f^{1d}
    + Re^\infty \textbf{u}_f^{1d} P_1 \grad\textbf{u}_f
    = 
    \div ( \bm\sigma_f^{1d} P_1)
\end{align*}
The averaged velocity, 
$\textbf{u}_f(\textbf{x},t) = \textbf{u}_f(\textbf{x}+ \textbf{r},t) + \textbf{r}\cdot \grad \textbf{u}_f(\textbf{x}+ \textbf{r},t)+ \ldots$ upon the assumption of linear quadratic etc flows one gets the desired equations in terms of the interpolated values of $\textbf{u}_f$. 

\section{Nearest particle closure}


\section*{Inertial part of the stresslet }

\begin{equation*}
    \textbf{S}_p 
    =
    \phi \mu Re
    \frac{59}{80} \left[
        - 3\textbf{U}\textbf{U}
        + (\textbf{U}\cdot\textbf{U})\bm\delta
    \right] 
\end{equation*}



\bibliography{Bib/bib_bulles.bib}
\appendix

\section*{Computation of the derive of $h_i$}

The definitions with $r_i = |\textbf{x}_i - \textbf{x}|$.  
\begin{equation*}
    h_i(\textbf{x},t,\FF)
    = 
    \frac{1}{N(\textbf{x},t,\FF)}
    \prod_j
    H(r_j - r_i)
\end{equation*}
and 
\begin{equation*}
    N(\textbf{x},t\FF)
    = \sum_i \prod_j 
    H(r_j - r_i)
\end{equation*}

The Gradient of this function gives
\begin{equation*}
    \pddt  h_i(\textbf{x},t,\FF)
    = 
    -  \frac{\pddt N(\textbf{x},t,\FF)}{N(\textbf{x},t,\FF)^2}
    \prod_j
    H(r_j - r_i)
    + \frac{1}{N(\textbf{x},t,\FF)}
    \pddt \prod_j
    H(r_j - r_i)
\end{equation*}
\begin{equation*}
    \pddx  h_i(\textbf{x},t,\FF)
    = 
    -  \frac{\pddx N(\textbf{x},t,\FF)}{N(\textbf{x},t,\FF)^2}
    \prod_j
    H(r_j - r_i)
    + \frac{1}{N(\textbf{x},t,\FF)}
    \pddx \prod_j
    H(r_j - r_i)
\end{equation*}
with, 
\begin{equation*}
    \pddt 
    \prod_j
    H(r_j - r_i)
    = 
    \sum_k 
    \delta(r_k - r_i)
    (\textbf{u}_k  \cdot \hat{\textbf{r}}_k - \textbf{u}_i  \cdot \hat{\textbf{r}}_i)
    \prod_{j\neq k}
    H(r_j - r_i)
\end{equation*}
\begin{equation*}
    \pddx
    \prod_j
    H(r_j - r_i)
    = 
    \sum_k 
    \delta(r_k - r_i)
    ( \hat{\textbf{r}}_i -  \hat{\textbf{r}}_k)
    \prod_{j\neq k}
    H(r_j - r_i)
\end{equation*}

\begin{align*}
    \pddt r_k
    = \pddt [(\textbf{x}_k(\FF,t) - \textbf{x})\cdot (\textbf{x}_k(\FF,t) - \textbf{x})]^{1/2}\\
    = 
    2 \textbf{u}_k(\FF,t)  \cdot (\textbf{x}_k(\FF,t) - \textbf{x})
    \frac{1}{2}[(\textbf{x}_k(\FF,t) - \textbf{x})\cdot (\textbf{x}_k(\FF,t) - \textbf{x})]^{- 1/2}
    = 
    \textbf{u}_k  \cdot \hat{\textbf{r}}_k
\end{align*}
\begin{align*}
    \pddx r_k
    = \pddx [(\textbf{x}_k(\FF,t) - \textbf{x})\cdot (\textbf{x}_k(\FF,t) - \textbf{x})]^{1/2}
    = - \hat{\textbf{r}}_k
\end{align*}

The $N$ function derivative gives 
\begin{equation*}
    \pddt 
    N 
    = \pddt 
    \sum_i \prod_j
    H(r_j - r_i)
    = 
    \sum_i 
    \sum_k 
    \delta(r_k - r_i)
    (\textbf{u}_k  \cdot \hat{\textbf{r}}_k - \textbf{u}_i  \cdot \hat{\textbf{r}}_i)
    \prod_{j\neq k}
    H(r_j - r_i) = 0 
\end{equation*}

Thus, we have \citet{zhang2021ensemble} 
\begin{align*}
    \pddt  h_i(\textbf{x},t,\FF)
    = 
    \frac{1}{N(\textbf{x},t,\FF)}
    \sum_k 
    \delta(r_k - r_i)
    (\textbf{u}_k  \cdot \hat{\textbf{r}}_k - \textbf{u}_i  \cdot \hat{\textbf{r}}_i)
    \prod_{j\neq k}
    H(r_j - r_i)
    \\
    \textbf{u}_i \cdot \grad h_i(\textbf{x},t,\FF)
    = 
    \frac{1}{N(\textbf{x},t,\FF)}
    \sum_k 
    \delta(r_k - r_i)
    (\textbf{u}_i  \cdot \hat{\textbf{r}}_i -  \textbf{u}_i  \cdot\hat{\textbf{r}}_k)
    \prod_{j\neq k}
    H(r_j - r_i)
    \\
\end{align*}
Since $\delta(r_k - r_i)$ we have $r_k = r_i$ therefor $H(r_k - r_i) = H(0) = 1 $
\begin{align*}
    \pddt  h_i(\textbf{x},t,\FF)
    = 
    \sum_k 
    \delta(r_k - r_i)
    (\textbf{u}_k  \cdot \hat{\textbf{r}}_k - \textbf{u}_i  \cdot \hat{\textbf{r}}_i)
    h_i
    \\
    \textbf{u}^0 \cdot \grad h_i(\textbf{x},t,\FF)
    = 
    \sum_k 
    \delta(r_k - r_i)
    (\textbf{u}^0  \cdot \hat{\textbf{r}}_i -  \textbf{u}^0  \cdot\hat{\textbf{r}}_k)
    h_i
    \\
\end{align*}
Idea, maybe if i sum on i it cancel ? 
\begin{align*}
    \pddt  h_i(\textbf{x},t,\FF) +\textbf{u}^0 \cdot \grad h_i(\textbf{x},t,\FF)
    &= 
    \sum_k 
    \delta(r_k - r_i)
    (\textbf{u}_k  \cdot \hat{\textbf{r}}_k  - \textbf{u}_i  \cdot \hat{\textbf{r}}_k)
    h_i
    \\
    &= 
    \sum_k 
    \delta(r_k - r_i)
    [(\textbf{u}_k  - \textbf{u}^0) \cdot \hat{\textbf{r}}_k - (\textbf{u}_i - \textbf{u}^0)  \cdot \hat{\textbf{r}}_i]
    h_i
    \\
\end{align*}
This can be treated in a similar way that previously

ensemble average these equations gives 

\end{document}


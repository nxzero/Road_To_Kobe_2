\documentclass[12pt]{My_preprint}
\title{
    Theoretical calculation of the droplets-particles sedimentation speed or drag force.
    }

\author[1,2]{Nicolas Fintzi}
% \author[1]{Jean-Lou Pierson}
% \author[2]{Stephane Popinet}
\affil[1]{IFP Energies Nouvelles, Rond-point de l’echangeur de Solaize, 69360 Solaize}
\affil[2]{Sorbonne Universit\'e, Institut Jean le Rond d'Alembert, 4 place Jussieu, 75252 PARIS CEDEX 05, France}
\normalmarginpar


\begin{document}

\maketitle

\begin{abstract}
\end{abstract}


\section{Local scale equations}

\subsection{Local scale mixture equation}

\begin{align}
     \div \textbf{u}^0 &= 0, \\
    \rho_f(\pddt 
    + \textbf{u}^0 \cdot \grad) \textbf{u}^0
    &= 
    \div \bm\sigma_*^0
    +\rho_f \textbf{g}
    +\kappa(\bm\sigma_f^0\cdot \textbf{n})\delta_\Gamma,
\end{align}
with 
\begin{equation}
    \bm\sigma_*^0 = 
    \chi_f \bm\sigma_f^0
    + \zeta^{-1}(\chi_d \bm\sigma_d^0
    + \delta_\Gamma \bm\sigma_\Gamma^0)
\end{equation}
with $\kappa = (\zeta^{-1} - 1)$

\subsection{Dispersed phase equations}
\begin{align}
    \ddt \intO{\rho_d\textbf{u}_d^0}
    &= 
    \intS{\bm{\sigma}_f' \cdot \textbf{n}}
    + \intO{\div\bm\Sigma}
    + \rho_d \textbf{g}\intO{}\\
    \intS{ \bm{\sigma}_\Gamma^0}
    +\intO{ \bm{\sigma}_d'}
    &= 
    \intS{\textbf{r} \bm{\sigma}_f' \cdot \textbf{n}}
    + \intO{\textbf{r} \div\bm\Sigma}\\
    \intO{[\bm\sigma_d' \textbf{r} + (\bm\sigma_d' \textbf{r})^\dagger]}
    + \intS{[\bm\sigma_\Gamma^0 \textbf{r} + (\bm\sigma_\Gamma^0 \textbf{r})^\dagger]}
    &= 
    \intS{\textbf{rr}\bm{\sigma}_f'\cdot\textbf{n}}
    + \intO{\textbf{rr}(\div\bm\Sigma + \rho_d \textbf{g})} 
\end{align}
where we have introduced the `disturbance' stresses, 
\begin{equation}
    \bm{\sigma}_k'
    =
    \bm{\sigma}_k^0
    -
    \bm{\Sigma}
    =
    - (p_k^0 - p_f) \bm\delta
    + 2\mu_k [
        \grad(\textbf{u}_k^0 - \textbf{u})
        + ^\dagger\grad(\textbf{u}_k^0 - \textbf{u})
    ]
\end{equation}
Additionally we introduce the notation, 
\begin{align}
    \textbf{M}_\alpha^{(0)} &=
    \intS{\bm{\sigma}_f' \cdot \textbf{n}}
   +\intO{\div \bm\Sigma}
   \\
   \textbf{M}_\alpha^{(1)} &=
   \intS{\textbf{r}\bm{\sigma}_f' \cdot \textbf{n}}
   -2\mu_f \intO{\textbf{e}_d'}
   +\intO{\textbf{r}\div \bm\Sigma}
   \\
   \textbf{M}_\alpha^{(2)} &=
   \frac{1}{2}\intS{\textbf{rr}\bm{\sigma}_f' \cdot \textbf{n}}
   -2\mu_f \intO{\textbf{re}_d''}
   +\intO{\textbf{rr}(\div \bm\Sigma+ \rho_f\textbf{g})}
    \\
\end{align}
for reasons that will be clear latter note that it is $\rho_f$ that appear on the right-hand side of this eq. 

\section{Ensemble Averaged mixture equation}
\subsubsection{Two-fluid form}

\begin{align}
    \div \textbf{u} &= 0, \\
   \rho_f(\pddt 
   + \textbf{u} \cdot \grad) \textbf{u}
   + \avg{\textbf{u}'\textbf{u}'}
   &= 
   \div \bm\sigma_*
   +\rho_f \textbf{g}
   +\kappa\avg{\bm\sigma_f'\cdot \textbf{n}\delta_\Gamma}
   +\kappa\avg{\chi_d }\div \bm\Sigma
\end{align}
with 
\begin{equation}
   \bm\sigma_* = 
   \bm\Sigma
   + \zeta^{-1}\avg{\chi_d \bm\sigma_d'
   + \delta_\Gamma \bm\sigma_\Gamma^0}
   -  2\mu_f\avg{\chi_d  \textbf{e}_d'}
\end{equation}
\subsubsection{Hybrid form}
\begin{align}
    \div \textbf{u} &= 0, \\
   \rho_f(\pddt 
   + \textbf{u} \cdot \grad) \textbf{u}
   + \avg{\textbf{u}'\textbf{u}'}
   &= 
   \div \bm\sigma_*
   +\rho_f \textbf{g}
   +\kappa\pavg{\textbf{M}_\alpha^{(0)}}
\end{align}
with 
\begin{equation}
    \bm\sigma_* = 
    \bm\Sigma
    + \pavg{\textbf{M}_\alpha^{(1)}}
    - \div \pavg{\textbf{M}_\alpha^{(2)}}
    + \ldots
\end{equation}

\section{Conditional ensemble average equations}

Let, 
\begin{equation}
    \delta_\Pi[\textbf{X}(t,\FF),\FF,t]
\end{equation}
be a distribution that describe a microstate, such that $\textbf{x} \notin \textbf{X}$.And  $\textbf{X}$ can be for example a dirac pointing on the position of neighboring particles . 
Also, we consider, and it will be always true (if we do not consider NPS), 
\begin{equation}
    \pddt \delta_\Pi + \grad_X \dot{X}\delta_\Pi = 0
\end{equation}
Averaging this 
\begin{equation}
    \pddt P(\Pi) + \grad_X \avg{\dot{X}\delta_\Pi} = 0
\end{equation}


\subsubsection{Two-fluid form}
\begin{equation}
    P(\Pi) \div \textbf{u}^\Pi = 0,
\end{equation}
\begin{multline}
    P(\Pi) \rho_f(\pddt 
   + \textbf{u}^\Pi \cdot \grad) \textbf{u}^\Pi
   + \div \avg{\delta_\Pi \textbf{u}''\textbf{u}''}
   + \grad_X \avg{\dot{X} \delta_\Pi \textbf{u}^0}\\
   = 
   \div \bm\sigma_*^\Pi
   +P(\Pi)\rho_f \textbf{g}
   +\kappa\avg{\delta_\Pi \delta_\Gamma\bm\sigma_f''\cdot \textbf{n}}
   +\kappa\avg{\delta_\Pi \chi_d \div \bm\Sigma^\Pi}
\end{multline}
with 
\begin{equation}
   \bm\sigma_*^\Pi = 
   P(\Pi)\bm\Sigma^\Pi
   + \zeta^{-1}\avg{\delta_\Pi(\chi_d \bm\sigma_d''
   + \delta_\Gamma \bm\sigma_\Gamma^0)}
   -  2\mu_f\avg{\delta_\Pi \chi_d   \textbf{e}_d'}
\end{equation}
\subsubsection{Hybrid form}
\begin{align}
    P(\Pi) \div \textbf{u}^\Pi &= 0, \\
    P(\Pi) \rho_f(\pddt 
   + \textbf{u}^\Pi \cdot \grad) \textbf{u}^\Pi
   + \div \avg{\delta_\Pi \textbf{u}''\textbf{u}''}
   + \grad_X \avg{\dot{X} \delta_\Pi \textbf{u}^0}
   &= 
   \div \bm\sigma_*
   +\rho_f \textbf{g}
   +\kappa\pavg{\delta_\Pi\textbf{M}_\alpha^{\Pi(0)}}
\end{align}
with 
\begin{equation}
    \bm\sigma_*^0 = 
    \bm\Sigma^\Pi
    + \pavg{\delta_\Pi\textbf{M}_\alpha^{\Pi(1)}}
    - \div \pavg{\delta_\Pi\textbf{M}_\alpha^{\Pi(2)}}
    + \ldots
\end{equation}
with,
\begin{align}
    \textbf{M}_\alpha^{\Pi(0)} &=
    \intS{\bm{\sigma}_f'' \cdot \textbf{n}}
   +\intO{\div \bm\Sigma^\Pi}
   \\
   \textbf{M}_\alpha^{\Pi(1)} &=
   \intS{\textbf{r}\bm{\sigma}_f'' \cdot \textbf{n}}
   -2\mu_f \intO{\textbf{e}_d''}
   +\intO{\textbf{r}\div \bm\Sigma^\Pi}
   \\
   \textbf{M}_\alpha^{\Pi(2)} &=
   \frac{1}{2}\intS{\textbf{rr}\bm{\sigma}_f'' \cdot \textbf{n}}
   -2\mu_f \intO{\textbf{re}_d''}
   +\intO{\textbf{rr}(\div \bm\Sigma^\Pi+ \rho_f\textbf{g})}
    \\
\end{align}

\section{Disturbance field equations}


Substracting the ensemble avg eq multiplied by $P(\Pi)$ to the Conditional eq one directly obtain, 
Introducing the distribution $\delta_{\Pi'} = \delta_\Pi - P(\Pi)$

\subsubsection*{two-fluidform}
\begin{equation}
    P(\Pi) \div \textbf{u}^{\Pi'} = 0,
\end{equation}
\begin{multline}
    P(\Pi) \rho_f\pddt \textbf{u}_\Pi'
    + \div \avg{\delta_{\Pi'} \textbf{u}^0\textbf{u}^0}
    + \grad_X \avg{\dot{X} (\delta_{\Pi} \textbf{u}^0 - P(\Pi) \textbf{u})}\\
   = 
   \div \bm\sigma_*^{\Pi'}
   +\kappa\avg{\delta_\Gamma(\delta_\Pi \bm\sigma_f'' - P(\Pi) \bm\sigma_f')\cdot \textbf{n}}
   +\kappa\avg{ \chi_d (\delta_\Pi\div \bm\Sigma^\Pi- P(\Pi) \div\bm\Sigma)}
\end{multline}
\begin{equation}
    \bm\sigma_*^{\Pi'} = 
    P(\Pi)\bm\Sigma^{\Pi'}
    + \zeta^{-1}\avg{\chi_d (\delta_\Pi\bm\sigma_d'' - P(\Pi) \bm\sigma_d')
    + \delta_{\Pi'}\delta_\Gamma \bm\sigma_\Gamma^0}
    -  2\mu_f\avg{ \chi_d   (\delta_\Pi\textbf{e}_d'' -P(\Pi)\textbf{e}_d') }
 \end{equation}

\subsubsection*{Hybrid form}
\begin{align}
    P(\Pi) \div \textbf{u}^{\Pi'} = 0, \\
    P(\Pi) \rho_f\pddt \textbf{u}_\Pi'
   + \div \avg{\delta_{\Pi'} \textbf{u}^0\textbf{u}^0}
   + \grad_X \avg{\dot{X} (\delta_{\Pi} \textbf{u}^0 - P(\Pi) \textbf{u})}\\
    = 
   \div \bm\sigma_*^{\Pi'}
   +\kappa\pavg{(
    \delta_\Pi\textbf{M}_\alpha^{\Pi(0)}
    - P(\Pi)\textbf{M}_\alpha^{(0)}
   )}
\end{align}
with 
\begin{equation}
    \bm\sigma_*^{\Pi'} = 
    P(\Pi)\bm\Sigma^{\Pi'}
    + \pavg{(
        \delta_\Pi \textbf{M}_\alpha^{\Pi(1)}
        - P(\Pi) \textbf{M}_\alpha^{(1)}
    )}
    - \div \pavg{
        (\delta_\Pi\textbf{M}_\alpha^{\Pi(2)}
        - P(\Pi)\textbf{M}_\alpha^{(2)})
        }
    + \ldots
\end{equation}


\section{Computing the first order hydrodynamics}

Because the closure termes $\textbf{M}^{(n)}$ require the disturbance fields $\pavg{\bm\sigma_f'}= \bm\Sigma^{1'}$ we must solve for the Conditional averaged equation setting, 
\begin{equation}
    \delta_\Pi = \delta_1 = \sum_{i=1}^{N}\delta(\textbf{y} - \textbf{x}_i) 
\end{equation}
Note that at the leading order $\mathcal{O}(n_p)$ we have 
\begin{align}
    \delta_1 \chi_d
    =
    \sum_{i,j} H(a- |\textbf{x} - \textbf{x}_i|)\delta(\textbf{y} - \textbf{x}_i) 
    =
    H(a- |\textbf{x} - \textbf{y}|)
    \sum_i \delta(\textbf{y} - \textbf{x}_i)+ \mathcal{O}( n_p^2)\\ 
    \delta_1 \delta_\Gamma
    =
    \sum_{i,j} \delta(a- |\textbf{x} - \textbf{x}_i|)\delta(\textbf{y} - \textbf{x}_i) 
    =
    \delta(a- |\textbf{x} - \textbf{y}|)
    \sum_i \delta(\textbf{y} - \textbf{x}_i)+\mathcal{O}( n_p^2)\\ 
\end{align}
Hence by definition, 
\begin{align}
    \avg{\delta_\Gamma \delta_1 \bm\sigma_f''\cdot \textbf{n}}
    &=
    \delta(a- |\textbf{x} - \textbf{y}|) \avg{\delta_1 (\bm\sigma_f^0 - \bm\Sigma^1)\cdot \textbf{n}}
    + \mathcal{O}( n_p^2)
    = 
    \mathcal{O}( n_p^2)\\
    P(\textbf{y})\avg{\delta_\Gamma  \bm\sigma_f'}
    &=
    \mathcal{O}( n_p^2)\\
    \avg{\chi_d \delta_1 \div\bm\Sigma^1}
    &=
    H(a- |\textbf{x}-\textbf{y}|) P(\textbf{y}) \div\bm\Sigma^{1}
    + 
    \mathcal{O}( n_p^2)\\
    P(\textbf{y})\avg{\chi_d \div\bm\Sigma}
    &=
    \mathcal{O}( n_p^2)\\
    \avg{\delta_{1'} \delta_\Gamma \bm\sigma_\Gamma^0}
    &=
    \avg{\delta_1 \delta_\Gamma \bm\sigma_\Gamma^0}
    - \avg{P_1 \delta_\Gamma \bm\sigma_\Gamma^0}
    =
    \delta(a -|\textbf{x} - \textbf{y}|)P(\textbf{y})\bm\sigma_\Gamma^1
\end{align}
So basically it corresponds to the equation for an isolated buoyant droplet since out of the particle there is no source terms . 
Hence, the closure can be computed it reads, 


\begin{align}
    \pavg{\textbf{M}^{(0)}} 
    &=
    \phi
    \frac{\mu_f}{a^2}
    \frac{3(2+3\lambda)}{2(1+\lambda)}\textbf{u}_r
    + \phi\mu_f  \frac{3\lambda}{4(\lambda +1)} \grad^2 \textbf{u}
    + \pOavg{\div\bm\Sigma}
    \\
    \pavg{\textbf{M}^{(1)}} 
    &= \mu_f \phi 
    \frac{(5\lambda +2)}{(\lambda +1)}\textbf{E} 
    + \phi a^2 \mu_f \frac{\lambda}{2(\lambda +1)}\grad^2 \textbf{E}
    + \pOavg{\textbf{r}\div\bm\Sigma}
    \\
    \pavg{\textbf{M}^{(2)}} 
    &=
    - \mu_f \phi \frac{3\lambda}{4(\lambda +1)}[\bm\delta \textbf{u}_r + \frac{1}{2\lambda}\textbf{u}_r \bm\delta ]
    + \propto \grad\grad \textbf{u}
    +\pOavg{\textbf{rr}(\div\bm\Sigma+\rho_f \textbf{g})}
\end{align}


Because of the present symmetry the divergence of the last term might be given by,
\begin{equation}
    - \mu_f \frac{3\lambda}{4(\lambda +1)}[\grad (\phi \textbf{u}_r)+\grad (\phi\textbf{u}_r)] + \frac{3\lambda-2}{4(\lambda+1)}\div (\phi\textbf{u}_r) \bm\delta 
\end{equation} 



Thus, we end up with, 
\begin{align}
    \div\textbf{u}&=0, \\
    \rho_f (\pddt 
    + \textbf{u}\cdot \grad)
    \textbf{u}
    % +\div \avg{\textbf{u}'\textbf{u}'}
    - \div \bm\Sigma
    &= 
    +\div \bm{\sigma}_\text{eff} 
    % + \kappa \pOavg{\div\bm\Sigma} 
    + \rho_f \textbf{g} 
    + \kappa \phi \frac{\mu_f}{a^2} \frac{3(2+3\lambda)}{2(1+\lambda)}\textbf{u}_r
    + \kappa \phi\mu_f  \frac{3\lambda}{4(\lambda +1)} \grad^2 \textbf{u}
    + \kappa \phi \div\Sigma
\end{align}
\begin{align*}
    \bm\sigma_\text{eff}  = 
    - \avg{ \textbf{u}'\textbf{u}'}
    + \mu_f \phi  \frac{(5\lambda +2)}{(\lambda +1)}\textbf{E} 
    + \mu_f \frac{3\lambda}{4(\lambda +1)}[\grad (\phi \textbf{u}_r)+\grad (\phi\textbf{u}_r)] 
    - \frac{3\lambda-2}{4(\lambda+1)}\div (\phi\textbf{u}_r) \bm\delta 
\end{align*}

\section{Second order hydrodynamic}

At this order the closure terms to obtain become quite complicated/non-negligible. 
Let us note, 
\begin{align}
    \pavg{(\delta_1\textbf{M}_\alpha^{1(n)} - P(\textbf{y})\textbf{M}_\alpha^{(n)})}
    &=
    P(\textbf{y})[P(\textbf{x}|\textbf{y})\textbf{M}_{p}^{1(n)}(\textbf{x},\textbf{y})
    - P(\textbf{x})\textbf{M}_{p}^{(n)}(\textbf{x})]
\end{align}
With this notation we may note the equation (without inertia) and in homogeneous scenario as,
\begin{align}
     \div \textbf{u}^{1'} = 0, \\
    -\div \bm\Sigma^{1'}
    &= 
   n_p\kappa
   [g(\textbf{x},\textbf{y})\textbf{M}_{p}^{1(0)}(\textbf{x},\textbf{y})
    - \textbf{M}_{p}^{(0)}(\textbf{x})]\\
   &+n_p\div [g(\textbf{x},\textbf{y})\textbf{M}_{p}^{1(1)}(\textbf{x},\textbf{y})
    - \textbf{M}_{p}^{(1)}(\textbf{x})]\\
   &-n_p\grad\grad: [g(\textbf{x},\textbf{y})\textbf{M}_{p}^{1(2)}(\textbf{x},\textbf{y})
    - \textbf{M}_{p}^{(2)}(\textbf{x})] = \textbf{f}(\textbf{x},\textbf{y})\\
\end{align}
Note that because, $\textbf{f}(\textbf{x},\textbf{y})= \int_{\mathbb{R}^3} \textbf{f}(\textbf{x}',\textbf{y})\delta(\textbf{x}'-\textbf{x})d\textbf{x}' $ we may write,
\begin{align}
    -\div \bm\Sigma^{1'}
    &= 
   n_p\kappa
   \int_{\mathbb{R}^3}
   [g(\textbf{x}'-\textbf{y})\textbf{M}_{p}^{1(0)}(\textbf{x}',\textbf{y})
    - \textbf{M}_{p}^{(0)}(\textbf{x}')]
    \delta(\textbf{x}' - \textbf{x}) 
    d\textbf{x}'\\
   &+ 
   n_p\int_{\mathbb{R}^3}
   [g(\textbf{x}'-\textbf{y})\textbf{M}_{p}^{1(1)}(\textbf{x}',\textbf{y})
    - \textbf{M}_{p}^{(1)}(\textbf{x}')]
    \cdot \grad\delta(\textbf{x}' - \textbf{x}) 
    d\textbf{x}'\\
   &-
   n_p\int_{\mathbb{R}^3}
   [g(\textbf{x}'-\textbf{y})\textbf{M}_{p}^{1(2)}(\textbf{x}',\textbf{y})
    - \textbf{M}_{p}^{(2)}(\textbf{x}')]
    : \grad\grad\delta(\textbf{x}' - \textbf{x}) 
    d\textbf{x}'
\end{align}
these are all termes of $\mathcal{O}(n_p)$ or even more that can be neglect in the previous assumption but not here.

Inverting the int gives, 
\begin{align*}
    -\textbf{u}^{1'} &=
    n_p\kappa
    \int_{\mathbb{R}^3}
    [g(\textbf{x}'-\textbf{y})\textbf{M}_{p}^{1(0)}(\textbf{x}',\textbf{y})
     - \textbf{M}_{p}^{(0)}(\textbf{x}')]
      \mathcal{G}(\textbf{x}' - \textbf{x}) 
     d\textbf{x}'\\
    &+ 
    n_p\int_{\mathbb{R}^3}
    [g(\textbf{x}'-\textbf{y})\textbf{M}_{p}^{1(1)}(\textbf{x}',\textbf{y})
     - \textbf{M}_{p}^{(1)}(\textbf{x}')]
     \cdot \grad \mathcal{G}(\textbf{x}' - \textbf{x}) 
     d\textbf{x}'\\
    &-
    n_p\int_{\mathbb{R}^3}
    [g(\textbf{x}'-\textbf{y})\textbf{M}_{p}^{1(2)}(\textbf{x}',\textbf{y})
     - \textbf{M}_{p}^{(2)}(\textbf{x}')]
     : \grad\grad \mathcal{G}(\textbf{x}' - \textbf{x}) 
     d\textbf{x}'
\end{align*}

Applying the reciprocal theorem, between $(\bm\Sigma^{1'},\textbf{u}^{1'})$ accurate at $\mathcal{O}(\phi)$ and $(\bm\Sigma^{1'},\textbf{u}^{1'})$ at $\mathcal{O}(\phi^0)$, we may find out that the contribution from the higher order terms in $\phi$ might simply be expressed as, 
\begin{equation}
    \intS{\hat{\textbf{u}}\cdot \bm\Sigma^{1'}\cdot \textbf{n}}
    =
    \intS{\hat{\textbf{u}}^{1'}\cdot \hat{\bm\Sigma}\cdot \textbf{n}}
    - \intO[out]{\textbf{f}' \cdot \hat{\textbf{u}}} 
\end{equation}
Because the test problem can simply be described by, 
\begin{equation}
    \hat{\textbf{u}}
    =
    \textbf{U}\cdot \mathcal{U}^U(\textbf{r})
    +\bm\Gamma : \mathcal{U}^E(\textbf{r})
\end{equation}
at the surf of the particle $\hat{\textbf{u}} = \textbf{U} + \bm\Gamma \cdot \textbf{r}$,
\begin{align}
    \mathcal{U}^U(\textbf{r})
    &=
    \frac{3r^{-1}}{4}\left(
        \bm\delta
        + \textbf{nn}
    \right)
    +\frac{r^{-3}}{4}
    (\bm\delta - 3 \textbf{nn})\\
    \mathcal{U}^E(\textbf{r})
    &=
    r^{-4}(\bm\delta \textbf{n} - \frac{5}{2} \textbf{nnn})
    + \frac{5}{2} r^{-2} \textbf{nnn}\\
    \hat{\bm\Sigma}\cdot \textbf{n}
    &=
    \frac{-3}{2}\textbf{U}
    -3\bm\Gamma\cdot\textbf{n}
\end{align}
Hence, the Stresslet and drag force may be obtained using the simple formula, 
\begin{align}
    \intS{\bm\Sigma^{1'}\cdot \textbf{n}}
    =
    -\frac{3}{2}\intS{\hat{\textbf{u}}^{1'}}
    - \intO[out]{\textbf{f}' \cdot\mathcal{U}^U} \\
    \intS{\textbf{r} \bm\Sigma^{1'}\cdot \textbf{n}}
    =
    - 3\intS{\hat{\textbf{u}}^{1'}\textbf{n}}
    - \intO[out]{\textbf{f}' \cdot \mathcal{U}^E} \\
\end{align}


\subsection{sedimentation of force free particles}
Using the momentum balence of the particles we have, 
\begin{equation}
    \kappa P(\textbf{x})[g(\textbf{x}|\textbf{y})\textbf{M}_{p}^{1(0)}(\textbf{x},\textbf{y}) -\textbf{M}_{p}^{(0)}(\textbf{x})]
    =
    - \kappa \rho_d v \textbf{g} P(\textbf{x}) [g(\textbf{x}|\textbf{y}) - 1]
\end{equation}
where $\kappa\rho_d = \rho_f-\rho_d$ is in fact the buoyant force of the particle phase. 
\subsubsection*{contribution to the force}
This contribution of this term to the drag force might be found using the reciprocal the as, 
\begin{align}
    \textbf{f}_b 
    &= 
    (\rho_f -\rho_d) v \textbf{g}  
    \intO[out]{  P(\textbf{x}) [g(\textbf{x}|\textbf{y}) - 1] \cdot \mathcal{U}^U(\textbf{x}-\textbf{y})}\\
    &=
    (\rho_f -\rho_d) v \textbf{g}  P(\textbf{x})\cdot
    \int_{1 < r< 2}{  [3r^{-1}/4  (\bm\delta + \textbf{nn}) + r^{-3}/3 (\bm\delta - 3\textbf{nn})]  }d\textbf{x}
\end{align}
where we have assumed $g(\textbf{x}|\textbf{y}) = H(|\textbf{x}-\textbf{y}|-2a)$, meaning we consider Brownian motions. 
Because the second term of the integral vanish and because, 
\begin{equation}
    \int_{1<r<2} 
    r^{-1}(\bm\delta + \textbf{nn}) d\textbf{x}
    =
    \int_{1<r<2} 
    r dr 
    \int_S (\bm\delta + \textbf{nn}) dS
    =
    (2^2 - 1)/2 (4\pi + 4\pi/3)
    =
    8\bm\delta
\end{equation}
Hence, 
\begin{equation}
    \textbf{f}_b 
    =
    6 (\rho_f -\rho_d) v \textbf{g}  P(\textbf{x}) 
    =
    6 (\rho_f -\rho_d)  \textbf{g}  \phi
\end{equation}
Because the drag on a sedimen unit sphere can be written,
\begin{equation}
    \frac{2}{9\mu_f}(\rho_f - \rho_d)\textbf{g}
\end{equation}
\subsubsection*{contribution to the velocity field}
\begin{align}
    \textbf{u}^{1'}(\textbf{y}|\textbf{y}) 
    &= 
    (\rho_f -\rho_d) v \textbf{g}  
    \int_{r<2a}{  P(\textbf{x}') [g(\textbf{x}'|\textbf{y}) - 1] \cdot \mathcal{G}(\textbf{x}'-\textbf{y})}d\textbf{x}'\\
    &= 
    (\rho_f -\rho_d) v P(\textbf{y}) \textbf{g}  
    \frac{8\pi}{3}
\end{align}
\subsection*{contribution to the first moment }
\begin{equation}
    g(\textbf{x},\textbf{y})\textbf{M}_{p}^{1(1)}(\textbf{x},\textbf{y}) - \textbf{M}_{p}^{(1)}(\textbf{x})
    =
    g(\textbf{x},\textbf{y})\textbf{M}_{p}^{1'(1)}(\textbf{x},\textbf{y}) 
    + [g(\textbf{x},\textbf{y})-1]\textbf{M}_{p}^{(1)}(\textbf{x})
\end{equation}
According to faxèn law for the Stresslet it reads, 
\begin{align}
    \textbf{M}_{p}^{1(1)}(\textbf{x},\textbf{y})
    - \textbf{M}_{p}^{(1)}(\textbf{x})
    &=
    \frac{20}{6}\mu\pi a (1 + \frac{a^2}{10}\grad^2)(\grad \textbf{u}^{1'}  + ^\dagger\grad \textbf{u}^{1'})
    + \frac{a^2 v}{5} \grad\grad\cdot \bm\Sigma^{1'}\\
    &=
    \frac{20}{6}\mu\pi a^3 \textbf{U}\cdot(1 + \frac{a^2}{10}\grad^2)(\grad \mathcal{U}  + ^\dagger\grad \mathcal{U})
    +\mathcal{O}(\phi)
\end{align}
The disturbance velocity field is determined by the buoyancy force $\textbf{F} = (\rho_f - \rho_d)v \textbf{g} $ that have the reference particle or through its yet unkwon mean velocity disturbance field hence ; 
\begin{equation}
    \textbf{u}^{1'} (\textbf{x}|\textbf{y})
    =
    \textbf{U}\cdot \mathcal{U}(\textbf{r})
    + 
    \mathcal{O}(\phi)
\end{equation}
therefor the term to compute becomes, if one assume $g(\textbf{x},\textbf{y}) =  H(|\textbf{x}- \textbf{y}|-2a)$ 
\begin{multline}
    \div 
    \{H(|\textbf{x}-\textbf{y}|-2a)\textbf{M}_{p}^{1'(1)}(\textbf{x},\textbf{y}) 
    + [H(|\textbf{x}-\textbf{y}|-2a)-1]\textbf{M}_{p}^{(1)}(\textbf{x})\}\\
    =
    \delta_\Gamma \textbf{n}\cdot \textbf{M}_{p}^{1'(1)}(\textbf{x},\textbf{y})
    + \delta_\Gamma \textbf{n}\cdot \textbf{M}_{p}^{(1)}(\textbf{x})\\
    + H(|\textbf{x}- \textbf{y}|-2a ) \div \textbf{M}_{p}^{1'(1)}(\textbf{x},\textbf{y}) 
    + [H(|\textbf{x}- \textbf{y}|-2a )-1] \div \textbf{M}_{p}^{(1)}(\textbf{x},\textbf{y}) 
\end{multline}
where we recall that, 
\begin{equation}
    \div\textbf{M}_p^{1'}
    =
    \frac{20}{6}\mu\pi a^3 \textbf{U}\cdot (1+\frac{a^2}{10}\grad^2)(\grad^2 \mathcal{U})
    =
    \frac{20}{6}\mu\pi a^3 \textbf{U}\cdot (\grad^2 \mathcal{U})
\end{equation}
because the velocity is biharmonic and divergence free. 
Additionally we can assume that $\div \textbf{M}_{p}^{(1)} = 0,\textbf{M}^{(1)}_p =0$ because of the absence of background shearing motion at the leading order. 
\begin{align}
    \intO[out]{\textbf{f}\cdot \mathcal{U}}
    &=
    P(\textbf{x})\int_{r=2a}{(\textbf{n} \cdot \textbf{M}_p^{1'})\cdot \mathcal{U}}d\textbf{x}
    + P(\textbf{x})\frac{20}{6}\pi\mu a \textbf{U}\cdot \intO[out]{\grad^2\mathcal{U}\cdot \mathcal{U}}\\
    &
    = -P(\textbf{x})\frac{439}{256}\pi\textbf{U}
    + 
    \left( \frac{7057 \pi^{2}}{129024}  - \frac{92983 \pi^{2}}{4032}\right)\textbf{U}
    =
\end{align}

\subsection*{Neutrally buoyant particle in shear}
In this case the first expr vanish while the  second and third closure need to be provided. 
The closure must be calculated at $\mathcal{O}(\phi)$ because we already divided by $\phi$, hence being accurate at $\phi^2$ as desired. 
Note before that that we may use the expr, 
\begin{equation}
    g(\textbf{x},\textbf{y})\textbf{M}_{p}^{1(1)}(\textbf{x},\textbf{y}) - \textbf{M}_{p}^{(1)}(\textbf{x})
    =
    g(\textbf{x},\textbf{y})\textbf{M}_{p}^{1'(1)}(\textbf{x},\textbf{y}) 
    + [g(\textbf{x},\textbf{y})-1]\textbf{M}_{p}^{(1)}(\textbf{x})
\end{equation}
According to faxèn law for the Stresslet it reads, 
\begin{equation}
    \textbf{M}_{p}^{1(1)}(\textbf{x},\textbf{y})
    - \textbf{M}_{p}^{(1)}(\textbf{x})
    =
    \frac{20}{6}\mu\pi a (1 + \frac{a^2}{10}\grad^2)(\grad \textbf{u}^{1'}  + ^\dagger\grad \textbf{u}^{1'})
    + \frac{a^2 v}{5} \grad\grad\cdot \bm\Sigma^{1'}
\end{equation}
Because \textbf{M} must be accurate at $\mathcal{O}(\phi)$ only the expr of $\textbf{u}'$ at order $\phi$ is sufficient hence we could use 
\begin{equation}
    \textbf{u}'(\textbf{x},\textbf{y}) = \textbf{M}^{(1)}(\textbf{y})_p: \grad \mathcal{G}(\textbf{x}-\textbf{y}) + \mathcal{O}(\phi)
\end{equation}
which leads to,
\begin{align}
    \textbf{M}_{p}^{1(1)}(\textbf{x},\textbf{y})
    - \textbf{M}_{p}^{(1)}(\textbf{x})
    &\approx
    \textbf{M}^{(1)}(\textbf{y})_p:
    \frac{20}{6}\mu\pi a (1 + \frac{a^2}{10}\grad^2)(\grad \grad \mathcal{G}(\textbf{x}-\textbf{y})  + ^\dagger\grad \grad \mathcal{G}(\textbf{x}-\textbf{y}))
    + \frac{a^2 v}{5} \grad\grad\cdot \bm\Sigma^{1'}\\
    &\approx
    \frac{20}{6}\mu\pi a 
    \textbf{M}^{(1)}(\textbf{y})_p:\grad \grad \mathcal{G}(\textbf{x}-\textbf{y})
\end{align}

\begin{align*}
    \mathcal{G}_{ij} &= \delta_{ij} r^{-1} + x_ix_j r^{-3}\\
    \mathcal{G}_{ij,k} &= 
    (- \delta_{ij}x_k + \delta_{ik} x_j+ \delta_{jk} x_i) r^{-3}  
    - 3 x_ix_jx_k r^{-5}\\
    \mathcal{G}_{ij,kl} &= 
    (- \delta_{ij}\delta_{kl} + \delta_{ik} \delta_{jl}+ \delta_{jk} \delta_{il}) r^{-3}  
    -3 (- \delta_{ij}x_kx_l + \delta_{ik} x_jx_l+ \delta_{jk} x_ix_l) r^{-5}  
    - 3 x_ix_jx_k r^{-5}\\
\end{align*}

Let now consider that $g(\textbf{x},\textbf{y}) = H(|\textbf{x} - \textbf{y}| - 2a)$ such that the thing to integrate in the faxen law becomes, 
\begin{align}
    \textbf{f}'(\textbf{x},\textbf{y})
    &=
    \div [ H(|\textbf{x} - \textbf{y}| - 2a)\textbf{M}_{p}^{1'(1)}(\textbf{x},\textbf{y}) 
    + (H(|\textbf{x} - \textbf{y}|-2a)-1)\textbf{M}_{p}^{(1)}(\textbf{x})]\\
    &=
    \delta(|\textbf{x} - \textbf{y}| - 2a)\textbf{n}\cdot 
    \textbf{M}_{p}^{1(1)}(\textbf{x},\textbf{y}) 
    +  H(|\textbf{x} - \textbf{y}| - 2a) 
    \div \textbf{M}_{p}^{1'(1)}(\textbf{x},\textbf{y}) \\
    &+(H(|\textbf{x} - \textbf{y}|-2a)-1) \div \textbf{M}_{p}^{(1)}(\textbf{x}) 
\end{align}

Let consider a particle in pure extension flow for the test problem,
The velocity field may be written, $\hat{\textbf{u}} = (1 + a^2/10\grad^2)\grad\mathcal{G}(\textbf{y}-\textbf{x})$
\begin{equation}
    \intS{\textbf{r}\cdot \bm\Sigma^{1'}\cdot \textbf{n}}
    =
    \intS{\hat{\textbf{u}}^{1'}\cdot \hat{\bm\Sigma}\cdot \textbf{n}}
    + \intO[out]{\textbf{f}' \cdot \hat{\textbf{u}}} 
\end{equation}




\subsection*{homogeneous chaos }
Assuming an homogeneous medium such that $P(\textbf{y}) = P(\textbf{x}) = n_p$ and molecular chaos, i.e. $P(\textbf{x}|\textbf{y}) =P(\textbf{x}) H(|\textbf{x}-\textbf{y}|-2a)$ we arrive at,
\begin{align}
    \pavg{(\delta_1\textbf{M}_\alpha^{1(n)} - P(\textbf{y})\textbf{M}_\alpha^{(n)})}
    &=
    n_p^2 [H(|\textbf{x}-\textbf{y}|-2a) \textbf{M}_{p}^{1(n)}(\textbf{x},\textbf{y})
    -\textbf{M}_{p}^{(n)}(\textbf{x})]\\
    &=
    n_p^2 [H(|\textbf{x}-\textbf{y}|-2a) (\textbf{M}_{p}^{1(n)} - \textbf{M}_{p}^{(n)})
    -H(2a - |\textbf{x}-\textbf{y}|) \textbf{M}_{p}^{(n)}(\textbf{x})]
\end{align}

The conditioned eq becomes, 
\begin{align}
    P(\textbf{y}) \div \textbf{u}^{1'} = 0, \\
    - P(\textbf{y}) \div \bm\Sigma^{1'}
    = 
    \div\bm\sigma_*^{1'}
    +
    \kappa n_p^2 [H_{|\textbf{x}-\textbf{y}|-2a} (\textbf{M}_{p}^{1(0)} - \textbf{M}_{p}^{(0)})
    -H_{2a - |\textbf{x}-\textbf{y}|} \textbf{M}_{p}^{(n)}(\textbf{x})]
\end{align}
with 
\begin{align}
    \bm\sigma_*^{1'} &= 
    n_p^2 [H(|\textbf{x}-\textbf{y}|-2a) (\textbf{M}_{p}^{1(1)} - \textbf{M}_{p}^{(1)})
    -H(2a - |\textbf{x}-\textbf{y}|) \textbf{M}_{p}^{(1)}(\textbf{x})]\\
    &-\div[
        n_p^2 [H(|\textbf{x}-\textbf{y}|-2a) (\textbf{M}_{p}^{1(2)} - \textbf{M}_{p}^{(2)})
    -H(2a - |\textbf{x}-\textbf{y}|) \textbf{M}_{p}^{(2)}(\textbf{x})]
    ]
\end{align}

\subsection*{Solid spherical particles }
An other approch is, 
\begin{align*}
    \textbf{M}^{(0)}_p
    =
    6\pi\mu a(1 + \frac{a^2}{6}\grad^2)(\textbf{u})
    + v_p \div \Sigma\\
    \textbf{M}^{1(0)}_p
    =
    6\pi\mu a(1 + \frac{a^2}{6}\grad^2)(\textbf{u}^1)
    + v_p \div \Sigma^1
\end{align*}
Hence, 
\begin{equation}
    \textbf{M}^{1(0)}_p
    -\textbf{M}^{(0)}_p
    =
    6\pi\mu a(1 + \frac{a^2}{6}\grad^2)(\textbf{u}^{1'})
    + v_p \div \bm\Sigma^{1'}
\end{equation}
At $\mathcal{\phi}$ $\textbf{u}^{1'} \approx \textbf{M}^{1(0)}_p\cdot (1+\frac{a^2}{6}\grad^2)\mathcal{G}(\textbf{x}-\textbf{y})$. 
Nevertheless it is more smart to keep thatat 0. 



From now on we note $\textbf{M}'$ the disturbance forces and moment so that the following concerne just the disturbance part .
The droplet in $\textbf{y}$ produce a force given by $\textbf{M}_p^{(0)}(\textbf{x})$
At the zeorth order in volume fraction the velocity field generated by this particle is thus given by, 
\begin{equation}
    \textbf{v}_1(\textbf{z}-\textbf{y})
    =
    \textbf{M}_p^{'(0)}(\textbf{y})\cdot [1+ \frac{a^2}{6}\grad^2_z]\mathcal{G}(\textbf{z}-\textbf{y})
\end{equation}
The total drag force on the test sphere is given at the leading order by,
\begin{equation}
    \textbf{M}_p^{'(0)}(\textbf{y})
    =
    6\pi\mu a (\textbf{u}_r + \frac{a^2}{6}\grad^2\textbf{u})|_{\textbf{z}=\textbf{y}}
    % + 
    % v_p \div\bm\Sigma
\end{equation}

At the first order in accuracy, according to Faxen law, the disturbance force on the particle in  \textbf{x} due to the droplet in \textbf{y} is therefor given by, 
\begin{align}
    \textbf{F}^{(1)}(\textbf{x},\textbf{y}) 
    &=
    6\mu \pi a (1 + \frac{a^2}{6}\grad^2_z ) \textbf{v}_1|_{\textbf{z}=\textbf{x}}\\
    &=
    6\mu \pi a 
    \textbf{M}_p^{(0)}(\textbf{y})\cdot
    (1 + \frac{a^2}{6}\grad^2 )
    [1+ \frac{a^2}{6}\grad^2]\mathcal{G}(\textbf{x}-\textbf{y})\\
    &=
    6\mu \pi a 
    \textbf{M}_p^{(0)}(\textbf{y})\cdot
    (1 + \frac{a^2}{3}\grad^2 )\mathcal{G}(\textbf{x}-\textbf{y})
\end{align}
\begin{align}
    \textbf{S}^{(1)}(\textbf{x},\textbf{y}) 
    &=
    \frac{20}{3} \mu \pi a^3 (1 + \frac{a^2}{10}\grad^2_z ) \textbf{e}_1|_{\textbf{z}=\textbf{x}}\\
    &=
    \frac{20}{3} \mu \pi a^3
    \textbf{M}_p^{(0)}(\textbf{y})\cdot
    (1 + \frac{a^2}{6}\grad^2 )
    [1+ \frac{a^2}{10}\grad^2]\grad\mathcal{G}^\text{sym}(\textbf{x}-\textbf{y})\\
    &=
    \frac{20}{3} \mu \pi a^3
    \textbf{M}_p^{(0)}(\textbf{y})\cdot
    (1 + \frac{4a^2}{15}\grad^2 )\grad\mathcal{G}^\text{sym}(\textbf{x}-\textbf{y})
\end{align}
The total force on the particle in \textbf{x}, is given by it first order force assumption + the mean force + the reflexion component which yields; 
\begin{equation}
    \textbf{M}_p^{1(0)}(\textbf{x},\textbf{y})
    =
    6\pi\mu a (\textbf{u}_r + \frac{a^2}{6}\grad^2\textbf{u})(\textbf{x})
    + 
    6\mu \pi a 
    \textbf{M}_p^{(0)}(\textbf{y})\cdot
    (1 + \frac{a^2}{3}\grad^2 )\mathcal{G}(\textbf{x}-\textbf{y})
    + v_p \div\bm\Sigma^1
\end{equation} 
To which we need to remove the total drag from the particle at \textbf{x} unconditioned, namely 
\begin{equation}
    \textbf{M}_p^{'(0)}(\textbf{x})
    =
    6\pi\mu a (\textbf{u}_r + \frac{a^2}{6}\grad^2\textbf{u})|_{\textbf{z}=\textbf{x}}
    + 
    v_p \div\bm\Sigma
\end{equation}
Assuming the particle velocity of the conditioned state is the same than the mean 
\begin{equation}
    \textbf{M}_p^{1(0)} - \textbf{M}_p^{(0)}
    =
    6\mu \pi a 
    \textbf{M}_p^{(0)}(\textbf{y})\cdot
    (1 + \frac{a^2}{3}\grad^2 )\mathcal{G}(\textbf{x}-\textbf{y})
    + v_p \div\bm\Sigma^{1'}
\end{equation}

Because there is no mean shear, for now, 
\begin{equation}
    \textbf{M}_p^{(1)}(\textbf{x})
    =
    0
\end{equation}
Assuming the particle velocity of the conditioned state is the same than the mean 
\begin{equation}
    \textbf{M}_p^{1(1)} - \textbf{M}_p^{(1)}
    =
    \frac{20}{3} \mu \pi a^3
    \textbf{M}_p^{(0)}(\textbf{y})\cdot
    (1 + \frac{4a^2}{15}\grad^2 )\grad\mathcal{G}^\text{sym}(\textbf{x}-\textbf{y})
\end{equation}


% 
\section{Mixture momentum equaiton}

Averaging the above equations gives directly, 
\paragraph{Dispersed phase equations}
\begin{align}
    % m_b (\pddt + \textbf{u}_b \cdot \grad )n_b &= - n_b \div \textbf{u}_b\\
    m_p (\pddt + \textbf{u}_p \cdot \grad )n_p &= - n_p \div \textbf{u}_p\\
    n_p m_p (\pddt + \textbf{u}_p\cdot  \grad) \textbf{u}_p
    + \div \pavg{ m_p \textbf{u}_p' \textbf{u}_p'}
    &= 
    m_p n_p \textbf{g}
    + \pSavg{\bm\sigma_f^0 \cdot \textbf{n}_p}
\end{align}
For the higher moments balance we may neglect inertia as the relative velocity is negligible. 
Additionally we introduce the notation,

\paragraph{Momentum conservation of the mixture}

Local
\begin{align}
     \div \textbf{u}^0 &= 0, \\
    \rho_f(\pddt 
    + \textbf{u}^0 \cdot \grad) \textbf{u}^0
    &= 
    \div \bm\sigma_*^0
    +\rho_f \textbf{g}
    +\kappa(\bm\sigma_f^0\cdot \textbf{n})\delta_\Gamma,
\end{align}
with 
\begin{equation}
    \bm\sigma_*^0 = 
    \chi_f \bm\sigma_f^0
    + \zeta^{-1}(\chi_d \bm\sigma_d^0
    + \delta_\Gamma \bm\sigma_\Gamma^0)
\end{equation}

\begin{align}
    \label{eq:NS_not_dispersed_mass}
    \div\textbf{u}&=0, \\
    \rho_f (\pddt 
    + \textbf{u}\cdot \grad)
    \textbf{u}
    +\div\avg{\textbf{u}'\textbf{u}'}
    &= 
    (1 +\kappa\phi)\div \bm\Sigma
    +\div  \bm{\sigma}_\text{eff} 
    + \kappa \avg{\delta_\Gamma \bm{\sigma}_f' \cdot \textbf{n}} 
    + \rho_f \textbf{g} 
    \label{eq:NS_not_dispersed}
\end{align}
with $\kappa = (1/\zeta - 1)$. 
with the effective stress, 
\begin{equation}
    \bm\sigma_\text{eff}  = 
    \zeta^{-1} \avg{\chi_d  \bm\sigma_d' + \chi_\Gamma  \bm\sigma_\Gamma^0}
    -  2\mu_f\avg{\chi_d  \textbf{e}_d'}
\end{equation}
\paragraph{Momentum conservation of the mixture: hybrid}
\begin{align}
    \label{eq:NS_not_dispersed_mass}
    \div\textbf{u}&=0, \\
    \rho_f (\pddt 
    + \textbf{u}\cdot \grad)
    \textbf{u}
    +\div\avg{\textbf{u}'\textbf{u}'}
    &= 
    (1+\kappa \phi)\div \bm\Sigma
    +\div \bm{\sigma}_\text{eff} 
    + \kappa \pSavg{ \bm{\sigma}_f' \cdot \textbf{n}} 
    + \rho_f \textbf{g} 
\end{align}
with $\kappa = (1/\zeta - 1)$. 
with the effective stress, 
\begin{multline}
    \bm\sigma_\text{eff}  = 
    + \pSavg{\textbf{r}\bm\sigma_f' \cdot \textbf{n}}
    -2\mu_f  \pOavg{ \textbf{e}_d'}
    + \zeta^{-1} \pOavg{\textbf{r}\div \bm\Sigma}
    \\
    - \div \left[
        \frac{1}{2}
        \pSavg{\textbf{rr}\bm\sigma_f' \cdot \textbf{n}}
        -2\mu_f \pOavg{\textbf{r}\textbf{e}_d'}
        + \frac{1}{2}\zeta^{-1}\pOavg{\textbf{rr}(\div\bm\Sigma + \rho_d \textbf{g})} 
    \right] 
\end{multline}


Let us define,
\begin{align}
    n_p f^1 =\avg{\delta_1 f^0}
    && 
    f'' = f^0 -  f^1  
    && 
    \avg{\delta_1 f''} =\avg{\delta_1 f^0} -  \avg{\delta_1} f^1 = 0  \\
    n_p f^{1'}
    &=
    n_p (f^1 - f)
    &=
    \avg{(\delta_1 f^0 - n_p  f^0)}
    &=
    \avg{f^0(\delta_1 - n_p )}
    =
    \avg{f^0\delta_{1'}}
\end{align}
The single particle average gives directly,
\begin{equation}
    \pddt \delta_1  +  \div_r \textbf{w} \delta_1 = 0
 \end{equation}

\section{Conditional average}
 \paragraph*{Conditional eq:}
 \begin{align*}
    n_p \div \textbf{u}^1 &= 0, \\
   n_p \rho_f(\pddt 
   + \textbf{u}^1 \cdot \grad) \textbf{u}^1
   + \grad_r \avg{\delta_1 \textbf{u}^0 \textbf{w}}
   + \div \avg{\delta_1 \textbf{u}'' \textbf{u}''}
   &= 
   \div \bm\sigma_*^1
   +n_p \rho_f \textbf{g}
   +\kappa\avg{\delta_1\delta_\Gamma  \bm\sigma_f''\cdot \textbf{n}}
   +\kappa \phi^1 n_p\div \bm\Sigma^1,
\end{align*}
\begin{equation}
    \bm\sigma_*^1 = 
    n_p \bm\Sigma^1
    % + \kappa \avg{\chi_d \delta_1 \bm\Sigma^1}
    + \zeta^{-1}\avg{\delta_1(\chi_d \bm\sigma_d'' + \delta_\Gamma \bm\sigma_\Gamma^0)}
    - \avg{\delta_1 \chi_d \textbf{e}_d''}
\end{equation}
\paragraph*{Averaged equaiton times $n_p$}
\begin{align}
    n_p \div\textbf{u}&=0, \\
    n_p \rho_f (\pddt 
    + \textbf{u}\cdot \grad)
    \textbf{u}
    +\div_r\avg{n_p \textbf{u}^0\textbf{w}}
    +\div\avg{n_p \textbf{u}'\textbf{u}'}
    &= 
    \div \bm{\sigma}_*
    + \rho_f \textbf{g} 
    + \kappa \avg{\delta_\Gamma \bm{\sigma}_f' \cdot \textbf{n}} 
    + \kappa \phi n_p \div\bm\Sigma
\end{align}
\begin{equation}
    \bm\sigma_* = 
    n_p \bm\Sigma
    % + \kappa \avg{\chi_d \delta_1 \bm\Sigma^1}
    + n_p \zeta^{-1}\avg{\chi_d \bm\sigma_d' + \delta_\Gamma \bm\sigma_\Gamma^0}
    - n_p \avg{ \chi_d \textbf{e}_d'}
\end{equation}
\paragraph*{Disturbance fields equations}
\begin{equation}
    n_p \div \textbf{u}^{1'} = 0
\end{equation}
\begin{multline*}
   n_p \rho_f(\pddt \textbf{u}^{1'}
   + \textbf{u}^{1'} \cdot \grad \textbf{u}^{1'}
   + \textbf{u} \cdot \grad \textbf{u}^{1'}
   + \textbf{u}^{1'} \cdot \grad \textbf{u})
   +\textbf{w}\cdot  \grad_r n_p \textbf{u}^{1'}
   + \div \avg{\delta_1 \textbf{u}'' \textbf{u}''}
   - \div \avg{n_p \textbf{u}' \textbf{u}'}
   \\
   = 
   \div \bm\sigma_*^{1'}
   +\kappa\avg{\delta_\Gamma  (\delta_1\bm\sigma_f'' - n_p \bm\sigma_f')\cdot \textbf{n}}
   +\kappa \avg{\chi_d \div(\delta_1 \bm\Sigma^1 - n_p \bm\Sigma)},
\end{multline*}
\begin{equation}
    \bm\sigma_*^{1'} = 
    n_p \bm\Sigma^{1'}
    % + \kappa \avg{\chi_d \delta_1 \bm\Sigma^1}
    + \zeta^{-1}\avg{\chi_d (\delta_1\bm\sigma_d'' - n_p \bm\sigma_d') + (\delta_1-n_p)\delta_\Gamma \bm\sigma_\Gamma^0}
    - \avg{\chi_d (\delta_1 \textbf{e}_d''- n_p \textbf{e}_d')}
\end{equation}


\section{Zero-order hydrodynamic $\mathcal{O}( \phi^0)$}

This represents basically a single phase flow Newtonian turbulent flow, 
\begin{align}
    \div\textbf{u}&=0, \\
    \rho_f (\pddt 
    + \textbf{u}\cdot \grad)
    \textbf{u}
    &= 
    \div( \bm\Sigma
    + \avg{ \textbf{u}'\textbf{u}'} )
    + \rho_f \textbf{g} 
\end{align}


\section{First-order hydrodynamic $\mathcal{O}( \phi^1)$}

The first order hydrodynamic is given by solving the single particle conditioned eq at $\mathcal{O}(\phi)$, outside the particle it gives, 
\begin{equation}
    n_p \div \textbf{u}^{1'} = 0
\end{equation}
\begin{equation*}
    - n_p\div \bm\Sigma^{1'}
   = 
   0
\end{equation*}

\begin{align}
    \pSavg{\bm\sigma_f'\cdot \textbf{n}} 
    &=
    \phi
    \frac{\mu_f}{a^2}
    \frac{3(2+3\lambda)}{2(1+\lambda)}\textbf{u}_r
    + \phi\mu_f  \frac{3\lambda}{4(\lambda +1)} \grad^2 \textbf{u}
    \label{eq:drag_forces}
    \\
    \pSavg{\textbf{r}\bm\sigma_f'\cdot \textbf{n}} - \pOavg{2\mu_f\textbf{e}_d'} 
    &= \mu_f \phi 
    \frac{(5\lambda +2)}{(\lambda +1)}\textbf{E} 
    + \mathcal{O}(a^2/L^2)
    % + \phi a^2 \mu_f \frac{\lambda}{2(\lambda +1)}\grad^2 \textbf{E}
    \\
    \frac{1}{2}\pSavg{\textbf{rr}\bm\sigma_f'\cdot \textbf{n}} 
    - \pSavg{ 2\mu_f\textbf{r}\textbf{e}'_d} 
    &=
    - \mu_f \phi \frac{3\lambda}{4(\lambda +1)}[\bm\delta \textbf{u}_r + \frac{1}{2\lambda}\textbf{u}_r \bm\delta ]
    + \mathcal{O}(\phi a^2/L^2)
    % \\
    % \pOavg{\div\bm\Sigma} &=
    % \phi\div\bm\Sigma
    % + \mathcal{O}(\phi a^2/L^2)
    \\
    \pOavg{\textbf{r}\div\bm\Sigma} &=
    \mathcal{O}(\phi a^2/L^2)\\
    \pOavg{\textbf{rr}(\div\bm\Sigma+\rho_f \textbf{g})}&=
    \mathcal{O}(\phi a^2/L^2)
    \label{eq:second_moment_surf}
\end{align}

Because of the present symmetry the divergence of the last term might be given by,
\begin{equation}
    - \mu_f \frac{3\lambda}{4(\lambda +1)}[\grad (\phi \textbf{u}_r)+\grad (\phi\textbf{u}_r)] + \frac{3\lambda-2}{4(\lambda+1)}\div (\phi\textbf{u}_r) \bm\delta 
\end{equation} 



Thus, we end up with, 
\begin{align}
    \div\textbf{u}&=0, \\
    \rho_f (\pddt 
    + \textbf{u}\cdot \grad)
    \textbf{u}
    % +\div \avg{\textbf{u}'\textbf{u}'}
    &= 
    (1+\kappa \phi) \div \bm\Sigma
    +\div \bm{\sigma}_\text{eff} 
    % + \kappa \pOavg{\div\bm\Sigma} 
    + \rho_f \textbf{g} 
    +\phi \frac{\mu_f}{a^2} \frac{3(2+3\lambda)}{2(1+\lambda)}\textbf{u}_r
    + \phi\mu_f  \frac{3\lambda}{4(\lambda +1)} \grad^2 \textbf{u}
\end{align}
\begin{align*}
    \bm\sigma_\text{eff}  = 
    - \avg{ \textbf{u}'\textbf{u}'}
    + \mu_f \phi  \frac{(5\lambda +2)}{(\lambda +1)}\textbf{E} 
    + \mu_f \frac{3\lambda}{4(\lambda +1)}[\grad (\phi \textbf{u}_r)+\grad (\phi\textbf{u}_r)] 
    - \frac{3\lambda-2}{4(\lambda+1)}\div (\phi\textbf{u}_r) \bm\delta 
\end{align*}
\section{Second-order hydrodynamic $\mathcal{O}( \phi^2)$}

For the second order hydrodynamic problem one must solve the single particle conditioned problem but at $\phi^2$ instead.

The  one fluid equation at order phi and for force free particle s
\begin{align*}
    n_p \div \textbf{u}^{1'} &= 0\\
    -n_p\div\bm\Sigma^{1'} 
    &= 
   \div \bm\sigma_*^{1'}
   - (\rho_d-\rho_f)\zeta v_p\textbf{g}  \pavg{\delta_{1'}}
\end{align*}
\begin{multline}
    \bm\sigma_*^{1'}  = 
    + \pSavg{\textbf{r}(\delta_1\bm\sigma_f'' - n_p \bm\sigma_f') \cdot \textbf{n}}
    -2\mu_f  \pOavg{ (\delta_1 \textbf{e}_d''- n_p \textbf{e}_d')}\\
    +  \pOavg{\textbf{r}\div (\delta_1 \bm\Sigma^1-n_p \bm\Sigma)}
    \\
    - \div \left[
        \frac{1}{2}
        \pSavg{\textbf{rr}(\delta_1\bm\sigma_f'' - n_p \bm\sigma_f') \cdot \textbf{n}}
        -2\mu_f \pOavg{\textbf{r}(\delta_1 \textbf{e}_d''- n_p \textbf{e}_d')} \right.\\ \left.
        + \frac{1}{2}\pOavg{\textbf{rr}(\div(\delta_1 \bm\Sigma^1-n_p \bm\Sigma) + \delta_{1'} \rho_f \textbf{g})} 
    \right] 
\end{multline}
with 
\begin{equation}
    \textbf{u}^{1'}\cdot \textbf{n} =
     (\textbf{w} - \textbf{u})\cdot \textbf{n}
\end{equation}

Thus, one has to find these closure terms accurate at $\phi^2$.
Multiplying the `mean' or unconditioned closure above by $n_p$ one get directly these closure accurate at $\phi^2$. 
Because 
\begin{equation}
    \delta_1
    =
    \sum_i \delta(\textbf{y} - \textbf{x}_i)
\end{equation}
We have 
\begin{align}
    \pavg{\delta_1}
    =
    \avg{\sum_{i,j} \delta(\textbf{x}-\textbf{x}_i)\delta(\textbf{y}-\textbf{x}_j)}
    % &=
    % \delta(\textbf{x}-\textbf{y})\avg{\sum_{i} \delta(\textbf{x}-\textbf{x}_i)}
    % +\avg{\sum_{i,j\neq i} \delta(\textbf{x}-\textbf{x}_i)\delta(\textbf{x}+\textbf{r}-\textbf{x}_j)}\\
    =
    \delta(\textbf{x}-\textbf{y})n_p(\textbf{x})
    +n_p(\textbf{x},\textbf{y})\\
\end{align}
Hence, 
\begin{equation}
    \pavg{\delta_{1'}}
    =
    \delta(\textbf{x} - \textbf{y}) n_p(\textbf{x})
    + n_p(\textbf{x},\textbf{y})
    - n_p(\textbf{x}) n_p(\textbf{y}) 
\end{equation}
the first term represents the sphere at $\textbf{r} = \textbf{0}$ and the second and third the spatial correlation. 

For know the Stresslet term allow $i=j$, so first of all note that,
\begin{align}
    \pSavg{\textbf{r}\delta_1\bm\sigma_f'' \cdot \textbf{n}}
    &=
    % \avg{\sum_{i,j}\delta(\textbf{x}-\textbf{x}_i)\delta(\textbf{y}-\textbf{x}_j)\intS{\textbf{r}\bm\sigma_f'' \cdot \textbf{n}}}\\
    % &=
    % \delta(\textbf{x}-\textbf{y})\avg{\sum_{i}\delta(\textbf{x}-\textbf{x}_i)\intS{\textbf{r}\bm\sigma_f'' \cdot \textbf{n}}}\\
    % &+ \avg{\sum_{i,j\neq i}\delta(\textbf{x}-\textbf{x}_i)\delta(\textbf{y}-\textbf{x}_j)\intS{\textbf{r}\bm\sigma_f'' \cdot \textbf{n}}}\\
    % &=
    % \delta(\textbf{x}-\textbf{y})\avg{\sum_{i}\delta(\textbf{x}-\textbf{x}_i)\intS{\textbf{r}\bm\sigma_f'' \cdot \textbf{n}}}\\
     \avg{\sum_{i,j\neq i}\delta(\textbf{x}-\textbf{x}_i)\delta(\textbf{y}-\textbf{x}_j)\intS{\textbf{r}\bm\sigma_f'' \cdot \textbf{n}}}\\
\end{align}
By definition the first term cancel because $\delta(\textbf{x}-\textbf{y})\pavg{\bm\sigma_f''} = 0$, that proves the last equality. 
This is totally logical as when the two particle are at the same place this correspond to the single particle case which we have remove with the mean stress. 
Thus, this term corresponds to the stresslet on a particle centered at \textbf{x} knowing a particle is present at $\textbf{y}=\textbf{x}+\textbf{r}$, minus the mean stress conditioned on that a particle is at \textbf{y}. 
similar comments can be made for the higher moments. 


The mean stress reads,
\begin{align}
    \pOavg{\div (\delta_1 \bm\Sigma^1-n_p \bm\Sigma)}
    &=
    \pOavg{\div \delta_1 \bm\Sigma^1}
    - n_p\pOavg{\div \bm\Sigma}\\
    &=
    \delta(\textbf{x}- \textbf{y})\pOavg{\div \bm\Sigma^1}
\end{align}


Let consider $\bm\Sigma^{(0)}$ the stress at $\mathcal{O}(\phi^1)$ and $\bm\Sigma^{(1)}$ at $\mathcal{O}(\phi^2)$, then the reciprocal theorem gives, 

\begin{equation}
    \intO[out]{\textbf{n}\cdot \bm\Sigma^{(1)} \cdot \textbf{u}^{(0)}}
    + \intO[out]{\textbf{f} \cdot \textbf{u}^{(0)}}
    =
    \intS{\textbf{n}\cdot \bm\Sigma^{(0)} \cdot \textbf{u}^{(1)}}
\end{equation}

Let consider the test problem the one of a translating droplets $\textbf{u}^{(0)}= \mathcal{U}(\textbf{r})\cdot \textbf{U}_2$, then 

\subsection{two particles problem to interaction}
We are looking for an equation for 
\begin{equation}
    \pavg{\delta_1 (\textbf{u}^0  - \textbf{u}^1)}
    =
    \pavg{\delta_1 (\textbf{u}^0  - \textbf{u}^1)}
\end{equation}
Hence one need to conditional avg the local eq by $\avg{\delta_1\delta_2 \textbf{u}^0}$ and then remove the mean contribution etc. 
This reads, 
\begin{align*}
%     n_{2p} \div \textbf{u}^2 &= 0, \\
%    n_{2p} \rho_f(\pddt 
%    + \textbf{u}^2 \cdot \grad) \textbf{u}^2
%    + \grad_r \avg{\delta_{12} \textbf{u}^0 \textbf{w}}
%    + \div \avg{\delta_{12} \textbf{u}''' \textbf{u}'''}
0
   &= 
   \div \bm\sigma_*^2
   +n_{2p} \rho_f \textbf{g}
   +\kappa\avg{\delta_{12}\delta_\Gamma  \bm\sigma_f'''\cdot \textbf{n}}
   +\kappa \phi^2 n_{2p}\div \bm\Sigma^2,
\end{align*}
\begin{equation}
    \bm\sigma_*^2 = 
    n_{2p} \bm\Sigma^2
    % + \kappa \avg{\chi_d \delta_{12} \bm\Sigma^1}
    + \zeta^{-1}\avg{\delta_{12}(\chi_d \bm\sigma_d''' + \delta_\Gamma \bm\sigma_\Gamma^0)}
    - \avg{\delta_{12} \chi_d \textbf{e}_d'''}
\end{equation}

We deduce directly the disturbance field equation, 
\begin{equation}
    n_{2p}\div \textbf{u}^{2''}=0
\end{equation}
\begin{equation*}
    0
    = 
    \div \bm\sigma_*^{2''}
    +\kappa\avg{\delta_\Gamma  (\delta_{12}\bm\sigma_f''' - n_{2p} \bm\sigma_f'')\cdot \textbf{n}}
    +\kappa \avg{\chi_d \div(\delta_{12} \bm\Sigma^1 - n_{2p} \bm\Sigma)},
 \end{equation*}
 \begin{equation}
     \bm\sigma_*^{2''} = 
     n_{2p} \bm\Sigma^{2''}
     % + \kappa \avg{\chi_d \delta_{12} \bm\Sigma^1}
     + \zeta^{-1}\avg{\chi_d (\delta_{12}\bm\sigma_d''' - n_{2p} \bm\sigma_d'') + (\delta_{12}-n_{2p})\delta_\Gamma \bm\sigma_\Gamma^0}
     - \avg{\chi_d (\delta_{12} \textbf{e}_d'''- n_{2p} \textbf{e}_d'')}
 \end{equation}

 At the boundary of the droplet the velocity field is presecibed by,
\begin{equation}
   \textbf{u}^{2''} \cdot \textbf{n} 
   = (\textbf{w}_2 - \textbf{u}^1) \cdot \textbf{n}
\end{equation}
Note that $\textbf{u}^1$ is an unkown because at $\phi^2$ this field is still not knwon this  is the reason why the reflexion method existe?  

This equation is of course complicated however note that,
\begin{align*}
    \delta_p \delta_{12}
    &=
    \sum_{i,j,k\neq j} 
    \delta(\textbf{x}- \textbf{x}_i)
    \delta(\textbf{r}_1- \textbf{x}_j)
    \delta(\textbf{r}_2- \textbf{x}_k)\\
    &=
    \delta(\textbf{r}_2- \textbf{x})
    \sum_{i,j,k=i\neq j} 
    \delta(\textbf{x}- \textbf{x}_i)
    \delta(\textbf{r}_1- \textbf{x}_j)\\
    &+\delta(\textbf{r}_1- \textbf{x})
    \sum_{i,k\neq i} 
    \delta(\textbf{x}- \textbf{x}_i)
    \delta(\textbf{r}_2- \textbf{x}_k)\\
    &+\sum_{i,j\neq i,k\neq j,i} 
    \delta(\textbf{x}- \textbf{x}_i)
    \delta(\textbf{r}_1- \textbf{x}_j)
    \delta(\textbf{r}_2- \textbf{x}_k)\\
\end{align*}
Hence only pts forces at the particle location remain at $\phi^2$, likewise  


We deduce that,
\begin{align*}
    F_{ij}(\textbf{x},\textbf{x}+\textbf{r})
    &=
    -\left(\frac{3 r^{-1}}{4} + \frac{r^{-3} \beta^{2}}{4} + \frac{r^{-3}}{4}\right)\delta^{ij} + \left(- \frac{3 r^{-3}}{4} + \frac{3 r^{-5} \beta^{2}}{4} + \frac{3 r^{-5}}{4}\right)r^{i}r^{j}\\
    S_{ij}(\textbf{x},\textbf{x}+\textbf{r})
    &=
    -\left(\frac{5 r^{-3}}{6} - \frac{5 r^{-5} \beta^{2}}{6} - \frac{r^{-5}}{2} + \frac{35 r^{-7} \beta^{2}}{12}\right)\delta^{il}r^{j} 
    + \left(\frac{5 r^{-5} \beta^{2}}{6} + \frac{r^{-5}}{2} - \frac{5 r^{-7} \beta^{2}}{6}\right)\delta^{ij}r^{l} \\
    &+ \left(\frac{5 r^{-5} \beta^{2}}{6} + \frac{r^{-5}}{2} - \frac{5 r^{-7} \beta^{2}}{6}\right)\delta^{jl}r^{i} 
    + \left(\frac{5 r^{-5}}{2} - \frac{25 r^{-7} \beta^{2}}{6}\right)r^{i}r^{j}r^{l}
\end{align*}

\begin{align}
    \pOavg{\div (\delta_1 \bm\Sigma^1-n_p \bm\Sigma)}
    &=
    \pOavg{\div \delta_1 \bm\Sigma^1}
    - n_p\pOavg{\div \bm\Sigma}\\
\end{align}

\subsubsection*{Kim \& Karria the reflexion }

Let, 
\begin{align*}
    \mathcal{G} 
    &= \frac{\delta_{ij}}{r}
    + \frac{x_ix_j}{r^3}\\
    \mathcal{E}
    &= 
    \frac{\delta_{ik}x_j}{r^3}
    - 3\frac{x_ix_jx_k}{r^5}\\
\end{align*}
Let the particles \textbf{1} and \textbf{2} produce the disturbance field, 
\begin{align}
    \textbf{v}_1= - \textbf{F}_1^{(0)}\cdot \textbf{V}_1 = - \textbf{F}_1^{(0)}\left[1+\frac{1}{6}\grad^2 \right]\frac{\mathcal{G}(\textbf{x}-\textbf{x}_1)}{8}\\
    \textbf{v}_2= - \textbf{F}_2^{(0)}\cdot \textbf{V}_2 = - \textbf{F}_2^{(0)}\left[1+\frac{\beta}{6}\grad^2 \right]\frac{\mathcal{G}(\textbf{x}-\textbf{x}_2)}{8}
\end{align}
\begin{align}
    \textbf{e}_1= - \textbf{F}_1^{(0)}\cdot \textbf{E}_1 = - \textbf{F}_1^{(0)}\left[1+\frac{1}{6}\grad^2 \right]\frac{\mathcal{E}(\textbf{x}-\textbf{x}_1)}{8}\\
    \textbf{e}_2= - \textbf{F}_2^{(0)}\cdot \textbf{E}_2 = - \textbf{F}_2^{(0)}\left[1+\frac{\beta}{6}\grad^2 \right]\frac{\mathcal{E}(\textbf{x}-\textbf{x}_2)}{8}
\end{align}
with zeorth order forces being related to the droplet translation with 
\begin{align*}
    \textbf{F}^{(0)}_1 = 6 (\textbf{u} - \textbf{w}_1)\\
    \textbf{F}^{(0)}_2 = 6 (\textbf{u} - \textbf{w}_2)
\end{align*}


At the first order reflexion we may say that, 
\begin{align}
    \textbf{F}^{(1)}_1/(\pi\mu a) 
    = 6 (1 + \frac{1}{6}\grad^2) \textbf{v}_2|_{\textbf{x}=\textbf{x}_1}
    = - \textbf{F}_2^{(0)}\cdot 6 (1 + \frac{1}{6}\grad^2) \left[1+\frac{\beta}{6}\grad^2 \right]\frac{\mathcal{G}(\textbf{x}-\textbf{x}_2)}{8}\\
    \textbf{F}^{(1)}_2/(\pi\mu a) 
    = 6 \beta (1 + \frac{\beta^2}{6}\grad^2) \textbf{v}_1|_{\textbf{x}=\textbf{x}_2}
    = - \textbf{F}_1^{(0)}\cdot 6\beta (1 + \frac{\beta}{6}\grad^2) \left[1+\frac{1}{6}\grad^2 \right]\frac{\mathcal{G}(\textbf{x}-\textbf{x}_1)}{8}
\end{align}
\begin{align}
    \textbf{S}^{(1)}_1/(\pi\mu a^3) 
    = \frac{20}{3} (1 + \frac{1}{10}\grad^2) \textbf{e}_2|_{\textbf{x}=\textbf{x}_1}
    = - \textbf{F}_2^{(0)}\cdot\frac{20}{3} (1 + \frac{1}{10}\grad^2) \left[1+\frac{\beta}{6}\grad^2 \right]\frac{\mathcal{E}(\textbf{x}-\textbf{x}_2)}{8}\\
    \textbf{S}^{(1)}_2/(\pi\mu a^3) 
    = \frac{20}{3} \beta^3 (1 + \frac{\beta^2}{10}\grad^2) \textbf{e}_1|_{\textbf{x}=\textbf{x}_2}
    = - \textbf{F}_2^{(0)}\cdot\frac{20}{3} (1 + \frac{1}{10}\grad^2) \left[1+\frac{1}{6}\grad^2 \right]\frac{\mathcal{E}(\textbf{x}-\textbf{x}_1)}{8}
\end{align}


\begin{align*}
    \textbf{v}_{21}
    = - \textbf{F}_1^{(1)}\cdot \left[1+\frac{1}{6}\grad^2 \right]\frac{\mathcal{G}(\textbf{x}-\textbf{x}_1)}{8}
    + \textbf{S}_1^{(1)} :\grad  \frac{\mathcal{G}(\textbf{x}-\textbf{x}_1)}{8}\\
    \textbf{v}_{12}
    = - \textbf{F}_2^{(1)}\cdot \left[1+\frac{\beta^2}{6}\grad^2 \right]\frac{\mathcal{G}(\textbf{x}-\textbf{x}_1)}{8}
    + \textbf{S}_2^{(1)} :\grad  \frac{\mathcal{G}(\textbf{x}-\textbf{x}_1)}{8}\\
\end{align*}

Then the second reflexion can be obtained by computing the forces, 
\begin{align}
    \textbf{F}^{(2)}_1/(\pi\mu a) 
    &= 6 (1 + \frac{1}{6}\grad^2) \textbf{v}_{12}|_{\textbf{x}=\textbf{x}_1}\\
    &= 
    - \textbf{F}_2^{(1)}\cdot 6 (1 + \frac{1}{6}\grad^2) \left[1+\frac{\beta}{6}\grad^2 \right]\frac{\mathcal{G}(\textbf{x}-\textbf{x}_2)}{8}\\
    &+ \textbf{S}_2^{(1)}:  6 (1 + \frac{1}{6}\grad^2) \left[1+\frac{\beta}{6}\grad^2 \right]\frac{\grad\mathcal{G}(\textbf{x}-\textbf{x}_2)}{8}\\
    % \textbf{F}^{(1)}_2/(\pi\mu a) 
    % = 6 \beta (1 + \frac{\beta^2}{6}\grad^2) \textbf{v}_1|_{\textbf{x}=\textbf{x}_2}
    % = - \textbf{F}_1^{(0)}\cdot 6\beta (1 + \frac{\beta}{6}\grad^2) \left[1+\frac{1}{6}\grad^2 \right]\frac{\mathcal{G}(\textbf{x}-\textbf{x}_1)}{8}
\end{align}
% \begin{align}
%     \textbf{S}^{(1)}_1/(\pi\mu a^3) 
%     = \frac{20}{3} (1 + \frac{1}{10}\grad^2) \textbf{e}_2|_{\textbf{x}=\textbf{x}_1}
%     = - \textbf{F}_2^{(0)}\cdot\frac{20}{3} (1 + \frac{1}{10}\grad^2) \left[1+\frac{\beta}{6}\grad^2 \right]\frac{\mathcal{E}(\textbf{x}-\textbf{x}_2)}{8}\\
%     \textbf{S}^{(1)}_2/(\pi\mu a^3) 
%     = \frac{20}{3} \beta^3 (1 + \frac{\beta^2}{10}\grad^2) \textbf{e}_1|_{\textbf{x}=\textbf{x}_2}
%     = - \textbf{F}_2^{(0)}\cdot\frac{20}{3} (1 + \frac{1}{10}\grad^2) \left[1+\frac{1}{6}\grad^2 \right]\frac{\mathcal{E}(\textbf{x}-\textbf{x}_1)}{8}
% \end{align}

\bibliography{Bib/bib_bulles.bib}
\appendix

\end{document}


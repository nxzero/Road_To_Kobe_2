\section{Conclusion}

The advancement conducted in this study provides a framework based on rigorous theoretical grounds to measure the microstructure geometry as well as the microstructure kinematic.
The major advancement can be summed up into 3 key points :
\begin{enumerate}
    \item Based on \ref{eq:dt_R} we could infer that the time of relaxation of $\textbf{R}(\textbf{x},t)$, is the mean age of interaction of the  nearest droplet pairs $\tau_p(\textbf{x},t)$. 
    \item Likewise, we could show that the relative velocity between droplet pairs scales as $\tau_p /d_p$ for all our cases, while its time of relaxation is also $\tau_p$. 
    The trends of $\tau_p$ have also been captured, it is shown to be longer for $\lambda = 1$, shorter for $\lambda = 10$, and even shorter for solid droplets at same $Ga$ and $\phi$. 
    All these remarks enabled us to build a model for the normal approach velocity between the nearest pair of droplets to feed the PBE models. 
    \item By studying \ref{eq:dt_R} we demonstrated that the correlation between $\textbf{w}_p^\text{nst}$ and \textbf{r}, acts as a source terms for $\textbf{R}(\textbf{x},t)$, which motivated us to study the droplets relative velocity conditioned on the relative position. 
    By a careful analysis of $\textbf{w}_p^\text{r}$ and $\textbf{u}_p^\text{r}$, we could show that for $\lambda = 10$ droplets rise faster with a nearest neighbor on top or bottom, while for $\lambda = 1$ droplets goes faster when the nearest neighbor is at large distances.
    Additionally, we could clearly identify and quantify phenomena such as the DKT mechanism and its derivatives. 
\end{enumerate}

This study has been motivated with the view of building a coalescence kernel as well as other closure terms for averaged dispersed two-phase flow equations and PBE. 
These coalescence models are often based on film drainage approximations which assume a normal approach between droplets \citet{chesters1991modelling}.  
However, in \ref{sec:results}, we have seen that the interactions are more likely to happen tangentially rather than with a normal approach. 
We also demonstrated that $\lambda$ and $Ga$ have a strong impact on the spatial arrangement of the droplets and on their relative kinematics. 
For these reasons we conclude that there is a clear need to take in account these mechanisms, with the objective of building more accurate coalescence kernels. 
This could be done by considering more sophisticated film drainage situations which would consider relative velocities consistent with $\textbf{w}^\text{nst}(\textbf{x},\textbf{r},t,a)$ distributions documented in this study.
More globally, the coalescence kernel might be re phrased in the context of the nearest particle statistic to incorporate the various forms of $P_\text{nst}$. 
These issues will be addressed in future work.  

Finally, this ``kinematic'' analysis must be completed by a ``dynamic'' analysis to understand completely the pairs' interaction between droplets. 
This also has to be postponed for a future study. 

% \section*{Acknowledgement}

% The computational power of  \textit{TGCC - tr\`es grand centre de calcul du CEA} is greatly appreciated. 
% % The author thanks anonymous reviewers for helpful comments.
% \section*{Data availability}

% All the data presented in this study are available upon the request at the author. 
% The buoyant emulsion simulations can be reproduced using the basilisk \texttt{.c} file \url{http://basilisk.fr/sandbox/fintzin/Rising-Suspension/RS.c}, and following the instruction herein. 
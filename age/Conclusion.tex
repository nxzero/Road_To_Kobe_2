


The advancement conducted in this study provide a numerical framework to perform statistically steady simulation of rising emulsion with non coalescing droplets.
Additionally, we provide a theoretical framework based on solid theoretical ground to measure the microstructure geometry and timescale of the microstructure as well as the one of the particles interactions.
The major advancement can be sum up into 4 key points :
\begin{enumerate}
    % \item The major advancements proposed in this work is twofold. 
    \item First, We developed an optimized Multi-VoF method within the \texttt{Basilisk} flow solver. 
    It enables us to avoid coalesce between an arbitrary number of droplets, while keeping the number of tracers inferior or equal to $7$. 
    Additionally, we show that the Multi-VoF method is capable of capturing the physics of interfaces interaction despite the coarse mesh definition at the scale of the film, between the droplets, see \ref{ap:validation} (\textit{Case 2}). 
    This enables us to perform massive DNS calculations of rising mono-disperse emulsion with various $\lambda$, $\phi$ and $Ga$ for an arbitrary amount of time.
    % Validation are also provided, see \ref{ap:validation} (\textit{Case 3}).  
    \item We made use of the recent \textit{nearest particle statistic} framework of \citet{zhang2023evolution} to introduce an objective and concise way to measure the microstructure via the nearest particle pair density function $P_\text{nst}(\textbf{x},\textbf{r},t,a)$. 
    By looking at the pair distribution function we could show qualitatively the influence of $Ga$, $\lambda$ and$\phi$ on the microstructure.
    Especially, it was found that 
    (1) At low $Ga$ isotropic clusters (see \ref{fig:scheme_clusters}(\textit{Case 2})) appear with increasing $\phi$. 
    (2) Non-isotropic clusters (see \ref{fig:scheme_clusterse}(\textit{Case 3})) are more likely to form for high \textit{Galileo} number.
    (3) The viscosity ratio $\lambda$ has an important impact on the microstructure, clearly more  clusters and layers are formed for $\lambda = 1$ than $\lambda = 10$. 
    And even less clusters are observed for solid particles suspensions. 
    \item Following \citet{zhang2023evolution} we show that the microstructure can be well described by the second moment of the probability density, noted $\textbf{R}(\textbf{x},t)$. 
    This constitutes the major finding of this work, indeed we provided evidences that $\textbf{R}(\textbf{x},t)$ is a reliable and objective way to measure the microstructure.
    As predicted by \citet{zhang2023evolution} its trace indicates the mean square distance between the nearest particles, ultimately a small trace witnesses of the presence of packed particles pair or clusters.
    Its anisotropic part indicates the presence of layers or side-by-side particle pairs in the flow. 
    % \item In a second step we showed that a conservation equation can be derived for $\textbf{R}(\textbf{x},t)$ based on kinetic-theory like concept. 
    Based on \ref{eq:dt_R} we could infer that the time of relaxation of $\textbf{R}(\textbf{x},t)$, is the mean age of interaction of the  nearest particles pairs $\tau_p(\textbf{x},t)$. 
    Likewise, we could show that the relative velocity between particles pairs scales as $\tau_p /d_p$ for all our cases, while its time of relaxation is also $\tau_p$. 
    The trends of $\tau_p$ have also been captured, it is shown to be longer for $\lambda = 1$, and shorter for $\lambda = 10$, and even shorter for solid particles at same $Ga$ and $\phi$. 
    \item 
    By studying \ref{eq:dt_R} we demonstrated that the correlation between $\textbf{w}_p^\text{nst}$ and \textbf{r}, acts as a source terms for $\textbf{R}(\textbf{x},t)$, which motivated us to study the particles relative velocity conditioned on the relative position. 
    By a careful analysis of $\textbf{w}_p^\text{r}$ and $\textbf{u}_p^\text{r}$, we could show that for $\lambda = 10$ particles goes faster with a nearest neighbor on top or bottom, while for $\lambda = 1$ particles goes faster when the nearest neighbor is at large distance.
    Additionally, we could clearly identify and quantify phenomena such as the DKT mechanism and its derivatives. 
\end{enumerate}
To the author knowledge, none of the previous studies, apart from \citet{zhang2023evolution} which introduced the concept, made use of such simple quantity to describe the microstructure. 
We believe that the \textit{Nearest particle statistics} framework is powerful to model multiphase-flow's microstructure as it provide an objective and concise way to measure its pair distribution. 
Additionally, its ability to be included in a kinetic theory, in opposition to other method,  such as the Voronoi cell volume approach \citep{senthil2005voronoi}, makes it promising. 

This study has been motivated with the view of building a coalescence kernel as well as other closure terms for averaged dispersed two-phase flow equations and PBE. 
These coalescence models are often based on film drainage approximations which assume a normal approach between droplets \citet{chesters1991modelling}.  
However, in \ref{sec:velocity}, we have seen that the interactions are more likely to happen tangentially rather than with a normal approach. 
We also demonstrated that $\lambda$ and $Ga$ have a strong impact on the spatial arrangement of the particles and on their relative kinematics. 
For these reasons we conclude that there is a clear need to take in account these mechanisms, with the objective of building more accurate coalescence kernels. 
This could be done by considering more sophisticated film drainage situations which would consider relative velocities consistent with $\textbf{w}^\text{nst}(\textbf{x},\textbf{r},t,a)$ distributions documented in this study.
More globally, the coalescence kernel might be re phrase in the context of the nearest particle statistic to incorporate the various form of $P_\text{nst}$. 
These issues will be addressed in future work.  



\section*{Acknowledgement}

The computational power of  \textit{TGCC - tr\`es grand centre de calcul du CEA} is greatly appreciated. 
% The author thanks anonymous reviewers for helpful comments.
\section*{Data availability}

All the data presented in this study are available upon the request at the author. 
The buoyant emulsion simulations can be reproduced using the basilisk \texttt{.c} file \url{http://basilisk.fr/sandbox/fintzin/Rising-Suspenion/RS.c}, and following the instruction herein. 
The \texttt{Basilisk} code has been validated numerous time in previous numerical studies. 
Especially, we can cite the recent studies of \citet{innocenti2020direct} and \citet{hidman2023assessing} which both performed DNS of rising suspension of bubbles. 
Nevertheless, in this work we investigate specific statistical distribution,
and we make use of a multi-VoF method to avoid droplets coalescence, therefore a meticulous validation of the DNS is in order. 


\subsection*{Mesh independence and statistical convergence for random array of drops}

Even through aforementioned studies carried validation of the \texttt{Basilisk} code for rising droplets or bubbles, almost all of them considered isolated droplets or bubbles as the only validation case. 
As far as the author's knowledge, to this date no published study presented a mesh independence study for random array of droplets nor bubbles of this scale. 
Nevertheless, as particles interaction and higher \textit{Galileo} numbers may be more challenging to model, it is primordial to investigate the mesh independence of the exact same DNS that are carried in this work. 
In this objective we performed DNS of random array of $N_b=125$ droplets, with the following parameters,
\begin{align*}
    \lambda = 10,
    && \zeta = 1.11,
    && Bo = 1,
    && Ga = 100,
    && \phi = 0.1,
    && N_b =125,
\end{align*}
and the mesh definition is, $d/\Delta = 7.37, 14.74, 29.9 58.97$. 
We expect that the most challenging DNS simulated in this work is for the case, $\lambda = 10$ and $Ga = 100$, since it is in this range of parameters that we induce the most vorticity, which ultimately require good mesh definition. 
Additionally, in opposition to the ordered array case, this case includes droplets interaction, which ultimately induce more numerical complexities to tackles. 
Based on this remark we can assume that if this case is mesh independent, then all cases from \ref{tab:simulations} must be since this is the most challenging scenario.   

Now let's study the mesh influence on the statistics. 
It is clear from \ref{fig:apstat} (left) that both mesh definition produce nearly the same radial distribution, no notable difference is identified. 
In \ref{fig:apstat} (middle) we can observe the age distribution for both mesh definition. 
It is clear that refining the mesh induce a difference in the age distribution. 
As, a matter of fact it has a small impact on the mean age, $\tau_p = 6.96$ for the lower definition, and $\tau_p = 6.14$ for the finest grid.
This makes a $10\%$ error, but as mentioned above this is probably the highest error that we could encounter among all cases. 
\begin{figure}
    \centering
    \includegraphics[height = 0.24\textwidth]{image/HOMOGENEOUS_NEW/VAL/Pr.pdf}
    \includegraphics[height = 0.24\textwidth]{image/HOMOGENEOUS_NEW/VAL/Pa.pdf}
    \includegraphics[height = 0.24\textwidth]{image/HOMOGENEOUS_NEW/VAL/w.pdf}
    \caption{
        Statistical averaged functions for two mesh definition. 
        (left) Radial normalized probability density function  $P_r(\textbf{x},|\textbf{r}|,t)/P_\text{th}$, in terms of the dimensionless radial position. 
        (middle) Probability density function of the age distribution $P_a(\textbf{x},t,a)$. 
        (right) Nearest averaged dimensionless approach velocity for both mesh definition, in terms of the dimensionless age. 
    }
    \label{fig:apstat}
\end{figure}
Even through an error is identified on the mean duration of interaction we still note that both nearest averaged dimensionless approach velocity on \ref{fig:apstat} (right) match perfectly. 
\begin{figure}[h!]
    \centering
    \includegraphics[height = 0.3\textwidth]{image/HOMOGENEOUS_NEW/VAL/U_rel_ndc_25.pdf}
    \includegraphics[height = 0.3\textwidth]{image/HOMOGENEOUS_NEW/VAL/U_rel_ndc_35.pdf}
    \caption{Quiver plots of the relative averaged velocity field $\textbf{w}^\text{r}(\textbf{x},\textbf{r},t)$ colored by the averaged dimensionless age $a^r(\textbf{x},\textbf{r},t)$, for $\phi = 0.05$ and $Ga = 100$. 
    (left) Low mesh definition.
    (right) High mesh definition. 
    }
    \label{fig:velap}
\end{figure}
Regarding, the 2D fields  $\textbf{w}^\text{r}(\textbf{x},\textbf{r},t)$ we can see that no notable difference can be identified, if it is not the slight difference in the value of the age scale. 


Overall, the one dimensional and two-dimensional conditioned statistics are almost independent of the mesh definition. 
By obtaining the same statistics with two independent DNS makes us confident on the fact that our numerical samples is large enough.
Indeed, if the samples were not sufficient we would have obtained two different distribution functions, thus we can be sure that the statistics have well converged. 
The slight difference in rising velocity (see \citet[Appendix A]{fintzi2024buoyancy}) and age distribution found for these Reynolds number must be acknowledged.
As mentioned at the beginning, this case is in fact very challenging as the volume fraction of droplets is consequent which induce numerous inertial interactions. 
Nevertheless, we can be sure that our final results is accurate at most with a $5\%$ error for this case, and probably less for the others cases. 
These, error testify for the very challenging  aspect of these simulations. 
Overall, we have great confidence in the statistical and physical representativity of our DNS results. 
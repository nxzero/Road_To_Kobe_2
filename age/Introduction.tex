\section{Introduction}





\begin{enumerate}
    \item We use extensively averaged models in industry 
    \item These models necessitate closure laws. These closure laws are highly dependent on the particles' microstructure. 
    \item In our previous work we performed a \textit{Nearest-particle-statistics} analysis of the microstructure geometry, or the steady state microstructure. 
    \item All the closures in the literature are performed in steady state system. 
    Therefore, in order to be valid in a Euler-Euler averaged context the time scales of the simulation must be greater than the time for the microstructure to reach its steady state. 
    Only then the assumption of quasi steady ness is valid.    
    \item Therefore, in this study we provide a theoritical and numerical analyssi of the timescale that drives the microstructure. 
    Doing so, we demonstrate that the particle interaction time is the main timescale involved in the microstructure relaxation time. 
    We therefor also study the relative kinematic
\end{enumerate}
Buoyancy-driven droplet flows are encountered in many chemical engineering processes such as gravity separators and liquid-liquid extractors. 
The usual engineering practice to model such facilities is to make use of the averaged Navier-Stokes equations and Population Balance Equations (PBE). 
\begin{itemize}
    \item Show that we use AVG model
    \item Show that these models need closure laws
    \item Show that all of these closure are performed in random arrays or in steady state system. 
    \item Show that for these closure to be true the "real" microstructure needs to be either random or that it need to reach it's steady state fast engouh. 
    \item Therefore it is useful to obtain an idea of the time that it takes for the microstructure to reach its steady state. 
    \item In this objective we study natturally the particles relative velocity interaction 
\end{itemize}
% \paragraph*{Why microstructure is interesting :}
However, these methods necessitate closure laws and a deep understanding of particle pair statistics.
Especially, these closures laws dependent highly on the physical properties of the fluids and of the statistically steady state particles' arrangement, that is called in this context : the microstructure. 
A simple example is given in \citet{yin2008lattice}, where they demonstrate that the mean rise velocity of buoyant mono disperse bubbly flows can be express with a power-law of $(1-\phi)$ for random particles' arrangement.
While when the microstructure exhibits anisotropic particles' structures this power-law doesn't fit the results anymore.
Consequently, the mean drag force or mean drift velocity of suspensions, which are crucial closures in two-phase flow problems, depend on the microstructure geometry.  
A second example, which is relevant for our industrial applications, is the one of poly-disperse flows.
In this case we might use PBE to model droplets' size distribution \citep{randolph2012theory}.
In this case PBE heavily rely on the coalescence kernel, which act as a source term describing the rate of coalescence of droplets.   
Ultimately, this kernel is microstructure dependent, and also depends on the relative kinematic between particles pairs \citep{chesters1991modelling}. 
Consequently, pair particles statistics which describe the relative particles' kinematic as well as the microstructure geometry are of utmost importance in the averaged models. 
Therefore, in this study we propose to characterize the microstructure of an emulsion as well as the relative particles pairs kinematic over a wide range of dimensionless parameters in the perspective of building more accurate models for averaged two phases flows equations and more generally for a better understanding of droplets flows.


Therefore, in this study, we follow \citet{zhang2023evolution} and describe the microstructure's geometry with a single second order tensor quantity. 
This tensor corresponds to the second moment of the nearest particles pair distribution.  
Additionally, within this framework we are able to characterize pair kinematics and particles time of interaction, which makes it particularly useful for our purpose. 
Indeed, the former aspect has often been overlooked in multiphase flows studies, although the time of interaction is crucial for coalescence kernel modeling.

% \paragraph*{Plan: }
Therefore, within a multiscale strategy, we conduct tri-periodic DNS of mono-disperse buoyancy-driven suspensions of drops. 
We begin in \ref{sec:methodo} by outlining the simulations' methodology, and describe our whole new algorithm which permits us to prevent numerical coalescence between droplets. 
Following that, we briefly introduce the \textit{Nearest Particle Statistics} framework. 
In \ref{sec:microstructure}, we delve into an analysis of the microstructure geometry and identify structures such as it is depicted \ref{fig:scheme_clusters} with respect to the dimensionless parameters of the present problem. 
Then, in \ref{sec:time}, we discuss the relevant timescale of the particles pair interaction, once again using the \textit{Nearest Particle Statistics}. 
We conclude this study in \ref{sec:velocity}, with a thorough analysis of the particles' pair relative velocity and draw conclusions on their implications for the averaged models.


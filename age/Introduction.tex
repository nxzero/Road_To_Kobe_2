\section{Introduction}

In the previous chapter, we conducted an analysis of the microstructure geometry, specifically quantifying the arrangement of the nearest droplet pairs within the flow at steady-state equilibrium. 
We also noted in the introduction that the primary closure terms needed for the averaged Navier-Stokes equations are highly dependent on the microstructure geometry. 
For example, the closure terms developed in the following chapters will only be valid under the assumption that the microstructure has already reached equilibrium, for a given set of dimensionless parameters.
This is because to compute the closures we approximate an ensemble average with a time average which requires that the given microstate (including the microstructure geometry) is statistically steady over time. 

Consequently, for these closures to be valid in an Euler-Euler simulation, the macroscopic simulation timescales must exceed the time required for the microstructure to reach a steady state, as the closure terms are not capable of describing the transient period required to reach this steady microstructure. 
Indeed, if one only has closure terms derived under the steady-state assumption, one can only accurately model timescales longer than the time needed for the microstructure to reach equilibrium. 
This ensures that transitional variations in the closure values (such as the drag force), which fluctuate due to the formation of the microstructure, can be safely disregarded.


Therefore, in this study, we provide a theoretical and numerical analysis of the timescale governing the microstructure. 
In other words, we examine the time it takes for the microstructure to reach equilibrium. 
We demonstrate, that the droplet interaction time is the primary timescale involved in microstructure relaxation. 
From this analysis, we find that the correlation between the position and relative velocity of the nearest pairs of droplets partly governs the formation of the microstructure. 
Consequently, a comprehensive analysis of the relative velocity between the nearest droplets is conducted, notably providing empirical closures for normal approach velocities between nearest pairs. 

This chapter is organized as follows: 
In \ref{sec:Theory} we develop the theoretical framework to describe the evolution of the second moment of the nearest pair distribution, denoted as \textbf{R}.
This study primarily builds upon the analysis of \citet{zhang2023evolution}. 
Then, in \ref{sec:methodo2} we briefly present the numerical methodology, which is similar to that used in \citet{fintzi2024buoyancy}, with additional details on the method for computing Lagrangian quantities. 
Finally, in \ref{sec:results} we present the numerical results, where we quantify and visualize the relative kinematics   between droplets for given values of the viscosity ratio $\lambda$, \textit{Galileo} number $Ga$, and volume fraction $\phi$.


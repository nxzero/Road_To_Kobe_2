\subsection{Equation of the mean square distance to the nearest neighbor}

We now derive an equation for the scalar $R = \bm\delta : \textbf{R}$ which is the averaged square distance to the nearest neighbor. 
This is done by applying the double contracted product of \ref{eq:dt_R} with $\bm\delta$, this leads directly to 
\begin{equation}
    \pddt (n_pR)
    + \pddx \cdot [n_p(\textbf{u}_pR
    + \bm\delta :\textbf{R}^\text{Re})]
    = 
    - \frac{n_pR}{\tau_p}
    +n_p  \bm\delta : (\textbf{B}+\textbf{D}+\textbf{W})
    \label{eq:dt_deltaR}
\end{equation}
% With,
% \begin{align*}
%     % R^\text{Re}(\textbf{x},t)
%     % =\bm\delta:\textbf{R}^\text{Re}(\textbf{x},t)\\
%     % B(\textbf{x},t)=\bm\delta : \textbf{B}(\textbf{x},t) \\
%     % D(\textbf{x},t) = \bm\delta: \textbf{D}(\textbf{x},t) \\
%     W(\textbf{x},t) = \frac{2}{n_p(\textbf{x},t)}
%     \int_{0}^\infty
%     \int_{\mathbb{R}^3} 
%         \textbf{r} \cdot \textbf{w}^\text{nst}_p
%     P_\text{nst}
%     d\textbf{r}
%     da.
% \end{align*}
Although this equation seems similar to \ref{eq:dt_R}, a major simplification makes it fundamentally different. 
Indeed, notice that in spherical coordinate the three terms on the right hand-side of \ref{eq:dt_deltaR} may be written
\begin{equation*}
    - \frac{n_pR}{\tau_p}
    + n_p  \bm\delta : (\textbf{B}+\textbf{D})
    = 
    \int_0^\infty r^2 \int_{0}^{\infty}\oint_{\mathbb{R}^2}\left[
        \delta(a)P(\textbf{x},\textbf{r},0,t)
    - \frac{P_\text{nst}(\textbf{x},\textbf{r},t,a)}{\tau^\text{nst}(\textbf{x},\textbf{r},t,a)}
    \right]dS(r) da dr.
\end{equation*}
Taking \ref{eq:int_dt_h} into account leads directly to $- \frac{n_pR}{\tau_p} + n_p  \bm\delta : (\textbf{B}+\textbf{D}) = 0$. 
In other worlds, the transport equation of $R(\textbf{x},t)$ reads 
\begin{equation}
    \pddt (n_pR)
    + \pddx \cdot [n_p(\textbf{u}_pR
    + \bm\delta : \textbf{R}^\text{Re})]
    = 
    n_p\bm\delta : \textbf{W}. 
    \label{eq:dt_deltaR2}
\end{equation}
Which simply state that the mean square distance to the nearest neighbor $R$, evolves according to the relative velocity-position correlation.
Therefore, in opposition to $\textbf{R}(\textbf{x},t)$, the evolution of $R(\textbf{x},t)$ is completely determined by the unknown scalar $W(\textbf{x},t)$. 


% \subsection{The homogeneous and steady state assumption}

% In and homogeneous and steady state regime all variables are constant with the macroscopic variables \textbf{x} and $t$.   
% Thus, from \ref{eq:dt_R} and \ref{eq:dt_deltaR2} we deduce that in this situation the following relations hold, 
% \begin{align}
%     \textbf{R}
%     = \tau_p (
%     \textbf{B}
%     + \textbf{D}
%     + \textbf{W})\\
%     W = \frac{2}{n_p(\textbf{x},t)}
%     \int_{0}^\infty
%     \int_{\mathbb{R}^3} 
%         \textbf{r} \cdot \textbf{w}^\text{nst}_p
%     P_\text{nst}
%     d\textbf{r}
%     da = 0. 
%     \label{eq:cdt_for_W}
% \end{align}
% The latter equation state that the resultant of the normal approach velocity between nearest neighbors is null. 

\subsection{The random destruction assumption}

It is possible to derive an analytical formula for the age distribution $P_a(a|\textbf{x},t)$, under the \textit{random destruction assumption} \citep{zhang2023evolution}.
This assumption first assume a steady and homogeneous situation, such that all derivatives on the macroscopic variables, i.e. \textbf{x} and $t$, are negligible.
Let us now recall the statement of this assumption. 
To that end we introduce the mean rate of pairs destruction of age $a$ as
\begin{equation*}
    1/\tau_p^a(\textbf{x},t,a) 
    = \frac{1}{n_p(\textbf{x},t)}\int_{\mathbb{R}^3} \frac{P_\text{nst}(\textbf{x},\textbf{r},t,a)}{\tau^\text{nst}(\textbf{x},\textbf{r},t,a)} d\textbf{r}.
\end{equation*}
This assumption states that the probability of particle pairs destruction averaged on all position, i.e. $1/\tau_p^a(\textbf{x},t,a)$, is uncorrelated with the age of interaction.
In other worlds, we consider that any pairs of nearest neighbors can be broken apart equally likely regardless of its current age of interaction, or that $\tau^a_p(\textbf{x},t,a) = \tau_p(\textbf{x},t)$.
It must be understood that $\tau^a_p$ is considered as independent of the age, which does not imply that $\tau^\text{nst}_p$ is independent of the relative position $\textbf{r}$, which is not true.

Under this assumption, which will be shown to be valid in our parameters' range, we can derive an analytical formula for the age distribution \citep{zhang2023evolution}, namely,
\begin{align}
    P_a(a|\textbf{x},t)  
    =\frac{e^{-a/\tau_p(\textbf{x},t)}}{\tau_p(\textbf{x},t)}.
    \label{eq:Pa}
\end{align} 
We can see that the age distribution is solely function of $\tau_p(\textbf{x},t)$ and that it is normalized to $1$.
Additionally, with this definition $\tau_p(\textbf{x},t)$, turns out to be equivalent to the first moment of the age distribution, indeed we have 
\begin{equation*}
    \int_{0}^\infty
    a P_\text{a}(\textbf{x},t,a)
    da
    =\tau_p(\textbf{x},t). 
\end{equation*}
Consequently, within the random destruction assumption $\textbf{R}(\textbf{x},t)$ has a relaxation time equal to the first moment of the age distribution $P_a(a|\textbf{x},t)$. 
Based on the definition of the ``age'' we can state that the mean age is equivalent to the mean particle interaction time.  
Thus, from \ref{eq:dt_R} and \ref{eq:Pa} we can state that the microstructure reaches a statistically steady-state equilibrium after a time of the order of the mean particle interaction time $\tau_p$. 

\tb{
\subsection*{Relation between rate of death and age}

If the above hypothesis does not hold, then the mean age is defined as, 
\begin{equation*}
    a_p(\textbf{x},t) = \int_{0}^{\infty} a P_a(\textbf{x},t,a) da
\end{equation*}
and the mean rate of destruction of particles $\tau_p(\textbf{x},t)$ as in \ref{eq:tau_p}. 
These two quantities are a priori uncorrelated. 
First notice that the mean rate of destruction is related to the mean rate of creation of nearest pair through \ref{eq:int_dt_h}
\begin{equation}
    \int_{0}^{\infty}\int_{\mathbb{R}^3}\left[
        \delta(a)P(\textbf{x},\textbf{r},t,a)
    \right]d\textbf{r} da    
    =
    \int_{0}^{\infty}\int_{\mathbb{R}^3}\left[
    \frac{P_\text{nst}(\textbf{x},\textbf{r},t,a)}{\tau^\text{nst}(\textbf{x},\textbf{r},t,a)}
    \right]dS(r) da    
\end{equation}
In other worlds we have for all $\textbf{x}$ and $t$, 
\begin{equation}
    P(a = 0|\textbf{x},t) 
    =
    \frac{1}{\tau_p(\textbf{x},t)}
    \Longleftrightarrow
    \frac{1}{P(a = 0|\textbf{x},t)}
    =
    \tau_p(\textbf{x},t)
\end{equation}
where $P(a = 0|\textbf{x},t)$ is the probability of finding a particle with \textit{age} $0$ knowing that a particle is present at $\textbf{x}$. 
Therefore, $1/\tau_p$ is the probability of finding a particle at $\textbf{x}$ 

The link, without any assumption between these two quantities is made by multiplying \ref{eq:dt_Pnst} by $a$, it yields
\begin{equation*}
    \pddt (a P_\text{nst})
    + \pdda (a P_\text{nst})
    + \pddx \cdot  ( a \textbf{u}^\text{nst}_p  P_\text{nst})
    + \pddr \cdot  ( a \textbf{w}^\text{nst}_p  P_\text{nst})
    = a \delta(a) P(\textbf{x},\textbf{r},t,a)
    % - \frac{P_\text{nst}(\textbf{x},\textbf{r},t,a)}{\tau^\text{nst}(\textbf{x},\textbf{r},t,a)}
    + P_\text{nst}(\textbf{x},\textbf{r},t,a)\left(
        1 - \frac{a}{\tau_p^{nst}}
    \right)
\end{equation*}
Then, integrating over $\iint_{\mathbb{R}^4} \ldots d\textbf{r}da$ gives, 
\begin{equation*}
    \pddt (n_p a_p)
    + \pddx \cdot  (n_p a_p \textbf{u}_p + n_p a_p^{Re})
    = 
    + n_p \left(
        1
    - \int_{\mathbb{R}} \frac{a P_a}{\tau_p^a}  da\right)
\end{equation*}
In homogeneous and steady state 
\begin{equation*}
    \int_{\mathbb{R}} \frac{a P_a(a|\textbf{x},t)}{\tau_p^a(\textbf{x},t,a)}  da
    = 
    \frac{a_p}{\tau_p}
    + \int_{\mathbb{R}} a P_a(a|\textbf{x},t)\left[
        \frac{1}{\tau_p^a(\textbf{x},t,a)} 
        - \frac{1}{\tau_p^a(\textbf{x},t)} 
    \right] da
    = 1 
\end{equation*}
Or 
\begin{equation*}
    a_p
    = \tau_p \left[
        1
        + 
        \int_{\mathbb{R}} a P_a(a|\textbf{x},t)\left(
            \frac{1}{\tau_p(\textbf{x},t)} 
            - \frac{1}{\tau_p^a(\textbf{x},t,a)} 
        \right) da
    \right]
\end{equation*}
The relation $a_p = \tau_p$ indeed require that $\tau_p^a = \tau_p$.

\paragraph*{Second strategy}
Instead of multiplying by $a$ we now multiply by $a^2$ and get
\begin{equation*}
    \pddt (a^2 P_\text{nst})
    + \pdda (a^2 P_\text{nst})
    + \pddx \cdot  ( a^2 \textbf{u}^\text{nst}_p  P_\text{nst})
    + \pddr \cdot  ( a^2 \textbf{w}^\text{nst}_p  P_\text{nst})
    = a^2 \delta(a) P(\textbf{x},\textbf{r},t,a)
    % - \frac{P_\text{nst}(\textbf{x},\textbf{r},t,a)}{\tau^\text{nst}(\textbf{x},\textbf{r},t,a)}
    - P_\text{nst}(\textbf{x},\textbf{r},t,a) 2 a
    - P_\text{nst}(\textbf{x},\textbf{r},t,a) \frac{a^2}{\tau_p^{nst}},
\end{equation*}
which upon integration give
\begin{equation*}
    \pddt (a^2_p n_p)
    + \pddx \cdot  ( a^2_p \textbf{u}_p  n_p + n_p a^{2-Re}_p)
    = 
    + 2 a_p n_p
    - \int_\mathbb{R} P_\text{nst}(\textbf{x},\textbf{r},t,a) \frac{a^2}{\tau_p^{nst}} da,
\end{equation*}
which gives in steady state 
\begin{equation}
    a_p
    = 
    \frac{1}{2}
    \int_\mathbb{R}
    P_a(\textbf{x},t,a) \frac{a^2}{\tau_p^{nst}} da,
    = \frac{1}{2\tau_p}
    \int_\mathbb{R}
    a^2 P_a(a|\textbf{x},t) da
    +  \frac{1}{2}
    \int_{\mathbb{R}} a^2 P_a(a|\textbf{x},t)\left(
        \frac{1}{\tau_p^a(\textbf{x},t,a)} 
        - \frac{1}{\tau_p(\textbf{x},t)} 
    \right) da
\end{equation}
which is equal to the std of the nearest pair distribution + the fluctuation of $\tau$ so it is not  adventageous. 
\textbf{My goals : }
\begin{itemize}
    \item $a_p$ is interesting by it self since it represent the time of interaction. 
    \item $\tau_p$ is also interesting since it might represent the microstructure relaxation (under very specifics circonstencies)
    \begin{itemize}
        \item In any case I can compute that and get the microstructure relaxation time so it is cool, we can give some closue
        \item Also 
    \end{itemize}
    \item They are both interesting, and under the random destruction assumption both are linked through $a_p = \tau_p$. 
\end{itemize}

\paragraph*{Numerical computation of the birth rate}

Let start from the relation 
\begin{equation}
    a_p 
    =
    \int_{\mathbb{R}} \frac{a P_a(a|\textbf{x},t)}{\tau_p^a(\textbf{x},t,a)}  da
    = \tau_p
\end{equation}

Pour verifier les hypothèse quantitativeent :
\begin{itemize}
    \item Soit je regarde le premier bin (mais pas precis et might be random)
    \item Soit je fit et je compare !! ! ! pas trop le choix de faire ca ! ! ! Obliger de verifier qualitativement par contre
\end{itemize}
}
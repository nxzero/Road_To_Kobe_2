\subsection{Equation of the mean square distance to the nearest neighbor}

We now derive an equation for the scalar $R = \bm\delta : \textbf{R}$ which is the ensemble-averaged square distance to the nearest neighbor. 
This is done by applying the double contracted product of \ref{eq:dt_R} with $\bm\delta$, this leads directly to 
\begin{equation}
    \pddt (n_pR)
    + \pddx \cdot [n_p(\textbf{u}_pR
    + \bm\delta :\textbf{R}^\text{Re})]
    = 
    - \frac{n_pR}{\tau_p}
    +n_p  \bm\delta : (\textbf{B}+\textbf{D}+\textbf{W})
    \label{eq:dt_deltaR}
\end{equation}
% With,
% \begin{align*}
%     % R^\text{Re}(\textbf{x},t)
%     % =\bm\delta:\textbf{R}^\text{Re}(\textbf{x},t)\\
%     % B(\textbf{x},t)=\bm\delta : \textbf{B}(\textbf{x},t) \\
%     % D(\textbf{x},t) = \bm\delta: \textbf{D}(\textbf{x},t) \\
%     W(\textbf{x},t) = \frac{2}{n_p(\textbf{x},t)}
%     \int_{0}^\infty
%     \int_{\mathbb{R}^3} 
%         \textbf{r} \cdot \textbf{w}^\text{nst}_p
%     P_\text{nst}
%     d\textbf{r}
%     da.
% \end{align*}
Although this equation seems similar to \ref{eq:dt_R}, a major simplification makes it fundamentally different. 
Indeed, notice that in spherical coordinates the three terms on the right-hand side of \ref{eq:dt_deltaR} may be written
\begin{equation*}
    - \frac{n_pR}{\tau_p}
    + n_p  \bm\delta : (\textbf{B}+\textbf{D})
    = 
    \int_0^\infty r^2 \int_{0}^{\infty}\oint_{\mathbb{R}^2}\left[
        \delta(a)P(\textbf{x},\textbf{r},0,t)
    - \frac{P_\text{nst}(\textbf{x},\textbf{r},t,a)}{\tau^\text{nst}(\textbf{x},\textbf{r},t,a)}
    \right]dS(r) da dr.
\end{equation*}
Taking \ref{eq:int_dt_h} into account leads directly to $- \frac{n_pR}{\tau_p} + n_p  \bm\delta : (\textbf{B}+\textbf{D}) = 0$. 
In other worlds, the transport equation of $R(\textbf{x},t)$ reads 
\begin{equation}
    \pddt (n_pR)
    + \pddx \cdot [n_p(\textbf{u}_pR
    + \bm\delta : \textbf{R}^\text{Re})]
    = 
    n_p\bm\delta : \textbf{W}. 
    \label{eq:dt_deltaR2}
\end{equation}
Which simply states that the mean square distance to the nearest neighbor $R$, evolves according to the relative velocity-position correlation.
Therefore, in opposition to $\textbf{R}(\textbf{x},t)$, the evolution of $R(\textbf{x},t)$ is completely determined by the unknown scalar $W(\textbf{x},t)$. 


In a homogeneous and steady-state regime all averaged quantities are constant with the \textbf{x} and $t$.   
Thus, from \ref{eq:dt_R} and \ref{eq:dt_deltaR2} we deduce that in this situation the following relation holds, 
\begin{align}
    W = \frac{2}{n_p(\textbf{x},t)}
    \int_{0}^\infty
    \int_{\mathbb{R}^3} 
        \textbf{r} \cdot \textbf{w}^\text{nst}_p
    P_\text{nst}
    d\textbf{r}
    da = 0. 
    \label{eq:cdt_for_W}
\end{align}
The latter equation states that the resultant of the normal approach velocity between nearest neighbors is null. 
This can be seen as a constraint to be respected for the normal approach velocity field $\textbf{r}\cdot \textbf{w}_p^\text{nst}$. 



\subsection{Nearest neighbor interaction time}

Let us define the first moment of the age distribution,
\begin{equation}
    a_p(\textbf{x},t) 
    = \int_{0}^{\infty} a P_a(a|\textbf{x},t) da. 
    = \frac{1}{n_p(\textbf{x},t)}
    \int_{\mathbb{R}^3}
    \int_{0}^{\infty} 
    a P_\text{nst}(\textbf{x},\textbf{r},a,t) da d\textbf{r}. 
    \label{eq:a_p}
\end{equation}
Based on the definition of the \textit{age}, we can state that the mean \textit{age} $a_p$ is the definition of the mean particle interaction time between nearest neighbors.  
In the aim to show the relation between the particle mean interaction time and the rate of destruction or birth of particles let us derive an equation for $a_p$. 
This is done by multiplying \ref{eq:dt_Pnst} with $a$ and integrating overall age and relative position, this yields
\begin{equation}
    \pddt (n_p a_p)
    + \pddx \cdot  (n_p a_p \textbf{u}_p + n_p a_p^{Re})
    = 
    n_p\left(1
    - \frac{a_p}{\tau_p}\right)
    - \int_{\mathbb{R}} a \left[
        \frac{1}{\tau_p}
        - \frac{1}{\tau_p^a}
    \right]P_a(a|\textbf{x},t) da,
    \label{eq:dt_a_p}
\end{equation}
where we have defined, the mean rate of destruction for pairs of age a, 
\begin{equation*}
    1/\tau_p^a(\textbf{x},t,a) 
    = \frac{1}{n_p(\textbf{x},t)}\int_{\mathbb{R}^3} \frac{P_\text{nst}(\textbf{x},\textbf{r},t,a)}{\tau^\text{nst}(\textbf{x},\textbf{r},t,a)} d\textbf{r}.
\end{equation*}
and the covariance term : $a^{Re}_p = \int_{0}^\infty\int_{\mathbb{R}^3} a (\textbf{u}^\text{nst}_p - \textbf{u}_p) P(\textbf{x},\textbf{r},t,a) d\textbf{r}da$. 
This equation describes the evolution of the particle mean interaction time within time and space. 

In steady state and homogeneous regime, this equation gives us the relation between the mean age of interaction $a_p$ and the mean rate of destruction $\tau_p$. 
Indeed, when the LHS of \ref{eq:dt_a_p} vanish we obtain the relation,
\begin{equation}
    a_p
    = \tau_p \left[
        1
        + 
        \int_{\mathbb{R}} a\left(
            \frac{1}{\tau_p(\textbf{x},t)} 
            - \frac{1}{\tau_p^a(\textbf{x},t,a)} 
        \right) P_a(a|\textbf{x},t) da
    \right]. 
    \label{eq:a_p2}
\end{equation}
The first term is simply the inverse of the rate of destruction and the section is the covariance term between the age and the rate of destruction. 

Following \citep{zhang2023evolution} we assume that the probability of particle pairs destruction averaged on all positions, i.e. $1/\tau_p^a(\textbf{x},t,a)$, is uncorrelated with the age of interaction. 
In other words, we consider that any pairs of nearest neighbors can be broken apart equally likely regardless of their current age of interaction, or that $\tau^a_p(\textbf{x},t,a) = \tau_p(\textbf{x},t)$.
Under this assumption, the second term of \ref{eq:a_p2} vanishes. 
We are left with
\begin{equation}
    a_p = \tau_p. 
    \label{eq:a_p_eq_tau_p}
\end{equation}
Consequently, within the random destruction assumption $\textbf{R}(\textbf{x},t)$ has a relaxation time equal to the first moment of the age distribution $P_a(a|\textbf{x},t)$. 
Thus, from \ref{eq:dt_R} and \ref{eq:Pa} we can state that the microstructure reaches a statistically steady-state equilibrium after a time of the order of the mean particle interaction time $\tau_p$. 
Additionally,  under this assumption, which will be shown to be valid in our parameters' range, we can derive an analytical formula for the age distribution \citep{zhang2023evolution}. 
It is done simply by integrating \ref{eq:dt_Pnst} over all age, making use of \ref{eq:norm} and using $\tau_p^a = \tau_p = a_p$ yields
\begin{align}
    P_a(a|\textbf{x},t)  
    =\frac{e^{-a/a_p(\textbf{x},t)}}{a_p(\textbf{x},t)}.
    \label{eq:Pa}
\end{align} 
We can see that the age distribution decrease monotonically with age. 
This means that the probability of finding nearest neighbors of an \textit{age} superior to $a_{max}$ can be arbitrarily low for an arbitrarily large $a_{max}$. 

In summary, $a_p$ is an interesting quantity by itself since it quantifies the mean time of interaction between two nearest particle pairs.
On the other side, $\tau_p$ quantifies the rate of destruction (or rate of birth) within the suspension, as discussed earlier this is also part of the microstructure relaxation time. 
Finally, under the random destruction assumption \citep{zhang2023evolution}, which will be shown to be true in our DNS, $a_p  =\tau_p$, therefore by measuring the mean age $a_p$ we directly obtain the microstructure relaxation time. 
 



% Objective of this section :
% \begin{itemize}
    % \item Introduce the dimensionless parameters.
    % \item Present the physical parameters of some industrial processes to locate our problematic. 
    % \item Introduce the dimensionless parameters range investigated in this study.
    % \item Present the tri-periodic box within which we add droplets in vof 
% \end{itemize}
\tb{re-écrire ou mettre la partie corrigé}
We investigate numerically the dynamic of homogeneous mono-disperse emulsion subject to buoyancy forces in a fully periodic domain. 
Both, the dispersed (resp. continuous) phase is considered as Newtonian fluid defined by viscosity $\mu_d$ (resp. $\mu_f$), and density $\rho_d$ (resp. $\mu_f$).
Throughout this work, the subscript $_d$ and $_f$ indicate properties belonging to the dispersed and continuous phase, respectively. 
The interface between both fluids is considered as infinitely thin and deprived of any impurities, it is described with the surface tension coefficient $\gamma$. 
In this work the density, viscosity, and surface tension coefficient, will be considered constant in each phase.
In dimensionless form this problem is completely characterized by $6$ dimensionless parameters :  the viscosity and density ratio, $\lambda = \mu_d / \mu_f$ and $\zeta = \rho_d / \rho_f$,  
the \textit{Galileo} number, 
\begin{equation*}
    Ga =\sqrt{\rho_f(\rho_f - \rho_d) g d^3} / \mu_f,
\end{equation*}
the \textit{Bond} number, 
\begin{equation*}
    Bo =\frac{(\rho_f - \rho_d) g d^2}{\gamma},
\end{equation*}
the number of droplets per domain $N_b$, and the dispersed phase volume fraction $\phi$. 
Here, $d$ represents the diameter of a sphere with the same volume as the droplets, and $g$ denotes the gravitational acceleration.
The \textit{Galileo} number measure the influence of the buoyancy forces against the viscous forces.
Whereas the \textit{Bond} number evaluate the ratio between buoyancy and capillary forces. 
% These parameters are the input parameters that we set at the beginning of a simulation


The primary objective of the study is to investigate the microstructure through the nearest particle pair distribution function.
Thus, it is crucial to obtain a sufficient number of DNS samples to ensure representativeness. 
Additionally, the physical quantities measured in these DNS must remain independent of the domain size. 
Therefore, we use a number of particles per domain of $N_b = 125$, which is roughly what \citet{hidman2023assessing} used for DNS of fully periodic buoyant rising bubbles.
Additionally, each DNS lasts for a time : $t^*_\text{end} = 1500 \sqrt{d/g}$.
It is shown in \ref{ap:validation} that these parameters  are sufficient to obtain well converged statistics.  
\begin{table}[h!]
    \centering
    \caption{Dimensionless parameter range investigated in this work.}
    \begin{tabular}{|ccccccc|ccc|}\hline
        \multicolumn{7}{|c|}{Primary parameters}&\multicolumn{3}{|c|}{Secondary parameters}\\\hline\hline
        $Ga$&$Bo$&$\phi$&$\lambda$&$\zeta$&$N_b$&$t^*_\text{end}$&$\mathcal{L}/d$&$Re$&$We$\\ \hline
        $5\rightarrow 100$&$0.2$&$1\% \rightarrow 20\%$&$10$ \& $1$&$0.9$&$125$&$1500$&$6.7\to 18.7$&$10^{-1}\to 170$&$10^{-4}\to 0.6$\\ \hline
    \end{tabular}
    \label{tab:simulations}
\end{table}
In this study we present DNS results with dimensionless parameters lying in ranges outlined \ref{tab:simulations}.
In summary, we investigated $5$ \textit{Galileo} number $Ga = 5,10,25,50,100$, $4$ different volume fractions $\phi = 0.01,0.05,0.1,0.2$, and two viscosity ratios $\lambda =1,10$ with $Bo = 0.2$ and $\zeta = 0.9$. 
This makes a total of $40$ representative simulations of $N_b = 125$ droplets for $t= 1500 \sqrt{d/g}$ time. 

\section{Numerical methodology}
\label{sec:methodo2}

The DNS results presented in this study are based on the same set of data as our previous work \citet{fintzi2024buoyancy} and \ref{chap:DNS}, where we conducted statistically steady-state simulations of rising mono-disperse suspensions of droplets in a fully periodic domain. 
This section 
%recalls briefly the numerical methods presented in \citet{fintzi2024buoyancy}, with 
presents additional information regarding the post-processing computation of statistics employed here. 
Note that the source code of the DNS and the post-processing tools used in this section are provided and documented on the wiki page: \href{http://basilisk.fr/sandbox/fintzin/Rising-suspension/RS.c}{RS.c}. 



% \subsection{Problem statement and numerical methods}


% We consider the dynamics of homogeneous mono-disperse emulsions subject to buoyancy forces in a fully periodic domain. 
% Both the dispersed and continuous phase are considered as Newtonian fluids defined by viscosity $\mu_d$ (resp. $\mu_f$) and density $\rho_d$ (resp. $\mu_f$).
% The interface between both fluids is modeled with a constant surface tension coefficient, $\gamma$. 
% Equally, the density and viscosity will be considered constant in each phase.
% In dimensionless form this problem is completely characterized by six dimensionless parameters:  the viscosity and density ratio, $\lambda = \mu_d / \mu_f$ and $\zeta = \rho_d / \rho_f$,  
% the \textit{Galileo} number, 
% \begin{equation*}
%     Ga =\sqrt{\rho_f(\rho_f - \rho_d) g d^3} / \mu_f,
% \end{equation*}
% the \textit{Bond} number, 
% \begin{equation*}
%     Bo =\frac{(\rho_f - \rho_d) g d^2}{\gamma},
% \end{equation*}
% the number of droplets per domain $N_b$, and the dispersed phase volume fraction $\phi$. 
% Here, $d$ represents the diameter of a sphere with the same volume as the droplets, and $g$ denotes the gravitational acceleration.
% The \textit{Galileo} number is a measure of the strength of the buoyancy forces relative to the viscous forces, whereas the \textit{Bond} number evaluates the ratio between buoyancy and capillary forces. 
% We used a number of droplets per domain of $N_b = 125$, which is roughly what \citet{hidman2023assessing} used for their DNS of fully-periodic buoyant rising bubbles.
% Moreover, each DNS lasts for a time: $t^*_\text{end} = 1500 \sqrt{d/g}$.
% These parameters are sufficient for to obtain a good statistical convergence \citet{fintzi2024buoyancy}. 
% \begin{table}[h!]
%     \centering
%     \caption{Dimensionless parameter ranges investigated in this work.}
%     \begin{tabular}{|ccccccc|ccc|}\hline
%         \multicolumn{7}{|c|}{Primary parameters}&\multicolumn{3}{|c|}{Secondary parameters}\\\hline\hline
%         $Ga$&$Bo$&$\phi$&$\lambda$&$\zeta$&$N_b$&$t^*_\text{end}$&$\mathcal{L}/d$&$Re$&$We$\\ \hline
%         $5\rightarrow 100$&$0.2$&$1\% \rightarrow 20\%$&$10$ \& $1$&$0.9$&$125$&$1500$&$6.7\to 18.7$&$10^{-1}\to 170$&$10^{-4}\to 0.6$\\ \hline
%     \end{tabular}
%     \label{tab:simulations}
% \end{table}
% In short, we present DNS with dimensionless parameters lying in ranges outlined in \ref{tab:simulations}.

% The governing equations describe the motion of two immiscible fluids of different densities and viscosities separated by an interface with surface tension. 
% We use the so-called ``one fluid'' formulation of the variable density and viscosity Navier-Stokes equations, which can be expressed as 
% \begin{align}
%     \pddt \rho+ \div(\rho\textbf{u})
%     &= 0,\\
%     \label{eq:dt_urho}
%     \pddt (\rho \textbf{u})
%     + \div (\rho  \textbf{u} \textbf{u} - \bm\sigma)
%     &= (\avg{\rho} - \rho)\textbf{g}
%     + \textbf{f}_\gamma,\\
%     \label{eq:dt_C}
%     \pddt C + \textbf{u}\cdot\grad C  
%     &= 0,
% \end{align}
% i.e. the mass, momentum, and colors function transport equations, respectively. 
% The scalar field $C$, represents the color function, which ranges between $0$ and $1$ to indicate the volumetric proportion of each phase.
% We introduced the fluid velocity vector $\textbf{u}$ and the Newtonian stress tensor $\bm{\sigma} = -p \textbf{I} + \mu (\grad \textbf{u}+ \grad \textbf{u}^\dagger)$ where $p$ is the pressure field and $^\dagger$ represents the transpose operator.
% Note that the material properties, $\rho$, and $\mu$, take the values of each phase in presence, using the arithmetic average: $\rho = (1-C)\rho_f + C \rho_d$ and $\mu = (1-C)\mu_f + C \mu_d$. 
% The capillary force is defined as, $\textbf{f}_\gamma =\textbf{n} \gamma \div \textbf{n} $, where \textbf{n} is the normal to the interface.
% Following  \citep{bunner2002dynamics}, we incorporated the artificial body force term, $\avg{\rho}\textbf{g}$, on the right-hand side of \ref{eq:dt_urho}, to maintain a zero-averaged velocity throughout the entire numerical domain.  

% To solve these equations we first initialized $125$ spherical droplets within a cubic domain with fully periodic boundary conditions. 
% We used the open source code \url{http///basilisk.fr} to discretize the governing equations. 
% Additionally, we used a novel algorithm (discussed in detail in \citep{mani2021numerical,fintzi2024buoyancy}) to prevent coalescence between droplets so that we preserve a mono-disperse population of droplets. 
% This algorithm is based on the Mult-VoF methodology. 
% The key principle is to use different VoF tracers for adjacent droplets, but allow several non-touching droplets to be in the same VoF tracer.
% This algorithm is shown to preserve the kinematics   behavior of interfaces that are in close contact, even though we do not resolve the thin liquid film between them \citep{fintzi2024buoyancy}. 
% Note that this is of major importance since this study focuses on the relative velocity of droplets in interaction. 

% \subsection{Numerical sampling and statistical convergence}

The age-included nearest-neighbor distribution function presented in the previous section requires the droplets' center of mass position and the position of its nearest neighbor, along with the current age of interaction.
% Equally, the center of mass velocity of the droplets together with the velocity of their nearest neighbor is also needed to reconstruct $\textbf{w}_p^{nst}$. 
% Indeed, when several droplets belong to the same VoF tracer it is challenging to distinguish them.
Therefore, we now provide additional comments on the numerical methodology employed to reconstruct these statistics. 

% As detailed in \citet{fintzi2024buoyancy} the \href{http://basilisk.fr/src/tag.h}{tag.h} algorithm permits us to distinguish the different droplet volumes within a given VoF field. 
% Let define $\texttt{t}_i[\text{c},t]$ as the ``tag'' field corresponding to the droplet $i$, evaluated at the cell ``c'' of the mesh and at time $t$.
% This field is equal to 1 for cells inside the volume of the droplet $i$ and 0 outside.
% Then, the \href{http://basilisk.fr/src/tag.h}{tag.h} algorithm provides us with $N$ ``tag'' field, i.e. one for each droplet. 
% Based on $\texttt{t}_i$, and on the other fields solved by \ref{eq:dt_C} and \ref{eq:dt_urho}, we can easily define the droplets Lagrangian properties mentioned above. 
% The mass, position, and momentum of a droplet are then computed as follows 
% \begin{align*}
%     m_i(t)
%     = \sum_{\text{c} = 1}^\text{last cell}
%     \rho [\text{c},t]
%     \texttt{t}_i[\text{c},t]dv[c],
%     \\ 
%     \textbf{x}_i(t)
%     = 
%     \frac{1}{m_i(t)}
%     \sum_{\text{c} = 1}^\text{last cell}
%     \textbf{x} 
%     \rho [\text{c},t]
%     \texttt{t}_i[\text{c},t]dv[c],
%     \\ 
%     \textbf{u}_i(t)
%     = 
%     \frac{1}{m_i(t)}
%     \sum_{\text{c} = 1}^\text{last cell}
%     \textbf{u} [\text{c},t]
%     \rho [\text{c},t]
%     \texttt{t}_i[\text{c},t]dv[c],
% \end{align*}
% where the sum from ``$\text{c} = 1$'' to ``$\text{last cell}$'' denotes a summation over the entire mesh grid, and $\rho[\text{c},t]$ and $\textbf{u}[\text{c},t]$ are the density field and velocity field evaluated at the cell ``c'' and at a simulation time step $t$. 
% Likewise, $\textbf{x}[\text{c}]$ and $dv[c]$ correspond to the Cartesian coordinate and the volume of the cell, respectively. 

For a given time step $t_n$, we compute the position $\textbf{x}_i(t_n)$ and velocity $\textbf{u}_i(t_n)$ of each droplet, as described in \ref{chap:DNS}. 
Then, by measuring each droplet pair distance, we identify the nearest neighbor for each droplet $i$. 
For convenience, the nearest neighbor of the droplet $i$ is indexed $j$. 
Once this is done we can compute the relative positions $\textbf{r}_{ij}(n) =\textbf{x}_i - \textbf{x}_j$, and the relative velocities $\textbf{w}_{ij} = \textbf{u}_i - \textbf{u}_j$ with the nearest neighbor for all droplets. 
At the simulation time step $t_{n+1}$ we repeat this procedure and compute all the $\textbf{x}_i(t_{n+1}),\;\textbf{u}_i(t_{n+1}),\;\textbf{r}_{ij}(t_{n+1}),\;\textbf{w}_{ij}(t_{n+1})$. 
If the nearest neighbor of the droplet $i$ at $t = t_{n+1}$, is still the same as at $t = t_n$, we increment the \textit{age} of interaction as $a_i = a_i + (t_{n+1} - t_n)$, where $a_i$ is the \textit{age} of the droplet $i$. 
Inversely, if the droplet $j$ of time $t= t_{n+1}$ is not the same as the one of time $t=t_n$ the \textit{age} is set back to zero. 
Note that at $t = 0$ all ages are set to zero. 


To reconstruct all these statistics from the DNS, we treat each simulation time step and droplets as an independent event, denoted as, $\FF$. 
As an example, to reconstruct a quantity such as $\textbf{w}^\text{nst}_p(\textbf{x},\textbf{r},t,a)$ we gather the relative velocities $\textbf{w}_{ij}(t)$ from the simulation and store them in $i$ intervals of ages $\Delta a_k$, and relative positions $\Delta \textbf{r}_k$ for $k = 1,\ldots, n$.
Then, we average on all the $\textbf{w}_{ij}(t;\FF)$ over each droplet and time step.
Formally, we write, 
\begin{equation}
    \textbf{w}^\text{nst}_p(\Delta\textbf{r}_k,\Delta a_k)
    = \frac{1}{E_k} 
    \sum^{E_k}_{\FF_k} 
    % \sum_i^{N_b}
    % \sum_{j\neq i}^{N_b}
    \textbf{w}_{ij}(t;\FF_k)
    % h_{ij}
    % \text{\;\;  with  \;\;}
    % \FF_k = \{\FF; \textbf{r}(\FF)\in\Delta \textbf{r}_k, a(\FF)\in  \Delta a_k\}
    \label{eq:vec_cond}
\end{equation}
where $\FF_k$ correspond to the events were the nearest droplet pair $i$ and $j$ follows $\textbf{r},a \in \Delta \textbf{r}_k ,\Delta a_k$.
$E_k$ is the total number of events fulfilling these constraints. 
% Thus, we obtained an approximation of $\textbf{w}^\text{nst}_p(\textbf{x},\textbf{r},t,a)$ which takes the intervals, $(\Delta\textbf{r}_i,\Delta a_i)$ as arguments.
At some point, it will be useful to study the averaged relative velocity, conditioned on the age or the radial distance, independently. 
Therefore, in a more general way if $p$ is a scalar property with $\Delta p_k$ its $n$ intervals, we can define the $p$-conditionally averaged relative velocity as, 
\begin{equation}
    \textbf{w}^\text{nst}_p(\Delta p_k)
    = \frac{1}{E_{k}} 
    \sum^{E_{k}}_{\FF_{k}}  
    % \sum_i^{N_b} 
    % \sum_{j\neq i}^{N_b}
    \textbf{w}_{ij}(t;\FF_k)
    % h_{ij}
    % \text{\;\;  with  \;\;}
    % \FF_k = \{\FF; p(\FF)\in\Delta p_k\}
    \label{eq:scalar_cond}
\end{equation}
In this definition, $\FF_{k}$ corresponds to all the events where the nearest neighbors respect the condition, $p \in \Delta p_k$, and $E_{k}$ represents the total number of events fulfilling this constraint. 

Note that due to the average over the macroscopic variables, $\textbf{w}_p^{nst}(\Delta\textbf{r}_k,\Delta a_k)$ and $\textbf{w}^\text{nst}_p(\Delta p_k)$ are not dependent on $\textbf{x}$ and $t$. 
In general, this will be the case for all quantities presented in \ref{sec:results}. 
Therefore, from now on we drop the \textbf{x} and $t$ arguments. 


It is clear that to obtain representative averaged quantities, the number of events $E_k$ per interval must be consequent. 
For 2D-conditioned quantities such as \ref{eq:vec_cond}, we estimated that $100$ samples per bin were sufficient to obtain good qualitative results. 
Consequently, the plots displayed in \ref{sec:results} have been generated by averaging the quantity of interest (velocity fields, age of interaction) with a minimum threshold of $E_k = 100$ samples for each bin.
Regarding the scalar-conditioned fields such as in \ref{eq:scalar_cond}, we gathered $E_k = 1000$ sample per bin to obtain accurate and quantitative results. 
Lastly, regarding the sampling, we collected data for each Lagrangian quantity every $10$ simulation time step.
On average, $200,000$ time steps are performed during a simulation with $N_b = 125$ droplets. 
This results in a total number of $E = 2,500,000$ samples. 
In \ref{ap:validation}, we demonstrate that our statistics are well-converged.


We would like to present what we think is the simplest way to solve the averaged equations for a mixture of two incompressible fluids.
In the first place, we present the ``single-phase-like'' mixture equations, which seem to be the most efficient way to solve the mixture equations. 
Then we incorporate the closure terms presented in this manuscript, inside this system of equation.

Our method is similar in many ways to the so-called ``drift flux'' models. 
These models focus on solving the averaged mixture-phases equations (i.e. solving for $\rho$ and $\textbf{u}_m$) and for the relative velocity between the particles and the mixture velocity $\textbf{u}- \textbf{u}_p$, which is termed the drift fluxes \citep{ishii1977one,guazzelli2011}.  
More recently,  \citet{kriaa2023two} used this formulation to model sediment transport with \texttt{Basilisk}.
Actually, \citet{kriaa2023two} used an approximation of this formulation, assuming that the ``bulk'' velocity $\textbf{u}$ satisfies Navier-Stokes equations.
The particles are accounted  only through a forcing term related to the weight of the particle in the NS equations. 
It this turns out to be valid under certain circumstances  according to Boussinesq, approximation. 

This derivation leads to an efficient formulation similar to the one of \citet{kriaa2023two}, but without any approximation or hypothesis.  
As it will be demonstrated due to the divergence-free property of \textbf{u} this formulation is particularly interesting to use (even more for one-dimensional model). 



\subsection{A ``single phase incompressible''-like formulation for the volume-averaged mixture velocity}

As described in \ref{chap:closure-disperse}, \ref{chap:daniel2} and \ref{chap:pseudoturbulence} by adding the local conservative form of the mass and momentum equations we can derive a conservation equation for the volumetric velocity $\textbf{u} =\phi_d\textbf{u}_d + \phi_f \textbf{u}_f$. 
At the local non-averaged scale we have: 
\begin{align}
    \label{eq:NSmass0}
    \div \textbf{u}^0 &= 0, \\
    (\pddt 
    + \textbf{u}^0 \cdot \grad) \textbf{u}^0
    &= 
    \div \bm\sigma^*
    +\textbf{g}
    +(\kappa/\rho_f)(\bm\sigma_f^0\cdot \textbf{n})\delta_\Gamma,
    \label{eq:NSmomentum0}
\end{align}
Where we noted $\textbf{u}^0 = \chi_f \textbf{u}f^0 + \chi_d \textbf{u}_d^0$, $\bm\sigma^* = \chi_f \bm\sigma_f^0/\rho_f  + \chi_d \bm\sigma_d^0/\rho_d + \delta_\Gamma \bm\sigma_\Gamma^0/\rho_d $ referred as the density-weighted stress, $\kappa = (1-\zeta)/\zeta$ and $\zeta$ the density ratio and the equivalent stress defined as.
Recall that $\bm\sigma_{f,d}^0 = -p_{f,d}\bm\delta + \mu_{f,d}\left[\grad \textbf{u}_{f,d}^0+ (\grad \textbf{u}_{f,d}^0)^\dagger\right]$ and $\bm\sigma_\Gamma^0 = \gamma (\bm\delta - \textbf{nn})$ represents the surface tension stress.

Averaging \ref{eq:NSmass0} and \ref{eq:NSmomentum0}, yields the averaged volume and acceleration conservation of the mixture phase, 
\begin{align}
    \div\textbf{u} &=0,\\
    (\pddt \textbf{u}  
    + \textbf{u} \cdot \grad )
    \textbf{u}
    &= 
    \div \bm{\sigma}^\text{eq} + 
    \textbf{g} 
    + (\kappa/\rho_f) \avg{\delta_\Gamma \bm{\sigma}_f^0 \cdot \textbf{n}},
    \label{eq:NS_firststep}
\end{align}
respectively. 
\begin{equation}
    \bm\sigma^\text{eq} = 
    - \avg{ \textbf{u}'\textbf{u}'}
    + \frac{\phi_f}{\rho_f}\bm\sigma_f
    + \frac{\phi_d}{\rho_d}\bm\sigma_d
    + \frac{\phi_\Gamma}{\rho_d} \bm\sigma_\Gamma. 
\end{equation}
At this stage, absolutely no assumptions have been made on the nature of the phases nor the flow regime, yet the velocity fields $\textbf{u}$ is divergence-free and possesses an advective term similar to the one encountered in the classic single-phase incompressible NS equations.
This expression can be the starting point to describe a $50\%$ $50\%$ mixture of fluid, that is not necessarily a dispersed multiphase flow. 

If one assumes that the phase ``f'' is dominant in the mixture, he can reformulate the effective stress $\bm\sigma_\text{eff}$ as a macroscopic Newtonian stress, written as $\Sigma = -  p_f \bm\delta  + \mu_f(\grad \textbf{u} + \grad \textbf{u}^\dagger)$, plus another effective stress related to the presence of the phase $d$.
Indeed, as demonstrated in \ref{chap:closure-disperse,chap:daniel2,chap:pseudoturbulence} we can write,
\begin{equation}
    \phi_f\bm\sigma_f
    = \bm\Sigma
    + \phi_d (p_f\bm\delta - 2\mu_f \textbf{e}_d)
    \label{eq:stress_formulation}
\end{equation}
Injecting \ref{eq:stress_formulation} into \ref{eq:NS_firststep} leads to, 
\begin{align}
    \label{eq:NS_not_dispersed_mass}
    \div\textbf{u}&=0, \\
    \rho_f (\pddt 
    + \textbf{u}\cdot \grad)
    \textbf{u}
    &= 
    \div( \bm\Sigma
    + \bm{\sigma}_\text{eff} )
    + \kappa \avg{\delta_\Gamma \bm{\sigma}_f^0 \cdot \textbf{n}} 
    + \rho_f \textbf{g} 
    \label{eq:NS_not_dispersed}
\end{align}
with the effective stress, 
\begin{equation}
    \bm\sigma_\text{eff} /\rho_f = 
    - \avg{ \textbf{u}'\textbf{u}'}
    + \frac{\phi_d}{\rho_f} (p_f\bm\delta - 2\mu_f \textbf{e}_d)
    + \frac{\phi_d}{\rho_d}\bm\sigma_d
    + \frac{\phi_\Gamma}{\rho_d} \bm\sigma_\Gamma. 
\end{equation}
As can be seen from \ref{eq:NS_not_dispersed}, the equation governing the velocity \textbf{u} is nothing else but the Navier-Stokes equations with fluid properties $\rho_f$ and $\mu_f$, and additional forcing terms related to the presence of the dispersed phase. 
At this stage note that we have not made any assumption on the topology of the dispersed phase. 

Let us consider that the phase "d" is a dispersed phase made of solid particles or droplets with yet arbitrary shapes. 
Upon reformulating the distribution $\chi_d$ and $\delta_\Gamma$ in terms of Dirac distributions (see \ref{chap:daniel1}) we obtain, 
\begin{align}
    \avg{\delta_I \bm{\sigma}_f^0 \cdot \textbf{n}} 
    &= 
    \pSavg{\bm{\sigma}_f^0 \cdot \textbf{n}} 
    -\div \pSavg{\textbf{r}\bm{\sigma}_f^0 \cdot \textbf{n}} 
    + \frac{1}{2}\div \pSavg{\textbf{r}\bm{\sigma}_f^0 \cdot \textbf{n}} \\
    \avg{ \textbf{u}'\textbf{u}'}
    &= 
    \avg{\chi_f \textbf{u}^0_f\textbf{u}^0_f}
    + \avg{\chi_d \textbf{u}^0_d\textbf{u}^0_d}
    - \textbf{uu}\\
    \avg{\chi_d \textbf{u}^0_d\textbf{u}^0_d}
    % &= 
    % \pOavg{\textbf{u}^0_d\textbf{u}^0_d}
    % -\div \pOavg{\textbf{r}\textbf{u}^0_d\textbf{u}^0_d}\\
    &= 
    v_p\pavg{\textbf{u}_\alpha\textbf{u}_\alpha}
    % \pOavg{\textbf{u}_\alpha\textbf{w}^0_d}
    % + \pOavg{\textbf{w}_d^0\textbf{u}_\alpha}
    + \pOavg{\textbf{w}_d^0\textbf{w}^0_d} \ldots
    -\div \pOavg{\textbf{r}\textbf{u}^0_d\textbf{u}^0_d}\\
    \textbf{uu} 
    % &= 
    % [\textbf{u}_f - \textbf{u}_f\phi_d + n_p v_p \textbf{u}_p - \div (n_p \textbf{P}_p)] 
    % [\textbf{u}_f\phi_f + n_p v_p \textbf{u}_p - \div (n_p \textbf{P}_p)] 
    % [\textbf{u}_f + \phi_d( \textbf{u}_d - \textbf{u}_f)] \\
    % &= 
    % \phi_f \textbf{u}_f \textbf{u}_f
    % + \phi_d [\textbf{u}_d \textbf{u}_f
    % + \phi_f  \textbf{u}_f ( \textbf{u}_d - \textbf{u}_f)
    % +  \textbf{u}_d \phi_d( \textbf{u}_d - \textbf{u}_f) ]\\
    &= 
    \textbf{uu} 
    + \phi_f \textbf{u}_f \textbf{u}_f
    - \phi_f \textbf{u}_f \textbf{u}_f
    + n_pv_p\textbf{u}_p \textbf{u}_p
    - n_pv_p\textbf{u}_p \textbf{u}_p
    % \\
    % \phi_d \textbf{u}_d \textbf{u}_d
    % &= 
    % [n_pv_p \textbf{u}_p- \div (\textbf{P}_p n_p)][\textbf{u}_p - \frac{1}{n_pv_p} \div \textbf{P}_p n_p ]\\
    % &\approx 
    % n_pv_p\textbf{u}_p\textbf{u}_p 
    % - \textbf{u}_p \div (\textbf{P}_p n_p) 
    % - \div (\textbf{P}_p n_p)\textbf{u}_p  
\end{align}
The averaged dispersed phase stresses, $\bm\sigma_d$ and $\bm\sigma_\Gamma$, can be reformulated into particle volume and surface quantities using the moment of momentum equations, which are given in \ref{chap:daniel1} and \ref{chap:closure-disperse}. 
Including all these terms into \ref{eq:NS_not_dispersed_mass} and \ref{eq:NS_not_dispersed} leads to dispersed two-phase flow formulation of the mixture volume and acceleration equation of the bulk phase, namely, 
\begin{align}
    \div\textbf{u}&=0, \\
    \rho_f (\pddt 
    + \textbf{u}\cdot \grad)
    \textbf{u}
    &= 
    \div (\bm\Sigma +  \bm{\sigma}_\text{eff} )
    + \kappa \pSavg{ \bm{\sigma}_f^0 \cdot \textbf{n}} 
    + \rho_f\textbf{g} 
    \label{eq:NS_dispersed}
\end{align}
with the effective stress re-written as, 
\begin{multline}
    \bm\sigma_\text{eff}/\rho_f = 
    - \avg{ \chi_f \textbf{u}_f'\textbf{u}_f'}
    - v_p \pavg{ \textbf{u}_\alpha'\textbf{u}_\alpha'}
    + (\phi_f \textbf{u}_f \textbf{u}_f
        +n_pv_p\textbf{u}_p \textbf{u}_p
        - \textbf{uu} 
    )\\
    % +  \pOavg{ 
    %     \textbf{w}_d^0\textbf{w}_d^0  
    %     }
    -\frac{1}{2}\pavg{\frac{d^2 }{dt^2}\intO{\textbf{rr}}}
    +\frac{1}{\rho_f}\pSavg{ 
        (\bm{\sigma}_1^0 \textbf{r}
        + \textbf{r}\bm{\sigma}_1^0 )
        \cdot \textbf{n}
        % - \mu_f(\textbf{u}_f^0\textbf{n}
        % + \textbf{n}\textbf{u}_f^0)
    }
    + \frac{\phi_d}{\rho_f} (p_f\bm\delta -2\mu_f \textbf{e}_d)
    \\
    +\frac{1}{2}\div\left\{
    - \pavg{\ddt\intO{  (\textbf{u}_d^0)_i r_j r_k }}
    +\frac{1}{\rho_f}\pSavg{  (\bm{\sigma}_f^0 \cdot \textbf{n}_d)_i r_{j}  r_{k}  } 
    + \div[\ldots]
    \right\}
    \label{eq:effective_stress}
\end{multline}
Each term of the closure term present in the effective stress is discussed in \ref{chap:closure-disperse}.
To summarize, the terms on the right-hand side of \ref{eq:NS_dispersed} represent the total drag force, and the terms on the first line of \ref{eq:effective_stress} represent the stress generated due to the velocity fluctuations, either from the continuous phase, or from the particle center of mass. 
Note that the fluid phase velocity $\textbf{u}_f$ must be considered as a closure term as we solve the equations for $n_p$ and  $\textbf{u}$ (the relation between the velocity is given in \citet{zaepffel2012multisize}).
Regarding the terms on the second line of \ref{eq:effective_stress}, the first terms represent in contribution of the particle's internal shear and the fluid phase pressure.
Regarding the first terms of the second line and the first term on the third line, they represent the influence of deformation or changes in orientation of the particle. 
Thus, these terms vanish if one considers spherical non-deformable particles, they can be either solid particles or spherical bubbles or droplets. 
The second terms of the first and second lines represent the contribution from the first and second momentum of hydrodynamic forces. 


\subsection{Particle phase equations}

As seen in \ref{chap:daniel1,chap:daniel2} one can add to the continuous or mixture phase equations an arbitrary number of equations describing the dispersed phase. 
If one considers a spherical mono-disperse suspension of droplets, we may simply just use the number density, $n_p$, and particle momentum $n_pm_p \textbf{u}_p$, conservation equations. 

According to \ref{chap:daniel1,chap:daniel15} we may write, 
\begin{align}
    (\pddt  
    + \textbf{u} \cdot\grad )n_p
    &= 
    \div (n_p\textbf{u}_r)
    \label{eq:dt_np_final}
    \\
    (\pddt + \textbf{u}_p \cdot \grad)  \textbf{u}_p
    &= 
    \textbf{g}
    + \frac{1}{n_pm_p}\left\{
        \pSavg{\bm\sigma_f^0 \cdot \textbf{n}}
        - \div \pavg{\textbf{u}_\alpha'\textbf{u}_\alpha'}
    \right\}.
    \label{eq:dt_up_final}
\end{align}
Where $\textbf{u}_r = \textbf{u} - \textbf{u}_p$ is the drift velocity. 
Under this form \ref{eq:dt_np_final} can be interpreted as an advection of $n_p$ along the mixture phase velocity $\textbf{u}$, plus a source terms accounting for the non-zero $\textbf{u}_r$. 


% \subsection{Summary of the available closure developed in this manuscript. }

% Although all of the closre presented in this manuscript are valid on wideer or note regime it is good to make a brief summary, 

% \subsubsubsection{Stokes dilute regime}
% \begin{align}
%     \pavg{\intS{(\bm{\sigma}_f^0 \cdot \textbf{n}_d)_ir_k}} -
% 2\pSavg{{\mu(\textbf{e}_d^0)_{ik}}} 
% &= 
% \phi_d p_f\bm\delta
% - \frac{5\lambda +2}{\lambda +1}
% \textbf{E}_f \phi \mu_f
% \label{eq:first}
% \\
% \frac{1}{2}\pavg{\intS{(\bm{\sigma}_f^0 \cdot \textbf{n}_d)_ir_kr_l}} -
% 2\pSavg{{\mu(\textbf{e}_d^0)_{ik} r_l}} 
% &= 
% \frac{\mu_f\phi_d}{2(\lambda +1) }
% \left[
%     \frac{3\lambda}{2} 
%     u_{fp,i}\delta_{kl}
%     +  u_{fp,l}\delta_{ki}
% \right]. 
% \label{eq:second}
% \end{align}

% \subsubsubsection{Low inertial dilute regime}
% The foreces traction reeads as, 
% \begin{equation}
%     \frac{a}{\mu_f U}\pSavg{\textbf{r}\bm{\sigma}_f^0 \cdot \textbf{n}_d}
%     % - 2 \mu_f \pOavg{\textbf{e}_d^0}
%     =
%      \phi Re [C_1
%     \textbf{u}_{r}\textbf{u}_{r} 
%     + C_2 (\textbf{u}_{r}\cdot \textbf{u}_{r}) \bm\delta]
%     + \phi p_f \bm\delta
% \end{equation}
% \begin{equation}
%     \frac{a }{\mu_f U}\avg{\chi_f\rho_f \textbf{u}_f'\textbf{u}_f'} =  Re \phi [
%         C_{EE}^1 \textbf{E}_f\cdot \textbf{E}_f
%        + C_{EE}^2 (\textbf{E}_f : \textbf{E}_f)\bm\delta
%     ]
% \end{equation}


% \subsubsubsection{Arbitrary higher regime}
% with $C_d$ the drag devellooped in Chp 8

% \begin{align*}
%     \frac{1}{n_pm_p}\pSavg{\bm{\sigma}_f^0\cdot \textbf{n}_d} &= 
%     \frac{1}{\rho_d} \div\bm\Sigma
%     + \frac{3\mu_f}{4\rho_d} \left(\frac{\lambda}{\lambda+1}\right)\grad^2\textbf{u} 
%     + \frac{1}{\zeta} \frac{3}{4} C_D\frac{1}{\phi_f^2} |\textbf{u}- \textbf{u}_p|(\textbf{u}- \textbf{u}_p)\\
%     \avg{\chi_f \textbf{u}_f'\textbf{u}_f'}
%     &= C^{(1)}_e \left[
%         \textbf{e}_p\textbf{e}_p
%         + \frac{\pavg{\textbf{u}_\alpha'\textbf{u}_\alpha'}}{n_p U^2}
%          - \frac{1}{3}(\textbf{e}_p\cdot \textbf{e}_p+2k_p/U^2)\bm\delta
%     \right]\\
%     &+ C^{(2)}_e 
%     (\textbf{e}_p\cdot \textbf{e}_p+2k_p/U^2)\bm\delta \\
%     % &+ C_{EE}^1 \textbf{E}_f\cdot \textbf{E}_f
%     % + C_{EE}^2 (\textbf{E}_f : \textbf{E}_f)\bm\delta\\
%     \pavg{\textbf{u}_\alpha'\textbf{u}_\alpha'} &= ?
% \end{align*}


% but also velocity gradient $\grad \textbf{u}$  and $\textbf{u}_p$

\subsection{General conclusion on the closure problem}

According to the general formulation of the closure problem it is clear that all closure terms present in the averaged two phase flow equations can be expressed as a function of the tensors $\textbf{u}_{r} =\textbf{u}- \textbf{u}_p$, $\grad \textbf{u}$, $\grad^2 \textbf{u}$, the dispersed phase volume fraction $\phi =\phi_d$, the viscosity ratio $\lambda$ and  $Re$. 


In the Stokes regime, due to the linearity of the Stokes equations, the disturbance fields must be proportional to $\textbf{u}_{r}$, $\grad \textbf{u}$, $\grad^2 \textbf{u}$, or the mean pressure of the fluid $p_f$. 
Therefore, based on this argument we might argue that the drag force term, which is a vector, must have the following functional form, 
\begin{align}
    % \frac{L^2}{\mu_f U_M} 
    \pSavg{\bm\sigma_f\cdot \textbf{n}} &= 
    \phi \{
    \div \bm\Sigma
    + 
    % (L^2/a^2) (U/U_M)  
    \mu_f a^{-2}
    C_u \textbf{u}_{r}
    +\mu_f C_{[\grad^2 u]} \grad^2 \textbf{u}
    \} + \ldots
    \label{eq:darg_force_general}
\end{align}
since $\div \bm\Sigma$, $\sim \textbf{u}_r$ and  $\sim \grad^2 \textbf{u}$ form a vector proportional to the velocity, in opposition to $\grad \textbf{u}$ which form a second order tensor, hence it cannot be part of the closure. 
% Note that the velocity scale have been made dimensionless using $U$ on the left-hand side of the equaiton. 
In \ref{eq:darg_force_general} we recognize that the dimensionless constant $C_u$ is related to the drag force coefficient developed in \ref{chap:mono-disperse}, hence $C_u$ is a function of $Re$, $\lambda$ and $\phi$.
We remark the presence of the radius of the particle $a^{-2}$ in front of $\textbf{u}_r$ so that $C_u$ is indeed a dimensionless variable. 
The constant $C_{[\grad^2 u]}$ correspond to the Fax\'en contribution, its closure is given in \ref{chap:daniel2}.
While, \ref{eq:darg_force_general} can apply to various situations by including $Re$, $\lambda$, and $\phi$ into those constants, it is still incomplete. 
Indeed, still considering Stokes regime, one might argue, according to the general form of the closure problem (see \ref{chap:daniel2}) that the vector, $\grad \phi$ must be included in the drag force formulation. 
We deduce that \ref{eq:darg_force_general} also includes the terms, 
\begin{equation}
    \ldots 
    +
    \phi C_{[E \grad\phi ]} \grad \textbf{u} \cdot \grad \phi 
    + \phi C_{[p_f \grad\phi ]} p_f \grad \phi 
    + \ldots
    \label{eq:drag_inhomo}
\end{equation}
Note the presence of the term $\phi \grad \phi$ and not simply $\grad \phi$. 
This is because in the absence of particles when $\phi=0$, we might still have gradient of volume fraction such that $\grad \phi \neq 0$ however, the force term must remain zero thus we must have $\phi\grad\phi$. 
Where  we remarked that the only vector linear with $\textbf{u}$ or $p_f$ (or their gradients), and $\grad \phi$, is given by $\grad \textbf{u} \cdot \grad \phi$ and $p_f \grad \phi$.
In \citet{wang1999longitudinal} they demonstrate analytically that the drift velocity of a particle under shear flow is of $\mathcal{O}(a^2 \grad \textbf{u}\cdot \grad \phi)$.
This means that \citet{wang1999longitudinal} founds a drag force term of  $\mathcal{O}( \grad \textbf{u} \cdot \phi \grad \phi)$ which implies that $C_{[E \grad\phi ]} \sim \mathcal{O}(1)$ at the leading order. 
As, stipulated in \citet{wang1999longitudinal,guazzelli2011} the physical interpretation of the first term of \ref{eq:drag_inhomo} is that this force is at the origin of the longitudinal drift velocity of a sheared dilute suspension of spheres. 
Regarding the pressure contribution \citet{du2020bubble} identified a term of the form $p_f \grad \phi$ and called it Laminar dispersion, this confirms that $C_{[p_f \grad\phi ]}\neq 0$. 

Of course, additional terms might appear if one considers higher gradients in the volume fraction and velocity field however we truncated the analysis to only the first-order derivative. 

Let us turn our attention to slightly inertial flows, or potential flows, the disturbance fields are now proportional to $\sim \textbf{uu}$ or $\sim \textbf{u}_r \textbf{u}_r$ and their derivatives. 
In the case of the drag force, there are numerous possibilities to construct a vector based on tensor formed of $\textbf{uu}$, $\textbf{u}_r\textbf{u}_r$, and $\phi$. 
Hence, here is a non-exhaustive list of the terms that must be added to the drag force closure, 
\begin{multline}
    % \frac{U^2 \rho^2}{L}
    \ldots
    + \phi \rho_f \{  C_{[E u]} \textbf{E} \cdot \textbf{u}_{r} 
    + C_{[\omega u]} \bm\omega_r \cdot \textbf{u}_{r} \}\\
    + \rho_f a^{-1}  \phi \grad\phi \cdot \{ C_{[uu\grad\phi]}^{(1)} \textbf{u}_{r}\textbf{u}_{r}
    +  C_{[uu\grad\phi]}^{(2)} (\textbf{u}_{r}\cdot \textbf{u}_{r})\bm\delta\} \\
    + \rho_f a  \phi \grad\phi \cdot \{ C_{[EE\grad\phi]}^{(1)} (\textbf{E} \cdot  \textbf{E}) 
    +  C_{[EE\grad\phi]}^{(2)} (\textbf{E} : \textbf{E})\bm\delta \}
\end{multline}
The term on the first lines represents the so-called Magnus effect which arises at finite inertia regime, where $\bm\omega_r = \bm\omega_p - \frac{1}{2}(\grad \textbf{u}-\grad \textbf{u}^\dagger)$ the relative rotation between the particles and the vorticity of the bulk. 
No study could be found reporting the values of the scalar $C_{[Eu]}$  and $C_{[\omega u]}$. 
The second term represents the collective contribution of inertia and non-homogeneity in the drag force. 
Such contributions have been reported in the recent study of \citet{wang2024effect}, thus the scalars $C_{[uu\grad\phi]}$ might have their importance as to describe the particles' migration fluxes that arise due to volume gradient of concentration. 
The terms on the last line, proportional to $\textbf{E}\cdot \textbf{E}$, have not been reported in the literature. 




Let us turns our attention to the effective stress, $\bm\sigma_\text{eff}$. 
In the first step let us consider the case where the disturbance fields satisfy Stokes equations, and $\grad \phi = 0$ 
Applying the same reasoning as for \ref{eq:darg_force_general}, leads us to the following general formulation for the effective suspension stress,  
\begin{align}
    % \frac{a}{\mu_f U}
    \bm\sigma_\text{eff} &= 
    \mu_f \phi 
    \{ 
    % +p_f\bm\delta
    + C_E  \textbf{E}
    +  C_{\grad u}^{(1)} [\grad \textbf{u}_{r} +(\grad \textbf{u}_{r})^\dagger + C_{\grad u}^{(2)}(\div  \textbf{u}_{r})\bm\delta] \}
    + \ldots
    \label{eq:stress_stokes}
\end{align}
The first scalar $C_E$ represents the response to the bulk stress from an applied mean shear $\textbf{E}$. 
Note that a contribution of the form $\sim[\grad \textbf{u}-(\grad \textbf{u})^\dagger]$ has been removed since as shown in \ref{chap:daniel2} the bulk stress of a suspension must remain symmetric if no body-torque are applied on the droplets. 
Therefore, $C_E$ simply represents the equivalent viscosity of the bulk coming from the first moment of the hydrodynamic forces.
It includes the Einstein equivalent viscosity term for example. 
The terms $C_{\grad u}^{(1)}$ and $C_{\grad u}^{(2)}$ represent the contribution of the second moment of hydrodynamic forces (see \ref{chap:closure-disperse}). 
On another hand, note that \citet{zhang2021stress} reported an additional stress, called the particle-fluid-particle stress, that arises due to long-range interaction which is due to multi-particle interactions through the fluid. 
This term is also proportional to $\grad \textbf{u}_{r}$. 
Consequently, in addition to the second hydrodynamic moment, the particle-fluid-particle stress contributes to $C_{\grad u}^{(1)}$ as $\sim \ldots+ \phi^{-1/3}$ according to \citep{zhang2021stress}. 
Consequently, the leading contribution of $C_{\grad u}$'s isn't $\sim \phi^0$ as suggested by the second moment of hydrodynamic force but rather $\phi^{-1/3}$ due to multi-particle interactions. 

Including inertial effects, we may include the terms, 
\begin{multline}
    \ldots
    % \frac{U^2\rho}
    +\phi \rho_f \{ C_{uu}^{(1)}
    \textbf{u}_{r}\textbf{u}_{r} 
    + C_{uu}^{(2)} (\textbf{u}_{r}\cdot \textbf{u}_{r}) \bm\delta \}
    + 
    \rho_f a^2 \{ C_{EE}^{(1)} \textbf{E}\cdot \textbf{E}
    +  C_{EE}^{(2)} (\textbf{E} : \textbf{E})\bm\delta \}
    \ldots
    \label{eq:stress2}
\end{multline}
As seen in \ref{chap:deformable,chap:pseudoturbulence} the first two terms proportional to, $\sim \textbf{u}_{r}\textbf{u}_{r}$, represent the Pseudo-turbulence tensor and the first moment of forces contribution. 
In addition, as reported in \citet{zhang2021stress} the functional form of the particle-fluid-particle stress also has this functional form at finite Reynolds number. 
Overall, the leading contribution from the $C_{uu}$'s is $\sim \phi^{-1/3}$. 
The third and fourth terms of \ref{eq:stress2} represent the pseudo-turbulence term as shown in \ref{chap:closure-disperse}. 
Additionally, as shown in \citet{stone2001inertial} the Stresslet term also includes a contribution of this form. 

If one considers that $\grad \phi \neq 0$ additional terms appear in the equivalent stress.  
The list of the second order symmetric tensors that can be built upon that consideration are, 
\begin{multline}
    % \frac{ a}{\mu_f U}\bm\sigma^\text{eq} &= 
    \ldots 
    +  \mu_f \{
    C_{\grad \phi u}^{(1)}
    (\textbf{u}_{r}
    \grad \phi 
    + 
    \grad \phi 
    \textbf{u}_{r}) 
    +C_{\grad \phi u}^{(2)} (\grad \phi \cdot 
    \textbf{u}_{r})\bm\delta \}\\
    +\rho_f a^2 \{
    C_{Eu \grad \phi}^{(1)} \textbf{E} (\textbf{u}_r\cdot \grad \phi) 
    + C_{Eu \grad \phi}^{(2)}  \textbf{u}_r (\textbf{E}\cdot\grad \phi+ \grad \phi\cdot \textbf{E})
    \}.
    \label{eq:stress_inho}
\end{multline}
The terms of the first line are the contribution of the second-order moment of force \eqref{chap:closure-disperse}. 
The terms on the second line represent the only possible contribution coming from the combination of $\phi\phi$ and inertial terms. 
Yet no study in the literature reported such a contribution.
A simple way to verify that the $C_{Eu \grad \phi}$'s are indeed non-zero would be to compute the second-order moment of hydrodynamic force for an isolated bubble in a linear potential flow. 

\subsection{Slightly non-uniform emulsions.}

As a first approach, one can just consider a slightly non-uniform suspension such as all the $\mathcal{O}(a^2/L^2) \approx 0$.  
For example, one can consider the following approximation for spherical droplets $\phi = n_pv_p + \textbf{V}_p :\grad^2 n_p = n_pv_p + \frac{v_p a^2}{5}\grad^2 n_p \approx n_pv_p + \mathcal{O}(a^2/L^2) = n_pv_p$.

Injecting, \ref{eq:darg_force_general,eq:stress_stokes} into the formulation \ref{eq:NS_dispersed,} gives us the final form of the NS equation, 
\begin{align}
    \div\textbf{u}&=0, \\
    (\pddt + \textbf{u}\cdot \grad)\phi &= \div(\phi \textbf{u}_r), \\
    \rho_d\phi (\pddt + \textbf{u}_p \cdot \grad)  \textbf{u}_p
    &= 
    \rho_d \phi \textbf{g}
    + 
        \pSavg{\bm\sigma_f^0 \cdot \textbf{n}}
        - m_p \div \pavg{\textbf{u}_\alpha'\textbf{u}_\alpha'}
    \\
    \rho_f (\pddt 
    + \textbf{u}\cdot \grad)
    \textbf{u}
    &= 
    \div (\bm\Sigma
    +  \bm{\sigma}_\text{eff} )
    + \kappa \pSavg{ \bm{\sigma}_f^0 \cdot \textbf{n}} 
    + \rho_f \textbf{g} 
    \label{eq:dtu_bulk}
\end{align}
% \begin{align}
%     \div\textbf{u}=&0, \\
%     (\pddt + \textbf{u}\cdot \grad)\phi =& \div(\phi \textbf{u}_r), \\
%     \phi (\pddt + \textbf{u}_p \cdot \grad)  \textbf{u}_p
%     =& 
%      \phi \textbf{g}
%     + 
%     \mu_f \phi \{
%     \grad p_f
%     + 
%     % (L^2/a^2) (U/U_M)  
%     a^{-2}
%     C_u \textbf{u}_{r}
%     +C_{[\grad^2 u]} \grad^2 \textbf{u}
%     \}\\
%     &- \div  \{\phi  C_{uu}^{(1)} \textbf{u}_{r}\textbf{u}_{r} 
%     + \phi C_{uu}^{(2)} (\textbf{u}_{r}\cdot \textbf{u}_{r}) \bm\delta \}
%     \\
%     \rho_f (\pddt 
%     + \textbf{u}\cdot \grad)
%     \textbf{u}
%     =& 
%     \div (\bm\Sigma
%     +  \bm{\sigma}_\text{eff} )
%     + \kappa 
%         \rho_d \phi (\pddt + \textbf{u}_p \cdot \grad) \textbf{u}_p
%         % -\rho_d \phi \textbf{g}
%     + \rho_f[1 - \phi (1  - \zeta) ] \textbf{g} 
% \end{align}
\begin{align*}
    \pSavg{\bm\sigma_f\cdot \textbf{n}} &= 
     \phi \{
    \div \bm\Sigma
    + 
    % (L^2/a^2) (U/U_M)  
    \mu_f a^{-2}
    C_u \textbf{u}_{r}
    +\mu_f C_{[\grad^2 u]} \grad^2 \textbf{u}
    \} 
    \\
    \bm\Sigma &= - p_f +2 \mu_f \textbf{E}\\
    \bm\sigma_\text{eff} &= 
    \mu_f \phi 
    \{ 
    % +p_f\bm\delta
    + C_E  \textbf{E}
    +  C_{\grad u}^{(1)} [\grad \textbf{u}_{r} +(\grad \textbf{u}_{r})^\dagger + C_{\grad u}^{(2)}(\div  \textbf{u}_{r})\bm\delta] \}\\
    &+\phi \rho_f \{ C_{uu}^{(1)}
    \textbf{u}_{r}\textbf{u}_{r} 
    + C_{uu}^{(2)} (\textbf{u}_{r}\cdot \textbf{u}_{r}) \bm\delta \}\\
    \pavg{\textbf{u}_\alpha'\textbf{u}_\alpha'} &= 
    \phi \{ C_{uu}^{(3)}
    \textbf{u}_{r}\textbf{u}_{r} 
    + C_{uu}^{(4)} (\textbf{u}_{r}\cdot \textbf{u}_{r}) \bm\delta \}
\end{align*}
Note that $\pavg{\textbf{u}_\alpha'\textbf{u}_\alpha'}$ has the same functional form as the pseudo-turbulence stress since it also results from the ensemble average of a disturbance field (that is therefore proportional to $\textbf{u}_r$, $\textbf{u}$, and so on \ldots). 















% Uncomment the following if need
% Uncomment the following if need
% Uncomment the following if need
% Uncomment the following if need
% Uncomment the following if need
% Uncomment the following if need
% Uncomment the following if need
% Uncomment the following if need
% Uncomment the following if need
% Uncomment the following if need
% Uncomment the following if need
% Uncomment the following if need

\subsection{Slightly non-uniform Stokestain dilute emulsions}

The governing equations for the dispersed phase in stokes regime read,
\begin{align}
    (\pddt + \textbf{u} \cdot \grad)\phi = \div(\phi \textbf{u}_r)\\
    0 
    = 
    \rho_d \phi \textbf{g}
    + \pSavg{{\bm{\sigma}_f^0 \cdot \textbf{n}_d}}. 
\end{align}
From which we directly deduce the form of the two terms on the right-hand side of \ref{eq:dtu_bulk}, namely
\begin{equation*}
    \rho_f \textbf{g} 
    + \kappa
    \pSavg{{\bm{\sigma}_f^0 \cdot \textbf{n}_d}}
    % =
    % \textbf{g}
    % +  
    % \phi (- 1 + \zeta)
    % \textbf{g}
    = \rho_f \textbf{g}(1 + \phi(\zeta-1) ). 
\end{equation*}

In Stokes regime the relative velocity $\textbf{u}_r$ can be closed algebraically from the dispersed phase momentum equation.  
From the closure in Stokes and dilute regime derived in \ref{chap:daniel2} we have, 
\begin{equation*}
    \pSavg{\bm{\sigma}_f^0\cdot \textbf{n}_d} = 
    \phi_d \div\bm\Sigma
    + \frac{3\phi_d\mu_f}{2 a^2} 
    \left(\frac{3\lambda+2}{\lambda+1}\right) \textbf{u}_{r} 
    + \frac{3\phi_d\mu_f}{4} \left(\frac{\lambda}{\lambda+1}\right)\grad^2\textbf{u}
    = - 
    \rho_d \phi \textbf{g}
\end{equation*}
% We deduce that, 
% \begin{align*}
%     \frac{3\phi_d\mu_f}{2 a^2}  
%     \left(\frac{3\lambda+2}{\lambda+1}\right) \textbf{u}_{f p} 
%     = 
%     \phi_d \grad p_f
%     - \rho_d \phi \textbf{g}
%     - \phi_d \mu_f \left[
%         1 + \frac{3}{4} \left(\frac{\lambda}{\lambda+1}\right)
%     \right]\grad^2 \textbf{u}
% \end{align*}
which simplify to 
\begin{align*}
    \frac{3\mu_f}{2 a^2}\left(\frac{3\lambda+2}{\lambda+1}\right)
   \textbf{u}_{r} 
    = 
    \grad p_f
    - \rho_f\zeta   \textbf{g}
    -  \mu_f \left[
        1 + \frac{3}{4} \left(\frac{\lambda}{\lambda+1}\right)
        \right]\grad^2 \textbf{u}
\end{align*}


Using the closure of the first- and second-moment of the hydrodynamic forces derived in \ref{chap:daniel2} we obtain the final set of equations, 
\begin{align}
    \div\textbf{u}&=0, \\
    (\pddt + \textbf{u}\cdot \grad)\phi &= \div(\phi \textbf{u}_r), \\
    \rho_f (\pddt 
    + \textbf{u}\cdot \grad)
    \textbf{u}
    &= 
    \div (\bm\Sigma + \bm{\sigma}_\text{eff})
    % + \kappa \pSavg{ \bm{\sigma}_f^0 \cdot \textbf{n}} 
    + \rho_f \textbf{g} (1 + \phi(\zeta-1) )\\
    \textbf{u}_{r} 
    &= 
    \frac{2 a^2}{3\mu_f}\left(\frac{\lambda+1}{3\lambda+2}\right)
    \left\{
        \grad p_f
        - \rho_f\zeta   \textbf{g}
        -  \mu_f \left[
            1 + \frac{3}{4} \left(\frac{\lambda}{\lambda+1}\right)
            \right]\grad^2 \textbf{u}
    \right\}
    \\
    % \pavg{\textbf{u}_\alpha'\textbf{u}_\alpha'} &= 
    % \phi \{ C_{uu}^{(1)}
    % \textbf{u}_{r}\textbf{u}_{r} 
    % + C_{uu}^{(2)} (\textbf{u}_{r}\cdot \textbf{u}_{r}) \bm\delta \}
    % \\
    \bm\Sigma& = 
    - p_f +2\mu_f  \textbf{E} \left[1 + \frac{\phi}{5}\left(\frac{5\lambda +2}{\lambda +1}\right)\right] \\
    \bm\sigma_\text{eff} &= \mu_f \phi      
    \frac{3\lambda}{4(\lambda + 1)}[\grad \textbf{u}_{r} +(\grad \textbf{u}_{r})^\dagger +\frac{2-3\lambda}{3\lambda}(\div  \textbf{u}_{r})\bm\delta] 
\end{align}
As we can observe this system reduces to the forced Navier-Stokes equations coupled with a transport equation for $\phi$. 
It is interesting to note that \citet{kriaa2023two} used the exact same set of equation, but with $\bm\Sigma + \bm\sigma_\text{eff} \to -p_f +2 \mu_f \textbf{E}$ instead. 
As their study provided realistic results this demonstrates that the effective stress can be neglected in certain circumstances. 


% To summary, we need to solve a forced Navier-Stokes equaiton for \textbf{u}, a momentum equation for $\textbf{u}_p$ and an adevection equaiton for $\phi$. 
% These three steps are done easliy with the \texttt{Basilisk C}. 

% We first initialize the gird, the navier-stokes solver and include the Bell-Collela-Glaz advection scheme,  
% \begin{lstlisting}[language=C][language=C]
%     #include "grid/multigrid.h"
%     #include "navier-stokes/centered.h"
%     #include "bcg.h"
% \end{lstlisting}

% The we initialize the main parameters and functions 
% \begin{lstlisting}[language=C][language=C]
%     #define zeta 0.9
%     #define G 1. 
%     #define d 1e-3
%     #define rhof 1. 
%     #define lambda 10.
%     #define l             ((2+3*lambda)/(lambda+1))
%     #define Ga 50

%     #define muf           (rhof/Ga  * d * sqrt( sqrt(sq(1-zeta))*G*d))

%     #define mu_ein(phi)   (muf*(1 + phi / 2 * (5*lambda +2 )/(lambda +1) ) )
%     #define mu_effU       (muf*3*lambda/(4*(lambda+1)))
% \end{lstlisting}


% After initializing the domain and initail volume fraction field we go on and add the closure terms to the solver. 
% The first thing to do is to add the Buoyancy force and efffective viscosity to the Navier-Stokes equation, that is easily done by overaloading the acceleration and properties events,  
% \begin{lstlisting}[language=C]
% event acceleration (i++)
% {
%   coord gg = {0,-G};
%   // gravity acceleration 
%   foreach_face(){
%     double phi_face = (PHI[] + PHI[-1])/2. ; 
%     av.x[] += (1 - phi_face  + phi_face * zeta )* gg.x; 
%   }
% event properties (i++) {
%   foreach_face(){
%     double phi_face = (PHI[] + PHI[-1])/2.;
%     muv.x[] =  mu_ein(phi_face);
%   }
% }
% \end{lstlisting}

% Regarding teh dispersed phase we obtain the velocity explicitely with the stokes drag force, 
% \begin{lstlisting}[language=C]

% event slip_velocity(i++){
%   coord gg = {0,-G};

%   foreach_face(){
%     double dDP = alpha.x[]*(p[] - p[-1])/Delta; 
%     double Lx = (uf.x[1] + uf.x[-1] + uf.x[0,1] + uf.x[0,-1] - 4*uf.x[])/(Delta*Delta);
%     ur.x[] =  d*d / (6*muf*l) * (dDP - rhof*zeta*gg.x  - muf*Lx *(1 +3/4*lambda/(lambda+1)) );
%     up.x[] =  (uf.x[] - ur.x[]); //particle phase vel
%   }  
% }
% \end{lstlisting}

% and transport the volume fraction fields accordingly, 

% \begin{lstlisting}[language=C]
% event tracer_advection (i++,last) {
%     dt = timestep (up,DT);
%     advection ((scalar *) {PHI}, up, dt);
% }
% \end{lstlisting}
    

% With these simples steps we could carry out efficient EulerEuler simulations. 


% \subsection{conclusion}


% \begin{enumerate}
%     \item We proposed a easy formulation 
%     \item We then discuss the rheorlogy of teh suspension 
% \end{enumerate}

% Perspectives : 
% \begin{enumerate}
%     \item Parler des meusure de viscosity
% \end{enumerate}
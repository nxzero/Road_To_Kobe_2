\documentclass[10pt,a4paper]{article}

\usepackage[utf8]{inputenc}
\usepackage{amsmath}
\usepackage{xcolor}
\usepackage{amsfonts}
\usepackage{amssymb}
\usepackage[bmargin=2.cm,tmargin=2.cm,lmargin=2cm,rmargin=2cm]{geometry}
\usepackage{natbib}

\newcommand{\tb}[1]{\color{blue}#1\color{black}}
\newcommand{\tr}[1]{\color{red}#1\color{black}}

\begin{document}

\title{Response to reviewer 1} 
\maketitle
\textbf{Paper entitled "Averaged equations for disperse two-phase flow with interfacial transport. IJMF"}
\bigskip

%\section*{Reviewer 1}

\textit{This manuscript derives the averaged equations for disperse multiphase flows. Authors are
certainly aware that this subject has been studied many times before by the cited literature. This
manuscript does have its uniqueness, especially in its treatments of the interface terms and the
moments. However, the authors do not provide an example to show how their method is more
advantageous or reveals something new. As currently written, the derivations are too abstract. To
publish this manuscript, I suggest the author to come up an example in which the closures can be
easily obtained by their methods. For instance, authors can take advantage of their treatment of
the interfaces study thermal capillary motion of bubbles to show the advantage of the method, or
study rotating or deformable particles to show some unique or non-Newtonian behavior..}

\tb{
    We thank the Referee for his /her positive feedback on our manuscript. 
    Before providing a point-by-point response to the Referee’s comments, we just wish to mention the main changes introduced in the paper to address the concerns raised by the other referees.
    In particular we have added new paragraphs in the introduction to outline the new contributions of the present paper and rewritten the conclusion.


%We thank the Referee for his/her appreciation of the paper. 
%Following the referee suggestion we have added two sections (denoted V and VI in the article). 
%The first (section V) desrcibe a set of averaged equations (including higher order moments) to desbrie the motion of droplets immersed in a Newtonian phase.
%In particular we emphasize the role on the equation for the second order mass moment equation.
%IN the second section (section VI) we compute the closure in the limit of dilute viscous dominated flows. 
%In particular we compute the effect of mean interfaccial gradient on the force and moments acting on the droplets.
%Moreover we clarify the role of the angular momentum equations as well as some covariance closures appearing in the averaged equations.

Following the Referee’s suggestion, we have added two new sections (Sections V and VI) to the revised version of the article.

Section V presents a set of averaged equations, including higher-order moments, to describe the dynamics of droplets immersed in a Newtonian fluid. 
Particular emphasis is placed on the choice of the stress decomposition and on the equation governing the second-order mass moment and first moment of momentum equation, which plays a key role in describing the deformation of the dispersed phase.

In Section VI, we derive the closure terms in the dilute, viscous-dominated regime. 
Specifically, we quantify the influence of surface tension gradient on the forces and moments acting on the droplets. 
Additionally, we discuss several covariance closure terms that emerge in the averaged equations.
Finally we demonstrate how the leading order deformation of the droplets can be obtained thanks to the second-order mass moment and first moment of momentum equation.

In light of these modifications, the title has been changed in "Averaged equations for disperse two-phase flow with interfacial properties and their closures for dilute suspension of droplets".
%Before providing a point-by-point response to the Referee’s comments, we just wish to mention the main changes introduced in the paper to address the concerns raised by the other referee.

}

\textit{Following are some technical suggestions.}


\begin{enumerate}
    \item Angular brackets are used for both inner product and averaged quantities. My suggestion is to use round brackets for the inner product to make the manuscript easier to read.
    \tb{Done.}%We thank the referee for his remark. After making the change it appears to us that round brackets were even more confusing}
    \item In the first paragraph of sec. 2.2: use “I” to denote the identity tensor, instead of delta to
    clearly distinguish it from the Dirac delta function.
    \tb{
        We chose not to use \textbf{I} for the identity tensor because it is already used to denote the inertia tensor of the droplets in Section V. 
        Additionally, the Dirac delta function is not typeset in bold, whereas the identity tensor is, which helps maintain a clear distinction between the two.
     }
    \item After eq. (2.15) “… with $\phi_I = <\delta_\Gamma>$ the probability of finding the interface at the point x at
    time t”. The probability of finding an interface at a given location is always zero. This
    quantity is not the probability. It is the interface area per unit volume, also called the specific
    area.\\
    \tb{We thank the referee for his/her suggestion. This has been corrected}
    \item Please note that although $\nabla_{||} n$ is well defined but not $\nabla n$; therefore, should not be used
    \\
    \tb{Although $\nabla n$ is not meaningful in the sense of function-since $\partial n / \partial n$ is not defined-we argue that it can still be interpreted in the sense of distributions, as shown in Appendix A and in the work of Orlando et al. (2023). 
    Moreover, the use of $\nabla n$ is common in the literature (see Nadim, 1996; Lhuillier, 2003; Orlando et al., 2023).
    Since we believe this notation does not cause significant confusion, we have decided to retain it.}
        
    %Although $\nabla n$ is not meaningfull in the sense of function (as $\partial \bm n$/ \partial n is not defined) we believe that it can still be defined in the sense of distribution (as demonstrated in appendix 1 and by Orlando et al. (2023)).
    %. Moreover $\nabla n$ is used by many auhors in the literature (see Nadim 1996, Lhuillier 2003, Orlando et al. (2023) and we believe this notation do not create much confusion so we have decided to keep it.%this  satisfactory to the author, as evidenced by Equation (65) in Orlando (2023), "On the evolution equations of interfacial variables in two-phase flows."}
\end{enumerate}


\end{document}


\documentclass[10pt,a4paper]{article}

\usepackage[utf8]{inputenc}
\usepackage{amsmath}
\usepackage{xcolor}
\usepackage{amsfonts}
\usepackage{amssymb}
\usepackage[bmargin=2.cm,tmargin=2.cm,lmargin=2cm,rmargin=2cm]{geometry}
\usepackage{natbib}

\newcommand{\tb}[1]{\color{blue}#1\color{black}}
\newcommand{\tr}[1]{\color{red}#1\color{black}}

\begin{document}

\title{Response to reviewer 1} 
\maketitle
\textbf{Paper entitled "Buoyancy driven motion of non-coalescing inertial drops: microstructure modeling with nearest particle statistics.
Acta Mechanica"}
\bigskip


\textit{In this work, the authors employ the open-source code basilisk to study three different particle arrangements of droplets. The simulations are based on a geometric volume-of-fluid method where a specific algorithm is used to prevent the coalescence between droplets and thus to keep the system mono-dispersed. In particular, the authors employ an optimized multi-marker method where multiple marker functions are used to avoid numerical coalescence. The authors perform simulations at different volume fractions, up to 20\%. Overall, the paper is well written and the results presented are of interest to the multiphase flow community. I think that after revision the paper can be accepted for publication.}


\color{blue}
We thank the Referee for his/her positive appreciation of the paper. 
Before providing a point-by-point response to the Referee’s comments, we just wish to mention the main changes 
introduced in the paper to address the concerns raised by the other referee.
\begin{itemize}
    \item In figure 9 (old figure 8) we plot the ratio $P_{(r-nst)}/P_{nst}^{th}$ instead of $P_{(r-nst)}/P_{nst}^{\phi\ll 1}$, the following discussion have been changed in consequences. 
    \item Additionally, we have included a plot of $P_{nst}^{th}$ for various $\phi$ to assist readers who may not be familiar with nearest neighbor distributions. 
\end{itemize}

% Additionally, as discussed before submission with the editor, we updated our DNS results for a more refined set of DNS (from $d/\Delta \approx 20$ to $d/\Delta \approx 25$) as the other set of results exhibited some doubtful results due to meshing problems. 
% The main changes arising from this new set of DNS are the followings, 
% \begin{itemize}
%     \item The bond number is now fixed to $Bo = 0.5$. 
%     \item The maximum \textit{Galileo} number is $80$ instead of $100$.
%     \item The appendix (B.3) have been updated yielding better converged results. 
% \end{itemize}  
% As expected, the results and conclusions made in this study remain unchanged.
% \color{black}



The following points should be improved:

1. What is the rationale for studying these three different configurations? How did the authors define these three configurations?

\tb{
    These configurations were selected because they represent distinct and fundamental types of microstructures observed in particle-laden flows. 
}

2. Compared to the standard VoF (single-marker), what is the computational overhead?

\tb{
    It is said in the text that:  
    ``In
    particular, it is observed that the no-coalesce.h algorithm accounts for approximately 4\% of the total
    computational time of a simulation in the densest scenario
    ''
    Do the referee needs more details ? 
}

3. It would be good to show some qualitative visualizations of the system behavior before moving to the discussion of the results.

\tb{
    The authors appreciate the referee's suggestion for including qualitative visualizations of the system behavior. 
    In our opinion, \textit{Figure 7} serves as a visualization of the system behavior, showcasing the distinct particle structures for each configuration (homogeneous, and layers). 
    This figure provides a clear qualitative comparison of the microstructures, thereby aligning with the referee's suggestion.
}

4. In Figure 4, it would be useful to add a legend also in the figure (and only in the caption). Also, it is difficult to read this figure. Will it be possible to see the results in different sub-panels? One for lambda=1 and one for lambda=10?

\tb{
    The authors agree, the legend have been added, and the figure is now split into one for each viscosity ratio $\lambda$. 
}

5. The deformation of the drops is limited (as shown in Figure 12). Did the authors perform simulations with more deformable drops? 

\tb{
    Unfortunately due to the high computational cost of these simulations additional DNS could not be performed, this will be investigated in a future work. 
}

6. Is turbulence forced in the triple-periodic box? Or the induced flow motion is just induced by the motion of the bubbles?

\tb{
    As mentioned in the text, we consider only the buoyancy forces in this problem. 
}

\end{document}

\documentclass[10pt,a4paper]{article}

\usepackage[utf8]{inputenc}
\usepackage{amsmath}
\usepackage{xcolor}
\usepackage{amsfonts}
\usepackage{amssymb}
\usepackage[bmargin=2.cm,tmargin=2.cm,lmargin=2cm,rmargin=2cm]{geometry}
\usepackage{natbib}

\newcommand{\tb}[1]{\color{blue}#1\color{black}}
\newcommand{\tr}[1]{\color{red}#1\color{black}}

\begin{document}

\title{Response to reviewer 2} 
\maketitle
\textbf{Paper entitled "Averaged equations for disperse two-phase flow with interfacial transport. IJMF"}
\bigskip


%\section*{Reviewer 2}

\textit{In this paper, the closure problem of disperse two-phase flow is derived at an exact mathematical level under general circumstances using ensemble averaging. When compared with previous such derivations (Drew, Jackson, Prosperetti, etc), this paper makes significant progress in two aspects in my opinion. First, it shows that the left-hand side of the sum of the solid phase and interface balance equations $(C_d+C_Gamma)$ is equal to a Taylor-like infinite sum over the left-hand sides of the moments $(M^(i))$ of the particle phase balance equations. In my understanding, previously, it was only established that the balance equation $C_d =0$ is equivalent to the zeroth moment particle phase balance $M^(0)=0$, a trivial result since any two things equal to 0, derived without approximation, are also equal to one another. Based on their finding, the authors argue in favor of a "hybrid model", which I believe they will continue working on in future papers. A second point of significant progress seems to me the author's treatment of the interface balance. It is treated with a mathematical rigor I have not seen before.
Given the complexity of the topic, the authors have done an excellent job in presenting their derivation. I found it very easy to follow the mathematics, except in a few instances where they (probably unintentionally) used different symbols for the same quantity (see comments below). Overall, I find this paper impressive and recommend its publication after just a few minor revisions.}

\tb{
We thank the Referee for his /her positive feedback on our manuscript. 
Before providing a point-by-point response to the Referee’s comments, we just wish to mention the main changes introduced in the paper to address the concerns raised by the other referees.
We have added two new sections (Sections V and VI) to the revised version of the article.

Section V presents a set of averaged equations, including higher-order moments, to describe the dynamics of droplets immersed in a Newtonian fluid. 
Particular emphasis is placed on the choice of the stress decomposition and on the equation governing the second-order mass moment and first moment of momentum equation, which plays a key role in describing the deformation of the dispersed phase.

In Section VI, we derive the closure terms in the dilute, viscous-dominated regime. 
Specifically, we quantify the influence of surface tension gradient on the forces and moments acting on the droplets. 
Additionally, we discuss several covariance closure terms that emerge in the averaged equations.
Finally we demonstrate how the leading order deformation of the droplets can be obtained thanks to the second-order mass moment and first moment of momentum equation.

We have also added new paragraphs in the introduction to outline the new contributions of the present paper and rewritten the conclusion.
In light of these modifications, the title has been changed in "Averaged equations for disperse two-phase flow with interfacial properties and their closures for dilute suspension of droplets"

}

Minor comments:
\begin{itemize}
    \item  different notations with the same meaning (subscripts 'I' and 'Gamma') are used for interface quantities: for example, at the beginning of section 2.2, just after Eq. (3.4), Eq. (3.30), and elsewhere
    \\
    \tb{this has been corrected}
    \item just before Eq. (2.11): "one obtains" \tb{corrected}
    \item writing just below Eq. (3.5), "using Eq. (3.10)", is not a good style as the latter equation has not been introduced at this point. At the very least, write something like, "as will become clear in Section 3.2. (Eq. 3.10)), …" Better, try to avoid this. \tb{corrected}
    \item "Note that Eq. (3.8) generalizes the usual expression" \tb{corrected}
    \item Eq. (3.12), a '+' too many \tb{corrected}
    \item Eq. (3.21) dGamma, not dSigma \tb{corrected}
\end{itemize}




\end{document}

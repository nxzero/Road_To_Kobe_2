\documentclass[10pt,a4paper]{article}

\usepackage[utf8]{inputenc}
\usepackage{amsmath}
\usepackage{xcolor}
\usepackage{amsfonts}
\usepackage{amssymb}
\usepackage[bmargin=2.cm,tmargin=2.cm,lmargin=2cm,rmargin=2cm]{geometry}
\usepackage{natbib}

\newcommand{\tb}[1]{\color{blue}#1\color{black}}
\newcommand{\tr}[1]{\color{red}#1\color{black}}

\begin{document}

\title{Response to reviewer 1} 
\maketitle
\textbf{Paper entitled "Averaged equations for disperse two-phase flow with interfacial transport. IJMF"}
\bigskip

\section*{Reviewer 1}

\textit{This manuscript derives the averaged equations for disperse multiphase flows. Authors are
certainly aware that this subject has been studied many times before by the cited literature. This
manuscript does have its uniqueness, especially in its treatments of the interface terms and the
moments. However, the authors do not provide an example to show how their method is more
advantageous or reveals something new. As currently written, the derivations are too abstract. To
publish this manuscript, I suggest the author to come up an example in which the closures can be
easily obtained by their methods. For instance, authors can take advantage of their treatment of
the interfaces study thermal capillary motion of bubbles to show the advantage of the method, or
study rotating or deformable particles to show some unique or non-Newtonian behavior..}

\tb{
    We thank the Referee for his /her positive feedback on our manuscript. 
    Before providing a point-by-point response to the Referee’s comments, we just wish to mention the main changes introduced in the paper to address the concerns raised by the other referees.
    In particular we have  added a paragraph in the introduction to outline the new contributions of the present paper.


%We thank the Referee for his/her appreciation of the paper. 
%Following the referee suggestion we have added two sections (denoted V and VI in the article). 
%The first (section V) desrcibe a set of averaged equations (including higher order moments) to desbrie the motion of droplets immersed in a Newtonian phase.
%In particular we emphasize the role on the equation for the second order mass moment equation.
%IN the second section (section VI) we compute the closure in the limit of dilute viscous dominated flows. 
%In particular we compute the effect of mean interfaccial gradient on the force and moments acting on the droplets.
%Moreover we clarify the role of the angular momentum equations as well as some covariance closures appearing in the averaged equations.

Following the Referee’s suggestion, we have added two new sections (Sections V and VI) to the revised version of the article.

Section V presents a set of averaged equations, including higher-order moments, to describe the dynamics of droplets immersed in a Newtonian fluid. 
Particular emphasis is placed on the choice of the stress decomposition and on the equation governing the second-order mass moment, which plays a key role in describing the deformation of the dispersed phase.

In Section VI, we derive the closure terms in the dilute, viscous-dominated regime. 
Specifically, we quantify the influence of surface tension gradient on the forces and moments acting on the droplets. 
Additionally, we clarify the role of the angular momentum balance and discuss several covariance closure terms that emerge in the averaged equations.
Finally we demonstrate how the leading order deformation of the droplets can be obtained thanks to the second-order mass moment equation.

%Before providing a point-by-point response to the Referee’s comments, we just wish to mention the main changes introduced in the paper to address the concerns raised by the other referee.

}

\textit{Following are some technical suggestions.}


\begin{enumerate}
    \item Angular brackets are used for both inner product and averaged quantities. My suggestion is to use round brackets for the inner product to make the manuscript easier to read.
    \item In the first paragraph of sec. 2.2: use “I” to denote the identity tensor, instead of delta to
    clearly distinguish it from the Dirac delta function.
    \item After eq. (2.15) “… with $\phi_I = <\delta_\Gamma>$ the probability of finding the interface at the point x at
    time t”. The probability of finding an interface at a given location is always zero. This
    quantity is not the probability. It is the interface area per unit volume, also called the specific
    area.\\
    \tb{This has been modified}
    \item Please note that although $\nabla_{||} n$ is well defined but not $\nabla n$; therefore, should not be used
    \\
    \tr{To the author it seems fine as demonstrated with Eq (65) of Orlando }
\end{enumerate}


\section*{Reviewer 2}

\textit{In this paper, the closure problem of disperse two-phase flow is derived at an exact mathematical level under general circumstances using ensemble averaging. When compared with previous such derivations (Drew, Jackson, Prosperetti, etc), this paper makes significant progress in two aspects in my opinion. First, it shows that the left-hand side of the sum of the solid phase and interface balance equations $(C_d+C_Gamma)$ is equal to a Taylor-like infinite sum over the left-hand sides of the moments $(M^(i))$ of the particle phase balance equations. In my understanding, previously, it was only established that the balance equation $C_d =0$ is equivalent to the zeroth moment particle phase balance $M^(0)=0$, a trivial result since any two things equal to 0, derived without approximation, are also equal to one another. Based on their finding, the authors argue in favor of a "hybrid model", which I believe they will continue working on in future papers. A second point of significant progress seems to me the author's treatment of the interface balance. It is treated with a mathematical rigor I have not seen before.
Given the complexity of the topic, the authors have done an excellent job in presenting their derivation. I found it very easy to follow the mathematics, except in a few instances where they (probably unintentionally) used different symbols for the same quantity (see comments below). Overall, I find this paper impressive and recommend its publication after just a few minor revisions.}

\tb{
We thank the Referee for his /her positive feedback on our manuscript. 
Before providing a point-by-point response to the Referee’s comments, we just wish to mention the main changes introduced in the paper to address the concerns raised by the other referees.
We have added two new sections (Sections V and VI) to the revised version of the article.

Section V presents a set of averaged equations, including higher-order moments, to describe the dynamics of droplets immersed in a Newtonian fluid. 
Particular emphasis is placed on the choice of the stress decomposition and on the equation governing the second-order mass moment, which plays a key role in describing the deformation of the dispersed phase.

In Section VI, we derive the closure terms in the dilute, viscous-dominated regime. 
Specifically, we quantify the influence of surface tension gradient on the forces and moments acting on the droplets. 
Additionally, we clarify the role of the angular momentum balance and discuss several covariance closure terms that emerge in the averaged equations.
Finally we demonstrate how the leading order deformation of the droplets can be obtained thanks to the second-order mass moment equation.

We have also added a paragraph in the introduction to outline the new contributions of the present paper.

}

Minor comments:
\begin{itemize}
    \item  different notations with the same meaning (subscripts 'I' and 'Gamma') are used for interface quantities: for example, at the beginning of section 2.2, just after Eq. (3.4), Eq. (3.30), and elsewhere
    \\
    \tb{this has been corrected}
    \item just before Eq. (2.11): "one obtains" \tb{OK}
    \item writing just below Eq. (3.5), "using Eq. (3.10)", is not a good style as the latter equation has not been introduced at this point. At the very least, write something like, "as will become clear in Section 3.2. (Eq. 3.10)), …" Better, try to avoid this. \tb{OK}
    \item "Note that Eq. (3.8) generalizes the usual expression" \tb{OK}
    \item Eq. (3.12), a '+' too many \tb{OK}
    \item Eq. (3.21) dGamma, not dSigma \tb{OK}
\end{itemize}

\section*{Reviewer 3}

\textit{This paper derives the continuum transport equations that govern two phase systems where
one phase is of the form of dispersed, but not necessarily rigid, particles. The method of
derivation, despite slight differences from previous approaches, appears to be quite standard.
The main extensions that I can detect are that transport are derived for properties associated
with the interfaces between the dispersed and continuous phases. The paper is nicely
introduced, the analysis is clear and appears to be correct - at least the standard two phase
flow equations for bulk properties are recovered. Overall, I like the clarity of exposition and
the connections the authors make with previous papers on the subject.
However, I do have some concerns, which are listed below. I believe most of these concerns
can be addressed if the authors put their minds to it, but it will require some rewriting, clearer
explanation and possibly a better discussion of the main advances.}

\tb{We thank the Referee for his /her positive feedback on our manuscript.
Before providing a point-by-point response to the Referee’s comments, we just wish to mention the main changes introduced in the paper to address the concerns raised by the other referees, some of which overlap with
those specified by the current referee.
We have added two new sections (Sections V and VI) to the revised version of the article.

Section V presents a set of averaged equations, including higher-order moments, to describe the dynamics of droplets immersed in a Newtonian fluid. 
Particular emphasis is placed on the choice of the stress decomposition and on the equation governing the second-order mass moment, which plays a key role in describing the deformation of the dispersed phase.

In Section VI, we derive the closure terms in the dilute, viscous-dominated regime. 
Specifically, we quantify the influence of surface tension gradient on the forces and moments acting on the droplets. 
Additionally, we clarify the role of the angular momentum balance and discuss several covariance closure terms that emerge in the averaged equations.
Finally we demonstrate how the leading order deformation of the droplets can be obtained thanks to the second-order mass moment equation.


}


\begin{enumerate}
    \item It is not clear what the purpose of the paper is, and in what substantive way it departs or
    extends previous studies, many of which the authors have already cited. Arriving at known
    results by another route might be a useful and rewarding exercise for the person doing it, but
    to qualify as a paper it must make an original contribution. What are the original
    contributions of the paper? To make this clear, I suggest the authors replace or expand the
    last paragraph of the introductory section to outline the new contributions in the paper and
    state in what way the analysis presented departs from previous papers.
    
    \tb{
        We agree with the referee. 
        We have expanded the last paragraph of the introduction to outline the new contributions of the present paper.
        Although this problem has been addressed by several research groups, including Lhuillier (1992) and Zhang et Prosperetti (1994), our contribution lies in presenting a unified and simple theoretical framework for deriving the hybrid set of governing equations.
        First the Lagrangian balance equations are derived without any simplifying assumptions, thus allowing for: (1) the incorporation of interfacial properties while retaining a Lagrangian framework, and (2) the representation of internal gradients within droplets through the use of distribution moments.
        Moreover we believe that even if the link between Lagrangian and Eulerian averaging approaches is well understood, it is often overlooked and has never been fully demonstrated in such general scenarios that the present one.
        A central feature of our approach is the emphasis on moment equations for the dispersed phase, as previously evidenced by Lhuillier \& Nadim (2009) in the context of solid particles. 
        We argue that, regardless of the specific problem under consideration - including those involving fluid particles of complex shape - the hybrid formulation offers a more physically grounded alternative to the traditional two-fluid model for describing dispersed flows.
        To illustrate our methodology, we provide closure laws in the dilute, viscous-dominated regime, with special attention to the role of surface tension gradients and the closure of higher-order moment equations.
        %in the limit of dilute viscous dominated flows, including diagient of surface tension effects with a special attention on the closure of the higher moment equations.

                %\begin{itemize}
        %    \item The Lagrangian balance equations are derived with no assumption, hence taking in account (1) the interfacial properties but still with a Lagrangian approches (2) the internal gradient of droplets internal properties through the moments of distribution. 
        %    \item 
        %    \item The equivalence between Lagrangian-averaged law and Eulerian one is rigorously demonstrated, that is an aspect often overlooked in the literature 
        %    \item Overall we put in the front line the moment equations describing the dispersed phase and reach the major conclusion that regardless the systems studies the hybrid model should be use over the two-fluid model for dispersed flows. 
        %\end{itemize}
        %\tr{This is more a pedagogical article which propose a clear derivation of the equations and }
    }
    \item One original contribution that I see is the transport equations of interfacial properties. This
    is a reasonable extension, but if it is the only one, the paper does not have to be this long and
    does not have to repeat the derivation of transport equations for bulk properties of the
    dispersed and continuous phases. 
    Additionally, what do the authors foresee the interfacial
    transport equations being useful for? For example, for a scalar property such as surfactant
    concentration, transport equations for the zeroth moment is quite simple - one can even write
    them down intuitively. I can see that transport equations for higher moments might be useful,
    for example in computing the mean Marangoni drift of bubbles in a liquid. There isn't much
    discussion of these aspects, which makes one wonder what the motivation is for deriving the
    equations.
    \tb{
        We thank the Referee for this excellent remark which motivated us to address the influence of surface tension gradients on the governing equations and closure relations. 
        These gradients give rise to the well-known Marangoni drift but also contribute to the effective stress in the suspension. 
        Those effects are now discussed in detail in Section VI of the revised manuscript.
        %to discuss the effect of surface tension gradients on the governing equations and closure laws.
        %It appears that surface tension gradients induced the well known Marangoni drift contribution to the effective stress of the suspension.
        %Those effect are discussed in section VI ot the revised manuscript.
        %M but also stresslet  
        
        In response to the reviewer’s comment that " the paper does not have to be this long and
        does not have to repeat the derivation of transport equations for bulk properties of the
        dispersed and continuous phases." , we have chosen to retain the derivations for both the bulk and interfacial transport equations. 
        We believe that including both contributes to a more pedagogical and coherent presentation of the overall derivation.
        %Furthermore, since the %bulk and interfacial formulations are intrinsically linked through the definition of Lagrangian properties, omitting either would compromise the generality and clarity of the manuscript. %Removing these derivations could hinder the reader’s understanding of how the different components of the model interact.
        
        %We have decided to conserve both bulk and interfaces equations derivation as we beleive this make the whole derivation more pedagogical.
        %Moreover since both equations (interfacial and bulk) are interconnected trough the definition of lagragian properties we believe that removing them will make the paper less general and readable.
        %\begin{itemize}
        %    \item The interfacial equation is crutial to well take in account interfacial momentum jump because of surafce tension (in the first moment of momentum), or partciles surface Energies etc... 
        %    \item Regarding surfactant the zero order is indeed wuite simple and obvious. The first order balance would represent where does these surfacant are located on the surface and how this location evolve. Indeed contaminated droplets form spherical caps, the first moment balence would predict where that is, an how it is formed
        %    \item The discussion are not provided if not at the end because i would be too long 
        %\end{itemize}
        %\begin{itemize}
        %    \item Pk les premier moments de Surface: ? (application) et comments cela s'agence dans les equations moyennées 
        %    Utiliser la solution de RAJA
        %    \item expliquer les retomber sur le modele hybrid 
        %\end{itemize}
    }
    \item A related point is the utility of the transport equations for the moments of particle
    properties. The utility of the first moment of the momentum of the momentum balance is
    immediately obvious - its anti-symmetric part is the balance of angular momentum. Where
    would one need balances for higher moments? Moreover, the main difficulty with these
    equations, as the authors have already found, is that one needs closures for covariances, for
    example between concentration and velocity fluctuations.
    
    \tb{
% Le second moment de la mass !  pour predire les deformation
%    In general if the particle inner motion have n degree of freedom n moments equations are nessesary.
%    However we beleive that in the majority of pratical applications only the first few moments are necessary to compute with a reasonable accuracy the motion of the dispersed and continuous phase.
%    we demonstrate how the leading order deformation of the droplets can be
%obtained thanks to the second-order mass moment equation.
%The remark of the referee motivated us also to discuss the discuss several covariance closure terms that emerge
%in the averaged equations. In particular in the viscous dominated flows the covariance between Therefore, in the Stokes regime, the quantities are uncorrelated with , implying
%that the covariance terms  is equal zero. However, these conclusions no longer hold at finite inertia.
%We also gives a particular emphasis n the co-variance of the velocity which is not zero at non-zero Reynolds number. 
%Alhtough we do not derive a closed form expression for it (due to the well known divergence paradox for this term in dilyte flows) we provide functional form closure for this term.

In general, if the internal motion or shape of the droplets exhibits $n$ degrees of freedom, then $n$ moment equations are required to fully describe their dynamics. 
However, we believe that in most practical applications, only the first few moments are needed to accurately capture the motion of both the dispersed and continuous phases.
In this work, we demonstrate that the leading-order deformation of the droplets can be effectively described using the second-order mass moment equation. 

Motivated by the referee's comments, we also include a discussion of several covariance closure terms that arise in the averaged equations.
In particular, for viscous-dominated flows (i.e., in the Stokes regime), the relevant quantities are uncorrelated, leading to vanishing covariance terms. 
However, this assumption no longer holds at finite inertia, where these covariance terms become non-negligible. 
We place special emphasis on the covariance of velocity, which is non-zero at finite Reynolds numbers.
Although we do not derive an explicit closed-form expression for the velocity covariance—owing to the well-known divergence paradox associated with this term in dilute flows—we provide a functional tensorial closure form that can be used in practical modeling.

    }
    %\begin{itemize}
        %\item  
        %\item In practice the higher moment are mainly needed in the continuous phase averaged equations where they appears anyway and need to be modeled.
        %Citer lhuillier 2009 ou il dit que c'est indispensable.   
        %\item Yes of course the modeling becomes difficult because it is a yet general formulation, but this is the only set of equatio availbe to describe such physics 
    %\end{itemize}}
    \item The discussion of the Lagrangian versus Eulerian description of the dispersed phases is
    misleading - only the equations describing individual particle properties are Lagrangian.
    Once they are volume (or ensemble) averaged, the continuum equations become Eulerian.
    The many references to Lagrangian versus Eulerian descriptions makes it confusing for the
    reader. If I have misunderstood the argument the authors are making, it is likely that an
    average reader would too.
    \tb{Indeed we have replaced the term ``Lagrangian'' by ``Lagrangian-based'' model} %model each time we mention it  averaged eq}
    \item A lot of discussion in the paper is about the equivalence between the 'phase averaged
    equations' and 'particle averaged' transport equations. I see that they do cite the previous
    studies that have shown this equivalence, but I'm wondering what the point to showing the
    equivalence if it has already been done.
    \tb{
    %\begin{itemize}
        %\item 
        In previous studies, the equivalence between the 'phase averaged equations' and 'particle averaged' has been demonstrated only in the context of the momentum equation, without accounting for interfacial properties or mass transfer (although Lhuillier 2001 considered specifically the interfacial terms in a different context).
        In this work, we extend the demonstration to the more general case, including these effects.
        Beyond the technical contribution, our aim is also pedagogical: we seek to clearly illustrate how Eulerian-based and Lagrangian-based formulations are interconnected.%—an aspect that is often overlooked in the literature.
        %In the previous work the equivalence have been demonstrated only for the momentum equation without interfacial properties nor transfer of mass. 
        %Here we propose to re-demonstrated it in the general cases. 
        %Moreover         the objective is rather pedagogical, we want to show how the Eulerian-based and Lagrangian-based equations are connected together since it is often overlooked in the literature

        %\item 
        %\item 
    %\end{itemize}
    }
\end{enumerate}
Minor points: 
\begin{itemize}
    \item In the Abstract and elsewhere, it is stated 'Notably, the non-convective flux inside the inclusion does not appear in the conservation law using this formulation.' Isn't this obvious?
    The non-convective flux will simply move properties from one place to another within a particle.
    \tb{Yes it is obvious however it remains unclear for the interfacial convective fluxes, therefore it seemed important for us to point that out,  espetially Regarding the surface forces}
    \item Abstract and elsewhere: what is the 'distributional form' of the interfacial transport equation.
    \tr{ Jean Lou know the ref $\to$ def a coter de l'eq}
    \item p. 5: 'We also recognize a term related to mass transfer proportional to $(u_\Gamma - u_k)$.' Isn't this is only true if $f = \rho$?
    \tb{Indeed this is not accurate the comments have been remove}
    \item p. 6: 'The ensemble average quantities are assumed to satisfy the following properties …'. Is it an assumption or is it exact?
    \tb{It is an assumption, this is the basic axiom of averaging processes}
    \item After (2.15): Is it probability or probability density?
    \tb{Neither this point has been raised by another Reviewer and has been modified}
    \item  p. 8: 'indexed, $\alpha$,' - delete the commas
    \tb{OK}
    \item  p. 11: 'pioneered by (Lhuillier, 1992)' - should be pioneered by Lhuillier (1992)
    \tb{OK}
\end{itemize}



\end{document}

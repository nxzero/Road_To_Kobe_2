\documentclass[10pt,a4paper]{article}

\usepackage[utf8]{inputenc}
\usepackage{amsmath}
\usepackage{xcolor}
\usepackage{amsfonts}
\usepackage{amssymb}
\usepackage[bmargin=2.cm,tmargin=2.cm,lmargin=2cm,rmargin=2cm]{geometry}
\usepackage{natbib}

\newcommand{\tb}[1]{\color{blue}#1\color{black}}
\newcommand{\tr}[1]{\color{red}#1\color{black}}

\begin{document}

\title{Response to reviewer 1} 
\maketitle
\textbf{Paper entitled "Averaged equations for disperse two-phase flow with interfacial transport. IJMF"}
\bigskip

\section*{Reviewer 1}

\textit{This manuscript derives the averaged equations for disperse multiphase flows. Authors are
certainly aware that this subject has been studied many times before by the cited literature. This
manuscript does have its uniqueness, especially in its treatments of the interface terms and the
moments. However, the authors do not provide an example to show how their method is more
advantageous or reveals something new. As currently written, the derivations are too abstract. To
publish this manuscript, I suggest the author to come up an example in which the closures can be
easily obtained by their methods. For instance, authors can take advantage of their treatment of
the interfaces study thermal capillary motion of bubbles to show the advantage of the method, or
study rotating or deformable particles to show some unique or non-Newtonian behavior..}
\tb{
\begin{itemize}
    \item We propose computing the mean Marangoni drift and stresslet generated by that drift 
\end{itemize}}

We thank the Referee for his/her appreciation of the paper. 
Before providing a point-by-point response to the Referee’s comments, we just wish to mention the main changes introduced in the paper to address the concerns raised by the other referee.



\textit{Following are some technical suggestions.}


\begin{enumerate}
    \item Angular brackets are used for both inner product and averaged quantities. My suggestion is to use round brackets for the inner product to make the manuscript easier to read.
    \item In the first paragraph of sec. 2.2: use “I” to denote the identity tensor, instead of delta to
    clearly distinguish it from the Dirac delta function.
    \item After eq. (2.15) “… with $\phi_I = <\delta_\Gamma>$ the probability of finding the interface at the point x at
    time t”. The probability of finding an interface at a given location is always zero. This
    quantity is not the probability. It is the interface area per unit volume, also called the specific
    area.\\
    \tb{This has been modified}
    \item Please note that although $\nabla_{||} n$ is well defined but not $\nabla n$; therefore, should not be used
    \\
    \tr{To the author it seems fine as demonstrated with Eq (65) of Orlando }
\end{enumerate}


\section*{Reviewer 2}

\textit{In this paper, the closure problem of disperse two-phase flow is derived at an exact mathematical level under general circumstances using ensemble averaging. When compared with previous such derivations (Drew, Jackson, Prosperetti, etc), this paper makes significant progress in two aspects in my opinion. First, it shows that the left-hand side of the sum of the solid phase and interface balance equations $(C_d+C_Gamma)$ is equal to a Taylor-like infinite sum over the left-hand sides of the moments $(M^(i))$ of the particle phase balance equations. In my understanding, previously, it was only established that the balance equation $C_d =0$ is equivalent to the zeroth moment particle phase balance $M^(0)=0$, a trivial result since any two things equal to 0, derived without approximation, are also equal to one another. Based on their finding, the authors argue in favor of a "hybrid model", which I believe they will continue working on in future papers. A second point of significant progress seems to me the author's treatment of the interface balance. It is treated with a mathematical rigor I have not seen before.
Given the complexity of the topic, the authors have done an excellent job in presenting their derivation. I found it very easy to follow the mathematics, except in a few instances where they (probably unintentionally) used different symbols for the same quantity (see comments below). Overall, I find this paper impressive and recommend its publication after just a few minor revisions.}



Minor comments:
\begin{itemize}
    \item  different notations with the same meaning (subscripts 'I' and 'Gamma') are used for interface quantities: for example, at the beginning of section 2.2, just after Eq. (3.4), Eq. (3.30), and elsewhere
    \\
    \tb{this has been corrected}
    \item just before Eq. (2.11): "one obtains" \tb{OK}
    \item writing just below Eq. (3.5), "using Eq. (3.10)", is not a good style as the latter equation has not been introduced at this point. At the very least, write something like, "as will become clear in Section 3.2. (Eq. 3.10)), …" Better, try to avoid this. \tb{OK}
    \item "Note that Eq. (3.8) generalizes the usual expression" \tb{OK}
    \item Eq. (3.12), a '+' too many \tb{OK}
    \item Eq. (3.21) dGamma, not dSigma \tb{OK}
\end{itemize}

\section*{Reviewer 3}

\textit{This paper derives the continuum transport equations that govern two phase systems where
one phase is of the form of dispersed, but not necessarily rigid, particles. The method of
derivation, despite slight differences from previous approaches, appears to be quite standard.
The main extensions that I can detect are that transport are derived for properties associated
with the interfaces between the dispersed and continuous phases. The paper is nicely
introduced, the analysis is clear and appears to be correct - at least the standard two phase
flow equations for bulk properties are recovered. Overall, I like the clarity of exposition and
the connections the authors make with previous papers on the subject.
However, I do have some concerns, which are listed below. I believe most of these concerns
can be addressed if the authors put their minds to it, but it will require some rewriting, clearer
explanation and possibly a better discussion of the main advances.}

\begin{enumerate}
    \item It is not clear what the purpose of the paper is, and in what substantive way it departs or
    extends previous studies, many of which the authors have already cited. Arriving at known
    results by another route might be a useful and rewarding exercise for the person doing it, but
    to qualify as a paper it must make an original contribution. What are the original
    contributions of the paper? To make this clear, I suggest the authors replace or expand the
    last paragraph of the introductory section to outline the new contributions in the paper and
    state in what way the analysis presented departs from previous papers.
    \tb{
        \begin{itemize}
            \item The Lagrangian balance equations are derived with no assumption, hence taking in account (1) the interfacial properties but still with a Lagrangian approches (2) the internal gradient of droplets internal properties through the moments of distribution. 
            \item The connections between Lagrangian and Eulerian averaged approach eventhoug well understood is often overlook and have never been demonstrated completly in such general senarios. 
            \item The equivalence between Lagrangian-averaged law and Eulerian one is rigorously demonstrated, that is an aspect often overlooked in the literature 
            \item Overall we put in the front line the moment equations describing the dispersed phase and reach the major conclusion that regardless the systems studies the hybrid model should be use over the two-fluid model for dispersed flows. 
        \end{itemize}
        \tr{This is more a pedagogical article which propose a clear derivation of the equations and }
    }
    \item One original contribution that I see is the transport equations of interfacial properties. This
    is a reasonable extension, but if it is the only one, the paper does not have to be this long and
    does not have to repeat the derivation of transport equations for bulk properties of the
    dispersed and continuous phases. 
    \tb{Yes it does because bulk and interfaces equations are  interconnected and work as a whole}
    Additionally, what do the authors foresee the interfacial
    transport equations being useful for? For example, for a scalar property such as surfactant
    concentration, transport equations for the zeroth moment is quite simple - one can even write
    them down intuitively. I can see that transport equations for higher moments might be useful,
    for example in computing the mean Marangoni drift of bubbles in a liquid. There isn't much
    discussion of these aspects, which makes one wonder what the motivation is for deriving the
    equations.
    \tb{
        \begin{itemize}
            \item The interfacial equation is crutial to well take in account interfacial momentum jump because of surafce tension (in the first moment of momentum), or partciles surface Energies etc... 
            \item Regarding surfactant the zero order is indeed wuite simple and obvious. The first order balance would represent where does these surfacant are located on the surface and how this location evolve. Indeed contaminated droplets form spherical caps, the first moment balence would predict where that is, an how it is formed
            \item The discussion are not provided if not at the end because i would be too long 
        \end{itemize}
        \begin{itemize}
            \item Pk les premier moments de Surface: ? (application) et comments cela s'agence dans les equations moyennées 
            Utiliser la solution de RAJA
            \item expliquer les retomber sur le modele hybrid 
        \end{itemize}
    }
    \item A related point is the utility of the transport equations for the moments of particle
    properties. The utility of the first moment of the momentum of the momentum balance is
    immediately obvious - its anti-symmetric part is the balance of angular momentum. Where
    would one need balances for higher moments? Moreover, the main difficulty with these
    equations, as the authors have already found, is that one needs closures for covariances, for
    example between concentration and velocity fluctuations.\tb{
    \begin{itemize}
        \item In general if the particles inned motion have N degree of freedom N equations are nessesary. Le second moment de la mass !  pour predire les deformation 
        \item In practice the higher moment are mainly needed in the continuous phase averaged equations where they appears anyway and need to be modeled.
        Citer lhuillier 2009 ou il dit que c'est indispensable.   
        \item Yes of course the modeling becomes difficult because it is a yet general formulation, but this is the only set of equatio availbe to describe such physics 
    \end{itemize}}
    \item The discussion of the Lagrangian versus Eulerian description of the dispersed phases is
    misleading - only the equations describing individual particle properties are Lagrangian.
    Once they are volume (or ensemble) averaged, the continuum equations become Eulerian.
    The many references to Lagrangian versus Eulerian descriptions makes it confusing for the
    reader. If I have misunderstood the argument the authors are making, it is likely that an
    average reader would too.
    \tb{Indeed we propose replacing the ``Lagrangian'' by ``Lagrangian-based'' model each time we speak about averaged eq}
    \item A lot of discussion in the paper is about the equivalence between the 'phase averaged
    equations' and 'particle averaged' transport equations. I see that they do cite the previous
    studies that have shown this equivalence, but I'm wondering what the point to showing the
    equivalence if it has already been done.
    \tb{
    \begin{itemize}
        \item In the previous work the equivalence have been demonstrated only for the momentum equation without interfacial properties nor transfer of mass. 
        Here we propose to re-demonstrated it in the general cases. 
        \item the objective is rather pedagogical, we want to show how the Eulerian-based and Lagrangian-based equations are connected together since it is often overlooked in the literature
        \item 
    \end{itemize}
    }
\end{enumerate}
Minor points: 
\begin{itemize}
    \item In the Abstract and elsewhere, it is stated 'Notably, the non-convective flux inside the inclusion does not appear in the conservation law using this formulation.' Isn't this obvious?
    The non-convective flux will simply move properties from one place to another within a particle.
    \tb{Yes it is obvious however it remains unclear for the interfacial convective fluxes, therefore it seemed important for us to point that out,  espetially Regarding the surface forces}
    \item Abstract and elsewhere: what is the 'distributional form' of the interfacial transport equation.
    \tr{ Jean Lou know the ref $\to$ def a coter de l'eq}
    \item p. 5: 'We also recognize a term related to mass transfer proportional to $(u_\Gamma - u_k)$.' Isn't this is only true if $f = \rho$?
    \tb{Indeed this is not accurate the comments have been remove}
    \item p. 6: 'The ensemble average quantities are assumed to satisfy the following properties …'. Is it an assumption or is it exact?
    \tb{It is an assumption, this is the basic axiom of averaging processes}
    \item After (2.15): Is it probability or probability density?
    \tb{Neither this point has been raised by another Reviewer and has been modified}
    \item  p. 8: 'indexed, $\alpha$,' - delete the commas
    \tb{OK}
    \item  p. 11: 'pioneered by (Lhuillier, 1992)' - should be pioneered by Lhuillier (1992)
    \tb{OK}
\end{itemize}



\end{document}

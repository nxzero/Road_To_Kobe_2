\documentclass[10pt,a4paper]{article}

\usepackage[utf8]{inputenc}
\usepackage{amsmath}
\usepackage{xcolor}
\usepackage{amsfonts}
\usepackage{amssymb}
\usepackage[bmargin=2.cm,tmargin=2.cm,lmargin=2cm,rmargin=2cm]{geometry}
\usepackage{natbib}

\newcommand{\tb}[1]{\color{blue}#1\color{black}}
\newcommand{\tr}[1]{\color{red}#1\color{black}}

\begin{document}

\title{Response to reviewer 2} 
\maketitle
\textbf{Paper entitled "Buoyancy driven motion of non-coalescing inertial drops: microstructure modeling with nearest particle statistics.
Acta Mechanica"}
\bigskip

\textit{This manuscript uses the Volume of Fluid (VoF) method to perform particle-resolved numerical simulations and then study microstructures of droplets. In general, this paper is well-written and is of interest to the multiphase flow community. I recommend its publication after minor revisions listed below.}


\color{blue}
We thank the Referee for his/her positive appreciation of the paper. 
Before providing a point-by-point response to the Referee’s comments, we just wish to mention the main changes 
introduced in the paper to address the concerns raised by the other referee: 
\begin{itemize}
    \item  We modified slightly Figure 4 for clarity of reading. 
\end{itemize}

% Additionally, as discussed before submission with the editor, we updated our DNS results for a more refined set of DNS (from $d/\Delta \approx 20$ to $d/\Delta \approx 25$) as the other set of results exhibited some doubtful results due to meshing problems. 
% The main changes arising from this new set of DNS are the followings, 
% \begin{itemize}
%     \item The bond number is now fixed to $Bo = 0.5$. 
%     \item The maximum \textit{Galileo} number is $80$ instead of $100$.
%     \item The appendix (B.3) have been updated yielding better converged results. 
% \end{itemize}  
% As expected, the results and conclusions made in this study remain unchanged.
% \color{black}

The following points should be improved:

\begin{enumerate}
    \item 
 In the main text, all$\Delta/d$ should be $d/\Delta$. They are correct in Appendix B.

 \tb{This is corrected. }

\item
 In the last paragraph of 4.1, "We can observe that $P_{nst}$ is larger close to the test particle ($r/d = 1$) in the high volume fraction cases than in the low volume fraction cases. In practice, if particles are more likely to be close to one another, it means that densely packed regions of particles are present in the flow. This suggests that isotropic clusters, as represented in Figure 1 (Case 2), are likely to form in the present context." Each of these sentences is correct but putting together seems to imply that large $P_nst$ itself indicates cluster. I would say that clusters happens if $P_nst$ around $r/d = 1$ in the flow is greater than the corresponding value at the same volume fraction with uniform particle distribution (i.e. the value of $P_nst^th(r=d)$ in eq 4.3).

\tb{The authors thank the referee suggestion and have added the following sentences to avoid any confusion: 
``
In practice, if particles are more likely to be close to one another, it means that densely packed regions of particles are present in the flow.
More specifically, if the value of $P_\text{nst}$ at $r=d$ is larger than the value of the corresponding uniform particle distribution at $r=d$  we may stipulate that clusters are present in the flow. 
Quantitative results supporting this assumption will be provided in the following section. 
For instance, we assert that Figure 4 suggests that isotropic clusters, as represented in Figure 1 (\textit{Case 2}), are likely to form in the flow (for $\phi = 0.2$). 
''
}

\item
 The ratio between $P_{nst}^{th}(r)$ and $P_nst^(\phi\ll 1)(r)$ represents the geometric effects of larger particle volume fraction.  Since $P_{nst}^{th}(r)$ has been obtained analytically by Torquato et al. (1990), the ratio is not very interesting to study. Given this paper is about the fluid dynamic effects on particle microstructures, it will be more interesting to plot $P_{(r-nst)}/P_{nst}^{th}$ in Fig. 8. This ratio represents effects of fluid dynamics and particle interactions.

 \tb{
    Thank for the valuable suggestion. 
    We have revised the analysis to more clearly highlight the effects of fluid dynamics on the final pair distribution. 
    As recommended, we have modified the entire paragraph and updated Figure 8 (which now corresponds to Figure 9) to plot $P_{(r-nst)}/P_{nst}^{th}$ instead of $P_{(r-nst)}/P_{nst}^{\phi\ll 1}$.
    Additionally, we included a plot of $P_{nst}^{th}$ (Fig 8) for various $\phi$ to assist readers who may not be familiar with nearest neighbor distributions. 
 }

\end{enumerate}

\end{document}

\section{The original problem with classic pair stats}

we should have a factor of $1/N$ or something in front of the int 


\section{Strategy}

As in the previous chapter we reformulate teh ensemble average of the partcile phase velocity variance, 
\begin{equation}
    \pavg{\textbf{u}_\alpha'\textbf{u}_\alpha'}
\end{equation}
in terms of the nearest particle statistics conditioned quantities. 
We use (2.15) of \citet{zhang2021ensemble}, namely, 
\begin{equation}
    \int_{\mathbb{R}^3}
    \sum_{j\neq i}
    \delta(\textbf{x}_j[\FF,t] - \textbf{y}) h_{ij}[t,\FF] d\textbf{y}
    = 1
\end{equation}
And introduce the relation, 
\begin{equation}
    \pavg{\textbf{u}_\alpha'\textbf{u}_\alpha'}
    = 
    n_p[\textbf{x},t]
    \int_{\mathbb{R}^3}
    (\textbf{v}^\text{nst}_p
    \textbf{v}^\text{nst}_p)[\textbf{x},\textbf{y},t]
    P_\text{nst}[\textbf{y}|\textbf{x},t]
    d\textbf{y}
    + 
    \int_{\mathbb{R}^3}
    \pavg{
        \sum_{j\neq i }
        \delta_j 
        h_{ij} 
        (\textbf{u}_\alpha'' - \textbf{u}^\text{nst}_p)
        (\textbf{u}_\alpha'' - \textbf{u}^\text{nst}_p)
    }d \textbf{y}
\end{equation}
where we recall that $P_{nst}$ is the probability of finding the nearest neighbor at the position of \textbf{r} knowing that the particle is present at \textbf{x} at time $t$. 
$\textbf{v}^\text{nst}_p = \textbf{u}_p^\text{nst} - \textbf{u}_p$ where $\textbf{u}_p^\text{nst}$ si the mean particle velocities with nearest neighbor at $\textbf{r}$. 
And $\textbf{u}_\alpha'' = \textbf{u}_\alpha - \textbf{u}^\text{nst}_p$ is the particle velocity fluctuation around the conditional mean $\textbf{u}^\text{nst}_p$. 

where we have defined the conditional velocity, 
\begin{align}
    \textbf{u}_p^\text{nst} P_\text{nst}[\textbf{y},\textbf{x},t]
    &= \frac{1}{v_p} \avg{\sum_{i,j\neq i}^N \delta(\textbf{x}_i - \textbf{x})\delta(\textbf{x}_j - \textbf{y}) h_{ij} \intO[i]{\textbf{u}_d^0[\textbf{z},\FF,t]}(\textbf{z})} \\
    &= \frac{1}{v_p} \int_{|\textbf{x} - \textbf{z}|<a}\avg{\sum_{i,j\neq i}^N \delta(\textbf{x}_i - \textbf{x})\delta(\textbf{x}_j - \textbf{x} - \textbf{r}) h_{ij} \textbf{u}_d^0[\textbf{z},\FF,t]}d\textbf{z} \\
    &= \frac{P_\text{nst}[\textbf{x},\textbf{y},t] }{v_p} \int_{|\textbf{x} - \textbf{z}|<a} \textbf{u}_d^\text{nst}[\textbf{z}|\textbf{x},\textbf{r},t] d\textbf{z} \\
\end{align}


\section{How to derive of the conditional particle velocity}

Bring the problem to the two particle conditional field. 
This is done by making a balence between the velocity and the buoyancy. 


The eveolution of teh pdf can be obtained as, 
\begin{align*}
    \pddt \delta_i + \textbf{u}_i \cdot \grad \delta_i = 0\\
    \pddt \delta_j + \textbf{u}_i \cdot \grad \delta_j  + \textbf{u}_{ji}\cdot \pddr \delta_j = 0\\
\end{align*}
Multiplying one by another and adding the eq gives,  
\begin{align*}
    \delta_j(\pddt \delta_i + \textbf{u}_i \cdot \grad \delta_i) + \\
    \delta_i (\pddt \delta_j + \textbf{u}_i \cdot \grad \delta_j  + \textbf{u}_{ji}\cdot \pddr \delta_j) = 0\\
\end{align*}
\begin{equation}
    (\pddt 
    + \textbf{u}_i \cdot \grad 
    % +  \textbf{u}_i \cdot \pddr(\delta_i \delta_j)  
    + \textbf{u}_{ji} \cdot \pddr) (\delta_i \delta_j) = D_t(\delta_i \delta_j) = 0
\end{equation}
including $h_{ij}$ in the eq gives, 
\begin{equation}
    D_t(h_{ij} \delta_i \delta_j )
    = \delta_i \delta_j \pddt h_{ij}
\end{equation}
For simplicity we will consider, 
\begin{equation}
    \delta_\text{nst}\to 
    \sum_{i,j\neq i = 1}^N \delta(\textbf{x}_i-\textbf{x})\delta(\textbf{x}_j-\textbf{x} - \textbf{r}) h_{ij}
\end{equation}
Averaging this, 
givesn 
\begin{equation}
    \pddt (\delta_\text{nst} - P_\text{nst}) 
    + \div (\delta_\text{nst} \textbf{u}_i - \textbf{u}_p^\text{nst}P_\text{nst})
    + \pddr \cdot(\delta_\text{nst} \textbf{u}_{ij} -  \textbf{u}_{pq}^\text{nst}P_\text{nst})
    = (\delta_i \delta_j \pddt h_{ij} -  \avg{\delta_i \delta_j \pddt h_{ij}})
\end{equation}
Note that averaging this equation gives $0$. 

Evaluating the NS equation at \textbf{z} gives, 
\begin{align}
    \div \textbf{u}^0 = 0, \\
    \pddt \textbf{u}^0
    + \div (\textbf{u}^0\textbf{u}^0 )
    = 
    \div \bm\sigma^*
    + \textbf{g}
    +\frac{\zeta-1}{\zeta\rho_f}(\bm\sigma_f^0\cdot \textbf{n})\delta_\Gamma,
\end{align}
Notice that, 
\begin{equation}
    \avg{(\delta_{nst} - P_\text{nst}) \textbf{u}^0}
    = P_\text{nst}(\textbf{u}^\text{nst}[\textbf{z},\textbf{x},\textbf{y},t] - \textbf{u})
\end{equation}
Thus Multiplying by $\delta_{nst} - P_\text{nst}$ and averaging gives the conservation eq for $\textbf{v}^\text{nst}$. 


\begin{align}
    P_\text{nst} \pddz \cdot \textbf{v}^{nst}
    = 0 
    \\
    \pddt (P_\text{nst} \textbf{v}^\text{nst})
    + \pddz \cdot\avg{(\textbf{u}^0 \textbf{u}^0 )(\delta_\text{nst} - P_\text{nst})} 
    + \div \avg{(\delta_\text{nst} \textbf{u}_i  - \textbf{u}_p P_\text{nst})\textbf{u}^0 } \nonumber \\ 
    + \pddy\cdot\avg{(\delta_\text{nst} \textbf{u}_{ij}  - \textbf{u}_{pd} P_\text{nst})\textbf{u}^0 }
    % + \pddw\cdot \avg{(\delta_\text{nst}\textbf{a}_i - P_\text{nst}\textbf{a}_p) \textbf{u}^0}
    =  
    \pddz\cdot\avg{(\delta_\text{nst} - P_\text{nst}) \bm\sigma^*}
    + \frac{1 - \zeta}{\zeta\rho_f}\avg{ (\delta_\text{nst} - P_\text{nst}) \delta_\Gamma \bm\sigma_f^0\cdot \textbf{n}}
    + \avg{\textbf{u}^0 S_\text{nst}},
\end{align}
The stress might be reformulated as before, 
\begin{multline}
    \avg{\bm\sigma^* (\delta_\text{nst} - P_\text{nst})} 
    = 
    % \avg{\chi_f \bm\sigma^0_f (\delta_\text{nst} - P_\text{nst})}/\rho_f 
    % + \avg{[\chi_d \bm\sigma^0_d  + \delta_\Gamma \bm\sigma^0_\Gamma] (\delta_\text{nst} - P_\text{nst})}/\rho_d
    % \nonumber \\
    % &= 
    % - P_\text{nst} [
    %     \phi_f^\text{nst} p_f^\text{nst}
    %     - \phi_f p_f
    % ]\bm\delta
    % + P_\text{nst} \mu_f [\grad \textbf{v}^\text{nst}+(\grad \textbf{v}^\text{nst})^\dagger] \nonumber \\
    % &+ \avg{[\chi_d (\bm\sigma_d^0 - 2 \mu_f \textbf{e}^0_d ) + \chi_\Gamma \bm\sigma_\Gamma ]  (\delta_\text{nst} - P_\text{nst})}/\zeta \nonumber \\
    % &= 
    - P_\text{nst}  p_{f\rho}^\text{nst-d}\bm\delta
    + P_\text{nst} \nu_f [\grad \textbf{v}^\text{nst}+(\grad \textbf{v}^\text{nst})^\dagger] \\
    +P_\text{nst} [\phi_d^\text{nst-d} p_{f\rho}
    - \phi_d^\text{nst} p_{f\rho}^\text{nst-d}] \bm\delta
    + \avg{[\chi_d (\bm\sigma_d^0 - 2 \mu_f \textbf{e}^0_d \zeta) + \chi_\Gamma \bm\sigma_\Gamma ]  (\delta_\text{nst} - P_\text{nst})} /\rho_d \nonumber
\end{multline}
where $p_{\rho f} = p_f/ \rho_f$ and $\nu_f = \mu_f/\rho_f$. 
It is clear that all the terms on the left-hand side will vanish in stokes regime,


Using the moments equaitons etc ,
\begin{align}
    P_\text{nst} \pddz \cdot  \textbf{v}^\text{nst} = 0 \\
    \pddt (P_\text{nst} \textbf{v}^\text{nst})
    + P_\text{nst} \pddz \cdot  (
     \textbf{v}^\text{nst} \textbf{v}^\text{nst}  
    + \textbf{u} \textbf{v}^\text{nst} 
    + \textbf{v}^\text{nst} \textbf{u} 
    )\nonumber\\
    + \pddx\cdot(P_\text{nst} \textbf{u}_p \textbf{v}^\text{nst} + \avg{\delta_1 \textbf{u}_{i}' \textbf{u}^0} )
    + \pddr\cdot(P_\text{nst} \textbf{u}_{pq} \textbf{v}^\text{nst} + \avg{\delta_1 \textbf{u}_{ij}' \textbf{u}^0} )\\
    = \frac{P_\text{nst}}{\rho_f}[
        - \pddz p_f^\text{nst}
        + \mu_f \pddz^2 \textbf{v}^\text{nst}
    ]
    +\pddz\cdot \bm\sigma_\text{eff}^\text{nst}
    + \left(\frac{1-\zeta}{\zeta\rho_f}\right) \pavg{ (\delta_\text{nst} - P_\text{nst}) \intO{\bm\sigma_f^0\cdot \textbf{n}}}
    + \avg{\textbf{u}^0 S_\text{nst}},
    % \label{eq:NS_dilute_inertiel}
\end{align}
\begin{multline*}
    \bm\sigma^\text{nst}_\text{eff}
    = + \pavg{\delta_\text{nst} \textbf{u}''\textbf{u}''}
    - P_\text{nst} \pavg{ \textbf{u}'\textbf{u}'}
    - \frac{1}{\rho_f }P_\text{nst} [\phi_d^\text{nst-d} p_f
    - \phi_d^{nst} p_f^\text{nst-d}]\bm\delta \\
    -  \pavg{(\delta_\text{nst} - P_\text{nst}) \intO{\textbf{w}_d^0\textbf{w}_d^0 }}
    +  \frac{1}{2}\pavg{(\delta_\text{nst} - P_\text{nst})\frac{d^2}{dt^2} \intO{\textbf{rr}} } \\
    - \frac{1}{\rho_f}\pavg{(\delta_\text{nst} - P_\text{nst}) \intS{\left[
        \textbf{r}\bm{\sigma}_f^0 \cdot \textbf{n}
        -  2 \mu_f (\textbf{u}_f^0 \textbf{n} + \textbf{n} \textbf{u}_f^0)
        \right] 
    }}
    + \frac{1}{2}\div\pavg{(\delta_\text{nst} - P_\text{nst} )\ldots}
\end{multline*}



\subsubsection*{Stokes and dilute regime}

In the stokes regime the particle phase equaiton directly gives us a closure 
\begin{equation}
    \pavg{ (\delta_\text{nst} - P_\text{nst}) \intO{\bm\sigma_f^0\cdot \textbf{n}}}
    = 
    -  \pavg{ (\delta_\text{nst} - P_\text{nst}) m_\alpha \textbf{g}}
    % = 
    % - \rho_d v_p \textbf{g} \pavg{ (\delta_\text{nst} - P_\text{nst})}
    = 
    - \rho_d v_p \textbf{g} n_p^\text{nst-d} P_\text{nst}
\end{equation}
where $n_p^\text{nst-d}$ is teh probability of finding a particle in \textbf{z}, knowing that a particle is present in \textbf{x} with its NN at \textbf{y}
Note that, 
\begin{equation}
    - \left(1-\zeta\right)  v_p \textbf{g} n_p^\text{nst-d} P_\text{nst}
\end{equation}

Then, the final eq is, 
\begin{align}
    P_\text{nst} \pddz \cdot  \textbf{v}^\text{nst} = 0 \\
    \frac{P_\text{nst}}{\rho_f}[
        - \pddz p_f^\text{nst}
        + \mu_f \pddz^2 \textbf{v}^\text{nst}
    ]
    = 
    -\pddz\cdot \bm\sigma_\text{eff}^\text{nst}
    + \left(1-\zeta\right)  v_p \textbf{g} n_p^\text{nst-d} P_\text{nst}
    \label{eq:NS_dilute_inertiel}
\end{align}
\begin{multline*}
    \bm\sigma^\text{nst}_\text{eff}
    =
    - \frac{P_\text{nst}}{\rho_f } [\phi_d^\text{nst-d} p_f
    - \phi_d^{nst} p_f^\text{nst-d}]\bm\delta \\
    - \frac{1}{\rho_f}\pavg{(\delta_\text{nst} - P_\text{nst}) \intS{\left[
        \textbf{r}\bm{\sigma}_f^0 \cdot \textbf{n}
        -  2 \mu_f (\textbf{u}_f^0 \textbf{n} + \textbf{n} \textbf{u}_f^0)
        \right] 
    }},
    + \frac{1}{2}\div\pavg{(\delta_\text{nst} - P_\text{nst} )\ldots}
\end{multline*}
note that the stresslet should compensate teh pressure term


Note that all of our closure terms / source terms are of the form, 
\begin{equation}
    \pavg{(\delta_\text{nst} - P_\text{nst}) \ldots}
    = \avg{
        \sum_\alpha^N 
        \sum_i^N 
        \sum_{j\neq i}^N
        \delta(\textbf{x}_\alpha - \textbf{z})
        \delta(\textbf{x}_i - \textbf{x})
        \delta(\textbf{x}_j - \textbf{y}) 
        h_{ij} \ldots
    }
\end{equation}
This can be re-formulated as, 
\begin{align*}
    \pavg{\delta_\text{nst} \ldots}
    &= 
    \avg{
        \sum_i^N 
        \sum_{j\neq i}^N
        \delta(\textbf{x}_{\alpha = i } - \textbf{z})
        \delta(\textbf{x}_i - \textbf{x})
        \delta(\textbf{x}_j - \textbf{y}) 
        h_{ij} \ldots
    } \\
    &+ 
    \avg{
        \sum_i^N 
        \sum_{j\neq i}^N
        \delta(\textbf{x}_{\alpha = j} - \textbf{z})
        \delta(\textbf{x}_i - \textbf{x})
        \delta(\textbf{x}_j - \textbf{y}) 
        h_{ij} \ldots
    } \\
    &+ 
    \avg{
        \sum_{\alpha=i,j}^N 
        \sum_i^N 
        \sum_{j\neq i}^N
        \delta(\textbf{x}_\alpha - \textbf{z})
        \delta(\textbf{x}_i - \textbf{x})
        \delta(\textbf{x}_j - \textbf{y}) 
        h_{ij} \ldots
    } \\
    &= 
    \delta(\textbf{x} - \textbf{z})
    \avg{
        \sum_i^N 
        \sum_{j\neq i}^N
        \delta(\textbf{x}_i - \textbf{x})
        \delta(\textbf{x}_j - \textbf{y}) 
        h_{ij} \ldots
    } \\
    &+ 
    \delta(\textbf{y} - \textbf{z})
    \avg{
        \sum_i^N 
        \sum_{j\neq i}^N
        \delta(\textbf{x}_i - \textbf{x})
        \delta(\textbf{x}_j - \textbf{y}) 
        h_{ij} \ldots
    } 
    \\
    &+ 
    \avg{
        \sum_{\alpha=i,j}^N 
        \sum_i^N 
        \sum_{j\neq i}^N
        \delta(\textbf{x}_\alpha - \textbf{z})
        \delta(\textbf{x}_i - \textbf{x})
        \delta(\textbf{x}_j - \textbf{y}) 
        h_{ij} \ldots
    } \\
\end{align*}
The first two sum are therefor proportional to $\delta(\textbf{x} - \textbf{y})P_\text{nst}$ and $\delta(\textbf{x} - \textbf{y})P_\text{nst}$ which is not negligible and the last $n_p^\text{nst}P_\text{nst}$,
Thus at $\mathcal{O}(\phi^2)$ we have, 
\begin{equation}
    \left(1-\zeta\right)  v_p \textbf{g} n_p^\text{nst-d} P_\text{nst}
    \approx 
    \left(1-\zeta\right)  v_p \textbf{g} P_\text{nst} [\delta(\textbf{z} - \textbf{y}) + \delta(\textbf{z} - \textbf{x}) - n_p H(|\textbf{y} - \textbf{x}| - |\textbf{z} - \textbf{x}|)]  
\end{equation}
Note that while the last term seems $\mathcal{O}((n_p)^3)$ it turns out that it is not since the integrated volume is $\sim L^3$ and can be arbitrary large and s. 

Accounting for that methode we may write, 
\begin{align*}
    P_\text{nst} \pddz \cdot  \textbf{v}^\text{nst} = 0 \\
    \frac{P_\text{nst}}{\rho_f}[
        - \pddz p_f^\text{nst}
        + \mu_f \pddz^2 \textbf{v}^\text{nst}
    ]
   & = 
    \left(1-\zeta\right)  v_p \textbf{g} P_\text{nst} [\delta(\textbf{z} - \textbf{y}) + \delta(\textbf{z} - \textbf{x}) - n_p H(|\textbf{y} - \textbf{x}| - |\textbf{z} - \textbf{x}|)]  \\
    &+P_\text{nst}\pddz\cdot  [\delta(\textbf{z} - \textbf{x}) \textbf{S}_p^\text{nst-x}
    +\delta(\textbf{z} - \textbf{x}) \textbf{S}_p^\text{nst-y}
    - n_p \textbf{S}_p H(|\textbf{y} - \textbf{x}| - |\textbf{z} - \textbf{x}|)
    ]\\
    &+P_\text{nst}\frac{1}{2}\pddz\pddz:   [\delta(\textbf{z} - \textbf{x}) \textbf{S}_{2p}^\text{nst-x}
    +\delta(\textbf{z} - \textbf{x}) \textbf{S}_{2p}^\text{nst-y}
    - n_p \textbf{S}_{2p} H(|\textbf{y} - \textbf{x}| - |\textbf{z} - \textbf{x}|)
    ]\\
   & = 
    \left(1-\zeta\right)  v_p \textbf{g} P_\text{nst} [\delta(\textbf{z} - \textbf{y}) + \delta(\textbf{z} - \textbf{x}) - n_p H(|\textbf{y} - \textbf{x}| - |\textbf{z} - \textbf{x}|)]  \\
    &+P_\text{nst}  [\bm\delta'(\textbf{z} - \textbf{x})\cdot  \textbf{S}_p^\text{nst-x}
    +\bm\delta'(\textbf{z} - \textbf{x}) \cdot  \textbf{S}_p^\text{nst-y}
    - n_p \textbf{S}_p \cdot \textbf{H}'(|\textbf{y} - \textbf{x}| - |\textbf{z} - \textbf{x}|)
    ]\\
    &+P_\text{nst}\frac{1}{2}   [\bm\delta''(\textbf{z} - \textbf{x}) :  \textbf{S}_{2p}^\text{nst-x}
    +\bm\delta''(\textbf{z} - \textbf{x}) : \textbf{S}_{2p}^\text{nst-y}
    - n_p \textbf{S}_{2p} : \textbf{H}''(|\textbf{y} - \textbf{x}| - |\textbf{z} - \textbf{x}|)
    ]\\
    &= \sum_n [\bm\delta^{(n)}(\textbf{z} - \textbf{x}) \odot   \textbf{S}_{(n)p}^\text{nst-x}
    + \bm\delta^{(n)}(\textbf{z} - \textbf{y}) \odot   \textbf{S}_{(n)p}^\text{nst-y}
    + \bm H^{(n)}(|\textbf{y} - \textbf{x}| - |\textbf{z} - \textbf{x}|) \odot   \textbf{S}_{(n)p} n_p]
\end{align*}
\begin{multline}
    % P_\text{nst} \pddz \cdot  \textbf{v}^\text{nst} = 0 
    \frac{P_\text{nst}}{\rho_f}[
        - \pddz p_f^\text{nst}
        + \mu_f \pddz^2 \textbf{v}^\text{nst}
    ]\\
    = \sum_n [\bm\delta^{(n)}(\textbf{z} - \textbf{x}) \odot   \textbf{S}_{(n)p}^\text{nst-x}
    + \bm\delta^{(n)}(\textbf{z} - \textbf{y}) \odot   \textbf{S}_{(n)p}^\text{nst-y}
    + \bm H^{(n)}(|\textbf{y} - \textbf{x}| - |\textbf{z} - \textbf{x}|) \odot   \textbf{S}_{(n)p} n_p]
\end{multline}
The conclusion of this study is that the fields $\textbf{v}^\text{nst}[\textbf{z},\textbf{x},\textbf{y}]$ is equivalent to a forced velocity fields, by to dipole, with the additional boby force related to the partcile free zone. 
At this stage only the zeroth moment is known but the other might be computed with the method of reflexion. 



\section{derivation of the conditional average with the method of reflexion}

\tb{we must start from the condiitonal nearest particle equation and explain it reduces to the eq of a single part}

According to \citet{kim2013microhydrodynamics,zhang2021ensemble} teh particle phase velcoity fluctuation can be obtained directly with the faxen laws.
Following the notation of \citet{kim2013microhydrodynamics} we note $\textbf{v}_1$ and $\textbf{v}_2$ the velocity fields generated by the particle at \textbf{x} and at \textbf{y} respectively. 
The external body forces inducing the motion of the particles are noted $\textbf{b}$ and $\textbf{b}_2$. 
For instance, we keep them arbitrary however note that in the DNS $\textbf{b}_1 = \textbf{b}_2 = m\textbf{g} = \textbf{b}$. 

The substantial difference between this analysis and \citet{kim2013microhydrodynamics} analysis is that $\textbf{v}_2$ is the nearest particle averaged.
Meaning that, $\textbf{v}_2$ is given by, 
\begin{equation}
    \textbf{v}_2 = 
    \textbf{b} \cdot \left[
        1
        + \frac{\lambda}{2(3\lambda +2)}\grad^2
    \right]\frac{\mathcal{G}(\textbf{z},\textbf{y})}{8a\pi\mu_f}
    - \phi \frac{\textbf{b}}{4 \pi \mu_f a} |\textbf{z} - \textbf{y}|^2 
\end{equation}
where all the distance have been made dimenisonless by $a$. 
and, 
\begin{align}
    \textbf{b} = 2 \pi \mu_f a \left(\frac{2+3\lambda}{1+\lambda}\right) \textbf{U}_1^{(0)}\\
    \textbf{b} = 2 \pi \mu_f a \left(\frac{2+3\lambda}{1+\lambda}\right) \textbf{U}_2^{(0)}
\end{align}
with $\textbf{k}_2 = \textbf{b}_2/U$. 
We recall that, 
\begin{equation}
    \mathcal{G}(\textbf{z},\textbf{y})
    = \frac{\bm\delta}{r}+\frac{\textbf{rr}}{r^3}
\end{equation}
with $\textbf{r} = |\textbf{z} - \textbf{x}|$. 
Notice and it will be useful for the following analysis that the gradient of $\mathcal{G}$ and $r$ over its first variable reads as, 
\begin{align}
    \partial_k \mathcal{G}_{ij}
    = 
    \frac{( - \delta_{ij}r_k+ \delta_{ki}r_j +\delta_{kj}r_i)}{r^3}
    - 3\frac{r_ir_jr_k}{r^5}\\
    \partial_{kl} \mathcal{G}_{ij}
    = 
    \frac{( - \delta_{ij}\delta_{kl}+ \delta_{ki}\delta_{jl} +\delta_{kj}\delta_{il})}{r^3}
    - 3\frac{( - \delta_{ij}r_kr_l+ \delta_{ki}r_jr_l +\delta_{kj}r_ir_l)}{r^5}\\
    - 3\frac{(\delta_{il}r_jr_k + r_ir_k \delta_{jl} + r_ir_j\delta_{kl})}{r^5}
    + 15\frac{r_ir_jr_kr_l}{r^7}
    \\
    \grad^2 \mathcal{G}_{ij}
    = 
    \frac{2 \bm\delta}{r^3}
    - 6\frac{\textbf{rr}}{r^5}\\
    \partial_k \grad^2 \mathcal{G}_{ij}
    = 
    - 6\frac{(\delta_{ij} r_k + \delta_{ki} r_j+ r_i \delta_{jk})}{r^5}
    + 30\frac{r_ir_jr_k}{r^7}
    \\
    \grad^4 \mathcal{G}_{ij}
    = 
   0
    \\
    % - 3\frac{(r_jr_i + r_ir_j + r_ir_j3)}{r^5}
    % + 15\frac{r_ir_j}{r^5}
    % \\
    \partial_k r^2 = 2 r_k\\
    \partial_{kl} r^2 = 2 \delta_{kl}
\end{align}
Regarding, $\textbf{v}_1$ it can be considered at the lowest order as the one of an isolated particle. 
\begin{equation}
    \textbf{v}_1 = 
    \textbf{b} \cdot \left[
        1
        + \frac{\lambda}{2(3\lambda +2)}\grad^2
    \right]\frac{\mathcal{G}(\textbf{z},\textbf{x})}{8\pi\mu_f}. 
    % + \phi A |\textbf{x} - \textbf{y}|^2 \textbf{k}_2
\end{equation}

At the first order in reflextion the velocity of the dorplet at \textbf{x} knowing its nearest neighbor is at \textbf{y}, can be computed according to faxen law as, 
\begin{align}
    2 \pi \mu_f a \textbf{U}_1^{(1)}
    &= \left(1 + \frac{\lambda}{2(3\lambda +2)}\grad^2\right)\textbf{v}_2|_{\textbf{z} = \textbf{x}}
\end{align}
where $\textbf{U}^{(1)}$ is the first reflexion. 
From the expression of $\mathbf{G}$ and its derivative evaluated at $\textbf{x}$ we may write, 
\begin{align}
    2 \pi \mu_f a \textbf{U}_1^{(1)}
    &= 
    \frac{\textbf{b}}{4}\cdot\left(1+ \frac{\lambda}{2(3\lambda +2)}\grad^2\right) \left\{\left(1 + \frac{\lambda}{2(3\lambda +2)}\grad^2\right)\mathcal{G}|_{\textbf{z} = \textbf{x}}  - 2 \phi r^2 \right\} \\
    &= 
    \frac{\textbf{b}}{4}\cdot\left(1+ \frac{\lambda}{3\lambda +2}\grad^2+ \frac{\lambda^2}{4(3\lambda +2)^2}\grad^4\right)\mathcal{G}|_{\textbf{z} = \textbf{x}}
    - \phi \textbf{b} (\frac{r^2}{2} +1 )\\
    &=\frac{\textbf{b}}{4}\cdot\left\{ \frac{\bm\delta}{r}+ \frac{\textbf{rr}}{r^3}+ \frac{2\lambda}{3\lambda +2} \left(
        \frac{\bm\delta}{r^3} - 3 \frac{\textbf{rr}}{r^5}
    \right)
    - \phi \bm\delta (\frac{r^2}{2} +1 )
    \right\}
\end{align}

In other word, 
\begin{align}
    \textbf{U}_1^{(1)}
    &=\left(\frac{2+3\lambda}{\lambda +1}\right)\frac{    \textbf{U}_1^{(0)}    }{4}\cdot\left\{ \frac{\bm\delta}{r}+ \frac{\textbf{rr}}{r^3}+ \frac{2\lambda}{3\lambda +2} \left(
        \frac{\bm\delta}{r^3} - 3 \frac{\textbf{rr}}{r^5}
    \right)
    - \phi \bm\delta (\frac{r^2}{2} +1 )
    \right\}
\end{align}




The particle phase pdf is given in dimensionless form by, 
\begin{equation*}
    P_\text{nst}[\textbf{y}|\textbf{x}]
    =
    n_p e^{- \phi (r^3 - 8)}
\end{equation*}

Integrating the first int we have, 
\begin{equation*}
    \pavg{\textbf{u}_\alpha'\textbf{u}_\alpha'}
    = 
    n_p[\textbf{x},t]
    \int_{\mathbb{R}^3}
    (\textbf{v}^\text{nst}_p
    \textbf{v}^\text{nst}_p)[\textbf{x},\textbf{y},t]
    P_\text{nst}[\textbf{y}|\textbf{x},t]
    d\textbf{y}
\end{equation*}
which gives, 
\begin{equation*}
    \pavg{\textbf{u}_\alpha'\textbf{u}_\alpha'} / n_p 
    = 
\end{equation*}


\section*{Comparaison of the distribution}
\section*{Comparaison of the granular temperature}

\subsection{Evolution of the nearest PDF eulerian. }
% In this section we wish to derive the evolution equation of the state funciton $\Pi(\textbf{x},t,\textbf{r}) = \chi_k(\textbf{x},t,\CC) \delta(\textbf{x} + \textbf{r}- \textbf{x}_i(t,\CC))h_i(\textbf{x},t,\CC)$ where $\chi_k(\textbf{x},t,\CC)$ is the phase indicator function of teh phase $k$. 
% Second option 
% \begin{align*}
%     \pddt \delta^i + \pddr \cdot (\textbf{u}_i \delta^i) &= \pddt \delta^i + \pddx \cdot (\textbf{u}_i \delta^i)  = 0\\
%     \pddt \chi_k +  \textbf{u}_I^0 \cdot \partial_\textbf{x}  \chi_k &= 0 \\
%     \pddt (\delta^i h_i) + \pddr \cdot (h_i \textbf{u}_i \delta^i) &= \delta_i \pddt h_i  \\
%     \pddt (\Pi) + \textbf{u}_i  \cdot \pddr\Pi &= \delta_i \pddt h_i  \\
%     \pddt (\Pi) + \textbf{u}_i  \cdot \pddx\Pi &= \delta_i (\pddt h_i + \textbf{u}_i \cdot \pddx h^i)  \\
%     \textbf{u}_i \cdot \pddx \Pi &= \textbf{u}_i \cdot (\delta_i \pddx h_i + \pddr \Pi)
% \end{align*}
In this section we wish to derive the evolution equation of the state funciton $\Pi(\textbf{x},t,\textbf{r},a) =  \delta(\textbf{x} + \textbf{r}- \textbf{x}_i(t,\CC))h_i(\textbf{x},t,\CC)\delta(t - a-t^i(\CC))$. 
Second option 
\begin{align*}
    \pddt \delta^i + \textbf{u}_i \cdot \pddr \delta^i &= \pddt \delta^i + \textbf{u}_i \cdot \pddx \delta^i  = 0\\
    \pddt (h_i \delta^i) + \textbf{u}_i \cdot \pddr (h_i \delta^i) 
    &= \delta_i \pddt h_i\\
    \pddt (h_i \delta^i) + \textbf{u}_i \cdot \pddx (h_i \delta^i) 
    &= \delta_i (\pddt h_i + \textbf{u}_i \cdot \pddx h_i)\\
\end{align*}
using the fact that $\pddt \delta_a = \pdda \delta_a$ we obtain 
\begin{align*}
    \pddt (h_i \delta^i \delta_a) + \pdda ( h_i \delta^i \delta_a) + \textbf{u}_i \cdot \pddx (h_i \delta^i\delta_a) 
    &=  \delta_a \delta_i (\pddt h_i + \textbf{u}_i \cdot \pddx h_i)\\
\end{align*}
alternatively we could have, 
\begin{align*}
    \pddt (h_i \delta^i \delta_a) 
    + \pdda (h_i \delta^i \delta_a) 
    + \pddr\cdot  ( \textbf{u}_i h_i \delta^i\delta_a) 
    &= \delta_a \delta_i \pddt h_i\\
\end{align*}
\begin{align*}
    \pddt \Pi
    + \pdda \Pi
    + \pddr\cdot  ( \textbf{u}_i \Pi) 
    &= \delta_a \delta_i \pddt h_i\\
\end{align*}
which seems easier. 
From the definition of $h_i$,
\begin{equation*}
    h_{i}(\textbf{x},t) 
    = \frac{1}{N_\textbf{x}}
    \prod_{j \neq i}
    H(|\textbf{x}_j - \textbf{x}| - |\textbf{x}_i - \textbf{x}|)
    = \frac{1}{N_\textbf{x}}
    \prod_{j \neq i}
    H(r_{ij})
\end{equation*}
\begin{equation*}
    N_\textbf{x}
    = 
    \sum_{i}
    \prod_{j\neq i}
    H(|\textbf{x}_j - \textbf{x}| - |\textbf{x}_i - \textbf{x}|)
    = 
    \sum_{i}
    \prod_{j\neq i}
    H(r_{ij})
\end{equation*}
 we can demonstrate that : 
\begin{align*}
    \pddt h_i
    &=
    \sum_{e \neq i} (\textbf{u}_e \cdot \hat{\textbf{r}}_e - \textbf{u}_i \cdot \hat{\textbf{r}}_i)
    \delta(|\textbf{x}_e - \textbf{x}| - |\textbf{x}_i - \textbf{x}|) h_i\\
    \pddx h_i
    &= \delta(|\textbf{x}_e - \textbf{x}| - |\textbf{x}_i - \textbf{x}|) h_i
\end{align*}

By Applying the ensemble average on the previous transport equaiton and defining $P_{nst}(\textbf{x},\textbf{r},t,a) = \int \Pi \PP$ we obtain,
\begin{align*}
    \pddt P_{nst}
    + \pdda P_{nst}
    + \pddr\cdot  ( \nstavg{\textbf{u}_i} P_{nst}) 
    &= \delta(a)P_{nst}(\textbf{x},\textbf{r},t,a) - \frac{P_{nst}(\textbf{x},\textbf{r},t,a)}
    {\tau_d(\textbf{x},\textbf{r},t,a)}\\
\end{align*} 
with, 
\begin{equation}
    \delta(a)P_{nst}(\textbf{x},\textbf{r},t,a)
    = 
    \int
    \sum_i \delta_i \delta_a \sum_{e \neq i} (\textbf{u}_e \cdot \hat{\textbf{r}}_e - \textbf{u}_i \cdot \hat{\textbf{r}}_i)
    \delta(|\textbf{x}_e - \textbf{x}| - |\textbf{x}_i - \textbf{x}|)^+ h_i
    \PP
\end{equation}
\begin{equation}
    \frac{P_{nst}(\textbf{x},\textbf{r},t,a)}
    {\tau_d(\textbf{x},\textbf{r},t,a)}
    = 
    -\int
    \sum_i \delta_i \delta_a \sum_{e \neq i} (\textbf{u}_e \cdot \hat{\textbf{r}}_e - \textbf{u}_i \cdot \hat{\textbf{r}}_i)
    \delta(|\textbf{x}_e - \textbf{x}| - |\textbf{x}_i - \textbf{x}|)^- h_i
    \PP
\end{equation}
\subsubsection{Age distribution}
The aim of this section is to find the form of $\tau_d$
The random destruction here isn't a hypothesis anymore since we regard the particle relative to eulerian points. 
But still here we assume homogeneity such as the macro scale gradients are negligible. 
\begin{align*}
    \pdda P_{nst}
    + \pddr\cdot  ( \nstavg{\textbf{u}_i} P_{nst}) 
    &= \delta(a)P_{nst}(\textbf{x},\textbf{r},t,a)
    - \frac{P_{nst}(\textbf{x},\textbf{r},t,a)}
    {\tau_d(\textbf{x},\textbf{r},t,a)}\\
\end{align*} 
Now we integrate this equation over \textbf{r} yielding the following equation, 
\begin{align*}
    \pdda P_{a}
    &= \delta(a)P_{a}(\textbf{x},t,a)
    - \frac{P_{a}(\textbf{x},t,a)}
    {\ravg{\tau_d}(\textbf{x},t,a)}\\
\end{align*} 
Assuming a constant rate of destruction we obtain, 
\begin{align*}
    P_a(\textbf{x},t, a)  
    =\frac{e^{-a/\avgcond{\tau_d}(\textbf{x},t)}}{\avgcond{\tau_d}(\textbf{x},t)}
\end{align*} 
\subsubsection{Position distribution}
The probability density of having a particle center of mass at \textbf{x} is the number density $n_p(\textbf{y})$. 
fFrom \citet{zhang2021ensemble}, \todo{not correct because we do not consider $\chi_k$}
\begin{equation}
    P_{nst}(\textbf{x},\textbf{r})
    = n_p e^{-4\pi n_p (r^3- r_b^3)/3}
\;\;\;
\text{for}
\;\;\;
|\textbf{r}| > a
\end{equation}
Assuming that the age distribution isn't correlated with \textbf{r} we have, 
\begin{equation}
    P_{nst}(\textbf{x},\textbf{r},t,a)
    = n_p(\textbf{x},t) 
    \frac{e^{-a/\avgcond{\tau_d}(\textbf{x},t) -4\pi n_p(\textbf{x},t) (|\textbf{r}|^3- r_b^3)/3}}{\avgcond{\tau_d}(\textbf{x},t)}
\end{equation}

From these arguments we only needs to solve $P_{nst}$ in the global phase space $\textbf{x},t$. 
Knowing that it only depends on $\tau$ and $n_p$. 
\subsection{Microscale description}

In this section we propose to describe the Microscale thought the use of the tensor $\mathcal{R}_p  = \pavg{ \textbf{rr}}$. 
This tensor represents the mean rapprochement between particles. 
\begin{figure}[h!]
    \centering
    \begin{tikzpicture}[scale =1.5]
      
    \foreach \i in {1,...,5} {
    \pgfmathsetmacro{\x}{rnd*0.7}
    \pgfmathsetmacro{\y}{rnd*0.3}
    \draw[fill=gray] ($(\x,\y)$) circle (0.1);
    }
    \foreach \i in {1,...,5} {
    \pgfmathsetmacro{\x}{rnd*0.7}
    \pgfmathsetmacro{\y}{rnd*0.3}
    \draw[fill=gray] ($(\x+2,\y)$) circle (0.1);
    }
    \foreach \i in {1,...,5} {
    \pgfmathsetmacro{\x}{rnd*0.6}
    \pgfmathsetmacro{\y}{rnd*0.4}
    \draw[fill=gray] ($(\x+1,\y-1)$) circle (0.1);
    }
    \foreach \i in {1,...,5} {
    \pgfmathsetmacro{\x}{rnd*0.6}
    \pgfmathsetmacro{\y}{rnd*0.4}
    \draw[fill=gray] ($(\x-1,\y-1)$) circle (0.1);
    }
    \draw (1,-1)node[below]{$\mathbb{R}/r_m^2 \ll 1$};
  
      \foreach \i in {1,...,20} {
        \pgfmathsetmacro{\x}{rnd*4}
        \pgfmathsetmacro{\y}{rnd*2}
        \draw[fill=gray] ($(\x+5,\y-1)$) circle (0.1);
        }
        \draw (6.5,-1)node[below]{$\mathbb{R}/r_m^2 < 1$};
      \end{tikzpicture}
      \hfill
  \end{figure}
Let $\mathcal{R}_p^* = r_m^2 \textbf{I}$ being the isotropic microscale representation at random. 
It can be computed from the random distribution funciton $P_{nst}(\textbf{r}) =     P_{nst}(\textbf{x},\textbf{r})
= n_p e^{-4\pi n_p [r^3- (d)^3]/3}$ \citep{zhang2021ensemble}. 
Therefore, we have, 
\begin{equation}
    r_m^2 \textbf{I}
    = \int_{|\textbf{r}| > d}
    n_p(\textbf{x},t) \textbf{rr} e^{-4\pi n_p(\textbf{x},t) [|\textbf{r}|^3- (d)^3]/3}
    d\textbf{r}
\end{equation}
Since this quantity is isotropic let's focus on the $zz$ components using the change of variable : 
\begin{equation*}
    \textbf{r} = 
    \left\{
        \begin{tabular}{cc}
            $x =$&$ r \sin\theta \cos\phi$\\
            $y =$&$ r \sin\theta \sin\phi$\\
            $z =$&$ r \cos\theta$
        \end{tabular}
    \right.
\end{equation*}
Thus, the previous integral reads, 
\begin{align*}
    r_m^2 
    &= n_p \int_{0}^{2\pi}\int_{0}^{\pi}\int_{d}^{\infty}
     r^4 \cos^2\theta\sin\theta e^{-4\pi n_p [r^3- (d)^3]/3}
    drd\theta d\phi\\
    &= n_p \int_{0}^{2\pi}d\phi 
    \int_{0}^{\pi} \cos^2\theta\sin\theta d\theta 
    \int_{d}^{\infty}
     r^4  e^{-4\pi n_p [r^3- (d)^3]/3} dr\\
    &= n_p \pi
    \frac{4}{3} 
    e^{-4\pi n_p d^3 /3}
    \int_{d}^{\infty}
     r^4  e^{-4\pi n_p r^3 /3}
      dr\\
\end{align*}
Since this last integral isn't computable we will use $u = r^3$ thus $du = 3 r^2 dr$ or $dr = \frac{u^{-2/3}}{3} du$
\begin{equation*}
    \int_{d}^{\infty}
     r^4  e^{-4\pi n_p r^3 /3}
      dr
      = \frac{1}{3} 
      \int_{d^{1/3}}^{\infty}
        u^{2/3}  e^{-4\pi n_p u /3}
      du
      = \frac{1}{3}\left.
        -\frac{\Gamma(\frac{5}{2}, 4\pi n_p/3 u )}{(4\pi n_p/3)^{5/3}}
        \right|_{d^{1/3}}^\infty
      = 
        \frac{\Gamma(\frac{5}{2}, 4\pi n_p/3 d^{1/3} )}{3(4\pi n_p/3)^{5/3}}
\end{equation*}
The final results reads as,
\begin{equation*}
    r_m^2 =
    \frac{4n_p \pi e^{-4\pi n_p d^3 /3}}{9(4\pi n_p / 3)^{5/3}}
    \Gamma\left(\frac{5}{2}, 4\pi n_p d^{1/3} / 3 \right)
\end{equation*}
However this results is true if and only if we are in a dilute regime since it is under this assumption that $P_{nst}$ is defined. 
In practice for $\lambda = 1$ we recover this behavior. 

\section{Expansion of the nearest PDF}

Let consider the $P_\text{nst}$ evaluated in,
\begin{equation}
  P_{nst}(\textbf{x}-\textbf{r}/2,\textbf{y},t,a)= 
  \int \sum_{i}\delta(\textbf{x}-\textbf{r}/2-\textbf{x}^i(\CC,t))
  \sum_{j\neq i}\delta(\textbf{x}+\textbf{r}-\textbf{x}^j(\CC,t)) 
  \delta(t+a-t_c^{ij}(\CC,t)) 
  h_{ij}(\CC,t) d\mathscr{P} 
\end{equation}
using the distribution formalism it is easy to show that, 
\begin{equation}
  \delta(\textbf{x}-\textbf{x}^i(\CC,t))
  = 
  \delta(\textbf{x}-\textbf{r}/2-\textbf{x}^i(\CC,t))
  +\textbf{r}/2\cdot \pddx
  \delta(\textbf{x}-\textbf{x}^i(\CC,t))
  + \ldots
\end{equation}
Thus, 
\begin{equation}
  \nstavg{q}P_{nst}(\textbf{x},\textbf{y},t,a)
  = \nstavg{q}P_{nst}(\textbf{x} - \textbf{r}/2,\textbf{y},t,a)
  + \frac{1}{2}
  \div\left[
    \textbf{r} \nstavg{q}P_{nst}(\textbf{x},\textbf{y},t,a)
  \right]
\end{equation}
Averaging on all position \textbf{r} and age $a$ then yields, 
\begin{equation}
  n_p q_p 
  = n_p q_\text{pm}
  + \frac{1}{2}
  \div\left[
    n_p \mathcal{Q}_\text{pfp}
  \right]
\end{equation}

\begin{align}
  \pddt (\phi_1 \rho_1)  
  + \div (
      \phi_1 \rho_1\textbf{u}_1
  )
  &= 
  0\\
  \pddt (\phi_1 \rho_1\textbf{u}_1)  
  + \div (
      \phi_1 \rho_1\textbf{u}_1\textbf{u}_1
      - \bm{\sigma}_1^\text{eq}
      % -n_p \mathcal{F}_p
      % +\Sigma_\text{pfp}
  )
  &= 
  \phi_1 \rho_1 \textbf{g} 
  -  n_p \textbf{f}_\text{pm}\\
  \pddt (\phi_1\rho_1E_1)  
  + \div (
      \phi_1\rho_1E_1\textbf{u}_1
      + \bm{q}_1^\text{eq}
      - \textbf{u}_1 \cdot \bm{\sigma}_1^\text{eq}
      % - \textbf{u}_1^0 \cdot \bm{\sigma}_1^0 
      % + \textbf{q}_1^0
      % - n_p \mathcal{C}_p
      )
  &= 
  \phi_1 \rho_1\textbf{u}_1 \cdot \textbf{g} 
  - n_p \textbf{c}_p
\end{align} 
\todo{Is this formulation usefull for the NRJ equaiton ? }
where we have defined, 
\begin{align*}
  &\bm{\sigma}_1^\text{eq}
  = \phi_1\rho_1 (
      \bm{\sigma}_1%- n_p \textbf{M}_p
      - \kavg{\textbf{u}_1'\textbf{u}_1'})
      +n_p \mathcal{F}_p
      - \mathcal{F}_\text{pfp}  
  &\textbf{q}_1^\text{eq}
  =\textbf{q}_1^\text{e} 
  +\textbf{q}_1^\text{k}  
  - n_p \mathcal{C}_p\\
  &\textbf{q}_1^\text{e}
  = \phi_1\rho_1 \kavg{\textbf{u}_1' e_1'} 
  + \phi_1\textbf{q}_1 
  &\textbf{q}_1^\text{k}
  = \phi_1\rho_1 \kavg{\textbf{u}_1' k_1} 
  - \phi_1\kavg{\textbf{u}_1' \cdot \bm{\sigma}_1^0} \\
  &n_p \textbf{f}_p= 
  \int_{\Sigma_i}
  \left(
      \mathbf{\sigma}_1^0 
      - \mathbf{\sigma}_1
  \right)  
  \cdot \textbf{n}_2d\Sigma
  &n_p \mathcal{F}_p
  = 
  \int_{\Sigma_i}
  \textbf{r}
  \left(
      \mathbf{\sigma}_1^0 
      - \mathbf{\sigma}_1
  \right)  
  \cdot \textbf{n}_2d\Sigma\\
  &n_p \textbf{c}_p= 
  \int_{\Sigma_i}
  \left(
      \textbf{u}_1^0 \cdot \mathbf{\sigma}_1^0 
      - \textbf{q}_1^0 
      % - \mathbf{\sigma}_1
  \right)  
  \cdot \textbf{n}_2d\Sigma
  &n_p \mathcal{C}_p
  = 
  \int_{\Sigma_i}
  \textbf{r}
  \left(
      \textbf{u}_1' \cdot \mathbf{\sigma}_1^0 
      - \textbf{q}_1^0 
  \right)  
  \cdot \textbf{n}_2d\Sigma\\
\end{align*}
\begin{equation*}
  \mathcal{F}_\text{pfp}
  = \int 
  \frac{\textbf{r}}{2}
  \textbf{f}^\text{nst} 
  P_\text{nst}(\textbf{x},\textbf{y},t,a)
  d\textbf{r}
\end{equation*}
As a consequence we can write, 

\begin{equation*}  
  n_p \textbf{c}_p= \pavg{
  \int_{\Sigma_i}
  \left(
    \textbf{u}_1^0 \cdot \mathbf{\sigma}_1^0 
    - \textbf{q}_1^0 
    % - \mathbf{\sigma}_1
    \right)  
    \cdot \textbf{n}_2d\Sigma}
    = n_p \textbf{c}_\text{pm}
    + \div \mathcal{C}_\text{pfp}
\end{equation*}
with, 
\begin{equation*}
  \mathcal{C}_\text{pfp}
  = \int 
  \frac{\textbf{r}}{2}
  \textbf{c}^\text{nst} 
  P_\text{nst}(\textbf{x},\textbf{y},t,a)
  d\textbf{r}
\end{equation*}
Besides, 
\begin{equation*}
  \textbf{u}_1 \cdot \mathcal{F}_\text{pfp}
  = \int 
  \frac{\textbf{r}}{2}
  \textbf{u}_1 \cdot \textbf{f}^\text{nst} 
  P_\text{nst}(\textbf{x},\textbf{y},t,a)
  d\textbf{r}
\end{equation*}
Let's focus on the term $\textbf{c}^\text{nst}$, 
\begin{align*}  
  n_p \textbf{c}^\text{nst}
  &= \int \delta_i \delta_j \delta_a h_{ij}
  \int_{\Sigma_i}
  \left(
    \textbf{u}_1^0 \cdot \mathbf{\sigma}_1^0 
    - \textbf{q}_1^0 
    % - \mathbf{\sigma}_1
    \right)  
    \cdot \textbf{n}_2d\Sigma d\CC\\
    &= \int \delta_i \delta_j \delta_a h_{ij}
    \int_{\Sigma_i}
    \left(
      \textbf{u}_1 \cdot \mathbf{\sigma}_1^0 
      - \textbf{q}_1^0 
      % - \mathbf{\sigma}_1
      \right)  
      \cdot \textbf{n}_2d\Sigma d\CC
    + \int \delta_i \delta_j \delta_a h_{ij}
    \int_{\Sigma_i}
    \left(
      \textbf{u}_1' \cdot \mathbf{\sigma}_1^0 
      - \textbf{q}_1^0 
      % - \mathbf{\sigma}_1
      \right)  
      \cdot \textbf{n}_2d\Sigma d\CC
\end{align*}
\begin{equation*}  
  n_p \textbf{u}_1  \cdot \textbf{f}^\text{nst}= \int \delta_i \delta_j \delta_a h_{ij}
  \int_{\Sigma_i}
  \left(
    \textbf{u}_1 \cdot \mathbf{\sigma}_1^0 
    \right)  
    \cdot \textbf{n}_2d\Sigma d\CC
\end{equation*}
Consequently, if $\textbf{c}^i = \int_{\Sigma_i} \textbf{u}'_1\cdot \bm{\sigma}_1^0 d\Sigma$ we can write, 
\begin{align*}  
  \int \delta_i \delta_j \delta_a h_{ij}
  \int_{\Sigma_i}
  \left(
    \textbf{u}_1^0 \cdot \mathbf{\sigma}_1^0 
    - \textbf{q}_1^0 
    % - \mathbf{\sigma}_1
    \right)  
    \cdot \textbf{n}_2d\Sigma d\CC
  = \textbf{u}_1 \cdot \textbf{f}^\text{nst}
  + n_p \textbf{c}^\text{nst}
\end{align*}
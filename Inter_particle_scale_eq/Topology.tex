\subsection{Topological equations}

Let $\textbf{x}^i(\CC,t)$ (resp. $\textbf{x}^j(\CC,t)$) be the position at time $t$ of the particle $i$ (resp. $j$) for a realization of the flow $\CC$. 
In the following we drop the argument $\CC$ and $t$. 
Then, $\delta^i(\textbf{x}) = \delta(\textbf{x}- \textbf{x}^i(\CC,t))$ is the indicator function of having a particle $i$ at  $\textbf{x}$.
Similarly, we introduce $\delta^j(\textbf{x},\textbf{r}) = \delta(\textbf{x} + \textbf{r} - \textbf{x}^j)$ which is the indicator function of having the particle indexed $j$ at $\textbf{x}+\textbf{r}$. 
The aim is to derive a transport equation for the indicator function of the following state : a particle $i$ is at \textbf{x} with its nearest neighbor located at $\textbf{x}+\textbf{r}$. 
Lastly, $\delta^a(a - a^{ij}(\CC,t))$ will be use as the indicator function for the age of the interaction. 
Such a function can be written $\Pi(\textbf{x},\textbf{r}) = \delta^a(a)\delta^i(\textbf{x}) \delta^j(\textbf{x},\textbf{r}) h^{ij}$. 
Using the distribution formalism and taking the partial time derivative on these indicators functions, leads us to, 
\begin{equation*}
    \pddt \delta^i(\textbf{x})
    + \textbf{u}^i  \cdot \partial_{\textbf{x}} \delta(\textbf{x} - \textbf{x}^i)
    = 0 
    % = \pddt \delta(\textbf{x} - \textbf{x}^i)
    % = \textbf{u}^i \partial_{\textbf{x}^i} \delta(\textbf{x} - \textbf{x}^i)
    % = - \textbf{u}^i \partial_{\textbf{x}} \cdot \delta(\textbf{x} - \textbf{x}^i)
\end{equation*}
\begin{equation*}
    \pddt \delta^j(\textbf{x})
    + \textbf{u}^i \cdot \partial_{\textbf{x}}  \delta(\textbf{x} + \textbf{r} - \textbf{x}^j)
    + (\textbf{u}^j - \textbf{u}^i) \cdot \partial_{\textbf{r}}  \delta(\textbf{x} + \textbf{r} - \textbf{x}^j)
    = 0 
\end{equation*}
\begin{equation*}
    \pddt \delta(a - a^{ij})
    + 
    \partial_a \delta(a - a^{ij})
    = 0 
\end{equation*}
where we added and subtracted $\textbf{u}^i \cdot \partial_{\textbf{x}}  \delta(\textbf{x} + \textbf{r} - \textbf{x}^j)$ in the second equation to make appear the $\partial_{\textbf{x}}$ derivative. 
Noticing that the Lagrangian quantities $\textbf{u}^i$ and $\textbf{u}^j$ are solely function of $(\CC,t)$ we can rewrite the equations like so, 
\begin{equation}
    \pddt \delta^i
    + \partial_{\textbf{x}} \cdot (\textbf{u}^i \delta^i)
    = 0 
    \label{eq:dt_delta_i}
\end{equation}
\begin{equation}
    \pddt \delta^j
    + \partial_{\textbf{x}} \cdot (\textbf{u}^i \delta^j)
    + \partial_{\textbf{r}}  \cdot (\textbf{w}^{ij}\delta^j)
    = 0 
\end{equation}
\begin{equation*}
    \pddt \delta^a
    + 
    \partial_a \delta^a
    = 0 
\end{equation*}
where $\textbf{w}^{ij} = (\textbf{u}^j - \textbf{u}^i)$ is teh relative velocity between nearest neighboring particles. 
Multiplying, this second equation by $\delta^a\delta^ih^{ij}$ and using \ref{eq:dt_delta_i} yields a transport equaiton for $\Pi(\textbf{x},\textbf{r},a,t)$, namely,
\begin{equation}
    \pddt \Pi(\textbf{x},\textbf{r},a,t)
    + \partial_a \Pi(\textbf{x},\textbf{r},a,t)
    + \partial_{\textbf{x}} \cdot (\textbf{u}^i \Pi(\textbf{x},\textbf{r},a,t))
    + \partial_{\textbf{r}}  \cdot (\textbf{w}^{ij} \Pi(\textbf{x},\textbf{r},a,t))
    = \delta^i\delta^j\delta^a \pddt h^{ij}
\end{equation}
The source term $\delta^a\delta^i\delta^j \pddt h^{ij}$ account for the creation and destruction of nearest particle pairs. 
The starting point to derive the evolution equation of the indicator function $\Pi$ is the Louisville equation

The starting point to derive the evolution equation of the indicator function $\Pi$ is the Louisville equation \citep{cercignani1988boltzmann}


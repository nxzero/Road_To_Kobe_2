\subsection{Topological equations}

Let $\textbf{x}^i(\CC,t)$ (resp. $\textbf{x}^j(\CC,t)$) be the position at time $t$ of the particle $i$ (resp. $j$) for a realization of the flow $\CC$. 
In the following we drop the argument $\CC$ and $t$. 
Then, $\delta^i(\textbf{x}) = \delta(\textbf{x}- \textbf{x}^i(\CC,t))$ is the indicator function of having a particle $i$ at  $\textbf{x}$.
Similarly, we introduce $\delta^j(\textbf{x},\textbf{r}) = \delta(\textbf{x} + \textbf{r} - \textbf{x}^j)$ which is the indicator function of having the particle indexed $j$ at $\textbf{x}+\textbf{r}$. 
The aim is to derive a transport equation for the indicator function of the following state : a particle $i$ is at \textbf{x} with its nearest neighbor located at $\textbf{x}+\textbf{r}$. 
Lastly, $\delta^a(t - a - t_c^{ij}(\CC,t))$ will be use as the indicator function indicating the age of the interaction for a particle pair created at time $t^{ij}_c$. 
Such a function can be written $\Pi(\textbf{x},\textbf{r},t,a) = \delta^a(a)\delta^i(\textbf{x}) \delta^j(\textbf{x},\textbf{r}) h^{ij}$. 
Using the distribution formalism and taking the partial time derivative on these indicators functions, leads us to these relations, 
\begin{equation*}
    \pddt \delta^i(\textbf{x})
    + \textbf{u}^i  \cdot \partial_{\textbf{x}} \delta(\textbf{x} - \textbf{x}^i)
    = 0 
    % = \pddt \delta(\textbf{x} - \textbf{x}^i)
    % = \textbf{u}^i \partial_{\textbf{x}^i} \delta(\textbf{x} - \textbf{x}^i)
    % = - \textbf{u}^i \partial_{\textbf{x}} \cdot \delta(\textbf{x} - \textbf{x}^i)
\end{equation*}
\begin{equation*}
    \pddt \delta^j(\textbf{x})
    + \textbf{u}^i \cdot \partial_{\textbf{x}}  \delta(\textbf{x} + \textbf{r} - \textbf{x}^j)
    + (\textbf{u}^j - \textbf{u}^i) \cdot \partial_{\textbf{r}}  \delta(\textbf{x} + \textbf{r} - \textbf{x}^j)
    = 0 
\end{equation*}
\begin{equation*}
    \pddt \delta^a(t - a - t_c^{ij})
    + 
    \partial_a \delta^a(t - a - t_c^{ij})
    = 0 
\end{equation*}
where we added and subtracted $\textbf{u}^i \cdot \partial_{\textbf{x}}  \delta(\textbf{x} + \textbf{r} - \textbf{x}^j)$ in the second equation to make appear the $\partial_{\textbf{x}}$ derivative. 
Noticing that the Lagrangian quantities $\textbf{u}^i$ and $\textbf{u}^j$ are solely function of $(\CC,t)$ we can rewrite the equations like so, 
\begin{equation}
    \pddt \delta^i
    + \partial_{\textbf{x}} \cdot (\textbf{u}^i \delta^i)
    = 0 
    \label{eq:dt_delta_i}
\end{equation}
\begin{equation}
    \pddt \delta^j
    + \partial_{\textbf{x}} \cdot (\textbf{u}^i \delta^j)
    + \partial_{\textbf{r}}  \cdot (\textbf{w}^{ij}\delta^j)
    = 0 
\end{equation}
\begin{equation*}
    \pddt \delta^a
    + 
    \partial_a \delta^a
    = 0 
\end{equation*}
where $\textbf{w}^{ij} = (\textbf{u}^j - \textbf{u}^i)$ is teh relative velocity between nearest neighboring particles. 
Multiplying, this second equation by $\delta^a\delta^ih^{ij}$ and using \ref{eq:dt_delta_i} yields a transport equaiton for $\Pi(\textbf{x},\textbf{r},a,t)$, namely,
\begin{equation}
    \pddt \Pi
    + \partial_a \Pi
    + \partial_{\textbf{x}} \cdot (\textbf{u}^i \Pi)
    + \partial_{\textbf{r}}  \cdot (\textbf{w}^{ij} \Pi)
    = G_{ijl}+D_{ijl}
    \label{eq:dt_Pi}
\end{equation}
The source term $\delta^a\delta^i\delta^j \pddt h^{ij} = G_{ijl}+D_{ijl}$ account for the creation and destruction of nearest particle pairs. 
The starting point to derive the evolution equation of the indicator function $\Pi$ is the Louisville equation
\tb{we could do the same using Louivill eq and $\pddt \delta = 0$ but in this case the age cannot be defined }

The source term on the RHS of \ref{eq:dt_Pi} can be computed by carrying out the algebra and yields, 
\begin{equation*}
    \pddt h_{ij} 
    = 
    \sum_{l \neq i,j} 
    (\hat{\textbf{r}}_{li}\cdot \textbf{w}_{li}
    - 
    \hat{\textbf{r}}_{ji}\cdot\textbf{w}_{ji})
    \delta(r_{li} - r_{ji})
        h_{ij}
\end{equation*}
Where $r$ refer to $|\textbf{r}|$ and $\hat{\textbf{r}}$ being the normal unit vector corresponding to $\textbf{r}$. 
We refer the reader too \citet[Appendix A.]{zhang2023evolution} for more detail about the computation of this term.
When a nearest pair of particles is created $(\hat{\textbf{r}}_{li}\cdot \textbf{w}_{li} - \hat{\textbf{r}}_{ji}\cdot\textbf{w}_{ji}) > 0$, when a nearest pair is destroyed $(\hat{\textbf{r}}_{li}\cdot \textbf{w}_{li} - \hat{\textbf{r}}_{ji}\cdot\textbf{w}_{ji}) < 0$ since the particle $j$ must go faster. 
Thus,
\begin{equation*}
    \pddt h_{ij} 
    = 
    \sum_{l \neq i,j} 
    (\hat{\textbf{r}}_{li}\cdot \textbf{w}_{li}
    - 
    \hat{\textbf{r}}_{ji}\cdot\textbf{w}_{ji})^+
    \delta(r_{li} - r_{ji})
        h_{ij}
    + \sum_{l \neq i,j} 
    (\hat{\textbf{r}}_{li}\cdot \textbf{w}_{li}
    - 
    \hat{\textbf{r}}_{ji}\cdot\textbf{w}_{ji})^-
    \delta(r_{li} - r_{ji})
        h_{ij}
\end{equation*}

\subsection{Evolution of the nearest PDF and particle properties.}

Now that the transport equation of $\Pi(\textbf{x},\textbf{r},a,t,\CC)$ is well set we can derive an equation for evolution of the nearest pair using the nearest average procedure defined in \ref{eq:P_nst}, such that : 
\begin{equation}
    \pddt P_{nst} 
    + \partial_a P_{nst} 
    + \partial_\textbf{x} \cdot \left(\nstavg{\textbf{u}} P_{nst}\right) 
    + \partial_\textbf{r} \cdot \left( \nstavg{\textbf{w}} P_{nst} \right) 
    =   P_{nst}(\textbf{x},\textbf{r},0,t)\delta(a)
    - \frac{P_{nst}(\textbf{x},\textbf{r},a,t)}{\tau_d(\textbf{x},\textbf{r},a,t)}
    \label{eq:dt_P_nst}
\end{equation}
where $P_{nst}(\textbf{x},\textbf{r},0,t)\delta(a)$ is the number of the nearest created pair and,
 $\tau(\textbf{x},\textbf{r},a,t)$ is the rate of destruction of the nearest particles, knowing the the nearest particle is at $\textbf{r}$ with age $a$.
Since for any new nearest pair of particles a pair is destroyed we have the following equallity :
 \begin{equation}
    \int P_{nst}(\textbf{x},\textbf{r},0,t)\delta(a) dS(r)
    =\iint \frac{P_{nst}(\textbf{x},\textbf{r},a,t)}{\tau_d(\textbf{x},\textbf{r},a,t)}
    dadS(r)
 \end{equation}
where we integrate on any sphere surface of radius $r$ around the particles. 

\subsubsection{Random destruction assumption}

Following \citet{zhang2023evolution} we assume that the mesoscal gradient are lower higher than the particle-fluid-particle scales gradients. 
In this case \ref{eq:dt_P_nst} reads, 
\begin{equation}
    \partial_a P_{nst} 
    + \partial_\textbf{r} \cdot \left( \nstavg{\textbf{w}} P_{nst} \right) 
    = P_{nst}(\textbf{x},\textbf{r},0,t)\delta(a)
    - \frac{P_{nst}(\textbf{x},\textbf{r},a,t)}{\tau_d(\textbf{x},\textbf{r},a,t)}
\end{equation}
integrating over all $\textbf{r}$ this equation gives, 
\begin{equation}
    \partial_a P_{a} (\textbf{x},a,t)
    = P_{a}(\textbf{x},0,t)\delta(a)
    - \frac{P_{a}(\textbf{x},a,t)}{\condavg{\tau_d}{\textbf{r}}(\textbf{x},a,t)}
    \label{eq:dt_P_a}
\end{equation}
where, 
\begin{equation}
    P_a(\textbf{x},a,t)n_p(\textbf{x},t)
    = \int P_{nst}(\textbf{x},\textbf{r},a,t) d\textbf{r}
\end{equation}
and,
\begin{equation}
    \frac{P_{a}(\textbf{x},a,t) n_p(\textbf{x},t)}{\condavg{\tau_d}{\textbf{r}}(\textbf{x},a,t)}
    = \int 
    \frac{P_{nst}(\textbf{x},\textbf{r},a,t)}{\tau_d(\textbf{x},\textbf{r},a,t)}
    d\textbf{r}.
\end{equation}
If we assume that the rate of destruction of the nearest pair isn't function of the relative position nor the age of interaction, it is possible to write $1/\avgcond{\tau_d}(\textbf{x},t) = 1/ \condavg{\tau_d}{r}(\textbf{x},a,t)$.
Then  \ref{eq:dt_P_a} can be written :
\begin{equation}
    \partial_a P_{a} (\textbf{x},a,t)
    = \delta(a) P_{a}(\textbf{x},0,t)
    - \frac{P_{a}(\textbf{x},a,t)}{\condavg{\tau_d}{}(\textbf{x},t)}
\end{equation}
Using the distribution formalism and the normalization condition $\int P_a(\textbf{x},t,a) da = 1$ we obtain : 
\begin{equation}
    P_a(\textbf{x},t,a)
    =\frac{e^{- a/\condavg{\tau_d}{}}(\textbf{x},t)}{\condavg{\tau_d}{}(\textbf{x},t)}
\end{equation}




\subsubsection{Flow induced anisotropy}

As we have shown in our resent paper the particle-fluid-particle stress in the vertical direction is manly caused by flow anisotropy. 
This flow anisotropy is measure by, 
\begin{equation}
    \condavg{\textbf{r}}{\textbf{r}}(\textbf{x},t,a)P_a(\textbf{x},a,t)n_p(\textbf{x},t)
    = \int \textbf{r} P_{nst}(\textbf{x},\textbf{r},a,t) d\textbf{r}
\end{equation}
\paragraph{Relative position equaiton :}
To derive an equation for $\textbf{r}$ we multiply \ref{eq:dt_Pi} and use this definition which gives in the first place :
\begin{multline}
    \pddt (\textbf{r} P_{nst})
    + \partial_a (\textbf{r} P_{nst})
    + \partial_{\textbf{x}} \cdot (\nstavg{\textbf{r}\textbf{u} } P_{nst})
    + \partial_{\textbf{r}}  \cdot (\nstavg{\textbf{r}\textbf{w}} P_{nst})
    =  \\
    \textbf{r} P_{nst}(\textbf{x},\textbf{r},0,t)\delta(a)
    - \textbf{r} \frac{P_{nst}(\textbf{x},\textbf{r},a,t)}{\tau_d(\textbf{x},t)}
    + \nstavg{\textbf{w}}P_{nst}
    \label{eq:dt_r_nst}
\end{multline}
Since we are actually interested in the average of $\textbf{r}$ over all \textbf{r}, namely, $\condavg{\textbf{r}}{\textbf{r}}$ we integrate this equation over all \textbf{r} and as before we assume no significant macroscopic changes. 
\begin{multline*}
    \partial_a (\condavg{\textbf{r}}{\textbf{r}}(\textbf{x},t,a)P_a(\textbf{x},a,t))
    =  
    \condavg{\textbf{r}}{\textbf{r}}(\textbf{x},t,a)P_a(\textbf{x},a,t)\delta(a)
     - \frac{\condavg{\textbf{r}}{\textbf{r}}(\textbf{x},t,a)P_a(\textbf{x},a,t)}{\tau_d(\textbf{x},t)} 
    + \condavg{\textbf{w}}{\textbf{r}}(\textbf{x},t,a)P_a(\textbf{x},a,t)
\end{multline*}
Noticing that the first term can be decomposed in $\partial_a (\condavg{\textbf{r}}{\textbf{r}}(\textbf{x},t,a)P_a(\textbf{x},a,t)) = \partial_a (\condavg{\textbf{r}}{\textbf{r}}(\textbf{x},t,a))P_a(\textbf{x},a,t) + \condavg{\textbf{r}}{\textbf{r}}(\textbf{x},t,a) \partial_a (P_a(\textbf{x},a,t))$ the previous expression simplify to : 
\begin{equation*}
    \partial_a (\condavg{\textbf{r}}{\textbf{r}}(\textbf{x},t,a))
    =  
    \condavg{\textbf{r}}{\textbf{r}}(\textbf{x},t,a)\delta(a)
    + \condavg{\textbf{w}}{\textbf{r}}(\textbf{x},t,a)
    \label{eq:da_r}
\end{equation*}
This equation holds for components of $\textbf{r}$ not its norm. 

\paragraph{Relative distance correlation :}
Now let's derive the first moment of this equation by multiplying \ref{eq:dt_P_nst} by $\textbf{rr}$. 
It reads, 
\begin{multline}
    \pddt (\nstavg{\textbf{rr}} P_{nst})
    + \partial_a (\nstavg{\textbf{rr}} P_{nst})
    + \partial_{\textbf{x}} \cdot (\nstavg{\textbf{rr}\textbf{u} } P_{nst})
    + \partial_{\textbf{r}}  \cdot (\nstavg{\textbf{rr}\textbf{w}} P_{nst})
    =  \\
    \nstavg{\textbf{rr}} P_{nst}(\textbf{x},\textbf{r},0,t)\delta(a)
    -  \frac{\nstavg{\textbf{rr}} P_{nst}(\textbf{x},\textbf{r},a,t)}{\tau_d(\textbf{x},t)}
    + \nstavg{\textbf{rw}}P_{nst}
    + \nstavg{\textbf{wr}}P_{nst}
\end{multline}
averaging over all \textbf{r} and considering homogeneous medium gives,
\begin{equation*}
    \pddt (\condavg{\textbf{rr}}{\textbf{r}} P_{a})
    + \partial_a (\condavg{\textbf{rr}}{\textbf{r}} P_{a})
    + \partial_{\textbf{x}} \cdot (\condavg{\textbf{rr}\textbf{u} }{\textbf{r}} P_{a})
    =  
    \condavg{\textbf{rr}}{\textbf{r}} P_{a}\delta(a)
    -  \frac{\condavg{\textbf{rr}}{\textbf{r}} P_{a}}{\tau_d(\textbf{x},t)}
    + \condavg{\textbf{rw}}{\textbf{r}} P_{a}
    + \condavg{\textbf{wr}}{\textbf{r}} P_{a}
\end{equation*}
or 
\begin{equation*}
    % \pddt (\condavg{\textbf{rr}}{\textbf{r}} P_{a})
    \partial_a (\condavg{\textbf{rr}}{\textbf{r}} P_{a})
    % + \partial_{\textbf{x}} \cdot (\condavg{\textbf{rr}\textbf{u} }{\textbf{r}} P_{a})
    =  
    -  \frac{\condavg{\textbf{rr}}{\textbf{r}} P_{a}}{\tau_d(\textbf{x},t)}
    + \condavg{\textbf{rr}}{\textbf{r}} P_{a}\delta(a)
    + \condavg{\textbf{rw}}{\textbf{r}} P_{a}
    + \condavg{\textbf{wr}}{\textbf{r}} P_{a}
\end{equation*}
If we consider that $\pdda P(a) = - \frac{e^{-a/\tau_a}}{\tau_a^2} = - \frac{- P(a)}{\tau_a}$ then, 
\begin{equation*}
    % \pddt (\condavg{\textbf{rr}}{\textbf{r}} P_{a})
     \partial_a (\condavg{\textbf{rr}}{\textbf{r}})
    % + \partial_{\textbf{x}} \cdot (\condavg{\textbf{rr}\textbf{u} }{\textbf{r}} )
    =  
    % -  \frac{\condavg{\textbf{rr}}{\textbf{r}} }{\tau_d(\textbf{x},t)}
    % + \condavg{\textbf{rr}}{\textbf{r}} \frac{1}{\tau_a}
    + \condavg{\textbf{rr}}{\textbf{r}} \delta(a)
    + \condavg{\textbf{rw}}{\textbf{r}} 
    + \condavg{\textbf{wr}}{\textbf{r}} 
\end{equation*}
Meaning that $\condavg{\textbf{rr}}{\textbf{r}}$ equal its initial value plus the velocity function. 
Integrating over all $a$ gives,
\begin{equation*}
    % \pddt (\condavg{\textbf{rr}}{\textbf{r}} P_{a})
     \condavg{\textbf{rr}}{\textbf{r}}
    % + \partial_{\textbf{x}} \cdot (\condavg{\textbf{rr}\textbf{u} }{\textbf{r}} )
    =  
    % -  \frac{\condavg{\textbf{rr}}{\textbf{r}} }{\tau_d(\textbf{x},t)}
    % + \condavg{\textbf{rr}}{\textbf{r}} \frac{1}{\tau_a}
    + \condavg{\textbf{rr}}{\textbf{r}}(a=0)
    + \int_0^\infty \condavg{\textbf{rw}}{\textbf{r}} da
    + \int_0^\infty \condavg{\textbf{wr}}{\textbf{r}} da
\end{equation*}
which give the evolution of the microstructure along time, but again the initial condition is required. 

Averaging over all $a$ then gives;
\begin{equation*}
    \pddt (\condavg{\textbf{rr}}{1} n_p)
    % + \partial_a (\condavg{\textbf{rr}}{1} n_p)
    + \partial_{\textbf{x}} \cdot (\condavg{\textbf{rr}\textbf{u} }{1} n_p)
    =  
    \condavg{\textbf{rr}}{1} n_pP_a(a=0)
    -  \frac{\condavg{\textbf{rr}}{1} n_p}{\tau_d(\textbf{x},t)}
    + \condavg{\textbf{rw}}{1} n_p
    + \condavg{\textbf{wr}}{1} n_p
\end{equation*}
In a global scale perspective, 
\begin{equation*}
    \pddt (\textbf{R} n_p)
    % + \partial_a (\condavg{\textbf{rr}}{1} n_p)
    + \partial_{\textbf{x}} \cdot (\condavg{\textbf{rr}\textbf{u} }{1} n_p)
    +  \frac{\textbf{R} n_p}{\tau_d(\textbf{x},t)}
    =  
    \textbf{R}P_an_p(\textbf{x},t,0) 
    + \condavg{\textbf{rw}}{1} n_p
    + \condavg{\textbf{wr}}{1} n_p
\end{equation*}
\begin{equation*}
    \pddt (\textbf{A} n_p)
    % + \partial_a (\condavg{\textbf{rr}}{1} n_p)
    + \partial_{\textbf{x}} \cdot (\condavg{(\textbf{rr}-\textbf{r}\cdot\textbf{r})\textbf{u} }{1} n_p)
    +  \frac{\textbf{A} n_p}{\tau_d(\textbf{x},t)}
    =  
    \textbf{A}P_an_p(\textbf{x},t,0) 
    + \condavg{\textbf{rw}}{1} n_p
    + \condavg{\textbf{wr}}{1} n_p
    - 2/3 \condavg{\textbf{w}\cdot \textbf{r}}{1} n_p
\end{equation*}
The anisotropy is therefore due to cdt initi plus velocity position corr

In an homogeneous medium it is clear that 
\begin{equation*} 
    \condavg{\textbf{rr}}{1} /\tau_d
    =  
    \condavg{\textbf{rr}}{1} P_a(a = 0)
    + \condavg{\textbf{rw}}{1} 
    + \condavg{\textbf{wr}}{1} 
\end{equation*}
Therefore the microstructure tensor, which is the mean square distance to the nearest neighbor is entirely determined by the average relative velocity distance correlation time the mean time. 
The first term can be viewed as initial condition. 
If we now focus on the trace of this tensor we obtain the relation : 
\begin{equation*} 
    \condavg{\textbf{r}\cdot \textbf{r}}{1} /\tau_d
    =  
    \condavg{\textbf{r}\cdot \textbf{r}}{1} P_a( a = 0)
    + 2\condavg{\textbf{r}\cdot \textbf{w}}{1}
\end{equation*}
Let's us define, 
\begin{equation*}
    \textbf{R}(\textbf{x},t) =\frac{1}{n_p(\textbf{x},t)} \int \textbf{rr} P_\text{nst}(\textbf{r}) d\textbf{r}
\end{equation*}
Since our objective is also to point out the anisotropy of the microstructure we are interested in the deviatoric part of this tensor, namely, 
\begin{equation*}
    \textbf{A}(\textbf{x},t) = \textbf{R}(\textbf{x},t) - \frac{1}{3} [\textbf{R}(\textbf{x},t) : \textbf{I}] \textbf{I}
\end{equation*}
To derive an equation for the asymmetry tensor we must subtract 1/3 of the previous equation to the tensorial one. 
Which gives, 
\begin{equation*} 
    \textbf{A}(\textbf{x},t)/\tau_d
    =  
    \textbf{A} P_a(\textbf{x}; a = 0)
    + 2\condavg{\textbf{rw} - 1/3\textbf{r}\cdot \textbf{w}}{1} 
\end{equation*}
Along the $A_{xx}$ component we have, 
\begin{equation*} 
    A_{xx}(\textbf{x},t)/\tau_d
    =  
    A_{xx} P_a(\textbf{x}; a = 0)
    + 2\condavg{r_xw_x - 1/3\textbf{r}\cdot \textbf{w}}{1}
\end{equation*}

Let measure the anisotropy with, 
\begin{equation*}
    \textbf{R}(\textbf{x},t)
    = \int \frac{\textbf{rr}}{|\textbf{r}|^2} P_\text{nst}(\textbf{x},\textbf{r},t,a) d\textbf{r} da
\end{equation*}
Then the derivative of this function is, 
\begin{equation*}
    \pddt \textbf{R}
    + \pddx\cdot (\textbf{u}_p \textbf{R})
    + \frac{\textbf{R}}{\tau}
    = \textbf{R}_{a=0}
    + \textbf{W}
\end{equation*}
with, 
\begin{align*}
    \textbf{W}(\textbf{x},t) = 
    \int_{0}^\infty
    \int_{\mathbb{R}^3} \left[
         \textbf{w}^\text{nst}_p \cdot \pddr (\textbf{rr}/|\textbf{r}|^2)
    \right]P_\text{nst}
    d\textbf{r}
    da,
    = \int_{0}^\infty
    \int_{\mathbb{R}^3} \left[
        \textbf{r} \textbf{w}^\text{nst}_p
        + \textbf{w}^\text{nst}_p\textbf{r}
    \right]P_\text{nst}
    d\textbf{r}
    da,
    \\
    \textbf{R}_{a = 0}(\textbf{x},t)
    =
    \int_{0}^\infty
    \int_{\mathbb{R}^3}
    \textbf{rr}
    P(\textbf{x},\textbf{r},t,a=0)\delta(a)
    d\textbf{r}da, \\
    \textbf{R}^\text{Re}(\textbf{x},t)
    =
    \int_{0}^\infty
    \int_{\mathbb{R}^3}
    \textbf{rr}(\textbf{u}^\text{nst}_p - \textbf{u}_p)
    P(\textbf{x},\textbf{r},t,a)
    d\textbf{r}da, 
\end{align*} 
The derivative of the dyadic reads as, 
\begin{align*}
    \partial_i (x_jr^{-2})
    = 
    x_jr^{-2} \delta_{ki}
    + x_ir^{-2} \delta_{jk}
    + x_ix_j \partial_k (r^{-2})\\
    % = 
    % x_jr^{-2} \delta_{ki}
    % + x_ir^{-2} \delta_{jk}
    % + x_ix_j  \partial_k r \partial_r (r^{-2})
    = 
    x_jr^{-2} \delta_{ki}
    + x_ir^{-2} \delta_{jk}
    + -2 x_ix_jx_k (r^{-4})\\
    = \frac{\textbf{xI}}{r^2}
    + \frac{\textbf{Ix}}{r^2}
    - \frac{2 \textbf{xxx}}{r^4}
\end{align*}

Let's say we transport, $\textbf{xx}/r$ instead then the source term yields, 
\begin{equation}
    \partial_k (x_i x_j r^{-1})
    = 
    x_j \delta_{ik} r^{-1}
    + x_i \delta_{jk} r^{-1}
    - x_ix_jx_k/r^3
\end{equation}


The closure, 
\begin{align*}
    \int_0^\infty \int_{\mathbb{R}^3} 
    r_k w_{p,k}^\text{nst}P_\text{nst}
    d\textbf{r}da 
    =
    \int_0^\infty \int_{\mathbb{R}^3} 
    \frac{1}{3}\partial_i (r_i) r_k w_{p,k}^\text{nst}P_\text{nst}
    d\textbf{r}da \\
    = 
    % \int_0^\infty \int_{\mathbb{R}^3} 
    % \frac{1}{3}\partial_{r,i} (r_i r_k w_{p,k}^\text{nst}P_\text{nst})
    % d\textbf{r}da 
    - \int_0^\infty \int_{\mathbb{R}^3} 
    \frac{1}{3} r_i r_k \partial_{r,i} ( w_{p,k}^\text{nst}P_\text{nst})
    d\textbf{r}da 
    - \int_0^\infty \int_{\mathbb{R}^3} 
    \frac{1}{3} r_k w_{p,k}^\text{nst}P_\text{nst}
    d\textbf{r}da 
\end{align*}
using the transport eq for $P_\text{nst}$ we have something useless, 

Let assume the mean square distance is purely determined by the initial condition, 
\begin{align*}
    \int_0^\infty 
    (r_k w_{p,k})^\text{nst-a}P_\text{nst-a}
    da 
    = 0 
    \int_0^\infty 
    (r_k w_{p,k})^\text{nst-a}\frac{e^{-a/\tau_a}}{\tau_a}
    da 
    = 0 
\end{align*}


\subsubsection{Dynamical description for particle without mass transfer}
First we define the Lagrangian balance equation for a particle $i$ such that 
\begin{equation}
    \ddt \textbf{u}_i
     = \frac{
        \textbf{b}_i
        + \textbf{f}_i
     }{m_i},
     \label{eq:dt_u_i}
 \end{equation}
From this equation we can also define an equation for the evolution of the relative velocity $\textbf{w}$ :
\begin{equation}
    \ddt \textbf{w}
    = \frac{
        \textbf{b}_i
        + \textbf{f}_i
    }{m_i}
    -\frac{
        \textbf{b}_j
        + \textbf{f}_j
    }{m_j}
    =\frac{
        \textbf{f}_i
    }{m_i}
    -\frac{
        \textbf{f}_j
    }{m_j}
    = 
    \textbf{a}_i
    - \textbf{a}_j,
    = \textbf{z}
    \label{eq:dt_w}
\end{equation}
\paragraph[short]{Relative velocity equation :}
Proceeding as before we derive the evolution equation of the averaged relative velocity fields, 
namely, 
\begin{multline}
    \pddt (\nstavg{\textbf{w}} P_{nst})
    + \partial_a (\nstavg{\textbf{w}} P_{nst})
    + \partial_{\textbf{x}} \cdot (\nstavg{\textbf{w}\textbf{u} } P_{nst})
    + \partial_{\textbf{r}}  \cdot (\nstavg{\textbf{w}\textbf{w}} P_{nst})
    =  \\
    \nstavg{\textbf{w}} P_{nst}(\textbf{x},\textbf{r},a,t)\delta(a)
    - \nstavg{\textbf{w}}  \frac{P_{nst}(\textbf{x},\textbf{r},a,t)}{\tau_d(\textbf{x},t)}
    + \nstavg{\textbf{z}}P_{nst}(\textbf{x},\textbf{r},a,t)
\end{multline}
Integrating over $dad\textbf{r}$ gives
\begin{equation}
    \pddt (\pnnavg{\textbf{w}} n_p)
    + \partial_{\textbf{x}} \cdot (\pnnavg{\textbf{w}\textbf{u} } n_p)
    =  \\
    \pnnavg{\textbf{w}} n_p(\textbf{x},t|a=0)
    - \pnnavg{\textbf{w}}  \frac{n_p(\textbf{x},t)}{\tau_d(\textbf{x},t)}
    + \pnnavg{\textbf{z}}n_p(\textbf{x},t)
\end{equation}
Applying the hypothesis of slow varying macroscopic variables : 
\begin{equation}
    \partial_a (\nstavg{\textbf{w}} P_{nst})
    +\partial_{\textbf{r}}  \cdot (\nstavg{\textbf{w}\textbf{w}} P_{nst})
    =  
    \nstavg{\textbf{w}} P_{nst}(\textbf{x},\textbf{r},a,t)\delta(a)
    - \nstavg{\textbf{w}}  \frac{P_{nst}(\textbf{x},\textbf{r},a,t)}{\tau_d(\textbf{x},t)}
    + \nstavg{\textbf{z}}P_{nst}(\textbf{x},\textbf{r},a,t)
\end{equation}
Integrating over all $\textbf{r}$ :
\begin{equation}
    \partial_a (\condavg{\textbf{w}}{\textbf{r}}(\textbf{x},t,a))
    =  
    \condavg{\textbf{w}}{\textbf{r}}(\textbf{x},t,a)\delta(a)
    + \condavg{\textbf{z}}{\textbf{r}}(\textbf{x},t,a)
    \label{eq:da_w}
\end{equation}
We have determined numerically that,  
\begin{equation*}
    \condavg{\textbf{w}}{\textbf{r}}(\textbf{x},t,a)
    = \frac{d_p}{\tau_a}(- 0.4 e^{ - 2a/\tau_a}+0.1)
\end{equation*}
therefor, 
\begin{equation*}
    \condavg{\textbf{w}}{\textbf{r}}(\textbf{x},t,a)\delta(a)
    = \frac{d_p}{\tau_a}(-0.3)\delta(a)
\end{equation*}
and, 
\begin{equation*}
    \pdda\condavg{\textbf{w}}{\textbf{r}}(\textbf{x},t,a)
    =  2\frac{d_p}{\tau_a^2} e^{- 2 a/\tau_a}
\end{equation*}
Therefore, we might expect, 
\begin{equation}
    \condavg{\textbf{z}}{\textbf{r}}(\textbf{x},t,a)
    = \partial_a (\condavg{\textbf{w}}{\textbf{r}}(\textbf{x},t,a))
    - \condavg{\textbf{w}}{\textbf{r}}(\textbf{x},t,a)\delta(a)
    =  2\frac{d_p}{\tau_a^2} e^{- 2 a/\tau_a}
    + 0.3 \frac{d_p}{\tau_a}\delta(a)
\end{equation}
Doesn't make much sens. 

\tb{we are Regarding the radial components here only that. }
\paragraph{Relative velocity correlation :}
By multiplying \ref{eq:dt_Pi} by $\textbf{wr} (\CC,t)$ and integrating over $\CC$ gives,
\begin{multline}
    \pddt (\nstavg{\textbf{rw}} P_{nst})
    + \partial_a (\nstavg{\textbf{rw}} P_{nst})
    + \partial_{\textbf{x}} \cdot (\nstavg{\textbf{rw}\textbf{u} } P_{nst})
    + \partial_{\textbf{r}}  \cdot (\nstavg{\textbf{rw}\textbf{w}} P_{nst})
    =  \\
    \nstavg{\textbf{rw}} P_{nst}\delta(a)
    -  \frac{\nstavg{\textbf{rw}} P_{nst}}{\tau_d(\textbf{x},t)}
    + \nstavg{\textbf{ww}}P_{nst}
    + \nstavg{\textbf{rz}}P_{nst}
\end{multline}
Again we average over all $a$ : 
\begin{multline}
    \pddt (\condavg{\textbf{rw}}{\textbf{r}} P_{a}n_p)
    + \partial_a (\condavg{\textbf{rw}}{\textbf{r}} P_{a}n_p)
    + \partial_{\textbf{x}} \cdot (\condavg{\textbf{rw}\textbf{u} }{\textbf{r}} P_{a}n_p)
    =  
    \condavg{\textbf{rw}}{\textbf{r}} P_{a}n_p\delta(a)
    -  \frac{\condavg{\textbf{rw}}{\textbf{r}} P_{a}n_p}{\tau_d(\textbf{x},t)}
    + \condavg{\textbf{ww}}{\textbf{r}}P_{a}n_p
    + \condavg{\textbf{rz}}{\textbf{r}}P_{a}n_p
\end{multline}

Thus, in steady state averaged case, 
\begin{equation}
      \frac{\pnavg{\textbf{rw}}}{\tau_d(\textbf{x},t)}
    =  
    \pnavg{\textbf{rw} \delta(a)}
    + \pnavg{\textbf{rw} 1/\tau'}
    + \pnavg{\textbf{ww}}
    + \pnavg{\textbf{rz}}
\end{equation}


\subsection{The momentum equation}
Multiplying the topological equation by $\textbf{u}^i$ yields the momentum equations, 
\begin{multline}
    \pddt (\nstavg{m\textbf{u}} P_{nst})
    + \partial_a (\nstavg{m\textbf{u}} P_{nst})
    + \partial_{\textbf{x}} \cdot (\nstavg{m\textbf{u}\textbf{u} } P_{nst})
    + \partial_{\textbf{r}}  \cdot (\nstavg{m\textbf{u}\textbf{w}} P_{nst})
    =  \\
    \nstavg{\textbf{u}} P_{nst}(\textbf{x},\textbf{r},a,t)\delta(a)
    - \nstavg{\textbf{u}}  \frac{P_{nst}(\textbf{x},\textbf{r},a,t)}{\tau_d(\textbf{x},t)}
    + \nstavg{\textbf{b}+\textbf{f}}P_{nst}(\textbf{x},\textbf{r},a,t)
\end{multline}
Let's integrate over $\textbf{r}$ and $a$, 
\begin{multline}
    \pddt (\pnnavg{m\textbf{u}} n_p)
    + \partial_{\textbf{x}} \cdot (\pnnavg{m\textbf{u}\textbf{u} } n_p)
    =  
    \textbf{b}_p n_p 
    +\textbf{f}_p  n_p
    + \iint \left[
        \nstavg{m\textbf{u}} P_{nst}(\textbf{x},\textbf{r},a,t)\delta(a)
        - \nstavg{m\textbf{u}}  \frac{P_{nst}(\textbf{x},\textbf{r},a,t)}{\tau_d(\textbf{x},t)}
        \right] da d\textbf{r}
\end{multline}
The last term can be computed such that,
\begin{multline}
   + \iint \left[
       \nstavg{m \textbf{u}} P_{nst}(\textbf{x},\textbf{r},a,t)\delta(a)
       - \nstavg{m \textbf{u}}  \frac{P_{nst}(\textbf{x},\textbf{r},a,t)}{\tau_d(\textbf{x},t)}
       \right] da d\textbf{r}\\
    =
   + \avgcond{m \textbf{u}}^{\textbf{r}} P^\textbf{r}_{nst}(\textbf{x},t|wa =0) d\textbf{r}
   - \frac{n_p \pnnavg{m \textbf{u}}  }{\tau_d(\textbf{x},t)}
\end{multline}
Using the expression of the probability density we obtain
Consequently, the momentum equation can be rewritten, 
\begin{equation}
    \pddt (\pnnavg{m\textbf{u}} n_p)
    + \partial_{\textbf{x}} \cdot (\pnnavg{m\textbf{u}\textbf{u} } n_p)
    =  
    \textbf{b}_p n_p 
    +\textbf{f}_p  n_p 
    + \frac{\avgcond{m \textbf{u}}^{\textbf{r}}(\textbf{x},t|a =0) - n_p \pnnavg{m \textbf{u}}  }{\tau_d(\textbf{x},t)}
\end{equation}
where the last term is the derivative of the momentum difference in an averaged interaction.

\subsection*{The moment of momentum equaiton}
Multiplying the topological equation by $\textbf{r}\textbf{u}^i$ yields the momentum equations, 
\begin{multline}
    \pddt (\nstavg{m\textbf{ru}} P_{nst})
    + \partial_a (\nstavg{m\textbf{ru}} P_{nst})
    + \partial_{\textbf{x}} \cdot (\nstavg{m\textbf{ru}\textbf{u} } P_{nst})
    + \partial_{\textbf{r}}  \cdot (\nstavg{m\textbf{ru}\textbf{w}} P_{nst})
    =  
    + \nstavg{\textbf{wu}}P_\text{nst}
    \\
    \nstavg{\textbf{ru}} P_{nst}(\textbf{x},\textbf{r},a,t)\delta(a)
    - \nstavg{\textbf{ru}}  \frac{P_{nst}(\textbf{x},\textbf{r},a,t)}{\tau_d(\textbf{x},t)}
    - \nstavg{m\textbf{uu}}P_{nst}(\textbf{x},\textbf{r},a,t)
    + \nstavg{\textbf{r}(\textbf{f}+\textbf{b})}P_{nst}(\textbf{x},\textbf{r},a,t)
\end{multline}
integrating yields
\begin{multline}
    \pddt (\pnnavg{m\textbf{ru}} n_p)
    + \partial_{\textbf{x}} \cdot (\pnnavg{m\textbf{ru}\textbf{u} } n_p)
    =  
    + \pnnavg{\textbf{wu}}P_\text{nst}
    +  \frac{\pnnavg{\textbf{ru}}n_p(a=0) -\pnnavg{\textbf{ru}} n_p}{\tau_d(\textbf{x},t)}
    - \pnnavg{m\textbf{uw}}n_p 
    + \pnnavg{\textbf{r}(\textbf{f}+\textbf{b})}n_p 
\end{multline}
In the steady homogeneous case, 
\begin{multline}
    0
    =  
    + \pnnavg{\textbf{wu}}P_\text{nst}
    +  \frac{\pnnavg{\textbf{ru}}n_p(a=0) -\pnnavg{\textbf{ru}} n_p}{\tau_d(\textbf{x},t)}
    - \pnnavg{m\textbf{uw}}n_p 
    + \pnnavg{\textbf{r}(\textbf{f}+\textbf{b})}n_p 
\end{multline}

\section{flotation problem}

Let say we have a bubbles phase denoted by $b$, a droplet phase $d$ and a continuous  phase $f$. 
\begin{align}
    \pddt f_k^0
    +\div \left(
        f_k^0\textbf{u}_k^0
        - \mathbf{\Phi}_k^0
        \right)
    &= 
    s_k^0
    & \text{ in } \Omega_k(t),&\\
    \pddt f_I^0 
    +\divI
    (
        f_I^0 \textbf{u}_I^0
        - \mathbf{\Phi}_{I||}^0 
    )
    &= 
    s_I^0
    - \Jump{
        f_k (\textbf{u}_I^0 - \textbf{u}_k^0)
        + \mathbf{\Phi}_k^0
     } 
    & \text{ on } \Sigma(t),&
\end{align}
where $k = g,b,c$ and $I = I_{gb},I_{gc},I_{bc}$.
If we omit the triple line At the Lagrangian scale we can write 
If we model the gas phase and droplet phase by Lagrangian Dirac functions such that the Dirac function of the $b$ phase is represented by $\delta_k^i(\textbf{x}- \textbf{x}^i_k)$. 
The dirac delta function of the droplets is represented by $\delta(\textbf{x} - \textbf{x}^j)$. 
At the local scale they all obey the following conservation equation : 
\begin{equation*}
    \pddt \delta^i(\textbf{x})
    + \textbf{u}^i  \cdot \partial_{\textbf{x}} \delta(\textbf{x} - \textbf{x}^i)
    = 0 
\end{equation*}
\begin{equation*}
    \pddt \delta^j(\textbf{x})
    + \textbf{u}^i \cdot \partial_{\textbf{x}}  \delta(\textbf{x} + \textbf{r} - \textbf{x}^j)
    + (\textbf{u}^j - \textbf{u}^i) \cdot \partial_{\textbf{r}}  \delta(\textbf{x} + \textbf{r} - \textbf{x}^j)
    = 0 
\end{equation*}
\begin{equation*}
    \pddt \delta^a(t - a - t_c^{ij})
    + 
    \partial_a \delta^a(t - a - t_c^{ij})
    = 0 
\end{equation*}
Overall the state function evolve as, 
\begin{equation}
    \pddt \Pi
    + \partial_a \Pi
    + \partial_{\textbf{x}} \cdot (\textbf{u}^i \Pi)
    + \partial_{\textbf{r}}  \cdot (\textbf{w}^{ij} \Pi)
    = G_{ijl}+D_{ijl}
\end{equation}
Here we define the nearest particles' statistics by the following equaitons,
\begin{equation}
    P_{nst}(\textbf{x},\textbf{y},t,a)= 
    \int \sum_{i}\delta(\textbf{x}-\textbf{x}^i(\CC,t))
    \sum_{j}\delta(\textbf{x}+\textbf{r}-\textbf{x}^j(\CC,t)) 
    \delta(t+a-t_c^{ij}(\CC,t)) 
    h_{ij}(\CC,t) d\mathscr{P} 
\end{equation}
\begin{equation}
    \nstavg{q^{ij}}P_{nst}(\textbf{x},\textbf{y},t,a)= \int \sum_{i}\delta_i(\textbf{x}-\textbf{x}^i(\CC,t))
    \sum_{j}\delta_j(\textbf{x}+\textbf{r}-\textbf{x}^j(\CC,t)) h_{ij}(\CC,t) q^{ij}(\CC,t) d\mathscr{P} 
\end{equation}
notice that the index $j$ can be equal to $i$ since we sum over two different population of particles. 
Equally, the function $h_{ij}$ must be redefined but it is the same thing. 
If we consider $i=1\ldots N$ and $j = N+1\ldots 2N$ for example the indices doesn't interfere so it is working. 
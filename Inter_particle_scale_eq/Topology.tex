\subsection{Topological equations}

Let $\textbf{x}^i(\CC,t)$ (resp. $\textbf{x}^j(\CC,t)$) be the position at time $t$ of the particle $i$ (resp. $j$) for a realization of the flow $\CC$. 
In the following we drop the argument $\CC$ and $t$. 
Then, $\delta^i(\textbf{x}) = \delta(\textbf{x}- \textbf{x}^i(\CC,t))$ is the indicator function of having a particle $i$ at  $\textbf{x}$.
Similarly, we introduce $\delta^j(\textbf{x},\textbf{r}) = \delta(\textbf{x} + \textbf{r} - \textbf{x}^j)$ which is the indicator function of having the particle indexed $j$ at $\textbf{x}+\textbf{r}$. 
The aim is to derive a transport equation for the indicator function of the following state : a particle $i$ is at \textbf{x} with its nearest neighbor located at $\textbf{x}+\textbf{r}$. 
Lastly, $\delta^a(t - a - t_c^{ij}(\CC,t))$ will be use as the indicator function indicating the age of the interaction for a particle pair created at time $t^{ij}_c$. 
Such a function can be written $\Pi(\textbf{x},\textbf{r},t,a) = \delta^a(a)\delta^i(\textbf{x}) \delta^j(\textbf{x},\textbf{r}) h^{ij}$. 
Using the distribution formalism and taking the partial time derivative on these indicators functions, leads us to these relations, 
\begin{equation*}
    \pddt \delta^i(\textbf{x})
    + \textbf{u}^i  \cdot \partial_{\textbf{x}} \delta(\textbf{x} - \textbf{x}^i)
    = 0 
    % = \pddt \delta(\textbf{x} - \textbf{x}^i)
    % = \textbf{u}^i \partial_{\textbf{x}^i} \delta(\textbf{x} - \textbf{x}^i)
    % = - \textbf{u}^i \partial_{\textbf{x}} \cdot \delta(\textbf{x} - \textbf{x}^i)
\end{equation*}
\begin{equation*}
    \pddt \delta^j(\textbf{x})
    + \textbf{u}^i \cdot \partial_{\textbf{x}}  \delta(\textbf{x} + \textbf{r} - \textbf{x}^j)
    + (\textbf{u}^j - \textbf{u}^i) \cdot \partial_{\textbf{r}}  \delta(\textbf{x} + \textbf{r} - \textbf{x}^j)
    = 0 
\end{equation*}
\begin{equation*}
    \pddt \delta^a(t - a - t_c^{ij})
    + 
    \partial_a \delta^a(t - a - t_c^{ij})
    = 0 
\end{equation*}
where we added and subtracted $\textbf{u}^i \cdot \partial_{\textbf{x}}  \delta(\textbf{x} + \textbf{r} - \textbf{x}^j)$ in the second equation to make appear the $\partial_{\textbf{x}}$ derivative. 
Noticing that the Lagrangian quantities $\textbf{u}^i$ and $\textbf{u}^j$ are solely function of $(\CC,t)$ we can rewrite the equations like so, 
\begin{equation}
    \pddt \delta^i
    + \partial_{\textbf{x}} \cdot (\textbf{u}^i \delta^i)
    = 0 
    \label{eq:dt_delta_i}
\end{equation}
\begin{equation}
    \pddt \delta^j
    + \partial_{\textbf{x}} \cdot (\textbf{u}^i \delta^j)
    + \partial_{\textbf{r}}  \cdot (\textbf{w}^{ij}\delta^j)
    = 0 
\end{equation}
\begin{equation*}
    \pddt \delta^a
    + 
    \partial_a \delta^a
    = 0 
\end{equation*}
where $\textbf{w}^{ij} = (\textbf{u}^j - \textbf{u}^i)$ is teh relative velocity between nearest neighboring particles. 
Multiplying, this second equation by $\delta^a\delta^ih^{ij}$ and using \ref{eq:dt_delta_i} yields a transport equaiton for $\Pi(\textbf{x},\textbf{r},a,t)$, namely,
\begin{equation}
    \pddt \Pi
    + \partial_a \Pi
    + \partial_{\textbf{x}} \cdot (\textbf{u}^i \Pi)
    + \partial_{\textbf{r}}  \cdot (\textbf{w}^{ij} \Pi)
    = G_{ijl}+D_{ijl}
    \label{eq:dt_Pi}
\end{equation}
The source term $\delta^a\delta^i\delta^j \pddt h^{ij} = G_{ijl}+D_{ijl}$ account for the creation and destruction of nearest particle pairs. 
The starting point to derive the evolution equation of the indicator function $\Pi$ is the Louisville equation
\tb{we could do the same using Louivill eq and $\pddt \delta = 0$ but in this case the age cannot be defined }

The source term on the RHS of \ref{eq:dt_Pi} can be computed by carrying out the algebra and yields, 
\begin{equation*}
    \pddt h_{ij} 
    = 
    \sum_{l \neq i,j} 
    (\hat{\textbf{r}}_{li}\cdot \textbf{w}_{li}
    - 
    \hat{\textbf{r}}_{ji}\cdot\textbf{w}_{ji})
    \delta(r_{li} - r_{ji})
        h_{ij}
\end{equation*}
Where $r$ refer to $|\textbf{r}|$ and $\hat{\textbf{r}}$ being the normal unit vector corresponding to $\textbf{r}$. 
We refer the reader too \citet[Appendix A.]{zhang2023evolution} for more detail about the computation of this term.
When a nearest pair of particles is created $(\hat{\textbf{r}}_{li}\cdot \textbf{w}_{li} - \hat{\textbf{r}}_{ji}\cdot\textbf{w}_{ji}) > 0$, when a nearest pair is destroyed $(\hat{\textbf{r}}_{li}\cdot \textbf{w}_{li} - \hat{\textbf{r}}_{ji}\cdot\textbf{w}_{ji}) < 0$ since the particle $j$ must go faster. 
Thus,
\begin{equation*}
    \pddt h_{ij} 
    = 
    \sum_{l \neq i,j} 
    (\hat{\textbf{r}}_{li}\cdot \textbf{w}_{li}
    - 
    \hat{\textbf{r}}_{ji}\cdot\textbf{w}_{ji})^+
    \delta(r_{li} - r_{ji})
        h_{ij}
    + \sum_{l \neq i,j} 
    (\hat{\textbf{r}}_{li}\cdot \textbf{w}_{li}
    - 
    \hat{\textbf{r}}_{ji}\cdot\textbf{w}_{ji})^-
    \delta(r_{li} - r_{ji})
        h_{ij}
\end{equation*}

\subsection{Evolution of the nearest PDF and particle properties.}

Now that the transport equation of $\Pi(\textbf{x},\textbf{r},a,t,\CC)$ is well set we can derive an equation for evolution of the nearest pair using the nearest average procedure defined in \ref{eq:P_nst}, such that : 
\begin{equation}
    \pddt P_{nst} 
    + \partial_a P_{nst} 
    + \partial_\textbf{x} \cdot \left(\nstavg{\textbf{u}} P_{nst}\right) 
    + \partial_\textbf{r} \cdot \left( \nstavg{\textbf{w}} P_{nst} \right) 
    =   P_{nst}(\textbf{x},\textbf{r},0,t)\delta(a)
    - \frac{P_{nst}(\textbf{x},\textbf{r},a,t)}{\tau_d(\textbf{x},\textbf{r},a,t)}
    \label{eq:dt_P_nst}
\end{equation}
where $P_{nst}(\textbf{x},\textbf{r},0,t)\delta(a)$ is the number of the nearest created pair and,
 $\tau(\textbf{x},\textbf{r},a,t)$ is the rate of destruction of the nearest particles, knowing the the nearest particle is at $\textbf{r}$ with age $a$.
Since for any new nearest pair of particles a pair is destroyed we have the following equallity :
 \begin{equation}
    \int P_{nst}(\textbf{x},\textbf{r},0,t)\delta(a) dS(r)
    =\iint \frac{P_{nst}(\textbf{x},\textbf{r},a,t)}{\tau_d(\textbf{x},\textbf{r},a,t)}
    dadS(r)
 \end{equation}
where we integrate on any sphere surface of radius $r$ around the particles. 

\subsubsection{Random destruction assumption}

Following \citet{zhang2023evolution} we assume that the mesoscal gradient are lower higher than the particle-fluid-particle scales gradients. 
In this case \ref{eq:dt_P_nst} reads, 
\begin{equation}
    \partial_a P_{nst} 
    + \partial_\textbf{r} \cdot \left( \nstavg{\textbf{w}} P_{nst} \right) 
    = P_{nst}(\textbf{x},\textbf{r},0,t)\delta(a)
    - \frac{P_{nst}(\textbf{x},\textbf{r},a,t)}{\tau_d(\textbf{x},\textbf{r},a,t)}
\end{equation}
integrating over all $\textbf{r}$ this equation gives, 
\begin{equation}
    \partial_a P_{a} (\textbf{x},a,t)
    = P_{a}(\textbf{x},0,t)\delta(a)
    - \frac{P_{a}(\textbf{x},a,t)}{\condavg{\tau_d}{\textbf{r}}(\textbf{x},a,t)}
    \label{eq:dt_P_a}
\end{equation}
where, 
\begin{equation}
    P_a(\textbf{x},a,t)n_p(\textbf{x},t)
    = \int P_{nst}(\textbf{x},\textbf{r},a,t) d\textbf{r}
\end{equation}
and,
\begin{equation}
    \frac{P_{a}(\textbf{x},a,t) n_p(\textbf{x},t)}{\condavg{\tau_d}{\textbf{r}}(\textbf{x},a,t)}
    = \int 
    \frac{P_{nst}(\textbf{x},\textbf{r},a,t)}{\tau_d(\textbf{x},\textbf{r},a,t)}
    d\textbf{r}.
\end{equation}
If we assume that the rate of destruction of the nearest pair isn't function of the relative position nor the age of interaction, it is possible to write $1/\avgcond{\tau_d}(\textbf{x},t) = 1/ \condavg{\tau_d}{r}(\textbf{x},a,t)$.
Then  \ref{eq:dt_P_a} can be written :
\begin{equation}
    \partial_a P_{a} (\textbf{x},a,t)
    = \delta(a) P_{a}(\textbf{x},0,t)
    - \frac{P_{a}(\textbf{x},a,t)}{\condavg{\tau_d}{}(\textbf{x},t)}
\end{equation}
Using the distribution formalism and the normalization condition $\int P_a(\textbf{x},t,a) da = 1$ we obtain : 
\begin{equation}
    P_a(\textbf{x},t,a)
    =\frac{e^{- a/\condavg{\tau_d}{}}(\textbf{x},t)}{\condavg{\tau_d}{}(\textbf{x},t)}
\end{equation}




\subsubsection{Flow induced anisotropy}

As we have shown in our resent paper the particle-fluid-particle stress in the vertical direction is manly caused by flow anisotropy. 
This flow anisotropy is measure by, 
\begin{equation}
    \condavg{\textbf{r}}{\textbf{r}}(\textbf{x},t,a)P_a(\textbf{x},a,t)n_p(\textbf{x},t)
    = \int \textbf{r} P_{nst}(\textbf{x},\textbf{r},a,t) d\textbf{r}
\end{equation}
\paragraph{Relative position equaiton :}
To derive an equation for $\textbf{r}$ we multiply \ref{eq:dt_Pi} and use this definition which gives in the first place :
\begin{multline}
    \pddt (\textbf{r} P_{nst})
    + \partial_a (\textbf{r} P_{nst})
    + \partial_{\textbf{x}} \cdot (\nstavg{\textbf{r}\textbf{u} } P_{nst})
    + \partial_{\textbf{r}}  \cdot (\nstavg{\textbf{r}\textbf{w}} P_{nst})
    =  \\
    \textbf{r} P_{nst}(\textbf{x},\textbf{r},0,t)\delta(a)
    - \textbf{r} \frac{P_{nst}(\textbf{x},\textbf{r},a,t)}{\tau_d(\textbf{x},t)}
    + \nstavg{\textbf{w}}P_{nst}
    \label{eq:dt_r_nst}
\end{multline}
Since we are actually interested in the average of $\textbf{r}$ over all \textbf{r}, namely, $\condavg{\textbf{r}}{\textbf{r}}$ we integrate this equation over all \textbf{r} and as before we assume no significant macroscopic changes. 
\begin{multline*}
    \partial_a (\condavg{\textbf{r}}{\textbf{r}}(\textbf{x},t,a)P_a(\textbf{x},a,t))
    =  
    \condavg{\textbf{r}}{\textbf{r}}(\textbf{x},t,a)P_a(\textbf{x},a,t)\delta(a)
     - \frac{\condavg{\textbf{r}}{\textbf{r}}(\textbf{x},t,a)P_a(\textbf{x},a,t)}{\tau_d(\textbf{x},t)} 
    + \condavg{\textbf{w}}{\textbf{r}}(\textbf{x},t,a)P_a(\textbf{x},a,t)
\end{multline*}
Noticing that the first term can be decomposed in $\partial_a (\condavg{\textbf{r}}{\textbf{r}}(\textbf{x},t,a)P_a(\textbf{x},a,t)) = \partial_a (\condavg{\textbf{r}}{\textbf{r}}(\textbf{x},t,a))P_a(\textbf{x},a,t) + \condavg{\textbf{r}}{\textbf{r}}(\textbf{x},t,a) \partial_a (P_a(\textbf{x},a,t))$ the previous expression simplify to : 
\begin{equation*}
    \partial_a (\condavg{\textbf{r}}{\textbf{r}}(\textbf{x},t,a))
    =  
    \condavg{\textbf{r}}{\textbf{r}}(\textbf{x},t,a)\delta(a)
    + \condavg{\textbf{w}}{\textbf{r}}(\textbf{x},t,a)
    \label{eq:da_r}
\end{equation*}
This equation holds for components of $\textbf{r}$ not its norm. 

\paragraph{Relative distance correlation :}
Now let's derive the first moment of this equation by multiplying \ref{eq:dt_P_nst} by $\textbf{rr}$. 
It reads, 
\begin{multline}
    \pddt (\nstavg{\textbf{rr}} P_{nst})
    + \partial_a (\nstavg{\textbf{rr}} P_{nst})
    + \partial_{\textbf{x}} \cdot (\nstavg{\textbf{rr}\textbf{u} } P_{nst})
    + \partial_{\textbf{r}}  \cdot (\nstavg{\textbf{rr}\textbf{w}} P_{nst})
    =  \\
    \nstavg{\textbf{rr}} P_{nst}(\textbf{x},\textbf{r},0,t)\delta(a)
    -  \frac{\nstavg{\textbf{rr}} P_{nst}(\textbf{x},\textbf{r},a,t)}{\tau_d(\textbf{x},t)}
    + \nstavg{\textbf{rw}}P_{nst}
    + \nstavg{\textbf{wr}}P_{nst}
\end{multline}
averaging over all \textbf{r} and considering homogeneous medium gives,
\begin{multline}
    \pddt (\condavg{\textbf{rr}}{\textbf{r}} P_{a}n_p)
    + \partial_a (\condavg{\textbf{rr}}{\textbf{r}} P_{a}n_p)
    + \partial_{\textbf{x}} \cdot (\condavg{\textbf{rr}\textbf{u} }{\textbf{r}} P_{a}n_p)
    =  
    \condavg{\textbf{rr}}{\textbf{r}} P_{a}n_p\delta(a)
    -  \frac{\condavg{\textbf{rr}}{\textbf{r}} P_{a}n_p}{\tau_d(\textbf{x},t)}
    + \condavg{\textbf{rw}}{\textbf{r}} P_{a}n_p
    + \condavg{\textbf{wr}}{\textbf{r}} P_{a}n_p
\end{multline}




\subsubsection{Dynamical description for particle without mass transfer}
First we define the Lagrangian balance equation for a particle $i$ such that 
\begin{equation}
    \ddt \textbf{u}_i
     = \frac{
        \textbf{b}_i
        + \textbf{f}_i
     }{m_i},
     \label{eq:dt_u_i}
 \end{equation}
From this equation we can also define an equation for the evolution of the relative velocity $\textbf{w}$ :
\begin{equation}
    \ddt \textbf{w}
    = \frac{
        \textbf{b}_i
        + \textbf{f}_i
    }{m_i}
    -\frac{
        \textbf{b}_j
        + \textbf{f}_j
    }{m_j}
    =\frac{
        \textbf{f}_i
    }{m_i}
    -\frac{
        \textbf{f}_j
    }{m_j}
    = 
    \textbf{a}_i
    - \textbf{a}_j,
    = \textbf{z}
    \label{eq:dt_w}
\end{equation}
\paragraph[short]{Relative velocity equation :}
Proceeding as before we derive the evolution equation of the averaged relative velocity fields, 
namely, 
\begin{multline}
    \pddt (\nstavg{\textbf{w}} P_{nst})
    + \partial_a (\nstavg{\textbf{w}} P_{nst})
    + \partial_{\textbf{x}} \cdot (\nstavg{\textbf{w}\textbf{u} } P_{nst})
    + \partial_{\textbf{r}}  \cdot (\nstavg{\textbf{w}\textbf{w}} P_{nst})
    =  \\
    \nstavg{\textbf{w}} P_{nst}(\textbf{x},\textbf{r},a,t)\delta(a)
    - \nstavg{\textbf{w}}  \frac{P_{nst}(\textbf{x},\textbf{r},a,t)}{\tau_d(\textbf{x},t)}
    + \nstavg{\textbf{z}}P_{nst}(\textbf{x},\textbf{r},a,t)
\end{multline}
Applying the hypothesis of slow varying macroscopic variables : 
\begin{equation}
    \partial_a (\nstavg{\textbf{w}} P_{nst})
    +\partial_{\textbf{r}}  \cdot (\nstavg{\textbf{w}\textbf{w}} P_{nst})
    =  
    \nstavg{\textbf{w}} P_{nst}(\textbf{x},\textbf{r},a,t)\delta(a)
    - \nstavg{\textbf{w}}  \frac{P_{nst}(\textbf{x},\textbf{r},a,t)}{\tau_d(\textbf{x},t)}
    + \nstavg{\textbf{z}}P_{nst}(\textbf{x},\textbf{r},a,t)
\end{equation}
Integrating over all $\textbf{r}$ :
\begin{equation}
    \partial_a (\condavg{\textbf{w}}{\textbf{r}}(\textbf{x},t,a))
    =  
    \condavg{\textbf{w}}{\textbf{r}}(\textbf{x},t,a)\delta(a)
    + \condavg{\textbf{z}}{\textbf{r}}(\textbf{x},t,a)
    \label{eq:da_w}
\end{equation}
 
\paragraph{Relative velocity correlation :}
By multiplying \ref{eq:dt_Pi} by $\textbf{wr} (\CC,t)$ and integrating over $\CC$ gives,
\begin{multline}
    \pddt (\nstavg{\textbf{rw}} P_{nst})
    + \partial_a (\nstavg{\textbf{rw}} P_{nst})
    + \partial_{\textbf{x}} \cdot (\nstavg{\textbf{rw}\textbf{u} } P_{nst})
    + \partial_{\textbf{r}}  \cdot (\nstavg{\textbf{rw}\textbf{w}} P_{nst})
    =  \\
    \nstavg{\textbf{rw}} P_{nst}\delta(a)
    -  \frac{\nstavg{\textbf{rw}} P_{nst}}{\tau_d(\textbf{x},t)}
    + \nstavg{\textbf{ww}}P_{nst}
    + \nstavg{\textbf{rz}}P_{nst}
\end{multline}
Again we average over all $a$ : 
\begin{multline}
    \pddt (\condavg{\textbf{rw}}{\textbf{r}} P_{a}n_p)
    + \partial_a (\condavg{\textbf{rw}}{\textbf{r}} P_{a}n_p)
    + \partial_{\textbf{x}} \cdot (\condavg{\textbf{rw}\textbf{u} }{\textbf{r}} P_{a}n_p)
    =  
    \condavg{\textbf{rw}}{\textbf{r}} P_{a}n_p\delta(a)
    -  \frac{\condavg{\textbf{rw}}{\textbf{r}} P_{a}n_p}{\tau_d(\textbf{x},t)}
    + \condavg{\textbf{ww}}{\textbf{r}}P_{a}n_p
    + \condavg{\textbf{rz}}{\textbf{r}}P_{a}n_p
\end{multline}
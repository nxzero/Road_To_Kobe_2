% \chapter{Equivalence between the momentum dispersed averaged and particular averaged equations.}
\chapter{Complete development of the particular averaged equations for the moments of inertia and momentum of the dispersed phase.}
\label{ap:exp}

This appendix aim to prove the Equivalence between the continuous and particular averaged momentum equations derived in \ref{sec:introavg} and \ref{sec:Lagrange_to_Euler} but for arbitrary order of the expansion.
We start by the derivation of the mass related equations, i.e. the expansion of the continuous mass balance and the transport of the moment of inertia of the particles for arbitrary order.
Then we carry out the same process for the momentum equations, and we show that the equivalence of \citet{nott2011suspension} is false.
Finally, we explore specific cases, as the solid spherical particle assumption, the axisymmetric shaped particles, and the case of viscous spherical droplets.

\tb{
\section{Generalized formula for the Taylor expansion of a dispersed averaged quantity.}
\label{sec:expansion_generalized}
In the following parts we will make use of specific notations to express expansion at arbitrary order.
Therefore, in this section we introduce the above named notation.
Let, $f$ be a physical quantity inside the dispersed phase.
Note that $\bm{f}$ could be a vector $f_i$, a tensor $f_{ij}$, it really doesn't matter at this point.
By carrying out an expansion of $g$ at a particle center $\alpha$ (see \ref{eq:expansion}), we show that for any tensor $f$,
\begin{equation*}
    \int_{V_\alpha} g \bm{f} dV
    = g_\alpha \int_{V_\alpha} \bm{f} dV
    - \partial_i g|_\alpha \int_{V_\alpha} r_i\bm{f}  dV
    % + \partial_i\partial_j g|_\alpha \frac{1}{2} \int_{V_\alpha} \bm{f} r_irj dV
    + \ldots,
    + \frac{(-1)^q}{q!}\partialp{1}{q} g|_\alpha \int_{V_\alpha} \pri{1}{q}\bm{f}  dV
\end{equation*}
where we have use $\frac{\partial}{\partial\bm{y}} g(\bm{x},\bm{y}) = - \frac{\partial}{\partial \bm{x}} g(\bm{y},\bm{x})$.
We recall that $\bm{y}$ is the local position vector and $\bm{x}$ the global one, thus any function of $\bm{x}$ can go out the integration by $dV$.
The $|_\alpha$ notation indicate that we take the value of the receding variable at $\bm{y}_\alpha$, e.g. $g|_\alpha = g(\bm{x},\bm{y}_\alpha)$.
Then it follows,
\begin{align}
    \phi\left<\bm{f}\right>^d
    &= \sum_\alpha g_\alpha \int_{V_\alpha} \bm{f} dV
    - \partial_i \cdot \sum_\alpha g_\alpha \int_{V_\alpha} r_i \bm{f} dV
    % + \frac{1}{2}\bm{\nabla\nabla} : \sum_\alpha g_\alpha \int_{V_\alpha} f \bm{r} \bm{r} dV
    \ldots
    + \frac{(-1)^q}{q!}\partialp{1}{q} \sum_\alpha g_\alpha \int_{V_\alpha} \pri{1}{q} \bm{f}  dV
    \nonumber\\
    % &= \sum_\alpha g_\alpha \bm{Q}_\alpha^0
    % - \bm{\nabla} \cdot \sum_\alpha g_\alpha \bm{Q}_\alpha^1
    % % + \frac{1}{2}\bm{\nabla\nabla} : \sum_\alpha g_\alpha  \bm{Q}_\alpha^2
    % \ldots
    % + \frac{(-1)^q}{q!}\partialp{1}{q} \sum_\alpha g_\alpha \bm{Q}_\alpha^q\\
    % \\
    &= \sum_{q=0}^\infty \left[\frac{(-1)^q}{q!} \partialp{1}{q} \sum_{\alpha} g_\alpha \int_{V_\alpha} \pri{1}{q} \bm{f}  dV\right]
    \nonumber\\
    \label{eq:monoexp}
    &= \sum_{q=0}^\infty \frac{(-1)^q}{q!} \partialp{1}{q} 
        \pavg{\int_{V_\alpha} \pri{1}{q} \bm{f}  dV}
\end{align}
where we can note that the integral is the volume moment of order $q$ of the quantity $\bm{f}$ of each particle $\alpha$.
One can note that all the terms in the sum over the $\alpha$ index correspond to particles phase average.
Certainly, we can even go further in the development, if we consider the expansion of the quantity $\bm{f}$ at the $n^{th}$ order around the center of mass of each particle.
Indeed, by doing so, it yields,
\begin{align*}
    \phi\left<\bm{f}\right>^d &= \sum_{q=0}^\infty \left[\frac{(-1)^q}{q!} \partialp{1}{q} \sum_{\alpha} g_\alpha \int_{V_\alpha} \pri{1}{q} \sum_{n=0}^\infty \left(
        \frac{1}{n!} \hatpartialp{1}{n}\bm{f}|_\alpha \prj{1}{n}
    \right) dV\right]\\
    &= \sum_{q=0}^\infty \sum_{n=0}^\infty\left[\frac{(-1)^q}{n!q!} \partialp{1}{q} \sum_{\alpha} g_\alpha 
    \hatpartialp{1}{n}\bm{f}|_\alpha 
    \int_{V_\alpha} \pri{1}{q}  
        \prj{1}{n}
    dV\right] 
\end{align*}
where we recall that $\hat{\partial}$ is the derivative over the local variable $\bm{y}$ thus it cannot go out the particular average.  
In the integral we recognize the definition of the moment of inertia $\mathcal{G}$, of order $q+n$. 
This series is a double infinite sum on $n$ and $q$, which is not very convenient when one want to keep only the term below a given order. 
Therefore, we introduce a new index $l$ which correspond to the maximum order of the series. 
% \begin{align}
    %     \rho_d \phi\left<\bm{f}\right>^d
    %     = \sum_{q=0}^\infty\sum^\infty_{n=1}\bm{\nabla}^q \sum_{\alpha} g_\alpha  \left[\frac{(-1)^q}{q!n!}   \bm{\hat{\nabla}^n u}|_{y_\alpha}\bm{\mathcal{G}_\alpha}^{q+n}\right]\\
    % \end{align}
So, we rearrange this equation in terms of the index $l = n + q$, and manipulate the product limits to keep only the $i_m$ index and remove the $j_m$.
By doing so we can group the product inside the integral such that, 
\begin{equation*}
    \phi\left<\bm{f}\right>^d 
    = \sum_{l=0}^\infty \sum_{q}^l\left[\frac{(-1)^{q}}{q!(l-q)!} \partialp{1}{q} \sum_{\alpha} g_\alpha 
    \hatpartialpi{q+1}{l}\bm{f}|_\alpha 
    \int_{V_\alpha} 
        \pri{1}{l}  
    dV\right] 
\end{equation*}
% \begin{align}
%     \rho_d \phi\left<\bm{f}\right>^d
%     =\sum_{\alpha} g_\alpha \sum_{l=0}^\infty\sum^l_{q=1} \left[\frac{(-1)^q}{q!(l-q)!}\bm{\nabla^q} \bm{\hat{\nabla}^{l-q} f}|_{y_\alpha}\bm{\mathcal{G}_\alpha}^{l}\right]\\
%     \label{eq:final}
% \end{align}
which is the results given by the application of the general Leibniz's rule when applying the $l^{th}$ order derivative over the product $g\bm{f}$.
In terms of particular average this expression yields,
\begin{equation}
    \davg{f}
    = \sum_{l=0}^\infty
    \sum_{q=0}^l \partialp{1}{q} \frac{(-1)^q}{q!(l-q)!}
    \pavg{\hatpartialpi{q+1}{l}\bm{f}|_\alpha \mathcal{G}^\alpha_{i_1i_2\ldots i_l}}
    \label{eq:doubleexpansion}
\end{equation}
where we used the expression for the $l^{th}$ order moment of inertia tensor. 
}

\section{General equivalence for an arbitrary conservation law}

A general conservation equation over the $k$ phase is written as, 
\begin{multline}
    \frac{\partial}{\partial t} (\phi_k\kavg{f})
    + \grad \cdot \left(
        \phi_k \kavg{f \textbf{u}}
    \right)\\
    = \grad \cdot \left(
        \phi_k \kavg{\bm{\Phi}}
    \right)
    + \phi_k \kavg{\textbf{S}}
    + a_I \Iavg{
        \bm{\Phi}_k \cdot \textbf{n}_k
        + f_k 
        \left(
            \textbf{u}_I
            - \textbf{u}_k
        \right) \cdot \textbf{n}_k
    } 
\label{ap:eq:avg_k_global}
\end{multline}
besides the local balance available inside the volume of the $k$ phase,
\begin{equation}
    \pddt f_k
    = \grad \cdot \left(
        \bm{\Phi}_k
        - f_k\textbf{u}_k
        \right)
    + \textbf{S}_k
\end{equation}
In the particular average formalism the same balance can be written as, 
\begin{equation}
    \pddt   \left(\pavg{q_\alpha}\right)
    + \grad \cdot \left(\pavg{q_\alpha \textbf{u}_\alpha}\right)
    = \pavg{\int_{V_\alpha} \textbf{S}_k dV }
    + \pavg{\int_{S_\alpha} \left[\bm{\Phi}_k + f (\textbf{u}_I-\textbf{u}_k) \right] \cdot \textbf{n}_k d S}
\end{equation}
where $q_\alpha = \int_{V_\alpha} f_k dV$.
The expansion of all the terms inside the phase average equation are rather straight forward.
Nevertheless, the divergence of the averaged non-convective flux is of a particular interest. 
Indeed, it is the only term that doesn't appear inside the particular average balance, which is normal since we transport integrated quantities.
It can be show that, 
\begin{equation}
    \partial_{i_{l+1}}
    (\phi_k \kavg{\bm{\Phi}_{i_{l+1}}})=
    \sum_l^\infty
    \left[
        \frac{(-1)^{l}}{l!}
        \prod^{l+1}_{m=1}
        \partial_{i_m}
        \sum_{\alpha}
        g_{\alpha}
        \int_{V_\alpha}
        \prod^{l}_{m=1}
        r_{i_m} \bm{\Phi}_{i_{l+1}}dV
    \right],
\end{equation}
By decomposition into symmetric and antisymmetric part we arrive to the expression, 
\begin{equation}
    \prod^{l}_{m=1} r_{i_m} \bm{\Phi}_{i_{l+1}}
    = \frac{1}{l+1}
    \underbrace{\left[
    \sum_{n=1}^{l+1} \bm{\Phi}_{i_{n}}\prod^{l+1}_{\substack{m=1 \\ m \neq n}} r_{i_m} \right.}_{\text{Symmetric part}}
    +\underbrace{\left.\sum_{n=1}^{l} (r_{i_n}\bm{\Phi}_{i_{l+1}} - r_{i_{l+1}}\bm{\Phi}_{i_{n}}) \prod^{l}_{\substack{m=1 \\ m \neq n}} r_{i_m} \right]}_{\text{Antisymmetric part}}.
\end{equation}
Taking the divergence of the same product yields, 
\begin{align*}
    \grad \cdot \left(\prod^{l+1}_{m=1} r_{i_m} \bm{\Phi}\right)
    &= \prod^{l+1}_{m=1} r_{i_m} \grad \cdot \bm{\Phi}
    + \bm{\Phi} \cdot \grad \prod^{l+1}_{m=1} r_{i_m}\\
    &= \prod^{l+1}_{m=1} r_{i_m} \grad \cdot \bm{\Phi}
    + \bm{\Phi}  \cdot\sum_{m=1}^l \grad r_{i_m}  \prod^{l}_{\substack{n=1 \\ n \neq m}} r_{i_m}\\,
    &= \prod^{l+1}_{m=1} r_{i_m} \grad \cdot \bm{\Phi}
    + \sum_{m=1}^l \bm{\Phi}_{i_m}  \prod^{l}_{\substack{n=1 \\ n \neq m}} r_{i_m}\\,
\end{align*}
Which, by considering the previous relation and the fact that the anti symmetric part is null under derivation, gives, 
\begin{equation}
    \prod^{l}_{m=1} r_{i_m} \bm{\Phi}_{i_{l+1}}
    =\frac{1}{l+1}\sum_{m=1}^{l+1} \bm{\Phi}_{i_m}  \prod^{l+1}_{\substack{n=1 \\ n \neq m}} r_{i_m}
    =\frac{1}{l+1}\grad\cdot \left(\prod^{l+1}_{m=1} r_{i_m} \bm{\Phi}\right)
    - \frac{1}{l+1}\prod^{l+1}_{m=1} r_{i_m} \grad\cdot \bm{\Phi}.
\end{equation}
Using the micro scale conservation equation for the second term it reads, 
\begin{equation}
    (l+1) \prod^{l}_{m=1} r_{i_m} \bm{\Phi}_{i_{l+1}}
    =\grad \cdot \left(\prod^{l+1}_{m=1} r_{i_m} \bm{\Phi}\right)
    - \prod^{l+1}_{m=1} r_{i_m} \left(
        \pddt f 
        + \grad \cdot (f \textbf{u})
        - \textbf{S}
    \right).
\end{equation}
Injecting this term into the initial expansion and using Gauss theorem on the first term gives,
\begin{align*}
    \partial_{i_{l+1}}
    (\phi_k \kavg{\bm{\Phi}_{i_{l+1}}})
    & =\sum_l^\infty
    \left[
        \frac{(-1)^{l}}{(l+1)!}
        \prod^{l+1}_{m=1}
        \partial_{i_m}
        \sum_{\alpha}
        g_{\alpha}
        \int_{S_\alpha}
        \prod^{l+1}_{m=1}
        r_{i_m} \bm{\Phi}_k \cdot \textbf{n}_k dS
    \right],\\
    & +\sum_l^\infty
    \left[
        \frac{(-1)^{l}}{(l+1)!}
        \prod^{l+1}_{m=1}
        \partial_{i_m}
        \sum_{\alpha}
        g_{\alpha}
        \int_{V_\alpha}
        \prod^{l+1}_{m=1}
        r_{i_m} \textbf{S} dV
    \right],\\
    & +\sum_l^\infty
    \left[
        \frac{(-1)^{l+1}}{(l+1)!}
        \prod^{l+1}_{m=1}
        \partial_{i_m}
        \sum_{\alpha}
        g_{\alpha}
        \int_{V_\alpha}
        \prod^{l+1}_{m=1}
        r_{i_m} \pddt f dV
    \right],\\
    & +\sum_l^\infty
    \left[
        \frac{(-1)^{l+1}}{(l+1)!}
        \prod^{l+1}_{m=1}
        \partial_{i_m}
        \sum_{\alpha}
        g_{\alpha}
        \int_{V_\alpha}
        \prod^{l+1}_{m=1}
        r_{i_m} \grad \cdot (f \textbf{u}) dV
    \right],\\
\end{align*}
where we recognize the expansion of the source term \textbf{S} and the one of the non-convective flux dotted with the normal, i.e. $\bm{\Phi}\cdot\textbf{n}$.
Therefore, those series cancel directly with the expansion of the terms in the general phase averaged conservation equation, only the zeroth moments remain, yielding the particular average of the zeroth moment of \textbf{S} and $\bm{\Phi}\cdot \textbf{n}_k$. 
Now let's focus on the third and fourth term. Namely, 
\begin{align*}
    \int_{V_\alpha}\pri{1}{l+1}\left[
        \pddt f 
        + 
        \grad \cdot (f \textbf{u}) 
    \right]dV
    &= \int_{V_\alpha} \left[
    \pddt \left(\pri{1}{l+1}  f \right)
    + 
    \grad \cdot \left(
    \prod^{l+1}_{m=1}
    r_{i_m} f \textbf{u}
    \right) \right]dV   \\ 
    &- \int_{V_\alpha}f\left[
        \pddt \left(\prod^{l+1}_{m=1}
        r_{i_m} \right)
        +
        \textbf{u} \cdot
        \grad \pri{1}{l+1}  
    \right]
    dV.\\
\end{align*}
Noticing that $\pddt \textbf{r} = - \textbf{u}_\alpha$, $\grad \textbf{r} = \textbf{I}$, and using the Reynolds transport theorem on the first term on the RHS (\ref{eq:q_alpha_dt}), gives, 
\begin{align*}
    \int_{V_\alpha}\pri{1}{l+1}\left[
        \pddt f 
        + 
        \grad \cdot (f \textbf{u}) 
    \right]dV
    &= \ddt \int_{V_\alpha} 
    \pri{1}{l+1}  f dV   \\
    &- \int_{S_\alpha} 
    \pri{1}{l+1}  f (\textbf{u}_I - \textbf{u})\cdot \textbf{n} dV \\
    &- \int_{V_\alpha}f \sum_{e=1}^{m=1} 
    \prod^{l+1}_{\substack{m=1\\ m\neq e}} 
    r_{i_m} 
    w_{i_e}
    dV.\\
\end{align*}
The second term on the RHS, i.e. the moments of the phase transfer term, cancels directly with the expansion of the interfacial average term in the phase average balance \ref{ap:eq:avg_k_global}.
In the end, the expansion of the non-convective flux reads as,
\begin{align*}
    \partial_{i_{l+1}}
    (\phi_k \kavg{\bm{\Phi}_{i_{l+1}}})
    & =\sum_l^\infty
    \left[
        \frac{(-1)^{l+1}}{(l+1)!}
        \prod^{l+1}_{m=1}
        \partial_{i_m}
        \sum_{\alpha}
        g_{\alpha}\ddt \int_{V_\alpha}
        \left(\pri{1}{l+1}  f \right)dV
    \right]\\
    &+\sum_l^\infty
    \left[
        \frac{(-1)^{l}}{(l+1)!}
        \prod^{l+1}_{m=1}
        \partial_{i_m}
        \sum_{\alpha}
        g_{\alpha}\int_{V_\alpha}
        f\sum_{e=1}^{m=1} 
        \prod^{l+1}_{\substack{m=1\\ m\neq e}} 
        r_{i_m} 
        w_{i_e}
         dV
    \right].\\
\end{align*}
Note that the time derivative doesn't commute with the particular average operator, instead one must use \ref{eq:q_alpha_dt_avg}, giving the following expression, 
\begin{align}
    \partial_{i_{l+1}}
    (\phi_k \kavg{\bm{\Phi}_{i_{l+1}}})
    & =\sum_l^\infty
    \left[
        \frac{(-1)^{l+1}}{(l+1)!}
        \prod^{l+1}_{m=1}
        \partial_{i_m}
        \pddt
        \sum_{\alpha}
        g_{\alpha} \int_{V_\alpha}
        \pri{1}{l+1}  f dV
    \right]
    \\
    & + \sum_l^\infty
    \left[
        \frac{(-1)^{l+1}}{(l+1)!}
        \prod^{l+1}_{m=1}
        \partial_{i_m}
        \grad \cdot
        \sum_{\alpha}
        g_{\alpha} \textbf{u}_\alpha 
        \int_{V_\alpha}
        \pri{1}{l+1}  f  dV
        \right]\\
    % & - \sum_l^\infty
    % \left[
    %     \frac{(-1)^{l+1}}{(l+1)!}
    %     \prod^{l+1}_{m=1}
    %     \partial_{i_m}
    %     \sum_{\alpha}
    %     g_{\alpha} \psi
    %     \int_{V_\alpha}
    %     \pri{1}{l+1}  f  dV
    %     \right]\\
        &+\sum_l^\infty
    \left[
        \frac{(-1)^{l}}{(l+1)!}
        \prod^{l+1}_{m=1}
        \partial_{i_m}
        \sum_{\alpha}
        g_{\alpha}\int_{V_\alpha}
        f\sum_{e=1}^{m=1} 
        \prod^{l+1}_{\substack{m=1\\ m\neq e}} 
        r_{i_m} 
        w_{i_e}
        dV
    \right].
    \label{ap:eq:partial_Phi}
\end{align}
In this expression we clearly recognize the time derivative of the moments of $\pri{1}{l+1}f$, which will obviously cancel with the first term on the LHS of \ref{ap:eq:avg_k_global}.
Now let's compare the two remaining term to the expansion of the advection term in \ref{ap:eq:avg_k_global}, which reads as, 
\begin{equation}
    \grad \cdot \phi_k \kavg{\textbf{u} f}
    = \sum_{l=0}^\infty  
    \left[
        \frac{(-1)^l}{l!} \prod^{l}_{m=1}\partial_{i_m}
        \grad \cdot
        \sum_\alpha  g_\alpha \int_{V_\alpha} \prod^l_{m=1}r_{i_m} \textbf{u} f dV
    \right],
\end{equation}
using the decomposition of the velocity, $\textbf{u} = \textbf{u}_\alpha + \textbf{w}$, 
\begin{align}
    \grad \cdot \phi_k \kavg{\textbf{u} f}
    &= \sum_{l=0}^\infty  
    \left[
        \frac{(-1)^l}{l!} \prod^{l}_{m=1}\partial_{i_m}
        \grad \cdot
        \sum_\alpha  g_\alpha \textbf{u}_\alpha  \int_{V_\alpha} \prod^l_{m=1}r_{i_m} f dV
    \right]\\
    &+ \sum_{l=0}^\infty  
    \left[
        \frac{(-1)^l}{l!} \prod^{l}_{m=1}\partial_{i_m}
        \grad \cdot
        \sum_\alpha  g_\alpha \int_{V_\alpha} \prod^l_{m=1}r_{i_m} \textbf{w} f dV
    \right]
    \label{ap:eq:partial_uf}
\end{align}
It is now clear that the first term of this equation will cancel out with the second term on the RHS of \ref{ap:eq:partial_Phi} except for the zeroth order moments that are not present in the latter equation.
Besides, the difference of second term of \ref{ap:eq:partial_uf} with \ref{ap:eq:partial_Phi} can be written as follows, 
\begin{equation}    
    \sum_{l=0}^\infty  
    \left[
        \frac{(-1)^l}{l!} \prod^{l+1}_{m=1}\partial_{i_m}
        \sum_\alpha  g_\alpha 
        \int_{V_\alpha} f_k\left(
            \prod^l_{m=1}r_{i_m} w_{i_{l+1}} 
            -
            \frac{1}{l+1}
            \sum_{e=1}^{m=1} 
            \prod^{l+1}_{\substack{m=1\\ m\neq e}} 
            r_{i_m} 
            w_{i_e}
        \right)
        dV
    \right]
    \label{ap:eq:diff_rw_term}
\end{equation}
Note that the integral of \ref{ap:eq:diff_rw_term} is the difference of the tensor $\pri{1}{l}w_{l_{l+1}}$ and its own antisymmetric part, i.e. $\frac{1}{l+1} \sum_{e=1}^{m=1} \prod^{l+1}_{\substack{m=1\\ m\neq e}} r_{i_m} w_{i_e}$.
Consequently, the remaining term will be an antisymmetric tensor, in the indices $i_1i_2\ldots i_{l+1}$, which will ultimately cancel upon the operator $\partialp{1}{l+1}$.

At last, the only terms remaining in \ref{ap:eq:avg_k_global} are the following,
\begin{multline*}
    \pddt   \left(\pavg{\int_{V_\alpha} f_k dV}\right)
    + \grad \cdot \left(\pavg{\textbf{u}_\alpha \int_{V_\alpha} f_k  dV  }\right)\\
    = \pavg{\int_{V_\alpha} \textbf{S}_k dV }
    + \pavg{\int_{S_\alpha} \left[\bm{\Phi}_k 
    + f (\textbf{u}_I-\textbf{u}_k) \right] \cdot \textbf{n}_k d S},
\end{multline*}
If we note, $q_\alpha = \int_{V_\alpha} f_k dV$ it yields the particular averaged equation, i.e. 
\begin{equation}
    \pddt   \left(\pavg{q_\alpha}\right)
    + \grad \cdot \left(\pavg{\textbf{u}_\alpha q_\alpha  }\right)
    = \pavg{\int_{V_\alpha} \textbf{S}_k dV }
    + \pavg{\int_{S_\alpha} \left[\bm{\Phi}_k 
    + f (\textbf{u}_I-\textbf{u}_k) \right] \cdot \textbf{n}_k d S},
\end{equation}
Thus, the phase average on the $k$ phase is equivalent to the particular average on the same phase. 




\section{Particular surface averaged equation}
\begin{equation}
    \pddt a_I
    + \grad \cdot \left(
        a_I
        \Iavg{\textbf{u}}
    \right)
    = a_I \Iavg{\grad \cdot \textbf{u}}
\end{equation}

Each term give,
also, 
\begin{align*}
    \pddt a_I &= \frac{(-1)^{n}}{n!}
        \pddt
        \prod^{n}_{m=1}
        \partial_{i_m}
        \sum_{\alpha}
        g_{\alpha}
        \int_{S_\alpha}
        \pri{1}{n}dS \\
    \grad \cdot (a_I \Iavg{\textbf{u}^I}) &= \frac{(-1)^{n}}{n!}
        \grad\cdot\prod^{n}_{m=1}
        \partial_{i_m}
        \sum_{\alpha}
        g_{\alpha}
        \int_{S_\alpha}
        \pri{1}{n} \textbf{u}_IdS\\
    a_I \Iavg{\grad \cdot \textbf{u}^I} &= \frac{(-1)^{n}}{n!}
        \prod^{n}_{m=1}
        \partial_{i_m}
        \sum_{\alpha}
        g_{\alpha}
        \int_{S_\alpha}
        \pri{1}{n} (\grad \cdot \textbf{u}_I)dS\\
\end{align*}
Using the relation derived in \ref{ap:cinematic}, 
\begin{align*}
    \int_{S_\alpha} 
    \prod_{\substack{m = 1 \\m \neq e}}^{n} r_{i_m}
    \partial_k^I u^I_k dS 
    &=
    \ddt \int_{S_\alpha} \pri{1}{n}dS
    + \sum_{e=1}^n u^\alpha_{i_e} \int_{S_\alpha} 
    \prod_{\substack{m = 1 \\m \neq e}}^{n} r_{i_m}dS \\
    &+ \sum_{e=1}^n \int_{S_\alpha} u^I_k
    \prod_{\substack{m = 1 \\m \neq e}}^{n} r_{i_m}
    (n_kn_{i_e} - \delta_{ki_e}) dS \\
\end{align*}
we have,
\begin{align*}
    a_I \Iavg{\grad \cdot \textbf{u}^I} 
    &= \frac{(-1)^{n}}{n!}
    \pddt 
    \prod^{n}_{m=1}
    \partial_{i_m}
    \sum_{\alpha}
    g_{\alpha} 
    \int_{S_\alpha} \pri{1}{n}dS \\
    &+ \frac{(-1)^{n}}{n!}
    \grad \cdot
    \prod^{n}_{m=1}
    \partial_{i_m}
    \sum_{\alpha}
    g_{\alpha} 
     \textbf{u}_\alpha \int_{S_\alpha} \pri{1}{n}dS \\
    & + \frac{(-1)^{n}}{n!}
    \prod^{n}_{m=1}
    \partial_{i_m}
    \sum_{\alpha}
    g_{\alpha} 
    \sum_{e=1}^n u^\alpha_{i_e} \int_{S_\alpha} 
    \prod_{\substack{m = 1 \\m \neq e}}^{n} r_{i_m}dS \\
    & +  \frac{(-1)^{n}}{n!}
    \prod^{n}_{m=1}
    \partial_{i_m}
    \sum_{\alpha}
    g_{\alpha} 
    \sum_{e=1}^n \int_{S_\alpha} u^I_k
    \prod_{\substack{m = 1 \\m \neq e}}^{n} r_{i_m}
    (n_kn_{i_e} - \delta_{ki_e}) dS
\end{align*}
We can see that the time derivatives cancels out, therefore it reads, 
\begin{align*}
    &\pddt \pavg{\int_{S_\alpha} dS}
     + \grad \cdot \left(
         \pavg{\textbf{u}_\alpha \int_{S_\alpha} dS} 
    \right)
    = \pavg{\int_{S_\alpha} \grad \cdot \textbf{u}_I dS} \\
    &+ \frac{(-1)^{n+1}}{n!}
    \grad\cdot\prod^{n}_{m=1}
    \partial_{i_m}
    \pavg{
    \int_{S_\alpha}
    \pri{1}{n} \textbf{u}_IdS}\\
    &+ \frac{(-1)^{n}}{n!}
    \grad \cdot
    \prod^{n}_{m=1}
    \partial_{i_m}
    \pavg{
     \textbf{u}_\alpha \int_{S_\alpha} \pri{1}{n}dS }\\
    & + \frac{(-1)^{n}}{n!}
    \prod^{n}_{m=1}
    \partial_{i_m}
    \pavg{
    \sum_{e=1}^n u^\alpha_{i_e} \int_{S_\alpha} 
    \prod_{\substack{m = 1 \\m \neq e}}^{n} r_{i_m}dS} \\
    & +  \frac{(-1)^{n}}{n!}
    \prod^{n}_{m=1}
    \partial_{i_m}
    \pavg{
    \sum_{e=1}^n \int_{S_\alpha} u^I_k
    \prod_{\substack{m = 1 \\m \neq e}}^{n} r_{i_m}
    (n_kn_{i_e} - \delta_{ki_e}) dS}
\end{align*}
It is evident that the $3^{th}$ and fourth line simplify, 
\begin{align*}
    &\pddt \pavg{\int_{S_\alpha} dS}
     + \grad \cdot \left(
         \pavg{\textbf{u}_\alpha \int_{S_\alpha} dS} 
    \right)
    = \pavg{\int_{S_\alpha} \grad \cdot \textbf{u}_I dS} \\
    &+ \frac{(-1)^{n+1}}{n!}
    \grad\cdot\prod^{n}_{m=1}
    \partial_{i_m}
    \pavg{
    \int_{S_\alpha}
    \pri{1}{n} \textbf{u}_IdS - 
    \left(
        \int_{S_\alpha}
        \pri{1}{n} \textbf{u}_IdS
    \right)^S}\\
    &+ \frac{(-1)^{n}}{n!}
    \grad \cdot
    \prod^{n}_{m=1}
    \partial_{i_m}
    \pavg{
     \textbf{u}_\alpha \int_{S_\alpha} \pri{1}{n}dS 
     - \left(\textbf{u}_\alpha \int_{S_\alpha} \pri{1}{n} dS\right)^S }\\\\
    & +  \frac{(-1)^{n+1}}{n!}
    \prod^{n+1}_{m=1}
    \partial_{i_m}
    \pavg{\left(\int_{S_\alpha} u^I_k
    \pri{1}{n} n_kn_{i_{n+1}}  dS\right)^S}
\end{align*}
where $(...)^S$ is the symmetric part of the argument. 
Both terms are thus antisymmetric and vanish under the divergence operators. 
\begin{align*}
    &\pddt \pavg{\int_{S_\alpha} dS}
     + \grad \cdot \left(
         \pavg{\textbf{u}_\alpha \int_{S_\alpha} dS} 
    \right)
    = \pavg{\int_{S_\alpha} \grad_I \cdot \textbf{u}_I dS} \\
    & +  \frac{(-1)^{n+1}}{n!}
    \prod^{n+1}_{m=1}
    \partial_{i_m}
    \pavg{\int_{S_\alpha} u^I_k n_kn_{i_{n+1}} \pri{1}{n} dS}^S
\end{align*}

\section{Particular mass averaged equation}
Now that we have defined a formalism to carry out Taylor expansion of any dispersed averaged quantity $\bm{f}$, it is rather easy to express dispersed average mass balance equation in terms of particular average quantities. 
First, we recall the dispersed averaged mass conservation equation.
\begin{equation*}
    \partial_t \phi + \partial_k (\davg{u_k} \phi)
    = 0
\end{equation*}
As stated before, we start by taking the expansion of each continuous averaged terms independently.
So, we make use of \ref{eq:doubleexpansion} to carry out this expansion, it gives the following relations,
\begin{equation*}
    \phi
    = \sum^\infty_l \frac{(-1)^l}{l!} \partialp{0}{l} \pavg{\mathcal{G}^\alpha_{i_1i_2\ldots i_l}},
\end{equation*}
for the expansion of the volume fraction, and,
\begin{equation*}
    \phi \davg{u_k}
    = \sum^\infty_l
    \sum_{q=0}^l
    \frac{(-1)^q}{q!(l-q)!}
    \partialp{1}{q} 
        \pavg{
        \mathcal{G}^\alpha_{i_1i_2\ldots i_l}
        \hatpartialp{q}{l} u_k
    },
\end{equation*}
Therefore, we can rewrite the continuous mass balance equation as,
\begin{multline*}
    \partial_t \sum_l^\infty
    \left[
        \frac{(-1)^l}{l!} \partialp{0}{l} \pavg{\mathcal{G}^\alpha_{i_1i_2\ldots i_l}}
    \right]\\
    + \partial_k
    \sum_l^\infty \sum_{q=0}^l
    \left[
        \frac{(-1)^q}{q!(l-q)!}
        \partialp{1}{q} \pavg{
            \mathcal{G}^\alpha_{i_1i_2\ldots i_l}
            \hatpartialp{q}{l} u_k
        }
    \right]
    = 0.
\end{multline*}
This equation depicts the contribution of the $l^{th}$ order moment of inertia times the derivative of the velocity of order $l$.
Alternatively, the second term of this equation can be written using \ref{eq:monoexp}, yielding,
\begin{equation}
    \partial_t
    \sum_l^\infty
    \left[
        \frac{(-1)^l}{l!} \partialp{1}{l} \pavg{\mathcal{G}^\alpha_{i_1i_2\ldots i_l}}
    \right]
    + \partial_k
    \sum_{q=0}^\infty
    \left[
        \frac{(-1)^q}{q!}
        \partialp{1}{q}
        \pavg{
            \mathcal{P}^\alpha_{i_1i_2\ldots i_l k}
        }
    \right]
    = 0.
    \label{eq:mass_order_l}
\end{equation}
In this form we make appear the $q^{th}$ moment of momentum which isn't necessarily a term of order $l$, since it contain higher order contribution within itself (see expansion in the previous section).
Nevertheless, some simplifications are in order. 
Indeed, using 









The reader must keep in mind that this equation remain a scalar equation, therefore, in order to be solvable it needs, an additional set of equations.
That is why in the following sections we will present additional equations to complete the system.


\tb{
\section{Particular average of the momentum equation}
Now let's apply the same process for the momentum averaged equation. 
We recall the dispersed averaged equation of the momentum,
\begin{equation}
    \rho_d\phi \left<\frac{D \bm{u}}{D t}\right>^d
    =
    \rho_d\frac{\partial}{\partial t} (\phi \left<\bm{u}\right>^d)
    + \rho_d\bm{\nabla}\cdot(\phi \left<\bm{uu}\right>^d)
    = \bm{\nabla}\cdot(\phi \left<\bm{\sigma}\right>^d)
    +\sum_\alpha\int_{S_\alpha}\bm{n}\cdot\bm{\sigma} g dS
    +\phi\left<\bm{b}\right>^d,
    \label{eq:momentumd}
\end{equation}
where we explicitly show the two possible formulation for the left-hand side term. 
Each term can be expressed with the series expansion of $g$ around the center of the particle, thus using \ref{eq:monoexp} we have,
\begin{equation}
    \phi\left<b\right>^d_q
    % = \sum_{l=0}^\infty \sum_\alpha \left[\frac{(-1)^l}{l!} \bm{\nabla}^l g|_\alpha \int (\r_\alpha)^l\bm{b} dV\right],
    = \sum_{l=0}^\infty \left[\frac{(-1)^l}{l!} \prod^l_{m=1}\partial_{i_m} \sum_\alpha g_\alpha \int_{V_\alpha} \prod^l_{m=1}r_{i_m}b_q dV\right],
    \label{eq:C1}
\end{equation}
% \begin{equation}
%     \rho_d\phi \left<\hat{D} u_q\right>^d
%     % = \sum_{l=0}^\infty \sum_\alpha \left[\frac{(-1)^l}{l!} \bm{\nabla}^l g|_\alpha \int (\r_\alpha)^l\bm{u} dV\right],
%     = \sum_{l=0}^\infty  \left[\frac{(-1)^l}{l!} \prod^l_{m=1} \partial_{i_m}  \sum_\alpha  g_\alpha \int_{V_\alpha} \rho_d \prod^l_{m=1}r_{i_m}  \hat{D} u_q dV\right],
%     % \label{eq:C3}
% \end{equation}
\begin{equation}
    \partial_t \phi \left< u\right>^d_q
    % = \sum_{l=0}^\infty \sum_\alpha \left[\frac{(-1)^l}{l!} \bm{\nabla}^l g|_\alpha \int (\r_\alpha)^l\bm{u} dV\right],
    = \sum_{l=0}^\infty  \left[\frac{(-1)^l}{l!} \prod^l_{m=1} \partial_{i_m} \partial_t \sum_\alpha  g_\alpha \int_{V_\alpha} \prod^l_{m=1}r_{i_m}  u_qdV\right],
    \label{eq:dt}
\end{equation}
\begin{equation}
    \partial_{i_{l+1}} \phi \left< u_{{i_{l+1}}} u_q\right>^d
    % = \sum_{l=0}^\infty \sum_\alpha \left[\frac{(-1)^l}{l!} \bm{\nabla}^l g|_\alpha \int (\r_\alpha)^l\bm{u} dV\right],
    = \sum_{l=0}^\infty  \left[\frac{(-1)^l}{l!} \prod^{l+1}_{m=1}\partial_{i_m}\sum_\alpha  g_\alpha \int_{V_\alpha} \prod^l_{m=1}r_{i_m} u_{i_{l+1}} u_q dV\right],
    \label{eq:u_pu_q1}
\end{equation}
\begin{equation}
    \sum_\alpha\int_{S_\alpha}n_p\sigma_{pq} g dS
    = \sum_{l=0}^\infty  \left[\frac{(-1)^l}{l!} \prod^l_{m=1}\partial_{i_m} \sum_\alpha g_\alpha\int_{S_\alpha} \prod^l_{m=1}r_{i_m} n_p\sigma_{pq} dS\right].
    \label{eq:C4}
\end{equation}
\begin{equation}
    \partial_{i_{l+1}}
    \phi\left<\sigma_{i_{l+1}q}\right>^d=
    \sum_l^\infty
    \left[
        \frac{(-1)^{l}}{(l)!}
        \prod^{l+1}_{m=1}
        \partial_{i_m}
        \sum_{\alpha}
        g_{\alpha}
        \int_{V_\alpha}
        \prod^{l}_{m=1}
        r_{i_m} \sigma_{i_{l+1} q}dV
    \right],
    \label{eq:expsig}
\end{equation}
where $\hat{D} u_q$ is the total derivative defined as $\hat{D} u_q = \partial_t u_q + u_k \hat{\partial}_k u_q$.
The $(l+2)^{th}$ order tensor in the latter equation $\prod^{l}_{m=1} r_{i_m} \sigma_{qp}$ can be rewritten as the sum of its symmetric part and antisymmetric part \citep{nott2011suspension}.
We recall that the symmetric part of a tensor of arbitrary order is just the sum of all its permutation divided by the number of permutation, and the antisymmetric part is the sum of the differences of all opposite permutations.
\begin{equation}
    \prod^{l}_{m=1} r_{i_m} \sigma_{i_{l+1}q}
    = \frac{1}{l+1}
    \underbrace{\left[\sum_{n=1}^{l+1} \sigma_{i_{n}q}\prod^{l+1}_{\substack{m=1 \\ m \neq n}} r_{i_m} \right.}_{\text{Symmetric part}}
    +\underbrace{\left.\sum_{n=1}^{l} (r_{i_n}\sigma_{i_{l+1}q} - r_{i_{l+1}}\sigma_{i_{n}q}) \prod^{l}_{\substack{m=1 \\ m \neq n}} r_{i_m} \right]}_{\text{Antisymmetric part}}.
    \label{eq:symanisym}
\end{equation}
The antisymmetric part of this tensor will cancel upon the operation $\partial_{i_n}\partial_{i_l}$ since it is antisymmetric on these indices, therefore only the symmetric part matter \citep{nott2011suspension}.
Besides, if we consider the divergence of this same product we can derive the following relation,
\begin{align*}
    \hat{\partial_p} \left(\prod^{l+1}_{m=1} r_{i_m} \sigma_{pq}\right)
    &= \prod^{l+1}_{m=1} r_{i_m} \hat{\partial_p} \sigma_{pq}
    + \sigma_{pq} \hat{\partial_p} \prod^{l+1}_{m=1} r_{i_m}\\
    &= \prod^{l+1}_{m=1} r_{i_m} \hat{\partial_p} \sigma_{pq}
    + \sigma_{pq}  \sum_{m=1}^l \hat{\partial_p}  r_{i_m}  \prod^{l}_{\substack{n=1 \\ n \neq m}} r_{i_m}\\,
\end{align*}
noticing that the gradient of $\bm{r}$ is the identity matrix, and reversing the equation yields,
\begin{equation}
    \sum_{m=1}^{l+1} \sigma_{i_mp}  \prod^{l+1}_{\substack{n=1 \\ n \neq m}} r_{i_m}
    =\hat{\partial_p} \left(\prod^{l+1}_{m=1} r_{i_m} \sigma_{pq}\right)
    - \prod^{l+1}_{m=1} r_{i_m} \hat{\partial_p} \sigma_{pq}.
    \label{eq:prod}
\end{equation}
Injecting \ref{eq:prod} and \ref{eq:symanisym} into \ref{eq:expsig} gives,
\begin{equation*}
    \partial_{i_{l+1}}
    \phi\left<\sigma\right>_{i_{l+1}q}^d=
    \sum_l^\infty
    \left\{
        \frac{(-1)^{l}}{(l+1)!}
        \prod^{l+1}_{m=1}
        \partial_{i_m}
        \sum_{\alpha}
        g_{\alpha}
        \int_{V_\alpha}
        \left[
            \hat{\partial_p} \left(\prod^{l+1}_{m=1} r_{i_m} \sigma_{pq}\right)
            - \prod^{l+1}_{m=1} r_{i_m} \hat{\partial_p} \sigma_{pq}.
        \right]
        dV
    \right\},
\end{equation*}
by making use of the divergence theorem for the first term in the integral, and using the momentum balance \ref{eq:CmomentumOnefluide} for the second term,  gives,
\begin{align}
    \label{eq:S}
    \partial_{i_{l+1}}
    \phi\left<\sigma\right>_{i_{l+1}q}^d
    &=
    \sum_l^\infty
    \frac{(-1)^{l}}{(l+1)!}
    \prod^{l+1}_{m=1}
    \partial_{i_m}
    \sum_{\alpha}
    g_{\alpha}
    \int_{S_\alpha}
    \prod^{l+1}_{m=1} r_{i_m} n_p \sigma_{pq}
    dS\\
    \label{eq:Dt}
    &+\sum_l^\infty
    \frac{(-1)^{l+1}}{(l+1)!}
    \prod^{l+1}_{m=1}
    \partial_{i_m}
    \sum_{\alpha}
    g_{\alpha}
    \int_{V_\alpha}
    \rho_d
    \prod^{l+1}_{m=1} r_{i_m} 
    \hat{D}u_q
    dV \\  
    % &+\sum_l^\infty
    % \frac{(-1)^{l+1}}{(l+1)!}
    % \prod^{l+1}_{m=1}
    % \partial_{i_m}
    % \partial_t
    % \sum_{\alpha}
    % g_{\alpha}
    %         \int_{V_\alpha}
    %         \rho_d
    %         \prod^{l+1}_{m=1} r_{i_m} u_q
    %     dV \\  
    % \label{eq:dt2}
    % &+\sum_l^\infty
    % \frac{(-1)^{l+1}}{(l+1)!}
    % \prod^{l+1}_{m=1}
    % \partial_{i_m}
    % \sum_{\alpha}
    % g_{\alpha}
    % \sum_{f=1}^{l+1}
    % u^\alpha_{i_f}
    %         \int_{V_\alpha}
    %         \rho_d
    %         \prod^{l+1}_{\substack{m=1\\ m\neq f}} r_{i_m} u_q
    %     dV \\  
    % \label{eq:u_pu_q}
    % &+ \sum_l^\infty
    % \frac{(-1)^{l+1}}{(l+1)!}
    % \prod^{l+1}_{m=1}
    % \partial_{i_m}
    % \sum_{\alpha}
    % g_{\alpha}
    %         \int_{V_\alpha}
    %         \rho_d
    %         \prod^{l+1}_{m=1} r_{i_m}  \hat{\partial}_p (u_pu_q)
    %     dV \\
    \label{eq:b}
    &+
    \sum_l^\infty
    \frac{(-1)^{l}}{(l+1)!}
    \prod^{l+1}_{m=1}
    \partial_{i_m}
    \sum_{\alpha}
    g_{\alpha}
            \int_{V_\alpha}
            \prod^{l+1}_{m=1} r_{i_m} b_q
        dV,
\end{align}
where we have used the notation $\hat{D}$ for the total derivative.
We observe that each terms of order $l$ in this expansion correspond to the terms of order $l+1$ in \ref{eq:S}, \ref{eq:Dt} and \ref{eq:b}.
Consequently, they all cancel out except the zeroth order terms of the latter equations (as already demonstrated by \citet{prosperetti2004average}).
Therefore, keeping only the first order terms in \ref{eq:momentumd} gives the simplified form,
\begin{equation*}
    \rho_d \sum_\alpha g_\alpha \int_{V_\alpha} \hat{D} \bm{u}  dV
    =  \sum_\alpha g_\alpha \int_{V_\alpha} \bm{b} dV
    +\sum_\alpha g_\alpha\int_{S_\alpha}  \bm{n}\cdot\bm{\sigma} dS.
\end{equation*}
At this point if we want to recover the particular average equation, we need to carry out some modifications.
Indeed, the right hands side correspond rigorously to the particular averaged quantities.
However, at the left hands side we recover the particular average of the total derivative while we should obtain the total derivative of the particular averaged quantities.

Another way to express the left hands side ($\textbf{LHS}$) term is to carry out the difference of the expansion while keeping the time derivative and divergence terms separated.  
This term can be derived by carrying out the difference between \ref{eq:Dt}, \ref{eq:dt}.
For conciseness, we will remove the summation sign over $l$ and $\alpha$ and treat them as implicit.   
Besides, remarks that in the following expression we incremented the order of the temporal term (\ref{eq:dt}), thus we added the zeroth order term at the beginning. 
It yields,
\begin{multline*}
    \textbf{LHS}
    = \partial_t \pavg{\int_{V_\alpha} u_q dV}
    + \frac{(-1)^l}{l!} \partialp{1}{l+1}
    \left[
        + \int_{V_\alpha} \pri{1}{l}
        \left(
            u_{i_{l+1}}u_q 
            +\frac{r_{i_{l+1}}}{l+1}  
            u_p \hat{\partial}_p u_q 
        \right) dV
    \right]\\
    + \frac{(-1)^{l+1}}{l!} \partialp{1}{l}
    \left[
        \sum_{e=1}^{l}
        u_{i_e}^\alpha \int_{V_\alpha}  
        \prod_{\substack{m=1 \\ m \neq e}}^{l} r_{i_m} u_q dV
    \right]
\end{multline*}
Considering the symmetry of the last term we can introduce the following simplification,
\begin{multline*}
    \textbf{LHS}
    = \partial_t \pavg{\int_{V_\alpha} u_q dV}
    + \frac{(-1)^l}{l!} \partialp{1}{l+1}
    \left[
        \int_{V_\alpha} \pri{1}{l}
        \left(
            u_{i_{l+1}}u_q 
            +\frac{r_{i_{l+1}}}{l+1}  
            u_p \hat{\partial}_p u_q 
        \right) dV
    \right]\\
    + \frac{(-1)^{l+1}}{(l-1)!} \partialp{1}{l}
    \left[
        u_{i_l}^\alpha \int_{V_\alpha}  
        \prod_{m=1}^{l-1} r_{i_m} u_q dV
    \right]
\end{multline*}
So, under that form we can clearly see that the first term is the transport of the moment of momentum of order $l$.
The additional term are source terms due to the internal motion of the particles.
The idea now, is to express $u_q$ as a Taylor expansion and let the other velocity as an Eulerian field so that we can introduce the different moment of momentum into this expression.
After carrying out the algebra it yields,
% \begin{multline*}
%     \int_{V_\alpha} \pri{1}{l}
%         \left(
%             u_{i_{l+1}}u_q 
%             +\frac{r_{i_{l+1}}}{l+1}  
%             u_p \hat{\partial}_p u_q 
%         \right) dV
%     \\= \sum_{n=0}^\infty  \sum_{l=0}^n \frac{1}{(n-l)!} 
%     \hatpartialpi{l+2}{n+1} u_q|_\alpha
%     \int_{V_\alpha} 
%     \left(
%         \prod_{\substack{m=1 \\ m \neq l+1}}^{n+1} r_{i_m}
%         u_{i_{l+1}} 
%         +\frac{n-l}{l+1}  
%         \prod_{m=1}^{n}
%         r_{i_m} u_{i_{n+1}}
%     \right) dV
% \end{multline*}
\begin{multline*}
    \textbf{LHS}
    = \partial_t \pavg{\int_{V_\alpha} u_q dV}\\
    + 
    \sum_{l=0}^n
    \frac{(-1)^l}{l!(n-l)!} \partialp{1}{l+1}
    \left[
    \hatpartialpi{l+2}{n+1} u_q|_\alpha
    \int_{V_\alpha} 
    \left(
        \prod_{\substack{m=1 \\ m \neq l+1}}^{n+1} r_{i_m}
        u_{i_{l+1}} 
        +\frac{n-l}{l+1}  
        \prod_{m=1}^{n}
        r_{i_m} u_{i_{n+1}}
    \right) dV
    \right]\\
    + \frac{(-1)^{n+1}}{(n-1)!} \partialp{1}{n}
    \left[
        u_{i_n}^\alpha \int_{V_\alpha}  
        \prod_{m=1}^{n-1} r_{i_m} u_q dV
    \right]
\end{multline*}
After averaging and expressing the integrals terms with moment of momentum expression, this equation reads,
\begin{multline*}
    \textbf{LHS}
    = \partial_t \pavg{\int_{V_\alpha} u_q dV}
    + \frac{(-1)^{n+1}}{(n-1)!} \partialp{1}{n}
    \left[
        u_{i_n}^\alpha   
        \mathcal{P}_{i_1 \ldots i_q}
    \right]\\
    + 
    \sum_{l=0}^n
    \frac{(-1)^l}{l!(n-l)!} \partialp{1}{l+1}
    \left[
    \hatpartialpi{l+2}{n+1} u_q|_\alpha
    \left(
        \mathcal{P}_{i_1 \ldots i_n i_{l+1}}
        +\frac{n-l}{l+1}  
        \mathcal{P}_{i_1 \ldots i_n i_{n+1}}
    \right) 
    \right]
\end{multline*}
note that the last index of $\mathcal{P}$ correspond to the index of the velocity vector.
The moment of momentum tensor $\mathcal{P}$ can be decomposed into a symmetric part $\mathcal{S}$ and an antisymmetric part $\mathcal{A}$.
The former represents at the second order the \textit{stretching momentum} of the particle, the latter the rate of rotation.
Considering the index of the derivative operator and the symmetry property of the tensor, the previous equation yields, 
\begin{multline*}
    \textbf{LHS}
    = \partial_t \pavg{\int_{V_\alpha} u_q dV}
    + \frac{(-1)^{n+1}}{(n-1)!} \partialp{1}{n}
    \left[
        u_{i_n}^\alpha   
        \left(
            \mathcal{S}_{i_1 \ldots i_q}
            +\mathcal{A}_{i_1 \ldots i_q}
        \right)
    \right]\\
    + 
    \sum_{l=0}^n
    \frac{(-1)^l}{l!(n-l)!} \partialp{1}{l+1}
    \left[
    \hatpartialpi{l+2}{n+1} u_q|_\alpha
    \left(
        \mathcal{S}_{i_1 \ldots i_n i_{l+1}}
        +\frac{n-l}{l+1}  
        \mathcal{S}_{i_1 \ldots i_n i_{n+1}}
    \right) 
    \right]
\end{multline*}

Likewise, instead of expressing the expansion with moment of momentum we could introduce the expansion of the second velocity vector.
In this case, we obtain only derivative of the velocity of arbitrary order.
Below, is the global expression, were we have used the index $k$ for the expansion of the second velocity component.
After manipulating the indices we get,
\begin{multline*}
    \sum_\alpha g_\alpha \int_{V_\alpha} \hat{\partial}_p(u_pu_q)  dV
    =
    \sum_{k=0}^\infty
    \sum_{n=0}^k
    \sum_{l=0}^n
    \frac{(-1)^l}{l!(n-l)!(k-n)!}
    \prod^{l+1}_{m=1}
    \partial_{i_m}
    \sum_{\alpha}
    g_{\alpha}\\
    \hatpartialpi{l+2}{n+1}u_q|_\alpha
    \left[
    \hatpartialpi{n+2}{k+1}  
    u_{i_{l+1}}|_\alpha
    \int_{V_\alpha}
        \prod_{\substack{m=1\\ m\neq l+1}}^{k+1} r_{i_m}
    dV
    -
    \sum_{e=l+2}^{n+1}
    \hatpartialpi{n+2}{k+1}
    u_{i_e}|_\alpha
    \int_{V_\alpha}
        \prod_{\substack{m=1\\ m\neq e}}^{k+1} r_{i_m}
    dV
    \right]
\end{multline*}
Also, we need to expand the first term of \ref{eq:first_order},
\begin{equation*}
    \partial_t (n\pavg{p^\alpha_i})
    = \partial_t \sum_\alpha g_\alpha 
    \sum_{k=0}^\infty \hatpartialpi{1}{k} u_i|_\alpha \int_{V_\alpha} \pri{1}{k} dV, 
\end{equation*}
then we obtain the complete expansion of the momentum equation of the dispersed phase.  


\section{Transport of the moment of inertia and momentum.}

As demonstrated by \ref{eq:first_order} and \ref{eq:mass_order_l}, the particular averaged moment of inertia and momentum both appear in the expansion of the dispersed averaged equation of mass and momentum.
While, we bought two unknown tensor at each order of the expansion, i.e. $\mathcal{G}$ and  $\mathcal{P}$ (the moments of momentum can be replaced by derivative of the velocity at $\bm{y}_\alpha$), the equation stays a scalar for the mass conservation and a vector for the momentum conservation.
Thus, to complete the system of equation we need the particular averaged transport equation introduced in \ref{ap:cinematic}.
Namely,
\begin{equation*}
    \partial_t\left(n\left<\mathcal{G}_{i_1i_2\ldots i_l}^\alpha\right>^p\right)
    +\partial_k\left(n\left<u_k\mathcal{G}_{i_1i_2\ldots i_l}^\alpha\right>^p\right)
    = n\;l\left<(\mathcal{P}^\alpha_{i_1i_2\ldots i_l})^\text{Sym}\right>
    +n\left<T^\alpha_{i_1i_2\ldots i_l}\right>,
\end{equation*}
for the moments of inertia of order $l$ and,
\begin{multline*}
    \partial_t\left(n\left<\mathcal{P}_{i_1i_2\ldots i_l}^\alpha\right>^p\right)
    +\partial_k\left(n\left<u_k\mathcal{P}_{i_1i_2\ldots i_l}^\alpha\right>^p\right)
    =
    n\pavg{M^\alpha_{i_1\ldots i_l}}
    +\rho_d n \pavg{ \int_{V_\alpha} \sum_{e=1}^{l-1}  \prod^{l-1}_{\substack{ m=1 \\   m \neq e}} r_{i_m} w_{i_e} u_{i_l}dV}\\
    -\rho_d n \pavg{\int_{V_\alpha} \sum^{l-1}_{e=1} \prod^{l-1}_{\substack{m=1\\ m\neq e}}r_{i_m}  \sigma_{i_ei_l}^d dV}
    + n \pavg{T^\alpha_{i_1i_2\ldots i_l}}.
\end{multline*}
To synthesize, in any dispersed two phases flow were no assumption is possible on the dispersed phase we have as unknown (excluding the closure terms), the particular averaged $l^th$ first moment of inertia, $l^th$ first moment of momentum, the number density of particles  $n$ and the averaged velocity of particles $\bm{u}_\alpha$. 
Which makes us $1+3+2\sum_{n=0}^l n^2 = 4+\frac{l(l+1)(2l+1)}{3}$ unknown. 
The above system of equation, provides $2\sum_{n=0}^l n^2 = \frac{l(l+1)(2l+1)}{3}$ equations. 
Then the $4$ remaining equations must be the \textit{linking equations}, i.e. the balance of the momentum \ref{eq:first_order} and the mass expanded mass balance \ref{eq:mass_order_l}.
Although one can note that the gradient of the velocity also appear into the latter equation 


This set of equation, is only needed completely if we ignore everything about the dispersed phase.
In practical cases we usually know if the dispersed phase is solid or the kind of inner flow of t isn't.  
Therefore, in the next section we investigate the form of the equations assuming specific cases. 
\section{Application to specific cases.}

In the first place we derive the $^n$ order system of equation for spherical particles. 
It has been done possible by knowing by advance the form of the inertial tensor at arbitrary order. 
Then we investigate the first term of the expansion when considering axisymmetric particles. 
And finally we consider droplet by describing the inner flow as hill vortex. 

\subsection{The case of solid particles}
The velocity inside a solid particle can express as, $\bm{u} = \bm{u}^\alpha + \bm{\Omega}_\alpha\cdot\bm{r} = \bm{u}^\alpha + \bm{\omega}_\alpha\times\bm{r}$, or in indices notation $u_p = u^\alpha_p + \Omega_{pj}^\alpha r_j = \epsilon_{pjk} \omega_k^\alpha r_j$.
Then, we seek the expression of, 
\begin{align*}
    u_pu_q &= (u^\alpha_p + \Omega_{pj}^\alpha r_j)(u^\alpha_q + \Omega_{qe}^\alpha r_e)\\
           &= u^\alpha_pu^\alpha_q
           + u^\alpha_p\Omega_{qe}^\alpha r_e
           + u^\alpha_q\Omega_{pj}^\alpha r_j
           +\Omega_{pj}^\alpha \Omega_{qe}^\alpha r_e r_j\\
           &= u^\alpha_p u^\alpha_q
           + u^\alpha_p \epsilon_{qea}\omega^\alpha_a r_e
           + u^\alpha_q \epsilon_{pia} \omega^\alpha_a r_e
           -\epsilon_{pja} \omega^\alpha_a \epsilon_{qeb}\omega^\alpha_b   r_e r_j,
\end{align*}
Keep in mind that $\bm{u}_\alpha$ and $\bm{\Omega}_\alpha$ are solely function of time, since it is a purely Lagrangian quantity.
The term in the integral of the second term of \ref{eq:sigmaexp} is the advection of the velocity $u_q$ along $u_p$.
Therefore, $u_p = u_p + \Omega_{pj}^\alpha r_j$ is the velocity fields, and $u_q = u^\alpha_q + \Omega_{qe}^\alpha r_e$ the Lagrangian velocity presented above.
\begin{align*}
    u_p\hat{\partial}_p(u_q)
    &= (u_p^\alpha + \Omega_{pj}^\alpha r_j) \hat{\partial}_p(u^\alpha_q + \Omega_{qe}^\alpha r_e)\\
    &= u_p^\alpha \Omega^\alpha_{qp} 
    + \Omega_{pj}^\alpha  \Omega_{qp}^\alpha r_j \\
    &= u_p^\alpha \Omega^\alpha_{qp} \epsilon_{pqb} \omega^\alpha_b 
    - \epsilon_{pja}  \epsilon_{pqb} \omega^\alpha_a\omega^\alpha_b r_j,
\end{align*}
Therefore, it yields,
\begin{align}
    &\partial_{i_{l+1}} \phi \left< u_{i_{l+1}} u_q\right>^d
    % = \sum_{l=0}^\infty \sum_\alpha \left[\frac{(-1)^l}{l!} \bm{\nabla}^l g|_\alpha \int (\r_\alpha)^l\bm{u} dV\right],
    =\\
    & \sum_{l=0}^\infty
        \frac{(-1)^l}{l!}
        \prod^{l+1}_{m=1} \partial_{i_m}
        \sum_\alpha  g_\alpha
        \int_{V_\alpha}
        \prod^l_{m=1}r_{i_m}
        \left(
            u^\alpha_{i_{l+1}} u^\alpha_q
            + u^\alpha_{i_{l+1}}\Omega_{qe}^\alpha r_e
            + u^\alpha_q\Omega_{i_{l+1}j}^\alpha r_j
            +\Omega_{qe}^\alpha \Omega_{i_{l+1}j}^\alpha  r_e r_j
        \right) dV
    \\
    =& \sum_{l=0}^\infty
        \frac{(-1)^l}{l!}
        \prod^{l+1}_{m=1} \partial_{i_m}
        \sum_\alpha  g_\alpha
        \left(
            \mathcal{G}_{i_1\ldots i_l}^l u^\alpha_{i_{l+1}} u^\alpha_q
            +\mathcal{G}_{i_1\ldots i_l e}^{l+1} u^\alpha_{i_{l+1}}\Omega_{qe}^\alpha
            +\mathcal{G}_{i_1\ldots i_l j}^{l+1} u^\alpha_q\Omega_{i_{l+1}j}^\alpha
            +\mathcal{G}_{i_1\ldots i_l ej}^{l+2}\Omega_{qe}^\alpha \Omega_{i_{l+1}j}^\alpha
        \right),
    \label{eq:C17}
\end{align}
and the third and fourth term of \ref{eq:sigmaexp} reads,
\begin{align}
    &-\sum_l^\infty
    \frac{(-1)^{l+1}}{(l+1)!}
    \prod^{l+1}_{m=1}
    \partial_{i_m}
    \sum_{\alpha}
    g_{\alpha}
    \sum_{f=1}^{l+1}
    u^\alpha_{i_f}
        \int_{V_\alpha}
        \rho_d
        \prod^{l+1}_{\substack{m=1\\ m\neq f}} r_{i_m} (u_q^\alpha + \Omega_{qe} r_e)
    dV \\  
    &+\sum_l^\infty
    \frac{(-1)^{l+1}}{(l+1)!}
    \prod^{l+1}_{m=1}
    \partial_{i_m}
    \sum_{\alpha}
    g_{\alpha}
            \int_{V_\alpha}
            \rho_d
            \prod^{l+1}_{m=1} r_{i_m}  \left(
                u_p^\alpha \Omega^\alpha_{qp} 
                +\Omega_{qp}^\alpha \Omega_{pj}^\alpha   r_j
            \right)
        dV \\
    &=
    \sum_l^\infty
    \frac{(-1)^{l+1}}{(l+1)!}
    \prod^{l+1}_{m=1}
    \partial_{i_m}
    \sum_{\alpha}
    g_{\alpha}\left(
        \mathcal{G}_{i_1\ldots i_{l+1}}^{l+1}u_p^\alpha \Omega^\alpha_{qp}     
        +\mathcal{G}_{i_1\ldots i_{l+1}j}^{l+2}\Omega_{qp}^\alpha \Omega_{pj}^\alpha
    \right)\\
    &-\sum_l^\infty
    \frac{(-1)^{l+1}}{(l+1)!}
    \prod^{l+1}_{m=1}
    \partial_{i_m}
    \sum_{\alpha}
    g_{\alpha}
    \sum_{f=1}^{l+1}
    \left(
        u^\alpha_{i_f}
        u_q^\alpha 
        \mathcal{G}_{i_1\ldots i_{l+1}}^l
        +u_{i_f}^\alpha \Omega_{qe} 
        \mathcal{G}_{i_1\ldots i_{l+1}e}^{l+1}
    \right).\\
    \label{eq:C19}
\end{align}
As one can note, the terms of the two above equations do not directly cancel with the terms of the latter one.  
Considering the symmetry of the terms 
Indeed, if we gather all the terms together taking care of the sign of each term we find, 
\begin{multline*}
    \sum_{l=0}^\infty
        \frac{(-1)^l}{l!}
        \prod^{l+1}_{m=1} \partial_{i_m}
        \sum_\alpha  g_\alpha
        \left[
            \mathcal{G}_{i_1\ldots i_l}^l u^\alpha_{i_{l+1}} u^\alpha_q
            +\mathcal{G}_{i_1\ldots i_l e}^{l+1} u^\alpha_{i_{l+1}}\Omega_{qe}^\alpha
            +\mathcal{G}_{i_1\ldots i_l j}^{l+1} u^\alpha_q\Omega_{i_{l+1}j}^\alpha
            +\mathcal{G}_{i_1\ldots i_l ej}^{l+2}\Omega_{qe}^\alpha \Omega_{i_{l+1}j}^\alpha
            \phantom{\sum^{l+1}_l}
        \right.\\
            +\frac{1}{l+1}
            \left(
                \mathcal{G}_{i_1\ldots i_{l+1}}^{l+1}u_p^\alpha \Omega^\alpha_{qp}     
                +\mathcal{G}_{i_1\ldots i_{l+1}j}^{l+2}\Omega_{qp}^\alpha \Omega_{pj}^\alpha
            \right)
        \\
        \left.
            +\frac{1}{l+1}
            \sum_{f=1}^{l+1}
            \left(
                u^\alpha_{i_f}
                u_q^\alpha 
                \mathcal{G}_{i_1\ldots i_{l+1}}^l
                +u_{i_f}^\alpha \Omega_{qe} 
                \mathcal{G}_{i_1\ldots i_{l+1}e}^{l+1}
            \right)
        \right].
\end{multline*}
The two term on the third line end up being symmetrical on all indices, from $i_1$ to $i_{l+1}$, moreover, we apply the operator $\prod^{l+1}_{m=1} \partial_{i_m}$ on those same terms.
Considering those facts the terms on the third line are the same than the two first terms on the first line. 
At this point the equation reduce to :
\begin{multline*}
    \sum_{l=0}^\infty
        \frac{(-1)^l}{l!}
        \prod^{l+1}_{m=1} \partial_{i_m}
        \sum_\alpha  g_\alpha
        \left[
            2\mathcal{G}_{i_1\ldots i_l}^l u^\alpha_{i_{l+1}} u^\alpha_q
            +2\mathcal{G}_{i_1\ldots i_l e}^{l+1} u^\alpha_{i_{l+1}}\Omega_{qe}^\alpha
            +\mathcal{G}_{i_1\ldots i_l j}^{l+1} u^\alpha_q\Omega_{i_{l+1}j}^\alpha
            +\mathcal{G}_{i_1\ldots i_l ej}^{l+2}\Omega_{qe}^\alpha \Omega_{i_{l+1}j}^\alpha
        \right.\\
        \left.
            +\frac{1}{l+1}
            \left(
                \mathcal{G}_{i_1\ldots i_{l+1}}^{l+1}u_p^\alpha \Omega^\alpha_{qp}     
                +\mathcal{G}_{i_1\ldots i_{l+1}j}^{l+2}\Omega_{qp}^\alpha \Omega_{pj}^\alpha
            \right)
        \right].
\end{multline*}
No more simplification can be made. 

\subsection{Spherical particles}
Up to now we didn't consider the shape of the particles, we just considered arbitrary solid particles. 
Now, if we consider spherical particles the we can express the shape tensor as such,
\begin{equation}
    \mathcal{G}_{i_1\ldots i_{l+1}j}^{l+2}
    =\rho_d \frac{R^{l+3}}{l+3}\sum_{k=1}^{l+1} \delta_{j i_k}
    \int_{V_\alpha}  \prod^{l+1}_{\substack{m=1\\ m\neq k}} r_{i_m} dV
    = \sum_{k=1}^{l+1} \delta_{j i_k} \mathcal{G}^l,
\end{equation}
where $\mathcal{G}^l$ is an $l^{th}$ order tensor containing all indices except the ones already contained in the Kronecker symbol (see \ref{ap:cinematic}).
For the other shape tensor appearing in \ref{eq:C17} and \ref{eq:C19} we can derive similar relations with different indices.
Then, if we make use of this expression, we get for each terms of order $l$ in \ref{eq:C17} the following expression,
\begin{equation}
    \mathcal{G}_{i_1\ldots i_l j}^{l+1} u^\alpha_q\Omega_{ji_{l+1}}^\alpha
    = \sum_{k=1}^{l}
    u^\alpha_q\Omega_{i_{l+1}i_k}^\alpha \mathcal{G}^{l-1}_{i_1\ldots i_l},
    \label{eq:C23}
\end{equation}
\begin{equation}
    \mathcal{G}_{i_1\ldots i_l ej}^{l+2}\Omega_{qe}^\alpha \Omega_{i_{l+1}j}^\alpha =
    -\Omega_{qe}^\alpha \Omega_{ei_{l+1}}^\alpha \mathcal{G}^l_{i_1 \ldots i_l}
    +\sum_{k=1}^{l}
    \Omega_{qe}^\alpha \Omega_{i_{l+1}i_k}^\alpha \mathcal{G}^l_{i_1\ldots i_le}
    \label{eq:C24}
\end{equation}
and for \ref{eq:C19},
\begin{equation}
    \frac{1}{l+1}\mathcal{G}_{i_1\ldots i_l i_{l+1}j}^{l+2}\Omega_{qp}^\alpha \Omega_{pj}^\alpha
    =
    \frac{1}{l+1}
    \sum_{k=1}^{l+1}
    \Omega_{qp}^\alpha \Omega_{p i_k}^\alpha \mathcal{G}^l
    =
    \Omega_{qp}^\alpha \Omega_{p i_{l+1}}^\alpha \mathcal{G}^l_{i_1 \ldots i_l},
    \label{eq:C21}
\end{equation}
for the second equality, we have used the fact that the tensor inside the sum is symmetric over all $i_k$ indices, besides, by applying the operator $\prod_{m=1}^{l+1}\partial_{i_e}$ which is also symmetric, all the terms in the sum are the exact same.  
Then, note that $i_k \in \left\{ i_1, \dots, i_{l+1}\right\}$ for \ref{eq:C21},  and $i_k \in \left\{ i_1, \dots, i_{l}\right\}$ for the other terms.
Therefore, the antisymmetric tensor $\Omega_{i_ki_{l+1}}^\alpha$ vanish under the operator $\prod^{l+1}_{m=1} \partial_{i_m}$ since it contain partial derivative of both indices.
The first term of equation \ref{eq:C24} is the scalar product of the rotation tensors.
In equation \ref{eq:C21} we recover the same term, thus the two series cancel out.
Therefore, for solid spherical particles the equation \ref{eq:first_order} reduce to,
\begin{equation*}
    \rho_d \frac{\partial}{\partial t} \left(n\left<V^\alpha u_q\right>^p\right)
    +\sum_{l=0}^\infty \left[
        \left(n
        \left<
            \mathcal{G}_{i_1\ldots i_l e}^{l+1} u^\alpha_{i_{l+1}}\Omega_{qp}^\alpha
        \right>^p\right)
    \right]
    = n \left<V^\alpha b^{ext}_q\right>^p
    + n\left<f_q^\alpha\right>^p,
\end{equation*}
If we make use of the general expression of the moment of inertia tensor (see \ref{eq:shapeT}), we can write the following infinite sum expression for the first and second terms inside the average, 
\begin{equation*}
    \mathcal{G}_{i_1\ldots i_l}^l u^\alpha_{i_{l+1}} u^\alpha_q
    = A^l V_\alpha
    \delta_{i_l i_a}
    \delta_{i_{l-1} i_b}
    \delta_{i_{l-2} i_c}
    \ldots
    \delta_{i_2i_1}
    u^\alpha_{i_{l+1}} u^\alpha_q,
\end{equation*}
\begin{equation*}
    \mathcal{G}_{i_1\ldots i_l e}^{l+1} u^\alpha_{i_{l+1}}\Omega_{qe}^\alpha
    = A^{l+1} V_\alpha
    \delta_{i_l i_a}
    \delta_{i_{l-1} i_b}
    \delta_{i_{l-2} i_c}
    \ldots
    \delta_{i_1e}
    u^\alpha_{i_{l+1}}\Omega_{qe}^\alpha.
\end{equation*}
First, note that for any even order $l$ the second term cancel, and for any odd order the first term cancel (see \ref{ap:cinematic}). 
Since, any $\partial_{i_e} \partial_{i_m} \delta_{i_ei_m} = \partial_{i_e}\partial_{i_e}$ which is a scalar quantity (i.e. the Laplacian sign), the expression simplify to,
\begin{multline*}
    \rho_d \frac{\partial}{\partial t} \left(n\left<V^\alpha u_q^\alpha\right>^p\right)
    +\rho_d\partial_{i_{l+1}} \sum_{l=0}^\infty 
    \left[
        \frac{A^l(-1)^l}{l!}
        (\partial_i\partial_i)^{l/2}
        \left(n 
        \left<V_\alpha u^\alpha_{i_{l+1}} u^\alpha_q \right>^p
        \right)
     \right.\\
     \left.
        +
        \frac{A^{l+1}(-1)^l}{l!}
        (\partial_i\partial_i)^{(l-1)/2}\partial_e 
        \left(n 
        \left<V_\alpha u^\alpha_{i_{l+1}}\Omega_{qe}^\alpha \right>^p
        \right)
    \right]
    = n \left<V^\alpha b^{ext}_q\right>^p
    + n\left<f_q^\alpha\right>^p
\end{multline*}
The $A^l$ and $A^{l+1}$ are coefficient solely function of the shape of the particle and not of the volume, therefore since it is all spherical particles they can be removed. 

\subsection{Case of axisymmetric particles}

For axisymmetric particles the only change is the shape tensor.
As developed in the previous appendix at second order $\mathcal{G}^2_{ij} = p_ip_j G_{||} + (\delta_{ij} - p_ip_j) G_{\bot}$.
Therefore, at the zeroth order the kinematics   part of the expansions read as,
\begin{align}
    \partial_{i} \phi \left< u_{i} u_q\right>^d
    % = \sum_{l=0}^\infty \sum_\alpha \left[\frac{(-1)^l}{l!} \bm{\nabla}^l g|_\alpha \int (\r_\alpha)^l\bm{u} dV\right],
    &=
    \partial_{i}
    \sum_\alpha  g_\alpha
    \left(
        V_\alpha u^\alpha_{i} u^\alpha_q
        -\mathcal{G}_{ej}^{2}\Omega_{qe}^\alpha \Omega_{ji}^\alpha
    \right)
    = \ldots
    (p_ep_j G_{||} + (\delta_{ej} - p_ep_j)G_{\bot})\Omega_{qe}^\alpha \Omega_{ji}^\alpha)
\end{align}
and the kinematics   term inside the divergence of the stress reads as,
\begin{align}
    \partial_i \left<\sigma_{iq}\right>^d =
    -\partial_{i}
    \sum_{\alpha}
    g_{\alpha}
    (\mathcal{G}_{ij}^{2}\Omega_{qp}^\alpha \Omega_{pj}^\alpha )
    =\ldots
    (p_ip_j G_{||} + (\delta_{ij} - p_ip_j) G_{\bot})\Omega_{qp}^\alpha \Omega_{pj}^\alpha)
\end{align}
If we subtract both equations and keep only rotation terms for conciseness we get,
\begin{align}
    \Delta \Omega
    &= - (p_ip_j G_{||} + (\delta_{ij} - p_ip_j) G_{\bot})\Omega_{qp}^\alpha \Omega_{pj}^\alpha
    + (p_ep_j G_{||} + (\delta_{ej} - p_ep_j)G_{\bot})\Omega_{qe}^\alpha \Omega_{ji}^\alpha\\
    &=- (p_ip_j G_{||} - p_ip_j  G_{\bot})\Omega_{qp}^\alpha \Omega_{pj}^\alpha
    + (p_ep_j G_{||} - p_ep_j G_{\bot})\Omega_{qe}^\alpha \Omega_{ji}^\alpha\\
    &= (G_{||} - G_{\bot})
    (p_ep_j\Omega_{qe}^\alpha \Omega_{ji}^\alpha
    - p_ip_j \Omega_{qp}^\alpha \Omega_{pj}^\alpha)\\
    &= (G_{||} - G_{\bot}) p_j \Omega_{qe}^\alpha
    (p_e \Omega_{ji}^\alpha
    - p_i \Omega_{ej}^\alpha)\\
    % &= (G_{||} - G_{\bot})
    % (p_ep_j\epsilon_{qea} \omega_a^\alpha \epsilon_{jib}\omega_b^\alpha
    % - p_ip_j \epsilon_{qpa}\epsilon_{pjb}\omega_a^\alpha \omega_b^\alpha) \\
    % &= (G_{||} - G_{\bot})
    % (p_ep_j\epsilon_{qea}  \epsilon_{jib} \omega_b^\alpha \omega_a^\alpha
    % + p_ip_j (\delta_{qj}\delta_{ab} - \delta_{qb}\delta_{aj})\omega_a^\alpha \omega_b^\alpha)\\
    % &= (G_{||} - G_{\bot})
    % (p_ep_j\epsilon_{qea}  \epsilon_{jib} \omega_b^\alpha \omega_a^\alpha
    % + p_ip_q\omega_a^\alpha \omega_a^\alpha
    % -p_i p_a \omega_a^\alpha \omega_q^\alpha)\\
\end{align}
Now, if we derive the form of the first order terms it reads,
\begin{align}
    \partial_{i_2} \phi \left< u_{i_2} u_q\right>^d
    &=
    -
    \partial_{i_1}\partial_{i_2}
        \sum_\alpha  g_\alpha
        \left(
            \mathcal{G}_{i_1 e}^{2} u^\alpha_{i_2}\Omega_{qe}^\alpha
            +\mathcal{G}_{i_1 j}^{2} u^\alpha_q\Omega_{i_{2}j}^\alpha
        \right)\\
    &=
    -
    \partial_{i_1}\partial_{i_2}
        \sum_\alpha  g_\alpha
        \left(
            (p_{i_1}p_e G_{||} + (\delta_{{i_1}e} - p_{i_1}p_e) G_{\bot}) u^\alpha_{i_2}\Omega_{qe}^\alpha
            +(p_{i_1}p_j G_{||} + (\delta_{{i_1}j} - p_{i_1}p_j) G_{\bot}) u^\alpha_q\Omega_{i_{2}j}^\alpha
        \right)
\end{align}

Then to solve the averaged equations at zeroth order accuracy, one has to solve the following system of equation,
The mass and shape conservation :
\begin{equation}
    \frac{\partial }{\partial t}(n\left<V_\alpha\right>^p)
    + \bm{\nabla}\cdot(n\left<\bm{u_\alpha}V_\alpha\right>^p)
    = 0,
\end{equation}
\begin{equation}
    \frac{\partial }{\partial t}\left(
        n\left<(G_{||}^\alpha-G_{\bot}^\alpha)\bm{p}\bm{p}+G_\bot^\alpha\bm{I}\right>^p
    \right)
    +\bm{\nabla}\cdot\left(
    n\left<(G_{||}^\alpha-G_{\bot}^\alpha)\bm{u}_\alpha\bm{p}\bm{p}+G_\bot^\alpha\bm{u}_\alpha\bm{I}\right>^p
    \right)
    = 0.
\end{equation}
The transport equation of the momentum equation,
\begin{equation}
    \rho_d\left[
        \frac{\partial }{\partial t}(n\left<V_\alpha\bm{u}_\alpha\right>^p)
        + \bm{\nabla}\cdot(n\left<V_\alpha\bm{u_\alpha}\bm{u}_\alpha\right>^p)
    \right]
    = n \left<V_\alpha\bm{b_{ext}}\right>^p
    + n\left<\bm{f_\alpha}\right>^p,
\end{equation}
The moment of momentum transport equation,
\begin{equation}
    \rho_d\frac{\partial }{\partial t}\left(n\left<(G_{||}^\alpha-G_{\bot}^\alpha)\bm{\Omega}_{\alpha} \cdot\bm{p}\bm{p}+G_\bot^\alpha\bm{\Omega}_{\alpha}\right>^p\right)
    + \rho_d\bm{\nabla}\cdot\left(n\left<(G_{||}^\alpha-G_{\bot}^\alpha)\bm{u_\alpha}\bm{\Omega}_{\alpha} \cdot\bm{p}\bm{p}+G_\bot^\alpha\bm{u_\alpha}\bm{\Omega}_{\alpha}\right>^p\right)
    = \left<\bm{T}_\alpha^{h}\right>^p
    + \left< \bm{T}_\alpha^{b}\right>^p
\end{equation}
Up to now we have 1 equation for the volume conservation, 9 for the shape conservation, 3 and 9 for the momentum and angular momentum equations, which gives us a total of 22 equations.
Regarding the unknown we have the 3 components of the velocity, the averaged volume, the 9 component of the shape tensor, the number density, and the 9 component of the rotation tensor, which makes a total of 23 unknown.
Therefore, to solve the complete system we need to add the scalar equation for the number density, which at the second order reads,
\begin{multline}
    \frac{\partial }{\partial t}\left[n\left<V_\alpha\right>^p
    +\frac{1}{2}\bm{\nabla}^2  n \left<(G_{||}^\alpha-G_{\bot}^\alpha)\bm{p}\bm{p}+G_\bot^\alpha\bm{I}\right>^p
    \right]
    + \bm{\nabla}\cdot\left[
        \frac{1}{2} \bm{\nabla}^{2} : \left(n \left<(G_{||}^\alpha-G_{\bot}^\alpha)\bm{p}\bm{p}\bm{u}_\alpha +G_\bot^\alpha\bm{I} \bm{u}_\alpha \right>^p\right)
    \right.\\\left.
        - \bm{\nabla} \cdot \left(n \left<(G_{||}^\alpha-G_{\bot}^\alpha)\bm{p}\bm{p}\cdot \bm{\Omega}_\alpha+G_\bot^\alpha\bm{\Omega}_\alpha \right>^p\right)
        +n\left<\bm{u_\alpha}V_\alpha\right>^p
    \right]
    =0,
\end{multline}
If we add the momentum conservation equation of the continuous phases at the zeroth order, namely,
\begin{equation}
    \rho_d
    \frac{\partial }{\partial t}(n\left<V_\alpha\bm{u}_\alpha\right>^p)
    + \rho_d\bm{\nabla}\cdot\left[n\left<V_\alpha\bm{u}_\alpha\bm{u}_\alpha\right>^p
    +\left<
    (G_{||}^\alpha - G_{\bot}^\alpha)
    \bm{\Omega}^\alpha\cdot
    (\bm{p}\bm{p}\cdot\bm{\Omega}^\alpha
    - \bm{\Omega}^\alpha\cdot \bm{p}  \bm{p})
    \right>^p
    \right]
    = n \left<V_\alpha\bm{b_{ext}}\right>^p
    + n\left<\bm{f_\alpha}\right>^p,
\end{equation}
which lead to the following equation (considering the identity \ref{eq:comutativity}),
\begin{equation}
    \bm{\nabla}\cdot\left<
    (G_{||}^\alpha - G_{\bot}^\alpha)
    \bm{\Omega}^\alpha\cdot
    (\bm{p}\bm{p}\cdot\bm{\Omega}^\alpha
    - \bm{\Omega}^\alpha\cdot \bm{p}  \bm{p})
    \right>^p
    = 0,
\end{equation}
where we have omitted the higher order terms
}
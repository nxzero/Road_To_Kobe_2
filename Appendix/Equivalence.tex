\chapter{Equivalence between volume average and weighted average for Lagrangian poly-disperse particles}
\label{ap:equivalence}
As mentioned in \ref{chap:avg} the velocity fields $\left<\bm{u}\right>^L_\gamma$ is needed in \ref{eq:PBM_QBMM}.
While, with the particular-average, \ref{eq:classic_hybrid_momentum_c}, we solve for $\left<\bm{u}\right>^p$.
Also, in \ref{eq:classic_hybrid_momentum_c} we have assumed an even distribution of particles size where this is clearly not the case. 
In the following we show the equivalence between the two averaged quantities and the correctness of the uniformity assumption. 
This appendix aim to prove rigorously the link between the population balance model and the particular-average equation.
The strategy is to derive \ref{eq:classic_hybrid_momentum_p} from Liouville equation following the method of \citet{curtiss1956kinetic} and \citet[chapter~7]{rao2008introduction}.
This way we show an equivalence by identification between the quantities mentioned above.

Let's consider the Boltzmann equation for a distribution $P(\textbf{x},\mathscr{C})$.
The distribution must remain physical, thus it must respect $\lim_{\mathscr{C} \rightarrow \partial\mathscr{C}} = 0$, where $\partial \mathscr{C}$ correspond to the boundary of the domain of definition of each component of $\mathscr{C}$. 
It yields, 
\begin{equation}
    \label{eq:Liouville}
    \pddt P(\textbf{x},\mathscr{C})
    + \bm{\nabla} \cdot \left(\textbf{u} P(\textbf{x},\mathscr{C})\right)
    + \grad_\mathscr{C} \cdot \left(\frac{d\mathscr{C}}{dt} P(\textbf{x},\mathscr{C})\right) 
    = \Psi
\end{equation}
Now let's consider a function $f(\mathscr{C},t)$ which represent any physical quantity function of the internal coordinate and the time. 
Multiplying \ref{eq:Liouville} by the function, $q_\alpha$, and integrating over all the internal coordinates yields Maxwell equation,
\begin{equation}
    \int q_\alpha \pddt P d\mathscr{C}
    + \int q_\alpha \grad \cdot \left(\textbf{u} P\right)d\mathscr{C}
    + \int q_\alpha \grad_\mathscr{C} \cdot \left(\frac{d\mathscr{C}}{dt} P\right) d\mathscr{C} = \int q_\alpha \Psi d\mathscr{C}.
\end{equation}
The first two terms derivative can be swapped with the integral since they are not function of $\mathscr{C}$.
Then, we make use of the divergence theorem on the third term, 
yielding,
\begin{multline}
    \pddt \int q_\alpha  P d\mathscr{C}
    + \grad \cdot\int q_\alpha \left(\textbf{u} P\right)d\mathscr{C} 
    % + \int_{\partial \lambda} \grad_\mathscr{C} \cdot \left(f \textbf{u}_\lambda P\right) d\partial\mathscr{C} 
    = \int q_\alpha \Psi d\mathscr{C},
    +\int \left(\frac{d\mathscr{C}}{dt} P\right) \cdot \grad_\mathscr{C} f  d\mathscr{C} 
    \label{eq:inttt}
\end{multline}
The second integral on RHS has been derived using the divergence theorem and considering that the distributions $P$ vanish near its boundaries. 
After applying the average operator \ref{eq:inttt} reads as,
\begin{equation}
    \pddt \left( n \avg{q_\alpha}\right)
    + \bm{\nabla} \cdot \left(n \avg{\textbf{u} q_\alpha}\right)
    = n \avg{\Psi q_\alpha}
    + n  \avg{\frac{d\mathscr{C}}{dt} \cdot \grad_\mathscr{C} q_\alpha}
    \label{ap:eq:maxwell}
\end{equation}
Note that up to now we have made no assumption on the shape of $P$ and on the nature of the internal coordinates $\mathscr{C}$. 
Nevertheless, only the number of particles  $n$, appear in \ref{ap:eq:maxwell}.
At this point of the development we must specify the nature of the internal coordinates.
Thus, in the next section we investigate different situation. 
In this context the scalar $\Psi$ represent the source term due to 3 things. 
The coalescence and break-up phenomena, and the inter particular collision. 
The source term can be thus expressed under this form \citep{rao2008introduction,curtiss1956kinetic},  
\begin{equation*}
    \Psi(\mathscr{C}) = B(\mathscr{C}) + D(\mathscr{C}) + \Pi(\mathscr{C}) + \nabla \cdot \Theta(\mathscr{C})
\end{equation*}
where $B$ is the birth term, $D$ the death term, $\Pi$ the collision source term, and $\Theta$ the collision flux. 

\section{Point mass particles}
We start by considering only point of mass particles. 
Therefore, we consider only the momentum of the particles $\textbf{p}_\alpha$ and the volume $V_\alpha$ as an internal coordinate.
Thus, \ref{ap:eq:maxwell} yields,
\begin{equation}
    \pddt \left( n \avg{q_\alpha}\right)
    + \bm{\nabla} \cdot \left(n \avg{q_\alpha \textbf{u}_\alpha}\right)
    - n  \avg{\frac{\partial V_\alpha}{\partial t} \cdot \frac{\partial q_\alpha}{\partial V_\alpha}}
    - n  \avg{\frac{\partial \textbf{p}_\alpha}{\partial t} \cdot \frac{\partial q_\alpha}{\partial \textbf{p}_\alpha}}
    = n \avg{\Psi q_\alpha},
\end{equation}
From this equation we can recover the conservation equations of the particular phase by replacing $q_\alpha$ by the right quantity. 
Consider, $f = 1$, for example, we get, 
\begin{equation}
    \pddt n
    + \grad \cdot \left(n \avg{\textbf{u}_\alpha}\right)
    = n\avg{\Psi}
\end{equation}
which is the conservation of the number density of the particles. 
Where the right-hand side is to coalesce and break up kernel.
Similarly, the mass balance equation can be obtained with $f = V_\alpha \rho_d$, yielding,
\begin{equation}
    \pddt (n\avg{m_\alpha})
    + \bm{\nabla} \cdot \left(n \avg{\textbf{u}_\alpha m_\alpha}\right)
    =
    n  \avg{\int_{S_\alpha} M_d dS},
\end{equation}
where we have use \ref{eq:dt_m_alpha} to make appear the mass transfer term. 
Note that the source term is null since coalescence and break-up conserve the mass. 
Again, for the momentum conservation equation we set $f = \textbf{p}_\alpha$.
\begin{equation}
    \pddt \left( n \avg{\textbf{p}_\alpha}\right)
    + \grad \cdot \left(n \avg{\textbf{u}_\alpha \textbf{p}_\alpha}\right)
    =  \nabla \cdot \Theta
    + n  \avg{\frac{\partial \textbf{p}_\alpha}{\partial t}}
\end{equation}
where $B$, $D$ and $\Pi$ cancel out since they doesn't change the momentum of the phase. 
Nevertheless, note that the collision flux term stay \citep{rao2008introduction}.
Using, the momentum balance for a single particle \ref{eq:dt_p_alpha}, we can simplify the last term yielding, 
\begin{multline*}
    \pddt \left( n \avg{\textbf{p}_\alpha}\right)
    + \bm{\nabla} \cdot \left(n \avg{\textbf{u}_\alpha \textbf{p}_\alpha }\right)\\
    = 
    \nabla \cdot \Theta
    + n\avg{\int_{V_\alpha} \textbf{b}_k dV}
    + \avg{\int_{S_\alpha} \left(
    \textbf{T}_k\cdot\textbf{n}_k
    - M_k \textbf{u}_k
    \right)dS},
\end{multline*}
The system of equation obtained here is equivalent to \ref{eq:avg_p_momentum}, but with an additional source $\Theta$ terms representing the particle stress.

We can also set $f = \textbf{u}_\alpha$ instead of $\textbf{p}_\alpha$ and consider only $\textbf{u}_\alpha$ as an internal coordinate.
Then by using \ref{eq:u_alpha_dt} we directly have, 
\begin{multline*}
    \pddt \left( n \avg{\textbf{u}_\alpha}\right)
    + \bm{\nabla} \cdot \left(n \avg{\textbf{u}_\alpha \textbf{u}_\alpha }\right)
    = 
    \nabla \cdot \Theta
    + n\avg{\frac{1}{m_\alpha}\int_{V_\alpha} \textbf{b}_k dV} \\
    + \avg{\frac{1}{m_\alpha}\int_{S_\alpha} \left(
    \textbf{T}_k\cdot\textbf{n}_k
    - M_k \textbf{w}_k
    \right)dS}
    + \avg{\frac{1}{m_\alpha} \ddt \int_{S_\alpha} 
        \textbf{r} M_k dS},
\end{multline*}

% \section{Higher order description of the particles}
% In the previous section we only considered the center of mass velocity.
% Here we consider also the first order moment of inertia and momentum.
% Therefore, we consider the following quantities as internal coordinate,
% the momentum $\textbf{p}_\alpha$, the moment of momentum $\mathcal{P}_\alpha$ and the moment of inertia $\mathcal{G}$.
% In the following we will adopt indices notation to be more rigorous. 
% Besides, we will neglect all terms related to the mass transfer.
% \ref{ap:eq:maxwell} now reads as, 
% \begin{multline*}
%     \pddt \left( n \avg{q_\alpha}\right)
%     + \frac{\partial}{\partial x_i} \left(n \avg{\textbf{u}_\alpha q_\alpha}\right) 
%     = n \left<Jf\right>^\lambda.\\
%     + n  \left<\frac{d p^\alpha_i}{dt} \frac{\partial q_\alpha}{\partial p_i} \right>^{\lambda} 
%     + n  \left<\frac{d\mathcal{G}^\alpha_{ij}}{dt} \frac{\partial q_\alpha}{\partial\mathcal{G}_{ij}}\right>^{\lambda} 
%     + n  \left<\frac{d\mathcal{P}^\alpha_{ij}}{dt} \frac{\partial q_\alpha}{\partial\mathcal{P}_{ij}}\right>^{\lambda} 
% \end{multline*}
% Now, we can apply the same process as earlier to get the conservation equations.
% For $f = \mathcal{G}_{ij}^\alpha$ we get,
% \begin{equation*}
%     \pddt \left( n \left<\mathcal{G}_{ij}\right>^\lambda\right)
%     + \frac{\partial}{\partial x_k} \cdot \left(n \left<\mathcal{G}_{ij} u_k^\alpha \right>^\lambda\right)
%     = n \left<J \mathcal{G}_{ij}\right>^\lambda.
%     + n  \left<\mathcal{S}_\alpha\right>^{\lambda} 
% \end{equation*}
% where we have use \ref{eq:Gorderl}, with $\mathcal{S} = \mathcal{P} + \mathcal{P}^T$ ($^T$ being the transpose operator). 
% As one can note we recover the particular averaged equation for the transport of $\mathcal{G}$.
% Besides, we assumed that $J$ does not cancel since it might not conserve the shape of particle. 
% Now, setting $f = \mathcal{P}_{ij}$ and using \ref{eq:momentMumdeq_\alpha} yields the moment of momentum equation,
% \begin{multline*}
%     \pddt \left( n \left<\mathcal{P}\right>^\lambda\right)
%     + \frac{\partial}{\partial x_i} \left(n \left<\mathcal{P} u_i^\alpha\right>^\lambda\right) 
%     = n \left<J\mathcal{P}\right>^\lambda \\
%     + \lavg{M_{ij}^\alpha}
%     - \lavg{\int_{V_\alpha} \sigma_{ij}dV}
%     + \lavg{\int_{V_\alpha} \rho_d u_i w_j dV}.
% \end{multline*}
% Again we kept the source term $J$, as we have no information on its nature while considering the moment of momentum and shape tensor. 
% As proven in \ref{ap:cinematic} we could derive additional equations for the higher order of inertia and momentum. 

% \section{Alternative derivation of the momentum equation}

% In this section we consider the following internal coordinate, the center of mass velocity $\bm{u}^\alpha$, the moment of inertia $\mathcal{G}$ and the volume $V_\alpha$.
% We would like to empathize that the center of mass velocity isn't rigorously equivalent to the moment $\textbf{p}_\alpha$ (see \ref{ap:cinematic}). 
% Indeed, the latter can be expressed as a Taylor expansion at the center of the particle. 
% Therefore, it is interesting to consider the derivation of the momentum equation using those internal coordinates.
% The global conservation equation now reads as,
% \begin{equation*}
%     \pddt \left( n \avg{q_\alpha}\right)
%     + \frac{\partial}{\partial x_i} \left(n \avg{\textbf{u}_\alpha q_\alpha}\right) 
%     - n  \left<\frac{d u^\alpha_i}{dt} \frac{\partial q_\alpha}{\partial u_i} \right>^{\lambda} 
%     - n  \left<\frac{d V_\alpha}{dt} \frac{\partial q_\alpha}{\partial V}\right>^{\lambda} 
%     - n  \left<\frac{d \mathcal{G}^\alpha_{ij}}{dt} \frac{\partial q_\alpha}{\partial \mathcal{G}_{ij}} \right>^{\lambda} 
%     = n \left<Jf\right>^\lambda.
% \end{equation*}
% According to \ref{ap:cinematic} we can express the linear momentum as $p^\alpha_j = V_\alpha u_j^\alpha + \frac{1}{2} \mathcal{G}_{kl}  K^\alpha_{jkl}+\ldots$, with $K^\alpha_{jkl} = \frac{\partial u_j^\alpha}{\partial y_k\partial y_l}|_\alpha$, where we neglected the third or above order terms. 
% Then, setting $f = p^\alpha_i$ gives, 
% \begin{multline*}
%     \pddt \left( n \left<V_\alpha u_j^\alpha + \frac{1}{2} \mathcal{G}_{kl}  K^\alpha_{jkl}\right>^\lambda\right)
%     + \frac{\partial}{\partial x_i} \left(n \left<
%         \left(V_\alpha u_j^\alpha 
%         + \frac{1}{2} \mathcal{G}_{kl}  K^\alpha_{jkl}\right) u_i^\alpha
%     \right>^\lambda\right) \\
%     + \frac{1}{2} n  \left<\frac{d\mathcal{G}_{kl}  K^\alpha_{jkl}}{dt}\right>^{\lambda} 
%     - \frac{1}{2} n  \left<\frac{d \mathcal{G}^\alpha_{kl}}{dt} K^\alpha_{jkl} \right>^{\lambda} 
%     = n \left<Jf\right>^\lambda
%     + \lavg{\bm{q_\alpha}_\alpha}
%     +\lavg{\bm{b}_{ext}}.
% \end{multline*}
% Since the averaging operator is linear, we can group the derivative yielding, 
% \begin{multline*}
%     \pddt \left( n \left<V_\alpha u_j^\alpha + \frac{1}{2} \mathcal{G}_{kl}  K^\alpha_{jkl}\right>^\lambda\right)
%     + \frac{\partial}{\partial x_i} \left(n \left<
%         \left(V_\alpha u_j^\alpha 
%         + \frac{1}{2} \mathcal{G}_{kl}  K^\alpha_{jkl}\right) u_i^\alpha
%     \right>^\lambda\right) \\
%     + \frac{1}{2} n  \left<\frac{d K^\alpha_{jkl}}{dt}\mathcal{G}_{kl}\right>^{\lambda} 
%     = n \left<Jf\right>^\lambda
%     + \lavg{\bm{q_\alpha}_\alpha}
%     +\lavg{\bm{b}_{ext}}.
% \end{multline*}


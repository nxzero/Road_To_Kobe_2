\chapter{A detailed derivation of the \textit{one-fluid}, \textit{two-fluid} and volume averaged formulation of a generalized balance equations.}
\label{ap:average}


In this appendix we derive the \textit{two-fluid} and \textit{one-fluid} formulation of a generalized conservation equation, namely,
\begin{equation}
    \frac{\partial}{\partial t} f_k
    = \nablabh \cdot (\bm{\Phi}_k - f_k\textbf{u}_k)
    + \textbf{S}_k.
    \label{ap:eq:global_balance}
\end{equation}
Then we derive the phase averaged and global averaged equation of this general equation of conservation. 
First, we recall the relations related to the phase indicator function, 
\begin{equation}
    \frac{\partial}{\partial t} \chi_k
    + \textbf{u}_I  \nablabh \chi_k 
    = 0, \;\;\;\;\text{and}\;\;\;\; 
    \nablabh \chi_k 
    = - \delta_I \textbf{n}_k.
    \label{ap:eq:phase_properties}
\end{equation}
Now we can derive the \textit{two-fluid} formulation by multiplying \ref{ap:eq:global_balance} by the phase indicator function \ref{eq:phase_indicator}. 
It yields, 
\begin{equation*}
    \frac{\partial}{\partial t} (\chi_k f_k)
    = \nablabh \cdot (\chi_k \bm{\Phi}_k - \chi_k f_k \textbf{u}_k)
    + \chi_k \textbf{S}_k.
    + f_k \frac{\partial}{\partial t} \chi_k
    + \left(
        f_k \textbf{u}_k 
        - \bm{\Phi}_k
    \right) \cdot \nablabh \chi_k
    % \label{ap:eq:global_balance}
\end{equation*}
where we have included the phase function $\chi_k$ into the derivative operators. 
Now, using the \ref{ap:eq:phase_properties} we get, 
\begin{equation}
    \frac{\partial}{\partial t} (\chi_k f_k)
    = \nablabh \cdot (\chi_k \bm{\Phi}_k - \chi_k f_k \textbf{u}_k)
    + \chi_k \textbf{S}_k
    + \left[
        \bm{\Phi}_k 
        + f_k 
        \left(
            \textbf{u}_I
            - \textbf{u}_k
        \right) 
    \right]
    \cdot \textbf{n}_k \delta_I 
    \label{ap:eq:two-fluid_global}
\end{equation}
where the last term is the interfacial source term, such as the drag force if $f_k$ is the momentum or the mass transfer if $f_k$ is the density. 
In this equation notice that all quantities are factor of the phase indicator function $\chi_k$. 
So we transport $\chi_k f_k$ which is field quantity defined over the whole domain. 
Thus, \ref{ap:eq:two-fluid_global} is valid over the entire domain.   
Nevertheless, notice that the last term contain term factor of $\delta_I$. 
The jump condition across the interface will be defined as the sum of the interfacial term on each phase $k$. 
It will or will not be null depending on the nature of $f_k$, therefor we note,
\begin{equation}
    \sum_k 
    \left[
        \bm{\Phi}_k 
        + f_k 
        \left(
            \textbf{u}_I
            - \textbf{u}_k
        \right) 
    \right]
    \cdot \textbf{n}_k
    = \textbf{J}_I
    \label{ap:eq:general_jump}
\end{equation}
were $\textbf{J}_I$ i the \textit{jump quantity} related to $f_k$.
This \textit{jump quantity} is the surface tension force in the case where $f_k$ is the momentum.

Now, let's derive the \textit{one-fluid} formulation of \ref{ap:eq:global_balance}.
To do so we sum on add phases \ref{ap:eq:two-fluid_global}, and we make use of the jump condition \ref{ap:eq:general_jump} for the interfacial term. 
Besides, we define any quantities $q$ being the sum of $q_k\chi_k$ on all phases, i.e. $q = \sum_k \chi_k q_k$.
Then it is trivial to show that, 
\begin{equation}
    \frac{\partial}{\partial t} f
    = \nablabh \cdot (\bm{\Phi} - f \textbf{u})
    + \textbf{S}
    + \textbf{J}_I \delta_I,
    \label{ap:eq:one-fluid_global}
\end{equation}
which is the \textit{one-fluid} formulation. 

The volume average of \ref{ap:eq:two-fluid_global} is then straight forward, by using the definition of the averaged operator, we get, 
\begin{equation*}
    \frac{\partial}{\partial t} (\phi_k\kavg{f})
    = \nablabh \cdot \left(
        \phi_k \kavg{\bm{\Phi} - f \textbf{u}}
    \right)
    + \phi_k \kavg{\textbf{S}}
    + a_I \Iavg{
        \bm{\Phi}_k \cdot \textbf{n}_k
        + f_k 
        \left(
            \textbf{u}_I
            - \textbf{u}_k
        \right) \cdot \textbf{n}_k
    } 
    \label{ap:eq:avg_k_global}
\end{equation*}
Similarly, the bulk average can be obtained by averaging \ref{ap:eq:one-fluid_global} yielding, 
\begin{equation*}
    \frac{\partial}{\partial t} \avg{f}
    = \nablabh \cdot \avg{\bm{\Phi} - f \textbf{u}}
    + \avg{\textbf{S}}
    + a_I\avg{\textbf{J}_I},
    \label{ap:eq:avg_global}
\end{equation*}
\section{A rigorous derivation of the point velocity}
Consider a particle of center of mass $\textbf{y}_\alpha$ defined such as
\begin{equation*}
    m_\alpha \textbf{y}_\alpha
    = \int_{V_\alpha} \rho_k \textbf{y}_k dV,
\end{equation*}
its velocity can be solely the derivation of $\textbf{y}_\alpha$ whitin time.
Yielding, 
\begin{align*}
    \ddt \textbf{y}_\alpha (t)
    &=
    \ddt \left(
        \frac{1}{m_\alpha} \int_{V_\alpha} \rho_k \textbf{y}_k dV
    \right)\\
    &= \frac{1}{m_\alpha}
    \ddt 
    \left(
        \int_{V_\alpha} \rho_k \textbf{y} dV
    \right)
    - \frac{1}{m_\alpha^2} \ddt \int_{V_\alpha} \rho_k dV \int_{V_\alpha} \rho_k \textbf{y}_k dV
    \\
    &= \frac{1}{m_\alpha}\int_{V_\alpha} \left[
        \pddt (\rho_k \textbf{y}) + \nablabh \cdot\left(\rho_k \textbf{y}\textbf{u}_k\right) dV 
    \right]\\
    &+ \frac{1}{m_\alpha}\int_{S_\alpha} \textbf{y} M_k d S
    -  \frac{1}{m_\alpha^2} \int_{S_\alpha} M_k dS  \int_{V_\alpha} \rho_k \textbf{y}_k dV
    \\
    &= \frac{1}{m_\alpha}\int_{V_\alpha} \textbf{y} \left[
    \pddt (\rho_k) + \nablabh \cdot\left(\rho_k \textbf{u}_k\right) dV 
    \right]dV
    + \frac{1}{m_\alpha}\int_{V_\alpha} \rho_k  \textbf{u}_k  \cdot \nablabh \textbf{y} dV \\
    &+ \frac{1}{m_\alpha}\int_{S_\alpha} \textbf{y}_k M_k d S
    - \frac{1}{m_\alpha}  \textbf{y}_\alpha \int_{S_\alpha} M_k dS
\end{align*}
By considering the mass conservation \ref{eq:single-fluid_mass} for the first term,  noticing that $\nablabh \textbf{y} = \textbf{I}$ where $\textbf{I}$ is the identity tensor for the second term and introducing \textbf{r} in the third term gives, we get the following relation,
\begin{equation*}
    \textbf{u}_\alpha
    = \frac{1}{m_\alpha} \left(
        \textbf{p}_\alpha
        +  \int_{S_\alpha} \textbf{r} M_k dS
    \right)
    % = \frac{1}{m_\alpha}  \left(
    %     \textbf{p}_\alpha
    % - \int_{V_\alpha} \rho_k \textbf{w} dV
    % \right)
\end{equation*}

\section{Decomposition of the particular energy balance.}
This section is a detailed derivation of the energy equation for a whole fluid particle, namely,
\begin{equation*}
    \label{ap:eq:E_alpha_dt}
    \ddt E_\alpha 
    % = \ddt \int_{V_\alpha} \rho_k E_k dV
    = \int_{V_\alpha} \textbf{b}_k \cdot \textbf{u}_k dV
    + \int_{S_\alpha} \left[
        (\textbf{T}\cdot \textbf{u} 
    - \textbf{q})\cdot\textbf{n}_k 
    + M_k E_k 
    + \textbf{f}_I \cdot \textbf{u}_I 
    \right]dS, 
\end{equation*}
Since we can decompose the velocity fields of a particle following $\textbf{u}_k = \textbf{u}_\alpha + \textbf{w}$ it is then possible to rewrite the total energy, 
Yielding, 
\begin{multline*}
    \int_{V_\alpha} \rho_k E_k dV
    = \int_{V_\alpha} \rho_k e_k dV
    + \frac{1}{2} \int_{V_\alpha} \rho_k \textbf{u}_\alpha\cdot\textbf{u}_\alpha dV\\
    + \int_{V_\alpha} \rho_k \textbf{u}_\alpha\cdot\textbf{w} dV
    + \frac{1}{2} \int_{V_\alpha} \rho_k \textbf{w}\cdot\textbf{w} dV
\end{multline*}
by applying the relation \ref{eq:M_alpha_dt} on the second term, it is possible to show that,
\begin{equation*}
    E_\alpha
    = \int_{V_\alpha} \rho_k e_k dV
    + \frac{1}{2} \textbf{u}_\alpha\cdot\textbf{u}_\alpha  m_\alpha
    + \textbf{u}_\alpha\cdot \int_{V_\alpha} \rho_k \textbf{w} dV
    + \frac{1}{2} \int_{V_\alpha} \rho_k \textbf{w}\cdot\textbf{w} dV.
\end{equation*}
We clearly see that the energy can be decomposed into internal and kinetic energy. 
We define the integrated internal energy of the particle by $e_\alpha = \int_{V_\alpha} = \rho_k e_k dV$.
It is the straight forward (using \ref{eq:one-fuild_internal_energy}) to show that, 
\begin{equation*}
    \ddt e_\alpha
    = \int_{V_\alpha} \textbf{T}:\nablabh\textbf{u}_k dV
    + \int_{S_\alpha} \left(
        e_k M_k
        - \textbf{q}_k \cdot \textbf{n}_k
    \right) dS.
\end{equation*}
Similarly, the kinetic energy equation for a whole fluid particle can be obtained deriving the local kinetic energy ,namely,
\begin{equation*}
    \ddt \int_{V_\alpha} \rho_k \frac{u_k^2}{2} dV
    = \int_{V_\alpha}\textbf{u}_k \cdot  \left(
        \textbf{b}_k
        + \nablabh \cdot \textbf{T}_k
    \right)dV
    + \int_{S_\alpha} \frac{u_k^2}{2} M_k dS.
\end{equation*}
Using the velocity decomposition one can deduce, 
\begin{multline}
    \frac{1}{2} \ddt \left(
        m_\alpha \textbf{u}_\alpha \cdot \textbf{u}_\alpha
        + 2\textbf{u}_\alpha \cdot \int_{V_\alpha}  \rho_k \textbf{w}_k dV
        + \int_{V_\alpha} \rho_k \textbf{w}_k \cdot \textbf{w}_k dV
    \right)\\
    =  \textbf{u}_\alpha \left[
        \cdot\int_{V_\alpha} \textbf{b}_k dV
        +  \int_{S_\alpha} \left(
            \textbf{T}_k \cdot \textbf{n}_k
            + \frac{1}{2} \textbf{u}_\alpha M_k 
            + \textbf{w}_k M_k 
        \right)dS
    \right]\\
    + \int_{V_\alpha} \left(
        \textbf{w}_k\cdot\textbf{b}_k
        -\textbf{T}_k : \nablabh \textbf{w}_k
    \right)dV
    + \int_{S_\alpha} 
        \textbf{w}_k\cdot(\textbf{T}_k\cdot \textbf{n}_k)
    dS\\
    + \int_{S_\alpha} \frac{1}{2} \textbf{w}_k \cdot \textbf{w}_k M_k dS.
    \label{ap:eq:u_2_dt}
\end{multline}
Besides, taking the dot product of the centered velocity $\textbf{u}_\alpha$, with the momentum equation \ref{eq:dt_p_alpha}, gives, 
\begin{multline*}
    \frac{1}{2}\ddt (m_\alpha \textbf{u}_\alpha \cdot \textbf{u}_\alpha)
    + \textbf{u}_\alpha \cdot \ddt \int_{V_\alpha} \rho_k \textbf{w}_k dV \\
    = \textbf{u}_\alpha \cdot \int_{V_\alpha} \textbf{b}_k dV
    + \textbf{u}_\alpha \cdot \int_{S_\alpha} \left(
    \textbf{T}_k \cdot\textbf{n}_k
    +\frac{\textbf{u}_\alpha}{2} M_k
    +\textbf{w}_k M_k
    \right)dS,
\end{multline*}
We can notice that the RHS of this equation correspond rigorously to the second line of \ref{ap:eq:u_2_dt}.
Therefore, taking the difference between those equations, yields the internal motion equation of an arbitrary particle, namely, 
\begin{multline*}
    \frac{1}{2}\ddt \int_{V_\alpha} \frac{\rho_k}{2} \textbf{w}_k \cdot \textbf{w}_k dV
    = \int_{V_\alpha} \left(
        \textbf{w}_k\cdot\textbf{b}_k
        -\textbf{T}_k : \nablabh \textbf{w}_k
    \right)dV\\
    + \int_{S_\alpha} 
        \textbf{w}_k\cdot(\textbf{T}_k\cdot \textbf{n}_k)
    dS
    + \int_{S_\alpha} \frac{1}{2} \textbf{w}_k \cdot \textbf{w}_k M_k dS.
\end{multline*}

\section{Derivation of kinetic Turbulence Evolution Equations}

The aim of this section is to derive the transport equation for the granular temperature scalar. 
Let's define the grain temperature, $T$ as such, $T =\frac{1}{2} \textbf{u}'\cdot\textbf{u}'$, where $\textbf{u}'$ is the fluctuation velocity of a given average procedure. 

\subsection{For a continuous phase}
We start this derivation for the continuous phase $k$, so that $\textbf{u}'_k = \textbf{u}_k - \kavg{\textbf{u}}$.
We start from the averaged kinetic energy equation over the $k$ phase, namely : 
\begin{multline*}
    \frac{\rho_k}{2}\frac{\partial }{\partial t}\left(
        \phi_k
        \kavg{u^2}
    \right)
    +\frac{\rho_k}{2}\nablab\cdot \left(
        \phi_k
        \kavg{u^2\textbf{u}}
    \right)
    =
    \nablab\cdot\left(
        \phi_k
        \kavg{\textbf{u}\cdot \textbf{T}}
    \right)
    +\phi_k\kavg{\textbf{u}\cdot\textbf{b} - \textbf{T}: \nablabh\textbf{u}}\\
    +a_I \Iavg{
        (\textbf{T}_k\cdot\textbf{u}_k)\cdot\textbf{n}_k
        + \frac{u^2_k}{2} M_k}.
\end{multline*}
Breaking down the LHS of this equation times $\frac{2}{\rho_k}$, yields,
\begin{align*}
    &\frac{\partial }{\partial t}\left(
        \phi_k
        \kavg{u^2}
    \right)
    +
    \nablab\cdot \left(
        \phi_k
        \kavg{u^2\textbf{u}}
    \right)\\
    &=
    \phi_k\frac{\partial }{\partial t}\kavg{u^2}
    +\kavg{u^2}\frac{\partial }{\partial t}\phi_k
    +\nablab\cdot \left(
        \phi_k
        \kavg{u^2}\kavg{\textbf{u}}
        +\phi_k
        \kavg{u^2\textbf{u}'}
    \right)\\
    &=
    \phi_k
    \left[
        \frac{\partial }{\partial t}\kavg{u^2}
        + 
        \kavg{\textbf{u}}
        \cdot\nablab 
        \kavg{u^2}
    \right]
    +\kavg{u^2} \left[
        \frac{\partial }{\partial t}\phi_k
        + \nablab\cdot \left(
            \phi_k
            \kavg{\textbf{u}}
        \right)
    \right]\\
    &+ \nablab\cdot \left(
        2\phi_k
        \kavg{\textbf{u}}\cdot\kavg{\textbf{u'}\textbf{u'}} + \phi_k \kavg{T \textbf{u'}}
    \right)
\end{align*}
where we used $\kavg{u^2} = \kavg{u}^2 + T$. 
By using the averaged mass conservation \ref{eq:avg_k_mass} times $\frac{1}{2}\kavg{u^2}$, namely, 
\begin{equation*}
    \frac{\rho_k}{2} \kavg{u^2} \pddt \phi_k 
    + \frac{\rho_k}{2} \kavg{u^2} \nablab \cdot\left(\phi_k\kavg{\textbf{u}}\right)
    = \frac{a_I}{2}\kavg{u^2}\Iavg{M_k},
\end{equation*}
we can rewrite the energy balance in conservative form. 
It reads as, 
\begin{multline}
    \phi_k\frac{\rho_k}{2}  \left[
        \frac{\partial }{\partial t}
        \kavg{u^2}
        +\kavg{\textbf{u}}\cdot\nablab 
        \kavg{u^2}
    \right]
    =
    \nablab\cdot\left(
        \phi_k
        \kavg{\textbf{u}\cdot \textbf{T}
        - \rho_k\kavg{\textbf{u}}\cdot\textbf{u'u'}
        - \frac{\rho_k}{2}T\textbf{u'}}
    \right)\\
    +\phi_k\kavg{\textbf{u}\cdot\textbf{b} - \textbf{T}: \nablabh\textbf{u}}
    +a_I \Iavg{
        (\textbf{T}_k\cdot\textbf{u}_k)\cdot\textbf{n}_k
        + \frac{u^2_k - \kavg{u^2}}{2} M_k}.
    \label{ap:eq:avg_k_u_2}
\end{multline}
On the other hand the momentum equation reads as, 
\begin{multline*}
    \rho_k\pddt (\phi_k\kavg{\textbf{u}}) 
    + \rho_k\nablab\cdot(\phi_k\kavg{\textbf{u}}\kavg{\textbf{u}})\\
    = \nablab\cdot\left[
        \phi_k \kavg{\textbf{T}
        - \rho_k \textbf{u'u'}}
    \right]
    +\phi_k\kavg{\textbf{b}}
    + a_I\Iavg{M_k \textbf{u}_k +\textbf{n}_k\cdot\textbf{T}_k},
\end{multline*}
using the mass balance times $\kavg{\textbf{u}}$, we can write the momentum equation as, 
\begin{multline*}
    \rho_k \phi_k \left[
        \pddt \kavg{\textbf{u}}
        + \kavg{\textbf{u}}\cdot\nablab\kavg{\textbf{u}}
    \right]\\
    = \nablab\cdot\left[
        \phi_k \kavg{\textbf{T}
        - \rho_k \textbf{u'u'}}
    \right]
    +\phi_k\kavg{\textbf{b}}
    + a_I\Iavg{M_k \left(\textbf{u}_k - \kavg{\textbf{u}}\right) +\textbf{n}_k\cdot\textbf{T}_k}.
\end{multline*}
Then taking the dot product of this equation with $\kavg{\textbf{u}}$ gives, 
\begin{multline}
    \phi_k\frac{\rho_k}{2}  \left[
        \pddt \kavg{u}^2
        + \kavg{\textbf{u}}\cdot\nablab\kavg{u}^2
    \right]\\
    = \nablab\cdot\left[
        \phi_k \kavg{\textbf{u}}\cdot\kavg{\textbf{T}
        - \rho_k  \textbf{u'u'}}
    \right]
    +\phi_k\kavg{\rho_k \textbf{u'u'} - \textbf{T}}:\nablab
         \kavg{\textbf{u}}\\
    +\phi_k\kavg{\textbf{u}}\cdot\kavg{\textbf{b}}
    + a_I\Iavg{M_k \left(\textbf{u}_k\cdot\kavg{\textbf{u}} 
    - \kavg{u}^2\right) +\textbf{n}_k\cdot(\kavg{\textbf{u}}\cdot\textbf{T}_k)}.
    \label{ap:eq:avg_k_u_u}
\end{multline}
Finally, we can obtain the transport equation of $T$ by subtracting \ref{ap:eq:avg_k_u_2} with \ref{ap:eq:avg_k_u_u}. 
Yielding the following equation, 
\begin{multline}
    \phi_k\rho_k  \left[
        \frac{\partial }{\partial t}
        \kavg{T}
        +\kavg{\textbf{u}}\cdot\nablab 
        \kavg{T}
    \right]\\
    =
    \nablab\cdot\left(
        \phi_k
        \kavg{\textbf{u'}\cdot \textbf{T'}
        - \rho_k T\textbf{u'}}
    \right)
    +\phi_k\kavg{\rho_k \textbf{u'u'}}:\nablab
         \kavg{\textbf{u}}\\
    +\phi_k\kavg{\textbf{u'}\cdot\textbf{b'} + \textbf{T'}: (\nablabh\textbf{u})'}
    +a_I \Iavg{
        (\textbf{T}_k'\cdot\textbf{u}'_k)\cdot\textbf{n}_k
        + T M_k}.
    \label{ap:eq:avg_k_T}
\end{multline}


\subsection*{Dispersed phase}

In the same spirit as the previous section we carry out the derivation for the transport of the kinetic turbulence evolution $T$. 
The only difference is that $T$ is now defined relative to the mean velocity of the particular phase $\pavg{u}$.  
From the previous section we know that we can write the energy equation under the following form, 
\begin{multline*}
    \frac{1}{2}\ddt (m_\alpha u^2_\alpha)
    + \textbf{u}_\alpha \cdot \ddt \int_{V_\alpha} \rho_k \textbf{w}_k dV 
    = \textbf{u}_\alpha \cdot \int_{V_\alpha} \textbf{b}_k dV\\
    + \textbf{u}_\alpha \cdot \int_{S_\alpha} \left[
    \textbf{T}_k \cdot\textbf{n}_k
    +\frac{\textbf{u}_\alpha}{2} M_k
    +\textbf{w}_k M_k
    \right]dS.
\end{multline*}
Applying the particular average operator yields the averaged point of mass kinetic energy equation, 
\begin{multline*}
    \frac{1}{2}\pddt   \left(\pavg{m_\alpha} \pnavg{u^2_\alpha}\right)
    + \frac{1}{2}\nablab \cdot \left(\pavg{m_\alpha} \pnavg{u^2_\alpha \textbf{u}_\alpha}\right) 
    = \pavg{\textbf{u}_\alpha \cdot \int_{V_\alpha} \textbf{b}_k dV}\\
    - \frac{1}{2}\pddt \left(\pavg{m_\alpha'(u_\alpha^2)'}\right)
    - \frac{1}{2}\nablab\cdot\left(\pavg{m_\alpha' (u_\alpha^2 \textbf{u}_\alpha)'}\right)\\
    - \pavg{\textbf{u}_\alpha \cdot \ddt \int_{V_\alpha} \rho_k \textbf{w}_k dV} 
    + \pavg{\textbf{u}_\alpha \cdot \int_{S_\alpha} \left[
    \textbf{T}_k \cdot\textbf{n}_k
    +\frac{\textbf{u}_\alpha}{2} M_k
    +\textbf{w}_k M_k
    \right]dS}.
\end{multline*}
From this step, we can carry out the same decomposition as the previous section for the RHS/2. 
Namely, 
\begin{multline*}
    \frac{\partial }{\partial t}\left(
        \pavg{m_\alpha}
        \pnavg{u^2_\alpha}
    \right)+
    \nablab\cdot \left(
        \pavg{m_\alpha}
        \pnavg{u^2_\alpha\textbf{u}_\alpha}
    \right) \\
    =
    \pavg{m_\alpha}
    \left[
        \frac{\partial }{\partial t}\pnavg{u^2_\alpha}
        + 
        \pnavg{\textbf{u}_\alpha}
        \cdot\nablab 
        \pnavg{u^2_\alpha}
    \right]
    +\pnavg{u^2_\alpha} \left[
        \frac{\partial }{\partial t}\pavg{m_\alpha}
        + \nablab\cdot \left(
            \pavg{m_\alpha}
            \pnavg{\textbf{u}_\alpha}
        \right)
    \right]\\
    + \nablab\cdot \left(
        2\pavg{m_\alpha}
        \pnavg{\textbf{u}_\alpha}\cdot\pnavg{\textbf{u'}\textbf{u'}} 
        + \pavg{m_\alpha} \pnavg{T \textbf{u'}}
    \right)
\end{multline*}
As we can observe the possible polydispersity of the flow, add supplementary terms linked to the fluctuation of the mass, $m_\alpha'$.
The energy equations in the conservative form then reads as, 
\begin{multline*}
    \frac{\pavg{m_\alpha}}{2}
    \left[
        \frac{\partial }{\partial t}\pnavg{u^2_\alpha}
        + 
        \pnavg{\textbf{u}_\alpha}
        \cdot\nablab 
        \pnavg{u^2_\alpha}
    \right]
    = \pavg{\textbf{u}_\alpha \cdot \int_{V_\alpha} \textbf{b}_k dV} + \nablab \cdot \textbf{L}\\
    - \pavg{\textbf{u}_\alpha \cdot \ddt \int_{V_\alpha} \rho_k \textbf{w}_k dV} 
    - \pavg{u_\alpha^2}\pnavg{\int_{S_\alpha} M_k d S}\\
    + \pavg{\textbf{u}_\alpha \cdot \int_{S_\alpha} \left[
    \textbf{T}_k \cdot\textbf{n}_k
    +\frac{\textbf{u}_\alpha}{2} M_k
    +\textbf{w}_k M_k
    \right]dS},
\end{multline*}
where \textbf{L} represent all the fluctuation terms derived in the two previous equations. 

Now let's look at the particular averaged momentum equation. 
Using similar decomposition as in the previous part,
and by using the decomposition of the momentum, $\textbf{p}_\alpha = m_\alpha \textbf{u}_\alpha + \int_{V_\alpha} \textbf{w}_k dV$, the equation aforesaid reads as, 
\begin{multline*}
    \pavg{m_\alpha} \left[
        \pddt \pnavg{\textbf{u}_\alpha}
        + \pnavg{\textbf{u}_\alpha}\cdot\nablab\pnavg{\textbf{u}_\alpha}
    \right]
    = \pavg{\int_{V_\alpha} \textbf{b}_k dV}
    + \nablab \cdot \textbf{R}\\
    - \pavg{\ddt \int_{V_\alpha} \rho_k \textbf{w}_k dV} 
    + \pavg{\int_{S_\alpha} \left[\textbf{T}_k + \rho_k \textbf{u}_k (\textbf{u}_I-\textbf{u}_k) \right] \cdot \textbf{n}_k d S}
\end{multline*}
where \textbf{R} is the term that gather all the fluctuation terms. 
Multiplying, this equation by $\pnavg{\textbf{u}_\alpha}$ and taking the difference with the energy equation yielding, 
\begin{multline*}
    \frac{\pavg{m_\alpha}}{2}
    \left[
        \frac{\partial }{\partial t}\pnavg{T_\alpha}
        + 
        \pnavg{\textbf{u}_\alpha}
        \cdot\nablab 
        \pnavg{T_\alpha}
    \right]
    = \pavg{\textbf{u}_\alpha' \cdot \left(\int_{V_\alpha} \textbf{b}_k dV\right)'} 
    + \nablab \cdot \left(\textbf{L}-\textbf{R}\right)\\
    - \pavg{\textbf{u}_\alpha' \cdot \left(\ddt \int_{V_\alpha} \rho_k \textbf{w}_k dV\right)'} \\
    - \pavg{T_\alpha}\pnavg{\int_{S_\alpha} M_k d S}
    + \pavg{\textbf{u}_\alpha \cdot \int_{S_\alpha} \left[
    \textbf{T}_k \cdot\textbf{n}_k
    +\frac{\textbf{u}_\alpha}{2} M_k
    +\textbf{w}_k M_k
    \right]dS},
\end{multline*}
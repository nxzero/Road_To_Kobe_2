\documentclass{sintefbeamer}

% packages, font, color, and newcommands
\usepackage{amsfonts, amsmath, oldgerm, lmodern, bm}

\usepackage[font={footnotesize}]{caption}
\newcommand{\size}{0.22\textwidth}
\newcommand{\avg}[1]{\left<#1\right>}
\newcommand{\davg}[1]{\left<#1\right>_d}
\newcommand{\cavg}[1]{\left<#1\right>_c}
\newcommand{\kavg}[1]{\left<#1\right>_k}
\newcommand{\Iavg}[1]{\left<#1\right>_I}
\newcommand{\pavg}[1]{n \left<#1\right>_p}
\newcommand{\mavg}[1]{\left<#1\right>_m}
\newcommand{\lavg}[1]{\theta_0\left<#1\right>^\lambda}
\newcommand{\partials}[1]{\partial_{i_1}\partial_{i_2}\ldots\partial{i_{#1}}}
\newcommand{\partialp}[2]{ \prod_{m=#1}^{#2} \partial_{i_m}}
\newcommand{\hatpartialp}[2]{ \prod_{m=#1}^{#2} \hat{\partial}_{j_m}}
\newcommand{\hatpartialpi}[2]{ \prod_{m=#1}^{#2} \hat{\partial}_{i_m}}
\newcommand{\pri}[2]{ \prod_{m=#1}^{#2} r_{i_m}}
\newcommand{\prj}[2]{ \prod_{m=#1}^{#2} r_{j_m}}
\newcommand{\nablab}{\bm{\nabla}}
\newcommand{\nablabh}{\hat{\bm{\nabla}}}
\newcommand{\ddt}{\frac{d}{d t}}
\newcommand{\pddt}{\frac{\partial}{\partial t}}
\usefonttheme{serif}
% \themecolor{blue} 
% meta-data
\title{The hybrid model for arbitrary dispersed two-phase flows.}
\subtitle{Generalization to liquid dispersed phase.}
\author{\href{mailto:qilong-kirov.liu@connect.polyu.hk}{Fintzi Nicolas}}
\date{Created on May 22, 2022}

\titlebackground{image/pic/bulles.png}

% document body
\begin{document}
\maketitle

\section{Classic two-phase flow formalism}

\begin{frame}{Governing equations at the microscopic scale.}
  The fluid in the phase $k$ of the flow follow :
  \begin{equation}
    \pddt f_k
    = \nablabh \cdot \left(
        \bm{\Phi}_k
        - f_k\textbf{u}_k
        \right)
    + \textbf{S}_k,
    \label{eq:general_conservation}
\end{equation}
where,
\begin{itemize}
  \item $f_k$ is an arbitrary quantity, (mass, momentum\ldots).
  \item $\textbf{u}_k$ is the velocity fields present in the phase $k$. 
  \item $\bm{\Phi}_k(f_k)$ is the non-convective term related to $f_k$.
  \item $\textbf{S}_k(f_k)$ is the source term related to $f_k$.
\end{itemize}
\end{frame}

\begin{frame}{The phase indicator function.}
  The phase indicator function (PIF) of the phase $k$ :
  \begin{equation}
    \chi_k(\textbf{y}) =  \left\{
      \begin{tabular}{cc}
        $1 \;\text{if} \;\textbf{y} \in V_k$\\
        $0 \;\text{if} \;\textbf{y} \notin V_k$
      \end{tabular}
      \right.,
      \label{eq:phase_indicator}
\end{equation}
Topological equation of the PIF :
\begin{equation}
  \pddt \chi_k
  + \textbf{u}_I  \nablabh \chi_k 
  = 0, \;\;\;\;\text{and}\;\;\;\;
    \nablabh \chi_k 
    = - \delta_I \textbf{n}_k.
  \label{eq:phaseindicator_transport}
\end{equation}

\begin{itemize}
  \item \textbf{y} local position vector,
  \item $\nablabh = \frac{\partial}{\partial \textbf{y}}$, is the local gradient operator.
  \item  $V_k$ is the volume occupied by the phase $k$.
  \item $\delta_I = \delta(\textbf{y} - \textbf{y}_I)$ with $\textbf{y}_I$ the position of the interface.
  \item $\textbf{n}_k$ normal exterior of phase $k$.
\end{itemize}
\end{frame}


\begin{frame}
  {The two fluid formulation}
  By multiplying the general conservation law with the PIF we get the \textit{two-fluid} formulation (valid over the whole domain) :
  \begin{equation}
    \pddt (\chi_k f_k)
    = \nablabh \cdot (\chi_k \bm{\Phi}_k - \chi_k f_k \textbf{u}_k)
    + \chi_k \textbf{S}_k
    + \underbrace{
    \left[
        \bm{\Phi}_k 
        + f_k 
        \left(
            \textbf{u}_I
            - \textbf{u}_k
        \right) 
    \right]
    \cdot \textbf{n}_k \delta_I }_{\text{Interfacial source term}},
    \label{eq:two-fluid_global}
\end{equation}
With the jump condition, 
\begin{equation}
  \sum_k 
  \left[
      \bm{\Phi}_k 
      + f_k 
      \left(
          \textbf{u}_I
          - \textbf{u}_k
      \right) 
  \right]
  \cdot \textbf{n}_k
  = \textbf{J}_I
  \label{eq:general_jump}
\end{equation}
\begin{itemize}
  \item $\textbf{J}_I$ Jump quantity, (surface tension \ldots). 
\end{itemize}
\end{frame}

\begin{frame}{The single fluid formulation.}
  Adding the \textit{two-fluid} formulation on all phases $k$, we obtain the single fluid formulation, 
  \begin{equation}
    \pddt f
    = \nablabh \cdot (\bm{\Phi} - f \textbf{u})
    + \textbf{S}
    +\underbrace{ 
    \textbf{J}_I \delta_I,}_{\text{Interfacial source term}},
    \label{eq:one-fluid_global}
\end{equation}
where,
\begin{itemize}
  \item $f = \sum_k f_k \chi_k$
  \item $\textbf{S} = \sum_k \textbf{S}_k \chi_k$
  \item $\bm{\Phi} = \sum_k \bm{\Phi}_k \chi_k$
\end{itemize}
\end{frame}


\section{Continuous averaged equations.}
\begin{frame}{Continuous volume average}
Definition of the volume average, 
  \begin{equation}
    \left<f\right>(\textbf{x},t) = \int g(\textbf{x},\textbf{y}) f(\textbf{y},t)dV,
    \label{eq:avg}
\end{equation}
where this operator follow those rules,
\begin{align}
  \left<f+g\right> = \left<f\right>+\left<g\right>, \;\;\;\;
  \left<\left<f\right>g\right> = \left<f\right>\left<g\right>, \\
  \left<\frac{\partial f}{\partial t}\right> 
  = \pddt\left<f\right>, \;\;\;\;
  \left<\nablabh f\right> 
  = \nablab\left<f\right>. 
  \label{eq:avg_properties}
\end{align}
\begin{itemize}
  \item \textbf{x} global position vector. 
  \item $g(\textbf{x},\textbf{y})$ the weighting function. 
  \item $\nablabh = \frac{\partial}{\partial \textbf{x}}$, is the global gradient operator.
\end{itemize}
\end{frame}

\begin{frame}{Continuous averaged equations}
  Continuous average on the phase $k$ : 
  \begin{equation*}
    \pddt (\phi_k\kavg{f})
    = \nablab \cdot \left(
        \phi_k \kavg{\bm{\Phi} - f \textbf{u}}
    \right)
    + \phi_k \kavg{\textbf{S}}
    + a_I \Iavg{
        \bm{\Phi}_k \cdot \textbf{n}_k
        + f_k 
        \left(
            \textbf{u}_I
            - \textbf{u}_k
        \right) \cdot \textbf{n}_k
    }.
    \label{eq:avg_k_global}
\end{equation*}
Averaged jump condition, 
\begin{equation}
  \sum_k 
  \Iavg{
      \bm{\Phi}_k 
      \cdot \textbf{n}_k
      + f_k 
      \left(
          \textbf{u}_I
          - \textbf{u}_k
      \right) 
      \cdot \textbf{n}_k
  }
  = \Iavg{\textbf{J}_I}
  \label{eq:general_jump}
\end{equation}
Continuous average on the whole domain, 
\begin{equation*}
  \pddt \avg{f}
  = \nablab \cdot \avg{\bm{\Phi} - f \textbf{u}}
  + \avg{\textbf{S}}
  + a_I\avg{\textbf{J}_I},
  \label{eq:avg_global}
\end{equation*}
\begin{itemize}
  \item $\phi_k = \int g \chi_k dV$ is the volume fraction of phase $k$. 
  \item $a_I = \int g \delta_I dV$ is the interfacial concentration. 
\end{itemize}
\end{frame}

\begin{frame}
  {Application to mass and momentum equations}
  Mass conservation law for the phase $k$ :
  \begin{equation}
    \pddt (\phi_k \rho_k)
    + \nablab \cdot \left(\phi_k 
        \kavg{\rho_k \textbf{u}}
    \right) 
    = a_I\Iavg{M_k},
    \label{eq:avg_k_mass}
\end{equation}
  Momentum conservation of the $k$ phase : 
  \begin{equation}
    \pddt (\phi_k\kavg{\rho_k\textbf{u}}) 
    % + \nablab\cdot(\phi_k\kavg{\textbf{uu}})
    = \nablab\cdot\left[
        \phi_k \kavg{\textbf{T}
        - \rho_k \textbf{uu}}
    \right]
    +\phi_k\kavg{\textbf{b}}
    + a_I\Iavg{M_k \textbf{u}_k +\textbf{n}_k\cdot\textbf{T}_k},
\end{equation}
where, 
\begin{itemize}
  \item $M_k = \rho_k (\textbf{u}_k-\textbf{u}_I) \cdot \textbf{n}_k$ is the mass transfer term.
  \item $\textbf{T}_k$ is the stress tensor in phase $k$.
  \item $\textbf{b}_k$ are the body forces in phase $k$. 
\end{itemize}
\end{frame}

\section{Lagrangian description of the dispersed phase.}

\begin{frame}{Evolution of a single particle's property.}
  For a Lagrangian property $q_\alpha$, where,
  \begin{equation}
    q_\alpha
    = \int_{\Omega_\alpha(t)} f_k(\textbf{y}) d\Omega,
    \label{eq:q_alpha}
\end{equation}
where $\Omega_\alpha$ is defined as, $\Omega_\alpha \subseteq  V_\alpha$.

With the Reynolds transport theorem, we can show that for $\Omega_\alpha = V_\alpha$ we have, 
\begin{align*}
  \ddt  q_\alpha 
  &= \int_{V_\alpha}\left[ \pddt f_k + \nablabh \cdot\left(f_k\textbf{u}_k\right) \right]dV 
    + \int_{S_\alpha} f_k (\textbf{u}_I-\textbf{u}_k)\cdot \textbf{n}_k d S,\\
  &= \int_{V_\alpha} \textbf{S}_k dV 
  + \int_{S_\alpha} \left[\bm{\Phi}_k + f_k (\textbf{u}_I-\textbf{u}_k) \right] \cdot \textbf{n}_k d S.
\end{align*}
\begin{itemize}
  \item $V_\alpha$ volume of the particle $\alpha$.
  \item $S_\alpha$ surface of the particle $\alpha$.
  \item $q_\alpha$ integrated property of the particle $\alpha$ (mass, momentum \ldots)
\end{itemize}
\end{frame}

\begin{frame}{From lagrangian to Eulerian descripton of particles.}
  Any Lagrangian quantity $q_\alpha(t)$ will be represented by the fields $q_\alpha(t)\delta(\textbf{y}-\textbf{y}_\alpha)$.

  Similarly The Lagrangian quantity $\ddt q_\alpha$ will be represented by the fields $\delta_\alpha \ddt q_\alpha$.

  It can be shown that : 
  \begin{equation*}
    \pddt \delta_\alpha
    + \nablabh (\delta_\alpha \textbf{u}_\alpha)
    = 0,
    \label{eq:delta_q_alpha_dt}
\end{equation*}
\begin{equation*}
    \delta_\alpha \ddt q_\alpha
    = \pddt (\delta_\alpha q_\alpha)
    + \nablabh (\delta_\alpha q_\alpha \textbf{u}_\alpha),
    \label{eq:delta_q_alpha_dt}
\end{equation*}
Besides, using the Reynolds transport theorem we define the velocity of the particle $\alpha$ as,  
\begin{equation*}
  \textbf{u}_\alpha
  = \ddt \textbf{y}_\alpha
  = \frac{1}{m_\alpha} \left(
      \int_{V_\alpha} \rho_k \textbf{u}_k dV
      +  \int_{S_\alpha} \textbf{r} M_k dS
  \right)
  % = \frac{1}{m_\alpha}  \left(
  %     \textbf{p}_\alpha
  % - \int_{V_\alpha} \rho_k \textbf{w} dV
  % \right)
\end{equation*}

If there is more than one particle in the flow, we replace the fields $q_\alpha(t)\delta(\textbf{y}-\textbf{y}_\alpha)$ by its sum on every particles, i.e. : the field, $\sum_k q_\alpha(t)\delta(\textbf{y}-\textbf{y}_\alpha)$.
Notice that the above relations on the derivative still holds.
\end{frame}


\begin{frame}{Particular averaged equaiton}  
  The particular average is then the volume average of $\sum_k = \delta_\alpha q_\alpha$, i.e.
  \begin{equation*}
    \pavg{q}(\textbf{x},t)
    = \avg{\sum_\alpha \delta_\alpha q_\alpha} (\textbf{x},t)
    = \int_V g(\textbf{x},\textbf{y}) \sum_\alpha \delta_\alpha(\textbf{y}- \textbf{y}_\alpha) q_\alpha(t) dV 
    =  \sum_\alpha g(\textbf{x},\textbf{y}_\alpha) q_\alpha(t).
\end{equation*}
The averaged Lagrangian balance for an arbitrary quantity $q_k$, yields, 
\begin{equation}
  \pddt   \left(\pavg{q_\alpha}\right)
  + \nablab \cdot \left(\pavg{q_\alpha \textbf{u}_\alpha}\right) 
  = \pavg{\int_{V_\alpha} \textbf{S}_k dV}
  + \pavg{\int_{S_\alpha} \left[\bm{\Phi} + f (\textbf{u}_I-\textbf{u}) \right] \cdot \textbf{n}_k d S}
  \label{eq:avg_p_global}
\end{equation}
\end{frame}

\begin{frame}
  {Application to mass and momentum equaitons}
  Setting $f_k = \rho_k$ we obtain the mass conservation equation :
  \begin{equation}
    \pddt   \left(\pavg{m_\alpha}\right)
    + \nablab \cdot \left(\pavg{m_\alpha \textbf{u}_\alpha}\right) 
    = 
     \pavg{\int_{S_\alpha} M_k d S}
    \label{eq:avg_p_global}
\end{equation}

Similarly, if $f_k = \rho_k \textbf{u}_k$ we obtain the momentum conservation equation :
\begin{equation}
    \pddt   \left(\pavg{\textbf{p}_\alpha}\right)
    + \nablab \cdot \left(\pavg{\textbf{p}_\alpha \textbf{u}_\alpha}\right) 
    = \pavg{\int_{V_\alpha} \textbf{b}_k dV}
    + \pavg{\int_{S_\alpha} \left[\textbf{T}_k + \rho_k \textbf{u}_k (\textbf{u}_I-\textbf{u}_k) \right] \cdot \textbf{n}_k d S}
    \label{eq:avg_p_global}
\end{equation}
\begin{itemize}
  \item $m_\alpha = \int_{V_\alpha} \rho_k dV$ is the mass of the particle $\alpha$.
  \item $\textbf{p}_\alpha = \int_{V_\alpha} \rho_k \textbf{u}_k dV$ is the momentum of the particle $\alpha$.
\end{itemize}
\end{frame}
  
\begin{frame}
  {Application to the diploe of mass and momentum}
  The transport of the dipole of mass is obtained by setting $f_k = \rho_k \textbf{rr}$ :
  \begin{equation*}
    \pddt   \left(\pavg{\mathcal{G}_\alpha}\right)
    + \nablab \cdot \left(\pavg{\mathcal{G}_\alpha \textbf{u}_\alpha}\right) 
    = 2 \pavg{\mathcal{S}_\alpha}
    + \pavg{\int_{S_\alpha} \textbf{rr} M_k dS},
\end{equation*}
  Lastly, for $f_k = \rho_k \textbf{r} \textbf{u}_k$ we get the moment of momentum balance :
  \begin{multline}
    \pddt   \left(\pavg{\mathcal{P}_\alpha}\right)
    + \nablab \cdot \left(\pavg{\mathcal{P}_\alpha \textbf{u}_\alpha}\right) 
    = \pavg{\int_{V_\alpha} \left( 
        \textbf{r} \textbf{b}_k 
        - \textbf{T}_k
        + \rho_k \textbf{w}_k  \textbf{w}_k 
    \right) dV}\\
    + \pavg{
    \int_{S_\alpha} \textbf{r} \left[
        \textbf{T}_k \cdot \textbf{n}_k
        + \textbf{w}_k M_k
    \right] dS}.
\end{multline}
\begin{itemize}
  \item $\mathcal{G}_\alpha = \int_{V_\alpha} \rho_k\textbf{rr} dV$ is the dipole of mass of the particle $\alpha$.
  \item $\mathcal{P}_\alpha = \int_{V_\alpha} \rho_k \textbf{r}\textbf{u}_k dV$ is the moment of momentum of the particle $\alpha$.
  \item $2\mathcal{S}_\alpha = \int_{V_\alpha} \rho_k (\textbf{r}\textbf{u}_k+\textbf{u}_k\textbf{r}) dV$ is the symmetric part of $\mathcal{P}$. 
  \item $2\mathcal{A}_\alpha = \int_{V_\alpha} \rho_k (\textbf{r}\textbf{u}_k-\textbf{u}_k\textbf{r}) dV$ is the antisymmetric part of $\mathcal{P}$, which is also the angular momentum. 
\end{itemize}
\end{frame}

\section{Equivalence between particular and continuous average.}


\begin{frame}{What is the link between both formalism ?}
  \textbf{Continuous average} :
  \begin{equation*}
      \pddt (\phi_k\kavg{f})
      = \nablab \cdot \left(
          \phi_k \kavg{\bm{\Phi} - f \textbf{u}}
      \right)
      + \phi_k \kavg{\textbf{S}}
      + a_I \Iavg{
          \bm{\Phi}_k \cdot \textbf{n}_k
          + f_k 
          \left(
              \textbf{u}_I
              - \textbf{u}_k
          \right) \cdot \textbf{n}_k
      },
      \label{eq:avg_k_global}
  \end{equation*}

  \textbf{Particular average :}
  \begin{equation}
    \pddt   \left(\pavg{q_\alpha}\right)
    + \nablab \cdot \left(\pavg{q_\alpha \textbf{u}_\alpha}\right) 
    = \pavg{\int_{V_\alpha} \textbf{S}_k dV}
    + \pavg{\int_{S_\alpha} \left[\bm{\Phi} + f (\textbf{u}_I-\textbf{u}) \right] \cdot \textbf{n}_k d S}
    \label{eq:avg_p_global}
\end{equation}
\begin{itemize}
  \item 
  Notice the presence of the non-convective term in the continuous average, which is the only term non-present in the particular averaged equaiton. 
\end{itemize}
\end{frame}


\begin{frame}{Taylor expansion of the continuous quantities}
Expansion of the weighed function around any center of mass $\textbf{y}_\alpha$,
\begin{equation}
    g(\textbf{x},\textbf{y})
    = g_\alpha(\textbf{x},\textbf{y}_\alpha)
    - \textbf{r} \cdot \nablab g(\textbf{x},\textbf{y})|_{\textbf{y}_\alpha}
    + \frac{1}{2} \textbf{r}\textbf{r} : \nablab\nablab g(\textbf{x},\textbf{y})|_{\textbf{y}_\alpha}
    + \ldots
    \label{eq:g_exp}
\end{equation} 
Therefore,
\begin{align*}
  \phi_k \kavg{f_k} = \sum_\alpha \int_{V_\alpha} g f_k dV 
  &=  \pavg{q_\alpha}        
      - \nablab \cdot  \left
      (\pavg{Q_k}\right)        
      + \frac{1}{2} \nablab\nablab : \left(\pavg{\textbf{Q}_k^2}\right)
      + \ldots  \\
  a_I \Iavg{f_k} = \sum_\alpha \int_{S_\alpha} g f_k dS 
  &=  \pavg{q_\alpha}        
      - \nablab \cdot  \left(\pavg{Q_k}\right)        
      + \frac{1}{2} \nablab\nablab : \left(\pavg{\textbf{Q}_k^2}\right)
      + \ldots,
\end{align*}  
where, $Q_k = \int_{V_\alpha} \textbf{r} f_k dV$, and $Q_k = \int_{V_\alpha} \textbf{r} \textbf{r} f_k dV$  are the higher moments of $f_k$.
\begin{itemize}
  \item To analyzes the similarities between both formalism we carry out the expansion of the continuous averaged equation into particular average quantities. 
\end{itemize}
\end{frame}

\begin{frame}  {Expansion of the non-convective term}
  The $n^{th}$ order expansion of the divergence of the non-convective term yields:
  \begin{equation}
    \nablab\cdot
    (\phi_k \kavg{\bm{\Phi}})=
    \sum_l^\infty
    \left[
        \frac{(-1)^{l}}{l!}
        \nablab^{n+1}
        \sum_{\alpha}
        g_{\alpha}
        \int_{V_\alpha}
        \prod^{l}_{m=1}
        r_{i_m} \bm{\Phi}dV
    \right],
\end{equation}
With,
\begin{equation}
    (l+1) \prod^{l}_{m=1} r_{i_m} \bm{\Phi}
    =\nablabh \cdot \left(\prod^{l+1}_{m=1} r_{i_m} \bm{\Phi}\right)
    - \prod^{l+1}_{m=1} r_{i_m} \left(
        \pddt f 
        + \nablabh \cdot (f \textbf{u})
        - \textbf{S}
    \right).
\end{equation}
In agreement with \citet{nott2011suspension}Nott and \citet{prosperetti2004average}prosperetti2004.
\end{frame}

\begin{frame}
  {Expansion of the non-convective term}
  After the use of the divergence theorem fo rthe first term and the Reynolds transport theorem for the second term we get :
  \begin{align*}
    \nablab \cdot
    (\phi_k \kavg{\bm{\Phi}})
    & =\sum_l^\infty
    \frac{(-1)^{n}}{(n+1)!}
    \nablab^{n+1}
    \cdot
    \sum_{\alpha}
    g_{\alpha} \\
  &\left[
    \int_{S_\alpha}
    \textbf{r}^{n+1}
    (\bm{\Phi}_k \cdot \textbf{n}_k) dS
    \right.
      +\int_{V_\alpha}
      \textbf{r}^{n+1}
      \textbf{S}_k dV \rightarrow \text{    Moments of $\textbf{S}_k$ and $\bm{\Phi}_k$.}\\
      &- \pddt
      \int_{V_\alpha}
      \textbf{r}^{n+1}  f_k dV \rightarrow \text{Time derivative moments of $f_k$.}\\
      &
        + \int_{S_\alpha} 
          f_k\textbf{r}^{n+1} 
          \left(\textbf{u}_I - \textbf{u}\right) \cdot \textbf{n}dS
          \rightarrow \text{Moments of the transfer term.}\\
      &- \nablab \cdot\left(
        \textbf{u}_\alpha 
        \int_{V_\alpha}
        \textbf{r}^{n+1}  f_k dV
      \right)
      \left.+\frac{1}{l+1}\int_{V_\alpha}
      f\sum_{e=1}^{m=1} 
      \prod^{l+1}_{\substack{m=1\\ m\neq e}} 
      r_{i_m} 
      w_{i_e}
      dV\right]\\
  \end{align*}
\end{frame}

\begin{frame}
  {Expansion of the advection term}
  The expansion of the divergence of the advection term reads as, 
  \begin{align*}
    \nablab \cdot \phi_k \kavg{\textbf{u} f}
    &= \sum_{l=0}^\infty  
    \frac{(-1)^l}{l!} 
    \nablab^{n+1} \cdot
    \sum_\alpha  g_\alpha 
    \left[
      \textbf{u}_\alpha  \int_{V_\alpha} \textbf{r}^{n} f dV
    + \int_{V_\alpha} \prod^l_{m=1}r_{i_m} \textbf{w} f dV
    \right]
    \label{ap:eq:partial_uf}
\end{align*}
where, $\textbf{u} = \textbf{u}_\alpha + \textbf{w}$.
Where the first term on the RHS is equivalent to the last term of the expansion of $\bm{\Phi}_k$.
\end{frame}

\begin{frame}
  {Difference of the $\bm{\Phi}$ and $\textbf{u}f$ series}
If we carry out the differences of the remaining moments we are left with the series :
\begin{equation*}    
  \sum_{l=0}^\infty  
  \left[
      \frac{(-1)^l}{l!} \prod^{l+1}_{m=1}\partial_{i_m}
      \sum_\alpha  g_\alpha 
      \int_{V_\alpha} f
      \underbrace{\left(
          \prod^l_{m=1}r_{i_m} w_{i_{l+1}} 
          -
          \underbrace{\frac{1}{l+1}
          \sum_{e=1}^{m=1} 
          \prod^{l+1}_{\substack{m=1\\ m\neq e}} 
          r_{i_m} 
          w_{i_e}}_{\text{Symmetric part}}
      \right)}_{Anti symmetric tensor}
      dV
  \right] = 0
  \label{ap:eq:diff_rw_term}
\end{equation*}  
\begin{itemize}
  \item As this term is antisymmetric over the indices $i_1i_2\ldots i_{l+1}$ the tensor vanish under the operator $\partialp{1}{l+1}$
\end{itemize}
\end{frame}

\begin{frame}{Equivalence between both averaging methods}
  \textbf{Continuous average} :
  \begin{equation*}
      \pddt (\phi_k\kavg{f})
      = \nablab \cdot \left(
          \phi_k \kavg{\bm{\Phi} - f \textbf{u}}
      \right)
      + \phi_k \kavg{\textbf{S}}
      + a_I \Iavg{
          \bm{\Phi}_k \cdot \textbf{n}_k
          + f_k 
          \left(
              \textbf{u}_I
              - \textbf{u}_k
          \right) \cdot \textbf{n}_k
      },
      \label{eq:avg_k_global}
  \end{equation*}

  \begin{center}
    \begin{equation*}
      \Longleftrightarrow 
    \end{equation*}
  \end{center}
  \textbf{Particular average :}
  \begin{equation*}
    \pddt   \left(\pavg{q_\alpha}\right)
    + \nablab \cdot \left(\pavg{q_\alpha \textbf{u}_\alpha}\right) 
    = \pavg{\int_{V_\alpha} \textbf{S}_k dV}
    + \pavg{\int_{S_\alpha} \left[\bm{\Phi} + f (\textbf{u}_I-\textbf{u}) \right] \cdot \textbf{n}_k d S}
    \label{eq:avg_p_global}
\end{equation*}
\begin{itemize}
  \item Those two equations, are therefore the representation of the \textbf{same} physical problem.
  \item The second equation furnish us with particular quantities, where the first one provides us phase averaged quantities. 
\end{itemize}
\end{frame}

\backmatter
\end{document}





\section{Perturbed solution of the Hybrid-model}

Remark that at this order in accuracy $\mathcal{O}(Ca^0)$ the linear momentum equations are not coupled with the dispersed phase moments ($\textbf{S}_p,\bm\mu_p$, and $\textbf{M}_p$).  
Consequently, in this first approach \ref{eq:dt_hybrid_Sp,eq:dt_hybrid_Mp,eq:dt_hybrid_mup} are not needed.
Hence, one can simply inject \ref{eq:drag_forces} to \ref{eq:mean_contributions}, into  \ref{eq:dt_hybrid_up} and \ref{eq:dt_uf} to obtain a closed form of the hybrid model, namely,  
\begin{align}
    \label{eq:first}
    \phi_f + \phi &= 1\\
    \textbf{u}_f &= \textbf{u}  + \phi \textbf{u}_r/\phi_f \approx \textbf{u} + \phi \textbf{u}_r\\
    \div \textbf{u} &= 0 \\
    (\pddt + \textbf{u} \cdot \grad)\phi
    &= \div (\phi\textbf{u}_r)\\
    \rho_d \phi (\pddt + \textbf{u}_p \cdot \grad)\textbf{u}_p
    &=
    \phi(\div \bm\Sigma
    + \rho_d  \textbf{g})
    + \div \bm\sigma_p^\text{eff}
    + \textbf{F}
    \\
    \phi_f \rho_f(\pddt + \textbf{u}_f  \cdot \grad) \textbf{u}_f
    % - \div \avg{\chi_f\rho_f \textbf{u}'\textbf{u}'}
    &= \phi_f 
    \left(\div \bm{\Sigma}
    + \rho_f \textbf{g}\right)
    + \div \bm\sigma^\text{eff}
    -\textbf{F}\\
    \label{eq:last}
\end{align}

\begin{align}
    \textbf{F}=&
    \phi
    \frac{\mu_f}{a^2}
    \frac{3(2+3\lambda)}{2(1+\lambda)}\textbf{u}_r
    + \phi\mu_f  \frac{3\lambda}{4(\lambda +1)} \grad^2 \textbf{u}\\
    \bm\sigma_p^\text{eff}
    =&
    -\rho_d C^1_{up}(\phi,\lambda) \textbf{u}_r \textbf{u}_r
    -\rho_d C^2_{up}(\phi,\lambda) (\textbf{u}_r \cdot \textbf{u}_r)\bm\delta\\
    % \bm\sigma_f^\text{eff}
    % =&
    % \bm\sigma_f^\text{eff-1}
    % + \mu_f a^2 \bm\sigma_f^\text{eff-2} \\
    \bm\sigma_f^\text{eff}
    =&
     \mu_f \phi \frac{5\lambda +2}{\lambda+1} \textbf{E}
    + \mu_f \frac{3\lambda}{4(\lambda+1)} [
    \grad(\phi \textbf{u}_r)
    + \grad(\phi \textbf{u}_r)^\dagger]
    - \mu_f \frac{3\lambda - 2}{4(\lambda+1)} \div(\phi \textbf{u}_r)  \bm\delta\nonumber\\
    &-\rho_f C^1_{uu}(\phi,\lambda)  \textbf{u}_r \textbf{u}_r
    -\rho_f C^2_{uu} (\phi,\lambda) (\textbf{u}_r \cdot \textbf{u}_r)\bm\delta
    \label{eq:sigma_feffff}
\end{align}

\subsubsection{We neglect the $O(Re)$ terms}
Or in an other form the total momentum ea,
\begin{align}
    \div \textbf{u}  &= 0\\
    (\pddt + \textbf{u}\cdot \grad) \phi &= \div(\textbf{w})\\
    0 &= 
    - \grad p_f 
    + \mu_f \grad^2\textbf{u}
    + (\rho_f +\phi \rho)\textbf{g}
    + \div (\bm\sigma^\text{eff}) \\
    % 0 &= 
    % - \grad^2 p_f  
    % + \rho \textbf{g}\cdot \grad \phi 
    % + \grad\grad : \bm\sigma^\text{eff}\\
    \bm\sigma^\text{eff}
    &=
     \mu_f \phi \frac{5\lambda +2}{\lambda+1} (\grad\textbf{u}+^\dagger \grad \textbf{u})
    + \mu_f \frac{3\lambda}{4(\lambda+1)} 
    \grad\textbf{w}
    + \mu_f \frac{ 2}{4(\lambda+1)} (\div\textbf{w} ) \bm\delta\\
    \textbf{w} &= - \phi \rho \frac{a^22(\lambda+1)}{\mu_f 3(3\lambda+2)}\textbf{g} - \frac{a^2\lambda}{2(3\lambda+2)} \phi \grad^2 \textbf{u} \\
\end{align} 
where $\rho = \rho_d - \rho_f$ and , 
\begin{align}
    - \phi \grad^2 p_f = O(\phi^2)\\
    \phi (-\grad p_f+\mu_f \grad^2 \textbf{u} )
    &=
    -\phi \textbf{g}\rho_f
    + O(\phi^2)\\
    \phi (-\grad\grad p_f+\mu_f \grad^2 \grad\textbf{u} )
    &=
    O(\phi^2)\\
    \phi \grad^4 \textbf{u}
    =
    O(\phi^2)
\end{align}  

Let assume a small perturbed state aroud ($\phi_0 = cst$ and $\textbf{u}_0 = 0$ ) such that, 
\begin{align*}
    \phi &= \phi_0 + \varphi\\
    \textbf{u} &= \textbf{u}_0 + \textbf{v}\\
    p &= p_0 + q\\
    \textbf{w} &= \textbf{w}_0 + \textbf{w}_1\\
    \bm\sigma^\text{eff} &= \bm\sigma^\text{eff}_0 + \bm\sigma^\text{eff}_1\\
\end{align*}
Hence the equilibrium varibale follows, 
\begin{align*}
    \textbf{u}_0\cdot \grad \phi_0  &= \div\textbf{w}_0\\
    0 &= 
    - \grad p_0 
    + (\rho_f +\phi_0 \rho)\textbf{g}
    + \div (\bm\sigma^\text{eff}_0) \\
    0 &= 
    - \grad^2 p_0  
    + \grad\grad : \bm\sigma^\text{eff}_0\\
    \bm\sigma^\text{eff}_0
    &=
    + \mu_f \frac{3\lambda}{4(\lambda+1)} 
    \grad\textbf{w}_0
    + \mu_f \frac{ 2}{4(\lambda+1)} (\div\textbf{w}_0)  \bm\delta\\
    \textbf{w}_0 &= - \phi_0 \rho \frac{a^22(\lambda+1)}{\mu_f 3(3\lambda+2)}\textbf{g}\\
\end{align*} 
Including the perturbed sol neglecting the non-linear termes and removing the stable eq gives
\begin{align*}
    \pddt \varphi &= \div \textbf{w}_1\\
    0 &= 
    - \grad  q
    + \mu_f \grad^2\textbf{v}
    + \varphi \rho \textbf{g}
    + \div \bm\sigma^\text{eff}_1 \\
    0 &= 
    - \grad^2 q
    + \rho \textbf{g}\cdot \grad \varphi 
    + \grad\grad : \bm\sigma^\text{eff}_1\\
    \bm\sigma^\text{eff}_1
    &=
     \mu_f \phi_0 \frac{5\lambda +2}{\lambda+1} (\grad\textbf{v}+^\dagger \grad \textbf{v})
    + \mu_f \frac{3\lambda}{4(\lambda+1)} 
    \grad  \textbf{w}_1
    + \mu_f \frac{ 2}{4(\lambda+1)} \div  \textbf{w}_1  \bm\delta\\
    \textbf{w}_1 &= -  \varphi \rho \frac{a^22(\lambda+1)}{\mu_f 3(3\lambda+2)}\textbf{g} - \frac{a^2\lambda}{2(3\lambda+2)} \phi_0 \grad^2 \textbf{v} 
\end{align*} 
of course we recall that $\div \textbf{v}= 0$. 
Hence the last two equation may be injected in the first three one to get three eq, for $\varphi, q, \textbf{v}$ namely, 
\begin{align*}
    \pddt \varphi &= - \rho \frac{a^22(\lambda+1)}{\mu_f 3(3\lambda+2)}\textbf{g}\cdot \grad \varphi \\
    0 &= 
    - \grad  q
    + \mu_f (1+\phi_0 \frac{5\lambda +2}{\lambda+1}) \grad^2\textbf{v}
    + \varphi \rho \textbf{g}
    -  \frac{a^2}{ 6(3\lambda+2)} \rho \textbf{g}\cdot (
    + 3\lambda \bm\delta \grad^2   
    +  2 \grad\grad  
    )\varphi\\
    0 &= 
    - \grad^2 q
    + \rho \textbf{g} \cdot(1 - \frac{a^2}{6}\grad^2) \grad \varphi \\
    \bm\sigma^\text{eff}_1
    &=
     \mu_f \phi_0 \frac{5\lambda +2}{\lambda+1} (\grad\textbf{v}+^\dagger \grad \textbf{v})\\
    &+ \mu_f \frac{3\lambda}{4(\lambda+1)}  (
    - \rho \frac{a^22(\lambda+1)}{\mu_f 3(3\lambda+2)}\textbf{g} \grad\varphi  
    - \frac{a^2\lambda}{2(3\lambda+2)} \phi_0 \grad^2 \grad\textbf{v} 
    )\\
    &+ \mu_f \frac{ 2}{4(\lambda+1)} (
        -   \rho \frac{a^22(\lambda+1)}{\mu_f 3(3\lambda+2)}\textbf{g} \cdot \grad\varphi  
    ) \bm\delta\\
    \textbf{w}_1 &= 
    -  \varphi \rho \frac{a^22(\lambda+1)}{\mu_f 3(3\lambda+2)}\textbf{g} 
    - \frac{a^2\lambda}{2(3\lambda+2)} \phi_0 \grad^2 \textbf{v} 
\end{align*} 

\paragraph*{Using fourier transform}
Let now consider the fourier transform, 
\begin{equation}
    \textbf{A}(\textbf{k},\omega)
    =
    \iiiint
    \textbf{A}(\textbf{x},t)
    e^{-i(\textbf{k}\cdot \textbf{x} + \omega t)}
    d\textbf{k}d\omega
\end{equation}
So the the above system of equaiton reads, 
\begin{align}
    \omega \varphi &= - \rho \frac{a^22(\lambda+1)}{\mu_f 3(3\lambda+2)}\textbf{g}\cdot \textbf{k} \varphi \\
    0 &= 
    i \textbf{k}  q
    - \mu_f (1+\phi_0 \frac{5\lambda +2}{\lambda+1}) k^2\textbf{v}
    + \varphi \rho \textbf{g}
    -  \frac{a^2}{ 6(3\lambda+2)} \rho \textbf{g}\cdot (
    - 3\lambda \bm\delta k^2   
    -  2 \textbf{kk}
    )\varphi\\
    0 &= 
    + k^2 q
    - \rho \textbf{g} \cdot(1 + \frac{a^2}{6}k^2) i\textbf{k} \varphi 
\end{align}
Or in matrix form, 
\begin{equation}
    \omega
    \begin{pmatrix}
        \varphi\\
        0\\
        0
    \end{pmatrix}
    =
    \begin{pmatrix}
        - \rho \frac{a^22(\lambda+1)}{\mu_f 3(3\lambda+2)}\textbf{g}\cdot \textbf{k}
        & 0 & 0 \\
        \rho \textbf{g}
        + \frac{a^2}{ 6(3\lambda+2)} \rho \textbf{g}\cdot (
        3\lambda \bm\delta k^2   
        +  2 \textbf{kk}
        )
        &
        - \mu_f (1+\phi_0 \frac{5\lambda +2}{\lambda+1}) k^2
        &
        i \textbf{k}  \\
        - \rho \textbf{g} \cdot(1 + \frac{a^2}{6}k^2) i\textbf{k}  
        &
        0
        &k^2 
    \end{pmatrix}
    \cdot 
    \begin{pmatrix}
        \varphi\\    
        \textbf{v}\\    
        q\\    
    \end{pmatrix}
\end{equation}

\section{A simple example : 1D sedimentation}
We consider the 1D sedimentation of spherical droplet in a fluid at rest although or model can be generalized to cases with non-zero bulk velocity.
For now we neglect the effect of bulk and micro scale inertia. 
Although in 1D the bulk inertia vanishes (but the unsteady terms remain), while we may incorporate the incluence of finite inertia on the effective stress (TO DO).
The above equation of motion reduces to,
\begin{align}
    \phi _f + \phi &= 1\\
    \phi _f u_f &= -\phi u_p  = \phi u_r\quad \text{or} \quad  u = 0 \\
\pddt \phi +\partial _z(u_p\phi)&=0  \\
0&= \phi \partial _z \Sigma _{zz}
- \phi \rho _d g
- \phi\frac{\mu_f}{a^2}
\frac{3(2+3\lambda)}{2(1+\lambda)}u_p\\
0    &= \phi_f\partial _z \Sigma _{zz}
- \phi_f\rho_f  g
+ \mu_f \frac{3\lambda - 2}{4(\lambda+1)} \partial _{zz}(\phi u_p)
+ \phi\frac{\mu_f}{a^2}
\frac{3(2+3\lambda)}{2(1+\lambda)}u_p
\end{align}
From which we immediately evidenced the effect of the second order force moment : spatial diffusion of the volume fraction and of the relative velocity.
However if we inject this term in mass conservation equation we obtain a dispersion term !
Moreover in constrast to the classical Richardson-Zaki  scaling (or in our dilute configuration the Hadamard force), the relative velocity is not uniform in space.
The previous system of equaitions lead to,
\begin{align}
\pddt \phi +\partial _z(u_p\phi)&=0  \\
0    &= 
 (1-\phi)g (\rho _d -\rho_f)
+ \mu_f \frac{3\lambda - 2}{4(\lambda+1)} \partial _{zz}(\phi u_p)
+ \frac{\mu_f}{a^2}
\frac{3(2+3\lambda)}{2(1+\lambda)}u_p
\end{align}
which leads to a more complex equations than the traditional continuity wave equation.



\section{TO DO : perform the stability analysis \textit{a la Jackson} considering inertia}
Do the dilute fluidisation of droplet is stable ?
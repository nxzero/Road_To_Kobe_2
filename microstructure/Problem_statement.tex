\subsection{Problem statement}

We investigate numerically the dynamics of homogeneous mono-disperse emulsions subject to buoyancy forces in a fully periodic domain. 
Both the dispersed and continuous phase are considered as Newtonian fluids defined by viscosity $\mu_d$ (resp. $\mu_f$) and density $\rho_d$ (resp. $\mu_f$).
Throughout this work, the subscript $_d$ and $_f$ indicate properties belonging to the dispersed and continuous phase respectively. 
The interface between both fluids is considered as infinitely thin and deprived of any impurities and a constant surface tension with a coefficient $\gamma$ is assumed. 
The density and viscosity will be considered constant in each phase.
In dimensionless form this problem is completely characterized by six dimensionless parameters :  the viscosity and density ratio, $\lambda = \mu_d / \mu_f$ and $\zeta = \rho_d / \rho_f$,  
the \textit{Galileo} number, 
\begin{equation*}
    Ga =\sqrt{\rho_f(\rho_f - \rho_d) g d^3} / \mu_f,
\end{equation*}
the \textit{Bond} number, 
\begin{equation*}
    Bo =\frac{(\rho_f - \rho_d) g d^2}{\gamma},
\end{equation*}
the number of droplets per domain $N_b$, and the dispersed phase volume fraction $\phi$. 
Here, $d$ represents the diameter of a sphere with the same volume as the droplets, and $g$ denotes the gravitational acceleration.
The \textit{Galileo} number is a measure of the strength of the buoyancy forces relative to the viscous forces, whereas the \textit{Bond} number evaluates the ratio between buoyancy and capillary forces. 

To provide a brief overview of the range of interest for these numbers in an industrial context, let us consider the example of a vegetable oil/water system.
In most liquid-liquid system encountered in industrial processes the diameter of the droplets lies in the range $d = [50 \mu \text{m}, 3 \text{mm}]$.
The density and viscosity of water are approximately $\rho_f = 1000 \text{kg/m}^3$ and $\mu_f = 10^{-3} \text{Pa.s}$, respectively.
The density and viscosity of oil are close to $\rho_d = 900 \text{kg/m}^3$ and $\mu_d = 10^{-2} \text{Pa.s}$, respectively.
We consider the gravitational acceleration on earth, thus $g= 9.81 \text{m.s}^{-2}$.
The surface tension of the oil/water system is approximately $\sigma = 0.05 \text{N/m}^2$ \citep{de2015gouttes}. 
\begin{table}[h!]
    \centering
    \caption{Dimensionless parameters of a water/oil system.}
    \begin{tabular}{|c||c|c|c|c|c|}
        \hline&$Ga$&$Bo$&$\phi$&$\lambda$&$\zeta$\\ \hline
        \hline Oil/Water&$[0.35,160]$&$[10^{-5};10^{-1}]$&$<0.2$&$10$&$0.9$\\ \hline
    \end{tabular}
    \label{tab:parameters_exp}
\end{table}
\ref{tab:parameters_exp} gives the corresponding dimensionless parameters.  
Notice that the \textit{Bond number} is relatively low, indicating that the droplets are nearly spherical in these processes.
Additionally, the maximum volume fraction is set to $\phi = 0.2$, indeed above such $\phi$ particles coalesce easily and the topology of the flow cannot be considered as dispersed anymore. 


Following \ref{tab:parameters_exp}, to approach real-life applications we conducted DNS for four volume fractions, specifically $\phi = 0.01,0.05,0.1,0.2$.
In contrast to most of the previous studies, we choose to keep the number of droplets constant while changing the volume fraction $\phi$. 
We then modify the domain size $\mathcal{L}$ accordingly. 
This introduces another dimensionless parameter of interest : $\mathcal{L}/d$, which measures the confinement of the particles within the finite numerical domain. 
This parameter is purely determined by $\phi$ and $N_b$, and will thus be refereed as a \textit{secondary parameter}.

As mentioned, the \textit{Bond} numbers of our targeted application is very low.
Therefore, and it will stay constant throughout this study, the \textit{Bond} number is set to $Bo = 0.2$.
DNS with lower \textit{Bond} numbers become excessively expensive due to the restrictive capillary time step constraint. 
However, we assert that for $Bo \leq 0.2$ the droplet shape essentially remains spherical at least for small \textit{Galileo} numbers. 
Additionally, the ratio between inertia and surface tension forces is given by the \textit{Weber} number, 
\begin{equation*}
    We = \frac{Bo \cdot Re^2}{Ga},
\end{equation*}
where $Re = \frac{\rho_f d U}{\mu_f}$ is the Reynolds number based on the phase drift velocity $U$.
Values of \textit{Reynolds} numbers for each DNS are provided in \ref{ap:age} \ref{fig:Reall}. 
Extreme values of $We$ reached in these simulations are displayed in \ref{tab:simulations}. 
It is clear that for $We=0.6$ we might expect some deformations, nevertheless, for most of the cases $We$ stays below these values. 
Consequently, whether it is in the viscous regime or inertial regime, the droplets are expected to remain spherical according to the values of $Bo$ and $We$.
This statement will be verified in \ref{sec:microstructure}. 

Density and viscosity ratio of droplets in real life applications are reported in \citet[Figure 1.]{balla2020effect}.
As depicted in \citet[Figure 1.]{balla2020effect}, the viscosity and density ratio of fluid-fluid systems range between, $\lambda \in [10^{-4} : 10^4]$ and $\zeta \in [10^{-1} : 10^1]$, respectively. 
In this study we restrict our attention to a single density ratio, $\zeta = 0.9$.
Regarding the viscosity ratio, we accomplished DNS for 2 different values, namely $\lambda = 1,10$.
Lastly, to explore the effect of inertia on the microstructure, the \textit{Galileo} number will vary within the range $Ga \in [5,100]$.

The primary objective of the study is to investigate the microstructure through the nearest particle pair distribution function.
Thus it is crucial to obtain a sufficient number of DNS samples to ensure statistical convergence. 
Also, the physical quantities measured in these DNS must remain independent of the domain size. 
Therefore, we use a number of particles per domain of $N_b = 125$, which is roughly what \citet{hidman2023assessing} used for their DNS of fully-periodic buoyant rising bubbles.
Moreover, each DNS lasts for a time : $t^*_\text{end} = 1500 \sqrt{d/g}$.
% It is shown in \ref{ap:validation} that these parameters  are sufficient to obtain well converged statistics.  
\begin{table}[h!]
    \centering
    \caption{Dimensionless parameter range investigated in this work.}
    \begin{tabular}{|ccccccc|ccc|}\hline
        \multicolumn{7}{|c|}{Primary parameters}&\multicolumn{3}{|c|}{Secondary parameters}\\\hline\hline
        $Ga$&$Bo$&$\phi$&$\lambda$&$\zeta$&$N_b$&$t^*_\text{end}$&$\mathcal{L}/d$&$Re$&$We$\\ \hline
        $5\rightarrow 100$&$0.2$&$1\% \rightarrow 20\%$&$10$ \& $1$&$0.9$&$125$&$1500$&$6.7\to 18.7$&$10^{-1}\to 170$&$10^{-4}\to 0.6$\\ \hline
    \end{tabular}
    \label{tab:simulations}
\end{table}
In this study we present DNS results with dimensionless parameters lying in ranges outlined in \ref{tab:simulations}.
In summary, we investigated $5$ \textit{Galileo} number $Ga = 5,10,25,50,100$, $4$ different volume fractions $\phi = 0.01,0.05,0.1,0.2$, and two viscosity ratios $\lambda =1,10$ with $Bo = 0.2$ and $\zeta = 0.9$. 
This makes a total of $40$ representative simulations of $N_b = 125$ droplets which last for $t= 1500 \sqrt{d/g}$. 



\subsection{Problem statement}
% Objective of this section :
% \begin{itemize}
    % \item Introduce the dimensionless parameters.
    % \item Present the physical parameters of some industrial processes to locate our problematic. 
    % \item Introduce the dimensionless parameters range investigated in this study.
    % \item Present the tri-periodic box within which we add droplets in vof 
% \end{itemize}
We investigate numerically the dynamic of homogeneous mono-disperse emulsion subject to buoyancy forces in a fully periodic domain. 
Both, the dispersed (resp. continuous) phase is considered as Newtonian fluid defined by viscosity $\mu_d$ (resp. $\mu_c$), and density $\rho_d$ (resp. $\mu_c$).
Throughout this work, the indices $d$ and $c$ indicate properties belonging to the dispersed and continuous phase, respectively. 
The interface between both fluid is considered as infinitely thin and deprived of any impurities so that it can only be described with the surface tension coefficient $\gamma$. 
In this work the density, viscosity, and surface tension coefficient, will be considered constant in each phase.
In dimensionless form this problem is completely determined by $6$ dimensionless parameters;  the viscosity and density ratio, $\lambda = \mu_d / \mu_c$ and $\zeta = \rho_d / \rho_c$,  
the \textit{Galileo} number, 
$
    Ga =\sqrt{\rho_c(\rho_c - \rho_d) g d^3} / \mu_c
$
the \textit{Bond} number, 
$
    Bo =\frac{(\rho_c - \rho_d) g d^2}{\gamma}
$
the number of droplets per domain $N_b$ and the dispersed phase volume fraction $\phi$. 
Here, $d$ is the diameter of the sphere of same volume as the droplets and $g$ the gravitational acceleration.
The \textit{Galileo} number measure the influence of the buoyancy forces against the viscous forces.
Whereas the \textit{Bond} number evaluate the ratio between buoyancy and capillary forces. 

To give a brief idea about the range of interest of those numbers in the industrial context, we take the example of a vegetal oil/water system.
In most liquid-liquid system encountered in industrial processes the diameter of the droplets lies in the range $d = [50 \mu \text{m}, 3 \text{mm}]$.
The density and viscosity of water are approximately $\rho_c = 1000 \text{kg/m}^3$ and $\mu_c = 10^{-3} \text{Pa.s}$.
The density and viscosity of oil are close to $\rho_d = 900 \text{kg/m}^3$ and $\mu_d = 10^{-2} \text{Pa.s}$.
We consider the gravity acceleration on earth, $g= 9.81 \text{kg.m.s}^{-2}$.
The surface tension of the system oil/water is known to be approximately $\sigma = 50 \text{mJ/m}^2$ \citep{de2015gouttes}. 
We display the values of dimensionless parameters computed with those physical parameters in \ref{tab:parameters_exp}.
\begin{table}[h!]
    \centering
    \caption{Dimensionless parameters in a water/oil emulsion.}
    \begin{tabular}{|c||c|c|c|c|c|}
        \hline&$Ga$&$Bo$&$\phi$&$\lambda$&$\zeta$\\ \hline
        \hline Oil/Water&$[0.35,160]$&$[10^{-5};10^{-1}]$&$<0.2$&$10$&$0.9$\\ \hline
    \end{tabular}
    \label{tab:parameters_exp}
\end{table}
We can notice that the \textit{Galileo number} is rather low therefore can already state that it will be necessary to investigate the stokes flow limit. 
Likewise, The \textit{Bond number} is very low, meaning that the droplets are nearly spherical.
Additionally, a maximum volume fraction has been set empirically to $0.2$, not because higher volume fraction flow exist, but rather because above this limit particles highly coalesce and the topology of the flow cannot be considered as dispersed anymore. 


To approach real life applications we carried out DNS for four volume fractions, namely $\phi = 0.01,0.05,0.1,0.2$.
In opposition to most of the previous studies, we choose to keep the number of droplets constant while changing the volume fraction $\phi$. 
Instead, we modify the domain size $\mathcal{L}$ accordingly. 
As mentioned the \textit{Bond} numbers of our targeted application is extremely low (see \ref{tab:parameters_exp}).
However, below a certain limit the \textit{Bond} number has no impact on the physics. 
% This threshold value has been fixed empirically at $Bo = 0.2$, below which we observed no modification in the drop dynamic. 
Therefore, and it will stay constant throughout this study, the \textit{Bond} number is set to $Bo = 0.2$.
Lower \textit{Bond} number cases could not be archived due to the capillary time step restriction which becomes excessively costly, nevertheless we consider that under $Bo = 0.2$ the \textit{Bond} number has no notable impact since the droplet shape remain essentially spherical. 
Density and viscosity ratio of droplets in real life applications are reported in \citet[Figure 1.]{balla2020effect}.
It is clear from their \textit{Figure 1}, that the viscosity and density ratio of fluid-fluid systems, range between the intervals : $\lambda = [10^{-4} : 10^4]$ and $\zeta = [10^{-1} : 10^1]$, respectively. 
In this study we restrict our attention to a single density ratio of $\zeta = 0.9$, corresponding to a water-oil emulsion.
However, regarding the viscosity ratio, we accomplished DNS for 3 different values, namely $\lambda = 0.1,1,0.1$.
The \textit{Galileo} number will range from $Ga = 1$ to $Ga = 100$ to study the influence of inertial effects.

As the purpose of the study is to investigate particle pair distribution function it is primordial to obtain a representative number of DNS samples. 
Therefore, we use a number of particles per domain of $N_b = 125$, which roughly what \citet{hidman2023assessing} uses for tri-periodic buoyant rising bubbly flow. 
Additionally, each DNS lasts for a time : $t^*_\text{end} = 500 \sqrt{d/g}$, which is sufficient to obtain well converged statistics (see \ref{ap:validation}).  
\begin{table}[h!]
    \centering
    \caption{Dimensionless parameter range investigated in this work.}
    \begin{tabular}{ccccccc}\hline
        $Ga$&$Bo$&$\phi$&$\lambda$&$\zeta$&$N_b$&$t^*_\text{end}$\\ \hline\hline
        $1\rightarrow 100$&$0.2$&$1\% \rightarrow 20\%$&$10$ \& $1$&$0.9$&$125$&$500$\\ \hline
    \end{tabular}
    \label{tab:simulations}
\end{table}
In this study we present DNS with dimensionless parameters lying in ranges display \ref{tab:parameters}.
In summary, we investigated $6$ \textit{Galileo} number $Ga = 1,5,10,25,50,100$, $4$ different volume fractions $\phi = 0.01,0.05,0.1,0.2$, and two viscosity ratios $\lambda =1,10$ with $Bo = 0.2$ and $\zeta = 0.9$. 
This makes a total of $60$ representative simulations of $N_b = 125$ droplets. 
In the next sections we develop our numerical strategy to make efficient multi-VoF simulations. 








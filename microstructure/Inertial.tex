\subsubsection*{High inertia regimes }
We now turn our attention to the high inertia regimes ($Ga =100$).
In this situation it is expected that the presence of particle wakes modify the interactions between particles \citep{yin2006}. 
\begin{figure}[h!]
    \centering
    \includegraphics[height=0.21\textwidth]{image/HOMOGENEOUS_NEW/Dist/Pnst_l_10_Ga_100_PHI_0_05.pdf}
    \includegraphics[height=0.21\textwidth]{image/HOMOGENEOUS_NEW/Dist/Pnst_l_1_Ga_100_PHI_0_05.pdf}
    \includegraphics[height=0.21\textwidth]{image/HOMOGENEOUS_NEW/Dist/Pnst_l_10_Ga_100_PHI_0_2.pdf}
    \includegraphics[height=0.21\textwidth]{image/HOMOGENEOUS_NEW/Dist/Pnst_l_1_Ga_100_PHI_0_2.pdf}
    \caption{Histogram of the normalized function $P_\text{nst}^n$ at high inertia $Ga = 100$.
    The color map represents the values of the nearest pair distribution function. %of $P_\text{nst}^n$.
    The origin corresponds to the position of the test particle.
    The dimensionless radial and azimuthal coordinates, $|\textbf{r}|/d$ and $\theta$, correspond to the nearest neighbor position.
    The vertical direction corresponds to the flow direction, which is also the axis of symmetry for $P_\text{nst}^n$.
    (left) Low volume fraction cases $\phi=0.05$ for $\lambda = 1,10$.
    (right) High volume fraction cases $\phi=0.2$ for $\lambda = 1,10$.}
    \label{fig:Pnst_high_Ga}
\end{figure}
%If we compare \ref{fig:Pnst_high_Ga} (right) with their counterparts from \ref{fig:Pnst_low_Ga} (right) we observe that $P^n_\text{r}$ becomes even larger at contact of the particles for $Ga=100$.
%Again, this could witness of the presence of clustering of particles. 
%In general, all $P_\text{nst}^n$ from \ref{fig:Pnst_high_Ga} exhibit some differences compared to the cases \ref{fig:Pnst_low_Ga}. 
%In the high inertial cases (\ref{fig:Pnst_high_Ga}), we can notice that $P_\text{nst}^n$ is larger on the sides of the test particle for the iso-viscous emulsions ($\lambda = 1$).
The presence of anisotropy in \ref{fig:Pnst_high_Ga} for $\lambda=1$ is particularly striking compared to \ref{fig:Pnst_low_Ga}. 
In the former a higher concentration of particles is identified at $\theta \approx 0$, as seen in \ref{fig:Pnst_high_Ga}. 
A higher concentration of $P_\text{nst}^n$ around $\theta \approx 0$ indicates the presence of horizontal rafts of particles. 
In this case the microstructure is non-homogeneous and anisotropic, this situation is illustrated in \ref{fig:scheme_clusters} (\textit{Case 3: ``layers''}). 
As the \textit{Galileo} number ($Ga$) increases and for low values of the viscosity ratio ($\lambda$), the probability of having neighbors on the horizontal plane of the test particle increases. 
This leads to an increase in the anisotropy of the microstructure which is more pronounced for low volume fractions. 
The high viscosity drops display an isotropic distribution of nearest particles around the test particle. 
This observation suggests the presence of isotropic clustering of particles.



%In comparison the high viscosity drops show an isotropic discitrbution of nearest particle around the test particle. This could witness of the presence of isotropic clustering of particles.

%We do not observe a significant effect of the volume fraction on the anisotropy of the distribution.
%However, at this stage it remains unclear if increasing $\phi$ have a positive or negative impact on the anisotropy of the distribution. 

To illustrate the impact of $\lambda$ on the microstructure, \ref{fig:images} displays snapshots of two DNS at $\phi = 0.05$ and $Ga = 100$. 
As predicted by $P_\text{nst}^n$, we observe layers and particles in close contact for $\lambda = 1$, contrasting with the seemingly more evenly dispersed microstructure for $\lambda = 10$.
\begin{figure}[h!]
   \centering
   \includegraphics[width=0.4\textwidth]{image/HOMOGENEOUS_NEW/P_PHI_5_l_10_Ga_100.png}
   \includegraphics[width=0.4\textwidth]{image/HOMOGENEOUS_NEW/P_PHI_5_l_1_Ga_100.png}
   \caption{Snapshot of a simulation at $t^* = 150$ for $\phi=0.05$ and $Ga=100$.
   Color map : values of the vertical component of the velocity, field on the vertical plane defined by the equation $z=0$. 
   (left)  $\lambda = 1$.
   (right)  $\lambda = 10$.
   }
   \label{fig:images}
\end{figure}
In fact, for $\lambda = 10$, in \ref{fig:images} (right), we can still observe horizontal rafts of droplets or droplets rising side-by-side, but this effect is clearly not as pronounced as for $\lambda = 1$. 
Clearly, drops with high viscosity ratio maintain a significant distance between other drops, which prevents the creation of structures such as droplets layers.
% This might be because of a higher vorticity around the viscous droplets.   
% As discussed in \citet{zhang2021three}, rising pairs of spherical bubbles may reach a stable side-by-side configuration, which tends to generate horizontal clusters.
% They range of dimensionless parameters is consistent with the ones presented in this study, making this hypothesis valuable for iso-viscous emulsions. 
% In \citet{legendre2003hydrodynamic} they study the interaction of a bubble pair rising side-by-side. 
% They stipulate that for two bubbles at moderate \textit{Reynolds} number $50-100$, the interaction forces are found to be repulsive, while it is attractive or null for higher \textit{Reynolds} number. 
% In our case it is reasonable to think that such pair attraction / repulsion mechanisms might drive the clustering mechanism.
%On another note, we can observe on \ref{fig:images} (left) that the distance between the layers is roughly equal to the length of the numerical domain. 
%Indeed, only one layer of droplets is present in the domain. 
%Therefore, the current microstructure is constrained by the size of the numerical domain, it is probably not representative of the real microstructure that we would obtain in an infinite non-periodic domain. 
%Additionally, one might argue that the layers appear due to collective effects drove by the size of the box.
%Indeed, it is exactly what we observe for small number of bubbles ($N_b = 4$) rising in a periodic domain, see \citet{loisy2017}. 
%However, we might expect that horizontal layers such as the one observed in \ref{fig:images} (left) still remain for lager boxes since the number of droplets is consequent.
%In our case, the presence of horizontal raft might be the consequence of pairwise interactions mechanism, as discussed above. 
%Therefore, it is likely that that layers still appear regardless of the size of the box.
%Nevertheless, the distance between these layers is still constrained by the size of the numerical domain, despite the consequent number of droplets used here. 
%In all rigor, DNS in a larger domain with more particles would be required to evaluate the microstructure dependence on the domain size. 
%Nevertheless, due to evident numerical constrains it has not been performed in this study.  
From the present analysis of $P_\text{nst}^n$ and the actual microstructure presented in \ref{fig:images} we can infer that the \textit{nearest particle statistics} is able to predict features in the microstructure such as layers and clusters. 


\subsubsection*{Nearest particle radial distribution function }

Although \ref{fig:Pnst_high_Ga} and \ref{fig:Pnst_low_Ga} give a good qualitative representation of the particle-pair azimuthal distribution, they fall short in delivering a quantitative depiction of the radial distribution.
For a random isotropic distribution of hard spheres it is possible to derive a theoretical prediction for $P_\text{nst}^n(r)$ obtained in the vanishing volume fraction limit. 
Indeed, it is shown in \citet{zhang2021ensemble} that for a dilute random arrangement of particles $P_\text{nst}^n(r)$, reads as
\begin{equation}
    P_\text{nst}^\text{th}(r) = \exp\left[- 8\phi\left(r_*^3-1\right)\right].
    \label{eq:Pnst_dilute}
\end{equation}
where we introduced the dimensionless radial distance $r_* = r/d$. 
It must be understood that this formula is accurate only at $\mathcal{O}(\phi)$, therefore in most of our cases it is not expected to be representative.
Additionally \citet{torquato1990nearest} derived a radial distribution function for impenetrable hard spheres at arbitrary volume fractions $\phi$. 
In our notation this distribution can be written
\begin{equation}
    P_\text{nst}^\text{tor}(r) = 
        \left(e+\frac{f}{r_*} +\frac{g}{r_*^2}\right)
    \exp\left\{-\phi\left[8e\left(r_*^3-1\right)+12 f\left(r_*^2-1\right)+24g\left(r_*-1\right)\right]\right\}
    \label{eq:torquato}
\end{equation}
with, 
\begin{align*}
    && e= \frac{1+\phi}{(1-\phi)^3},
    && f= \frac{-\phi (3+\phi)}{2(1-\phi)^3},
    && g= \frac{\phi^2}{2(1-\phi)^3}.
\end{align*}
In both cases the hard sphere model differs from the droplet pair distribution.
Indeed, in the latter case $P_\text{nst}^\text{th} = 0$ for $r<d$ while in our case particles might deform at contact, meaning that $P_\text{nst}^\text{th}$ is finite for certain $r<d$. 
However it remains valuable to use these theoretical probability density functions for comparative purposes. 

\begin{figure}[h!]
    \centering
    % \includegraphics[height=0.3\textwidth]{image/HOMOGENEOUS_NEW/Dist/Pr_l_1_Ga_100.pdf}
    % \includegraphics[height=0.3\textwidth]{image/HOMOGENEOUS_NEW/Dist/Pr_l_10_Ga_100.pdf}
    \includegraphics[height=0.3\textwidth]{image/HOMOGENEOUS_NEW/Dist/Pr_l_1_Ga_10.pdf}
    \includegraphics[height=0.3\textwidth]{image/HOMOGENEOUS_NEW/Dist/Pr_l_10_Ga_10.pdf}
    \includegraphics[height=0.3\textwidth]{image/HOMOGENEOUS_NEW/Dist/Pr_l_1_Ga_100.pdf}
    \includegraphics[height=0.3\textwidth]{image/HOMOGENEOUS_NEW/Dist/Pr_l_10_Ga_100.pdf}
    \caption{Radial probability density function $P_\text{nst}^n(r)$ (symbols) and $P_r^\text{torq}$ (dashed lines) divided by the theoretical distribution $P_\text{nst}^{th}$ \ref{eq:Pnst_dilute}, as functions of the dimensionless distance $r/d$, for  $Ga = 100$ and $Ga =10$.
    (left)  $\lambda = 1$.
    (right) $\lambda = 10$.
    ($\pmb\bigcirc$) $\phi = 0.01$; ($\pmb\triangle$) $ \phi = 0.05$; ($\pmb\square$) $\phi = 0.1$ ($\pmb\lozenge$) $\phi = 0.2$.
    (dashed lines) Theoretical prediction : $P_\text{nst}^n/P_\text{nst}^\text{th} = 1$. 
    For $r<d$ we arbitrarily set $P_\text{nst}^\text{th} = 1$ so that the distribution can be visualized.
    }
    \label{fig:Pr}
\end{figure}

In \ref{fig:Pr}  we plot the radial distribution $P_\text{nst}^n(r)$. %averaged on all $\theta$.
We displayed two different viscosity ratios and multiple volume fractions as functions of the dimensionless distance $r/d$. 
In the dilute regime ($\phi = 0.01$) and for $\lambda=1$, we observe on \ref{fig:Pr} (left) that the radial distribution follows the dilute random arrangement of particles predicted by the theory (\ref{eq:Pnst_dilute}). 
In contrast, when $\lambda = 10$ with the same $\phi$, it becomes evident that fewer particles gather in close vicinity to the test particle. Consequently, on average, the droplets are positioned at greater distances from each other compared to the dilute random distribution. 
For most values of $\phi$ except $\phi=0.2$, we observe that the nearest particle distribution is higher at the contact of the particle of reference ($r/d = 1$), for $\lambda = 1$ than for $\lambda = 10$. 
%Although, we selected a small \textit{Bond} number ($Bo = 0.2$), 
It is clear from \ref{fig:Pr} that the particles deform at contact, as witnessed by the non-vanishing value of $P_\text{nst}$ for $r/d<1$.

Lastly, \ref{eq:torquato} brings a clear improvement on the prediction of the radial distribution compared to the random distribution. 
Indeed, we clearly see in the DNS results that the volume fraction of particles is higher at contact, while \ref{eq:torquato} predicts the same behavior. 


In brief, we observed that both the radial and azimuthal distributions were affected by the inertial effects measured by the \textit{Galileo} number. 
The major effect coming with high inertia is the generation of strong anisotropy in the particle pair distribution, as well as a more important concentration of neighboring particles at $r/d < 1$. 
Furthermore, for increasing volume fractions, the density at contact of the test particle is higher. 
Regarding the viscosity ratio, it has a strong impact on the microstructure, but only at high $Ga$, whereas at low $Ga$ the change in viscosity ratio has no notable impact on $P_\text{nst}$. 

\subsubsection*{Macroscopic modeling of the microstructure}
Up to now we have presented visualizations of the microstructure with 2D or 1D distributions. 
For a more concise description we would like to adopt a different approach. 
Following \citet{zhang2023evolution} we opt to describe the microstructure using the second moment of $P_\text{nst}$ with respect to \textbf{r}, which reads
\begin{equation}
    \textbf{R} =\frac{1}{n_p} 
    \int_0^\infty 
    \int_{\mathbb{R}^3} \textbf{rr} P_\text{nst}(\textbf{r}) d\textbf{r}.
    \label{eq:R}
\end{equation}
The tensor $\textbf{R}$ allows us to measure the mean-square distance between a particle and its nearest neighbor in the three dimensions of space.
%This second-rank tensor measures the spread of the nearest neighbor distribution in a given direction. 
It is worth noting that such a quantity is computable only because $\lim_{|\textbf{r}|\to \infty} P_\text{nst}(\textbf{r}) = 0$, which enables the integral of \ref{eq:R} to converge. 
This would not be the case for classical pair distributions, which is the primary reason for using the \textit{nearest particle statistics} framework. 

Since our objective is to measure the anisotropy of the microstructure, we are particularly interested in the deviatoric part of this tensor, namely
\begin{equation*}
    \textbf{A} = \textbf{R} - \frac{1}{3}  \text{tr}(\textbf{R}) \textbf{I}.
\end{equation*}
$\textbf{A}$ represents the likelihood of having a particle in a given direction compared to the mean radial square distance. 
Therefore, in an isotropic suspension, we have $A_{xx} = A_{yy} = 0$, where $y$ represents the coordinate aligned with gravity. 
In a situation like the one depicted in \ref{fig:images} (left), particles are more likely to have closer neighbors on the lateral sides rather than vertically positioned, hence  $A_{yy} < 0$ and $0 < A_{xx}$. 
To facilitate the understanding of the subsequent findings, it is useful to calculate the distance $r_m$. 
This distance represents the average distance to the closest neighboring particle within a random arrangement of dilute hard spheres. 
Through direct integration of equation \ref{eq:Pnst_dilute}, we get
\begin{align}
    r_m^2
    = \frac{1}{n_p}
    \int_{\mathbb{R}^3} r^2 P_\text{nst}^\text{th}(\textbf{r}) d\textbf{r} 
    =  \frac{d^2}{8^{2/3}}\frac{\Gamma\left({{5}\over{3}} , 8\,\phi\right)e^{8\phi}}{\phi^{2/3}},
    % = d^2{{4^{{{2}\over{3}}}\,\Gamma\left({{5}\over{3}} , 8\,\phi\right)
    % \,e^{8\,\phi}}\over{2^{{{10}\over{3}}}\,\phi^{{{2}\over{3}}}}}
\end{align}
where $\Gamma(z,a) = \int_a^\infty t^{z-1} e^{-t} dt$ is the upper incomplete gamma function.
Since the upper incomplete gamma function is not so common we have plotted in \ref{ap:age} \ref{fig:ap:agee} the dimensionless distance $r_m/d$ in terms of $\phi$ to help comprehension. 

\begin{figure}[h!]
    \centering
    \includegraphics[height=0.3\textwidth]{image/HOMOGENEOUS_NEW/PA/trR.pdf}
    \includegraphics[height=0.3\textwidth]{image/HOMOGENEOUS_NEW/PA/Axx.pdf}
    \caption{
        (left) Dimensionless trace of $\textbf{R}$ as a function of the volume fraction.%Trace of the second moment of the probability density function $P_\text{nst}(\textbf{r})$ divided by the square diameter of the particles $d^2$. 
        (right) Horizontal components of the anisotropy tensor divided by the trace of $\textbf{R}$. %the second moment of the probability density function.
    ($\pmb\bigcirc$) $\phi = 0.01$; ($\pmb\triangle$) $ \phi = 0.05$; ($\pmb\square$) $\phi = 0.1$ ($\pmb\lozenge$) $\phi = 0.2$.
    The hollow symbols correspond to $\lambda = 1$, the filled symbols to $\lambda = 10$.
    For $r<d$ we arbitrarily set $P_\text{nst}^\text{th} = 1$ so that the distribution can be visualized.
    Black symbols represent the results of \citet{zhang2023evolution} for sedimenting spherical particles with $\phi = 0.016,0.056,0.134,0.262$ and $\zeta = 2.56$. %$\phi = 0.0168,0.0565,0.1341,0.2622$ 
    Corresponding to $\pmb\times,\pmb +, \pmb\star , \pmb\triangledown$, respectively.
    }
    \label{fig:A}
\end{figure}
In \ref{fig:A} (left), the dimensionless mean square distance between nearest neighbors, represented as $3\cdot\text{tr}(\textbf{R})/r_m^2 = \textbf{R}:\textbf{I}/r_m^2$, is illustrated for all configurations explored in this study. 
The simulation marked by the symbol \textcolor{col1}{$\pmb\circ$} in \ref{fig:A} (left), corresponding to parameters $\lambda = 1$, $Ga = 100$, and $\phi = 0.01$, yields a value of $\textbf{I}:\textbf{R}/r_m^2 \approx 1$. 
This observation agrees with the quasi-hard-sphere distribution previously reported for this particular case in \ref{fig:Pr} (left).
The dependence of the mean square distance on the volume fraction is displayed on \ref{fig:A} (left). 
We observe a decrease in $\textbf{I}:\textbf{R}/r_m^2$ as $\phi$ increases, indicating that particles tend to come closer to each other on average compared to a dilute random distribution of hard spheres. 
This observation, as highlighted by \citet{zhang2023evolution}, suggests the emergence of clusters when increasing the particle volume fraction. 
The dependence on the \textit{Galileo} number is non-monotonic. 
Indeed $\textbf{I}:\textbf{R}/r_m^2$ increases until $Ga = 25$ and then decreases until $Ga = 100$. We also note that the distance to the nearest neighbor is not significantly affected by the viscosity ratio, particularly when dealing with high volume fractions. 
In \ref{fig:A} (left), the symbols $\pmb\star$, $\pmb\times$, $\pmb +$, and $\pmb\triangledown$ depict the findings from the study conducted by \citet{zhang2023evolution} on the sedimentation of solid spheres in a liquid.
As observed, the value of $\textbf{R}:\textbf{I}$ is on average closer to the mean $r_m^2$ than our simulations, but it maintains the same trend i.e. clusters appear as the volume fraction increases.

 
\ref{fig:A} (right) illustrates the anisotropy of the microstructure. We can see on \ref{fig:A} (right) that we have $A_{xx} \ge 0$ for nearly all our cases, meaning that the emulsion is either isotropic ($A_{xx} = 0$), or exhibits a tendency towards an horizontal alignment of particles on average ($A_{xx} >0$). % or with particles that are in average more aligned horizontally ($A_{xx} >0$). 
%Moreover $A_{xx}$ increases with $Ga$ until $Ga = 50$ where we reach a maximum, and then decrease until $Ga =100$  but still remains consequent. 
Additionally, $A_{xx}$ rises with $Ga$ up to $Ga = 50$, reaching a peak, and subsequently decreases until $Ga = 100$, although it remains significant.
In agreement with the remarks of the previous section, the value of $A_{xx}$ is greater for $\lambda = 1$ and lower for  $\lambda = 10$.
Although not obvious at first we observe a non-monotonic trend with the volume fraction. $A_{xx}$ increases up to a peak value at $\phi = 0.1$ (indicated by \textcolor{col3}{$\pmb\square$} on \ref{fig:A} (right)), but then decreases for $\phi=0.2$ (shown by the \textcolor{col4}{$\pmb\lozenge$} symbols). %$A_{xx}$ first increases up to a maximum value for $\phi =0.1$ (represented by \textcolor{col3}{$\pmb\square$} on \ref{fig:A} (right)) and then decreases for $\phi=0.2$ (represented by the \textcolor{col4}{$\pmb\lozenge$} symbols).
%Although, it is not quite obvious we observe a non-monotonic trend with the volume fraction, $A_{xx}$ first increases up to a maximum value for $\phi =0.1$ (represented by \textcolor{col3}{$\pmb\square$} on \ref{fig:A} (right)) and then decreases for $\phi=0.2$ (represented by the \textcolor{col4}{$\pmb\lozenge$} symbols). 
This implies that at a certain volume fraction, around $\phi \approx 0.1$, higher $\phi$ makes the microstructure more isotropic, while at low volume fraction ($\phi < 0.1$) increasing $\phi$ favors the side-by-side configuration.
This phenomenon of isotropization at high $\phi$ has been reported in other studies such as in \citet{seyed2021sedimentation} for sedimentation of solid particles. 
However, at high \textit{Galileo} numbers, it seems that this effect is less pronounced. 


We conclude that the microstructure can be classified into four classes :
(1) The homogeneous microstructure.
(2) The non-homogeneous but isotropic microstructure for ($\textbf{R}:\textbf{I} > 1$) or dispersed arrangement. %ordered array.
(2 bis) The non-homogeneous but isotropic microstructure for ($\textbf{R}:\textbf{I} < 1$) or clustering. 
(3) The non-homogeneous and non-isotropic microstructure or layering ($\textbf{A}\neq \textbf{0}$). 
Each of these types are characterized by specific values of $\textbf{R}:\textbf{I}$, they are reported in \ref{tab:microstructure}. 
Additionally, to highlight the dependence of $\textbf{R}:\textbf{I}$ on $\phi$ and $Ga$, we display the values of $A_{xx}/tr(\textbf{R})$ and $\textbf{R}:\textbf{I}/r_m^2$ in phase diagrams on \ref{fig:phase}.
We observe that the mean square particle distance compared to a random case is decreasing when increasing the volume fraction, and is globally higher for viscous particles ($\lambda = 10$).
Meanwhile, the likelihood of finding the nearest neighboring particle on the horizontal is greater for $\lambda=1$ than $\lambda = 10$, and it is globally increasing with  $Ga$ and non-monotonic with $\phi$. 
It is seen that at $Ga \approx 50$ and $\phi \approx 0.1$ we reach the configuration with the maximum anisotropy for both viscosity ratios. 
\begin{table}[h!]
    \caption{Microstructure classification}
    \label{tab:microstructure}
    \centering
    \begin{tabular}{|lccccc|} \hline
        Microstructure types & Homogeneous & Isotropic & \ref{fig:scheme_clusters} & $\textbf{R}:\textbf{I}/r_m^2$ & $A_{xx}/tr(\textbf{R})$ \\
        Homogeneous & Yes & Yes &(\textit{Case 1}) & $ \approx 1$ & $\ll 1$ \\
        Dispersed &  No & Yes  &(\textit{Case 2}) & $ > 1$ & $\ll 1$ \\
        Clustering &  No & Yes  &(\textit{Case 2}) & $ < 1$ & $\ll 1$ \\
        Layering &    No & No  &(\textit{Case 4}) & $ - $ & $< 1$\\ \hline
    \end{tabular}
\end{table}
\begin{figure}[h!]
    \centering
    \begin{tikzpicture}[scale=0.8]
        \node (img) at (0,0) {\includegraphics[height=5.5cm]{image/HOMOGENEOUS_NEW/PA/phase_Rtr_l_1.pdf}};
        % \draw[dashed] (10cm,-1.6) ellipse (3 and 2);
        \node (txt) at (-2,1) {Clustering};
        \node (txt) at (-1,-1.6) {Dispersed};
        \draw[dashed] ($(-1,-1.6) + (-10:3 and 2)$(P) arc
        (-10:155:3 and 2);
        \node (img) at (10.5,0) {\includegraphics[height=5.5cm]{image/HOMOGENEOUS_NEW/PA/phase_Rtr_l_10.pdf}};
        % \draw[dashed] (10cm,-1.6) ellipse (3 and 2);
        \node (txt) at (8.5,1) {Clustering};
        \node (txt) at (10,-1.6) {Dispersed};
        \draw[dashed] ($(10,-2) + (-10:3 and 2)$(P) arc
        (-10:180:3 and 2);
    \end{tikzpicture}
    \begin{tikzpicture}[scale=0.8]
        \node (img) at (0,0) {\includegraphics[height=5.5cm]{image/HOMOGENEOUS_NEW/PA/phase_axx_l_1.pdf}};
        \draw[dashed] (1.2,0.3) ellipse (1.5 and 2.5);
        \node (txt) at (1.2,1) {Anisotropic};
        \node (txt) at (-2,1) {Isotropic};

        \node (img) at (10.5,0) {\includegraphics[height=5.5cm]{image/HOMOGENEOUS_NEW/PA/phase_axx_l_10.pdf}};
        % \draw[dashed] (11.7,-0.5) ellipse (0.75 and 1.75);
        % \node (txt) at (11.7,1) {Anisotropic};
        \node (txt) at (8,1) {Isotropic};
    \end{tikzpicture}
    \caption{
        (top) Phase diagram of the dimensionless mean square distance to the nearest neighbor, $\textbf{R}:\textbf{I}/r_m^2$.
        (bottom) Phase diagram of the dimensionless horizontal components of the anisotropy tensor, $A_{xx}/\text{tr}(\textbf{R})$.  
        (left) Iso-viscous emulsion $\lambda = 1$.
        (right) Viscous droplets $\lambda = 10$ }
    \label{fig:phase}
\end{figure}

\subsubsection*{Discussion}
%Although, previous studies mainly focused on bubbles or solid particles, it is reasonable to compare the $\lambda = 1$ and $\lambda = 10$ simulations, to the former and the latter cases, respectively. 
%In \citet{bunner2002dynamics} they performed tri-periodic simulation of buoyant bubbles at $Re \approx 10-30$ for various $\phi$.
%They reported a preference for the bubbles to be aligned in pair.
While earlier studies primarily focused on bubbles or solid particles, it is reasonable to draw parallels between simulations with $\lambda = 1$ and $\lambda = 10$, corresponding to the former and latter scenarios, respectively. In \citet{bunner2002dynamics}, buoyant bubbles were simulated in a tri-periodic setup at Reynolds numbers around 10-30 for various volume fractions ($\phi$). The authors noted a tendency for the bubbles to align in pairs. % a finding that aligns with our observations.
This observation is consistent with what we observe in \ref{fig:phase} ($\lambda = 1$) since the anisotropy tensor : $A_{xx} \approx 0.15$, for $Ga = 25$ (which corresponds roughly to $Re = 25$). 
% Additionally, it is seen that these layers structures are lost at lower volume fraction ($\phi = 0.01$). 
Moreover, \citet{zhang2021direct} conducted Direct Numerical Simulations (DNS) of buoyant bubbly flows within tri-periodic domains. The Reynolds numbers varied from $Re=18$ to $22.8$ across different volume fractions, $\phi$, ranging from $0.05$ to $0.2$, with a fixed Galileo number, $Ga$, of $29$. They observed the presence of anisotropic clusters for $\phi > 0.1$, while noting their absence at lower volume fractions. This observation aligns with findings depicted in \ref{fig:phase} (left), where a small decrease in the value of $A_{xx}$ may be observed for the lowest $\phi$.
%Furthermore, in \citet{zhang2021direct} they carried out DNS of buoyant bubbly flows in tri-periodic domains.
%Their \textit{Reynolds} numbers range form $Re=22.5\to 18$ for various volume fraction : $\phi = 0.05\to 0.2$ and a fixed \textit{Galileo} number of $Ga = 29$. 
%They observe anisotropic clusters for $\phi >0.1$ as well, and they report that at lower volume fraction these structures are not present. 
%It is consistent with the results reported in \ref{fig:phase} (left) where we can see that the value of $A_{xx}$ clearly decreases for lower $\phi$. 

For solid particles at $Ga = 144$, it is observed in \citet{shajahan2023inertial} that in the dilute regime, ($\phi \approx 0.02$), verticals raft of particles are formed. This phenomenon was attributed to the presence of a well-defined wake around individual particles, which effectively traps neighboring particles within the wake without causing repulsion towards the sides. %This effect was explained by the presence of a more developed wake for dilute solid particles which trap neighboring particles within the wake without repulsing it on the sides. 
In our case, we notice a greater concentration of particles in the vertical directions for $\lambda = 10$ compared to $\lambda = 1$ at $Ga = 100$, as evidenced by the smaller value of $A_{xx}$ in the former case. Although not immediately apparent, this phenomenon could be attributed to similar factors, namely, the wake of the viscous drop potentially inducing fewer instances of particles aligning side-by-side and more instances of vertically stable configurations. DNS at higher $Ga$ and $\lambda$ would be necessary to confirm or not the presence of the wake trapping effect. %To conclusively confirm or refute the presence of this wake trapping effect, it would be necessary to conduct DNS at higher $Ga$ and $\lambda$ values.
%In our case we observe more particles in the verticals directions for $\lambda = 10$ than $\lambda =1$ at $Ga =100$, since $A_{xx}$ is smaller in the former cases.
%Although it is not quite obvious it might be the consequence of the same effects, i.e., the wake of the viscous drop might induce less side-by-side configuration and more vertical nearly stable configuration. 
%DNS at higher $Ga$ and $\lambda$ would be necessary to confirm or not the presence of the wake trapping effect.
%In our case we could not observe such a phenomenon, meaning that it might arise at larger \textit{Galileo} number or that the approximation of solid particle at for $\lambda = 10$ is too coarse.
In the moderately dense regime,  $0.02 < \phi \le 0.1$  \citet{shajahan2023inertial} identified more configurations of particles situated side-by-side. 
As mentioned above even if it is less pronounced than for $\lambda = 1$ we indeed observe that $A_{xx}$ is higher in these cases, see \ref{fig:phase} (right). 
Additionally, \citet{almeras2021statistics} carried out experiments of liquid-solid fluidized bed with spherical particles. 
Their Reynolds numbers range between $150\leq Re \leq 360$ depending on the volume fraction $0.14 \leq \phi \leq 0.42$.
It is observed that particles are most concentrated on the horizontal plane of the reference particle when $\phi = 0.14$.
If the $\lambda = 10$ cases follow the same trend, it is reasonable to expect that the probability of horizontal configurations, already predominant at $Ga =100$, will continue to increase for higher \textit{Galileo} at $\phi  \approx 0.1$.


We would like to end this comparison with the literature with the study of \citet{yin2008lattice} which compares the microstructure of suspensions of rising bubbles with suspensions of sedimenting solid particles.
%They studied two \textit{Reynolds} numbers and two volume fractions : $Re = 5,20$ and  $\phi = 0.05, 0.2$, respectively.
%It is found that :
The study encompasses two Reynolds numbers ($Re = 5, 20$) and two volume fractions ($\phi = 0.05, 0.2$). Their findings reveal that: 
\enquote{    
     microstructure in bubble
    suspensions is more anisotropic and inhomogeneous than
    solid particle suspensions of the same volume fraction and
    \textit{Reynolds} number.    
}. 
%Although we compare emulsions with varying droplets' viscosity rather than bubbles and solid particles' suspension, our conclusion regarding the anisotropy of the flow is consistent with their study.
%Additionally, it is also observed in \citet{yin2008lattice} that the microstructure shape has a clear impact on the mean rising velocity of the dispersed phase.
%Indeed, they observed that a power-law function of $(1-\phi)$ fit perfectly the rising velocity of random and isotropic suspensions, while it is not the case when the microstructure exhibit anisotropic structures. 
%Therefore, as stated in introduction, the knowledge of the microstructure shape (reported in \ref{fig:phase}), is of utmost importance in the objective of building realistic averaged models. 
While our analysis focuses on emulsions with varying droplet viscosities rather than bubbles and suspended solid particles, our findings regarding the flow anisotropy align with the study of \citet{yin2008lattice}. 



% In the following we try to explain the origin of the striking difference between $\lambda = 1$ and $\lambda = 10$ on the particle pair distribution.
% With this objective in mind, we present a meticulous analysis of the particle time of interaction as well as the particles relative averaged velocity fields. 

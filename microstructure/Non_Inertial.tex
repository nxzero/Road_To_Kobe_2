This section present an analysis of the microstructure based on the pair probability density function  $P_\text{nst}(\textbf{x},\textbf{r},t,a)$.
After identifying the different forms of the microstructure with respect to the dimensionless parameters, we introduce a general and concise way to quantify it.

\subsubsection*{Nearest pair distribution function}
In this section, only the relative position between pairs will be of interest. 
Therefore, we introduce the reduced probability distribution,
\begin{equation*}
    P_\text{r}(\textbf{x},\textbf{r},t)
    = \int_0^\infty P_\text{nst}(\textbf{x},\textbf{r},t,a) da.
\end{equation*}
In the following graphs, we assume an axisymmetric with respect to the vertical axis due to the periodic boundaries of the numerical domain.
Therefore, it is convenient to represent the vector \textbf{r} with its radial, polar, and azimuthal coordinates,  $r = |\textbf{r}|$,$\beta$ and $\theta$, respectively.
Where $\theta$ is the angle between the vector \textbf{r} and the vertical direction, and $\beta$ the polar angle defined from $0$ to $2\pi$.
By consideration of the axis symmetry of the problem, we will be interested in the weighted probability density function $P_\text{nst}^n(\textbf{x},t,r,\theta)$, defined as, 
\begin{equation*}
    P_\text{r}^n(\textbf{x},t,r,\theta)
    =\frac{1}{2\pi \sin\theta r^2 dr d\theta n_p}
    \int_0^{2\pi}
    \int_0^\infty P_\text{nst}(\textbf{x},t,r,\theta,\beta,a) da d\beta.
\end{equation*}

\subsubsection*{Low inertial effects }
We begin with a detailed analysis of $P_\text{r}^n$ at $Ga =10$, to investigate the influence of $\lambda$ and $\phi$ on the microstructure when inertia effects are low.
As it will be seen in the next section the nearest particle statistic is not fore-after symmetric with respect to the horizontal plane, as it is the case for classic particle pair statistics. 
Nevertheless, it turns out that $P_\text{r}$ possesses a nearly symmetric distribution, such that $P_\text{r}^n(\textbf{x},t,r,\theta)\approx P_\text{r}^n(\textbf{x},t,r,- \theta)$. 
Therefore, in the following plots, displayed in \ref{fig:Pnst_low_Ga} and \ref{fig:Pnst_high_Ga}, we choose to show only the upper part of the distribution.
Additionally, in the discussion below we refer to the sphere at the origin of the graphs, located at $\textbf{x}=0$, as the \textit{test particle} or the \textit{reference particle}. 

\begin{figure}[h!]
    \centering
    \includegraphics[height=0.21\textwidth]{image/HOMOGENEOUS_NEW/Dist/Pnst_l_10_Ga_10_PHI_0_05.pdf}
    \includegraphics[height=0.21\textwidth]{image/HOMOGENEOUS_NEW/Dist/Pnst_l_1_Ga_10_PHI_0_05.pdf}
    \includegraphics[height=0.21\textwidth]{image/HOMOGENEOUS_NEW/Dist/Pnst_l_10_Ga_10_PHI_0_2.pdf}
    \includegraphics[height=0.21\textwidth]{image/HOMOGENEOUS_NEW/Dist/Pnst_l_1_Ga_10_PHI_0_2.pdf}
    \caption{Histogram of the normalized function, $P_\text{r}^n$, at low inertial effects $Ga = 10$.
    The color map represents the values of $P_\text{r}^n$.
    The origin corresponds to the position of the test particle.
    The dimensionless radial and azimuthal coordinates, $|\textbf{r}|/d$ and $\theta$, correspond to the nearest neighbor position.
    The vertical direction corresponds to the flow direction, which is also the axis of symmetry for $P_\text{r}^n$.
    (left) Low volume fraction cases $\phi=0.05$ for $\lambda = 1,10$.
    (right) High volume fraction cases $\phi=0.2$ for $\lambda = 1,10$ }
    \label{fig:Pnst_low_Ga}
\end{figure}
The first observation from \ref{fig:Pnst_low_Ga} indicates that the likelihood of finding the nearest neighboring particle at an angle $\theta$ is uniform across all $\theta$.
This suggests that $P_\text{r}^n$ is isotropic at these \textit{Galileo} numbers. 
If we compare the low volume fraction cases to the high volume fraction cases, we can observe that $P_\text{r}^n$ is more concentrated at near contact of the test particle ($r/d = 1$) in the latter case.
For solid particles it is also common that pair distributions are more concentrated at the contact of the test particle for increasing $\phi$. 
In practice, if particles are more likely to be close to one another, it means that densely packed regions of particles are present in the flow.
This suggests that isotropic clusters, as represented in \ref{fig:scheme_clusters} (\textit{Case 2}), must be present in the emulsion or suspension. 
Regarding the impact of the viscosity ratio, $P_\text{r}^n$ appears quite similar for both values of $\lambda$. 
Consequently, in this regime we find homogeneous microstructures at low $\phi$, and non-homogeneous but still isotropic microstructure at higher $\phi$. 


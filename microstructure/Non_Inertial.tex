\section{Microstructure}
\label{sec:microstructure}

This section presents an analysis of the microstructure based on the nearest neighbor probability density function. 
%age included nearest pair probability density function  $P_\text{nst}(\textbf{x},\textbf{r},t,a)$ and on .


%After identifying the different forms of the microstructure with respect to the dimensionless parameters, we introduce a general and concise way to quantify it.
By definition, $P_\text{nst}(\textbf{r}|\textbf{x},t)$ does not require symmetry with respect to the variable $\textbf{x}+\textbf{r}$ and \textbf{x}, as is the case for classical particle-pair distribution functions. 
Nevertheless, it turns out that $P_\text{nst}$ possesses a nearly-symmetric distribution, such that  $P_\text{nst}(r,\theta)\approx P_\text{nst}(r,- \theta)$ as demontrated below.
\begin{figure}[h!]
    \centering
    \includegraphics[height = 0.3\textwidth]{image/HOMOGENEOUS_NEW/PA/Ry_l_10.pdf}
    \includegraphics[height = 0.3\textwidth]{image/HOMOGENEOUS_NEW/PA/Ry_l_1.pdf}
    \caption{ Dimensionless first moment of the nearest droplet pair distribution in the direction of gravity $M_y/d$. 
    ($\pmb\bigcirc$) $\phi = 0.01$; ($\pmb\triangle$) $ \phi = 0.05$; ($\pmb\square$) $\phi = 0.1$ ($\pmb\lozenge$) $\phi = 0.2$.
    (right)  $\lambda  = 1$
    (left)  $\lambda  = 10$.
    % The hollow symbols correspond to $\lambda = 1$, the filled symbols to $\lambda = 10$.
    % For $r<d$ we arbitrarily set $P_\text{r}^\text{th} = 1$ so that the distribution can be visualized.
    % Black symbols represent the results of \citet{zhang2023evolution} for hard sphere suspension with $\phi = 0.016,0.056,0.134,0.262$  %$\phi = 0.0168,0.0565,0.1341,0.2622$ 
    % corresponding to $\pmb\times,\pmb +, \pmb\star , \pmb\triangledown$, respectively.
    }
    \label{fig:ap:RY}
\end{figure}
%This is demonstrated by \ref{fig:ap:RY} 
%where we can see that the first moment of $P_\text{nst}$, namely
We define the first moment of $P_\text{nst}$ as
\begin{equation}
 \textbf{M} = \int_{\mathbb{R}^3} \textbf{r} P_\text{nst}(\textbf{r}) d\textbf{r}.
\end{equation}
\ref{fig:ap:RY} illustrates the projection of $\textbf{M}$ along the direction of gravity. 
The relatively small but finite values of $M_y$ indicate that $P_\text{nst}$ exhibits a nearly symmetric distribution with respect to $\theta$.
Nevertheless, this indicates that the nearest neighbor is more likely to be located in the upstream direction. 
Note that this is consistent with the findings of \citet{zhang2023evolution}.
Even though this slight asymmetry might have its importance \cite{zhang2023evolution}, we discard it in this study. 
Therefore, we choose to show only the upper part of the distribution in the following plots, (displayed in \ref{fig:Pnst_low_Ga} and \ref{fig:Pnst_high_Ga}) since qualitatively it remains the same as the lower part.  
Additionally, in the discussion below, we refer to the sphere at the origin of the graphs, located at $\textbf{x}=0$, as the \textit{test droplet}.%\textit{test droplet} or the \textit{test droplet}. 

\subsection{Low inertia regimes}
We begin with a detailed analysis of $P_\text{nst}$ at $Ga =10$, to investigate the influence of $\lambda$ and $\phi$ on the microstructure when inertial effects are small.
\begin{figure}[h!]
    \centering
    \includegraphics[height=0.21\textwidth]{image/HOMOGENEOUS_NEW/Dist/Pnst_l_10_Ga_10_PHI_0_05.pdf}
    \includegraphics[height=0.21\textwidth]{image/HOMOGENEOUS_NEW/Dist/Pnst_l_1_Ga_10_PHI_0_05.pdf}
    \includegraphics[height=0.21\textwidth]{image/HOMOGENEOUS_NEW/Dist/Pnst_l_10_Ga_10_PHI_0_2.pdf}
    \includegraphics[height=0.21\textwidth]{image/HOMOGENEOUS_NEW/Dist/Pnst_l_1_Ga_10_PHI_0_2.pdf}
    \caption{Histogram of the probability density function, $P_\text{nst}(r,\theta)$, for low inertia $Ga = 10$.
    The color map represents the nearest pair distribution function. %the values of $P_\text{nst}$.
    The origin corresponds to the position of the \textit{test droplet}.
    The dimensionless radial and azimuthal coordinates, $|\textbf{r}|/d$ and $\theta$, correspond to the nearest neighbor position.
    The vertical direction corresponds to the flow direction, which is also the axis of symmetry for $P_\text{nst}$.
    (left) Low volume fraction cases $\phi=0.05$ for $\lambda = 1,10$.
    (right) High volume fraction cases $\phi=0.2$ for $\lambda = 1,10$.
    }
    \label{fig:Pnst_low_Ga}
\end{figure}
The first observation from \ref{fig:Pnst_low_Ga} indicates that the likelihood of finding the nearest neighboring droplet at an angle $\theta$ is uniform across all $\theta$.
This suggests that $P_\text{nst}$ is isotropic at these \textit{Galileo} numbers. We can observe that $P_\text{nst}$ is larger close to the \textit{test droplet} ($r/d = 1$) in the high volume fraction cases than in the low volume fraction cases.
%If we compare the low volume fraction cases to the high volume fraction cases, we can observe that $P_\text{nst}$ is larger at near contact of the test droplet ($r/d = 1$) in the latter case.
% For solid droplets it is also common that pair distributions are more concentrated at the contact of the test droplet for increasing $\phi$. 
In practice, if droplets are more likely to be close to one another, it means that densely packed regions of droplets are present in the flow.
Specifically, if the value of $P_\text{nst}$ at $r=d$ is larger than the value of the corresponding uniform droplet distribution at $r=d$ (given by \ref{eq:Pnst_dilute}) we may stipulate that clusters are present in the flow. 
Quantitative results supporting this assumption will be provided in the following section. 
For instance we assert that \ref{fig:Pnst_low_Ga} suggests that isotropic clusters, as represented in \ref{fig:scheme_clusters} (\textit{Case 2}), are likely to form in the flow for $\phi = 0.2$. 
Regarding the effect of the viscosity ratio, $P_\text{nst}$ are very similar for both values of $\lambda$ except that for the highest volume fraction the region of highest probability is thinner for the lowest aspect ratio. 
Consequently, in this regime we find homogeneous microstructures at low $\phi$, and non-homogeneous but still isotropic microstructure (\textit{``clusters''}) at higher $\phi$. 


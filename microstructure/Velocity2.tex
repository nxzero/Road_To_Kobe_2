\subsection{Carrier phase velocity fields}

In the previous section we explained the microstructure formation with kinematic arguments.
Although we indeed provided an explanation the question that arise now id :
Why does the relative velocity behave as such.
The answer might be obtained based on dynamical arguments as it is done often, in such a way we could explain the relative kinematic.
Nevertheless, the dynamical aspect of the interaction is out of the scope of this study and will be treated in a future work. 

Instead, we propose to study the particles averaged wakes to explain the possible difference in interaction between the iso-viscous and viscous droplets cases. 
Again we make use of the nearest particle averaged statistic to compute the carrier fluid phase velocity conditionally on the presence of a particle at \textbf{x}, it reads,
\begin{equation*}
    \textbf{u}^\text{nst}_f P_{nst}(\textbf{x},\textbf{r},t)= 
    \int \sum_{i}^{N_b} \delta(\textbf{x}-\textbf{x}_i(\FF,t))
    h_{i} 
    \textbf{u}_f^0(\textbf{x}+\textbf{r},t,\FF)
    d\mathscr{P} 
\end{equation*}
where $h_{i} = 1$ if the particle $i$ center of mass is the nearest point to the eularian coordinate \textbf{x}+\textbf{r}. 
This velocity fields can be reconstructed as well with our DNS. 
On \ref{fig:stream} we display the reconstructed velocity field $\textbf{u}^\text{nst}_f$ for (left) the iso-viscous case $\lambda =1$ and (right) the viscous droplets' case $\lambda = 10$, for different value of the volume fraction. 
\begin{figure}[h!]
    \centering
    \includegraphics[height=0.4\textwidth]{image/HOMOGENEOUS_NEW/Stream/Stream_PHI_1_Ga_100_l_100}
    \includegraphics[height=0.4\textwidth]{image/HOMOGENEOUS_NEW/Stream/Stream_PHI_1_Ga_100_l_10}
    \includegraphics[height=0.4\textwidth]{image/HOMOGENEOUS_NEW/Stream/Stream_PHI_5_Ga_5_l_100}
    \includegraphics[height=0.4\textwidth]{image/HOMOGENEOUS_NEW/Stream/Stream_PHI_5_Ga_5_l_10}
    \includegraphics[height=0.4\textwidth]{image/HOMOGENEOUS_NEW/Stream/Stream_PHI_20_Ga_100_l_100}
    \includegraphics[height=0.4\textwidth]{image/HOMOGENEOUS_NEW/Stream/Stream_PHI_20_Ga_100_l_10}
    % \includegraphics{image/HOMOGENEOUS_NEW/Stream/Stream_PHI_5_Ga_100_l_1.pdf}
    \caption{Nearest averaged carrier phase velocity fields. }
    \label{fig:stream}
\end{figure}
This velocity field is evaluated at $\textbf{x}+\textbf{r}$ conditioned on the presence of the nearest particle at $\textbf{x}$. 


\tb{compare with \citet{shajahan2023inertial} for explaination }

To study the emulsions' microstructure, we chose to adopt the \textit{nearest particle statistics} framework recently revisited by \citet{zhang2021ensemble}.
We now recall some definitions of the \textit{nearest particle statistics} averaging procedure. 
For further details, readers are encouraged to refer to \citet{zhang2023evolution} on which this work mostly relies.

% \subsubsection*{Theoretical framework}
Let $P(\FF)$ be the probability density function that describes the probability of finding the flow in the configuration $\FF$, where $\FF = (\lambda_1,\lambda_2,\lambda_3,\ldots)$ is the set of all parameters describing the flow configuration.
% \footnote{We assume that the flow can be described by a finite number of parameters related to both phase.}. 
Then, we define $d\mathscr{P} = P(\FF)d\FF$ as the probable number of realizations in the incremental region of the flow' phase space, $d\FF$ around $\FF$.
Additionally,  $\textbf{x}_i(t,\FF)$ and $\textbf{x}_j(\FF,t)$ refers to the Lagrangian position vectors of the particles $i$ and $j$, respectively. 
Note that the particles positions are function of time and of the flow configuration $\FF$. 
% We also introduce the age of the interaction, denoted as $a$, which represents the time elapsed since two particles became nearest neighbors.
Then, the nearest pair probability density function is given by \citep{zhang2021ensemble,zhang2023evolution},
\begin{equation}
    P_{nst}(\textbf{x},\textbf{r},t)= 
    \int \sum_{i}^{N_b}\delta(\textbf{x}-\textbf{x}_i(t,\FF))
    \sum_{j\neq i}^{N_b}\delta(\textbf{x}+\textbf{r}-\textbf{x}_j(t,\FF)) 
    % \delta(t+a-t_c^{ij}) 
    h_{ij} (t,\FF)
    d\mathscr{P},
    \label{eq:P_nstij}
\end{equation}
where we introduced the function $h_{ij}(\FF,t)$, which is defined as,
\begin{equation*}
    h_{ij}(\FF,t)
    = \left\{
        \begin{tabular}{cc}
            $1/n(\FF,t)$ & if $j$ is one of the $n^{th}$ nearest neighbors of $i$ \\
            0& otherwise
        \end{tabular}
        \right.
\end{equation*}
In this definition we considered the situation were $n$ particles could be simultaneously nearest neighbor to the particle $i$. 
In DNS, having two nearest neighbors to one particle may never occur, thus in most of the cases $h_{ij}$ is either $1$ or $0$. 
Nevertheless, the coefficient $1/n$ must be retained in the definition for theoretical consistency.
% Formally, we have $a = t - t^{ij}_c(\FF,t)$.
Anyhow, $P_{nst}(\textbf{x},\textbf{r},t)$ is the probability of having a particle center of mass located at $\textbf{x}$ at time $t$ with it nearest neighbor at $\textbf{x}+\textbf{r}$. 
Note that $P_\text{nst}(\textbf{x},\textbf{r},t)$ is related to the number density $n_p(\textbf{x},t)$ through the relationship, 
\begin{equation*}
    \int_{\mathbb{R}^3}
     P_\text{nst}(\textbf{x},\textbf{r},t,a) d\textbf{r}  = n_p(\textbf{x},t). 
    \label{eq:Pnst}
\end{equation*}
This establishes consistency between the nearest particle statistic and kinetic theory \citep{zhang2021ensemble}. 


It is convenient to represent the vector \textbf{r} with its radial, polar, and azimuthal coordinates,  $r = |\textbf{r}|$,$\beta$ and $\theta$, respectively. $\theta$ is the angle between the vector \textbf{r} and the vertical direction, and $\beta$ the polar angle defined from $0$ to $2\pi$. In particular, due to the symmetries of the problem under consideration we consider the weighted probability density function $P_\text{nst}^n(r,\theta|\textbf{x},t)$, defined as, 
\begin{equation}
    n_p(\textbf{x},t) P_\text{nst}^n(r,\theta|\textbf{x},t)dr d\theta 
    =\frac{1}{2\pi \sin\theta r^2 }
    \int_0^{2\pi}
    % \int_0^\infty 
    P_\text{nst}(\textbf{x},t,r,\theta,\beta) 
    % da 
    d\beta.
    \label{eq:Ptheta_r}
\end{equation}
Note that with this definition $P_\text{nst}^n(r,\theta|\textbf{x},t)$ is the probability of finding the nearest neighboring particle at ($r$, $\theta$) relative to \textbf{x}, given that there is a particle in $\textbf{x}$. 
Thus, $P_\text{nst}^n$ is a conditioned probability function, in opposition to $P_\text{nst}(\textbf{x},\textbf{r},t)$. 

To reconstruct all these statistics from the DNS, we treat each simulation time step  as an independent flow configuration, denoted as $\FF$. 
Indeed, time average and ensemble average are identical when the system is \textit{ergodic} \citep{hansen2013theory} which is the case here.  
Precisely, we collected data for each Lagrangian quantity every $10$ simulation time steps. 
The simulation time step is determined either by a Courant Friedrichs Lewy (CFL) condition or by the capillary time step, depending on the dimensionless numbers involved.
On average, $200,000$ time steps are performed during a simulation with $N_b = 125$ droplets. 
This results in a total number of $E = 2,500,000$ events. 
Under those conditions, the ensemble particle average operators can be rewritten as :
\begin{equation}
    \int \sum_i \delta(\textbf{x}-\textbf{x}_i(\FF,t)) \ldots d\PP
    = \frac{1}{E}\sum_\FF^\text{E} \ldots 
    \label{eq:discrete_ensemble_average}
\end{equation}  
When performing conditional average based on the position of the nearest neighboring particle, the methodology is slightly different. 
To reconstruct a quantity such as $P_\text{nst}(\textbf{x},\textbf{r},t)$ we gather all the relative position $\textbf{r}_{ij}(t;\FF)$ from the DNS and store them in $n$ intervals of relative positions $\Delta \textbf{r}_k$ for $k = 1,\ldots, n$.
Then, we apply the discrete ensemble average for each group independently.
Formally, we write, 
\begin{equation}
    P_\text{nst}(\Delta\textbf{r}_k,\textbf{x},t)
    = \frac{1}{E_k} 
    \sum^{E_k}_{\FF_k} 
    1
    \label{eq:vec_cond}
\end{equation}
where $\FF_k$ correspond to the events were the nearest particle pair $i$ and $j$ respects $\textbf{r} \in \Delta \textbf{r}_k$.
$E_k$ is the total number of events fulfilling these constraints. 
Finally, we obtained an approximation of $P_\text{nst}(\textbf{x},\textbf{r},t,a)$ which takes the intervals, $(\Delta\textbf{r}_i)$ as input.

Since, we model a statistically steady state and homogeneous configuration, the values of $P_\text{nst}^n$ are averaged over the variables $\mathbf{x}$ and $t$ for a given simulation. 
Consequently, for our concern, $P_\text{nst}^n$ is not dependent on $\mathbf{x}$ and $t$, but it remains a function of the global flow parameters :  $Ga$, $\phi$, $Bo$, $\zeta$, and $\lambda$.
Therefore, from now on we drop $\mathbf{x}$ and $t$ in our notation. 
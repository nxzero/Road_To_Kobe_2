
In order to study the microstructure of our suspension we choose to adopt the \textit{nearest-particle-statistics} framework recently revisited by \citet{zhang2021ensemble}.
Within this framework we are able to study pair interaction based on the nearest neighbor only. 
It must be understood that it is by no mean restricted to pair interaction in dilute suspension. 
Indeed, even through our statistics focus on a particle and its nearest neighbor, the properties are averaged on all flow configuration therefore the nearest pair probability is still representative even in dense flows \citet{zhang2021ensemble}. 



\paragraph*{Theory :}
Let, $P(\FF)$ be the probability density function that describe the probability of finding the flow in the configuration $\FF$, were $\FF = (\lambda_1,\lambda_2,\lambda_3,\ldots)$ is a finite set of all the parameters describing the initial flow configuration.
% \footnote{We assume that the flow can be described by a finite number of parameters related to both phase.}. 
Then, we define $d\mathscr{P} = P(\FF)d\FF$ as the probable number of flow in the incremental region of the particles' phase space $d\FF$ around $\FF$. 
In the following sections we wish to investigate the particle arrangements and clustering effects as well as the history of interactions. 
In this objective We introduce the probability density function defined such that $P_{nst}(\textbf{r},a)d\textbf{r}da$ is the probable number of having  a nearest neighboring particle at a disatnce $\textbf{r}$ from a test particle at $\textbf{r} = 0$ having an age of interaction of $a$. 
Let $\textbf{x}^i(t,\FF)$ and $\textbf{x}^j(\FF,t)$ be the Lagrangian position vector of the particle $i$ and $j$ function of the initial configuration of the flow $\FF$ and the time $t$. 
Then, the nearest pair probability density function is defined such as, 
\begin{equation}
    P_{nst}(\textbf{x},\textbf{r},a,t)= 
    \int \sum_{i}\delta(\textbf{x}-\textbf{x}^i(\FF,t))
    \sum_{j\neq i}\delta(\textbf{x}+\textbf{r}-\textbf{x}^j(\FF,t)) 
    \delta(t+a-t_c^{ij}(\FF,t)) 
    h_{ij}(\FF,t) d\mathscr{P} 
    \label{eq:P_nstij}
\end{equation}
where we introduced the function : $h_{ij}$, which is equal to $1$ if the particle $j$ is one of the nearest neighbor from the particle $i$, and $h_{ij} = 0$ if it is not \citet{zhang2021ensemble}. 
We also introduced the age of the interaction $a$, which is the time at which the particles started to be nearest $t_c^{ij}(\FF,t)$ neighbor to the current time $t$. 
Then, $P_{nst}(\textbf{x},\textbf{r},a,t)$ is the probability of having a particle center of mass located at $\textbf{x}$ with it nearest neighbor at $\textbf{y}$ knowing the pair of particles have been nearest neighbor for $a$ time. 
This pair probability density function is related to the number density $n_p(\textbf{x},t)$ such that, 
\begin{equation*}
    \int_0^\infty 
    \int_{\mathbb{R}^3}
    P_\text{nst}(\textbf{x},\textbf{r},t,a) d\textbf{r} da = n_p(\textbf{x},t). 
    \label{eq:Pnst}
\end{equation*}
The probable number of nearest neighbor ultimately decrease when $\textbf{r}\to\infty$ while the probable number  of neighbor doesn't. 
Which makes the second integral of \ref{eq:Pnst} convergent. 
Therefore, the main difference of the \textit{nearest particle statistics} with classic pair statistics, is that the latter is not convergent when carrying out such an integration over the second space variable $\textbf{r}$. 

It will prove useful in this work to investigate relative properties between two nearest neighboring particles. 
To this end we introduce the relative Lagrangian property : $q_{ij}(t,\FF)$. 
An example of such a quantity is the relative velocity between the particle $i$ and $j$ which is noted : $\textbf{w}_{ij}(t,\FF) = \textbf{u}_j(t,\FF) - \textbf{u}_i(t,\FF)$ where $\textbf{u}_j(t,\FF) - \textbf{u}_i(t,\FF)$ are the Lagrangian center of mass velocity of the particle $i$ and $j$ respectively. 
The ensemble average of such quantity can be written formally as, 
\begin{equation*}
    \textbf{w}^\text{nst} P_{nst}(\textbf{x},\textbf{r},t,a)= 
    \int \sum_{i}\delta(\textbf{x}-\textbf{x}^i(\FF,t))
    \sum_{j\neq i}\delta(\textbf{x}+\textbf{r}-\textbf{x}^j(\FF,t)) 
    \delta(t+a-t_c^{ij}(\FF,t)) 
    \textbf{w}_{ij}(t,\FF)
    h_{ij}(\FF,t) 
    d\mathscr{P} 
    \label{eq:q_nstij}
\end{equation*}
With this expression, $\textbf{w}^\text{nst}(\textbf{x},\textbf{r},t,a)$ is the averaged relative velocity between a particle and its nearest neighbor, conditionally on the event : there is a particle center of mass at \textbf{x} with its nearest neighboring particle center of mass at \textbf{y}, with age $a$, at time $t$. 
The velocity can be replaced by any particle properties $q_{i}(t,\FF)$ such as the resultant of the force on a particle, or any relative properties $q_{ij}(t,\FF)$. 
In this study we especially focus on the relative velocity statistic, therefore we concentrate on  $\textbf{w}^\text{nst}(\textbf{x},\textbf{r},t,a)$. 

In the following section it will be useful to compare the distribution $P_\text{nst}$ reconstructed from DNS, to theoretical prediction of $P_\text{nst}$ obtained in limiting cases. 
One of which is the isotropic random nearest pair distribution.
It is shown in \citet{zhang2021ensemble} that for a dilute random arrangement of particle the probability density of the nearest PDF reads, 
\begin{equation}
    P_\text{nst}^\text{th}(\textbf{x},t,r) = n_p \exp[{-4 \pi n_p(\textbf{x},t) (r^3 - d^3)/3}].
    \label{eq:Pnst_dilute}
\end{equation}
where $P_\text{nst}(\textbf{x},t,r) = \int_0^\infty P_\text{nst}(\textbf{x},\textbf{r},t,a) da$.

Lastly, since we model a statistically homogeneous configuration within space and time, the variable \textbf{x} and $t$ are of no interest, therefor in the following we restrict our attention to the function $P_{nst}(\textbf{r},a)$. 
In fact the dependence on $\textbf{x}$ and $t$ can ve viewed as the dependence on the global flow parameter $Ga$, $\phi$ and so on. 


\paragraph*{Numerical samples :}
In order to re-construct all these functions and PDF from the DNS we consider that each time step can be considered as flow configuration, $\FF$. 
Based on this assumption the ensemble average operators might be re-written as, 
\begin{equation}
    \int  d\PP\ldots
    = \frac{1}{N}\sum_{\FF_i}^N \ldots 
\end{equation}
In this way we are able to compute every of these PDFs from the Lagrangian properties of the particles within the DNS. 

When performing conditional average on the position of the neighboring particle or the age of the interaction, the methodology is slightly different. 
In fact the continuous conditional operator used in \ref{eq:P_nstij} can be discretized such as, 
\begin{equation}
    \int  d\PP \sum_i^N \sum_{j\neq i}  \delta_i \delta_j \delta_a h_{ij} \ldots
    = \frac{1}{N}\sum_{\FF_i}^N 
    \sum_i^N \sum_{j\neq i}  \delta_i \delta_j \delta_a h_{ij}
    \ldots 
\end{equation}

it is important to obtain representative quantities per bins. 
Therefore, all plotted quantities in the following will display only the value of the bins where at least 1000 sample have been reaches. 



\tb{How to do teh bins ? }


\tb{Number of samples per bins  ? 5000 ??? }


\begin{itemize}
    \item We presented a New multiVOF method developed in the \texttt{Basilisk.fr} flow solver. 
    \item We reconstruct the microstructure fields such as the velocity and streamlines in  anon dilute emulsion. 
    \item We could show qualitatively the influence of $Bo$, $Ga$, $\lambda$ and$\phi$ on the microstructure.
    \item Also, on the relative kinetic of the particles.  
\end{itemize}

In this study we developed a new kind of algorithm which prevent numerical coalescence between VOF tracer within a simulation of buoyant emulsion. 
This enabled us to perform statistically steady simulation of buoyant droplets with various \textit{Galileo} number, volume fraction and viscosity ratios. 

By the mean of the nearest neighboring statistics we described the flow structure around an averaged situation. 
We analyze the velocity field in the vicinity of the particles together with the relative position of the particles in an average manner. 

These results enabled us to make physical points on the closure terms results presented in the followed section. 
Finally, after a clear explanation of the force and Reynolds stress we propose semi-empirical formula to close the averaged equation in those regimes. 

Perspective : a lots of work is needed to well close the RANS especially one must studies the particles interactions. 

In this work, we carried out tri-periodic DNS of buoyancy driven motion of suspensions of drops for a wide range of dimensionless parameters. 
We provided evidence on the nature of the pairwise interactions by studying the probability density $P_\text{nst}(\textbf{r})$ and the relative conditional velocity field $\textbf{w}(\textbf{r},a)$. 
Within this framework  we could identify and quantify phenomena such as the DKT mechanism and the shape of the microstructure. 
% We demonstrate that $\mu_r$ and $Ga$ has a strong impact on the spatial arrangement of the particles and also on their relative behavior. 
% To fully understand to interaction mechanism in an emulsion, these arguments must be completed by a dynamical analysis of the interactions between particles, however, for succinctness it will not be presented in this document.  
This study has been motivated with the view of building a coalescence kernel for population balance equations. 
These coalescence models are often based on film drainage approximations which assume a normal approach between droplets [1].  
However, in Figure \ref{fig:icmf} (panel (e) and (f)), we showed that the interactions are more likely to happen tangentially rather than with a normal approach. 
We also demonstrated that $\mu_r$ and $Ga$ have a strong impact on the spatial arrangement of the particles and on their relative kinematics. 
For these reasons we conclude that there is a clear need to take in account these mechanisms, with the objective of building more accurate coalescence kernels. 
This could be done by considering more sophisticated film drainage situations which would consider relative velocities consistent with the $\textbf{w}(\textbf{r},a)$ distributions documented in this study.
This issue will be addressed in future work.  


\section*{ACKNOWLEDGMENTS}

\section*{DATA AVAILABILITY}

All the data presented in this study are available upon the request at the author. 
All the simulations can be reproduced using the basilisk \texttt{.c} file : \url{http://basilisk.fr/sandbox/fintzin/Rising-Suspenion/RS.c} by following the instruction herein. 
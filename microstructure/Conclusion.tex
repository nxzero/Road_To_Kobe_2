\section{Conclusion}
\label{sec:conclusion}
%The advancement conducted 
In this study, we provided a numerical framework to perform statistically steady simulations of rising emulsions with non-coalescing droplets. 
We then provided quantitative results to measure the microstructure geometry.
Three key points can summarize the major advances:
\begin{enumerate}
    % \item The major advancements proposed in this work is twofold. 
    \item We developed an optimized multi-VoF method within the \texttt{Basilisk} flow solver. 
    It enabled us to avoid coalescence between an arbitrary number of droplets while keeping the number of tracers inferior or equal to $7$. 
    Additionally, we showed that the multi-VoF method is capable of capturing the physics of interfacial interactions despite the coarse mesh definition at the scale of the film between the droplets, see \ref{ap:validation} (\textit{Case 2}). 
    This enabled us to perform massive DNS calculations of rising mono-disperse emulsion with various $\lambda$, $\phi$, and $Ga$ for long durations.
    \item We made use of the recent \textit{nearest particle statistic} framework of \citet{zhang2023evolution} to introduce an objective and concise way to measure the microstructure via the nearest particle pair density function $P_\text{nst}(\textbf{x},\textbf{r},t)$. 
    By looking at the pair distribution function, we could show qualitatively the influence of $Ga$, $\lambda$, and $\phi$ on the microstructure.
    Especially it was found that 
    (1) At low $Ga$ isotropic clusters (see \ref{fig:scheme_clusters}(\textit{Case 2})) appear with increasing $\phi$. 
    (2) Non-isotropic clusters (see \ref{fig:scheme_clusters}(\textit{Case 3})) are more likely to form for high \textit{Galileo} number.
    (3) The viscosity ratio $\lambda$ has an important impact on the microstructure: more clusters and layers are formed for $\lambda = 1$ than for $\lambda = 10$. 
    Even fewer clusters are observed for solid particle suspensions. 
    \item Following \citet{zhang2023evolution}, we show that the microstructure is well described by the second moment of the probability density, noted $\textbf{R}$. 
    This constitutes the central finding of this work. Indeed, we provided evidence that $\textbf{R}$ is a reliable and objective way to measure the microstructure.
    As predicted by \citet{zhang2023evolution} its trace measures the mean square distance between the nearest particles, ultimately a small trace witnesses of the presence of packed particles pair or clusters.
    Its anisotropic part indicates the presence of layers or side-by-side particle pairs in the flow. 
    With a reasonable estimation of the values of $\text{tr}(\textbf{R})$ and $A_{xx}$, one can predict the presence of side-by-side arrangements or clusters.
    % \item In a second step, we showed that a conservation equation can be derived for $\textbf{R}(\textbf{x},t)$ based on a kinetic-theory-like concept.
\end{enumerate}

%To the authors' knowledge, none of the previous studies, apart from \citet{zhang2023evolution} which introduced the concept, made use of a quantity as simple as is \textbf{R} to describe the microstructure. 
We believe that the \textit{nearest particle statistics} framework is powerful enough to model the microstructure of multiphase-flow as it provides an objective and concise way to measure its pair distributions. 
Additionally, its ability to be included in a kinetic theory \citep{zhang2023evolution}, in opposition to other methods,  such as the Voronoi cell volume approach \citep{senthil2005voronoi}, makes it promising. As a perspective, based on the value of $\textbf{R}$, an approximation of the distribution $P_\text{nst}$ could be reconstructed by assuming a particular functional form with two degrees of freedom.
Each degree of freedom would correspond to the scalars $\text{tr}(\textbf{R})$ and $A_{xx}$.
This approximated distribution could then be used to take in account the various microstructures in some theoretical problems involving the theoretical pair distribution function (\citet{batchelor1972sedimentation}, \citet{hinch1977averaged,wang1999longitudinal}, and \citet{zhang2021ensemble}).

This work focused solely on the geometry of the microstructure in the steady state regimes. 
However, an important yet unexplored aspect of this problem is related to the timescales that drive the formation or evolution of the microstructure. 
In other words, what time does it take for the microstructure to reach its steady state? 
To determine how $\textbf{R}$ evolves in time and space, one can make use of the transport equation of $P_\text{nst}$ which directly yields a conservation equation for $\textbf{R}$. 
Then, this equation will eventually help us understand the kinematic of $\textbf{R}$ and, therefore, the kinematic of the microstructure. 
This will be investigated in a future study. 

 

\section*{Acknowledgement}

The computational power of  \textit{TGCC - tr\`es grand centre de calcul du CEA} is greatly appreciated. 
% The author thanks anonymous reviewers for helpful comments.
\section*{Data availability}

All the data presented in this study are available upon request to the author. 
The buoyant emulsion simulations can be reproduced using the basilisk \texttt{.c} file \url{http://basilisk.fr/sandbox/fintzin/Rising-suspension/RS.c}, and following the instructions herein. 

The advancement conducted in this study provide a global method to perform statistically steady rising emulsion with non coalescing droplets.
But also we provide a theoretical framework based on solid theoretical ground to measure the microstructure geometry and timescale.
The major advancement can be sum up into 4 key points :
\begin{enumerate}
    % \item The major advancements proposed in this work is twofold. 
    \item First, We developed an optimized Multi-VoF method within the \texttt{Basilisk} flow solver. 
    It enables us to avoid coalesce between an arbitrary large amount of droplets, while keeping the number of VoF tracer inferior or equal to $7$. 
    Additionally, we show that the Multi-VoF method is capable of capturing the interface interaction, despite the coarse mesh definition at the scale of the film between the drop, see \ref{ap:validation} (\textit{Case 2}). 
    This enables us to perform massive DNS calculations of rising mono-disperse emulsion with various $\lambda$, $\phi$ and $Ga$, which are validated as well, see \ref{ap:validation} (\textit{Case 3}).  
    \item We made use of the recent \textit{nearest particle statistic} framework of \citet{zhang2023evolution} to introduce an objective way to measure the microstructure via the nearest particle pair density function $P_\text{nst}(\textbf{x},\textbf{r},t,a)$. 
    By looking at the pair distribution function we could show qualitatively the influence of $Ga$, $\lambda$ and$\phi$ on the microstructure.
    Especially, it was found that (1) Aligned clusters are more likely to form for high \textit{Galileo} number. (2) For increasing volume fraction, above $\phi = 0.1$, fewer layers is observed. (3) The viscosity ratio $\lambda$ has a huge impact on the microstructure, clearly more  clusters are formed for $\lambda = 1$ than $\lambda = 0.1$. 
    \item Following \citet{zhang2023evolution} we show that the microstructure can be well described by the second moment of the probability density, $P_\text{nst}(\textbf{x},\textbf{r},t,a)$ with respect to the relative position, \textbf{r}, which is noted $\textbf{R}(\textbf{x},t)$. 
    This constitutes the major finding of this work, indeed we provided a reliable and objective way to measure the microstructure by the use of a single tensor $\textbf{R}(\textbf{x},t)$, its trace indicates the mean square distance between the nearest particles, ultimately a small trace witnesses of the presence of packed particles pair or clusters, and the anisotropic part indicate the presence of  any anisotropy in the microstructure such as layers of particles. 
    % \item In a second step we showed that a conservation equation can be derived for $\textbf{R}(\textbf{x},t)$ based on kinetic-theory like concept. 
    Based on \ref{eq:dt_R} we could infer that the time of relaxation of $\textbf{R}(\textbf{x},t)$, is the mean age of interaction of the  nearest particles pairs $\tau_p(\textbf{x},t)$. 
    Likewise, we could show that the relative velocity between particles pairs scales as $\tau_p /d_p$ for all our cases, while its time of relaxation is also $\tau_p$. 
    \item 
    By studying \ref{eq:dt_R} we demonstrated that the correlation between $\textbf{w}_p^\text{nst}$ and \textbf{r}, acts as a source terms for $\textbf{R}(\textbf{x},t)$, which motivated us to study the particles relative velocity conditioned on the relative position. 
    By a careful analysis of $\textbf{w}_p^\text{r}$ and $\textbf{u}_p^\text{r}$ we could show that for $\lambda = 10$ particles goes faster with a nearest neighbor on top or bottom, while for $\lambda = 1$ particles goes faster when the nearest neighbor is at large distance.
    Additionally, we could clearly identify and quantify phenomena such as the DKT mechanism and its derivatives. 
\end{enumerate}
None of the previous studies in the literature apart from \citet{zhang2023evolution} which introduced the concept made use of such simplistic quantity to describe the microstructure. 
We believe that the \textit{Nearest particle statistic} framework is powerful to model multiphase-flow microstructure as it provide an objective and concise way to measure its geometry. 
Additionally, its availability to be included in a kinetic theory-like model, in opposition to other method,  such as the Voronoi cell volume approach previously used \citep{senthil2005voronoi}, makes it promising. 

The microstructure will eventually impact the value of the drift velocity or that of the drag forces Euler-Euler model. 
We believe that,  $P_\text{nst}$ through the value of \textbf{R}, could be taken as an input for theoretical problem such as the sedimentation velocity problem, treated in \citet[Appendix A]{zhang2021ensemble}.
 

Numerous, kinetic model are based on 
On a different note we think that in a future study pair interactions such as the relative velocity between particles could be modeled within this framework. 
This could in turn be used to provide a more accurate description of the 

interactions, firstly because any particles pair properties could be investigated such as the forces and surface of contacts useful for coalesce modeling, and second because it is formulated on solid theoretical ground \citep{zhang2021ensemble}. 

\tb{oppening on dynamics}


This study has been motivated with the view of building a coalescence kernel for population balance equations. 
These coalescence models are often based on film drainage approximations which assume a normal approach between droplets \citet{chesters1991modelling}.  
However, in the Figures displayed in \ref{sec:velocity}, we show that the interactions are more likely to happen tangentially rather than with a normal approach. 
We also demonstrated that $\lambda$ and $Ga$ have a strong impact on the spatial arrangement of the particles and on their relative kinematics. 
For these reasons we conclude that there is a clear need to take in account these mechanisms, with the objective of building more accurate coalescence kernels. 
This could be done by considering more sophisticated film drainage situations which would consider relative velocities consistent with $\textbf{w}^\text{nst}(\textbf{x},\textbf{r},t,a)$ distributions documented in this study.
More globally, the coalescence kernel might be re phrase in the context of the nearest particle statistic to incorporate the various form of $P_\text{nst}$. 
These issues will be addressed in future work.  


\section*{Acknowledgement}

The author thanks the \textit{TGCC - trÈs grand centre de calcul du CEA} which 

\section*{Data availability}

All the data presented in this study are available upon the request at the author. 
The buoyant emulsion simulations can be reproduced using the basilisk \texttt{.c} file \url{http://basilisk.fr/sandbox/fintzin/Rising-Suspenion/RS.c}, and following the instruction herein. 
\subsection{The Basilisk flow solver}

In this problem, the governing equations consist of the one-fluid formulation of the mass and momentum equation, with an additional transport equation for the dispersed phase indicator function. 
We recall their form here, 
\begin{align}
    \pddt \rho+ \div(\rho\textbf{u})
    &= 0,\\
    \label{eq:dt_urho}
    \pddt (\rho \textbf{u})
    + \div (\rho  \textbf{u} \textbf{u} - \bm\sigma)
    &= (\avg{\rho} - \rho)\textbf{g}
    + \textbf{f}_\gamma,\\
    \label{eq:dt_C}
    \pddt C + \textbf{u}\cdot\grad C  
    &= 0,
\end{align}
% \tb{mettre le link navier stokes solver et faire ca pour le reste }
which are the mass, momentum and colors function transport equation, respectively. 
The scalar fields $C$, represents the color function, which range between $0$ and $1$ to indicate the proportion of fluid and dispersed phase, respectively. 
We introduced the fluid velocity vector $\textbf{u}$ and the Newtonian stress tensor $\bm{\sigma} = -p \textbf{I} + \mu (\grad \textbf{u}+ \grad \textbf{u}^\dagger)$ where $p$ is the pressure fields and $^\dagger$ represents the transpose operator.
Note that the material properties, $\rho$ and $\mu$, take the value of the phases in presence, following the arithmetic average : $\rho = (1-C)\rho_f + C \rho_d$ and $\mu = (1-C)\mu_f + C \mu_d$. 
In our case the arithmetic mean turns out to perform better compared to the harmonic mean, which is often used to interpolate the viscosity for bubbly flows \citet{hidman2023assessing,innocenti2020direct}.
More details about this choice is provided in \ref{ap:validation} (\textit{Case 1.}). 
The capillary force is defined as, $\textbf{f}_\gamma =\textbf{n} \gamma \div \textbf{n} $, where \textbf{n} is the normal at the interface.
Following  \citep{bunner2002dynamics}, we incorporated the artificial body force term, $\avg{\rho}\textbf{g}$, on the right-hand side of \ref{eq:dt_urho}, to maintain a zero-averaged velocity throughout the entire numerical domain.  

To solve these equations we first initialized $125$ spherical droplets within a cubic domain with fully periodic boundary condition. 
We used the open source code \url{http///basilisk.fr} to discretize the governing equations in a multigrid solver. 
The Navier-Stokes equations are discretized with a centered scheme.
The two-phase flow solver uses the Volume Of Fluid (VOF) method. 
The interfaces between the droplets and the carrier fluid is reconstructed using the Piecewise Linear Inter-face Calculation or PLIC method \citet[Chapter 5.]{tryggvason2011direct}.
Regarding the treatment of the surface tension force term we refer the reader to \citet{popinet2018numerical} for more details. 
The Basilisk solver has been validated extensively, in the framework of bubbly flow. 
Most of the previous studies \citep{hidman2023assessing,innocenti2020direct} recommend a grid definition of $\Delta/d \ge  30$, where $\Delta$ is the grid spacing. 
In \ref{ap:validation} we carried a mesh independence study and demonstrate that a grid spacing of $\Delta = d/30$ is also suitable in our context.
For readers seeking more detailed information about the solvers, we recommend the wiki pages : \href{http://basilisk.fr/src/navier-stokes/centered.h}{centered.h}, \href{http://basilisk.fr/src/tension.h}{tension.h} and \href{http://basilisk.fr/src/grid/multigrid.h}{multigrid.h} where he can find the source code of the Navier-Stokes, surface tension and multigrid solver used in this work, respectively. 

With the VoF method droplets and bubbles may experience premature coalesce.
For a detailed discussion on this issue, see  \citet[Appendix B]{innocenti2020direct}.
However, in this work, it is imperative to conserve a specific (mono-disperse) population of droplets over time to accumulate sufficient statistics about the microstructure.
To tackle this issue we present in the next section a novel algorithm which prevent coalescence between droplets, while maintaining a reasonable cost. 






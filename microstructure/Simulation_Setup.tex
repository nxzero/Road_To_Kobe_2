\subsection{Numerical method}
The governing equations describe the motion of two immiscible fluids of different densities and viscosities separated by an interface with surface tension. 
We use the so-called ``one fluid'' formulation of the variable density and viscosity Navier-Stokes equations, which can be expressed as \citep{tryggvason2011direct}
\begin{align}
    \pddt \rho+ \div(\rho\textbf{u})
    &= 0,\\
    \label{eq:dt_urho}
    \pddt (\rho \textbf{u})
    + \div (\rho  \textbf{u} \textbf{u} - \bm\sigma)
    &= (\avg{\rho} - \rho)\textbf{g}
    + \textbf{f}_\gamma,\\
    \label{eq:dt_C}
    \pddt C + \textbf{u}\cdot\grad C  
    &= 0,
\end{align}
for the mass, momentum and colors function transport equations, respectively. 
The scalar field, $C$, represents the color function, which ranges between $0$ and $1$ to indicate the volumetric proportion of each phase.
We introduced the fluid velocity vector $\textbf{u}$ and the Newtonian stress tensor $\bm{\sigma} = -p \bm\delta + \mu (\grad \textbf{u}+ \grad \textbf{u}^\dagger)$ where $p$ is the pressure field, $^\dagger$ represents the transpose operator and $\bm\delta$ the unit tensor.
Note that the material properties, $\rho$, and $\mu$, take the values of each phase in presence, using the arithmetic average: $\rho = (1-C)\rho_f + C \rho_d$ and $\mu = (1-C)\mu_f + C \mu_d$. 
In our case, the arithmetic mean performs better than the harmonic mean, which is often used to interpolate the viscosity for bubbly flows \citet{hidman2023assessing,innocenti2020direct}.
More details about this choice are provided in \ref{ap:validation} (\textit{Case 1.}). 
The capillary force is defined as $\textbf{f}_\gamma =\textbf{n} \gamma \div \textbf{n} $, where \textbf{n} is the normal to the interface.
Following  \citep{bunner2002dynamics}, we incorporated the artificial body force term, $\avg{\rho}\textbf{g}$, on the right-hand side of \ref{eq:dt_urho}, to mimic a zero-averaged velocity throughout the entire numerical domain.  

Before solving these equations, we first initialized $N_b = 125$ spherical droplets within a cubic domain with fully periodic boundary conditions. 
We used the open source code \url{http///basilisk.fr} to discretize the governing equations. 
The Navier-Stokes equations are discretized with a centered scheme.
The two-phase flow solver uses the geometric Volume of Fluid (VoF) method. 
The interfaces between the droplets and the carrier fluid is reconstructed using the Piecewise Linear Inter-face Calculation or PLIC method \citet[Chapter 5.]{tryggvason2011direct}.
Regarding treating the surface tension force term, we refer the reader to \citet{popinet2018numerical} for more details. 
The Basilisk solver has been validated extensively in the framework of bubbly flows. 
Most previous studies \citep{hidman2023assessing,innocenti2020direct} recommend a resolution of $d/\Delta \ge  30$, where $\Delta$ is the grid spacing. 
In \ref{ap:validation}, we carry out a mesh-independence study and demonstrate that a grid spacing of $d/\Delta = 20$ is suitable in our context.
For readers seeking more detailed information about the solvers, we recommend the wiki pages: \href{http://basilisk.fr/src/navier-stokes/centered.h}{centered.h}, \href{http://basilisk.fr/src/tension.h}{tension.h} and \href{http://basilisk.fr/src/poissson.h}{poissson.h} where one can find the source code of the Navier--Stokes, surface tension and multigrid solver used in this work, respectively. 

With the VoF method, droplets and bubbles may experience premature coalescence.
See \citet[Appendix B]{innocenti2020direct} for a detailed discussion on this issue.
However, in this work, it is imperative to conserve a specific (mono-disperse) population of droplets over time to accumulate sufficient statistics about the microstructure.
To tackle this issue, we present in the next section a novel algorithm that prevents coalescence between droplets at a reasonable computational cost. 
Note that the wiki page \href{http://basilisk.fr/sandbox/fintzin/Rising-suspension/RS.c}{RS.c} complements this section, where the reader can access the source code used to conduct these DNS, as well as comments and notes to help comprehension. 





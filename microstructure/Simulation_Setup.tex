\subsection{The Basilisk flow solver}
% Objective : 
% \begin{itemize}
%     %\item Present the tri-periodic box within which we add droplets in vof 
%     \item Present : Governing equations under single fluid formulation 
%     \item Present in details the numerical scheme. 
%     \item Finally, present why the numerical coalescence is a problem
% \end{itemize}


In this problem, the governing equations are the one-fluid formulation of the mass and momentum equation, with an additional transport equation for the dispersed phase indicator function noted $C$. 
We recall their form here, 
\begin{align}
    \pddt \rho+ \div(\rho\textbf{u})
    = 0,\\
    \label{eq:dt_urho}
    \pddt (\rho \textbf{u})
    + \div (\rho  \textbf{u} \textbf{u} - \bm\sigma)
    = (\avg{\rho} - \rho)\textbf{g}
    + \textbf{f}_\gamma,\\
    \label{eq:dt_C}
    \pddt C + \textbf{u}\cdot\grad C = 0,
\end{align}
% \tb{mettre le link navier stokes solver et faire ca pour le reste }
for the mass, momentum and colors function transport equation, respectively. 
Were we introduced : the velocity vector $\textbf{u}$,  the Newtonian stress  tensor $\bm{\sigma} = -p \textbf{I} + 2\mu (\grad \textbf{u}+\grad \textbf{u}^T)$ with $p$ the pressure fields and $^T$ the transpose operator.
Note that the material properties, $\rho$ and $\mu$, take the value of the phases in presence according the arithmetic mean : $\rho = (1-C)\rho_c + C \rho_d$. 
Indeed, in our case the arithmetic mean turns out to perform better compared to the harmonic mean, which is often used for the viscosity in bubbly flows \citet{hidman2023assessing,innocenti2020direct}, for more details see \ref{ap:validation} (\textit{Case 1.}). 
The capillary force is defined as, $\textbf{f}_\gamma =\textbf{n} \gamma \div \textbf{n} $, where \textbf{n} is the normal at the interface.
Following  \citep{bunner2002dynamics}, we added the artificial body force term : $\avg{\rho}\textbf{g}$, on the right-hand side of \ref{eq:dt_urho}, to keep a null averaged velocity throughout the whole numerical domain.  

To solve these equations we first initialized $125$ spherical droplets within a square domain with periodic boundary condition on all sides. 
We used the open source code \url{http///basilisk.fr} to discretize the governing equations in a multigrid. 
The Navier-Stokes equations are discretized with a centered scheme.
The two-phase flow solver use the Volume Of Fluid (VOF) method. 
The interfaces between the droplets and the carrier fluid is reconstructed using the Piece wise Linear Inter-face Calculation or PLIC method \citet[Chapter 5.]{tryggvason2011direct}.
Regarding the treatment of the surface tension force term we refer the reader to \citet{popinet2018numerical}. 
The Basilisk solver has been validated extensively, in the framework of bubbly flow. 
Most of the previous studies such as \citep{hidman2023assessing} and \citet{innocenti2020direct} recommend a grid definition of $\Delta \le d/30$. 
In \ref{ap:validation} we carried an independence study on the grid definition and demonstrate that a grid spacing of $\Delta = d/30$ is indeed valid in our context to obtain a physically representative simulations.
If the reader seek for more precise information about the solvers used here, we recommend to look directly at the wiki pages : \href{http://basilisk.fr/src/navier-stokes/centered.h}{centered.h}, \href{http://basilisk.fr/src/tension.h}{tension.h} and \href{http://basilisk.fr/src/grid/multigrid.h}{multigrid.h} where he can find the source code of the Navier-Stokes, surface tension and multigrid solver, respectively. 

It is known that with the VoF method, we experience premature coalesce events between droplets, see  \citet[Appendix B]{innocenti2020direct}.
However, in this work we must be able to conserve a specific population of droplets within time to gather statistics.
In our case  we are interested in a mono-disperse configuration, to study the statistically steady state of the microstructure. 
To tackle this issue we developed a specific algorithm which prevent coalescence between droplets, while keeping a reasonable cost. 
This is the subject of the next section. 






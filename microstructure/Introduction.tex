\section{Introduction}


Buoyancy-driven droplet flows are commonly encountered in chemical engineering processes, including gravity separators and liquid-liquid extractors. Typically, these systems are modeled using the averaged Navier-Stokes equations as described by \citep{castellano2019} for instance. However, these methods necessitate closure laws and a thorough understanding of particle pair statistics \citep{simonin1996}. The accuracy of these closure laws significantly depends on the physical properties of the fluids and the arrangement of particles, referred to as the microstructure. %in this context.
For example, in the Stokes flow regime for a dilute ordered array consisting of a periodic arrangement of spherical inclusions, the dimensionless relative velocity of the suspension decreases proportionally to the cube root of the volume fraction. Conversely, in the case of a random array of freely moving particles, the dimensionless velocities decrease linearly with the volume fraction \citep{saffman1973}.
In inertial regimes, \citet{yin2007} demonstrate that the settling velocity of randomly arranged solid spherical particles can be characterized by a power-law function of $(1-\phi)$, where $\phi$ denotes the particle volume fraction.
However, this power-law relationship fails to accurately describe the observed data for anisotropic bubble microstructures, as shown by \citet{yin2008lattice} and \citet{loisy2017}.
Additionally, \citet{cartellier2009induced} conducted experiments of buoyant bubbly flows. 
They measured the mean velocity agitation of the continuous phase for various $\phi$, 
and they demonstrated that the change in microstructure influence the scaling of agitation with the volume fractions. 
Thus, the mean drag force or mean drift velocity as well as the continuous phase agitation of suspensions, which are essential parameters in the modeling of two-phase flow problems, is influenced by the geometry of the microstructure.

%In inertial regimes, \citet{yin2007} illustrate that the settling velocity of a random arrangement of solid spherical particle can be described by a power-law function of $(1-\phi)$, where $\phi$ represents the particle volume fraction.
%In constrast, for anisotropic bubble microstcuture, this power-law relationship does not accurately represent the observed data, as demonstrated by \citet{yin2008lattice,loisy2017}.


%, assuming a random particle arrangement.
%Buoyancy-driven droplet flows are encountered in many chemical engineering processes such as gravity separators and liquid-liquid extractors.
%The usual engineering practice to model such facilities is to use the averaged Navier-Stokes equations and Population Balance Equations (PBE) \citep{castellano2019}. % \paragraph*{Why microstructure is interesting :}
%These methods require closure laws and a deep understanding of particle pair statistics \citep{simonin1996}.
%These closures laws are strongly dependent on the physical properties of the fluids and on the particle arrangements, which is called in this context the microstructure.

%Specifically for a dilute ordered array consisting of a periodic arrangement of spherical inclusions the dimensionless velocity of the suspension decreases with the volcume fraction with a power 1/3 of the volume fraction. in contrast for a random array of freely moving particle the dimensionless velocitys decreases linearly with the volume fraction.
%For inertial regimes, an example is given in \citet{yin2008lattice}, where they demonstrate that the mean rise velocity of buoyant mono-disperse bubbly flows can be expressed as a power-law of $(1-\phi)$ (where $\phi$ is the particle volume fraction) for random arrangements of particles.
%In contrast, for anisotropic arrangements of particles, this power-law relation is a poor fit to the observation, as also exemplified in \citet{loisy2017}.
%Consequently, the mean drag force or mean drift velocity of suspensions, which are crucial closures in two-phase flow problems, depend on the microstructure geometry.


Another relevant illustration involves the presence of poly-disperse distribution in chemical engineering processes. In these scenarios, Population Balance Equations (PBE) are employed to depict the distribution of droplet sizes \citep{randolph2012theory}. PBE relies on a coalescence kernel, which acts as a source term delineating the rate at which droplets merge. This kernel depends on the microstructure and the relative motion of particle pairs \citep{chesters1991modelling}. Consequently, having statistical information on particle pairs becomes essential in PBE models. This study aims to analyze the microstructure of an emulsion and the relative motion of particle pairs across a wide range of dimensionless parameters.


%Another pertinent example involves poly-disperse flows in our industrial contexts.
%Here, PBE come into play for modeling droplet size distributions \citep{randolph2012theory}.
%PBE rely on a coalescence kernel, acting as a source term delineating droplet coalescence rates.
%This kernel varies with the microstructure and the relative kinematics   of particle pairs \citep{chesters1991modelling}.
%Hence, statistical data on particle pairs become crucial in averaged models.
%The purpose of the present study is thus to characterize the microstructure of an emulsion and the relative kinematics   of particle pairs across a broad spectrum of dimensionless parameters. %This endeavor seeks to enhance the accuracy of averaged two-phase flow equations and, more broadly, to deepen our understanding of droplet flows.

%A second example, which is relevant for our industrial applications, is the one of poly-disperse flows.
%In this case we might use Population Balance Equations (PBE)  to model droplets' size distribution \citep{randolph2012theory}. PBE relies on the coalescence kernel, which act as a source term describing the rate of coalescence of droplets. This kernel is microstructure dependent, and also depends on the relative kinematics   between particles pairs \citep{chesters1991modelling}.
%Consequently, pair particles statistics which describe the relative particles' kinematics   as well as the microstructure geometry are of utmost importance in the averaged models.
%Therefore, in this study we propose to characterize the microstructure of an emulsion as well as the relative particles pairs kinematics   over a wide range of dimensionless parameters in the perspective of building more accurate models for averaged two phases flows equations and more generally for a better understanding of droplets flows.

% \paragraph*{Biblio physics :}
%Computational methods have become increasingly valuable tools for investigating particle interactions and spatial distributions.

One of the pioneering studies dedicated to the microstructure of bubbly flows via Direct Numerical Simulations (DNS) was undertaken by \citet{bunner2002dynamics}.
They conducted tri-periodic simulations focusing on the behavior of nearly spherical rising bubbles in the regime of moderate inertia, characterized by a bubble Reynolds number of approximately 15.
In their investigation involving around 30 bubbles within a tri-periodic domain, they observed a tendency for the bubbles to align horizontally in pairs, along with the identification of horizontal arrays of particles.
An illustration depicting such a microstructure is provided in \ref{fig:scheme_clusters} (Case 4).
Numerous other studies have investigated the microstructures of bubbly flows using DNS, including those by \citet{yin2008lattice,zhang2021direct}. 
They also observe a tendency towards horizontal alignment of bubble pairs within the microstructure, even at high volume fractions. %that the microstructure has a strong tendency toward horizontal alignment of bubble pairs even at high volume fraction.


\citet{shajahan2023inertial} investigated the sedimentation of solid spherical particles at high particulate Reynolds numbers.
In the dilute regime, they observed the formation of vertical particle rafts, which transitioned into side-by-side particle arrangements as the particle volume fraction increased. In dense regimes ($\phi \geq 10 \%$), they observed a nearly random distribution of particles. % in space. %In the dilute regime, they observed vertical rafts of particles, and as the particle volume fraction increased, side-by-side particle formations started to appear.
%From an experimental perspective, \citet{almeras2021statistics} examined the microstructure of solid spherical particles within fluidized beds.
\citet{almeras2021statistics} investigated experimentally the microstructure of solid spherical particles within fluidized beds. They noted the emergence of horizontal particle rafts for moderate volume fraction $\phi \approx 0.2$, which tended to disappear as the volume fraction was increased.  %as the particle volume fraction approached $\phi = 0.14$.
%They observed the presence of horizontal rafts of particles when the particle volume fraction approached $\phi = 0.14$.
The studies above solely focused on spherical particles.
It is also known that the deformability of bubbles or more generally, the shapes of the particles greatly influence the microstructure geometry \citep{bunner2003effect,seyed2021sedimentation}.
%In the present study, we focus on monodisperse emulsion of nearly-spherical droplets.

All these studies reported the formation of particle structures that are dependent on the dimensionless parameters of the problem.
These structures can be classified into three categories: The homogeneous microstructure (\ref{fig:scheme_clusters} (\textit{Case 1: ``homogeneous''})); the non-homogeneous but isotropic microstructure \ref{fig:scheme_clusters} (\textit{Case 2: ``clusters'' }); and the non-homogeneous and non-isotropic microstructure \ref{fig:scheme_clusters} (\textit{Case 3: ``layers''}).
%Despite these studies, investigation into the microstructure of emulsions remains limited, and in general there is a notable absence of closure models for emulsions in the literature.
%However, understanding the microstructure and the physics of emulsions is crucial for the modeling closure terms which in turns allows us to model industrial processes.
%In practice, the microstructures exhibit a varied range of geometries, ranging from more dispersed to more clustered or layered arrangements.
Since the microstructures exhibit a varied range of geometries, having a suitable metric to quantify the microstructure becomes essential. 
Although traditional particle-pair distributions have been employed in previous studies \citep{yin2007,cartellier2009induced, seyed2021sedimentation}, this method has limitations, especially in closure law modeling, as it remains non-zero even at large inter-particle distances. 
Hence, in our study, we adopt the methodology proposed by \citep{zhang2023evolution} and characterize the microstructure using the nearest particle pair distribution.
Specifically, we use a second-order tensor defined as the second moment of nearest neighbor distribution, it is shown to provide a quantitative description of the microstructure.

If one wishes to build relevant statistics, one needs to conduct simulations of droplet flows for sufficient time to reach a statistically steady-state regime.
However, depending on the numerical method used, one may encounter difficulties in preventing coalescence between droplets which is necessary to reach a steady state regime.
Indeed, the Volume-of-Fluid (VoF) method, which is the method used in this work, is known to cause premature coalescence between droplets or bubbles \citep{innocenti2020direct}.
Several methods have been proposed to address this issue, including those by \citet{roghair2011drag,balcazar2015multiple,hidman2023assessing,zhang2023evolution}, and \citet{karnakov2022computing}.
These methods typically involve front-tracking, using a single volume-of-fluid tracer per droplet, or incorporating a numerical interfacial force to hinder coalescence. However, as detailed in the subsequent sections, these methods are only partially suitable for our objectives due to the potential introduction of non-physical behavior or excessive computational overhead. Consequently, we propose a novel algorithm to prevent coalescence, facilitating Direct Numerical Simulation (DNS) over extended periods while maintaining a constant droplet population within a VoF framework.
%The methods used in these studies are either based on front-tracking or by using one volume-of-fluid tracer per droplet, or by adding a numerical interfacial force to prevent coalescence.
%As will be discussed in the body of the text none of these methods are ideal for our purpose, either because they may lead to non-physical behavior or because they are computationally expensive.
%We therefore propose a new algorithm to prevent coalescence which enables us to perform DNS for arbitrarily long times with a fixed population of droplets within a VoF framework.

In this study, we employ tri-periodic Direct Numerical Simulations (DNS) to investigate buoyancy-driven suspensions of mono-disperse droplets. We start by detailing the simulation methodology in \ref{sec:methodo}, presenting our novel algorithm designed to prevent numerical coalescence among droplets. Subsequently, we introduce the Nearest Particle Statistics framework proposed by \citet{zhang2023evolution} in \ref{sec:nearest}. \ref{sec:microstructure} is dedicated to an in-depth examination of the microstructure geometry, where we identify and explore the structures illustrated in \ref{fig:scheme_clusters}, along with their occurrence depending on the dimensionless parameters of the system. Finally, \ref{sec:conclusion} concisely summarizes the key findings presented in this work.

%In this work, we conduct tri-periodic DNS of mono-disperse buoyancy-driven suspensions of drops.
%We begin in \ref{sec:methodo} by outlining the simulations methodology, and describe our new algorithm to prevent numerical coalescence between droplets.
%We then briefly introduce the Nearest Particle Statistics framework of \citet{zhang2023evolution} in \ref{sec:nearest}.
%In \ref{sec:microstructure}, we delve into an analysis of the microstructure geometry and identify the structures depicted in \ref{fig:scheme_clusters} and their correlations with the dimensionless parameters of the problem. \ref{sec:conclusion} summarizes the main findings of this work.
\begin{figure}
    \hfill
\begin{tikzpicture}
    \draw[->] (-1,2.5)--++(0,-2)node[midway, left]{$\textbf{g}$};
    \foreach \i in {1,...,5} {
    \pgfmathsetmacro{\x}{rnd}
    \pgfmathsetmacro{\y}{rnd}
    \draw[fill=gray!50] ($(\x,\y)$) circle (0.1);
    }
    \foreach \i in {1,...,5} {
    \pgfmathsetmacro{\x}{rnd}
    \pgfmathsetmacro{\y}{rnd}
    \draw[fill=gray!50] ($(\x+1,\y)$) circle (0.1);
    }
    \foreach \i in {1,...,5} {
    \pgfmathsetmacro{\x}{rnd}
    \pgfmathsetmacro{\y}{rnd}
    \draw[fill=gray!50] ($(\x+2,\y)$) circle (0.1);
    }
    \foreach \i in {1,...,5} {
        \pgfmathsetmacro{\x}{rnd}
        \pgfmathsetmacro{\y}{rnd}
        \draw[fill=gray!50] ($(\x+1,\y+1)$) circle (0.1);
    }
    \foreach \i in {1,...,5} {
    \pgfmathsetmacro{\x}{rnd}
    \pgfmathsetmacro{\y}{rnd}
    \draw[fill=gray!50] ($(\x+2,\y+1)$) circle (0.1);
    }
    \foreach \i in {1,...,5} {
    \pgfmathsetmacro{\x}{rnd}
    \pgfmathsetmacro{\y}{rnd}
    \draw[fill=gray!50] ($(\x,\y+1)$) circle (0.1);
    }
    \foreach \i in {1,...,5} {
        \pgfmathsetmacro{\x}{rnd}
        \pgfmathsetmacro{\y}{rnd}
        \draw[fill=gray!50] ($(\x+1,\y+2)$) circle (0.1);
    }
    \foreach \i in {1,...,5} {
    \pgfmathsetmacro{\x}{rnd}
    \pgfmathsetmacro{\y}{rnd}
    \draw[fill=gray!50] ($(\x+2,\y+2)$) circle (0.1);
    }
    \foreach \i in {1,...,5} {
    \pgfmathsetmacro{\x}{rnd}
    \pgfmathsetmacro{\y}{rnd}
    \draw[fill=gray!50] ($(\x,\y+2)$) circle (0.1);
    }
    \draw (1.5,3.4)node[below]{\textit{Case 1}};
\end{tikzpicture}
\hfill
\begin{tikzpicture}
    \foreach \i in {1,...,5} {
    \pgfmathsetmacro{\x}{rnd*0.4}
    \pgfmathsetmacro{\y}{rnd*0.4}
    \draw[fill=gray!50] ($(\x,\y)$) circle (0.1);
    }
    \foreach \i in {1,...,5} {
    \pgfmathsetmacro{\x}{rnd*0.3}
    \pgfmathsetmacro{\y}{rnd*0.3}
    \draw[fill=gray!50] ($(\x+1,\y)$) circle (0.1);
    }
    \foreach \i in {1,...,5} {
    \pgfmathsetmacro{\x}{rnd*0.3}
    \pgfmathsetmacro{\y}{rnd*0.3}
    \draw[fill=gray!50] ($(\x+2,\y)$) circle (0.1);
    }
    \foreach \i in {1,...,5} {
        \pgfmathsetmacro{\x}{rnd*0.5}
        \pgfmathsetmacro{\y}{rnd*0.5}
        \draw[fill=gray!50] ($(\x+0.5,\y+1)$) circle (0.1);
    }
    \foreach \i in {1,...,5} {
        \pgfmathsetmacro{\x}{rnd*0.4}
        \pgfmathsetmacro{\y}{rnd*0.4}
        \draw[fill=gray!50] ($(\x+2.5,\y+1)$) circle (0.1);
    }
    \foreach \i in {1,...,5} {
        \pgfmathsetmacro{\x}{rnd*0.4}
        \pgfmathsetmacro{\y}{rnd*0.4}
        \draw[fill=gray!50] ($(\x+1.5,\y+1)$) circle (0.1);
        }
    \foreach \i in {1,...,5} {
        \pgfmathsetmacro{\x}{rnd*0.5}
        \pgfmathsetmacro{\y}{rnd*0.5}
        \draw[fill=gray!50] ($(\x,\y+2)$) circle (0.1);
    }
    \foreach \i in {1,...,5} {
        \pgfmathsetmacro{\x}{rnd*0.4}
        \pgfmathsetmacro{\y}{rnd*0.4}
        \draw[fill=gray!50] ($(\x+2,\y+2)$) circle (0.1);
    }
    \foreach \i in {1,...,5} {
        \pgfmathsetmacro{\x}{rnd*0.4}
        \pgfmathsetmacro{\y}{rnd*0.4}
        \draw[fill=gray!50] ($(\x+1,\y+2)$) circle (0.1);
        }
    \draw (1.5,3.4)node[below]{\textit{Case 2}};
\end{tikzpicture}
\hfill
\begin{tikzpicture}
    \foreach \i in {1,...,5} {
    \pgfmathsetmacro{\x}{rnd*1.5}
    \pgfmathsetmacro{\y}{rnd*0.2}
    \draw[fill=gray!50] ($(\x,\y)$) circle (0.1);
    }
    \foreach \i in {1,...,5} {
    \pgfmathsetmacro{\x}{rnd*1.5}
    \pgfmathsetmacro{\y}{rnd*0.3}
    \draw[fill=gray!50] ($(\x+1,\y)$) circle (0.1);
    }
    \foreach \i in {1,...,5} {
    \pgfmathsetmacro{\x}{rnd*1.5}
    \pgfmathsetmacro{\y}{rnd*0.3}
    \draw[fill=gray!50] ($(\x+2,\y)$) circle (0.1);
    }
    \foreach \i in {1,...,5} {
        \pgfmathsetmacro{\x}{rnd*1.5}
        \pgfmathsetmacro{\y}{rnd*0.2}
        \draw[fill=gray!50] ($(\x+1+0.5,\y+1)$) circle (0.1);
    }
    \foreach \i in {1,...,5} {
    \pgfmathsetmacro{\x}{rnd*1.5}
    \pgfmathsetmacro{\y}{rnd*0.2}
    \draw[fill=gray!50] ($(\x+2+0.5,\y+1)$) circle (0.1);
    }
    \foreach \i in {1,...,5} {
    \pgfmathsetmacro{\x}{rnd*1.5}
    \pgfmathsetmacro{\y}{rnd*0.1}
    \draw[fill=gray!50] ($(\x+0.5,\y+1)$) circle (0.1);
    }
    \foreach \i in {1,...,5} {
        \pgfmathsetmacro{\x}{rnd*1.5}
        \pgfmathsetmacro{\y}{rnd*0.2}
        \draw[fill=gray!50] ($(\x+1,\y+2)$) circle (0.1);
    }
    \foreach \i in {1,...,5} {
    \pgfmathsetmacro{\x}{rnd*1.5}
    \pgfmathsetmacro{\y}{rnd*0.2}
    \draw[fill=gray!50] ($(\x+2,\y+2)$) circle (0.1);
    }
    \foreach \i in {1,...,5} {
    \pgfmathsetmacro{\x}{rnd*1.5}
    \pgfmathsetmacro{\y}{rnd*0.1}
    \draw[fill=gray!50] ($(\x,\y+2)$) circle (0.1);
    }
    \draw (1.5,3.4)node[below right]{\textit{Case 3}};
\end{tikzpicture}
\hfill
\caption{Sketch of the three different particle arrangements identified in this study.
The gray disks represent droplets in a three-dimensional space.
Gravity acts in the vertical direction.
(\textit{Case 1: ``homogeneous''}) A homogeneous arrangement of droplets is called a homogeneous microstructure or just ``homogeneous''.
(\textit{Case 2: ``clusters''}) Isotropic non-homogeneous microstructure where we observe the presence of ``clusters'' meaning that the particles are close to each other on average.
The opposite scenario, where particles are, on average, far from each other, also falls into this category.
(\textit{Case 3: ``layers''}) The non-isotropic non-homogeneous microstructure where we observe the presence of stratified arrangements.
This type of microstructure is referred to as layered microstructure or just ``layers''.
}
\label{fig:scheme_clusters}
\end{figure}

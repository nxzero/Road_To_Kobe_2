\section{Introduction}



\begin{enumerate}
    \item[Intro : ]
    Buoyancy-driven droplet flows are encountered in many chemical engineering processes such as gravity separators, liquid-liquid extractors, etc. The usual engineering practice to model such facilities is to make use of the averaged Navier-Stokes equations and population balance equations. 
    \item[Why is it interesting :]
    However, these methods necessitate closure laws and a deep understanding of particle pair statistics.
    Especially, closures laws are in fact microstructure dependent. 
    Therefore, in an objective to understand to physics these terms it is primordial to understand the microstructure and why it is like so. 
    \item[bibliography : ]
    Previous authors investigated the microstructure of bubbly flows. 
    \citet{bunner2002dynamics} reported the creation of horizontal raft for rising  spherical bubbles. 
    \citet{bunner2003effect} demonstrated taht due to wake trapping effect deformable bubbles have a tendency to align horizontally. 
    In a more recent study \citet{zhang2021direct} also found the creation of horizontal layers of particles. 
    Regarding the solid particles other studies have been conducted for the sedimentation of spheres \citet{shajahan2023inertial}, and they found a high probability of particle pair on  the side of the particle of reference.     
    \item[What is still needed :]
    None, of these studies focus on droplets' suspension. 
    So no closure are available. 
    Additionally, none of them proposed an algorithm to avoid coalesce. 
    \item[What good if i new :]
    An algorithm to prevent coalesce would allow us to perform DNS for arbitrary long time with a fixed population of droplets. 
    In addition to providing data for the closure terms appearing in averaged models, the simulations are of great interest to understand and describe the microstructure of suspensions. 
    \item[introduce the plan :]
    Therefore, within a multiscale strategy, we perform tri-periodic Direct Numerical Simulations (DNS) of monodisperse buoyancy driven suspension of drops.
    In this work we present a concise analysis of the microstructure by analyzing the \textit{Nearest Particle Statistics} recently revisited by \citet{zhang2021ensemble}. 
    We start in \ref{sec:methodo} to present the DNS methodology. 
    Then, we introduce quickly the \textit{nearest particle statistics}.
    In \ref{sec:microstructure} we present an analysis of the microstructure geometry and identify structure such as layers and cluster with respect to the dimensionless parameters. 
    Then, in \ref{sec:time} we describe the relevant timescale of the flow. 

\end{enumerate}
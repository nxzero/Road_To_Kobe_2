\documentclass[12pt]{My_preprint}
\usetikzlibrary{arrows.meta,
                chains,
                positioning,
                shapes.geometric}

%%%%%%%%%%%%%%%%%%%%%%%%%%%%%%%%%%%%%%%%%%%%%%%%%%%%%%%%%%%%%%%%%%%%%%%%%%%%%%%
\newcommand{\size}{0.22\textwidth}
\newcommand{\avg}[1]{\left<#1\right>}
\renewcommand{\avg}[1]{\left<#1\right>}
\newcommand{\condavg}[1]{\left<#1 | \mathscr{C}_1\right>}
\newcommand{\Exp}[1]{\overline{\overline{#1}}}
\newcommand{\davg}[1]{\left<#1\right>_d}
\newcommand{\cavg}[1]{\left<#1\right>_c}
\newcommand{\pavg}[1]{\avg{\delta_\alpha #1}}
% \newcommand{\pnavg}[1]{n\left<#1\right>_p}

\newcommand{\avgcond}[1]{\left<#1\right>}
\renewcommand{\avgcond}[1]{\overline{#1}}
\newcommand{\kavg}[1]{\avgcond{#1}^k}
\newcommand{\Iavg}[1]{\avgcond{#1}^I}
\newcommand{\pnnavg}[1]{\avgcond{#1}^{p}}
\newcommand{\pnavg}[1]{n_p\pnnavg{#1}}
\newcommand{\oneavg}[1]{\avgcond{#1}^1}
\newcommand{\twoavg}[1]{\avgcond{#1}^2}
\newcommand{\smallavg}[2]{\avgcond{#1}^{#2}}
\newcommand{\sym}[1]{\text{Sym}\left[#1\right]}

\newcommand{\nstavg}[1]{\overline{#1}_{nst}}
\newcommand{\nstrelavg}[1]{\overline{#1}_{nst}^{rel}}
\newcommand{\mavg}[1]{\left<#1\right>_m}
\newcommand{\gavg}[2][\gamma]{\left<#2\right>_{#1}}
\newcommand{\partials}[1]{\partial_{i_1}\partial_{i_2}\ldots\partial{i_{#1}}}
\newcommand{\partialp}[2]{ \prod_{m=#1}^{#2} \partial_{i_m}}
\newcommand{\hatpartialp}[2]{ \prod_{m=#1}^{#2} \hat{\partial}_{j_m}}
\newcommand{\hatpartialpi}[2]{ \prod_{m=#1}^{#2} \hat{\partial}_{i_m}}
\newcommand{\pri}[2]{ \prod_{m=#1}^{#2} r_{i_m}}
\newcommand{\prj}[2]{ \prod_{m=#1}^{#2} r_{j_m}}

\newcommand{\grad}{\mathbf{\nabla}}
\renewcommand{\div}{\mathbf{\nabla}\cdot}
\newcommand{\gradI}{\mathbf{\nabla}_{||}}
\newcommand{\divI}{\mathbf{\nabla}_{||}\cdot}

\newcommand{\ddt}{\frac{d}{d t}}
\newcommand{\pddt}{\frac{\partial}{\partial t}}
\renewcommand{\pddt}{\partial_t}
\newcommand{\norm}[1]{\hat{#1}}
\newcommand{\Jump}[1]{\llbracket #1 \rrbracket \cdot \textbf{n} }

\newcommand{\CC}{\mathscr{C}}
\newcommand{\PP}{\mathscr{P}}

%%% Utiliser pour les commentaires
\newcommand{\JL}[1]{\color{red}#1\color{black}}
\newcommand{\DL}[1]{\color{green}#1\color{black}}
\newcommand{\tb}[1]{\color{blue}#1\color{black}}
% \renewcommand{\alpha}{}
\renewcommand{\JL}[1]{}
% \renewcommand{\tb}[1]{}

\renewcommand{\size}[1]{0.3\textwidth}
\newcommand{\expo}[2][n]{\frac{(-1)^#1}{#1!} \partialp{1}{#1} \pavg{\int_{\Omega_\alpha} \pri{1}{#1}#2 d\Omega}}
\newcommand{\expoU}[2][n]{\frac{(-1)^#1}{#1!} \partialp{1}{#1} \pavg{\textbf{u}_\alpha\int_{\Omega_\alpha} \pri{1}{#1}#2 d\Omega}}
\newcommand{\expoS}[2][n]{\frac{(-1)^#1}{#1!} \partialp{1}{#1} \pavg{\int_{\Sigma_\alpha} \pri{1}{#1}#2 d\Sigma}}

% \newcommand{\numref}[1]{\ref{#1}}
\renewcommand{\ref}[1]{\autoref{#1}}

%%%%%%%%%%%%%%%%%%%%%%%%%%%%%%% Title & Author %%%%%%%%%%%%%%%%%%%%%%%%%%%%%%%%


\title{Notes}

\author[1,2]{Nicolas Fintzi}
\affil[1]{IFP Energies Nouvelles, Rond-point de l’changeur de Solaize, 69360 Solaize}
\affil[2]{Sorbonne Université, Institut Jean le Rond ∂’Alembert, 4 place Jussieu, 75252 PARIS CEDEX 05, France}

\begin{document}
\section{Incoherence dans le bilan energetique du modèle hybrid ?}
We consider a Newtonian fluid for both phase such that, $\bm{\sigma}_k = -p_k \textbf{I} + \bm{\tau}_k = -p_k \textbf{I} + \mu_k\textbf{e}_k$.
In this situation the fluid phase equations for mass, momentum and energy reads as,  
\begin{align}
    \pddt (\phi_1 \rho_1)  
    + \div (
        \phi_1 \rho_1\textbf{u}_1
    )
    &= 
    0\\
    \pddt (\phi_1 \rho_1\textbf{u}_1)  
    + \div (
        \phi_1 \rho_1\textbf{u}_1\textbf{u}_1
        - \bm{\sigma}_1^\text{eq}
    )
    &= 
    \phi_1  (\rho_1\textbf{g} - \grad p_1 )
    -  \textbf{f}_\text{p}\\
    \pddt (\phi_1\rho_1E_1)  
    + \div (
        \phi_1\rho_1E_1\textbf{u}_1
        + \bm{q}_1^\text{eq}
        - \textbf{u}_1 \cdot \bm{\sigma}_1^\text{eq}
        % - \textbf{u}_1^0 \cdot \bm{\sigma}_1^0 
        % + \textbf{q}_1^0
        )
    &= 
    n_p \mathcal{F}_p:\grad \textbf{u}_1
    + \phi_1 \textbf{u}_1 \cdot (\rho_1\textbf{g}
    - \grad p_1 
    )\\
    &- \phi_1 p_1 \div \textbf{u}_1
    - n_p \textbf{c}_\text{p}
    + n_p \textbf{e}_\text{p}
    - \textbf{u}_1 \cdot \textbf{f}_\text{p}
\end{align} 
\begin{align*}
    &\bm{\sigma}_1^\text{eq}
    = \phi_1 (
        \bm{\tau}_1%- n_p \textbf{M}_p
        - \rho_1\kavg{\textbf{u}_1'\textbf{u}_1'}) 
        + n_p \mathcal{F}_\text{p}
    &\textbf{q}_1^\text{eq}
    =\textbf{q}_1^\text{e} 
    +\textbf{q}_1^\text{k}  \\
    &\textbf{q}_1^\text{e}
    = \phi_1\rho_1 \kavg{\textbf{u}_1' e_1'} 
    + \phi_1\textbf{q}_1 
    + n_p \mathcal{E}_p
    &\textbf{q}_1^\text{k}
    = \phi_1\rho_1 \kavg{\textbf{u}_1' k_1} 
    - \phi_1\kavg{\textbf{u}_1' \cdot \bm{\sigma}_1^0}
    - n_p \mathcal{C}_p
\end{align*}
where $\mathcal{F}_\text{p},\mathcal{E}_p$ and $\mathcal{C}_p$ are the first moment of $\textbf{f}_\text{p}$, $\textbf{e}_\text{p}$ and $\textbf{c}_\text{p}$, respectively. 

\subsection{The mean Fluid stress}

Regarding the fluid stress it can be reformulated considering Newtonian fluid,
\begin{equation}
    \phi_1 \bm{\sigma}_1 
    = - \phi_1 p_1 \textbf{I}
    + \mu_1 \phi_1 \textbf{e}_1
\end{equation}
with $\textbf{e}_1$ being the averaged shear rate. 
The first model is then, 
\begin{align*}
    \phi_1 \textbf{e}_1
    = \phi_1 (\nabla \textbf{u}_1+ (\grad \textbf{u}_1)^T)
    + \avg{[(\textbf{u}_1^0 - \textbf{u}_1)  \textbf{n}_1 
    +  \textbf{n}_1(\textbf{u}_1^0 - \textbf{u}_1 )]\delta_I}
\end{align*}
Where the second term is the extra stress tensor. 
In \citet[chap 9]{ishii2010thermo} they assume,
\begin{equation}
    \avg{[(\textbf{u}_1^0 - \textbf{u}_1)  \textbf{n}_1 +  \textbf{n}_1(\textbf{u}_1^0 - \textbf{u}_1 )]\delta_I}\\
    = 
    (\textbf{u}_2 - \textbf{u}_1)  \grad \phi_1 +  \grad \phi_1(\textbf{u}_2 - \textbf{u}_1 )\\
\end{equation}
But I didn't find out where the derivation came from the reference article is too old there is no access.  

% \subsection*{Second approach}
Alternatively we can say that, 
\begin{align*}
    \phi_1 \textbf{e}_1
    = \nabla \textbf{u}+ (\grad \textbf{u})^T
    - \avg{\chi_2 (\grad\textbf{u}_2^0 + \grad(\textbf{u}_2^0 )^T)}
    = \textbf{e}
    - \phi_2 \textbf{e}_2
\end{align*}
Thus, the stress in the fluid phase of a suspension can be written,
\begin{equation}
    \bm{\sigma}_1 \phi_1
    =- \phi_1 p_1 \textbf{I}
    + \mu_1 \textbf{e}
    - \lambda \phi_2 \bm{\tau}_2
    % \bm{\sigma}_1 
    % &= - \left(p_1 + \frac{\lambda \phi_2}{\phi_1} p_2\right) \textbf{I}
    % + \frac{\mu_1}{\phi_1} \textbf{e}
    % - \frac{\lambda \phi_2}{\phi_1} \bm{\sigma}_2\\
    % \bm{\sigma}
    % &= - \phi_1 p_1  \textbf{I}
    % + \mu_1 \textbf{e}
    % + \bm{\sigma}_2 \phi_2 
    % +\phi_I \bm{\sigma}_I 
    % - \lambda \phi_2 \bm{\tau}_2
    \label{eq:sigma1}
\end{equation}
where $\lambda$ is the viscosity ratio vanishing for solid particles. 
% \subsection*{The fluid equivalent stress}

% As mentioned above the fluid equivalent stress can be written, 
% \begin{equation}
%     \bm{\sigma}_1^\text{eq}
%     = \phi_1 (
%         \bm{\tau}_1%- n_p \textbf{M}_p
%         - \rho_1\kavg{\textbf{u}_1'\textbf{u}_1'}) 
%         + n_p \mathcal{F}_\text{p}
% \end{equation}
% Using the developement below we can show that 
% \begin{equation}
%     \bm{\sigma}_1^\text{eq}
%     = - \phi_1\rho_1\kavg{\textbf{u}_1'\textbf{u}_1'}
%         + n_p \mathcal{F}_\text{p}
%         + \mu_1 \textbf{e}
%         - \lambda \phi_2 \bm{\tau}_2
% \end{equation}
\subsection*{Mixture stress}

Let's derive a geenral expression for the mixture stress $\bm{\sigma}$.
According to \ref{eq:sigma1}  the stress in the mixture phase of a suspension can be written,
\begin{equation}
    \bm{\sigma}
    = - \phi_1 p_1  \textbf{I}
    + \mu_1 \textbf{e}
    - \lambda \phi_2 \bm{\tau}_2
    + \bm{\sigma}_2 \phi_2 
    +\phi_I \bm{\sigma}_I 
    \label{eq:sigma}
\end{equation}
where $\phi_I \bm{\sigma}_I$ is the averaged surface stress $\bm{\sigma}_I^0 = \gamma (\textbf{I}  -  \textbf{nn})$.
We can reformulate the last two terms using the first moment of momentum balance equation, 
\begin{equation}
    -  \dot{\mathcal{P}_p}
    +  \mathscr{S}_p^*
    +  \mathscr{L}_p
    + \frac{1}{3}(\bm{\sigma}_1^0 \cdot \textbf{n}_2 \cdot \textbf{r})_p^\Sigma \textbf{I}
    + (\rho_2 \textbf{w}_2^0  \textbf{w}_2^0 )^\Omega
    =   (\bm{\sigma}_2^0)^\Omega
    + (\bm{\sigma}_I)^\Sigma,
\end{equation}
where we used the notation $n_p(\ldots)^\Sigma_p$ and $n_p(\ldots)^\Omega_p$ to denote the particle average of surface and volume integrated quantities. 
% Or in stokes condition, 
% \begin{equation}
%     n_p \mathscr{S}_p^*
% + n_p \mathscr{L}_p
% + n_p\frac{1}{3}(\bm{\sigma}_1^0 \cdot \textbf{n}_2 \cdot \textbf{r})_p^\Sigma \textbf{I}
%     = n_p \left(
%         \bm{\sigma}_2^0
%     \right)_p^\Omega
%     +n_p (\bm{\sigma}_I)^\Sigma_p
% \end{equation}
And we defined, 
\begin{align*}
    \mathscr{S}_p^* =\frac{1}{2}  (
        \textbf{r} \bm{\sigma}_1^0 \cdot \textbf{n}_2
        +  \bm{\sigma}_1^0 \cdot \textbf{n}_2\textbf{r}
        -
          \frac{2}{3}(\bm{\sigma}_1^0 \cdot \textbf{n}_2 \cdot \textbf{r})\textbf{I}
        )_p^\Sigma\\
    \mathscr{L}_p =\frac{1}{2} (
        \textbf{r} \bm{\sigma}_1^0 \cdot \textbf{n}_2
        - \bm{\sigma}_1^0 \cdot \textbf{n}_2\textbf{r}
        )_p^\Sigma\\
    \mathcal{\dot{P}}_p = \left(\ddt ( \textbf{r} \textbf{u}_2^0)^\Omega\right)_p
\end{align*}
which are the symmetric and anti-symmetric part of the first moment of momentum, respectively. 

Considering a homogeneous suspension (or neglecting higher moments in \ref{eq:sigma}), and using the first moment of momentum balance in \ref{eq:sigma1} we can write :
\begin{align*}
    \bm{\sigma}
    &= [- \phi_1 p_1 
    + n_p \frac{1}{3}(\bm{\sigma}_1^0 \cdot \textbf{n}_2 \cdot \textbf{r})^\Sigma_p] \textbf{I}
    + \mu_1 \textbf{e}
    + n_p \mathscr{S}^*
    - \lambda \phi_2 \bm{\tau}_2
    + n_p \mathscr{L}
    -  \dot{\mathcal{P}_p}
    + (\rho_2 \textbf{w}_2^0  \textbf{w}_2^0 )^\Omega
\end{align*}
which is the Bluk stress expression for dispersed two phase flow. 

In stokes flow condition the expression reduce to 
\begin{align*}
    \bm{\sigma}
    &= [- \phi_1 p_1 
    + n_p \frac{1}{3}(\bm{\sigma}_1^0 \cdot \textbf{n}_2 \cdot \textbf{r})^\Sigma_p] \textbf{I}
    + \mu_1 \textbf{e}
    + n_p \mathscr{S}
    + n_p \mathscr{L}
\end{align*}
where we introduced $\mathscr{S} = n_p \mathscr{S}_p^* - \lambda \phi_2 \bm{\tau}_2 \approx   \mathscr{S}_p^* - \lambda n_p (\bm{\tau}_2^0)^\Omega_p$  as being the Stresslet. 
Indeed, the original expression include internal the term $- \lambda \phi_2 \bm{\tau}_2 = $ within the stress let as it is demonstrated in some reference books \citep{kim2013microhydrodynamics,pozrikidis1992boundary}. 
From this expression and the calculation of $\mathscr{S} $ we can determine the equivalent Einstein viscosity etc\ldots 


\subsubsection{Calculation of the stresslet in stokes condition}

In VOF simulation it is often more accurate to compute the quantity with volume integral rather than surface integration.
Thus, we seek for a volume integral formulation of the Stresslet $\mathscr{S}$. 
From the first momentum equation we can deduce by neglecting the inertia that :
\begin{align*}
    n_p \mathscr{S}_p
+ n_p \mathscr{L}_p
+ n_p\frac{1}{3}(\bm{\sigma}_1^0 \cdot \textbf{n}_2 \cdot \textbf{r})_p^\Sigma \textbf{I}
    &= 
    n_p \left(
        \bm{\sigma}_2^0
    \right)_p^\Omega
    +n_p (\bm{\sigma}_I)^\Sigma_p
    - \lambda \phi_2 \bm{\tau}_2\\
    &= 
    - n_p \left(
        p_2^0
    \right)_p^\Omega \textbf{I}
    +n_p (\bm{\sigma}_I)^\Sigma_p
    + n_p (1 - \lambda)\left(
        \mu_2 \textbf{e}_2^0
    \right)_p^\Omega 
\end{align*}
Now if we isolated the symmetric part we get : 
\begin{equation}
    n_p \mathscr{S}_p
% + n_p \mathscr{L}_p\textsc{}
=
    +n_p (\bm{\sigma}_I - \frac{1}{3}(\bm{\sigma}_I : \textbf{I})\text{I})^\Sigma_p
    + n_p (1 - \lambda)\left(
        \mu_2 \textbf{e}_2^0
    \right)_p^\Omega 
    \label{eq:stresslet}
\end{equation}
Isolating the antisymmetric part gives
\begin{align*}
+ n_p \mathscr{L}_p\textsc{}
=
0
\end{align*}
Meaning that in stokes flow condition torque is only related to body force or inertial terms that we neglected here. 

In a more general manner the Stresslet can be given by : 
\begin{align*}
    n_p \mathscr{S}_p
% + n_p \mathscr{L}_p
% + n_p\frac{1}{3}(\bm{\sigma}_1^0 \cdot \textbf{n}_2 \cdot \textbf{r})_p^\Sigma \textbf{I}
    &= 
    % - n_p \left(
    %     p_2^0
    % \right)_p^\Omega \textbf{I}
    +n_p (\bm{\sigma}_I^\text{dev})^\Sigma_p
    + n_p (1 - \lambda)\left(
        \mu_2 \textbf{e}_2^0
    \right)_p^\Omega 
    + n_p \dot{\mathcal{S}_p}
    - n_p \rho_2 (\textbf{w}_2^0\textbf{w}_2^0 - \textbf{w}_2^0\cdot \textbf{w}_2^0)_p^\Sigma
\end{align*}

\section*{QUESTION ?}

If the particle is spherical ($Bond numebr \ll 1$ and therefore $(\bm{\sigma}_I - \frac{1}{3}(\bm{\sigma}_I : \textbf{I})\text{I})^\Sigma_p =  0$) the \ref{eq:stresslet} reads, 
\begin{align*}
    n_p \mathscr{S}_p
    &= 
    + n_p (1 - \lambda)\left(
        \mu_2 \textbf{e}_2^0
    \right)_p^\Omega 
\end{align*}
Which implies that a droplet with the same viscosity as the fluid $\lambda = 1$ will generate no stresslet. 
Which is of course wrong since the famous formula from \citet{rallison1978note} predict a non null $\mathscr{S}$ due to the non deformability of the drops. 

So the question is : 
\begin{itemize}
    \item 
    Where is the contribution of the droplet surface tension to $\mathscr{S}$ in \ref{eq:stresslet} when the droplet is spherical ? 
\end{itemize}


\bibliography{Bib/bib_bulles.bib}
\end{document}

\documentclass[11pt]{My_preprint}


\title{Averaged equations for disperse two-phase flow with interfacial properties and their closures for dilute suspension of droplets}

\author[1,2]{Nicolas Fintzi}
%\author[2]{Daniel Lhuillier}
\author[1]{Jean-Lou Pierson}
% \author[2]{Stephane Popinet}
\affil[1]{IFP Energies Nouvelles, Rond-point de l'echangeur de Solaize, 69360 Solaize}
\affil[2]{Sorbonne Universit\'e, Institut Jean le Rond $\partial$'Alembert, 4 place Jussieu, 75252 PARIS CEDEX 05, France}

\begin{document}

\maketitle

\begin{abstract}
This article provides a derivation of the averaged equations governing the motion of dispersed two-phase flows with interfacial transport. 
We begin by revisiting the two-fluid formulation, as well as the distributional form of the interfacial transport equation which holds on the entire domain. 
Following this, a general Lagrangian model is introduced, which accounts for the effects of both internal and interfacial properties of the dispersed inclusions (bubbles, droplets, or particles) within a continuous phase.
This is achieved by derivation of conservation laws for particle surface and volume-integrated properties. 
By summing the internal and interfacial conservation laws, we derive a conservation equation for an arbitrary Lagrangian property associated with the inclusion. 
We then proceed by deriving the lesser-known conservation equations for the moments of the volume and surface distribution of an arbitrary Lagrangian property.  
Next, the averaged equations for the dispersed phase are derived through two distinct approaches: the particle-averaged (or Lagrangian-based) formalism, and the phase-averaged method. 
One important conclusion of this work is the demonstration of the relationship between the particle-averaged and phase-averaged equations. 
We show that the dispersed phase-averaged equations can be interpreted as a series expansion of the particle-averaged moment equations. 
We then present a "hybrid" set of equations, consisting of phase-averaged equations for the continuous fluid phase, complemented by an arbitrary number of moment conservation equations for the dispersed phase.
To further illustrate the methodology, we derive the mass, momentum, second moment of mass and first moment of momentum equations for droplets or bubbles suspended in a Newtonian fluid. 
In particular, we highlight the role of the second-order moment of mass equation and first-order moment of momentum equation, which link droplet deformation to the stresslet. 
We then derive closure laws in the dilute, viscous-dominated regime, with particular emphasis on the effects of surface tension gradients.
Additionally, we discuss several covariance closure terms that emerge in the averaged equations. 
Finally we demonstrate how the leading order deformation of the droplets can be obtained thanks to the second-order mass moment and first moment of momentum equation.
\end{abstract}
% Notably, the non-convective flux inside the inclusion and of the interface does not appear in the conservation law using this formulation. 
\end{document}

\documentclass[12pt,a4paper]{article}
\usepackage[T1]{fontenc}
\usepackage{times}
\usepackage{graphicx}
\usepackage{xcolor}
\usepackage{fancyhdr}
\usepackage{natbib}
\usepackage{amssymb}
\usepackage{amsmath}
\usepackage{amsthm}
\usepackage{mathrsfs}
\usepackage{empheq}
\usepackage{bm}
\newcommand{\size}{0.22\textwidth}
\newcommand{\avg}[1]{\left<#1\right>}
\renewcommand{\avg}[1]{\left<#1\right>}
\newcommand{\Exp}[1]{\overline{\overline{#1}}}
\newcommand{\davg}[1]{\left<#1\right>_d}
\newcommand{\cavg}[1]{\left<#1\right>_c}
\newcommand{\pavg}[1]{\avg{\delta_\alpha #1}}
% \newcommand{\pnavg}[1]{n\left<#1\right>_p}

\newcommand{\avgcond}[1]{\left<#1\right>}
\renewcommand{\avgcond}[1]{\overline{#1}}
\newcommand{\condavg}[2]{\overline{#1}^{#2}}
\newcommand{\ravg}[1]{\avgcond{#1}^\textbf{r}}
\newcommand{\Tavg}[1]{\avgcond{#1}^T}
\newcommand{\Xavg}[1]{\avgcond{#1}^X}
\newcommand{\TXavg}[1]{\Tavg{\Xavg{#1}}}
\newcommand{\kavg}[1]{\avgcond{#1}^k}
\newcommand{\Iavg}[1]{\avgcond{#1}^I}
\newcommand{\pnnavg}[1]{\avgcond{#1}^{p}}
\newcommand{\pnavg}[1]{n_p\pnnavg{#1}}
\newcommand{\oneavg}[1]{\avgcond{#1}^1}
\newcommand{\twoavg}[1]{\avgcond{#1}^2}
\newcommand{\smallavg}[2]{\avgcond{#1}^{#2}}
\newcommand{\sym}[1]{\text{Sym}\left[#1\right]}

\newcommand{\nstavg}[1]{\overline{#1}^{nst}}
\newcommand{\nstrelavg}[1]{\overline{#1}^{nst}_{rel}}
\newcommand{\mavg}[1]{\left<#1\right>_m}
\newcommand{\gavg}[2][\gamma]{\left<#2\right>_{#1}}
\newcommand{\partials}[1]{\partial_{i_1}\partial_{i_2}\ldots\partial{i_{#1}}}
\newcommand{\partialp}[2]{ \prod_{m=#1}^{#2} \partial_{i_m}}
\newcommand{\hatpartialp}[2]{ \prod_{m=#1}^{#2} \hat{\partial}_{j_m}}
\newcommand{\hatpartialpi}[2]{ \prod_{m=#1}^{#2} \hat{\partial}_{i_m}}
\newcommand{\pri}[2]{ \prod_{m=#1}^{#2} r_{i_m}}
\newcommand{\prj}[2]{ \prod_{m=#1}^{#2} r_{j_m}}

\newcommand{\grad}{\mathbf{\nabla}}
\renewcommand{\div}{\mathbf{\nabla}\cdot}
\newcommand{\gradI}{\mathbf{\nabla}_{||}}
\newcommand{\divI}{\mathbf{\nabla}_{||}\cdot}

\newcommand{\ddt}{\frac{d}{dt}}
% \renewcommand{\ddt}{d_t}
\newcommand{\pddt}{\frac{\partial}{\partial t}}
\newcommand{\Dt}{D_t}
\newcommand{\pddx}{\frac{\partial}{\partial \textbf{x}}}
\newcommand{\pddr}{\frac{\partial}{\partial \textbf{r}}}
\newcommand{\pddy}{\frac{\partial}{\partial \textbf{y}}}
\newcommand{\pddw}{\frac{\partial}{\partial \textbf{w}}}
\renewcommand{\pddx}{{\partial_\textbf{x}}}
\renewcommand{\pddr}{{\partial_\textbf{r}}}
\renewcommand{\pddy}{{\partial_\textbf{y}}}
\renewcommand{\pddw}{{\partial_\textbf{w}}}
\newcommand{\pdda}{\partial_a}
\renewcommand{\pddt}{\partial_t}
\newcommand{\norm}[1]{\hat{#1}}
\newcommand{\Jump}[1]{\llbracket #1 \rrbracket \cdot \textbf{n}_k }
\renewcommand{\Jump}[1]{\sum_{k=d,f} \left[#1\right] \cdot \textbf{n}_k }

\newcommand{\intO}[1]{\int_{V_\alpha} #1 dV}
% \renewcommand{\intO}[1]{ ( #1 )_{\Omega}}
\newcommand{\intS}[1]{\oint_{S_\alpha} #1 dS}
% \renewcommand{\intS}[1]{ ( #1 )_{\Sigma}}
\newcommand{\pOavg}[1]{\pavg{\intO{#1}}}
\newcommand{\pSavg}[1]{\pavg{\intS{#1}}}
\newcommand{\pMavg}[1]{\mathscr{M}\!\!\left[#1\right]}
\newcommand{\pMOavg}[1]{\mathscr{M}_\Omega\!\!\left[#1\right]}
\newcommand{\pMSavg}[1]{\mathscr{M}_\Sigma\!\!\left[#1\right]}
\newcommand{\CC}{\mathscr{C}}
\newcommand{\PP}{\mathscr{P}}
\newcommand{\FF}{\mathscr{F}}

%%% Utiliser pour les commentaires
\newcommand{\JL}[1]{\color{red}#1\color{black}}
\newcommand{\DL}[1]{\color{green}#1\color{black}}
\newcommand{\tb}[1]{\color{blue}#1\color{black}}
% \renewcommand{\alpha}{}
% \renewcommand{\tb}[1]{}

\renewcommand{\size}[1]{0.3\textwidth}
\newcommand{\expo}[2][n]{\frac{(-1)^#1}{#1!} \partialp{1}{#1} \pavg{\int_{\Omega_\alpha} \pri{1}{#1}#2 d\Omega}}
\newcommand{\expoU}[2][n]{\frac{(-1)^#1}{#1!} \partialp{1}{#1} \pavg{\textbf{u}_\alpha\int_{\Omega_\alpha} \pri{1}{#1}#2 d\Omega}}
\newcommand{\expoS}[2][n]{\frac{(-1)^#1}{#1!} \partialp{1}{#1} \pavg{\int_{\Gamma_\alpha} \pri{1}{#1}#2 d\Sigma}}

% \newcommand{\numref}[1]{\ref{#1}}
% \renewcommand{\ref}[1]{\autoref{#1}}

% FORMAT OF PAGE
\addtolength{\textheight}{4.5cm}
\addtolength{\topmargin}{-1.5cm}
\addtolength{\footskip}{0cm}
\addtolength{\textwidth}{5cm}
\addtolength{\evensidemargin}{-2.5cm}
\addtolength{\oddsidemargin}{-2.5cm}

\definecolor{mygray}{gray}{0.6}
\renewcommand\refname{\textbf{\large References}}


\begin{document}

\pagestyle{fancy}
\fancyhf{}

\lhead{\textcolor{mygray}{12th International Conference on Multiphase flow}}
\rhead{\textcolor{mygray}{ICMF 2025, Toulouse, France, May 12-16, 2025}}
\lfoot{}
\cfoot{}
\rfoot{}

% TITLE IN CAPITAL LETTERS
\begin{center}
{\large {\bf Theoretical calculation of the droplet induced agitation (or pseudoturbulence) in mono disperse buoyant emulsions for low inertia and dilute regime.}}
\vspace{10pt}

% AUTHORS

\underline{Nicolas Fintzi}$^1$, Daniel Lhuillier$^2$, Jean-Lou Pierson$^1$\\
% AFFILIATIONS
{\it
$^1$IFP Energies Nouvelles, Solaize, 69360, France,jean-lou.pierson@ifpen.fr\\
$^2$Affiliation of second author: University, School, Department, Institute, City, Country, E-mail\\
%$^3$Affiliation of third author: University, School, Department, Institute, City, Country, E-mail\\
}
\end{center}

\vspace{10pt}
\noindent{\bf {\large Abstract}}:\\
Buoyancy-driven droplet flows are encountered in many chemical engineering processes such as gravity separators and liquid-liquid extractors. 
The usual engineering practice to model such facilities is to use the averaged Navier-Stokes equations. 
The present work focuses on the velocity fluctuations tensor, also known as the \textit{Reynolds stress} tensor, or Pseudo-turbulent tensor, which is a crucial closure term in these equations.
Formally, this stress is defined as $\avg{ \textbf{u}' \textbf{u}'}_f$, where $\textbf{u}'$ is the fluid velocity fluctuation, and $\avg{\ldots}_f$ denotes an ensemble average procedure applied over the continuous phase. 
The \textit{Reynolds stress} can be separated into two contributions : (1) The agitation generated due to the averaged wakes around the particles; (2) All other sources of fluctuations, such as those generated through particle interactions, and single-phase turbulence \cite{du2022analysis}.
This study only considers the first contribution. 
Specifically we compute $\avg{ \textbf{u}' \textbf{u}'}_f$, for buoyant rising droplets in an otherwise quiescent fluid. 
The derivation is restricted to spherical droplets (of radius $a$), at small particle Reynolds number, and low particle volume fraction ($\phi$). 
    \vspace{10pt}

\noindent{{\bf Keywords}}: PTKE equations, hybrid model, Averaged equations, dispersed multiphase flows



For buoyant inviscid bubbly flows under the dilute assumption, it is known since the study of \cite{van1982bubble} that the \textit{Reynolds stress} can be written as
\begin{equation}
    \avg{\textbf{u}'\textbf{u}'}_f (\textbf{x})
    \approx
    n_p \int_{|\textbf{r}| > a}\textbf{v}\textbf{v}  d\textbf{r}
    = \phi \left(\frac{3}{20} U^2\textbf{I} + \frac{1}{20} \textbf{UU} \right),
    % \\+\underbrace{\int_{\mathbb{R}^3} \textbf{R} P_{f}^1(\textbf{x},\textbf{r}) d\textbf{r}}_\text{(2)}
    % - \phi_f \textbf{u}_f\textbf{u}_f,
    \label{eq1}
\end{equation}
where $n_p$ is the number of particles per unit of volume, $\textbf{U}$ is the averaged particle velocity, and $\textbf{v}$ is the disturbance velocity field of an isolated particle. 
The integral in Equation (\ref{eq1}) could be computed since $\textbf{v} \sim \mathcal{O}(r^{-3})$, with $r =|\textbf{r}|$ in the case of potential flows. 
However, the disturbance velocity field of a translating droplet in Stokes flows (see Figure \ref{fig:wake}) decays as $\sim \mathcal{O}(r^{-1})$. 
Therefore, the integral in Equation (\ref{eq1}) cannot be computed for Stokes flows. 
This is the main reason why no theoretical model exists for $\avg{\textbf{u}'\textbf{u}'}_f$ in the limit of low Reynolds number. 

To bypass the problem of divergent integrals we make use of the  \textit{Nearest Particle Statistics} framework recently revisited by \cite{zhang2021ensemble}. 
In this context we can express the \textit{Reynolds stress} in terms of nearest-particles averaged quantities, which yields
\begin{equation*}
    \avg{\textbf{u}'\textbf{u}'}_f(\textbf{x},t)
    \approx
    \int_{r >a}  \textbf{v}  \textbf{v} P_\text{nst}^f(\textbf{r}|\textbf{x},t) d\textbf{r}.
\end{equation*}
\tb{maybe include v nst instead}
Where $P_\text{nst}^f(\textbf{r}|\textbf{x},t)$ is the probability that the nearest particle's center of mass is located at \textbf{r}, conditionally on the point \textbf{x} being occupied by the continuous phase.
In the homogeneous and dilute regime $P_\text{nst}^f(\textbf{r}|\textbf{x},t) = n_p e^{- \phi [(r/a)^3 - 8]}$ \cite{zhang2021ensemble}.
The rapid decay of $P_\text{nst}^f$ at large $r$ enables us to compute the integral of the disturbance velocity fields even in low inertia regimes. 
Carrying out the integration yields directly 
\begin{equation}
    \frac{\avg{\textbf{u}'_y\textbf{u}'_y}_f}{U^2}
    = \frac{\phi}{5(\lambda +1)^2}\left[
        \frac{7 \Gamma\left(1/3\right)}{12}(3\lambda+2)^2    \phi^{-1/3}
        - (17\lambda^2+22\lambda+7)
        \right]
        + \mathcal{O}(\phi^{4/3})
        \label{eq:results}
\end{equation}
\tb{results fr the trace }
where $\Gamma(1/3) = \int_0^\infty t^{z-1} e^{-t} dt$ is the Gamma function with $\Gamma(1/3)= 2.678$, $\lambda$ is the viscosity ratio between the dispersed and continuous phase, and $y$ is the direction of gravity. 
\begin{figure}[h!]
    \centering
    \includegraphics[height=0.25\textwidth]{image/Rising_Stokes.png}
    \includegraphics[height=0.3\textwidth]{image/HOMOGENEOUS_final/CA/KFliterature.pdf}
    \caption{(left) Streamlines of a translating droplet in Stokes flow. 
            (right) Dimensionless Reynolds stress for rising bubbly flows in the direction of gravity. 
            (dots) Experimental results of \cite{cartellier2009induced}. 
            Equation (\ref{eq1}) : Potential flow theory \cite{van1982bubble}. 
            Equation (\ref{eq:results}) Stokes flow theory. 
            }
    \label{fig:wake}
\end{figure}
As can be observed on Figure \ref{fig:wake}, the present theory is in very good agreement with the experimental results of \cite{cartellier2009induced}.  
Direct numerical simulations of rising buoyant emulsions have also been performed (with the code \texttt{http://basilisk.fr}) for $\lambda = 0.1,1,10$. Good agreement is also obtained. 

In conclusion we provided a \textit{Reynolds stress} closure valid in the dilute Stokes flow regime for arbitrary viscosity ratios. 

 

\bibliographystyle{abbrv}
\bibliography{Bib/bib_bulles.bib}
\end{document}


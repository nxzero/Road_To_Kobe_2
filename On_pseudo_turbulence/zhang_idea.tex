\section{Introduction}


We are seeking for an expression of the Reynolds stress term,
\begin{equation}
    \avg{\chi_f \textbf{u}'_f\textbf{u}_f'}[\textbf{x},t],
\end{equation}
which is evaluated at an arbitrary point \textbf{z} in space. 
We will see latter that it is easier to seek for a term relative to the volume averaged velocity \textbf{u} rather than $\textbf{u}_f$, to this end one can reformulate 
\begin{equation*}
    \avg{\chi_f \textbf{u}_f'\textbf{u}_f'}
    = \avg{\chi_f (\textbf{u}^0_f-\textbf{u}_f)(\textbf{u}^0_f-\textbf{u}_f)}
    % &= 
    % \avg{\chi_f \textbf{u}^0_f\textbf{u}^0_f}
    % - \phi_f \textbf{u}_f \textbf{u}_f 
    % + \phi_f \textbf{uu}
    % - \phi_f \textbf{uu},\\
    = 
    \avg{\chi_f \textbf{u}^*_f\textbf{u}^*_f}
    + \phi_f (\textbf{uu} - \textbf{u}_f \textbf{u}_f ). 
\end{equation*}
We have introduced the notation $\textbf{u}_f^*= \textbf{u}_f^0 - \textbf{u}$ to indicate that the local quantity $\textbf{u}_f^0$ is expressed relative to its  volume ensemble averaged counterpart, i.e. \textbf{u}.

We want to express this quantity as an integral of conditioned quantity. 
The condition may be anything, nevertheless since we are studying the effect of particles motion on this tensor we condition by the presence of a single particle. 
So that the integral is correctly normed we use the \textit{Nearest-Particle-Statistics} (NPS), introduced in \citep{zhang2021ensemble}.

For that purpose we introduce the generalized function, 
\begin{equation}
    \delta_N[\textbf{y},\textbf{y}+\textbf{r}]
    =  \sum_i^N \delta(\textbf{x}_i[\FF,t] - \textbf{y}- \textbf{r})h_i[\textbf{y},\FF,t].
\end{equation}
Details on the definition are given in \citet{zhang2021ensemble,fintzi2025}. 
As such $\delta_N$ is non-zero only in the situations where a particle center of mass is located at \textbf{y}+\textbf{r} and that it is the nearest particle to the point \textbf{y}. 
The key relation, 
\begin{equation}
    \int_{\mathbb{R}^3} \delta_N[\textbf{y},\textbf{r}] d^3 \textbf{r} = 1, 
\end{equation}
permits us to write, 
\begin{equation}
    \avg{\chi_f \textbf{u}^*_f\textbf{u}^*_f}[\textbf{y},t]
    =
    \int_{\mathbb{R}^3}
    \avg{(\chi_f \textbf{u}^*_f\textbf{u}^*_f) [\textbf{y},t] \delta_N [\textbf{y},\textbf{r},t]}
    d^3 \textbf{r}
\end{equation}
The term within bracket is now the average of the local quantity $\textbf{u}^*_f\textbf{u}^*_f$ at $(\textbf{y},t)$ knowing that fluid is present at \textbf{y} ($\chi_f$), and that the nearest particle to \textbf{y} is at $\textbf{y}+ \textbf{r}$. 

For the points $|\textbf{r}- \textbf{y}|<a$, with $a$ the radius of all particles, we know for sure that $\chi_f = 0$. 
When $|\textbf{r}- \textbf{y}|>a$ we know for sure that $\chi_f[\textbf{y},t]= 1$. 
Hence, 
\begin{equation}
    \int_{\mathbb{R}^3}
    \avg{(\chi_f \textbf{u}^*_f\textbf{u}^*_f) [\textbf{y},t] \delta_N [\textbf{y},\textbf{r},t]}
    d^3 \textbf{r}
    =
    \int_{|\textbf{r}-\textbf{y}|>a}
    \avg{\textbf{u}'\textbf{u}' \delta_N }[\textbf{y},\textbf{r},t]
    d^3 \textbf{r}
\end{equation}
So ideally one needs an expression for $\avg{\textbf{u}'\textbf{u}' \delta_N }[\textbf{y},\textbf{r}]$, which is the Reynolds averaged on every configuration where the nearest particle is at $\textbf{y} +  \textbf{r}$.

Because $\avg{\textbf{u}'\textbf{u}' \delta_N }[\textbf{y},\textbf{r}]$ lackes of Boundary condition we seek for an expression of the term, 
\begin{equation}
    \avg{\textbf{u}'\textbf{u}'[\textbf{x},t] \delta_N[\textbf{y},\textbf{r},t] },
\end{equation}
instead. 
Indeed, when it is evaluated at $\textbf{x} = \textbf{y}$ one fall backs on $\avg{\textbf{u}'\textbf{u}' \delta_N }[\textbf{y},\textbf{r}]$. 
In that configuration one can state that,
\begin{equation}
    \lim_{|\textbf{x}- \textbf{y}| ,  |\textbf{x}- \textbf{y}+\textbf{r}|\to\infty }
    \avg{\textbf{u}'\textbf{u}'[\textbf{x},t] \delta_N[\textbf{y},\textbf{r},t] }
    = 
    \avg{\textbf{u}'\textbf{u}'}[\textbf{x},t]
    \avg{\delta_N}[\textbf{y},\textbf{r},t] ,
\end{equation}
Because the particle-free zone and the particle at $\textbf{y}$ and $\textbf{y}+\textbf{r}$ have no influence on what happen on the average at \textbf{x} which is very far from these points. 
We then may be interested to compute, 
\begin{equation}
    \lim_{|\textbf{x}- \textbf{y}| ,  |\textbf{x}- \textbf{y}+\textbf{r}|\to\infty }
    \avg{\textbf{u}'\textbf{u}'[\textbf{x},t] \delta_N[\textbf{y},\textbf{r},t] }
    - \avg{\textbf{u}'\textbf{u}'}[\textbf{x},t]
    \avg{\delta_N}[\textbf{y},\textbf{r},t] 
    = \avg{\textbf{u}'\textbf{u}'}^N[\textbf{x},t] P_N[\textbf{y},\textbf{r},t] = 0
\end{equation}
which is the difference of ``kinetic energy'' when comparing to the conditioned situations to the ensemble averaged situations. 

\paragraph{summary:}
\begin{align*}
    \avg{\chi_f \textbf{u}'_f\textbf{u}_f'}[\textbf{y},t],
    &=
    \int_{|\textbf{r}-\textbf{y}|>a}
    \avg{\textbf{u}'\textbf{u}'}^N[\textbf{y},t] P_N[\textbf{y},\textbf{r},t]
    d^3 \textbf{r}
    + 
    \avg{\textbf{u}'\textbf{u}'}[\textbf{y},t]
    \int_{|\textbf{r}-\textbf{y}|>a}
    \avg{\delta_N}[\textbf{y},\textbf{r},t] 
    d^3 \textbf{r}\\
    &+ \phi_f (\textbf{uu} - \textbf{u}_f \textbf{u}_f ) 
\end{align*}
which is equivalent to write, 
\begin{align*}
    \avg{\chi_f \textbf{u}^0_f\textbf{u}_f^0}[\textbf{y},t]
    - \avg{\textbf{u}^0\textbf{u}^0}[\textbf{y},t]
    &=
    \int_{|\textbf{r}-\textbf{y}|>a}
    \avg{\textbf{u}'\textbf{u}'}^N[\textbf{y},t] P_N[\textbf{y},\textbf{r},t]
    d^3 \textbf{r}
    - \phi_d \textbf{uu} 
\end{align*}

\tb{
    The problem is that 
    \begin{align}
        \avg{ \textbf{u}^* \textbf{u}^*}[\textbf{y},t]
        &=
        \int_{\mathbb{R}^3}
        \avg{(\textbf{u}^*\textbf{u}^*   \delta_N) [\textbf{y},\textbf{r},t]}
        d^3 \textbf{r}\\
        \Longleftrightarrow 0 &=
        \int_{\mathbb{R}^3}
        \avg{(\textbf{u}^*\textbf{u}^*   \delta_N) [\textbf{y},\textbf{r},t]}
        - 
        \avg{\textbf{u}^*\textbf{u}^*}[\textbf{y},t] \avg{ \delta_N} [\textbf{y},\textbf{r},t]
        d^3 \textbf{r}\\
        % &+ 
        % \avg{\textbf{u}^*\textbf{u}^*}[\textbf{y},t] 
        % \int_{\mathbb{R}^3}
        % \avg{ \delta_N} [\textbf{y},\textbf{r},t]
        % d^3 \textbf{r}\\
    \end{align}
}

where $\textbf{u}^* = \textbf{u}^0 - \textbf{u}_f$ in opposition to $\textbf{u}_f' = \textbf{u}_f^0 - \textbf{u}_f$

\subsubsection{additional properties}
\begin{equation}
    \pavg{\textbf{u}_i'\textbf{u}_i'}[\textbf{y},t]
    +  \pOavg{\textbf{ww}}
    \approx
    \avg{\chi_d \textbf{u}_d^0 \textbf{u}_d^0}
    = 
    \avg{\textbf{u}^0 \textbf{u}^0}
    - \avg{\chi_f \textbf{u}_f^0 \textbf{u}_f^0}
\end{equation}
And so, 
\begin{equation}
    \pavg{\textbf{u}_i'\textbf{u}_i'}[\textbf{y},t]
    \approx
    - \int_{|\textbf{r}-\textbf{y}|>a}
    \avg{\textbf{u}'\textbf{u}'}^N[\textbf{y},t] P_N[\textbf{y},\textbf{r},t]
    d^3 \textbf{r}
    + \phi_d \textbf{uu} 
    - \avg{\chi_f \textbf{u}_f^0 \textbf{u}_f^0}
    - \pOavg{\textbf{ww}}
\end{equation}


\subsection{Unaveraged equations }
At the local non-averaged level we have,
\begin{align}
    \pddt \textbf{u}^0 + \div ( \textbf{u}^0\textbf{u}^0 + \bm\sigma^0_*)
    &= \textbf{f}^0,
    \;\;\\
    \bm\sigma^0_*
    &=
    -\sum_{k=\Gamma,f,d} \rho^{-1}_k \chi_k \bm\sigma_k^0
    =
      -  \rho_f^{-1} \chi_f \bm\sigma_f^0
    - \rho_d^{-1} (\chi_d \bm\sigma_d^0 + \delta_{\Gamma} \bm\sigma_{\Gamma}^0)\\
    \textbf{f}^0 &= 
    \textbf{g}
    +(\rho_d^{-1}-\rho_f^{-1}) \delta_{\Gamma}  \bm\sigma_f^0 \cdot \textbf{n}
    = 
    \textbf{g}
    +\kappa {\delta_{\Gamma}  \bm\sigma_f^0 \cdot \textbf{n}}
    % - \zeta^{-1} (\chi_d \bm\sigma_d^0 + \delta_{\Gamma} \bm\sigma_{\Gamma}^0)
\end{align}
$\zeta^{-1} \to $
\subsection{Ensemble averaged equation}
\begin{align}
    \pddt \textbf{u} + \div ( \textbf{u}\textbf{u} + \avg{\textbf{u}'\textbf{u}'}+ \bm\sigma_*)
    &= \textbf{f},\\
    \bm\sigma_*
    &=
    -\avg{\sum_{k=\Gamma,f,d} \rho^{-1}_k \chi_k \bm\sigma_k^0}\\
    \textbf{f} &= 
    \textbf{g}
    +\kappa \avg{\delta_{\Gamma}  \bm\sigma_f^0 \cdot \textbf{n}}
\end{align}
\subsection{Disturbance field}
Because $\textbf{u}^0 \textbf{u}^0 - \textbf{uu} = \textbf{uu}' + \textbf{u}' \textbf{u} + \textbf{u}' \textbf{u}'$
\begin{align}
    \pddt \textbf{u}'
    + \div (
         \textbf{u}\textbf{u}'
        +  \textbf{u}' \textbf{u}
        +  \textbf{u}' \textbf{u}'
        -\avg{\textbf{u}'\textbf{u}'}
        + \bm\sigma_*')
    = \textbf{f}',\\
    \bm\sigma_*'
    =
    -\sum_k \rho^{-1}_k \chi_k \bm\sigma_k^0
    + \avg{\sum_k \rho^{-1}_k \chi_k \bm\sigma_k^0  }\\
    \textbf{f}'
    =
    \kappa \delta_{\Gamma}  \bm\sigma_f^0 \cdot \textbf{n}
    - \kappa \avg{\delta_{\Gamma}  \bm\sigma_f^0 \cdot \textbf{n}}
    \\
\end{align}
\subsubsection{Energies}
\begin{align*}
    \pddt (\textbf{u}')^2/2
    + \div ( \textbf{u}(\textbf{u}')^2/2
    +  \textbf{u}' (\textbf{u}')^2/2
    + \textbf{u}'\cdot \bm\sigma_*')
    =
    - \textbf{u}'\textbf{u}' : \grad \textbf{u}
    + \textbf{u}' \cdot \div \avg{\textbf{u}'\textbf{u}'}
    + \bm\sigma_*':\grad \textbf{u}' +  \textbf{u}'\cdot \textbf{f}',
    \\
    \textbf{u}'\cdot \bm\sigma_*'
    =
    - \textbf{u}'\cdot \sum_k \rho^{-1}_k \chi_k \bm\sigma_k^0
    + \textbf{u}'\cdot \avg{\sum_k \rho^{-1}_k \chi_k \bm\sigma_k^0  }\\
    \textbf{u}'\cdot \textbf{f}'
    =
    \kappa   \textbf{u}' \cdot \delta_{\Gamma}  \bm\sigma_f^0 \cdot \textbf{n}
    - \kappa \textbf{u}' \cdot \avg{\delta_{\Gamma}  \bm\sigma_f^0 \cdot \textbf{n}}
    \\
\end{align*}
\subsubsection{REstress}
\begin{multline}
    \pddt (\textbf{u}'\textbf{u}')
    + \div (
         \textbf{u}\textbf{u}'\textbf{u}'
         +  \textbf{u}' \textbf{u}'\textbf{u}'
         + \textbf{u}'\bm\sigma_*'
         + ^\dagger\textbf{u}'\bm\sigma_*')
         \\
         = 
         +\textbf{u}'\div \avg{\textbf{u}'\textbf{u}'}
         +\textbf{u}'\div \avg{\textbf{u}'\textbf{u}'}^\dagger
         - \textbf{u}' (\textbf{u}'\cdot \grad)\textbf{u}
         - \textbf{u}' (\textbf{u}'\cdot \grad)\textbf{u}^\dagger\\
    + \bm\sigma_*'\cdot \grad \textbf{u}'
    +^\dagger \bm\sigma_*'\cdot \grad \textbf{u}'
    + \textbf{f}'\textbf{u}'
    + \textbf{u}'\textbf{f}'
\end{multline}

\section{Ensemble averaegd eq}


\subsubsection{Energies}
\begin{align*}
    \pddt k
    + \div ( \textbf{u}k
    +  \avg{\textbf{u}' (\textbf{u}')^2/2}
    + \avg{\textbf{u}'\cdot \bm\sigma_*'})
    =
    - \avg{\textbf{u}'\textbf{u}'} : \grad \textbf{u}
    % + \textbf{u}' \cdot \div \avg{\textbf{u}'\textbf{u}'}
    + \avg{\bm\sigma_*':\grad \textbf{u}'} + \avg{ \textbf{u}'\cdot \textbf{f}'},
    \\
    \avg{\textbf{u}'\cdot \bm\sigma_*'}
    =
    - \avg{\textbf{u}'\cdot \sum_k \rho^{-1}_k \chi_k \bm\sigma_k^0}\\
    \avg{\textbf{u}'\cdot \textbf{f}'}
    =
    \kappa   \avg{\textbf{u}' \cdot \delta_{\Gamma}  \bm\sigma_f^0 \cdot \textbf{n}}
    \\
\end{align*}
homogeneous steady-state, 
\begin{align*}
    \div \avg{\textbf{u}'\cdot \bm\sigma_*'}
    =
    \avg{\bm\sigma_*':\grad \textbf{u}'} + \avg{ \textbf{u}'\cdot \textbf{f}'},
    \\
    \avg{\textbf{u}'\cdot \bm\sigma_*'}
    =
    - \avg{\textbf{u}'\cdot \sum_k \rho^{-1}_k \chi_k \bm\sigma_k^0}\\
    \avg{\textbf{u}'\cdot \textbf{f}'}
    =
    \kappa   \avg{\textbf{u}' \cdot \delta_{\Gamma}  \bm\sigma_f^0 \cdot \textbf{n}}
    \\
\end{align*}
Maybe the mean part of the pressure does not cancel-out


\subsubsection{REstress}
\begin{multline}
    \pddt \avg{\textbf{u}'\textbf{u}'}
    + \div (
         \textbf{u}\avg{\textbf{u}'\textbf{u}'}
         +  \avg{\textbf{u}' \textbf{u}'\textbf{u}'}
         + \avg{\textbf{u}'\bm\sigma_*'}
         + ^\dagger\avg{\textbf{u}'\bm\sigma_*'})
         = 
        %  +\textbf{u}'\div \avg{\textbf{u}'\textbf{u}'}
        %  +\textbf{u}'\div \avg{\textbf{u}'\textbf{u}'}^\dagger
         - \avg{\textbf{u}'  \textbf{u}'}\cdot (\grad\textbf{u} + \grad\textbf{u}^\dagger)\\
    + \avg{\bm\sigma_*'\cdot \grad \textbf{u}'}
    +^\dagger \avg{\bm\sigma_*'\cdot \grad \textbf{u}'}
    + \avg{\textbf{f}'\textbf{u}'}
    + \avg{\textbf{u}'\textbf{f}'}
\end{multline}

In homogeneous steady-state regime we get, 
\begin{equation}
    % \pddt \avg{\textbf{u}'\textbf{u}'}
    % + \div (
        %  \textbf{u}\avg{\textbf{u}'\textbf{u}'}
        %  +  \avg{\textbf{u}' \textbf{u}'\textbf{u}'}
         \div (\avg{\textbf{u}'\bm\sigma_*'}
         + ^\dagger\avg{\textbf{u}'\bm\sigma_*'})
         = 
        %  +\textbf{u}'\div \avg{\textbf{u}'\textbf{u}'}
        %  +\textbf{u}'\div \avg{\textbf{u}'\textbf{u}'}^\dagger
        %  - \avg{\textbf{u}'  \textbf{u}'}\cdot (\grad\textbf{u} + \grad\textbf{u}^\dagger)\\
    + \avg{\bm\sigma_*'\cdot \grad \textbf{u}'}
    + ^\dagger \avg{\bm\sigma_*'\cdot \grad \textbf{u}'}
    + \avg{\textbf{f}'\textbf{u}'}
    + \avg{\textbf{u}'\textbf{f}'}
\end{equation}
Or, 
\begin{equation}
         \avg{\textbf{u}'\div \bm\sigma_*'}
         + ^\dagger\avg{\textbf{u}'\div \bm\sigma_*'}
         = 
    % + \avg{\bm\sigma_*'\cdot \grad \textbf{u}'}
    % + ^\dagger \avg{\bm\sigma_*'\cdot \grad \textbf{u}'}
    + \avg{\textbf{f}'\textbf{u}'}
    + \avg{\textbf{u}'\textbf{f}'}
\end{equation}
Because indeed, the pressure may not be constant, but rather have a constant derivatives.  


\begin{align}
    \avg{\textbf{u}' \textbf{f}'}
    &=
    \avg{\kappa   \textbf{u}'(\delta_{\Gamma}  \bm\sigma_f^* \cdot \textbf{n})}
    + \avg{\kappa   \delta_{\Gamma} \textbf{u}'  \textbf{n}}\cdot \bm\Sigma
\end{align}
these can be evaluated from the one particle problem. 

\section{Conditionally averaegd eq}
Notting 
\begin{align}
    \avg{\delta_N \textbf{u}'} = P_N \textbf{v}^N 
     && \textbf{u}'' = \textbf{u}' - \textbf{v}^N
\end{align}

\paragraph{Momentum }
\begin{align}
    \pddt (P_N \textbf{v}^N) 
    - \avg{\textbf{u}'\pddt \delta_N}
    + \div (
          P_N \textbf{u}\textbf{v}^N
        + P_N \textbf{v}^N \textbf{u}
        + P_N \textbf{v}^N \textbf{v}^N
        + \avg{\delta_N \textbf{u}''\textbf{u}''}
        - P_N \avg{\textbf{u}'\textbf{u}'}
        + \avg{\delta_N \bm\sigma_*'} )
    = \avg{\delta_N \textbf{f}'}
\end{align}
\paragraph{energy}
Let, $k^N P_N = \avg{\delta_N (\textbf{u}')^2 /2}$ 
\begin{multline*}
    \pddt (P_N k^N)
    - \avg{(\textbf{u}')^2/\pddt \delta_N }
    + \div ( P_N \textbf{u} k^N 
    +  P_N \textbf{v}^N  k^N 
    + \avg{\delta_N \textbf{u}'' (\textbf{u}')^2/2}
    + \avg{\delta_N \textbf{u}'\cdot \bm\sigma_*'})\\
    =
    - \avg{\delta_N \textbf{u}'\textbf{u}'} : \grad \textbf{u}
    + P_N \textbf{v}^N \cdot \div \avg{\textbf{u}'\textbf{u}'}
    + \avg{\delta_N \bm\sigma_*':\grad \textbf{u}'} 
    + \avg{\delta_N  \textbf{u}'\cdot \textbf{f}'} \\    
    \avg{\delta_N \textbf{u}'\cdot \bm\sigma_*'}
    =
    - \avg{\delta_N \textbf{u}'\cdot 
    \left(\sum_k \rho^{-1}_k \chi_k \bm\sigma_k^0
    - \avg{\sum_k \rho^{-1}_k \chi_k \bm\sigma_k^0  }
    \right)
    }
    \\
    \textbf{u}'\cdot \textbf{f}'
    =
    \avg{\kappa \delta_N   \textbf{u}' \cdot \left(\delta_{\Gamma}  \bm\sigma_f^0 \cdot \textbf{n}
    - \avg{\delta_{\Gamma}  \bm\sigma_f^0 \cdot \textbf{n}}\right)}
    \\
\end{multline*}
\tb{we must retrive the mean PTKE to this equation so that the unkown goes to 0. }
\paragraph{Re stress }
\begin{multline}
    \pddt \avg{\delta_N \textbf{u}'\textbf{u}'}
    - \avg{\textbf{u}'\textbf{u}' \pddt \delta_N }
    + \div (
         \textbf{u} \avg{\delta_N \textbf{u}'\textbf{u}'}
         +  \textbf{v}^N \avg{\delta_N \textbf{u}'\textbf{u}'}
         + \avg{\delta_N \textbf{u}'' \textbf{u}'\textbf{u}'}
        + \avg{\delta_N \textbf{u}'\bm\sigma_*'}
        + ^\dagger \avg{\delta_N \textbf{u}'\bm\sigma_*'}
        %  + \textbf{q}^N
         )
         \\
         = 
         +P_N \textbf{v}^N\div \avg{\textbf{u}'\textbf{u}'}
         +P_N \textbf{v}^N\div \avg{\textbf{u}'\textbf{u}'}^\dagger
         - \avg{\delta_N \textbf{u}' \textbf{u}'}\cdot \grad\textbf{u}
         - \avg{\delta_N \textbf{u}' \textbf{u}'}\cdot \grad\textbf{u}^\dagger\\
    + \avg{\delta_N \bm\sigma_*'\cdot \grad \textbf{u}'}
    + \avg{\delta_N ^\dagger \bm\sigma_*'\cdot \grad \textbf{u}'}
    + \avg{\delta_N \textbf{f}'\textbf{u}'}
    + \avg{\delta_N \textbf{u}'\textbf{f}'}
\end{multline}

\subsection{Closure for the energy equation in dilute uniform regime}

For any points where the probability of begin in the fluids is 100\% we then have 
\begin{equation}
    \rho_f \delta_N \bm\sigma_*'
    =
    \delta_N [p_f' \bm\delta
    - \mu_f (
        \grad \textbf{u}' 
        + ^\dagger \grad \textbf{u}' 
        )]
\end{equation}
\begin{equation}
    \avg{\delta_N \textbf{u}'\cdot \bm\sigma_*'}
    =
    - \avg{\delta_N \textbf{u}'\cdot [p_f' \bm\delta
    - \mu_f (
        \grad \textbf{u}' 
        + ^\dagger \grad \textbf{u}' 
        )]}
\end{equation}

for the energy equation we have
\begin{align*}
    \rho_f (\textbf{u}' \bm\sigma_*')_{iki}
    &=
    (\textbf{u}')_i (
        p_f' \delta_{ki}
    - \mu_f \grad_k (\textbf{u}')_i
    - \mu_f  \grad_i (\textbf{u}')_k
    )\\
    &= p_f'(\textbf{u}')_k  
    - \mu_f \grad_k (\textbf{u}'\textbf{u}')_{ii}/2
    - \mu_f (\textbf{u}')_i \grad_i (\textbf{u}')_k
    \\
    % &= p_f'[(\textbf{u}')_i  \delta_{kj}+ (\textbf{u}')_j  \delta_{ki}]
    % - \mu_f \grad_k (\textbf{u}'\textbf{u}')_{ij}
    % - \mu_f [
    %     \grad_j (\textbf{u}'_i\textbf{u}'_k) + \grad_i (\textbf{u}'_j \textbf{u}'_k) 
    % ]\\
    % &+ \mu_f [
    %     \textbf{u}'_k (\grad_j \textbf{u}'_i + \grad_i \textbf{u}'_j  )
    % ]
    % \\
\end{align*}
Also, 
\begin{equation}
    \bm\sigma_*' : \grad \textbf{u}'
    = 
    -\frac{\mu_f}{\rho_f}
    (\grad \textbf{u}'+ ^\dagger \grad \textbf{u}'): \grad \textbf{u}'
\end{equation}

Or for the energy
\begin{align*}
    \grad_k [\rho_f (\textbf{u}' \bm\sigma_*')_{iki}]
    &=
    \grad_k (p_f'(\textbf{u}')_k  )
    - \mu_f \grad_k^2 (\textbf{u}'\textbf{u}')_{ii}/2
    - \mu_f\grad_k[ (\textbf{u}')_i \grad_i (\textbf{u}')_k]\\
    &=
    \frac{1}{\rho_f} \grad_k (p_f'(\textbf{u}')_k  )
    - \frac{\mu_f}{\rho_f}\grad_k^2 (\textbf{u}'\textbf{u}')_{ii}/2
    - \frac{\mu_f}{\rho_f}\grad_k (\textbf{u}')_i \grad_i (\textbf{u}')_k\\
\end{align*}
\begin{equation}
    \textbf{u}'\cdot\div\bm\sigma'_*
    = 
    \textbf{u}'\cdot \grad p_f'
    - \frac{\mu_f}{\rho_f} \textbf{u}'\cdot (\grad^2 \textbf{u}' + \grad\div \textbf{u}')
    =
    \textbf{u}'\cdot \grad p_f'
    - \frac{\mu_f}{\rho_f} \textbf{u}'\cdot \grad^2 \textbf{u}'
\end{equation}
And 
\begin{equation}
    \textbf{u}'_k \grad_i \grad_i \textbf{u}'_k
    =
   \grad_i ( \grad_i \textbf{u}'_k\textbf{u}'_k /2)
   - \grad_i \textbf{u}'_k \grad_i \textbf{u}'_k
   =
   \grad^2 (\textbf{u}')^2/2
   - \grad_i \textbf{u}'_k \grad_i \textbf{u}'_k
\end{equation}
So this is consistent. 
Then, 
If we consider an 
\begin{itemize}
    \item homogeneous, steady state ensemble averaged situations. 
    \item So $\grad \textbf{A} = 0$ with $A$ an ensemble averaged quantities. 
    \item All terms including $\textbf{u}''$ may be neglected. 
\end{itemize}

\begin{equation*}
    % \pddt (P_N k^N)
    % - \avg{(\textbf{u}')^2/\pddt \delta_N }
    P_N(\textbf{u}+ \textbf{v}^N ) \cdot\grad k^N 
    % + \div \avg{\delta_N \textbf{u}'' (\textbf{u}')^2/2}
    -P_N \frac{\mu_f}{\rho_f} \grad^2k^N
    =
    -\frac{1}{\rho_f}\div\avg{\delta_N p_f'\textbf{u}'}
    % - \avg{\delta_N \textbf{u}'\textbf{u}'} : \grad \textbf{u}
    % + P_N \textbf{v}^N \cdot \div \avg{\textbf{u}'\textbf{u}'}
    -\frac{\mu_f}{\rho_f} \avg{\delta_N \grad_k\textbf{u}'_i:\grad_k \textbf{u}'_i} 
\end{equation*}
\tb{this is true only of the particles doesn't move with a non zero $\textbf{u}$ otherwise $\textbf{u}\to\textbf{u}_r$}
Either I treat each of these as closure either the second one may be considered as closure
How to derive the BC now ? 
\paragraph{Boundary condition}
At the local level I have for spherical particles 
\begin{align}
    \textbf{u}_d^0 -  \textbf{u}_f^0&=0 \\
    [\textbf{u}_d^0- \textbf{u}_\alpha]\cdot \textbf{n} &= 0\\
    (\bm\delta - \textbf{nn})\cdot  [ \bm e_f^0- \lambda \bm e_d^0]\cdot \textbf{n} &= 0\\
\end{align}
In terms of disturbance field this gives, 
\begin{align}
    \textbf{u}_d' -  \textbf{u}_f'&=0 \\
    [\textbf{u}_d' + (\textbf{u} - \textbf{u}_\alpha)]\cdot \textbf{n} &= 0\\
    (\bm\delta - \textbf{nn})\cdot  [\bm e_f'- \lambda \bm e_d' + (1-\lambda) \textbf{e}]\cdot \textbf{n} &= 0\\
\end{align}
Multiplying these equaiton by $\textbf{u}_f'$ or $\textbf{u}_d'$ may give Boundary for kinetic energy equaiton.

Using the first BC one may state that we don't care either it is $\textbf{u}_f'$ or $\textbf{u}_d'$, so let us multiply by $\textbf{u}'$ 
\begin{align}
    \textbf{u}_d'\textbf{u}_d' -  \textbf{u}_f'\textbf{u}_f'&=0 \\
    [\textbf{u}_{f/d}'\textbf{u}_{f/d}' + \textbf{u}_{d/f}'(\textbf{u} - \textbf{u}_\alpha)]\cdot \textbf{n} &= 0\\
    (\bm\delta - \textbf{nn})\cdot  [\textbf{u}_f'\bm e_f'- \lambda \textbf{u}_d'\bm e_d' + (1-\lambda) \textbf{u}_{f/d}'\textbf{e}]\cdot \textbf{n} &= 0\\
\end{align}

The first integral is possible because the velocity is continuous at the surface of the part. 
when conditioned by the state of the droplet and evalutaed at the points where the particle is we obtain 
\begin{align}
    \avg{\delta_N \textbf{u}_d'\textbf{u}_d'} -  \avg{\delta_N \textbf{u}_f'\textbf{u}_f'}&=0 \\
    [\avg{\delta_N \textbf{u}_{f/d}'\textbf{u}_{f/d}' }+ P_N \textbf{v}^N (\textbf{u} - \textbf{w} )]\cdot \textbf{n} &= 0\\
    (\bm\delta - \textbf{nn})\cdot  [\textbf{u}_f'\bm e_f'- \lambda \textbf{u}_d'\bm e_d' + (1-\lambda) \textbf{u}_{f/d}'\textbf{e}]\cdot \textbf{n} &= 0\\
\end{align}
and note that,
\begin{equation}
    2(\textbf{u}' \textbf{e}')_{ikj} 
    = 
    \textbf{u}'_i(\grad_k \textbf{u}'_j+ ^\dagger \grad_j \textbf{u}'_k)
    =
    (\grad_k \textbf{u}'_i\textbf{u}'_j+ ^\dagger \grad_j \textbf{u}'_i\textbf{u}'_k)
    - (\textbf{u}'_j\grad_k + ^\dagger \textbf{u}'_k\grad_j )\textbf{u}'_i
\end{equation}
Maybe teh first two BCs are suffficient 
This BC must be completed using the stress BC, 
\begin{align}
    \textbf{u}_d' -  \textbf{u}_f'&=0 \\
    [\textbf{u}_d' + (\textbf{u} - \textbf{u}_\alpha)]\cdot \textbf{n} &= 0\\
    (\bm\delta - \textbf{nn})\cdot  [\bm e_f'- \lambda \bm e_d' + (1-\lambda) \textbf{e}]\cdot \textbf{n} &= 0\\
\end{align}

\subsection{Solving the equation for solid parts}
The system to solve then becomes when $Re \ll 0$, 
\begin{align*}
    % \pddt (P_N k^N)
    % - \avg{(\textbf{u}')^2/\pddt \delta_N }
    % (\textbf{u}+ \textbf{v}^N ) \cdot\grad k^N 
    % + \div \avg{\delta_N \textbf{u}'' (\textbf{u}')^2/2}
    - \frac{\mu_f}{\rho_f} \grad^2k^N
    &=
    -\frac{1}{\rho_f}\div ( p_f^N \textbf{v}^N )
    % - \avg{\delta_N \textbf{u}'\textbf{u}'} : \grad \textbf{u}
    % +  \textbf{v}^N \cdot \div \avg{\textbf{u}'\textbf{u}'}
    -\frac{\mu_f}{\rho_f}   \grad_k\textbf{v}^N _i:\grad_k \textbf{v}^N _i \\
    \lim_{r \to \infty} k^N &= k \\
    k^N &= \textbf{u}_r \cdot \textbf{u}_r/2
\end{align*}


The solution for $k^N$ is then 
\begin{equation}
    k^N(\textbf{r})
    = 
    \textbf{A}(\textbf{r}): \textbf{u}_r \textbf{u}_r 
    + k(\textbf{r})
\end{equation}
with $\textbf{A}(\textbf{r})$ going to 0 as $r$ goes to infinity. 
Then 
\begin{equation}
    k(\textbf{x}) = \int_{out}
    (\textbf{A}(\textbf{r}): \textbf{u}_r \textbf{u}_r 
    + k(\textbf{r})) P_N d^3\textbf{r}\\
    =
    -\phi^{1/3} \Gamma(2/3)
    + \frac{9 3^{1/2} \phi^{2/3} \Gamma(1/3)  }{16}
    + k 
\end{equation}
if $k$ is uniform we are stuck, but then the NPS tells us that it is not how that work. 
Indeed, at the location \textbf{r} we are far from a particle hence the kinetic energy isn't supposed to tend to anything. 

\begin{equation}
    \lim_{|\textbf{z}- \textbf{y}|\to\infty }k^N(\textbf{z},\textbf{x},\textbf{y})= k(\textbf{z})
\end{equation}
if i substitu


\subsection{closure for Re stress equation}

all of this is under the sign of the divergence so that, 
\begin{align*}
    \grad_k [
        \rho_f (\textbf{u}' \bm\sigma_*')_{ikj}
        + \rho_f (\textbf{u}' \bm\sigma_*')_{jki}
    ]
    &=
    [\grad_j ((\textbf{u}')_i  p_f')+\grad_i ((\textbf{u}')_j p_f')]
    - \mu_f \grad_k^2 (\textbf{u}'\textbf{u}')_{ij}
    - \mu_f \grad_k [
        (\textbf{u}'_i \grad_j+ \textbf{u}'_j \grad_i )\textbf{u}'_k
    ]\\
    &=
    [\grad_j ((\textbf{u}')_i  p_f')+\grad_i ((\textbf{u}')_j p_f')]
    - \mu_f \grad_k^2 (\textbf{u}'\textbf{u}')_{ij}
    - \mu_f (\grad_j\textbf{u}'_k \grad_k \textbf{u}'_i+ \grad_i\textbf{u}'_k\grad_k \textbf{u}_j')  
    \\
\end{align*}
\begin{align*}
    \rho_f (\textbf{u}' \bm\sigma_*')_{ikj}
    + \rho_f (\textbf{u}' \bm\sigma_*')_{jki}
    &=
    (\textbf{u}')_i (
        p_f' \delta_{kj}
    - \mu_f \grad_k (\textbf{u}')_j
    - \mu_f  \grad_j (\textbf{u}')_k
    )
    + (\textbf{u}')_j (
        p_f' \delta_{ki}
    - \mu_f \grad_k (\textbf{u}')_i
    - \mu_f  \grad_i (\textbf{u}')_k
    )\\
    &= p_f'[(\textbf{u}')_i  \delta_{kj}+ (\textbf{u}')_j  \delta_{ki}]
    - \mu_f \grad_k (\textbf{u}'\textbf{u}')_{ij}
    - \mu_f [
        (\textbf{u}'_i \grad_j+ \textbf{u}'_j \grad_i )\textbf{u}'_k
    ]
    \\
    % &= p_f'[(\textbf{u}')_i  \delta_{kj}+ (\textbf{u}')_j  \delta_{ki}]
    % - \mu_f \grad_k (\textbf{u}'\textbf{u}')_{ij}
    % - \mu_f [
    %     \grad_j (\textbf{u}'_i\textbf{u}'_k) + \grad_i (\textbf{u}'_j \textbf{u}'_k) 
    % ]\\
    % &+ \mu_f [
    %     \textbf{u}'_k (\grad_j \textbf{u}'_i + \grad_i \textbf{u}'_j  )
    % ]
    % \\
\end{align*}


\section{Periodic green function}
\subsubsection{One pts conditional average}
The equaiton for the disturbance field in the pts particle approximation reads, 
\begin{align}
    \grad^2 \textbf{u} - \grad p + \textbf{f} \sum_{n\in \mathbb{Z}^3} \delta(\textbf{x}-\textbf{x}_n) - n_p \textbf{f} = 0 && \div \textbf{u} =0 
\end{align}
Indeed the disturbance fields are subject to the points force minus the mean forces. 
FT of these equations yields, 
\begin{align}
    -k^2 \textbf{u} - i\textbf{k} p + \textbf{f}\frac{8\pi^3}{V}\sum_{n\in \mathbb{Z}^3} \delta(\textbf{k}-\textbf{k}_n) - 8\pi^3 n_p \textbf{f}\delta(\textbf{k})= 0 
    && -i\textbf{k}\cdot  \textbf{u} =0 
\end{align}
In a single Periodic cell we may evaluate $n_p = 1/V$ because there is a single particle. 
\begin{equation}
    -k^2 \textbf{u} - i\textbf{k} p + \textbf{f}\frac{8\pi^3}{V}\sum_{n\in \mathbb{Z}^3\neq 0} \delta(\textbf{k}-\textbf{k}_n) = 0 
\end{equation}

Because the dirac at $\textbf{k}_n = 0$ is properly canceled we may have, 
\begin{align*}
    p =  -  i \frac{8\pi^3}{V} \frac{\textbf{f}\cdot \textbf{k}}{k^2}\sum_{n\in \mathbb{Z}^3} \delta(\textbf{k}-\textbf{k}_n)
    &&
    -i\textbf{k} p =  - \frac{8\pi^3}{V} \textbf{f}\cdot \frac{\textbf{kk}}{k^2}\sum_{n\in \mathbb{Z}^3} \delta(\textbf{k}-\textbf{k}_n)
\end{align*}
Hence, \textbf{u} may be evaluated as, 
\begin{equation}
    \textbf{u}(\textbf{k})
    =
    \frac{8\pi^3}{V} \textbf{f}\cdot \frac{\bm\delta k^2 - \textbf{kk}}{k^4}\sum_{n\in \mathbb{Z}^3\neq 0} \delta(\textbf{k}-\textbf{k}_n) 
    % + \textbf{f}\frac{8\pi^3}{V}\sum_{n\in \mathbb{Z}^3\neq 0} \delta(\textbf{k}-\textbf{k}_n) = 0 
\end{equation}
This fourier transform can be inverted quite easly, 
\begin{equation}
    \textbf{u}(\textbf{x})
    =
    \frac{1}{8\pi^3}\int \textbf{u}(\textbf{k}) e^{i\textbf{k}\cdot \textbf{x}} d^3\textbf{k}
    =
    \frac{1}{V} \textbf{f}\cdot \sum_{n\in \mathbb{Z}^3\neq 0} \frac{\bm\delta k_n^2 - \textbf{k}_n\textbf{k}_n}{k_n^4} e^{i \textbf{k}_n \cdot \textbf{x}}
    % + \textbf{f}\frac{8\pi^3}{V}\sum_{n\in \mathbb{Z}^3\neq 0} \delta(\textbf{k}-\textbf{k}_n) = 0 
\end{equation}
\paragraph{Sedimentation velocity}
To compute the Sedimentation velocity we remove the singularity at the origin and consider the 


We deduce that the periodic green func is, 
\begin{equation}
    \textbf{G}(\textbf{x})
    =
    \frac{1}{V}\sum_{n\in \mathbb{Z}^3\neq 0} \frac{\bm\delta k_n^2 - \textbf{k}_n\textbf{k}_n}{k_n^4} e^{i \textbf{k}_n \cdot \textbf{x}}
    % + \textbf{f}\frac{8\pi^3}{V}\sum_{n\in \mathbb{Z}^3\neq 0} \delta(\textbf{k}-\textbf{k}_n) = 0 
\end{equation}

The fluid velocity at $\textbf{r} = 0$ is 
\begin{equation}
    \textbf{G}(\textbf{x})
    =
    \frac{1}{V}\sum_{n\in \mathbb{Z}^3\neq 0} \frac{\bm\delta k_n^2 - \textbf{k}_n\textbf{k}_n}{k_n^4} 
    % + \textbf{f}\frac{8\pi^3}{V}\sum_{n\in \mathbb{Z}^3\neq 0} \delta(\textbf{k}-\textbf{k}_n) = 0 
\end{equation}

The convolution theorem, 
\begin{align*}
    \int \textbf{uu} d^3\textbf{x}
    &=
    \frac{1}{V^2} \textbf{ff}: \sum_{n\in \mathbb{Z}^3\neq 0} 
    \frac{\bm\delta k_n^2 - \textbf{k}_n\textbf{k}_n}{k_n^4} 
    \frac{\bm\delta k_n^2 - \textbf{k}_n\textbf{k}_n}{k_n^4} \\
    &=
    \frac{1}{V^2} \textbf{ff}: \sum_{n\in \mathbb{Z}^3\neq 0} 
    \frac{\bm\delta\bm\delta k_n^4 
    - \bm\delta k_n^2\textbf{k}_n\textbf{k}_n
    - \textbf{k}_n\textbf{k}_n\bm\delta k_n^2
    + \textbf{k}_n\textbf{k}_n\textbf{k}_n \textbf{k}_n }{k_n^8} 
\end{align*} 
which yield an explicit formula for the pseudoturbulence. 

Let us assume that,
\begin{equation}
    \frac{V}{2\pi}\textbf{k} = n_1\textbf{e}^{(1)}+n_2\textbf{e}^{(1)}+n_3\textbf{e}^{(1)}
\end{equation}
if the force is along $\textbf{e}^{(1)}$ then, 
\begin{equation}
    \textbf{ff}: 
    \frac{\bm\delta\bm\delta k_n^4 
    - \bm\delta k_n^2\textbf{k}_n\textbf{k}_n
    - \textbf{k}_n\textbf{k}_n\bm\delta k_n^2
    + \textbf{k}_n\textbf{k}_n\textbf{k}_n \textbf{k}_n }{k_n^8} 
    = 
    \frac{\textbf{e}^{(1)}\textbf{e}^{(1)} k_n^4 
    -  k_n^2 n_1\textbf{e}^{(1)}\textbf{k}_n
    -  k_n^2 n_1\textbf{k}_n\textbf{e}^{(1)}
    + n_1^2 \textbf{k}_n \textbf{k}_n }{k_n^8} 
\end{equation}
the velocity varience on $\textbf{e}_1$ becomes,
\begin{equation}
    \sum_{n\in \mathbb{N}^3} \frac{k_n^4 + n_1^4
    -  2k_n^2 n_1^2
      }{k_n^8} 
      = 
\end{equation}




\section{Introduction}


In this chapter we focus on the modeling of what is called the \textit{Pseudo-turbulent} stress tensor, also called the \textit{Reynolds stress}. 
Note that the former terminology is more appropriate since we consider the modeling of the velocity fluctuation that are generated by the particles. 
In other words we concentrate on the modeling of the velocity fluctuation generated by the wakes or the disturbance velocity fields of the droplets. 

Our first approach of modeling will be entirely theoretical.
Then in a second step, we extend the model obtained in the previous step with we use of the results obtained with the DNS presented in the last chapters.  

The \textit{Pseudo-turbulence} stress is the averaged local values of the velocity fluctuation generated by the particles present in the flow. 
Therefore, we must somehow model the disturbance field generated by a particle immersed in an arbitrary flows and then average on the carrier fluid phase to obtain the \textit{Pseudo-turbulence} stress.
In the first step we will consider only mean relative motion between the droplets and the carrier fluid, instead of an arbitrary flow. 

In this restricted situation the \textit{Pseudo-turbulence} stress is the average of the velocity fluctuation square, generated due to a droplet in translation relative to a quiescent fluid.
Within this context, \citet{van1998pseudo} computed a \textit{Pseudo turbulence} stress closure model for bubbles rising in a quiescent fluid. 
His model was based on the potential flow solution of the wake generated by a translating bubble in stokes flows.
In a first step we would like to carry out the same king of analysis but for a droplet translating in stokes flows instead of potential flows.
However, in this regime the wake generated by the droplet is proportional to $\mathcal{O}(r^{-1})$ with $\textbf{r}$ the distance from the droplet center of mass to a point in space. 
This slow decay of the disturbance field prevent us to use the same statistical considerations than \citet{van1998pseudo} did since the same method leads to divergent integral in stokes flow regime. 


To tackle this issue, we present in a first section the classical approach of \citet{van1998pseudo} but applied to the wake of a particle in stokes flows. 
Doing so, we demonstrate why dose this method cannot work and why  does the \textit{Pseudo-turbulence} stress computed with this method is a divergent integral. 
In a second step, we present the \textit{Nearest particle Statistics} framework of \citet{zhang2021ensemble} with slight adjustment. 
We demonstrate how the ensemble averaged \textit{Pseudo turbulent} stress is related to the \textit{nearest conditional} averaged wakes of the particles. 
Then, the \textit{nearest conditional} averaged wakes of the particles is assumed to follow the equation of an isolated particle translating in Stokes flow condition. 
This enables us to compute the \textit{Pseudo turbulent} stress from the relation just mentioned. 
Finally, we obtained a closure term of the same kind as the one of \citet{van1998pseudo} but for stokes flows and depending on the viscosity ratio.  

After validation of this model we propose to extend it validity by first showing how it is related to the dispersed phase velocity fluctuation tensor. 
Then, we show how to take in account mean shearing motion from the fluid phase. 
Finally, using the DNS results we try to extend the original closure for translating droplets for finite $Re$ and $\phi$. 
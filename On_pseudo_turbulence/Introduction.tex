
\section{Introduction}


In this chapter, we focus on the modeling of what is called the \textit{pseudo-turbulent} stress tensor, or \textit{Reynolds stress} tensor. 
The former terminology is more appropriate since we consider, in this chapter, the modeling of the velocity fluctuations that are generated by the motion of the particles. 
In other words, we focus on the modeling of the velocity fluctuation generated by the wakes or the disturbance velocity fields of the droplets. 

% Our first approach of modeling will be entirely theoretical.
% Then in a second step, we extend the model obtained in the previous step with we use of the results obtained with the DNS presented in the last chapters.  

The \textit{pseudo-turbulence} stress tensor represents the averaged local values of the velocity fluctuation generated by the motion of the particles present in the flow. 
Therefore, we first need to model the disturbance field generated by a particle immersed in an arbitrary flow and then average it over the carrier fluid phase to obtain the \textit{pseudo-turbulence} stress. 
To simplify the problem, we consider here only uniform relative motions between the droplets and the carrier fluid. 
% Subsequently, we will explore the possibility of extending this approach to arbitrary flow fields, based on the general singularity solution of \citet{kim2013microhydrodynamics}. 

In this restricted situation the \textit{pseudo-turbulence} stress corresponds to the velocity variance, generated due to a droplet in translation relative to a quiescent fluid.
Within this context, \citet{biesheuvel1984two,vanvan1998pseudo,zhang1994ensemble} computed a \textit{pseudo turbulence} stress closure model for mono-disperse bubbles rising in a quiescent fluid under the potential flow assumption. 
% His model is based on the potential flow solution of the wake generated by a translating bubble in stokes flows.
We perform a similar analysis for a droplet translating in Stokes flow instead of potential flow. In this regime, the wake generated by the droplet decays as $\mathcal{O}(r^{-1})$ with $\textbf{r}$ is the distance from the droplet's center of mass to a point in space. 
This slow decay of the disturbance field prevents us from using the same statistical considerations as \citet{van1998pseudo}  since this approach leads to divergent integrals in the Stokes flow regime, as discussed by \citet{caflisch1985variance}. 


To address this issue, we first present the classical approach of \citet{van1998pseudo} applied to the wake of a particle in Stokes flow. 
By doing so, we demonstrate why this method fails and why the computed \textit{pseudo-turbulence} stress results in a divergent integral. 
Next, we extend the \textit{Nearest Neighbor Statistics} framework of \citet{zhang2021ensemble} and show how the ensemble-averaged \textit{pseudo-turbulent} stress tensor is connected to the \textit{nearest neighbor conditionally averaged} wakes of the particles. 
We then demonstrate that the \textit{nearest neighbor conditionally averaged} disturbance velocity field around a particle satisfies the \textit{nearest neighbor conditionally averaged} momentum and mass equation. 
Since these equations are unsolvable in their general form, we consider a dilute emulsion and neglect the effects of inertia.
Solving these equations allows us to compute the \textit{pseudo-turbulent} stress tensor in the dilute and Stokes regime. 
At this stage, we obtain a closure term of the \textit{pseudo-turbulent} stress tensor adapted for Stokes flow and dependent on the viscosity ratio, the particle phase velocity variance and the particle fluid mean drift velocity.  


After validating this model by comparing it to the DNS results, we attempt to extend the original closure for translating droplets to account for finite $Re$ and $\phi$ based on DNS results.
This new semi-empirical model is then shown to exhibit very good agreement to both experimental and numerical studies in the literature. 

% Finally, based on theoretical grounds, we extend the model's applicability by incorporating mean shearing motion from the fluid phase into the \textit{nearest neighbor conditionally averaged} momentum and mass equations. 
% This results in a fully closed model that accounts for the mean gradients of the fluid phase.
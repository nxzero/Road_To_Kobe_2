\section{Conclusion}

In this study, we provided a semi-empirical formula to predict the values of the \textit{Reynolds} stress tensor for steady-state buoyant rising emulsions of mono-dispersed droplets. 
We then provided quantitative validation based on experimental and numerical studies found in the literature. 
The major advancements can be summarized as follows : 
\begin{enumerate}
    \item We first demonstrated why the usual method to derive continuous phase averaged ensemble quantities could not be applied to the closure of the \textit{Reynolds} stress tensor. 
    While this fact was known a long time ago, we provided a clear proof of why it is the case, and a solution to avoid this issue. 
    \item Based on the \textit{Nearest particle statistics} framework we have shown that the ensemble-averaged \textit{Reynolds} stress could be expressed through the integration of the \textit{nearest-neighbor conditionally averaged} wake around a droplet \eqref{eq:relation_ensemble_nst}. 
    After deriving the \textit{nearest-neighbor conditionally averaged} Stokes equations we could obtain an analytical formula for the \textit{nearest-neighbor conditionally averaged} wake around a droplet.
    This enabled us to compute the \textit{Reynolds} stress tensor in the low inertia and dilute regime for an arbitrary viscosity ratio $\lambda$. 
    \item  Then, we made use of DNS of buoyant rising emulsions of mono-dispersed droplets to extend the validity of this \textit{Reynolds} stress closure, given by \ref{eq:functional_form_avg}, to arbitrary $Re$ and $\phi$. 
    Outstanding agreements are obtained comparing our model to the present model in the literature.  
\end{enumerate}
Although, \ref{eq:functional_form_avg} provides us with an algebraic model for $\avg{\chi_f \textbf{u}_f'\textbf{u}_f'}$ it still needs the values of $\pavg{\textbf{u}_\alpha' \textbf{u}_\alpha'}$ to be computed. 
Therefore, the next step must be the creation of a model for the particle phase velocity variance, $\pavg{\textbf{u}_\alpha' \textbf{u}_\alpha'}$. 
This will be treated in a future study. 

Finally, we would like to reiterate that we have only examined the effect of mean uniform relative motion on $\avg{\chi_f \textbf{u}_f'\textbf{u}_f'}$. However, as shown in \ref{chap
}, $\avg{\chi_f \textbf{u}_f'\textbf{u}_f'}\sim \phi \textbf{E}_f \cdot \textbf{E}_f+\mathcal{O}(Re)$.
In other words, further work is necessary to develop a semi-empirical model for the \textit{Reynolds} stress in terms of mean shear at higher \textit{Reynolds} numbers.


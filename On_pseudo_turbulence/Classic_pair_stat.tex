\section{Closure with one-point statistics, and why it does not work for Stokes flows.}

The classical method used to close theoretically ensemble average terms, such as the \textit{Reynolds} stress tensor, namely 
\begin{equation*}
    \avg{\chi_f \textbf{u}_f' \textbf{u}_f'}[\textbf{x},t],
\end{equation*} 
is to use conditional averaged variables.
We recall that $\avg{\ldots}$ is an ensemble average procedure and that $\textbf{u}_f' = \textbf{u}_f^0 - \textbf{u}_f$ with,  $\textbf{u}_f^0$ and $\textbf{u}_f$ being the local and averaged velocity respectively.
This is what \citet{van1998pseudo} did in order to take in account the wake generated by translating bubble in potential flows. 
Here we revisit his method and show why it doesn't work for stokes flows. 

We introduce the distribution $\delta_1[\textbf{x},\textbf{w},\FF,t]$, which is non-null at the coordinate $\textbf{y}$ and $\textbf{w}$ as soon as a particle center of mass $\textbf{x}_i$ is located at $\textbf{y}$ with a center of mass velocity $\textbf{u}_i$ equal to $\textbf{w}$, namely, 
\begin{equation*}
    \delta_1[\textbf{x},\textbf{w},\FF,t] =\sum_i^N \delta(\textbf{x}_i[\FF,t] - \textbf{y})\delta(\textbf{u}_i[\FF,t] - \textbf{w}),
\end{equation*}
where $N$ is the total number of particles in the flow. 
Using this distribution and the fact that, 
\begin{equation*}
    \int_{\mathbb{R}^6}
    \delta_1
    d \textbf{y}
    d \textbf{w}
    = 1,
\end{equation*}
we can show without any assumption that, 
\begin{align}
    \avg{\chi_f \textbf{u}_f' \textbf{u}_f'}[\textbf{x},t]
    &= \frac{1}{N}
    \int_{\mathbb{R}^6}
    \avg{\delta_1\chi_f \textbf{u}_f' \textbf{u}_f'}[\textbf{x},t]
    d\textbf{y}
    d\textbf{w}\\
    &= 
    \frac{1}{N}
    \int_{\mathbb{R}^6}
    \textbf{u}_f^{1d}
    \textbf{u}_f^{1d}
    P_{1f}
    d\textbf{y}
    d\textbf{w}
    + 
    \frac{1}{N}
    \int_{\mathbb{R}^6}
    % \textbf{u}_f^{1d}
    % \textbf{u}_f^{1d}
    \avg{\delta_1 \chi_f\textbf{u}_f'' \textbf{u}_f''}
    d\textbf{y}
    d\textbf{w}
    \label{eq:classic_avg}
\end{align}
Where we have introduced,  
\begin{align}
    P_{1f} [\textbf{w},\textbf{y},\textbf{x},t]
    % = n_p[\textbf{y},\textbf{w},t] \phi_f^1[\textbf{x},t|\textbf{y},\textbf{w}], 
    =
    \avg{\chi_f \delta_1},
    % \textbf{u}_f^{1d}P_{1f} [\textbf{w},\textbf{y},\textbf{x},t]
    % = n_p[\textbf{y},\textbf{w},t] \phi_f^1[\textbf{x},t|\textbf{y},\textbf{w}], 
    % =
    % \avg{\chi_f \delta_1\textbf{u}_f^0}
    % - P_{1f} \textbf{u}_f,
\end{align}
as the probability of having a particle with velocity \textbf{w} at \textbf{y} while the continuous phase is present at \textbf{x}.
Notice that $P_{1f}$ can be subdivided into two distribution, namely, 
\begin{equation}
    P_{1f} [\textbf{w},\textbf{y},\textbf{x},t]
    = n_p[\textbf{y},\textbf{w},t] \phi_f^1[\textbf{x},t|\textbf{y},\textbf{w}],
\end{equation}
where $n_p$ is the number density and $\phi_f^1$ the fluid phase volume fraction at $\textbf{x}$ knowing a particle is present at $\textbf{y}$. 
Notice that for identical spherical particles of radius $a$, $\phi_f^1 = 0\; \forall \;|\textbf{x}-\textbf{y}| < a$, reducing the domain of integration in \ref{eq:batchlor_avg} from $\mathbb{R}^3$ to $|\textbf{x}-\textbf{y}| > a$. 
We also defined: 
The velocity fluctuation of the local value around the ensemble average : $\textbf{u}_f' = \textbf{u}_f^0 - \textbf{u}_f$;
The fluctuation of the local velocity value around the single particle conditional average $\textbf{u}_f'' = \textbf{u}_f^0 - \textbf{u}_f^1$;
The fluctuation of the single particle conditional average around the ensemble average : $\textbf{u}_f^{1d} = \textbf{u}_f^1 - \textbf{u}_f$.
In these definitions we have used the definition, $\textbf{u}_f^1 =\avg{\chi_f \delta_1 \textbf{u}_f^1}$.  
The latter definition corresponds to the averaged distance fields generated due to the particles at $\textbf{y}$ with velocity $\textbf{w}$. 
Based on these definitions and \ref{eq:classic_avg} we state that we can separate the \textit{Reynolds stress} into two distinct contribution :  (1) the agitation generated due to the averaged wakes around the particles; (2) all other source of fluctuations such as those generated through particles interactions and the single phase turbulence. 
The second contribution of \ref{eq:classic_avg} is shown to be of $\mathcal{O}(\phi^2)$ for the wake of a translating particle in potential flows \citet[Appendix A]{zhang1994averaged}.
Therefore, we assert that this second term might always be neglected, even for stokes flows. 
Regarding the first integral of this expression, we need the explicit expression of $\textbf{u}_f^{1d}$, and $P_{1f}$ to compute it.

Notice the presence of $N$, the total number of particles in \ref{eq:classic_avg}. 
In principle, we do not have this information, as millions of particles might be present in the industrial process at hand. 
We may try to reformulate, this constant with the number density term. 
Indeed, as the number density $n_p$ is just the number of particle per unit of volume, we may write, $n_p / N = 1/V$, where $V$ is the total volume of our process. 
Considering this last transformation, \citet{eq:classic_avg} reduce to a volume average over the entire $V$ domain considered here. 
However, just like $N$, $V$ is arbitrary here, as we focus on no particular processes. 

This last point is unpractical, and as discussed in length in \ref{chap:daniel2} we use the method originally introduced in \citet{batchelor1972sedimentation} to reformulate this conditional average which solve partially this issue. 
Indeed, using the hypothesis of \textit{additivity} of the particles wakes we arrive at the formula(see \ref{chap:daniel2} and \citet{batchelor1972sedimentation}), 
\begin{align}
    \avg{\chi_f \textbf{u}_f' \textbf{u}_f'}[\textbf{x},t] =
    % \int_{\mathbb{R}^6}
    % \avg{\delta_1\chi_f \textbf{u}_f' \textbf{u}_f'}[\textbf{x},t]
    % d\textbf{y}
    % d\textbf{w}\\
    % &= 
    \int_{\mathbb{R}^6}
    \textbf{u}_f^{1d}
    \textbf{u}_f^{1d}
    P_{1f}
    d\textbf{y}
    d\textbf{w}
    + 
    \int_{\mathbb{R}^6}
    % \textbf{u}_f^{1d}
    % \textbf{u}_f^{1d}
    \avg{\delta_1 \chi_f\textbf{u}_f'' \textbf{u}_f''}
    d\textbf{y}
    d\textbf{w}
    +
    \text{Error}
    \label{eq:batchlor_avg}
\end{align}
with, 
\begin{equation}
    \text{Error}
    = 
    \int_{\mathbb{R}^6}
    \avg{\sum_i
    \chi_f (\textbf{u}_f^0\textbf{u}_f^0)
    }\cdot\mathcal{O}(|\textbf{r}|)
    d\textbf{r}
    d\textbf{w}
    \label{eq:error}
\end{equation}
without going into the details, this formulation get rid of the number of particles $N$, but the counterpart is that it generated an error proportional to the integral over $\mathbb{R}^3$ of $\mathcal{O}(|\textbf{r}|)$.
Equation (2.10) \citet{batchelor1972sedimentation} is in fact equivalent to \ref{eq:batchlor_avg}, but is derived simply based on physical reasoning.
While \ref{eq:batchlor_avg} follows a rigorous ensemble average derivation, which enabled us to derive explicitly the error generated in that expression. 
This, is related to two hypotheses made in this derivation: (1) The hypotheses of additivity mentioned above, and (2) the hypotheses of homogeneity, i.e. the variables does, not depends on $\textbf{x}$, and $t$. 
Consequently, the formula \ref{eq:batchlor_avg} with the explicit expression of the ``Error'' term is original, the derivation can be found in \ref{chap:daniel3}. 
% The main difference between, \ref{eq:batchlor_avg} and (2.10) and \citet{batchelor1972sedimentation} is that the latter stipulated that the error was proportional to $\mathcal{O}(\phi^2)$ while we show that it is not as simple. 
We thus, generalized (2.10) of \citet{batchelor1972sedimentation} which stipulated that the error of \ref{eq:batchlor_avg} was only of $\mathcal{O}(\phi^2)$ under the condition that the integrals converge, since we now have an explicit expression for the error which is $\mathcal{O}(r)$. 


In \citet{van1998pseudo} they only consider the first term on the right-hand side of \ref{eq:batchlor_avg}. 
As firstly demonstrated by \citet{hinch1977averaged}, we can prove rigorously that, at $\mathcal{O}(\phi)$, $\textbf{u}_f^{1d}$ follows the equations of an isolated particle infinite medium of carrier fluid. 
In \ref{chap:daniel2}, we indeed proves with more details that $\textbf{u}_f^{1d}$ follows a set of \textit{Conditionally averaged equations}, and that by neglecting all $\mathcal{O}(\phi)$ terms we recover the system of equation describing an isolated particle. 
The solution of the wake of an isolated droplet in translation in potential flow is known. 
Therefore, following \citet{van1998pseudo,zhang1994averaged} we may use the approximation, 
\begin{equation}
    \textbf{u}_d^{1d}[\textbf{r},\textbf{w}]
    = 
    \frac{\textbf{U}}{2}\cdot \left[
        \frac{\bm\delta}{r^3}-\frac{3\textbf{rr}}{r^5}
    \right]
    + \mathcal{O}(\phi)
    \label{eq:potential_sol}
\end{equation}
where $\textbf{U} = \textbf{w} - \textbf{u}_f[\textbf{y},t]$ is the relative velocity, with $\textbf{u}_f$ the mean fluid phase velocity evaluated at the center of mass of the particle $\textbf{y}$. 
Using this velocity fields in \ref{eq:batchlor_avg} and considering that $\textbf{u}_f^{1d}\sim \frac{1}{r^3}$ at the leading order, yields an ``Error'' term of,  
\begin{equation*}
    \text{Error}
    \sim
    \int_{\mathbb{R}^3}
    \mathcal{O}(|\textbf{r}|/r^6)
    d\textbf{r}
    = \text{finite}. 
\end{equation*}
In light of this result, we conclude that due to the rapid decay of $\textbf{u}_f^{1d}$ in potential flow the error produced is therefore finite, enabling \ref{eq:batchlor_avg} to provide a physical result.

Indeed, by direct integration of the first term on the right-hand side of \citet{eq:batchlor_avg} we obtain, 
\begin{equation}
    \avg{\chi_f \textbf{u}_f'\textbf{u}_f'}
    = \phi \left\{
        \frac{1}{20}[\textbf{u}_{fp}\textbf{u}_{fp}+ \frac{1}{n_p}\pavg{\textbf{u}_\alpha\textbf{u}_\alpha}]
        + 
        \frac{3}{20} (\textbf{u}_{fp}\cdot \textbf{u}_{fp} + 2k_p)\bm\delta
    \right\},
    \label{eq:van_wingarden_sol}
\end{equation} 
where we have defined, $ \textbf{u}_{fp} = \textbf{u}_f - \textbf{u}_p$ as the relative phase velocity, and $\textbf{u}_\alpha' = \textbf{u}_\alpha - \textbf{u}_p$ and the particles fluctuation velocity. 
Notice that all the distances have been made dimensionless with the radius $a$ of the particles. 
This is the original closure obtained by \citet{van1998pseudo} with the addition of the particle phase velocity fluctuations, through the terms $\pavg{\textbf{u}_\alpha\textbf{u}_\alpha}$ and $k_p = \frac{1}{n_p}\pavg{\textbf{u}_\alpha'\cdot\textbf{u}_\alpha'}$ that have been found by \citet{zhang1994averaged}. 
The latter two terms can be obtained noticing that, 
\begin{equation}
    \int_{\mathbb{R}^3} \textbf{ww} n_p[\textbf{y},\textbf{w}] d\textbf{w}
    = \pavg{\textbf{u}_\alpha\textbf{u}_\alpha}. 
\end{equation}
For a better understanding we discard the particle phase velocity fluctuation until otherwise mentioned. 

Let us know consider the Stokes flow regime.
In this case we can prove that $\textbf{u}_f^{1d}$ follows the \textit{Conditionally averaged} Stokes flow equations. 
At $\mathcal{O}(\phi)$, it can be shown (see \ref{chap:daniel2}) that $\textbf{u}_f^{1d}$ corresponds to the velocity field of an isolated droplet.
Since we consider only uniform relative motions with the carrier phase, we may directly write, 
\begin{equation}
    \textbf{u}_f^{1d}
    = 
    \frac{1}{4}\left(\frac{3\lambda + 2}{\lambda +1}\right)
    \left(\frac{ \bm\delta}{r} + \frac{\textbf{rr}}{r^3}\right)  \cdot \textbf{U}
    - 
    \frac{1}{4}\left(\frac{\lambda}{\lambda +1}\right)
    \left(-\frac{\bm\delta}{r^3} + \frac{3 \textbf{rr} }{r^5}\right) 
    \cdot \textbf{U}
    + \mathcal{O}(\phi)
    \label{eq:stokes_sol}
\end{equation}
Again, in this formula the vector $\textbf{r}$ is made dimensionless using the radius $a$ of the particles. 
As already discussed in several studies in the literature \citet{caflisch1985variance}, 
in the Stokes flow regime $\textbf{u}_f^{1d} \sim \frac{1}{r}$, making the first integral on the right-hand side of \ref{eq:batchlor_avg} unable to converge. 
Indeed, in that case, 
\begin{equation}
    \int \textbf{u}_f^{1f} \textbf{u}_f^{1f}  d \textbf{y} = \infty.
    \label{eq:non_convergence}
\end{equation}
Notice that we have considered a finite but constant number density $n_p$ in this integral. 
This, was attributed to an infinite energy generated by the wake of an isolated droplet in Stokes flow \citet{caflisch1985variance}. 
Note that at finite Reynolds number the same problem is found \citet{koch1993hydrodynamic}. 
While this analysis remains true we would like to revisit this discussion with the help of the formula \ref{eq:batchlor_avg}, that is apparently new. 

Firstly, we would like to clarify a point of the most importance. 
Using, the solution provided by \ref{eq:stokes_sol}, does not imply that we are considering a single droplet translating in an infinite medium.
Indeed, it just witness of the fact that   $\textbf{u}_f^{1d}$ at order $\mathcal{O}(1)$ in $\phi$ follows the same equations as the wake of an isolated droplet. 
This means, that the arguments of \citet{caflisch1985variance} is true for the case were we consider such an isolated droplet in an infinite medium.
However, as ``isolated'' $\neq$ ``dilute regime'', this justification is not valid in our case.
Indeed, we are not considering a single particle here, but rather an infinite number of particles, in a dilute medium.   

In other words, it is logic that the variance of an isolated particle's wake is infinite since: in terms of pure math it is what we obtain, and physically it is not inconsistent since isolated particles in an infinite medium simply does not exist.
Indeed, in real life the continuous phase domain is always bounded, making the generation of an infinite amount of energy due to the particle wake impossible. 
Real stokes flows does not exist either, but since we know that this is not the source of the problem \citet{koch1993hydrodynamic} we discard this fact for now. 
Thus, we believe that the conclusion of \citet{caflisch1985variance} is in fact not in contradiction with physical principle since they are computing the variance of a situation that cannot exist in real application. 
So the question is rather, why does our methodology seem to lead us to the computation of the variance of an isolated particle, where $\phi =0$; while we considered all along the derivation a finite, but non-vanishing, volume fraction $\phi$ ?

We believe that the answer is in fact purely mathematical rather than physical. 
Indeed, thanks to the explicit derivation of the ``Error'' term of \ref{eq:batchlor_avg} which was not discovered up to now, we are now able to understand the source of this inconsistency. 
Absolutely, in Stokes flow, at $\mathcal{O}(\phi)$, $\textbf{u}_f^{1d} \sim 1/r$ at the leading order, thus in this situation, 
\begin{equation}
    \text{Error}
    = 
    \int_{\mathbb{R}^3} 
    \mathcal{O}(1/r) d\textbf{r}
    = \infty. 
    \label{eq:real_error}
\end{equation}
In light of \ref{eq:real_error} the error generated by the wake of a dilute emulsion of translating particles in Stokes flow is in fact infinite, making \ref{eq:batchlor_avg} unable to provide consistent results. 


To conclude, against previous studies, especially \citet{caflisch1985variance}, we assert that the non-converge issue encountered in the derivation of $\avg{\chi_d \textbf{u}_f'\textbf{u}_f'}$ is only due to the poor accuracy of Batchelor's formula \eqref{eq:batchlor_avg}, rather than a physical issue due to the wake of particle in Stokes flows. 
Thus, in the following sections, we propose to use another method than, \ref{eq:batchlor_avg} and \ref{eq:classic_avg} to reformulate the ensemble average quantities, as these traditional approaches are either inaccurate or inapplicable to our specific situation. 






\section{Pseudo-turbulence closure with one-point statistics, and why it does not work for Stokes flows}

The classical method used to close theoretically ensemble average terms, such as the \textit{Reynolds} stress tensor, namely 
\begin{equation*}
    \avg{\chi_f \textbf{u}_f' \textbf{u}_f'}[\textbf{x},t],
\end{equation*} 
is to use conditional averaged method \citet{van1998pseudo,zhang1994ensemble}.
We recall that $\avg{\ldots}$ is an ensemble average procedure and that $\textbf{u}_f' = \textbf{u}_f^0 - \textbf{u}_f$ with,  $\textbf{u}_f^0$ and $\textbf{u}_f$ being the local and ensemble averaged velocity field respectively.

As defined in \ref{chap:daniel2} we introduce the distribution $\delta_1[\textbf{y},\textbf{w},\FF,t]$, which is defined as being non-zero when a particle with its center of mass at $\textbf{y}_i[\FF,t]$ is located at $\textbf{y}$ and its center of mass velocity $\textbf{u}_i[\FF,t]$ equal to $\textbf{w}$, namely, 
\begin{equation*}
    \delta_1[\textbf{y},\textbf{w},\FF,t] =\sum_i^N \delta(\textbf{x}_i[\FF,t] - \textbf{y})\delta(\textbf{u}_i[\FF,t] - \textbf{w}),
\end{equation*}
where $N$ is the total number of particles in the flow, $\FF$ a flow configuration and $t$ the current time. 
Using this distribution and the fact that, 
\begin{equation*}
    \int_{\mathbb{R}^6}
    \delta_1
    d \textbf{y}
    d \textbf{w}
    = N,
\end{equation*}
we can show without any assumption that, 
\begin{align}
    \avg{\chi_f \textbf{u}_f' \textbf{u}_f'}[\textbf{x},t]
    &= \frac{1}{N}
    \int_{\mathbb{R}^6}
    \avg{\delta_1\chi_f \textbf{u}_f' \textbf{u}_f'}[\textbf{x},t]
    d\textbf{y}
    d\textbf{w}\\
    &= 
    \frac{1}{N}
    \int_{\mathbb{R}^6}
    \textbf{v}_f^{1}
    \textbf{v}_f^{1}
    P_{1f}
    d\textbf{y}
    d\textbf{w}
    + 
    \frac{1}{N}
    \int_{\mathbb{R}^6}
    % \textbf{u}_f^{1d}
    % \textbf{u}_f^{1d}
    \avg{\delta_1 \chi_f\textbf{u}_f'' \textbf{u}_f''}
    d\textbf{y}
    d\textbf{w}
    \label{eq:classic_avg}
\end{align}
Where we have introduced,  
\begin{align}
    P_{1f} [\textbf{w},\textbf{y},\textbf{x},t]
    % = n_p[\textbf{y},\textbf{w},t] \phi_f^1[\textbf{x},t|\textbf{y},\textbf{w}], 
    =
    \avg{\chi_f \delta_1},
    % \textbf{u}_f^{1d}P_{1f} [\textbf{w},\textbf{y},\textbf{x},t]
    % = n_p[\textbf{y},\textbf{w},t] \phi_f^1[\textbf{x},t|\textbf{y},\textbf{w}], 
    % =
    % \avg{\chi_f \delta_1\textbf{u}_f^0}
    % - P_{1f} \textbf{u}_f,
\end{align}
as the probability of finding a particle with velocity \textbf{w} at \textbf{y} with the continuous phase present at \textbf{x}.
Note that $P_{1f}$ can be subdivided into two distribution, namely, 
\begin{equation}
    P_{1f} [\textbf{w},\textbf{y},\textbf{x},t]
    = n_p[\textbf{y},\textbf{w},t] \phi_f^1[\textbf{x},t|\textbf{y},\textbf{w}],
\end{equation}
where $n_p$ is the number density and $\phi_f^1$ the fluid phase volume fraction at $\textbf{x}$ knowing a particle is present at $\textbf{y}$. 
Note that for identical spherical particles of radius $a$, $\phi_f^1 = 0$ when $|\textbf{x}-\textbf{y}| < a$, reducing the domain of integration in \ref{eq:classic_avg} from $\mathbb{R}^3$ to $|\textbf{x}-\textbf{y}| > a$. 
In \ref{eq:classic_avg} we also defined: 
(1) The velocity fluctuation of the local value around the ensemble average : $\textbf{u}_f' = \textbf{u}_f^0 - \textbf{u}_f$;
(2) The fluctuation of the local velocity value around the single particle conditional average $\textbf{u}_f'' = \textbf{u}_f^0 - \textbf{u}_f^1$, where, $\textbf{u}_f^1 =\avg{\chi_f \delta_1 \textbf{u}_f^1}/P_{1f}$.  
(3) The fluctuation of the single particle conditional average with respect to the ensemble average : $\textbf{v}_f^{1} = \textbf{u}_f^1 - \textbf{u}_f$.
The latter definition corresponds to the averaged velocity disturbance fields generated due to the particles at $\textbf{y}$ with velocity $\textbf{w}$. 
Based on these definitions and \ref{eq:classic_avg} we state that we can separate the \textit{Reynolds stress} into two distinct contribution :  (1) the agitation generated due to the averaged wakes around the particles, represented by $\textbf{v}_f^{1}$; (2) all other source of fluctuations such as those generated through particles interactions and the single phase turbulence, represented by $\textbf{u}_f''$. 
The second contribution on the right-hand side of \ref{eq:classic_avg} is shown to be of $\mathcal{O}(\phi^2)$ for the wake of a translating particle in potential flows \citet[Appendix A]{zhang1994averaged}.
In the following, we assert that this second term might always be neglected, even for stokes flows. 
Regarding the first integral of this expression, we need the expression of $\textbf{v}_f^{1}$, and $P_{1f}$ to compute it.

Note the presence of $N$ in \ref{eq:classic_avg}, which is the total number of particles in the domain. 
In principle, we do not have this information, as millions of particles might be present in the industrial process at hand. 
We may try to reformulate $N$ using the number density $n_p$. 
Indeed, as the number density $n_p$ is just the number of particles per unit of volume, we may write, $n_p / N = 1/V$, where $V$ is the total volume of our process, assuming that $n_p$ is constant. 
With this last transformation, \ref{eq:classic_avg} reduces to a volume average over the entire volume $V$ of the domain. 
However, just like $N$, $V$ is an unknown here, is unknown in this context, as we are not focusing on any specific processes. 

As discussed in length in \ref{chap:daniel2}, we use the method originally introduced in \citet{batchelor1972sedimentation} to reformulate \ref{eq:classic_avg}. 
Using the hypothesis of \textit{additivity} of the particles wakes we arrive at the formula (see \ref{chap:daniel2} and \citet{batchelor1972sedimentation}), 
\begin{align}
    \avg{\chi_f \textbf{u}_f' \textbf{u}_f'}[\textbf{x},t] =
    % \int_{\mathbb{R}^6}
    % \avg{\delta_1\chi_f \textbf{u}_f' \textbf{u}_f'}[\textbf{x},t]
    % d\textbf{y}
    % d\textbf{w}\\
    % &= 
    \int_{\mathbb{R}^6}
    \textbf{v}_f^1
    \textbf{v}_f^1
    P_{1f}
    d\textbf{y}
    d\textbf{w}
    + 
    \int_{\mathbb{R}^6}
    % \textbf{u}_f^{1d}
    % \textbf{u}_f^{1d}
    \avg{\delta_1 \chi_f\textbf{u}_f'' \textbf{u}_f''}
    d\textbf{y}
    d\textbf{w}
    +
    \text{Error}
    \label{eq:batchlor_avg}
\end{align}
with, 
\begin{equation}
    \text{Error}
    = 
    \int_{\mathbb{R}^6}
    \avg{\sum_i
    \chi_f\delta_1 (\textbf{u}_f^0\textbf{u}_f^0) \mathcal{O}(|\textbf{r}|/L)
    }
    \cdot\mathcal{O}(|\textbf{r}|)
    d\textbf{r}
    d\textbf{w}
    \label{eq:error}
\end{equation}
without going into the details, this formulation gets rid of the number of particles $N$, but the counterpart is that it generated an error proportional to the integral over $\mathbb{R}^3$ of $\mathcal{O}(|\textbf{r}|/L)$.
% Equation (2.10) of \citet{batchelor1972sedimentation} is similar to \ref{eq:batchlor_avg}, but the former is derived simply based on physical reasoning.
% While \ref{eq:batchlor_avg} follows a rigorous ensemble average derivation, which enabled us to derive explicitly the error term. 
The error arises from two hypotheses made in this derivation: (1) the assumption of additivity, and (2) the assumption of homogeneity, which implies that the variables do not depend on $\textbf{x}$, and $t$. 
% The formula \ref{eq:batchlor_avg} with the explicit expression of the ``Error'' term is original, the derivation can be found in \ref{chap:daniel2}. 
% Thus, we generalized equation (2.10) of \citet{batchelor1972sedimentation} which stated that the error in \ref{eq:batchlor_avg} was only of $\mathcal{O}(\phi^2)$, under the condition that the integrals converge. 
% We extended this reasoning by providing an explicit expression for the error in cases where the integral does not converge, which is  $\mathcal{O}(r)$. 


In \citet{van1998pseudo} they only consider the first term on the right-hand side of \ref{eq:batchlor_avg}. 
As firstly demonstrated by \citet{hinch1977averaged}, we can prove rigorously that, at $\mathcal{O}(\phi)$, $\textbf{v}_f^1$ follows the equations of an isolated particle immersed in a pure Newtonian fluid. 
In \ref{chap:daniel2}, we indeed prove (including more details) that $\textbf{v}_f^{1}$ follows a set of \textit{Single-particle conditionally averaged equations}, and that by neglecting all $\mathcal{O}(\phi)$ terms, we indeed recover the system of equations describing an isolated particle. 
% The solution of the wake of an isolated droplet in translation in potential flow is known. 
Therefore, following \citet{van1998pseudo,zhang1994averaged} we may use in potential flow the approximation, 
\begin{equation}
    \textbf{v}_f^{1}[\textbf{r},\textbf{w}]
    = 
    \frac{\textbf{w} - \textbf{u}_f}{2}\cdot \left[
        \frac{\bm\delta}{r^3}-\frac{3\textbf{rr}}{r^5}
    \right]
    + \mathcal{O}(\phi),
    \label{eq:potential_sol}
\end{equation}
with $\textbf{u}_f$ the mean fluid phase velocity evaluated at the center of mass of the test-particle located at $\textbf{y}$. 
In \ref{eq:potential_sol} the vector $\textbf{r}$ represents the dimensionless distance from the particle center of mass \textbf{y} to the point $\textbf{x}$, i.e. $\textbf{r} = (\textbf{x} - \textbf{y})/a$, where we recall that $a$ is the radius of the droplets. 
Using \ref{eq:potential_sol} in \ref{eq:batchlor_avg} and considering that $\textbf{v}_f^{1}\sim \frac{1}{r^3}$ at the leading order, yields an ``Error'' term of,  
\begin{equation*}
    \text{Error}
    \sim
    \int_{\mathbb{R}^3}
    \mathcal{O}(|\textbf{r}|/r^6)
    d\textbf{r}
    = \text{finite}. 
\end{equation*}
In light of this result, we conclude that due to the rapid decay of $\textbf{v}_f^{1}$ in potential flow, the error produced is therefore finite, enabling \ref{eq:batchlor_avg} to provide a physically meaningful result.
Indeed, by direct integration of the first term on the right-hand side of \ref{eq:batchlor_avg} using the solution \ref{eq:potential_sol}, we obtain, 
\begin{equation}
    \avg{\chi_f \textbf{u}_f'\textbf{u}_f'}
    = \phi \left\{
        \frac{1}{20}[\textbf{u}_{fp}\textbf{u}_{fp}+ \frac{1}{n_p}\pavg{\textbf{u}_\alpha'\textbf{u}_\alpha'}]
        + 
        \frac{3}{20} (\textbf{u}_{fp}\cdot \textbf{u}_{fp} + 2k_p)\bm\delta
    \right\},
    \label{eq:van_wingarden_sol}
\end{equation} 
where we have defined the mean relative phase velocity as $\textbf{u}_{fp} = \textbf{u}_f - \textbf{u}_p$, and the particles fluctuating velocity, $\textbf{u}_\alpha' = \textbf{u}_\alpha - \textbf{u}_p$. 
\ref{eq:van_wingarden_sol} is the original closure obtained by \citet{van1998pseudo} with the addition of the particle phase velocity fluctuations, with the terms $\pavg{\textbf{u}_\alpha\textbf{u}_\alpha}$ and $k_p = \frac{1}{n_p}\pavg{\textbf{u}_\alpha'\cdot\textbf{u}_\alpha'}$ that have been found latter by \citet[Appendix B]{zhang1994averaged}. 
The latter two terms can be obtained by noticing that, 
\begin{equation}
    \int_{\mathbb{R}^3} \textbf{ww} n_p[\textbf{y},\textbf{w}] d\textbf{w}
    = \pavg{\textbf{u}_\alpha\textbf{u}_\alpha}. 
\end{equation}


Let us now consider the Stokes flow regime. 
In this case, we can prove that $\textbf{v}_f^{1}$ follows the \textit{single-particle conditionally averaged} Stokes flow equations. 
Again, At $\mathcal{O}(\phi)$, it can be shown (see \ref{chap:daniel2}) that $\textbf{v}^1_f$ corresponds to the velocity field of an isolated droplet in a pure Newtonian fluid. 
Since we consider only uniform relative motions with the carrier phase, we can directly write (see \ref{chap:closure-disperse} and \citet{kim2013microhydrodynamics}), 
\begin{equation}
    \textbf{v}_f^{1}[\textbf{r},\textbf{w}]
    = 
    % \left(\frac{3\lambda + 2}{\lambda +1}\right)\frac{\textbf{w}- \textbf{u}_f}{4}\cdot
    % \left\{
    %     1
    %     + 
    %     \frac{\lambda}{2(3\lambda +2)}\grad^2
    % \right\}\mathcal{G}(\textbf{r})
    \mathcal{U}(\textbf{r}) \cdot (\textbf{w} - \textbf{u}_f)
    + \mathcal{O}(\phi),
    \label{eq:stokes_sol}
\end{equation}
% where $\mathcal{G}(\textbf{r})$ is teh Green function of the Stokes equations centerd at \textbf{y}, namely, 
% \begin{equation}
%     \mathcal{G}(\textbf{r}) = \frac{\bm\delta}{r} + \frac{\textbf{rr}}{r^3} .
% \end{equation}
where we recall that,
\begin{equation}
    \mathcal{U}(\textbf{r})= 
    \frac{1}{4}\left(\frac{3\lambda + 2}{\lambda +1}\right)
    \left(\frac{\bm\delta}{r} + \frac{\textbf{rr}}{r^3}\right) 
    + 
    \frac{1}{4}\left(\frac{\lambda}{\lambda +1}\right)
    \left(\frac{\bm\delta}{r^3} - \frac{3 \textbf{rr}}{r^5}\right)
    \label{eq:Umathcal}
\end{equation}
% Again In this formula, the vector $\textbf{r}$ is made dimensionless using the radius of the particles, $a$. 
As discussed in several studies in the literature \citep{caflisch1985variance}, 
since $\textbf{v}_f^{1} \sim \frac{1}{r}$ at the leading order in the stokes regime, the first integral on the right-hand side of \ref{eq:batchlor_avg} diverges. 
Indeed, we obtain,
\begin{equation}
    \int \textbf{v}_f^{1} \textbf{v}_f^{1}  d \textbf{y} = 
    \int \mathcal{O}(r^{-2}) d \textbf{y} = \infty.
    \label{eq:non_convergence}
\end{equation}
Note that we have considered a finite but constant number density $n_p$ in this integral. 
The divergence problem was attributed to the infinite energy generated by the slow decay of the velocity perturbation of an isolated droplet in Stokes flow \citep{caflisch1985variance}\footnote{\citet{caflisch1985variance} actually considered multiple sedimenting particles, however, as they mention in their conclusion, the derivation is equivalent to the calculation of the infinite kinetic energy generated due to an isolated sedimenting sphere. }. 
One could also argue that this inconsistency arises because pure Stokes flows do not exist in reality; however, we know that the consideration of small but finite $Re$ does not solve the divergence integral issues \citep{koch1993hydrodynamic}. 
While these conclusions remain valid for isolated particles, we aim to revisit this discussion using the formula \ref{eq:batchlor_avg}.

% Firstly, we would like to clarify a point of importance. 
The solution provided by \ref{eq:stokes_sol} does not imply that we are considering a single droplet translating in an infinite medium.
Indeed, it just witnesses the fact that  $\textbf{v}_f^{1}$ (at the leading order in $\phi$) has the same mathematical expression as the one of the disturbance field of an isolated droplet. 
% This means, that even if a single particle immersed in Stokes flow does indeed produce an infinite amount of energy in the solvent while translating.
% However, as ``isolated'' $\neq$ ``dilute regime'', this justification is not valid in our case.
Then, the variance of an isolated particle's disturbance field is infinite because: mathematically, this is the result we obtain, and physically, it makes sense because isolated particles in an infinite medium do not exist. 
Indeed, in reality, the continuous phase domain is always bounded, preventing the generation of an infinite amount of energy due to a translating particle.
However, in our derivation we are not considering a single droplet, but rather an arbitrary number of droplets in the dilute regime.   
% Thus, we believe that the conclusion of \citet{caflisch1985variance} is not in contradiction with physical principles since they are computing the variance of a situation that cannot exist in real applications. 
The question, therefore, is why our methodology appears to lead us to compute the variance of an isolated particle, where $\phi = 0$, even though we considered throughout the derivation a finite but non-zero volume fraction ($\phi > 0$).

We believe that the source of this inconsistency is purely mathematical rather than physical. 
Indeed, thanks to the explicit derivation of the "Error" term in \ref{eq:batchlor_avg}, which has not been presented in this explicit form until now, we are now able to identify the source of this inconsistency.
In Stokes flow, at $\mathcal{O}(\phi)$, $\textbf{v}_f^{1} \sim 1/r$ at the leading order, thus in this situation, 
\begin{equation}
    \text{Error}
    = 
    \int_{\mathbb{R}^3} 
    \mathcal{O}(1/r) d\textbf{r}
    = \infty. 
    \label{eq:real_error}
\end{equation}
In light of \ref{eq:real_error} the error generated by the wake of a dilute emulsion of translating particles in Stokes flow is infinite, making \ref{eq:batchlor_avg} unable to provide consistent results. 


To conclude, contrary to previous studies, particularly \citet{caflisch1985variance}, we argue that the non-convergence issue encountered in the derivation of $\avg{\chi_d \textbf{u}_f'\textbf{u}_f'}$ arise because of the limited accuracy of Batchelor's formula \eqref{eq:batchlor_avg}, rather than from any physical reasons.
Thus, in the following sections, we propose to use another method than, \ref{eq:batchlor_avg} and \ref{eq:classic_avg} to reformulate the ensemble average quantities, as these traditional approaches are either inaccurate or inapplicable to our specific situation. 






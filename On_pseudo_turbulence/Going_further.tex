\section{Compute the influence of two nearest particles ? }


\subsection{Expressing the Reynolds stress}

In the first place let us introduce the relation, 
\begin{align*}
    \avg{\chi_f \textbf{u}_f'\textbf{u}_f'}[\textbf{x},t]
    &= 
    \int_{\mathbb{R}^3}
    \avg{\chi_f \textbf{u}_f'\textbf{u}_f' 
    \sum_i^N \delta(\textbf{x}_i - \textbf{x}- \textbf{r}_1)
    h_i
    }
    d\textbf{r}_1\\
    &= 
    \int_{\mathbb{R}^3}
    \int_{\mathbb{R}^3}
    \avg{\chi_f \textbf{u}_f'\textbf{u}_f' 
    \sum_i^N \delta(\textbf{x}_i - \textbf{x}- \textbf{r})
    h_i
    \sum_{j\neq i}^N \delta(\textbf{x}_j - \textbf{x}- \textbf{r}_2)
    h_j
    }
    d\textbf{r}_2
    d\textbf{r}_1
\end{align*}
then let us introduce the disturbance field, 
\begin{align}
    \phi_f
    &= 
    \avg{\chi_f 
    % \sum_i^N \delta(\textbf{x}_i - \textbf{x}- \textbf{r})
    % h_i
    % \sum_{j\neq i}^N \delta(\textbf{x}_j - \textbf{x}- \textbf{r})
    % h_j
    }\\
    P_\text{1-f}
    &= 
    \avg{\chi_f  
    \sum_i^N \delta(\textbf{x}_i - \textbf{x}- \textbf{r})
    % h_i
    % \sum_{j\neq i}^N \delta(\textbf{x}_j - \textbf{x}- \textbf{r})
    % h_j
    }\\
    P_\text{nst-f}
    &= 
    \avg{\chi_f  
    \sum_i^N \delta(\textbf{x}_i - \textbf{x}- \textbf{r})
    h_i
    % \sum_{j\neq i}^N \delta(\textbf{x}_j - \textbf{x}- \textbf{r})
    % h_j
    }\\
    P_\text{2-f}
    &= 
    \avg{\chi_f  
    \sum_i^N \delta(\textbf{x}_i - \textbf{x}- \textbf{r})
    % h_i
    \sum_{j\neq i}^N \delta(\textbf{x}_j - \textbf{x}- \textbf{r})
    % h_j
    }\\
    P_\text{2-nst-f}
    &= 
    \avg{\chi_f  
    \sum_i^N \delta(\textbf{x}_i - \textbf{x}- \textbf{r})
    h_i
    \sum_{j\neq i}^N \delta(\textbf{x}_j - \textbf{x}- \textbf{r})
    h_j
    }\\
    P_\text{2-nst-f} \textbf{v}_f^\text{2-nst}
    &= 
    \avg{\chi_f \textbf{u}_f' 
    \sum_i^N \delta(\textbf{x}_i - \textbf{x}- \textbf{r})
    h_i
    \sum_{j\neq i}^N \delta(\textbf{x}_j - \textbf{x}- \textbf{r})
    h_j
    }\\
\end{align}
From these definitions we may deduce the relations, 
\begin{align}
    P_\text{2-nst-f}[\textbf{r}_2,\textbf{r}_1,\textbf{x}]
    &= 
    P_\text{nst-f}[\textbf{r}_1,\textbf{x}]
    P_\text{2-nst-f}[\textbf{r}_2|\textbf{r}_1,\textbf{x}]
    = 
    \phi_f[\textbf{x}]
    P_\text{nst-f}[\textbf{r}_1|\textbf{x}]
    P_\text{2-nst-f}[\textbf{r}_2|\textbf{r}_1,\textbf{x}]\\
    P_\text{2--f}[\textbf{r}_2,\textbf{r}_1,\textbf{x}]
    &= 
    P_\text{1-f}[\textbf{r}_1,\textbf{x}]
    P_\text{2-f}[\textbf{r}_2|\textbf{r}_1,\textbf{x}]
    = 
    \phi_f[\textbf{x}]
    P_\text{1-f}[\textbf{r}_1|\textbf{x}]
    P_\text{2-f}[\textbf{r}_2|\textbf{r}_1,\textbf{x}]
\end{align}
Additionally, following \citet{zhang2021ensemble} we may now introduce, 
\begin{align}
    h_\text{1-nst}[\textbf{r}_1, \textbf{x}]
    =
    P_\text{nst-f}[\textbf{r}_1 , \textbf{x}] / P_\text{1-f}[\textbf{r}_1 ,\textbf{x}]
    = 
    P_\text{nst-f}[\textbf{r}_1 | \textbf{x}] /P_\text{1-f}[\textbf{r}_1 |\textbf{x}]
\end{align}
where $h_\text{2-nst}[\textbf{r}_1, \textbf{x}]$ is the probability of finding no particles center of mass in the spherical shell $a < |\textbf{z} - \textbf{x}| < |\textbf{r}_1 - \textbf{x}|$. 
Similarly
\begin{align}
    h_\text{2-nst}[\textbf{r}_2,\textbf{r}_1, \textbf{x}]
    =
    P_\text{2-nst-f}[\textbf{r}_2 ,\textbf{r}_1 , \textbf{x}] / P_\text{2-f}[\textbf{r}_2 ,\textbf{r}_1 ,\textbf{x}]
    =
    P_\text{2- nst-f}[\textbf{r}_2 |\textbf{r}_1 , \textbf{x}] 
    / 
    P_\text{2-f}[\textbf{r}_2 |\textbf{r}_1 ,\textbf{x}]
    h_\text{1-nst}
\end{align}
where $h_\text{2-nst}[\textbf{r}_2,\textbf{r}_1, \textbf{x}]$ is the probability that $\textbf{r}_1$ and $\textbf{r}_2$ are the nearest particles to \textbf{x}, knowing both particles are present and fluid at \textbf{x}. 
In other worlds it is the probability that no-other particles are present in the spherical shell $a < |\textbf{z} - \textbf{x}| < |\textbf{r}_1 - \textbf{x}|$ excluding the volume that the first particle occupy. 

Finally, 
\begin{equation}
    P_\text{2-nst-f}[\textbf{r}_2 ,\textbf{r}_1 , \textbf{x}] 
    =
    h_\text{2-nst}[\textbf{r}_2,\textbf{r}_1, \textbf{x}] 
    / h_\text{1-nst}[\textbf{r}_2,\textbf{r}_1 ]
    P_\text{2-f}[\textbf{r}_2 ,\textbf{r}_1 ,\textbf{x}]
\end{equation}
\paragraph*{Second option (may be better): }

\begin{equation}
    h_2[\textbf{r}_2,\textbf{r}_1,\textbf{x}]
    = 
    P_\text{2-nst-f}[\textbf{r}_2|\textbf{r}_1,\textbf{x}]
    / 
    P_\text{2-f}[\textbf{r}_2|\textbf{r}_1,\textbf{x}]
\end{equation}
So that $h_2$ is the probability of finding no particle for the points \textbf{z} in the region $|\textbf{r}_1- \textbf{x}|<|\textbf{z}- \textbf{x}|<|\textbf{r}_2- \textbf{x}|$, making the particle 2 the second closested. 



\paragraph*{Summary}

\begin{equation}
    P_\text{2-nst-f}
    = 
    \phi_f 
    P_\text{nst-f}
    P_\text{2-nst-f}
    = 
    \phi_f 
    h_1 P_\text{1-f}
    h_2 P_\text{2-f}
\end{equation}
In the dilute regime where there is no overlap it is easy to compute all of these terms. 


To compute the probability $h_1$ it is easy. 
It is one minus th proba of finding a particle in the volume $V$. 
In that case the volume is, 
\begin{equation}
    V = \frac{4}{3}\pi (r_1^3 - a^3)
\end{equation}
then we introduce the infinitesimal volume $dv = V / N$. The probability of finding a part in sucha small volume is $n_p v$ and thus, 
\begin{equation}
    h_1 
    = \lim_{N\to\infty}
    (1-n_p \frac{V}{N})^N 
    = e^{-n_p V}
\end{equation}

\begin{tikzpicture}

    % Define points
    \coordinate (X) at (0,0); % Center X
    \coordinate (r1) at (3,0); % Point r1
    \coordinate (r2) at (2.6,3); % Point r2 (farther than r1)

    % Dashed circles centered at X
    \draw[dashed] (X) circle (1); % Dashed circle from X to r1
    \draw[dashed] (X) circle (3); % Dashed circle from X to r1
    \draw[dashed] (X) circle (4); % Dashed circle from X to r2
    \draw[dashed] (r1) circle (2); % Dashed circle from X to r2

    % Solid circle centered at r1 with radius 2a
    \draw[thick] (r1) circle (2);

    % Shaded region between r1 and r2, excluding the solid circle
    \begin{scope}
        \clip (r1) circle (2);
        % \fill (X) circle (6);
    \end{scope}

    % Draw labels
    \fill (X) circle (2pt) node[left] {\Large $\textbf{x}$};
    \fill (r1) circle (2pt) node[below] {\Large $\textbf{r}_1$};
    \fill (r2) circle (2pt) node[below] {\Large $\textbf{r}_2$};

\end{tikzpicture}

The same reasoning can be made for $h_2$ except that the volume in question is, 
\begin{equation}
    V = \{
        |\textbf{r}_1 - \textbf{x}| < |\textbf{z} - \textbf{x}| < |\textbf{r}_2 - \textbf{x}|
        \setminus
        |\textbf{z} - \textbf{r}_1| < 2a
    \}
\end{equation}
For consistness we intriducde 
\begin{equation}
    V = \{
        r_1 < |\textbf{z} - \textbf{x}| < r_2
        \setminus
        |\textbf{z} - \textbf{r}_1| < 2a
    \}
\end{equation}
We must find the volume of the spherical caps. 
The volume of the spherical shell excluding the sphere at $\textbf{r}_1$ can be computed as, 
\begin{equation}
    V 
    = 
    \int_0^{2\pi}
    \int_{r_1}^{r_2}
    \int_{\theta_{min}(r)}^{\pi}
    r^2
    \sin\theta
    d\theta 
    dr
    d\varphi
\end{equation}
where we have choosen a local coordinate system centered at $\textbf{x}$ in the direction of $\textbf{r}_1$. 
In this situation the eq of the surface of the sphere at $\textbf{r}_1$ might be written, 
\begin{align}
    x^2
    + y^2
    + (z-r_1)^2
    &= (2a)^2\\
    r^2 \sin^2\theta_{min}
    + (r \cos\theta_{min} -r_1)^2
    &= (2a)^2\\
    - r^2 
    + 2 \cos\theta_{min} r r_1 
    - r_1^2
    &= - (2a)^2\\
    + 2 \cos\theta_{min} r r_1 
    &= 
    r^2 
    + 
    (r_1+2a)(r_1-2a)
\end{align}
Thus, 
\begin{equation}
    \theta_{min}
    = \left\{
    \begin{tabular}{ll}
        $\arccos\{
        \frac{
            r^2 
            + r_1^2
             - (2a)^2
        }{2 r r_1}\}
        $&$
        \forall r < (r_1+2a)$\\
        $ 
        0
        $&$
        \forall r > (r_1+2a)$
    \end{tabular}
    \right.
\end{equation}


So for points $r_2 <r_1 +2a$ the integration goes like, 
\begin{align}
    V 
    &= 
    \int_0^{2\pi}
    d\varphi
    % \left[
    \int_{r_1}^{r_2}
    \int_{\theta_{min}(r)}^{\pi}
    % +
    %     \int_{r_1+2a}^{r_2}
    %     \int_{0}^{\pi}
    % \right]
    r^2
    \sin\theta
    d\theta 
    dr\\
    &= 
    \pi /r_1
    \int_{r_1}^{r_2} 
    [2r^2r_1 +  
        r^3 
        + r(r_1^2
        - (2a)^2)
    ]
    dr\\
    &= 
    \pi /r_1
    [2r^3/3r_1 
        +  r^4/4
        + r^2/2 (r_1^2 - (2a)^2)
    ]_{r_1}^{r_2} \\
    &= 
    \pi /r_1
    [\frac{2r_1}{3}(r_2^3-r_1^3) 
    + \frac{1}{4} (r_2^4-r_1^4)
        + \frac{1}{2}(r_2^2 - r_1^2) (r_1^2 - (2a)^2)]\\
    & = \frac{4\pi}{3}
    \frac{ 
     8 r_{1} \left(r_{2}^{3} - r_{1}^{3}\right) 
    + 3 (r_{2}^{4} - r_{1}^{4})
    + 6  \left(r_{1}^{2} - (2 a)^{2} \right) \left(r_{2}^{2} - r_{1}^{2}\right)}{16 r_{1}}
\end{align}

For points $r_2 > r_1 +2a$ we have, 
\begin{align}
    V &= 
    \pi /r_1
    [\frac{2r_1}{3}((r_1 + 2a)^3-r_1^3) 
    + \frac{1}{4} ((r_1 + 2a)^4-r_1^4)
        + \frac{1}{2}((r_1 + 2a)^2 - r_1^2) (r_1^2 - (2a)^2)]
    \\
    &+\frac{4\pi}{3}(r_2^3 - (r_1+(2a)^3))\\
    &
    =
    \frac{4\pi}{3}\left[
        -\frac{3a^4}{r_1}
        - 4 a^3 
        + r_2^3
        -r_1^3
    \right]
\end{align}

Assuming $r_1/a \to r_1$ and $r_2/a \to r_2$ we have, 
\begin{align}
    V 
    &= 
    \frac{4\pi a^3}{3}
    \frac{ 
     8 r_{1} \left(r_{2}^{3} - r_{1}^{3}\right) 
    + 3 (r_{2}^{4} - r_{1}^{4})
    + 6  \left(r_{1}^{2} - 4 \right) \left(r_{2}^{2} - r_{1}^{2}\right)}{16 r_{1}}
    \\
    V 
    &= 
    \frac{4\pi a^3}{3}\left[
        -\frac{3}{r_1}
        - 4 
        + r_2^3
        -r_1^3
    \right]
\end{align}



We conclude that, 
\begin{align}
    h_1 &= \exp\left\{-\phi (r_1^3 - 1)\right\}& \forall r_1 > a\\
    h_2 &= \exp\left\{-\phi 
        \frac{ 
        8 r_{1} \left(r_{2}^{3} - r_{1}^{3}\right) 
       + 3 (r_{2}^{4} - r_{1}^{4})
       + 6  \left(r_{1}^{2} - 4 \right) \left(r_{2}^{2} - r_{1}^{2}\right)}{16 r_{1}}
    \right\} &\forall  r_2 < r_1 + 2\\ 
    h_2 &= \exp\left\{-\phi 
    \left[
        -\frac{3}{r_1}
        - 4 
        + r_2^3
        -r_1^3
    \right]
    \right\} &\forall  r_2 > r_1 + 2 
\end{align}

\paragraph*{Results : }
For the $r_2 < r_1 + 2$ we have, 
\begin{equation}
    P_{2-nst}^f[\textbf{r}_2 | \textbf{r}_1,\textbf{x}]
    = 
    n_p[\textbf{r}_2]
    \exp\left\{-\phi 
        \frac{ 
        8 r_{1} \left(r_{2}^{3} - r_{1}^{3}\right) 
       + 3 (r_{2}^{4} - r_{1}^{4})
       + 6  \left(r_{1}^{2} - 4 \right) \left(r_{2}^{2} - r_{1}^{2}\right)}{16 r_{1}}
    \right\} 
\end{equation}
and for $r_2 > r_1 + 2$
\begin{equation}
    P_{2-nst}^f[\textbf{r}_2 | \textbf{r}_1,\textbf{x}]
    = 
    n_p[\textbf{r}_2]
    \exp\left\{-\phi 
    \left[
        -\frac{3}{r_1}
        - 4 
        + r_2^3
        -r_1^3
    \right]
    \right\} 
\end{equation}
and for both cases $P_{2-nst}^f = 0 $ for $r_2 < r_1$ and $|\textbf{r}_2-\textbf{r}_1|<2a$. 
Note that we should have 
\begin{equation}
    \int_{
        r_1 < r_2
        \setminus
        |\textbf{r}_2 - \textbf{r}_1| < 2a
    }
    P_{2-nst}^f[\textbf{r}_2 | \textbf{r}_1,\textbf{x}]
    d\textbf{r}_2 
    = 1 
\end{equation}


\paragraph*{Check : }

\begin{equation}
    \int_{
        \mathcal{D}_1 + \mathcal{D}_2
    }
    P_{2-nst}^f
    d\textbf{r}_2 
    = 
    \int_{
        \mathcal{D}_1 
    }
    P_{2-nst}^f
    d\textbf{r}_2 
    + 
    \int_{
        \mathcal{D}_2
    }
    P_{2-nst}^f
    d\textbf{r}_2 
\end{equation}

The first integral, 
\begin{multline}
    \int_{
        \mathcal{D}_1 
    }
    P_{2-nst}^f
    d\textbf{r}_2 
    \\=
    \frac{3 \phi}{4\pi}
    \int_0^{2\pi}
    \int_{r_1}^{r_1+2}
    \int_{\theta_{min}}^{\pi}
    \exp\left\{-\phi 
    \frac{ 
    8 r_{1} \left(r_{2}^{3} - r_{1}^{3}\right) 
   + 3 (r_{2}^{4} - r_{1}^{4})
   + 6  \left(r_{1}^{2} - 4 \right) \left(r_{2}^{2} - r_{1}^{2}\right)}{16 r_{1}}
    \right\} 
    r_2^2
    \sin\theta
    d\theta
    d\varphi
    dr_2
    \\=
    \frac{3 \phi}{4\pi}
    \int_0^{2\pi}
    \int_{r_1}^{r_1+2}
    \exp\left\{-\phi 
    \frac{ 
    8 r_{1} \left(r_{2}^{3} - r_{1}^{3}\right) 
   + 3 (r_{2}^{4} - r_{1}^{4})
   + 6  \left(r_{1}^{2} - 4 \right) \left(r_{2}^{2} - r_{1}^{2}\right)}{16 r_{1}}
    \right\} 
    r_2^2
    [-\cos\theta]_{\theta_{min}}^{\pi}
    d\varphi
    dr_2
    \\=
    \frac{3 \phi}{4\pi}
    2\pi
    \int_{r_1}^{r_1+2}
    r_2^2
    [1
    + \frac{
        r_2^2 
        + r_1^2
         - 4
    }{2 r_2 r_1}]
    \exp\left\{-\phi 
    \frac{ 
    8 r_{1} \left(r_{2}^{3} - r_{1}^{3}\right) 
   + 3 (r_{2}^{4} - r_{1}^{4})
   + 6  \left(r_{1}^{2} - 4 \right) \left(r_{2}^{2} - r_{1}^{2}\right)}{16 r_{1}}
    \right\} 
    dr_2
    \\=
    \frac{3 \phi}{4\pi}
    \frac{\pi}{r_1}
    \int_{r_1}^{r_1+2}
    [2r_2^2r_1
    + 
        r_2^3
        + r_2 (r_1^2
         - 4)]
    \exp\left\{-\phi 
    \frac{ 
    8 r_{1} \left(r_{2}^{3} - r_{1}^{3}\right) 
   + 3 (r_{2}^{4} - r_{1}^{4})
   + 6  \left(r_{1}^{2} - 4 \right) \left(r_{2}^{2} - r_{1}^{2}\right)}{16 r_{1}}
    \right\} 
    dr_2
    \\
    =
    1 
    - 
    \exp\left\{
        -\phi (6r_1^2 + 12 r_1 + 4 - \frac{3}{r_1})
        \right\}
\end{multline}

The second integral, 
\begin{multline}
    \int_{
        \mathcal{D}_2 
    }
    P_{2-nst}^f
    d\textbf{r}_2 
    =
    \frac{3 \phi}{4\pi}
    \int_0^{2\pi}
    \int_{r_1+2}^\infty
    \int_{0}^{\pi}
    \exp\left\{-\phi 
    \left[
        -\frac{3}{r_1}
        - 4 
        + r_2^3
        -r_1^3
    \right]
    \right\} 
    r_2^2
    \sin\theta
    d\theta
    d\varphi
    dr_2 \\
    =
    3 \phi
    \int_{r_1+2}^{\infty}
    \exp\left\{-\phi 
    \left[
        -\frac{3}{r_1}
        - 4 
        + r_2^3
        -r_1^3
    \right]
    \right\} 
    r_2^2
    dr_2 \\
    = \exp\left\{
        - \phi [
            6 r_1 
            + 12 r_1
            + 4
            - \frac{3}{r_1}
        ]
    \right\}
\end{multline}

We obtain indeed, 

\begin{equation}
    \int_{
        \mathcal{D}_1 + \mathcal{D}_2
    }
    P_{2-nst}^f
    d\textbf{r}_2 
    = 
    1
\end{equation}


\subsection{Evaluation of the effect of a second nearest part on the RS}

In the first place let us introduce the relation, 
\begin{align*}
    \avg{\chi_f \textbf{u}_f'\textbf{u}_f'}[\textbf{x},t]
    &= 
    \int_{\mathbb{R}^3}
    \avg{\chi_f \textbf{u}_f'\textbf{u}_f' 
    \sum_i^N \delta(\textbf{x}_i - \textbf{x}- \textbf{r}_1)
    h_i
    }
    d\textbf{r}_1\\
    &= 
    \int_{\mathbb{R}^3}
    \textbf{v}_f^\text{nst}
    \textbf{v}_f^\text{nst}
    P_f^\text{nst}
    d\textbf{r}_1
    + 
    \int_{\mathbb{R}^3}
    \avg{\chi_f \textbf{u}_f''\textbf{u}_f'' 
    \sum_i^N \delta(\textbf{x}_i - \textbf{x}- \textbf{r}_1)
    h_i
    }
    d\textbf{r}_1
    % \int_{\mathbb{R}^3}
    % \int_{\mathbb{R}^3}
    % \avg{\chi_f \textbf{u}_f'\textbf{u}_f' 
    % \sum_i^N \delta(\textbf{x}_i - \textbf{x}- \textbf{r})
    % h_i
    % \sum_{j\neq i}^N \delta(\textbf{x}_j - \textbf{x}- \textbf{r}_2)
    % h_j
    % }
    % d\textbf{r}_2
    % d\textbf{r}_1
\end{align*}
where $\textbf{u}_f'' = \textbf{u}_f^0 - \textbf{u}_f^\text{nst}$. 

Then, notice that the second term can be further expressed as, 
\begin{align}
    \avg{\chi_f \textbf{u}_f''\textbf{u}_f'' 
    \sum_i^N \delta(\textbf{x}_i - \textbf{x}- \textbf{r}_1)
    h_i
    }
    = 
    \int 
    \avg{\chi_f \textbf{u}_f''\textbf{u}_f'' 
    \sum_i^N \delta(\textbf{x}_i - \textbf{x}- \textbf{r}_1)h_i
    \sum_{j\neq i }^N \delta(\textbf{x}_j - \textbf{x}- \textbf{r}_2)h_j
    }
    d\textbf{r}_2\\
    =
    \int 
    \textbf{v}_f^\text{2-nst}
    \textbf{v}_f^\text{2-nst}
    P_\text{2-nst-f}
    d\textbf{r}_2
    + 
    \int 
    \avg{\chi_f \textbf{u}_f'''\textbf{u}_f''' 
    \sum_i^N \delta(\textbf{x}_i - \textbf{x}- \textbf{r}_1)h_i
    \sum_{j\neq i }^N \delta(\textbf{x}_j - \textbf{x}- \textbf{r}_2)h_j
    }
    d\textbf{r}_2
\end{align}
Where, $\textbf{u}_f''' = \textbf{u}_f^0 - \textbf{u}_f^\text{2-nst}$ and, 
\begin{equation}
    \textbf{v}_f^\text{2-nst}[\textbf{r}_2,\textbf{r}_1,\textbf{x}]
    = 
    \textbf{u}_f^\text{2-nst}[\textbf{r}_2,\textbf{r}_1,\textbf{x}]
    - \textbf{u}_f^\text{nst}[\textbf{r}_1,\textbf{x}]
\end{equation}
which might be obtained as, 
\begin{equation}
    \textbf{v}_f^\text{2-nst} P_\text{f-nst-2}
    = 
    (
    \textbf{u}_f^\text{2-nst}
    - \textbf{u}_f^\text{nst}
    )P_\text{f-nst-2}
    = 
    \avg{
        \chi_f \textbf{u}_f^0 \delta_\text{nst}
        (
            \delta_{nst-2}
            - P_\text{nst-2}^\text{f-nst}
        )
    }
\end{equation}
Anyhow the int can be expressed as, 

\begin{multline}
    \int 
    \textbf{v}_f^\text{2-nst}
    \textbf{v}_f^\text{2-nst}
    P_\text{2-nst-f}
    d\textbf{r}_2
    = 
    P_\text{nst-f}
    \int 
    \textbf{v}_f^\text{2-nst}
    \textbf{v}_f^\text{2-nst}
    P_2^\text{nst-f}
    d\textbf{r}_2\\
    = 
    P_\text{nst-f}
    \int 
    \mathcal{O}(\bm\delta/r_2^2)
    P_2^\text{nst-f}
    d\textbf{r}_2
\end{multline}
since $\textbf{v}_f^\text{2-nst}$ does not contains the mean effect of the particle in $r_1$ it is a most $\mathcal{O}(1/r_2)$.

Thus, the final contribution of this term is, 
\begin{multline}
    \int      P_\text{nst-f}
    \int 
    \mathcal{O}(1/r_2^2)
    P_2^\text{nst-f}
    d\textbf{r}_2
    d\textbf{r}_1
\end{multline}
We recall that, 
\begin{equation}
    P_\text{nst-f}
    = \frac{4\phi}{3\pi}
    \exp{[-\phi (r_1^3 - 1)]}
\end{equation}
Such that, 
\begin{multline}
    \int     \frac{4\phi}{3\pi}
    \exp{[-\phi (r_1^3 - 1)]}
    \int 
    \mathcal{O}(1/r_2^2)
    P_2^\text{nst-f}
    d\textbf{r}_2
    d\textbf{r}_1
\end{multline}

This integral can be computed into the two respective region of integration, yielding, 

The first integral, 
\begin{multline}
    \int_{
        \mathcal{D}_1 
    }
    \frac{1}{r_2^2}P_{2-nst}^f
    d\textbf{r}_2 
    =\\
    \frac{3 \phi}{4\pi}
    \frac{\pi}{r_1}
    \int_{r_1}^{r_1+2}
    [2r_1
    + 
        r_2
        +     
        \frac{1}{r_2}
        (r_1^2 - 4)]
    \exp\left\{-\phi 
    \frac{ 
    8 r_{1} \left(r_{2}^{3} - r_{1}^{3}\right) 
   + 3 (r_{2}^{4} - r_{1}^{4})
   + 6  \left(r_{1}^{2} - 4 \right) \left(r_{2}^{2} - r_{1}^{2}\right)}{16 r_{1}}
    \right\} 
    dr_2\\
\end{multline}

The second integral, 
\begin{multline}
    \int_{
        \mathcal{D}_2 
    }
    \frac{1}{r_2^2}P_{2-nst}^f
    d\textbf{r}_2 
    =
    3 \phi
    \int_{r_1+2}^{\infty}
    \exp\left\{-\phi 
    \left[
        -\frac{3}{r_1}
        - 4 
        + r_2^3
        -r_1^3
    \right]
    \right\} 
    dr_2 
\end{multline}



\section*{$N^{th}$ nearest neighbor}

\begin{equation}
    \avg{
        \chi_f \textbf{u}_f' \textbf{u}_f'
    }
    =
    \int d \textbf{r}^N \avg{
        \chi_f \textbf{u}_f' \textbf{u}_f'
        \prod_{k=0}^{N-1} \sum_{i_k \neq i_{k-1},i_{k-2},\ldots, i_0}^{N-k}
        \delta(\textbf{x}_{i_k} - \textbf{x} - \textbf{r}_k)
        h_{i_k}
    }
\end{equation}
The conditional field, 
\begin{equation}
    \textbf{u}^{N-nst} P_{N-nst}
    = 
    \sum_i^N (1+\grad^2) \mathcal{G}(\textbf{r}_i)+\ldots
\end{equation}



\section*{conditional eq for the two-nearest neighbor equaiton}


For consistness we note, 
\begin{align}
    \chi_f[\textbf{x}] \sum_i \delta(\textbf{x}_i-\textbf{x}-\textbf{r}_1)h_i
    &\to 
    \delta_\text{f-nst}[\textbf{x},\textbf{r}_1]\\
    \chi_f[\textbf{x}] 
    \sum_i \delta(\textbf{x}_i-\textbf{x}-\textbf{r}_1)h_i
    \sum_{j\neq i} \delta(\textbf{x}_i-\textbf{x}-\textbf{r}_2)h_j
    &\to 
    \delta_\text{f-2-nst}[\textbf{x},\textbf{r}_1,\textbf{r}_2]\\
\end{align}
We need an equaiton for the velocity field, 
\begin{equation}
    \textbf{v}_f^\text{2-nst}P_\text{f-2-nst}
    =
    \avg{
        \textbf{u}_f^0 
        (\delta_\text{2-nst-f} - \delta_\text{nst-f} P^\text{nst-f}_\text{2-nst})
    }
\end{equation}
At the points $\textbf{x}$ note that $\textbf{v}_f^\text{2-nst} = \textbf{v}^\text{2-nst}$ with,
\begin{equation}
    (\textbf{v}^\text{2-nst}P_\text{f-2-nst})[\textbf{z},\textbf{x},\textbf{r}_1,\textbf{r}_2]
    =
    \avg{
        \avg{
            \textbf{u}^0[\textbf{z}]
            (\delta_\text{2-nst-f} - \delta_\text{nst-f} P^\text{nst-f}_\text{2-nst})
        }
    }
\end{equation}

In order to be even more concise we note,
\begin{equation}
    \delta_\text{2-nst-f} - \delta_\text{nst-f} P^\text{nst-f}_\text{2-nst}
    \to 
    \Pi_\text{2-nst-f}^\text{nst-f}
\end{equation}


The local scale equaitons, 
\begin{align}
    \div \textbf{u}^0  &= 0 \\
    \rho_f \pddt \textbf{u}^0
    + \rho_f \div \textbf{u}^0\textbf{u}^0
    &= 
    \div \bm\sigma^* 
    + \rho_f \textbf{g}
    + \kappa \delta_\Gamma(\bm\sigma_f^0 \cdot \textbf{n})
\end{align}
with 
\begin{align}
    \bm\sigma^0 
    &= \chi_f \bm\sigma_f^0
    + \chi_f \bm\sigma_d^0/\zeta
    + \delta_\Gamma \bm\sigma_\Gamma^0/\zeta\\
    \zeta
    &= \rho_d / \rho_f\\
    \kappa
    &= (1-\zeta)/\zeta\\
\end{align}

Multiplying the right-hand side (Stokes eq) by  $\Pi_\text{2-nst-f}^\text{nst-f}$ and averaging yields, 
\begin{align}
    \div (P_\text{2-nst-f} \textbf{v}^\text{2-nst-f}) &= 0 \\
    \div \bm\sigma_\text{2-nst-f}^\text{eff}
    &= 
    - \kappa  \avg{\Pi_\text{2-nst-f}^\text{nst-f} \delta_\Gamma(\bm\sigma_f^0 \cdot \textbf{n})}
\end{align}
with, 
\begin{align}
    \bm\sigma_\text{2-nst-f}^\text{eff}
    &= 
    \avg{\Pi (\chi_f \bm\sigma_f^0
    + \chi_f \bm\sigma_d^0/\zeta
    + \delta_\Gamma \bm\sigma_\Gamma^0/\zeta
    )}\\
    &= 
    - P_\text{2-nst-f} \tau_f^\text{2-nst-f} \bm\delta
    +\mu_f P_\text{2-nst-f} (
        \grad \textbf{v}^\text{2-nst-f}
        + ^\dagger\grad \textbf{v}^\text{2-nst-f}
    )\\
    &+ P_\text{2-nst-f} (\phi_d^{nst} p_f^\text{2-nst-f} -\phi_d^{nst} p_f^\text{nst-f})
    + \avg{\Pi(+ \chi_f (\bm\sigma_d^0/\zeta - 2\mu_f \textbf{e}_d^0)
    + \delta_\Gamma \bm\sigma_\Gamma^0/\zeta)}
\end{align}
where $\tau$ is the disturbance pressure field $\tau^\text{2-nst-f}_f = p^\text{2-nst-f}_f - \tau^\text{nst-f}_f$.
Expanding the drag force term etc and keeping only the homogeneous terms yields,
\begin{align}
    \bm\sigma_\text{2-nst-f}^\text{eff}
    &= 
    \avg{\Pi (\chi_f \bm\sigma_f^0
    + \chi_f \bm\sigma_d^0/\zeta
    + \delta_\Gamma \bm\sigma_\Gamma^0/\zeta
    )}\\
    &= 
    - P_\text{2-nst-f} \tau_f^\text{2-nst-f} \bm\delta
    +\mu_f P_\text{2-nst-f} (
        \grad \textbf{v}^\text{2-nst-f}
        + ^\dagger\grad \textbf{v}^\text{2-nst-f}
    )\\
    &
    % + P_\text{2-nst-f} (\phi_d^{nst} p_f^\text{2-nst-f} -\phi_d^{nst} p_f^\text{nst-f})
    + \pavg{\Pi \intS[i]{\textbf{r}\bm\sigma_f^0\cdot \textbf{n} -2\mu_f \textbf{e}_d^0}}\\
    &-\frac{1}{2}\div  \left[
        \pavg{\Pi \intS[i]{\textbf{rr}\bm\sigma_f'\cdot \textbf{n} -2\mu_f (\textbf{r}\textbf{e}_d^0+\textbf{e}_d^0 \textbf{r})}}
        + \pavg{\Pi} \textbf{V} \rho_f \textbf{g}
    \right]
\end{align}
Indeed, in stokes flow regime we have, 
\begin{multline}
    \intO{ \textbf{r}(\bm{\sigma}^0_d)_{ik}+r_{k}(\bm{\sigma}^0_d)_{ji}}
    +\intS{ \textbf{r}(\bm{\sigma}^0_I)_{ik}+r_{k}(\bm{\sigma}_\Gamma^0)_{ji}}
    = 
    \intS{  \textbf{rr} (\bm{\sigma}_f^0\cdot\textbf{n}_d)_i }
    + \intO{ \textbf{rr}  \rho_d \textbf{g} } 
    \label{eq:dt_P2_alpha_bis}
\end{multline}



Accoring to the relation, 
\begin{equation}
    (\delta_\Gamma \ldots)[\textbf{z}]
    =
    \sum_k \delta(\textbf{x}_k - \textbf{z}) \intS{\ldots}
    - \div \sum_k \delta(\textbf{x}_k - \textbf{z}) \intS{\textbf{r}\ldots}
    % +\frac{1}{2} \grad\grad : \sum_i \delta(\textbf{x}_i - \textbf{z}) \intS{\textbf{rr}\ldots}
\end{equation}
Additionally, the product, 
\begin{align}
    \Pi_\text{2-nst-f}^\text{nst-f}
    \sum_i \delta(\textbf{x}_i - \textbf{z}) 
    &= 
    \chi_f[\textbf{x}] 
    \sum_i \delta(\textbf{x}_i-\textbf{x}-\textbf{r}_1)h_i
    \sum_{j\neq i} \delta(\textbf{x}_i-\textbf{x}-\textbf{r}_2)h_j
    \sum_k \delta(\textbf{x}_k - \textbf{z}) \\
    &- 
    P^\text{nst-f}_\text{2-nst}\chi_f[\textbf{x}] 
    \sum_i \delta(\textbf{x}_i-\textbf{x}-\textbf{r}_1)h_i
    % \sum_{j\neq i} \delta(\textbf{x}_i-\textbf{x}-\textbf{r}_2)h_j
    \sum_k \delta(\textbf{x}_k - \textbf{z}) \\
    &= 
    \chi_f[\textbf{x}] 
    \sum_i \delta(\textbf{x}_i-\textbf{x}-\textbf{r}_1)h_i
    \sum_{j\neq i} \delta(\textbf{x}_i-\textbf{x}-\textbf{r}_2)h_j
    [ \delta(\textbf{x}_j - \textbf{z}) +  \delta(\textbf{x}_i - \textbf{z})  + \sum_{k\neq i,j} \delta(\textbf{x}_k - \textbf{z}) ]\\
    &- P^\text{nst-f}_\text{2-nst}\chi_f
    \sum_i \delta(\textbf{x}_i-\textbf{x}-\textbf{r}_1)h_i
    [\delta(\textbf{x}_i - \textbf{z})  + \sum_{k\neq i} \delta(\textbf{x}_k - \textbf{z}) ]\\
\end{align}
Under the avg operator it is clear that, 
\begin{multline}
    \avg{(\ldots)\Pi_\text{2-nst-f}^\text{nst-f} \sum_i \delta(\textbf{x}_i - \textbf{z}) }
    =
    \delta(\textbf{r}_1 - \textbf{z}) \avg{(\ldots)\Pi_\text{2-nst-f}^\text{nst-f} }
    % -\delta(\textbf{r}_1 - \textbf{z}) P^\text{nst-f}_\text{2-nst} \avg{(\ldots)\delta_\text{2-nst-f}}\\
    +\delta(\textbf{r}_2 - \textbf{z}) \avg{(\ldots)\delta_\text{2-nst-f}}\\
    + \avg{(\ldots)\delta_\text{2-nst-f} \sum_{k\neq i,j} \delta(\textbf{x}_k - \textbf{z}) }
    - P^\text{nst-f}_\text{2-nst} \avg{(\ldots) \delta_\text{nst-f} \sum_{k\neq i} \delta(\textbf{x}_k - \textbf{z}) }\\
\end{multline}

In Stokes regime we may write, 
\begin{equation*}
    \intS[i] {\bm\sigma_f^0 \cdot \textbf{n}} =
    \rho_d \textbf{g} v
\end{equation*}
where $v=v_i$ for mono-disperse suspenison. 
Thus we obtain, 
\begin{multline}
    \avg{\Pi_\text{2-nst-f}^\text{nst-f} \sum_i \delta(\textbf{x}_i - \textbf{z}) \intS[i] {\bm\sigma_f^0 \cdot \textbf{n}}}
    =
    % \delta(\textbf{r}_1 - \textbf{z}) \avg{\rho_d \textbf{g} v \Pi_\text{2-nst-f}^\text{nst-f} }
    % -\delta(\textbf{r}_1 - \textbf{z}) P^\text{nst-f}_\text{2-nst} \avg{(\ldots)\delta_\text{2-nst-f}}\\
    + \rho_d \textbf{g} v \delta(\textbf{r}_2 - \textbf{z}) P_\text{2-nst-f}\\
    + \rho_d \textbf{g} v \avg{\delta_\text{2-nst-f} \sum_{k\neq i,j} \delta(\textbf{x}_k - \textbf{z}) 
    -  P^\text{nst-f}_\text{2-nst} \delta_\text{nst-f} \sum_{k\neq i} \delta(\textbf{x}_k - \textbf{z}) }\\
    =
    % \delta(\textbf{r}_1 - \textbf{z}) \avg{\rho_d \textbf{g} v \Pi_\text{2-nst-f}^\text{nst-f} }
    % -\delta(\textbf{r}_1 - \textbf{z}) P^\text{nst-f}_\text{2-nst} \avg{(\ldots)\delta_\text{2-nst-f}}\\
    + P_\text{2-nst-f} \rho_d \textbf{g} v \delta(\textbf{r}_2 - \textbf{z}) 
    - P_\text{2-nst-f} \rho_d \textbf{g} v n_p \int_{r_1<|\textbf{z}' - \textbf{x}| <r_2} \delta(\textbf{z}'-\textbf{z})d\textbf{z}\\
\end{multline}

\begin{multline}
    \avg{\textbf{V}\rho_f \textbf{g}\Pi_\text{2-nst-f}^\text{nst-f} \sum_i \delta(\textbf{x}_i - \textbf{z}) }
    =
    % \delta(\textbf{r}_1 - \textbf{z}) \avg{\textbf{V}\rho_f \textbf{g}\Pi_\text{2-nst-f}^\text{nst-f} }
    % -\delta(\textbf{r}_1 - \textbf{z}) P^\text{nst-f}_\text{2-nst} \avg{\textbf{V}\rho_f \textbf{g}\delta_\text{2-nst-f}}\\
    +\textbf{V}\rho_f \textbf{g}\delta(\textbf{r}_2 - \textbf{z}) \avg{\delta_\text{2-nst-f}}\\
    +\textbf{V}\rho_f \textbf{g} \avg{\delta_\text{2-nst-f} \sum_{k\neq i,j} \delta(\textbf{x}_k - \textbf{z}) }
    -\textbf{V}\rho_f \textbf{g} P^\text{nst-f}_\text{2-nst} \avg{ \delta_\text{nst-f} \sum_{k\neq i} \delta(\textbf{x}_k - \textbf{z}) }\\
    =
    + P_\text{2-nst-f} \bm\delta V \rho_f \textbf{g}\delta(\textbf{r}_2 - \textbf{z}) 
    - P_\text{2-nst-f} \bm\delta V \rho_f \textbf{g} n_p \int_{r_1<|\textbf{z}' - \textbf{x}| <r_2} \delta(\textbf{z}'-\textbf{z})d\textbf{z}
\end{multline}


Let call $S$ the stress on the particle $i$ then,
\begin{multline}
    \avg{\textbf{S}_k\Pi_\text{2-nst-f}^\text{nst-f} \sum_k \delta(\textbf{x}_k - \textbf{z}) }
    =
    \delta(\textbf{r}_1 - \textbf{z}) \avg{\Pi_\text{2-nst-f}^\text{nst-f}  \textbf{S}_i}
    % -\delta(\textbf{r}_1 - \textbf{z}) P^\text{nst-f}_\text{2-nst} \avg{\textbf{S}_k\delta_\text{2-nst-f}}\\
    +\delta(\textbf{r}_2 - \textbf{z}) \avg{\delta_\text{2-nst-f} \textbf{S}_j}\\
    + \avg{\delta_\text{2-nst-f} \sum_{k\neq i,j} \delta(\textbf{x}_k - \textbf{z}) \textbf{S}_k }
    - P^\text{nst-f}_\text{2-nst} \avg{\delta_\text{nst-f} \sum_{k\neq i} \delta(\textbf{x}_k - \textbf{z}) \textbf{S}_k }\\
    = 
    \delta(\textbf{r}_1 - \textbf{z}) (\textbf{S}_{1}^\text{2-nst-f} - \textbf{S}_{1}^\text{nst-f}) P_\text{2-nst-f}
    + \delta(\textbf{r}_2 - \textbf{z}) \textbf{S}_2^\text{2-nst-f} P_\text{2-nst-f}\\
    + P_\text{2-nst-f} n_p \int_{r_2 < |\textbf{z}' - \textbf{z}|}\textbf{S}[\textbf{z}'|\textbf{r}_1,\textbf{r}_2,\textbf{x}]\delta(\textbf{z}'-\textbf{z})d\textbf{z}
    - P_\text{2-nst-f} n_p \int_{r_1 < |\textbf{z}' - \textbf{z}|}\textbf{S}[\textbf{z}'|\textbf{r}_1,\textbf{x}] \delta(\textbf{z}'-\textbf{z}) d\textbf{z}
\end{multline}
if one assume that $\textbf{S}[\textbf{z}'|\textbf{r}_1,\textbf{r}_2,\textbf{x}] = \textbf{S}[\textbf{z}'|\textbf{r}_1,\textbf{x}]$ in the region $r_2 > |\textbf{z}-\textbf{x}|$ posses the same formula then 
\begin{multline}
    \avg{\textbf{S}_k\Pi_\text{2-nst-f}^\text{nst-f} \sum_k \delta(\textbf{x}_k - \textbf{z}) }
    =
    \delta(\textbf{r}_1 - \textbf{z}) (\textbf{S}_{1}^\text{2-nst-f} - \textbf{S}_{1}^\text{nst-f}) P_\text{2-nst-f}
    + \delta(\textbf{r}_2 - \textbf{z}) \textbf{S}_2^\text{2-nst-f} P_\text{2-nst-f}\\
    % + P_\text{2-nst-f} n_p \int_{r_2 < |\textbf{z}' - \textbf{z}|}\textbf{S}[\textbf{z}'|\textbf{r}_1,\textbf{r}_2,\textbf{x}]\delta(\textbf{z}'-\textbf{z})d\textbf{z}
    % - P_\text{2-nst-f} n_p \int_{r_2 < |\textbf{z}' - \textbf{z}|}\textbf{S}[\textbf{z}'|\textbf{r}_1,\textbf{x}] \delta(\textbf{z}'-\textbf{z}) d\textbf{z}
    - P_\text{2-nst-f} n_p \int_{r_1 < |\textbf{z}' - \textbf{z}|<r_2}\textbf{S}[\textbf{z}'|\textbf{r}_1,\textbf{x}] \delta(\textbf{z}'-\textbf{z}) d\textbf{z}
\end{multline}
Note that the terms with $\delta(\textbf{r}_{1,2} - \textbf{z})$ in factor arent funciton of $\textbf{z}$ them selfs thus the final eq might be written, (when $P_\text{2-nst-f}\neq =0$)
\begin{align}
     \div \textbf{v}^\text{2-nst-f} &= 0 \\
    - \grad\tau_f^\text{2-nst-f} 
    +\mu_f \grad^2\textbf{v}^\text{2-nst-f}
    = 
    &- (\rho_f - \rho_d) \textbf{g} v \delta(\textbf{r}_2 - \textbf{z}) \\
    &+ (\rho_f - \rho_d) \textbf{g} v n_p \int_{r_1<|\textbf{z}' - \textbf{x}| <r_2} \delta(\textbf{z}'-\textbf{z})d\textbf{z} \\
    &-  (\textbf{S}_{1}^\text{2-nst-f} - \textbf{S}_{1}^\text{nst-f})  \cdot \grad \delta(\textbf{r}_1 - \textbf{z}) \\
    &-  \textbf{S}_2^\text{2-nst-f}  \cdot \grad \delta(\textbf{r}_2 - \textbf{z}) \\
    &+  n_p \int_{r_1 < |\textbf{z}' - \textbf{z}|<r_2}\textbf{S}[\textbf{z}'|\textbf{r}_1,\textbf{x}] \cdot \grad\delta(\textbf{z}'-\textbf{z}) d\textbf{z}\\
    &+ \frac{1}{2} V \rho_f \textbf{g}\grad^2\delta(\textbf{r}_2 - \textbf{z}) \\
    &- \frac{1}{2} V \rho_f \textbf{g} n_p \int_{r_1<|\textbf{z}' - \textbf{x}| <r_2} \grad^2\delta(\textbf{z}'-\textbf{z})d\textbf{z} \\
\end{align}


\section{Recursive demonstration with simplified PDF}
Let us consider the pdf, 
\begin{equation}
    P_n [\textbf{r}_n \ldots \textbf{r}_1|\textbf{x}]
    = 
    \frac{3\phi}{4\pi} e^{-\phi (r_n^3 -  1)}
\end{equation}
Equally we may use the notation , 
\begin{equation}
    P_n [\textbf{r}_n | \textbf{r}_{n-1} \ldots, \textbf{r}_1, \textbf{x}]
    = 
    \frac{3\phi}{4\pi} e^{-\phi (r_n^3 -  (r_{n-1}))}
\end{equation}

Using the decomposition etc we obtain 
\begin{align}
    \avg{\chi_f \textbf{u}_f'\textbf{u}_f'}/ \phi_f
    &=
    \int_{0}^{\infty}
    \left\{
        \textbf{u}^1
        \textbf{u}^1
        P_1 
        +
        \int_{r_1}^{\infty}
        \left[
        \textbf{u}^2
        \textbf{u}^2
        P_2 
        + \ldots
        \textbf{u}^{n-1}
        \textbf{u}^{n-1}
        P_{n-1}
        + 
        \int_{r_{n-1}}^{\infty}
        \textbf{u}^n
        \textbf{u}^n
        P_n
        \ldots
        d\textbf{r}_n 
        \right]
        d\textbf{r}_2
    \right\}
    d\textbf{r}_1 \\
    &=
    \sum_{i=1}^N 
    \prod_{k=1}^i \left(
        P_{k_{k-1}}
        \int_{r_{k-1}}^{\infty}
        d\textbf{r}_k 
    \right)
    \textbf{u}^i
    \textbf{u}^i
\end{align}


Assuming $u^i$ is the stokslet of the part in $i^{th}$ wehavegot, 
\begin{equation}
    (\textbf{u}^i )^2
     = 1/r_i^2
\end{equation}
\begin{align}
    \avg{\chi_f \textbf{u}_f'\cdot \textbf{u}_f'}/ \phi_f
    &=
    \sum_{i=1}^N 
    \prod_{k=1}^{i+1} \left(
        4\pi r_k^2
        \int_{r_{k-1}}^{\infty}
        d\textbf{r}_k 
    \right)
    \frac{1}{r_{i+1}^2}
    e^{-\phi(r_{i+1}^3- 1)}\\
    &=
    \sum_{i=1}^N 
    \prod_{k=1}^{i+1} \left(
        4\pi r_k^2
        \int_{r_{k-1}}^{\infty}
        d\textbf{r}_k 
    \right)
    \frac{1}{r_{i+1}^2}
    e^{-\phi(r_{i+1}^3- 1)}\\
\end{align}




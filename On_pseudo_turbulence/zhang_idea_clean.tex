\section{PTKE equaiton}


\begin{equation}
    \avg{\textbf{u}'\textbf{u}'}
    =
    \int_{\mathbb{R}^3}\avg{\textbf{u}'\textbf{u}'\delta_{nst}} d\textbf{r}
\end{equation}
In the first place we are looking for an equation for,
\begin{equation}
    \avg{\textbf{u}'\cdot \textbf{u}'}
    =
    \int_{\mathbb{R}^3}\avg{\textbf{u}'\cdot \textbf{u}'\delta_{nst}} d\textbf{r}
\end{equation}




% \subsection{Unaveraged equations }
% \paragraph{Momentum: }
% At the local non-averaged level we have,
% \begin{align}
%     \pddt(\rho_f\textbf{u}^0) + \div (\rho_f \textbf{u}^0\textbf{u}^0 + \bm\sigma^0_*)
%     &= \textbf{f}^0,
%     \;\;(=
%     \rho_f \textbf{g}
%     +\kappa  \delta_{\Gamma}  \bm\sigma_f^0 \cdot \textbf{n})\\
%     \bm\sigma^0_*
%     &=
%     -\sum_k \zeta^{-1}_k \chi_k \bm\sigma_k^0
%     =
%       - \chi_f \bm\sigma_f^0
%     - \zeta^{-1} (\chi_d \bm\sigma_d^0 + \delta_{\Gamma} \bm\sigma_{\Gamma}^0)
%     % - \zeta^{-1} (\chi_d \bm\sigma_d^0 + \delta_{\Gamma} \bm\sigma_{\Gamma}^0)
% \end{align}
% \paragraph{Energy: }
% \begin{align}
%     \pddt(\rho_f(\textbf{u}^0)^2/2) + \div (\rho_f \textbf{u}^0(\textbf{u}^0)^2/2 + \textbf{u}^0 \cdot \bm\sigma^0_*)
%     = \bm\sigma_*^0 : \grad \textbf{u}^0 + \textbf{u}^0 \cdot \textbf{f}^0\\
%     \textbf{u}^0 \cdot \bm\sigma^0_*
%     =
%     -\textbf{u}^0 \cdot \sum_k \zeta^{-1}_k \chi_k \bm\sigma_k^0
% \end{align}
% \subsection{Ensemble averaged equation}
% \paragraph{Momentum}
% \begin{align}
%     \pddt(\rho_f\textbf{u}) + \div (\rho_f \textbf{u}\textbf{u} + \bm\sigma_*)
%     &= \textbf{f},\\
%     \bm\sigma_*
%     &=
%     \avg{\rho_f \textbf{u}' \textbf{u}'}
%     -\avg{\sum_k \zeta^{-1}_k \chi_k \bm\sigma_k^0  }
% \end{align}
% \paragraph{ENERGY}
% \begin{align}
%     \pddt(\rho_f\textbf{u}^2/2) + \div (\rho_f \textbf{u}\textbf{u}^2/2 +\textbf{u}\cdot \bm\sigma_*)
%     &=\bm\sigma_*:\grad \textbf{u} + \textbf{u}\cdot \textbf{f},\\
%     \textbf{u}\cdot \bm\sigma_*
%     &=
%     \textbf{u}\cdot \avg{\rho_f \textbf{u}' \textbf{u}'}
%     -\textbf{u}\cdot  \avg{\sum_k \zeta^{-1}_k \chi_k \bm\sigma_k^0  }
% \end{align}
% \subsection{Disturbance field}
% \paragraph{Momentum:}
% Because $\textbf{u}^0 \textbf{u}^0 - \textbf{uu} = \textbf{uu}' + \textbf{u}' \textbf{u} + \textbf{u}' \textbf{u}'$
% \begin{align}
%     \pddt(\rho_f\textbf{u}') + \div (\rho_f \textbf{u}\textbf{u}' + \bm\sigma_*')
%     = \textbf{f}',\\
%     \bm\sigma_*'
%     =
%     % \textbf{uu}' +
%     \textbf{u}' \textbf{u} + \textbf{u}' \textbf{u}'
%     - \avg{\rho_f \textbf{u}' \textbf{u}'}\\
%     -\sum_k \zeta^{-1}_k \chi_k \bm\sigma_k^0
%     + \avg{\sum_k \zeta^{-1}_k \chi_k \bm\sigma_k^0  }
% \end{align}
% \begin{align}
%     \pddt(\rho_f\textbf{u}')
%     + \div (
%         \rho_f \textbf{u}\textbf{u}'
%         +\rho_f  \textbf{u}' \textbf{u}
%         +\rho_f  \textbf{u}' \textbf{u}'
%         + \bm\sigma_*')
%     = \textbf{f}',\\
%     \bm\sigma_*'
%     =
%     % \textbf{uu}' +
%     % \textbf{u}' \textbf{u}
%     - \avg{\rho_f \textbf{u}' \textbf{u}'}\\
%     -\sum_k \zeta^{-1}_k \chi_k \bm\sigma_k^0
%     + \avg{\sum_k \zeta^{-1}_k \chi_k \bm\sigma_k^0  }
% \end{align}
% Or
% \paragraph{Energy 2nd method (may be better):}

% \begin{align}
%     \pddt(\rho_f(\textbf{u}')^2/2)
%     + \div (
%         \rho_f \textbf{u}(\textbf{u}')^2/2
%         +\rho_f  \textbf{u}' (\textbf{u}')^2/2
%         + \bm\sigma_*'\cdot \textbf{u}')
%     =
%     - \rho_f \textbf{u}'\textbf{u}':\grad \textbf{u}
%     + \bm\sigma_*' : \grad \textbf{u}'
%     + \textbf{u}'\cdot \textbf{f}',\\
%     \bm\sigma_*'
%     =
%     % \textbf{uu}' +
%     % \textbf{u}' \textbf{u}
%     - \avg{\rho_f \textbf{u}' \textbf{u}'}
%     -\sum_k \zeta^{-1}_k \chi_k \bm\sigma_k^0
%     + \avg{\sum_k \zeta^{-1}_k \chi_k \bm\sigma_k^0  }
% \end{align}

% \paragraph{Energy 1st method (bof one):}
% By multiplying this equaiton by $\textbf{u}$ we obtain:
% \begin{align}
%     \pddt(\rho_f(\textbf{u}')^2/2) + \div (\rho_f \textbf{u}(\textbf{u}')^2/2 + \textbf{u}'\cdot \bm\sigma_*')
%     =\bm\sigma_*':\grad \textbf{u}'+ \textbf{u}' \cdot \textbf{f}',\\
%     \textbf{u}' \cdot \bm\sigma_*'
%     =
%     % \textbf{uu}' +
%     \textbf{u}'   (\textbf{u} \cdot \textbf{u}')
%     + \textbf{u}' (\textbf{u}'\cdot \textbf{u}')
%     - \avg{\rho_f \textbf{u}' \textbf{u}'}\cdot \textbf{u}' \\
%     -\sum_k \zeta^{-1}_k \chi_k \bm\sigma_k^0   \cdot \textbf{u}'
%     + \avg{\sum_k \zeta^{-1}_k \chi_k \bm\sigma_k^0  } \cdot \textbf{u}'
% \end{align}
% taking the average
% \begin{align}
%     \pddt(\rho_f\avg{(\textbf{u}')^2/2})
%     + \div (\rho_f \textbf{u}\avg{(\textbf{u}')^2/2} + \avg{\textbf{u}'\cdot \bm\sigma_*'} )
%     =
%     \avg{\bm\sigma_*':\grad \textbf{u}'}+ \avg{\textbf{u}' \cdot \textbf{f}'},\\
%     \avg{\textbf{u}' \cdot \bm\sigma_*'}
%     =
%     % \textbf{uu}' +
%     \avg{\textbf{u}' \textbf{u}'}\cdot \textbf{u}
%     + \avg{\textbf{u}' (\textbf{u}'\cdot \textbf{u}')}
%     -\avg{\sum_k \zeta^{-1}_k \chi_k \bm\sigma_k^0   \cdot \textbf{u}'}\\
%     \avg{ \bm\sigma_*' :\grad \textbf{u}'}
%     =
%     \avg{\textbf{u}' \textbf{u} : \grad \textbf{u}'} + \avg{\textbf{u}' \textbf{u}':\grad \textbf{u}'}
%     -\avg{\sum_k \zeta^{-1}_k \chi_k \bm\sigma_k^0  :\grad \textbf{u}'}
%     % + \avg{\sum_k \zeta^{-1}_k \chi_k \bm\sigma_k^0  }
% \end{align}
% \paragraph{Energy bis}
% Taking the difference of the above equations we may alos obtain
% \begin{equation}
%     \textbf{u}^0 \cdot \textbf{u}^0
%     - \textbf{u}\cdot \textbf{u}
%     =
%     + \textbf{u}' \cdot \textbf{u}
%     + \textbf{u} \cdot \textbf{u}'
%     + \textbf{u}' \cdot \textbf{u}'
% \end{equation}
% \begin{equation}
%     \textbf{u}^0\textbf{u}^0\cdot \textbf{u}^0-  \textbf{uu}\cdot \textbf{u}
%     =
%     % \textbf{u} \textbf{u}\cdot \textbf{u}
%     + \textbf{u} \textbf{u}\cdot \textbf{u}'
%     + \textbf{u} \textbf{u}'\cdot \textbf{u}
%     + \textbf{u} \textbf{u}'\cdot \textbf{u}'
%     + \textbf{u}'\textbf{u}\cdot \textbf{u}
%     + \textbf{u}'\textbf{u}'\cdot \textbf{u}
%     + \textbf{u}'\textbf{u}\cdot \textbf{u}'
%     + \textbf{u}'\textbf{u}'\cdot \textbf{u}'
%     % -  \textbf{uu}\cdot \textbf{u}
% \end{equation}
% \begin{equation}
%     \avg{\textbf{u}^0\textbf{u}^0\cdot \textbf{u}^0-  \textbf{uu}\cdot \textbf{u}}
%     =
%     % \textbf{u} \textbf{u}\cdot \textbf{u}
%     + \textbf{u} \avg{\textbf{u}'\cdot \textbf{u}'}
%     + 2\avg{\textbf{u}'\textbf{u}'}\cdot \textbf{u}
%     + \avg{\textbf{u}'\textbf{u}'\cdot \textbf{u}'}
%     % -  \textbf{uu}\cdot \textbf{u}
% \end{equation}
% When averaged $\avg{\textbf{u}'}$ disappear hence,

% \begin{align*}
%     \pddt(\rho_f \avg{(\textbf{u}')^2/2} )
%     + \div (\rho_f \textbf{u} \avg{(\textbf{u}')^2/2}
%     + \avg{\textbf{u}'\textbf{u}'}\cdot \textbf{u}
%     + \avg{\textbf{u}'\textbf{u}'\cdot \textbf{u}'}/2
%     + \avg{\textbf{u}^0 \cdot \bm\sigma^0_*} - \textbf{u}\cdot \bm\sigma_* )\\
%     =
%     \avg{\bm\sigma_*^0 : \grad \textbf{u}^0}
%     - \bm\sigma_* : \grad \textbf{u}
%     +
%     \avg{\textbf{u}^0 \cdot \textbf{f}^0}
%     - \textbf{u} \cdot \textbf{f}
%     \\
%     + \avg{\textbf{u}'\textbf{u}'}\cdot \textbf{u}
%     + \avg{\textbf{u}'\textbf{u}'\cdot \textbf{u}'}/2
%     + \avg{\textbf{u}^0 \cdot \bm\sigma^0_*} - \textbf{u}\cdot \bm\sigma_*
%     =
%     % + \avg{\textbf{u}'\textbf{u}'}\cdot \textbf{u}
%     % + \avg{\textbf{u}'\textbf{u}'\cdot \textbf{u}'}/2
%     % -\avg{\textbf{u}^0 \cdot \sum_k \zeta^{-1}_k \chi_k \bm\sigma_k^0  }
%     % - \textbf{u} \cdot \avg{\rho_f \textbf{u}'\textbf{u}'}
%     % + \textbf{u} \cdot \avg{\sum_k \zeta^{-1}_k \chi_k \bm\sigma_k^0  }\\
%     % =
%     + \avg{\rho_f \textbf{u}'\textbf{u}'\cdot \textbf{u}'}/2
%     -\avg{\textbf{u}' \cdot \sum_k \zeta^{-1}_k \chi_k \bm\sigma_k^0  }
%     % -\textbf{u} \cdot \avg{ \sum_k \zeta^{-1}_k \chi_k \bm\sigma_k^0  }
%     % + \textbf{u} \cdot \avg{\sum_k \zeta^{-1}_k \chi_k \bm\sigma_k^0  }
%     \\
%     \avg{\bm\sigma_*^0 : \grad \textbf{u}^0}
%     - \bm\sigma_* : \grad \textbf{u}
%     =
%     \avg{\bm\sigma_*^0 : \grad \textbf{u}'}
%     +\avg{\rho_f \textbf{u}' \textbf{u}'} : \grad \textbf{u}
% \end{align*}

\subsection{Unaveraged equations }
At the local non-averaged level we have,
\begin{align}
    \pddt(\rho_f\textbf{u}^0) + \div (\rho_f \textbf{u}^0\textbf{u}^0 + \bm\sigma^0_*)
    &= \textbf{f}^0,
    \;\;(=
    \rho_f \textbf{g}
    +\kappa  \delta_{\Gamma}  \bm\sigma_f^0 \cdot \textbf{n})\\
    \bm\sigma^0_*
    &=
    -\sum_k \zeta^{-1}_k \chi_k \bm\sigma_k^0
    =
      - \chi_f \bm\sigma_f^0
    - \zeta^{-1} (\chi_d \bm\sigma_d^0 + \delta_{\Gamma} \bm\sigma_{\Gamma}^0)
    % - \zeta^{-1} (\chi_d \bm\sigma_d^0 + \delta_{\Gamma} \bm\sigma_{\Gamma}^0)
\end{align}
\subsection{Ensemble averaged equation}
\begin{align}
    \pddt(\rho_f\textbf{u}) + \div (\rho_f \textbf{u}\textbf{u} + \bm\sigma_*)
    = \textbf{f},\\
    \bm\sigma_*
    =
    \avg{\rho_f \textbf{u}' \textbf{u}'}
    -\avg{\sum_k \zeta^{-1}_k \chi_k \bm\sigma_k^0  }
\end{align}
\subsection{Disturbance field}
Because $\textbf{u}^0 \textbf{u}^0 - \textbf{uu} = \textbf{uu}' + \textbf{u}' \textbf{u} + \textbf{u}' \textbf{u}'$
\begin{align}
    \pddt(\rho_f\textbf{u}')
    + \div (
        \rho_f \textbf{u}\textbf{u}'
        +\rho_f  \textbf{u}' \textbf{u}
        +\rho_f  \textbf{u}' \textbf{u}'
        + \bm\sigma_*')
    = \textbf{f}',\\
    \bm\sigma_*'
    =
    - \avg{\rho_f \textbf{u}' \textbf{u}'}
    -\sum_k \zeta^{-1}_k \chi_k \bm\sigma_k^0
    + \avg{\sum_k \zeta^{-1}_k \chi_k \bm\sigma_k^0  }
\end{align}

\paragraph{Energy 2nd method (may be better):}

\begin{align*}
    \pddt(\rho_f(\textbf{u}')^2/2)
    + \div (
        \rho_f \textbf{u}(\textbf{u}')^2/2
        + \textbf{q}^0
        )
    =
    - \rho_f \textbf{u}'\textbf{u}':\grad \textbf{u}
    + \bm\sigma_*' : \grad \textbf{u}'
    + \textbf{u}'\cdot \textbf{f}',\\
    \textbf{q}^0 =
    % \textbf{u}' (\textbf{u}')^2/2 +
    % \textbf{u}'\cdot \bm\sigma_*'
    % =
    \textbf{u}' (\textbf{u}')^2/2
    -\textbf{u}'\cdot  \avg{\rho_f \textbf{u}' \textbf{u}'}
    - \textbf{u}' \cdot \sum_k \zeta^{-1}_k \chi_k \bm\sigma_k^0
    + \textbf{u}' \cdot  \avg{\sum_k \zeta^{-1}_k \chi_k \bm\sigma_k^0  }\\
    \bm\sigma_*': \grad \textbf{u}'
    =
    - \avg{\rho_f \textbf{u}' \textbf{u}'}:\grad \textbf{u}'
    -\sum_k \zeta^{-1}_k \chi_k \bm\sigma_k^0  :\grad \textbf{u}'
    + \avg{\sum_k \zeta^{-1}_k \chi_k \bm\sigma_k^0  }:\grad \textbf{u}'\\
    \textbf{f}' \cdot \textbf{u}'
    =
    \kappa \delta_\Gamma \bm\sigma_f^0 \cdot \textbf{n}  \cdot \textbf{u}'
    -  \kappa \avg{\delta_\Gamma \bm\sigma_f^0 \cdot \textbf{n}}
      \cdot \textbf{u}'
    \\
\end{align*}
\tb{both reynolds stress term may be grouped together as they cancel to $\textbf{u}' \cdot (\div\avg{\textbf{u}'\textbf{u}'})$ }
\paragraph{Averaged equation gives }
with $k = \avg{(\textbf{u}')^2/2}$,
\begin{align*}
    \pddt(\rho_f k )
    + \div (
        \rho_f \textbf{u} k
        + \textbf{q}
        )
    &= 
    - \rho_f \avg{\textbf{u}'\textbf{u}'}:\grad \textbf{u}
    + \avg{\bm\sigma_*' : \grad \textbf{u}'}
    + \avg{\textbf{u}'\cdot \textbf{f}'},\\
    \textbf{q}^0 &= 
    % \textbf{u}' (\textbf{u}')^2/2 +
    % \textbf{u}'\cdot \bm\sigma_*'
    % =
    \avg{\textbf{u}' (\textbf{u}')^2/2}
    % -\textbf{u}'\cdot  \avg{\rho_f \textbf{u}' \textbf{u}'}
    - \avg{\textbf{u}' \cdot \sum_k \zeta^{-1}_k \chi_k \bm\sigma_k^0}
    % + \textbf{u}' \cdot  \avg{\sum_k \zeta^{-1}_k \chi_k \bm\sigma_k^0  }
    \\
    \avg{\bm\sigma_*': \grad \textbf{u}'}
    &=
    % - \avg{\rho_f \textbf{u}' \textbf{u}'}:\grad \textbf{u}'
    -\avg{\sum_k \zeta^{-1}_k \chi_k \bm\sigma_k^0  :\grad \textbf{u}'}\\
    \avg{\textbf{f}'\cdot \textbf{u}'}
    &=
    \avg{\textbf{f}^0 \cdot \textbf{u}'}
    = 
    \avg{\kappa \delta_\Gamma (\textbf{u}'\cdot \bm\sigma_f^0 \cdot \textbf{n})}
\end{align*}
\paragraph{Monophasique: }
Assuming the dorplets contribution is small then, 
\begin{align}
    \avg{\kappa \delta_\Gamma (\textbf{u}'\cdot \bm\sigma_f^0 \cdot \textbf{n})}&= 0 \\
    \avg{\sum_k \zeta^{-1}_k \chi_k \bm\sigma_k^0  :\grad \textbf{u}'}
    &=
    \mu_f \avg{ \grad \textbf{u}'  :\grad \textbf{u}'}\\
    \avg{\textbf{u}' \cdot \sum_k \zeta^{-1}_k \chi_k \bm\sigma_k^0}
    &=
    -\avg{\textbf{u}' p'}
    + 
    \mu_f \div \avg{\textbf{u}'\textbf{u}'}  + \mu_f  \grad k 
\end{align}
which end up to gives 
\begin{align*}
    \pddt(\rho_f k )
    + \div (
        \rho_f \textbf{u} k
        +\avg{\textbf{u}' (\textbf{u}')^2/2}
        + \avg{\textbf{u}' p'}
        )
    - \mu_f \grad\grad : \avg{\textbf{u}'\textbf{u}'}  
    - \mu_f  \grad^2 k 
    &= 
    - \rho_f \avg{\textbf{u}'\textbf{u}'}:\grad \textbf{u}
    - \mu_f \avg{\grad \textbf{u}':\grad \textbf{u}'}
\end{align*}
For isotropic turbulence let $\avg{\textbf{u}'\textbf{u}'} = 2 k \bm\delta$, then, 
\begin{align*}
    \pddt(\rho_f k )
    + \div (
        \rho_f \textbf{u} k
        +\avg{\textbf{u}' (\textbf{u}')^2/2}
        + \avg{\textbf{u}' p'}
        )
    - \mu_f  3\grad^2 k 
    &= 
    % - \rho_f \avg{\textbf{u}'\textbf{u}'}:\grad \textbf{u}
    - \mu_f \avg{\grad \textbf{u}':\grad \textbf{u}'}
\end{align*}
\paragraph{conditionally averaged equation: }
We introduce the distribution $\delta_N$ which represent a state that may be evolving such that $\pddt \delta_N \neq 0$. 
The conditionally averaged field $\avg{\delta_N \textbf{u}'} = P_N \textbf{u}^N - P_N \textbf{u}= P_N \textbf{v}^N$ represents the difference between the ensemble averaged and conditionally averaged situations, i.e. the disturbance field created by the condition. 
Also we may note $\avg{\delta_N (\textbf{u}')^2/2} = P_N k^N$. 

Then, multiplying the PTKE equaiton by $\delta_N$ and averaging yields, 
\begin{align*}
    \pddt(\rho_f P_N k^N )
    + \div (
        \rho_f \textbf{u} P_N k^N 
        + \textbf{q}^0
        )
    =
    \avg{(\textbf{u}')^2/2\pddt \delta_N }
    - \rho_f \avg{\delta_N \textbf{u}'\textbf{u}'}:\grad \textbf{u}
    + \avg{\delta_N\bm\sigma_*' : \grad \textbf{u}'} 
    + \avg{\delta_N \textbf{u}'\cdot \textbf{f}'},\\
    \textbf{q}^0 =
    \avg{\delta_N \textbf{u}' (\textbf{u}')^2/2}
    -P_N \textbf{v}^N \cdot  \avg{\rho_f \textbf{u}' \textbf{u}'}
    - \avg{\delta_N \textbf{u}' \cdot \sum_k \zeta^{-1}_k \chi_k \bm\sigma_k^0}
    + P_N \textbf{v}^N  \cdot  \avg{\sum_k \zeta^{-1}_k \chi_k \bm\sigma_k^0  }\\
    \avg{\delta_N \bm\sigma_*': \grad \textbf{u}'}
    =
    -P_N \avg{\rho_f \textbf{u}' \textbf{u}'}:\grad \textbf{v}^N 
    - \avg{\delta_N \sum_k \zeta^{-1}_k \chi_k \bm\sigma_k^0  :\grad \textbf{u}'}
    +P_N  \avg{\sum_k \zeta^{-1}_k \chi_k \bm\sigma_k^0  }:\grad \textbf{v}^N\\
    \avg{\delta_N \textbf{f}' \cdot \textbf{u}'}
    =
    \kappa \avg{\delta_\Gamma\delta_N  \textbf{u}'\cdot \bm\sigma_f^0 \cdot \textbf{n} }
    - P_N  \kappa \avg{\delta_\Gamma \bm\sigma_f^0 \cdot \textbf{n}}
      \cdot \textbf{v}^N
    \\
\end{align*}
The bounday for $k^N$ is then, 
\begin{equation}
    \lim_{|\textbf{x}- \textbf{z}| \to \infty} k^N = k
\end{equation}
\section{Conditionally averaged equation}


Let $\delta_N(\textbf{y},\textbf{y}+ \textbf{r})$ be the fact that the nearest particle to the point $\textbf{y}$ is at $\textbf{y}+ \textbf{r}$, in the dilute limit one may expect that for all point $|\textbf{x}-( \textbf{y}+ \textbf{r})| = |\textbf{x}- \textbf{z}| > a $ one is located in pure liquid phase. 
For those points we can evaluate the following distributions, 
\begin{align}
    P_N \textbf{v}^N \cdot  \avg{\sum_k \zeta_k^{-1}\chi_k \bm\sigma_k^0 }
    &\approx  
    P_N\textbf{v}^N \cdot   (-p_f \bm\delta + \mu_f (\grad \textbf{u} + ^\dagger \grad \textbf{u} ) ) 
    + O(P_N^2)\\
    \avg{\delta_N \textbf{u}' \cdot \sum_k \zeta^{-1}_k \chi_k \bm\sigma_k^0}
    &\approx 
    \avg{\delta_N \textbf{u}' \cdot 
    (-p_f^0  \bm\delta + \mu_f (\grad \textbf{u}^0  + ^\dagger \grad \textbf{u}^0  ))
    }\\
    &=  
    P_N \textbf{v}^N  \cdot 
    (-p_f  \bm\delta + \mu_f (\grad \textbf{u}  + ^\dagger \grad \textbf{u}  ))\\
    &+   
    -\avg{\delta_N \textbf{u}' p_f'} 
    + \mu_f (\div \avg{\delta_N \textbf{u}'\textbf{u}'}  + P_N  \grad k^N) 
    \\
    P_N  \avg{\sum_k \zeta^{-1}_k \chi_k \bm\sigma_k^0  }:\grad \textbf{v}^N
    &= P_N \bm\Sigma : \grad \textbf{v}^N \\
    P_N  \avg{ \delta_N \grad \textbf{u}' : \sum_k \zeta^{-1}_k \chi_k \bm\sigma_k^0  }
    &= P_N \bm\Sigma : \grad \textbf{v}^N 
    +2  P_N   \avg{ \delta_N \grad \textbf{u}' : \grad \textbf{u}'}
\end{align}
In this situations we have for $\textbf{x}- \textbf{z}>a$, 

\begin{align*}
    \pddt(\rho_f P_N k^N )
    + \div (
        \rho_f \textbf{u} P_N k^N 
        + \textbf{q}^0
        )
    =
    \avg{(\textbf{u}')^2/2\pddt \delta_N }
    - \rho_f \avg{\delta_N \textbf{u}'\textbf{u}'}:\grad \textbf{u}
    + \avg{\delta_N\bm\sigma_*' : \grad \textbf{u}'} 
    % + \avg{\delta_N \textbf{u}'\cdot \textbf{f}'},
    \\
    \textbf{q}^0 =
    \avg{\delta_N \textbf{u}' (\textbf{u}')^2/2}
    -P_N \textbf{v}^N \cdot  \avg{\rho_f \textbf{u}' \textbf{u}'}
    +\avg{\delta_N \textbf{u}' p_f'} 
    - \mu_f \div \avg{\delta_N \textbf{u}'\textbf{u}'}  + P_N\mu_f  \grad k^N \\
    \avg{\delta_N \bm\sigma_*': \grad \textbf{u}'}
    =
    -P_N \avg{\rho_f \textbf{u}' \textbf{u}'}:\grad \textbf{v}^N 
    - 2  P_N   \avg{ \delta_N \grad \textbf{u}' : \grad \textbf{u}'}
\end{align*}
Or, 
\begin{multline*}
    \pddt(\rho_f P_N k^N )
    + \div (
        \rho_f \textbf{u} P_N k^N 
        + 
        \avg{\delta_N \textbf{u}' (\textbf{u}')^2/2}
    % -P_N \textbf{v}^N \cdot  \avg{\rho_f \textbf{u}' \textbf{u}'}
    +\avg{\delta_N \textbf{u}' p_f'} 
    - \mu_f \div \avg{\delta_N \textbf{u}'\textbf{u}'}  + P_N\mu_f  \grad k^N
        )\\
    =
    \avg{(\textbf{u}')^2/2\pddt \delta_N }
    - \rho_f \avg{\delta_N \textbf{u}'\textbf{u}'}:\grad \textbf{u}
    + P_N \textbf{v}^N \cdot \div\avg{\rho_f \textbf{u}' \textbf{u}'}
    - 2  P_N   \avg{ \delta_N \grad \textbf{u}' : \grad \textbf{u}'}
\end{multline*}
Assuming that $\avg{\delta_N \textbf{u}'\textbf{u}'} = \textbf{v}^N\textbf{v}^NP_N + \avg{\delta_N \textbf{u}'' \textbf{u}''} $ with $\textbf{u}'' = \textbf{u}' - \textbf{v}^N$. 


Let consider an homogeneous sedimentation in which case, 
\begin{itemize}
    \item All ensemble averaged quantities are constants $\div\avg{\textbf{u}'\textbf{u}'} = 0 $ and $\grad \textbf{u} = 0$. 
    \item The pressure $p_f$ may be function of coordinate but that is it. 
    \item Steady state regime
\end{itemize}
\begin{multline*}
    % \pddt(\rho_f P_N k^N )
    \rho_fP_N  \textbf{u} \cdot \grad k^N 
    + 
    + \div ( \avg{\delta_N \textbf{u}' (\textbf{u}')^2/2})
    +\avg{\delta_N \textbf{u}' p_f'} )
    - \mu_f \grad\grad : \avg{\delta_N \textbf{u}'\textbf{u}'}  
    + P_N\mu_f  \grad^2 k^N
\\
    =
    % - \rho_f \avg{\delta_N \textbf{u}'\textbf{u}'}:\grad \textbf{u}
    % + P_N \textbf{v}^N \cdot \div\avg{\rho_f \textbf{u}' \textbf{u}'}
    - 2  P_N   \avg{ \delta_N \grad \textbf{u}' : \grad \textbf{u}'}
\end{multline*}
so this is kind of diffusion equaiton, 
The BCs are, for solid particles 

At the local level we have on the surface of the droplets, 
\begin{equation}
    \delta_\Gamma(\textbf{u}^0 - \textbf{u} )= \delta_\Gamma (\textbf{u}_\alpha - \textbf{u})
\end{equation}
Or, 
\begin{equation}
    \avg{\delta_N\delta_\Gamma (\textbf{u}')^2/2} = \avg{\delta_N\delta_\Gamma (\textbf{u}_\alpha - \textbf{u})\cdot \textbf{u}'/2}
\end{equation}
At $|\textbf{x}-\textbf{z}| = a$ one is on an interface hence $\delta_\Gamma$ can be dropped, and the center of mass velocity is prescribed by $\delta_N$ also, 
\begin{equation}
    k^N = (\textbf{w} - \textbf{u})\cdot \textbf{v}^N /2 
\end{equation}
The expression of $\textbf{v}^N$ at the surface of the particle can ten be found solving the momentum equaiton, one find $\textbf{v}^N = (\textbf{w} - \textbf{u})$ of course hence, 
\begin{equation}
    k^N = (\textbf{w} - \textbf{u})\cdot (\textbf{w} - \textbf{u})/2 ,
\end{equation}
is without surprise the correct boundary condition. 


Let consider a sedimentation case in a static vessel in which case \textbf{u}, also let us neglect the closures terms, 
\begin{align}
     \avg{\delta_N \textbf{u}' (\textbf{u}')^2/2}
     =
     \textbf{v}^N  k^N 
     + \avg{\delta_N \textbf{u}'' (\textbf{u}')^2/2}
    &&
    \avg{\delta_N \textbf{u}' p_f'} 
    =
    P_N \textbf{v}^N p^N 
    + \avg{\delta_N \textbf{u}'' p_f''} \\
    \avg{ \delta_N \grad \textbf{u}' : \grad \textbf{u}'}
    =
    P_N \grad \textbf{v}^N  : \grad \textbf{v}^N
    + \avg{ \delta_N \grad \textbf{u}'' : \grad \textbf{u}''}
\end{align}
\begin{equation}
    \rho_f \textbf{v}^N \cdot \grad  k^N
    - \mu_f \grad\grad : \avg{\delta_N \textbf{u}'\textbf{u}'}  
    - P_N\mu_f  \grad^2 k^N
    =
    - 2 P_N \grad \textbf{v}^N  : \grad \textbf{v}^N
    - \div (\textbf{v}^N  p^N)
\end{equation}

Addmetons dans un premier temps que $\textbf{v}^N\sim \textbf{F}\cdot  (\bm\delta + \textbf{nn})r^{-1}$ 
the homogeneous equaiton yields
\begin{align}
    \frac{a \rho_f \textbf{v}^N}{\mu_f} \cdot \grad  k^N
    % - \mu_f \grad\grad : \avg{\delta_N \textbf{u}'\textbf{u}'}  
    -  \grad^2 k^N
    =
    - 2 \grad \textbf{v}^N  : \grad \textbf{v}^N
    - \div (\textbf{v}^N  p^N)\\
    \textbf{v}^N \sim  \textbf{F}\cdot  (\bm\delta + \textbf{nn})r^{-1}\\
    p^N = \textbf{F}\cdot \textbf{n}r^{-2}
\end{align}

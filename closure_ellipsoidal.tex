\documentclass[12pt]{My_preprint}

%%%%%%%%%%%%%%%%%%%%%%%%%%%%%%%%%%%%%%%%%%%%%%%%%%%%%%%%%%%%%%%%%%%%%%%%%%%%%%%

%%%%%%%%%%%%%%%%%%%%%%%%%%%%%%% Title & Author %%%%%%%%%%%%%%%%%%%%%%%%%%%%%%%%


%\title{The hybrid model for arbitrary dispersed multiphase flows with surface properties}
\title{Averaged equations for spheroidal particles  }

\author[1,2]{Nicolas Fintzi}
\author[1]{Jean-Lou Pierson}
% \author[2]{Stephane Popinet}
\author[2]{Daniel Lhuillier}
\affil[1]{IFP Energies Nouvelles, Rond-point de l’changeur de Solaize, 69360 Solaize}
\affil[2]{Sorbonne Université, Institut Jean le Rond d'Alembert, 4 place Jussieu, 75252 PARIS CEDEX 05, France}


\begin{document}

\maketitle

\begin{abstract}
    In this work we propose to derive an averaged set of equations for dilute suspension of spheroidal particles. 
    We consider small but non-zero effects of inertial, such that $Re<1$, where $Re$ is the Reynolds number based on the relative velocity. 
    The Reynolds number based on shear, and rotational motion are assumed to be small.
    The closure are derived using the methodology of \citet{brenner1963resistance} within the Stokes flow hypothesis. 
    Then the results are extended to slightly inertial flow using the reciprocal theorem based on the stokes flow solution. 
\end{abstract}


\section{Dynamic of a single particle}

We consider a spheroidal particle, which volume is described by the vectors, 
\begin{equation}
    \textbf{r} 
    = r [1 + \epsilon \textbf{H}:\textbf{S}(\textbf{n})]\textbf{e}_r,
    = r [1 + f]\textbf{n},
\end{equation}
where $r\in [0;1]$, $\textbf{S} = \textbf{nn} - \frac{1}{3}\bm\delta$ is the surface spherical harmonics of order $2$, and \textbf{H} the second order tensor quantifying the deformation of the spheroidal particle (see \citet{nadim1996concise} for a geometrical interpretation).  
$\epsilon\ll 1$ is a very small parameter. 

To understand how $\textbf{H}$ is related to the three radius of the particle we use the relation, 
\begin{equation}
    \int_{V} \textbf{rr} dV = (\frac{4\pi a^2}{3}) a^2/5(\bm\delta+ 2\epsilon \textbf{H})
\end{equation}
Also, by direct calculation in the principal axis of the spheroid, noted $\textbf{p}_n$ one obtain, 
\begin{equation}
    \int \textbf{rr} dV =  \frac{4\pi a^3}{3} ((a_1/a)^2 \textbf{p}_1 \textbf{p}_1+ (a_2/a)^2\textbf{p}_2 \textbf{p}_2 + (a_3/a)^2\textbf{p}_3 \textbf{p}_3)a^2/5
\end{equation}
see \url{https://mathworld.wolfram.com/Ellipsoid.html}. 
Hence, our definition of \textbf{H} is, 
\begin{equation}
    2\epsilon \textbf{H}= (a_1/a)^2 \textbf{p}_1 \textbf{p}_1+ (a_2/a)^2\textbf{p}_2 \textbf{p}_2 + (a_3/a)^2\textbf{p}_3 \textbf{p}_3 - \bm\delta,
\end{equation}
one may eventually express $\textbf{p}_3$ as $\textbf{p}_1 \times \textbf{p}_2$ hence only to independent vector constitutes $\textbf{H}$. 
Note also that the isotropic part vanish in the torque expression given below. 


For instance, we discard the effect of inertia. 
In this case the equations governing the disturbance pressure and velocity fields $(p,\textbf{u})$, around the isolated particle are, 
\begin{align}
    \div \textbf{u}= 0&& \grad^2 \textbf{u}=\grad p\\
    \textbf{u} = \textbf{u}_r\;\; \forall r = 1+\epsilon \textbf{H}:\textbf{S} 
    && \lim_{r\to \infty} (\textbf{u},p) = (\textbf{0},0)
\end{align}
where $\textbf{u}_r = \textbf{U}-   \textbf{u}_p$, with \textbf{U} the background velocity field, and $\textbf{u}_p$ the center of mass velocity field. 

Because we cannot apply these boundaries condition directly we express any of the above function as a taylor expansion in terms of $\epsilon$ such that, any tensor \textbf{F} can be expressed as, 
\begin{equation}
    \textbf{F}
    = \ps{0}{\textbf{F}}
    + \epsilon \ps{1}{\textbf{F}}
    + O(\epsilon^2)
\end{equation}
Then, any function to be evaluated on the droplet interface might be written as, 
\begin{equation}
    \textbf{F}(r=1+\epsilon f)
    % =
    % \textbf{F}|_{r=1}
    % + \epsilon f \frac{\partial \textbf{F}}{\partial r}|_{r=1}
    =
    \ps{0}{\textbf{F}}|_{r=1}
    + \epsilon \left[\ps{1}{\textbf{F}}
    + f \frac{\partial\ps{0}{\textbf{F}}}{\partial r}\right]_{r=1}
    + O(\epsilon)
    =
    \ps{0}{\textbf{F}}|_{r=1}
    + \epsilon \left[\ps{1}{\textbf{F}}
    + f (\textbf{n}\cdot \grad) \ps{0}{\textbf{F}} \right]_{r=1}
    + O(\epsilon). 
\end{equation}
Doing so one can solve a stokes problem for $\ps{0}{\textbf{u}}$ in terms of $\textbf{u}_r$ and of $\ps{1}{\textbf{u}}$ in terms of $\textbf{H}$. 
Note that at this stage $\textbf{u}_r$ is arbitrary, and can represent uniform, linear or quadratic flow. 

Once these solutions are obtained one can use the reciprocal theorem to find out the effect of inertia on the first moment on a translating spheroidal particle. 
This is done by applying the reciprocal theorem between $Eq(\textbf{u}^{Re})$ which are the equations governing the inertial flow, and $Eq(\textbf{u})$ which represent the stokes flow around that particle. 
In the configuration where $Eq(\textbf{u}^{Re})$ represent uniform translation of the spheroidal particle and $Eq(\textbf{u})$ a linear flow around the deformed droplet in stokes flow, one obtain the expression, 
\begin{equation}
    \intS[p]{\textbf{r}\bm\sigma^{Re}\cdot \textbf{n}}
    % - \intO[i]{2\textbf{e}_i}
    % - (1-\lambda) \intO[p]{(\textbf{u}_{i} + \textbf{u}_r)\cdot \grad^2 \hat{\textbf{u}}}
    % + \intS[p]{2(\textbf{u}_{i} + \textbf{u}_r)\cdot \hat{\textbf{b}}}
    % \\
    =
    % \intS[p]{\textbf{u}_{r}\cdot \textbf{S}^{(2)}_{o}\cdot \textbf{n}}
    % - \intO[i]{\textbf{S}^{(2)}_{i}:\grad\textbf{u}}
    % - (1-\lambda) \intO[p]{(\textbf{U}_{i}^{(2)} + \textbf{r}\bm\delta) \cdot \grad^2 \textbf{u} }\\ 
    % + \intS[p]{2(\textbf{U}_{i}^{(2)} + \textbf{r}\bm\delta) \cdot  \textbf{b}}
    % + \zeta Re \intO{(\textbf{U}_{i}^{(2)} + \textbf{r}\bm\delta)\cdot \textbf{f}_{i}} 
    \ldots + Re\intO[o]{\textbf{U}_{o}^{(2)}\cdot \textbf{f}_{o}},
    = \ldots + Re\intO[o]{\textbf{U}_{o}^{(2)}\cdot (\textbf{U}^{(1)} + \bm\delta)\cdot \grad \textbf{U}^{(1)}} : \textbf{u}_r\textbf{u}_r,
    \label{eq:first_mom}
\end{equation}
where $\textbf{U}_{o}^{(2)}\cdot \grad \textbf{U}$ is the disturbance velocity field generated by a spheroidal particle immersed in pure linear flow, and $\textbf{U}_{o}^{(2)}\cdot \grad \textbf{u}_r$ the disturbance field of the same particle immersed in a uniform flow of relative velocity $\textbf{u}_r$. 
The terms represented by the $\ldots$ represent the other non-inertial contribution partially  given by \citet{brenner1963resistance} (he does not provide the Stresslet). 

Carrying the computation and retaining only the term proportional to $Re \epsilon$ (neglecting the $\epsilon^2$ terms), yields, 

\begin{align}
    M_{ij}=
    \frac{1}{2}\intS[p]{[ r_i (\bm\sigma^{Re}\cdot \textbf{n})_j- r_j (\bm\sigma^{Re}\cdot \textbf{n})_i]}
    =
    \ldots
      Re \epsilon \pi
    \frac{29}{40}[H_{jk}(\textbf{u}_r)_i - H_{ik}(\textbf{u}_r)_j ] (\textbf{u}_r)_k
    \label{eq:first_mom}
\end{align}
where $\textbf{R}$ is the resistance matrix of the spheroidal particle in stokes flow. 

Then the torque may be obtained as, 
\begin{equation}
    T_l = - \frac{1}{2}\epsilon_{lij}M_{ij}
    % =- \frac{1}{2}\epsilon_{lij} Re \epsilon \pi 
    % \frac{29}{40}[H_{jk}(\textbf{u}_r)_i - H_{ik}(\textbf{u}_r)_j ] (\textbf{u}_r)_k
    =- \frac{1}{2} Re \epsilon \pi 
    \frac{29}{40} \epsilon_{lij}H_{jk}(\textbf{u}_r)_i  (\textbf{u}_r)_k
    =\tb{\pm(\frac{1}{2} ?)} Re \epsilon \pi 
    \frac{29}{40} (\textbf{u}_r\times \textbf{H})_{lk}  (\textbf{u}_r)_k
\end{equation}

\tb{Voir \citet[p.29]{guazzelli2011} pour une definitions des pseudo vecteur, car le $1/2$ me semble un peu arbitraire. Et surtout voir quelle convention utilise Cox. Je crois que pour le signe c'est aussi une affaire de right-hand rule ou non donc a verifier }
To make the results dimensional multiply the above expression by $a^2 \mu_f U$ where $a$ is the mean radius of the sphere of same volume. Because upon changing $\textbf{H}$ the volume does not change. 

\bibliography{Bib/bib_bulles.bib}



\appendix


\end{document}

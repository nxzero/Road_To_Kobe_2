\documentclass[12pt]{My_preprint}
\title{
    Singularity solutions
    }

\author[1,2]{Nicolas Fintzi}
\normalmarginpar

\newcommand{\ii}{\delta}

\begin{document}

\maketitle

\begin{abstract}
\end{abstract}

\section{Spherical harmonic}
The spherical harmonic of order $^n$ is noted,
\begin{equation}
    h^{n}_{i_1i_2\ldots i_n}
    =
    \prod_{e=1}^{n}
    \partial_{i_e}
    r^{-1}
    = \grad^{(n)}
    \odot 
    r^{-1}
\end{equation}
Note that, 
\begin{equation}
    \partial_i r^{-n}
    =
    - n
    x_i r^{-n-2}
\end{equation}
Hence, 
\begin{align*}
    h^{(0)} &= r^{-1}\\
    h^{(1)}_i &= - x_i r^{-3}\\
    h^{(2)}_{ij} &= (3 x_i x_j r^{-5} - \ii_{ij} r^{-3})\\
    h^{(3)}_{ijk} &= [3 (x_i \ii_{jk} + x_j \ii_{ki} + x_k \ii_{ij})r^{-5} - 15 x_ix_jx_k r^{-7}]\\
    h^{(4)}_{ijkl} &= [
    3 (\ii_{il} \ii_{jk} + \ii_{jl} \ii_{ki} + \ii_{kl} \ii_{ij})r^{-5} \\
    &- 15 (x_ix_l  \ii_{jk} + x_jx_l  \ii_{ki} + x_k x_l \ii_{ij} + \ii_{il}x_jx_k+ x_i x_k\ii_{jl}+x_ix_j\ii_{kl}) r^{-7}\\
    &+ 105  x_ix_jx_kx_l r^{-9}
    ]\\
    h^{(5)}_{ijklm} &= \{
    - 15 [
        x_i ( \ii_{lm} \ii_{jk}+\ii_{km}\ii_{jl}+\ii_{kl}\ii_{jm}) 
        + x_j ( \ii_{lm}\ii_{ki}+\ii_{il}\ii_{km}+\ii_{kl}\ii_{im})\\
        &+ x_k (\ii_{ij}\ii_{lm}+\ii_{il}\ii_{jm} +\ii_{jl}\ii_{im})
        + x_l (\ii_{im} \ii_{jk}+ \ii_{jm}\ii_{ki}+\ii_{ij}\ii_{im})
        + x_m (\ii_{il} \ii_{jk} + \ii_{jl} \ii_{ki} + \ii_{kl} \ii_{ij})
     ] r^{-7} \\
    &+ 105  (x_mx_ix_l  \ii_{jk} + x_m x_jx_l  \ii_{ki} + x_m x_k x_l\ii_{ij} + \ii_{il}x_jx_kx_m + x_i x_kx_m\ii_{jl}+x_ix_jx_m \ii_{kl}\\
    &+ \ii_{im}x_jx_kx_l+x_ix_kx_l\ii_{jm}+x_ix_jx_l\ii_{km}+x_ix_jx_k\ii_{lm}) r^{-9}\\
    &+ 945  x_ix_jx_kx_lx_m r^{-11}
    \} \\
\end{align*}
We conclude that 
\begin{itemize}
    \item The dimension of $h^{(n)}$ is always of $[L^{-n-1}]$ because of its definition. 
    \item $h^{(n)}$ is always symmetric on all its indices, because $\grad\grad\grad \ldots$ is a symmetric operator. 
    \item If $n$ is even, the first term of the series is $\Delta^{(n)} r^{-n-1}$, and when it is odd $(\textbf{x}\Delta)^{(n)} r^{-n-1}$, where $(\ldots)^{(n)}$ returns the fully symmetric tensor of the arguments, and $\Delta$ is the product of identity tensor.
    \item The last term is always $\textbf{x}\ldots\textbf{x} r^{-2n+1}$
\end{itemize}

\paragraph*{Additional relations}
\begin{align*}
    x_i h^{(0)} &= - h_i^{(1)} r^{2}\\
    x_j h^{(1)}_i &= -\frac{r^2}{3}(3 x_i x_j r^{-5} - \ii_{ij} r^{-3} + \ii_{ij} r^{-3})  \\
    h^{(2)}_{ij} &= (3 x_i x_j r^{-5} - \ii_{ij} r^{-3}) = \\
    h^{(3)}_{ijk} &= [3 (x_i \ii_{jk} + x_j \ii_{ki} + x_k \ii_{ij})r^{-5} - 15 x_ix_jx_k r^{-7}]\\
    h^{(4)}_{ijkl} &= [
    3 (\ii_{il} \ii_{jk} + \ii_{jl} \ii_{ki} + \ii_{kl} \ii_{ij})r^{-5} \\
    &- 15 (x_ix_l  \ii_{jk} + x_jx_l  \ii_{ki} + x_k x_l \ii_{ij} + \ii_{il}x_jx_k+ x_i x_k\ii_{jl}+x_ix_j\ii_{kl}) r^{-7}\\
    &+ 105  x_ix_jx_kx_l r^{-9}
    ]\\
    h^{(5)}_{ijklm} &= \{
    - 15 [
        x_i ( \ii_{lm} \ii_{jk}+\ii_{km}\ii_{jl}+\ii_{kl}\ii_{jm}) 
        + x_j ( \ii_{lm}\ii_{ki}+\ii_{il}\ii_{km}+\ii_{kl}\ii_{im})\\
        &+ x_k (\ii_{ij}\ii_{lm}+\ii_{il}\ii_{jm} +\ii_{jl}\ii_{im})
        + x_l (\ii_{im} \ii_{jk}+ \ii_{jm}\ii_{ki}+\ii_{ij}\ii_{im})
        + x_m (\ii_{il} \ii_{jk} + \ii_{jl} \ii_{ki} + \ii_{kl} \ii_{ij})
     ] r^{-7} \\
    &+ 105  (x_mx_ix_l  \ii_{jk} + x_m x_jx_l  \ii_{ki} + x_m x_k x_l\ii_{ij} + \ii_{il}x_jx_kx_m + x_i x_kx_m\ii_{jl}+x_ix_jx_m \ii_{kl}\\
    &+ \ii_{im}x_jx_kx_l+x_ix_kx_l\ii_{jm}+x_ix_jx_l\ii_{km}+x_ix_jx_k\ii_{lm}) r^{-9}\\
    &+ 945  x_ix_jx_kx_lx_m r^{-11}
    \} \\
\end{align*}

Meaning that, 
\begin{equation}
    \textbf{h}^n = \Delta
    + \ldots + \prod_{k=1}^{n/2} (2k+1)
\end{equation}
Based on this last remark we build growing harmonic based on, 
\begin{equation}
    g^{n}_{i_1i_2\ldots i_n}
    =
    r^{2n+1}
    \prod_{e=1}^{n}
    \partial_{i_e}
    r^{-1}
    =r^{2n+1} \grad^{(n)}
    \odot 
    r^{-1}
\end{equation}
which reads,
\begin{align*}
    g^{(0)} &= 1\\
    g^{(1)}_i &= - x_i \\
    g^{(2)}_{ij} &= (3 x_i x_j - \ii_{ij} r^{2})\\
    g^{(3)}_{ijk} &= [3 (x_i \ii_{jk} + x_j \ii_{ki} + x_k \ii_{ij})r^2 - 15 x_ix_jx_k ]\\
    g^{(4)}_{ijkl} &= [
    3 (\ii_{il} \ii_{jk} + \ii_{jl} \ii_{ki} + \ii_{kl} \ii_{ij})r^4 \\
    &- 15 (x_ix_l  \ii_{jk} + x_jx_l  \ii_{ki} + x_k x_l \ii_{ij} + \ii_{il}x_jx_k+ x_i x_k\ii_{jl}+x_ix_j\ii_{kl}) r^2\\
    &+ 105  x_ix_jx_kx_l 
    ]\\
    g^{(5)}_{ijklm} &= \{
    - 15 [
        x_i ( \ii_{lm} \ii_{jk}+\ii_{km}\ii_{jl}+\ii_{kl}\ii_{jm}) 
        + x_j ( \ii_{lm}\ii_{ki}+\ii_{il}\ii_{km}+\ii_{kl}\ii_{im})\\
        &+ x_k (\ii_{ij}\ii_{lm}+\ii_{il}\ii_{jm} +\ii_{jl}\ii_{im})
        + x_l (\ii_{im} \ii_{jk}+ \ii_{jm}\ii_{ki}+\ii_{ij}\ii_{im})
        + x_m (\ii_{il} \ii_{jk} + \ii_{jl} \ii_{ki} + \ii_{kl} \ii_{ij})
     ] r^4 \\
    &+ 105  (x_mx_ix_l  \ii_{jk} + x_m x_jx_l  \ii_{ki} + x_m x_k x_l\ii_{ij} + \ii_{il}x_jx_kx_m + x_i x_kx_m\ii_{jl}+x_ix_jx_m \ii_{kl}\\
    &+ \ii_{im}x_jx_kx_l+x_ix_kx_l\ii_{jm}+x_ix_jx_l\ii_{km}+x_ix_jx_k\ii_{lm}) r^2\\
    &+ 945  x_ix_jx_kx_lx_m 
    \} \\
\end{align*}


Contraction with a vector, 
\begin{align*}
    x_i h^{(1)}_{i} &= -x_ix_i r^{-3} = - h^{(0)}\\
    x_i h^{(2)}_{ij} &= 2  x_j r^{-3} = -2 h^{(1)}_j \\
    h^{(2)}_{ii} &= (3 x_i x_i r^{-5} - \ii_{ii} r^{-3}) = 0 \\
    h^{(3)}_{iji} &= [15 r^{-5} - 15 x_j r^{-5}] = 0\\
    x_i h^{(3)}_{ijk} &= [3 (x_i x_i \ii_{jk} + x_j x_i \ii_{ki} +x_i  x_k \ii_{ij})r^{-5} - 15 x_i x_ix_jx_k r^{-7}]\\
     &= 3 [ \ii_{jk}r^{-3} - 3 x_jx_k r^{-5}] = - 3 h_{jk}^{(2)} \\
     h^{(4)}_{ijki} &= [
    3 (\ii_{ii} \ii_{jk} + \ii_{ji} \ii_{ki} + \ii_{ki} \ii_{ij})r^{-5} \\
    &- 15 (x_ix_i  \ii_{jk} + x_jx_i  \ii_{ki} + x_k x_i \ii_{ij} + \ii_{ii}x_jx_k+ x_i x_k\ii_{ji}+x_ix_j\ii_{ki}) r^{-7}\\
    &+ 105  x_ix_jx_kx_i r^{-9}
    ] \\
     &= 0 \\
\end{align*}

It seems that, 
\begin{align*}
    h_{ii\ldots}^{(n)} &= 0\\
    x_i h_{ijk\ldots }^{(n)} &= -n h_{jk\ldots }^{(n-1)}  
\end{align*}
This is always true because, 
\begin{equation}
    \delta_{i_1 i_2} h^{n}_{i_1i_2\ldots i_n}
    =
    h^{n}_{i_1i_1\ldots i_n}
    =
    \prod_{e=3}^{n}
    \partial_{i_e}(
    \partial_{i_1}^2
    r^{-1})
    = \grad^{(n-2)}
    \odot 
    h_{ii}^{(2)}
    = 0 
\end{equation}
The second is a bit more complicated, 
\begin{align}
    x_{i_n} \partial_{i_1\ldots i_n} (1/r)
    &=
    \partial_{i_1\ldots i_{n-1}}  (x_{i_n} \partial_{i_n} (1/r))
    - 
    \sum_{k=1}^{n-1}
    \partial_{i_1\ldots i_{n-1}}  (1/r)\\
    &= 
    - \partial_{i_1\ldots i_{n-1}}  (1/r)
    - 
    \sum_{k=1}^{n-1}
    \partial_{i_1\ldots i_{n-1}}  (1/r)\\
    &= -n 
    \partial_{i_1\ldots i_{n-1}}  (1/r)
\end{align}

Derivatives of growing harmonic, 
Note that $\grad h^{(n)} = h^{(n+1)}$ however $\grad g^{(n)} \neq g^{(n+1)}$. 
\begin{align*}
    \partial_j g^{(1)}_i
    &=
    - \ii_{ij}\\
    \partial_k g^{(1)}_{ij}
    &=
    (3 (\ii x_j+x_i \ii_{jk}) - 2 x_k \ii_{ij})\\
    \partial_l g^{(3)}_{ijk} 
    &= [
        3 (\ii_{il} \ii_{jk} + \ii_{jl} \ii_{ki} + \ii_{kl} \ii_{ij})r^2 
        + 6 x_l (x_i \ii_{jk} + x_j \ii_{ki} + x_k \ii_{ij})
    - 15 (\ii_{il}x_jx_k+x_i\ii_{jl}x_k+x_ix_j\ii_{kl}) ]\\
\end{align*}

In general,
\begin{align*}
    \partial_{i_{n+1}} g^{n}_{i_1i_2\ldots i_n}
    =  g^{n}_{i_1i_2\ldots i_n,i_{n+1}}
    &=
    (2n+1)x_{i_{n+1}} r^{2n-1}
    \prod_{e=1}^{n}
    \partial_{i_e}
    r^{-1}
    + 
    r^{2n+1}
    \prod_{e=1}^{n+1}
    \partial_{i_e}
    r^{-1} \\
    &=
    (2n+1)x_{i_{n+1}} r^{2n-1}
    h_{i_1,\ldots,i_{n}}^{(n)}
    + 
    r^{2n+1}
    h_{i_1,\ldots,i_{n},i_{n+1}}^{(n+1)} \\
    &=
    r^{-2}[(2n+1) 
    x_{i_{n+1}} 
    g_{i_1,\ldots,i_{n}}^{(n)}
    + 
    g_{i_1,\ldots,i_{n},i_{n+1}}^{(n+1)}] \\
    &=
    r^{2n-1}[
    (2n+1)x_{i_{n+1}} 
    h_{i_1,\ldots,i_{n}}^{(n)}
    + 
    r^{2}
    h_{i_1,\ldots,i_{n},i_{n+1}}^{(n+1)} ]\\
\end{align*}
Then the properties of the decaying harmonic can be used.
Particularly,
\begin{align*}
    g_{ii\ldots}^{(n)} &= 
        r^{2n+1}
        h_{ii\ldots}^{(n)}
        = 0
    \\
    x_i g_{ijk\ldots }^{(n)} &= 
    r^{2n+1}
    x_i h_{ijk\ldots}^{(n)}
    = - n r^{2n+1} h_{jk\ldots }^{(n-1)}  
    = - n r^2 g_{jk\ldots }^{(n-1)}  \\
    g_{ijk\ldots,i}^{(n)} &= 
    r^{2n-1}[
    (2n+1)x_{i} 
    h_{ijk\ldots}^{(n)}
    + 
    r^{2}
    h_{ijk,\ldots,i}^{(n+1)} ]
    = - n (2n+1)  g_{jk,\ldots}^{(n- 1)} \\
    g_{ij\ldots,k}^{(n)}
    &= r^{-2}[
        (2n+1) 
    x_{k} 
    g_{ij\ldots}^{(n)}
    + 
    g_{ij,\ldots,k}^{(n+1)}] 
\end{align*}

Note that the last term of $g_{ij\ldots}^{(n)}$ will be of opposite sign than the last temr of $g_{ij,\ldots,k}^{(n+1)}$ hence they will exactly cancel each other thus, $(2n+1) x_{k} g_{ij\ldots}^{(n)}+ g_{ij,\ldots,k}^{(n+1)}$ is of the order of $g^{(n-1)}$. 


\section{Stokes equation and general solution}

In dimensionless form the Stokes equation reads, 
\begin{align}
    \div \textbf{u}&= 0\\
    -\grad p + \grad^2 \textbf{u} &= 0 
\end{align}
based on that we postulate that,
\begin{equation}
    \textbf{u} =
    -\frac{1}{2} p \textbf{x} 
    + \textbf{u}^H
\end{equation}
where $\textbf{u}^H$ and $p$ are linear combination of harmonic functions. 


\section{Droplets in pure linear flow}
We consider that, 
\begin{align}
    \div \textbf{u}&= 0\\
    -\grad p + \grad^2 \textbf{u} &= 0 
\end{align}
with, 
\begin{align}
    \textbf{u}\cdot\textbf{n} &= (\textbf{U} - \bm\Gamma \cdot \textbf{n} +\ldots) \cdot \textbf{n}\\
    \textbf{u}^{in} &=  \textbf{u}^{out}\\
    (\bm\sigma - \lambda\bm\sigma)\cdot\textbf{n} &= \gradI \gamma + \gamma \textbf{n} (\div \textbf{n}) \\
    (\bm\delta - \textbf{nn})\cdot  (\bm\sigma - \lambda\bm\sigma)\cdot\textbf{n} &= (\bm\delta - \textbf{nn}) \cdot \grad \gamma
\end{align}

In a first attempt we assume that the solution is proportional to $\Gamma$ hence we state that, 
\begin{align}
    u_i &= \mathcal{U}_{ijk} \Gamma_{jk}  \\
    p &= \mathcal{P}_{jk} \Gamma_{jk}\\
    \sigma_{il} &= \Sigma_{ijkl} \Gamma_{jk}  
\end{align}
Hence our new problem become, 
\begin{align}
    \partial_i \mathcal{U}_{ijk} &= 0\\
    - \partial_i \mathcal{P}_{jk}
    + \partial_l\partial_l \mathcal{U}_{ijk} &= 0
\end{align}
\begin{align}
    \mathcal{U}_{ijk} n_i  &= - n_jn_k \\
    \mathcal{U}_{ijk}^{in} &= \mathcal{U}_{ijk}^{out}\\
    (\delta_{im} - n_in_m) (\Sigma_{ijkl} - \lambda\Sigma_{ijkl})n_l &= (\delta_{im} - n_in_m) (\partial_i \gamma )_{jk}= 0
\end{align}

\subsection{General solution solutions}
Since the pressure and velocity are harmonic functions and Decay at infinity we have, 
\begin{align}
    p^{out}_{jk}
    &=
    C_{0f} \delta_{jk} h^{(0)} 
    + D_{0f} h^{(2)}_{jk}\\
    \mathcal{U}_{ijk}^{out}
    &=
    \frac{C_{0f}}{2} \delta_{jk} x_i   h^{(0)} 
    + 
    C_{1f} \delta_{ij} h^{(1)}_k
    + C_{2f} \delta_{ki} h^{(1)}_j
    + C_{3f} \delta_{jk} h^{(1)}_i
    + \frac{D_{0f}}{2}  x_i h^{(2)}_{jk}
    + D_{1f} h^{(3)}_{ijk}
\end{align}
and, 
\begin{align}
    p^{in}_{jk}
    &=
    C_{0d} \delta_{jk} g^{(0)} 
    + D_{0d} g^{(2)}_{jk}\\
    \mathcal{U}_{ijk}^{in}
    &=
    \frac{C_{0d}}{2} x_i  \delta_{jk} g^{(0)} 
    + 
    C_{1d} \delta_{ij} g^{(1)}_k
    + C_{2d} \delta_{ki} g^{(1)}_j
    + C_{3d} \delta_{jk} g^{(1)}_i
    + \frac{D_{0d}}{2}  x_i g^{(2)}_{jk}
    + D_{1d} g^{(3)}_{ijk}
\end{align}

We may even compute the stress, 
\begin{align}
    \partial_l \mathcal{U}_{ijk}^{out}
    &=
    \frac{C_{0f}}{2} \delta_{jk} (
        \ii_{il}  h^{(0)} 
        + x_i  h^{(1)}_l 
    )
    + 
    C_{1f} \delta_{ij} h^{(2)}_{kl}
    + C_{2f} \delta_{ki} h^{(2)}_{jl}
    + C_{3f} \delta_{jk} h^{(2)}_{il}
    + \frac{D_{0f}}{2}  (\ii_{il} h^{(2)}_{jk} + x_i h^{(3)}_{jkl})
    + D_{1f} h^{(4)}_{ijkl}
\end{align}
The symmetric part of this tensor gives, 
\begin{align}
    \partial_l \mathcal{U}_{ijk}^{out}
    + \partial_i \mathcal{U}_{ljk}^{out}
    &=
    \frac{C_{0f}}{2} \delta_{jk} (
        2\ii_{il}  h^{(0)} 
        + (x_i  h^{(1)}_l +x_l  h^{(1)}_i )
    )\\
    &+ 
    C_{1f} \delta_{ij} h^{(2)}_{kl}
    + C_{1f} \delta_{lj} h^{(2)}_{ki}
    + C_{2f} \delta_{ki} h^{(2)}_{jl}
    + C_{2f} \delta_{kl} h^{(2)}_{ji}
    + 2C_{3f} \delta_{jk} h^{(2)}_{il} \\
    &+ \frac{D_{0f}}{2}  (2\ii_{il} h^{(2)}_{jk} + x_i h^{(3)}_{jkl}+x_l h^{(3)}_{jki})
    +2 D_{1f} h^{(4)}_{ijkl}
\end{align}
we conclude that, 
\begin{align}
    \Sigma_{ijkl}
    &=
    + \frac{C_{0f}}{2} \delta_{jk} (
        + (x_i  h^{(1)}_l +x_l  h^{(1)}_i )
    )\\
    &+ 
    C_{1f} \delta_{ij} h^{(2)}_{kl}
    + C_{1f} \delta_{lj} h^{(2)}_{ki}
    + C_{2f} \delta_{ki} h^{(2)}_{jl}
    + C_{2f} \delta_{kl} h^{(2)}_{ji}
    + 2C_{3f} \delta_{jk} h^{(2)}_{il} \\
    &+ \frac{D_{0f}}{2}  (x_i h^{(3)}_{jkl}+x_l h^{(3)}_{jki})
    +2 D_{1f} h^{(4)}_{ijkl}
\end{align}

\subsection{Divergence of the velocity}
\begin{align}
    \partial_i \mathcal{U}_{ijk}^{out}
    &=
    \frac{C_{0f}}{2} \delta_{jk} ( 3   h^{(0)} + x_i   h^{(1)}_i )
    + C_{1f} \delta_{ij} h^{(2)}_{ki}
    + C_{2f} \delta_{ki} h^{(2)}_{ji}
    + C_{3f} \delta_{jk} h^{(2)}_{ii} \\
    &+ \frac{D_{0f}}{2}  (3 h^{(2)}_{jk} + x_i h^{(3)}_{ijk})
    + D_{1f} h^{(4)}_{ijki} \\
    &=
    C_{0f} \delta_{jk} h^{(0)} 
    + (C_{1f} + C_{2f})h^{(2)}_{kj}
\end{align}
\begin{align}
    \partial_i \mathcal{U}_{ijk}^{in}
    &=
    \frac{C_{0d}}{2} 3  \delta_{jk} g^{(0)} 
    + 
    C_{1d} (3 x_j g^{(1)}_k + g^{(2)}_{jk})r^{-2}
    + C_{2d} (3 x_k g^{(1)}_j + g^{(2)}_{jk}) r^{-2}
    - 3 C_{3d} \delta_{jk} g^{(0)} \\
    &+ 5/2 D_{0d}   g^{(2)}_{jk}
    -21 D_{1d}  g^{(2)}_{jk} \\
    &=
    \frac{C_{0d}}{2} 3  \delta_{jk} g^{(0)} 
    - C_{1d} \delta_{jk}g^{(0)}
    - C_{2d} \delta_{jk}g^{(0)}
    - 3 C_{3d} \delta_{jk} g^{(0)} \\
    &+ 5/2 D_{0d}   g^{(2)}_{jk}
    -21 D_{1d}  g^{(2)}_{jk} \\
\end{align}
Comparing the functional forms we deduce, 
\begin{align*}
    C_{0f} &= 0\\
    C_{1f} + C_{2f}&= 0  \\
    3C_{0d} - 2C_{1d}-2C_{2d} - 6 C_{3d} &=0 \\
    5 D_{0d} - 42 D_{1d} &= 0\\
\end{align*}


Let study the BC $\mathcal{U}_{ijk} n_i = \mathcal{U}_{ijk} x_i = - x_jx_k = - (h^2_{jk}r^2 + \ii_{jk})/3$ it gives,  
\begin{align}
    \mathcal{U}_{ijk} x_i
    =
    \frac{C_{0f}}{2}\delta_{jk} h^{(0)} 
    + 
    C_{1f} x_j h^{(1)}_k
    + C_{2f} x_k h^{(1)}_j
    - C_{3f} \delta_{jk} h^{(0)}
    + \frac{D_{0f}}{2} h^{(2)}_{jk}
    -3  D_{1f} h^{(2)}_{jk}
\end{align}
\begin{align}
    \mathcal{U}_{ijk} x_i
    &=
    \frac{C_{0f}}{2}\delta_{jk} h^{(0)} 
    - C_{1f} (h^{(2)}_{jk} r^2 + \ii_{jk}h^{(0)})/3
    - C_{2f} (h^{(2)}_{jk} r^2 + \ii_{jk}h^{(0)})/3
    - C_{3f} \ii_{jk} h^{(0)}\\
    &+ \frac{D_{0f}}{2} h^{(2)}_{jk}
    -3  D_{1f} h^{(2)}_{jk}
\end{align}
\begin{align}
    C_{0f}/2-C_{1f}/3-C_{2f}/3-C_{3f} = -1/3\\
    -C_{1f}/3-C_{2f}/3 + D_{0f}/2 - 3D_{1f} = -1/3\\
\end{align}

\begin{align*}
    \mathcal{U}_{ijk}^{out}
    - \mathcal{U}_{ijk}^{in}
    =
    - \frac{C_{0f}}{2}  \delta_{jk} h^{(1)}_i r^2
    + 
    C_{1f} \delta_{ij} h^{(1)}_k
    + C_{2f} \delta_{ki} h^{(1)}_j
    + C_{3f} \delta_{jk} h^{(1)}_i\\
    - \frac{D_{0f}}{2}  (h^{(3)}_{ijk}r^2 +(2\ii_{kj}h^{(1)}_i - 3\ii_{ik}h^{(1)}_j - 3\ii_{ij}h^{(1)}_k ))/5
    + D_{1f} h^{(3)}_{ijk}\\
    + \frac{C_{0d}}{2}  \delta_{jk} g^{(1)}_i
    - C_{1d} \delta_{ij} g^{(1)}_k
    - C_{2d} \delta_{ki} g^{(1)}_j
    - C_{3d} \delta_{jk} g^{(1)}_i\\
    + \frac{D_{0d}}{2}  ( (g^{(3)}_{ijk} + r^2 (2\ii_{kj}g^{(1)}_i -3\ii_{ik}g^{(1)}_j - 3\ii_{ij}g^{(1)}_k )  ))/ 5
    - D_{1d} g^{(3)}_{ijk}
\end{align*}
we deduce,
\begin{align*}
    -5C_{0f} + 10C_{3f} - 2D_{0f} + 5C_{0f} - 10 C_{3f} + 2 D_{0d} = 0 \\
    10 C_{1f} + 3 D_{0f} - 10C_{1d} - 3 D_{0d} = 0 \\
    10C_{2f} + 3 D_{0f} - 10C_{2d} - 3 D_{0d} = 0 \\
    -D_{0f}+ 10 D_{1f}+ D_{0d}-10D_{1f} =0 
\end{align*}
Still need two eqs 
\begin{equation}
    (\delta_{im} - n_in_m) (\Sigma_{ijkl} - \lambda\Sigma_{ijkl})n_l = (\delta_{im} - n_in_m) (\partial_i \gamma )_{jk}= 0
\end{equation}
Or even, 
\begin{align*}
    (\delta_{im} - n_in_m) \Sigma_{ijkl}^{out} x_l |_{r=1}
    &=
    + \frac{C_{0f}}{2} \delta_{jk} (
         h^{(1)}_m + h^{(0)} x_m
    )
    + 
    - 2C_{1f}  (h^{(1)}_{k}\delta_{jm} - h^{(1)}_{k}x_jx_m)
    + C_{1f}   (h^{(2)}_{km} x_j  +2 h^{(1)}_{k} x_j x_m) \\
    &- 2C_{2f}  ( h^{(1)}_{j}\delta_{km} - h^{(1)}_{j} n_kn_m)
    + C_{2f}   (h^{(2)}_{jm} x_k + 2 h^{(1)}_{j} x_k x_m)
    - 4C_{3f}  (\delta_{jk} h^{(1)}_{m} + \delta_{jk} h^{(0)} x_m) \\
    &+ \frac{D_{0f}}{2} ( h^{(3)}_{jkm} +3 x_m h^{(2)}_{jk} )
    -8 D_{1f}( h^{(3)}_{mjk}+3  h^{(2)}_{jk} x_m)
\end{align*}



\bibliography{Bib/bib_bulles.bib}
\appendix

\end{document}


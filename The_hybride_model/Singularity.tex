\section{Singularity solution}

In this appendix we consider the calculation of the zeroth and second moment of force density around a spherical droplet translating with a relative velocity \textbf{U} in the carrier fluid and having a viscosity ratio $\lambda$. 
In \citet{kim2013microhydrodynamics,pozrikidis1992boundary} the velocity \textbf{u},$\hat{\textbf{u}}$ and pressure fields $p$,$\hat{p}$ inside and outside (denoted by a hat ) is given in the form, 
\begin{align*}
    u_i(\textbf{r})
    = G_{ij}(\textbf{r}) g_j 
    + D_{ij}(\textbf{r}) d_j 
    &\;\;\; p(\textbf{r})
    = 2 \mu \frac{x_j}{r^3}g_j
    & \sigma_{ij}^0 
    = \mu_1 ( T^G_{ijk} g_j + T^D_{ijk} d_j )\\
    \hat{u}_i(\textbf{r})
    = c_i 
    + E_{ij}(\textbf{r}) e_j 
    &\;\;\;\hat{p}(\textbf{r})
    = 10 \mu r_je_j
    & \hat\sigma_{ij}^0 
    = \mu_2 T^E_{ijk} e_j 
\end{align*}
Where the constant are defined such that,
\begin{align*}
    &\textbf{g} = \frac{1}{4}\left(\frac{3\lambda + 2}{\lambda +1}\right) a \textbf{U}
    &\textbf{d} = -\frac{1}{4}\left(\frac{\lambda}{\lambda +1}\right) a^3 \textbf{U}\\
    &\textbf{c} = \frac{1}{2}\left(\frac{3\lambda + 3}{\lambda +1}\right) \textbf{U}
    &\textbf{e} = -\frac{1}{2}\left(\frac{1}{\lambda +1}\right) \frac{1}{a^2} \textbf{U}\\
\end{align*}
And the functions follow decaying harmonics functions \textbf{G}(\textbf{r}) and \textbf{D}(\textbf{r}) reads, 
\begin{align*}
    G_{ij}(\textbf{r})
    = \frac{\delta_{ij}}{r}
    + \frac{r_ir_j}{r^3}
    && T^G_{ijk}
    = - 6 \frac{r_ir_jr_k}{r^5}\\
    D_{ij}(\textbf{r})
    = -\frac{\delta_{ij}}{r^3}
    + 3 \frac{r_ir_j}{r^5}
    &&
    T^D_{ijk}
    = 6\frac{\delta_{ij}r_k+\delta_{ik}r_j + \delta_{jk}r_i}{r^5}
    -30 \frac{r_ir_jr_k}{r^7}
\end{align*}
Lastly the growing harmonics \textbf{E}(\textbf{r}) reads, 
\begin{align*}
    E_{ij}(\textbf{r})
    = 2 r^2 \delta_{ij}
    - r_ir_j. 
    &&
    T^E_{ijk}(\textbf{r})
    = 3(-4\delta_{ik} r_j + \delta_{ij}x_k + \delta_{kj}x_i )
\end{align*}


The aim is now to compute the integrals of the form, 
\begin{align*}    
    \pMSavg{\textbf{r}\bm{\sigma}_1^0\cdot \textbf{n}_2}
    - \pMOavg{\mu_1 \textbf{e}_2 },
\end{align*}
in the case of a relative linear flow fields. 
Thus, one must compute the integrals, 
\begin{align*}
    \pSavg{\bm\sigma_1^0\cdot \textbf{n}_2}
    &&
    \pSavg{\textbf{rr}\bm\sigma_1^0\cdot \textbf{n}_2 - \mu_1 \textbf{r}\textbf{e}_2^0}
\end{align*}
Since the first order moment is identically zero. 
To do so we first recall that, 
\begin{equation}
    \bm\sigma_k^0
    = 
    - p_k \textbf{I}
    + \mu_1 \left(
        \frac{\partial u_{k,i}^0}{\partial x_j}
        + \frac{\partial u_{k,j}^0}{\partial x_i}
    \right)
\end{equation}
The exterior stress filed

\subsubsection*{The drag force}

 \section{Lagrangian equation for a single particle}
\label{ap:particles_eq}
In this appendix we provide more details regarding the derivation of the mass, momentum and energy equation of a single particle. 

The volume equation of a single particle follow the notation \ref{eq:dt_q_alpha} and stay relatively straightforward to derive.
What is less evident thought is the derivation of the surface equations, or boundary conditions, integrated on the surface of a single particle. 
They are derived following the procedure of \ref{eq:dt_q_I_alpha} exposed in \ref{sec:Lagrangian}. 
Integrating the momentum, kinetic energy and internal energy jump conditions, i.e.  \ref{eq:surface_tension}, \ref{eq:dt_rhoI_uI3} and  \ref{eq:dt_rhoIe_I}  gives, 
\begin{align}
    \label{eq:int_u_I}
    \intS{\Jump{\bm{\sigma}_k^0} }
    &=
    \intS{\divI\bm\sigma^0_{I||}} = 0\\
    \label{eq:int_u_I2}
    \intS{\Jump{\textbf{u}_k^0 \cdot \bm{\sigma}_k^0}},
    &=
    -\intS{\gamma\textbf{n}\cdot \textbf{u}_{I}^0(\div \textbf{n})}
    = \gamma \ddt s_\alpha,\\
    \Jump{\textbf{q}_k^0}
    &=
    0,
\end{align}
respectively. 
The second equality of \ref{eq:int_u_I} is demonstrated using \ref{eq:gauss_surface}. 
The second equality of \ref{eq:int_u_I2} is derived using both \ref{eq:gauss_surface} and \ref{eq:reynolds_transport}. 
Consequently, on a closed surface the surface tension plays no role in the momentum jump condition.
Additionally, the work of the surface tension force is equal to $\gamma$ times the rate of change of the area of a given particle. 

Using these equations as boundary condition we easily derive the  mass, momentum and energy equations based on \ref{eq:dt_q_alpha_tot}   for a single particle, 
\begin{align}
    \label{eq:dt_m_alpha}
    \ddt m_\alpha
    &= 
    0\\
    \label{eq:dt_p_alpha}
    \ddt (m_\alpha \textbf{u}_\alpha)
    &= 
    m_\alpha\textbf{g}
    +  \intS{\bm{\sigma}_1^0 \cdot \textbf{n}_2}\\
    \label{eq:dt_E_alpha_tot}
    \ddt (m_\alpha E_\alpha^\text{tot})
    &= 
    m_\alpha \textbf{u}_\alpha \cdot \textbf{g}
    +\textbf{u}_\alpha \cdot \intS{\bm{\sigma}_1^0 \cdot \textbf{n}_2}
    +\intS{\textbf{w}_1^0 \cdot \bm{\sigma}_1^0 \cdot  \textbf{n}_2} 
    - \intS{\textbf{q}_1^0 \cdot \textbf{n}_2}
\end{align}
where  $\intS{  \bm{\sigma}_1^0 \cdot \textbf{n}_2 }$ is the resultants of the hydrodynamic force of the continuous phase, $\intS{\textbf{w}_1^0 \cdot \bm{\sigma}_1^0 \cdot  \textbf{n}_2} $ is the resultant of the work of the continuous phase stresses with the surface velocity, and $\intS{ \textbf{q}_1^0 \cdot \textbf{n}_2 }$ is the resultants of the surface heat flux. 
The second term on the right hands side of the energy equation is the work produced by the mean force and the translational motion of the droplets, while $\intS{\textbf{w}_1^0 \cdot \bm{\sigma}_1^0 \cdot  \textbf{n}_2}$ is the work produced by the local forces and local motion of the fluid at the surface of the particle.
Since we integrated the energy over the particle's volume and its surface, we explicitly made appear the surface energy $\gamma s_\alpha$ within the derivative operator using \ref{eq:int_u_I2}. 
Note that these equations does not explicitly account for inter-particle interactions. 
However, not that the surface external stress flux $\bm{\sigma}_1^0$ contains inter-particle forces mediated through the carrier fluid such as lubrication forces and long range inter-particle forces.

In the spirit of the energy decomposition exposed in \ref{eq:E_alpha_def} the total energy equation \ref{eq:dt_E_alpha_tot} can be split into three secondary equations. 
One for the center of mass kinetic energy $u_\alpha^2/2$, another for the particle internal kinetic energy $W_\alpha$, and a last one for the particle internal energy $e_\alpha$, it reads
\begin{align}
    \label{eq:dt_u2_alpha}
    \frac{1}{2}\ddt (m_\alpha u_\alpha^2)
    &= 
    \textbf{u}_\alpha\cdot
    \textbf{g}m_\alpha
    + 
    \textbf{u}_\alpha\cdot
    \intS{\bm\sigma_1^0 \cdot \textbf{n}_2},\\
    \label{eq:dt_w2_alpha}
    \ddt (W_\alpha + \gamma s_\alpha)
    &= 
    - \intO{ \bm{\sigma}_2^0 : \grad\textbf{u}_2^0 }
    + \intS {\textbf{w}_1^0 \cdot \bm{\sigma}_1^0 \cdot \textbf{n}_2 }
    \\
     \label{eq:dt_e_alpha}
    \ddt (m_\alpha e_\alpha )
    &= 
     \intO{ \bm{\sigma}_2^0 : \grad\textbf{u}_2^0  }
    -  \intS{\textbf{q}_1^0\cdot \textbf{n}_2 } 
\end{align}
respectively. 
The first equation is obtained by taking the dot product of \ref{eq:dt_p_alpha} with $\textbf{u}_\alpha$. 
The third equation is directly obtained using the local conservation of $e_k^0$ (\ref{eq:dt_rhoe_k}) and setting $f_2^0 = e_2^0$ in \ref{eq:dt_q_alpha_tot}.
The last equation is obtained by subtracting the first and third equations to \ref{eq:dt_E_alpha_tot}. 
Note that under this form the surface energy plays the role as a spring in the energy balance.   
From this set of equation we can easily see that the rate of dissipation terms $\intS{\bm{\sigma}_2^0 : \grad\textbf{u}_2^0}$ represent an energy sink in the equation of $W_\alpha$ while it is a source term in the internal energy equation. 
As it has been observed in the previous section, this terms convert the energy of internal motion to molecular agitation. 
However, the interplay between the center of mass  kinetic energy and the internal fluctuation is not obvious, indeed \ref{eq:dt_u2_alpha} shears no common term with the heat and internal kinetic energy equation.
In fact, we will see that the transfer between these scales is archived thought the fluid phase pseudo turbulent energy. 


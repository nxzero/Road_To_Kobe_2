 \section{Lagrangian equation for a single particle}
\label{ap:particles_eq}
\tb{MAYBE EXPLICITE A BIT MORE THE BOUNDARY FOR ENERRGY}
Following the same assumption as in \ref{sec:local_eq}, i.e. we consider no mass transfer and weightless interfaces, the Lagrangian  mass, momentum and energy equations for a single particle can be derived using the generic form \ref{eq:dt_q_alpha_tot} and reads as, 
\begin{align}
    \label{eq:dt_m_alpha}
    \ddt m_\alpha
    &= 
    0\\
    \label{eq:dt_p_alpha}
    \ddt (m_\alpha \textbf{u}_\alpha)
    &= 
    m_\alpha\textbf{g}
    +  \intS{\bm{\sigma}_1^0 \cdot \textbf{n}_2}\\
    \label{eq:dt_E_alpha_tot}
    \ddt (m_\alpha E_\alpha^\text{tot})
    &= 
    m_\alpha \textbf{u}_\alpha \cdot \textbf{g}
    +\textbf{u}_\alpha \cdot \intS{\bm{\sigma}_1^0 \cdot \textbf{n}_2}
    +\intS{\textbf{w}_1^0 \cdot \bm{\sigma}_1^0 \cdot  \textbf{n}_2} 
    - \intS{\textbf{q}_1^0 \cdot \textbf{n}_2}
\end{align}
where  $\intS{  \bm{\sigma}_1^0 \cdot \textbf{n}_2 }$ is the resultants of the hydrodynamic force and $\intS{ \textbf{q}_1^0 \cdot \textbf{n}_2 }$ is the resultants of the surface heat flux. 
The second term on the right hands side of the energy equation is the work produced by the mean force and the translational motion of the droplets, while $\intS{\textbf{w}_1^0 \cdot \bm{\sigma}_1^0 \cdot  \textbf{n}_2}$ is the work produced by the local forces and local motion of the fluid at the surface of the particle.
Since we integrated the energy over the particle's volume and its surface, we explicitly made appear the surface energy $\gamma s_\alpha$ within the derivative operator. 
Note that these equations does not explicitly account for inter-particle interactions. 
However, it is possible to include manually such forces by noticing that the surface external stress flux $\bm{\sigma}_1^0$ is the sum of hydrodynamic and particles-particles interaction forces, regardless it is pure contact forces from direct contact or a force mediated through the carrier fluid.
From this consideration it is possible to split every term involving the stress $\bm{\sigma}_1^0$ into two terms representing these contributions. 
Same comments can be made for the heat flux $\textbf{q}_1^0$. 
Although this distinction is important, for purpose of clearly we will stay general, and we will keep the fluxes $\bm{\sigma}_1^0$ and $\textbf{q}_1^0$ as such. 

In the spirit of the energy decomposition exposed in \ref{eq:E_alpha_def} the total energy equation can be split into three equations, one for the center of mass kinetic energy, internal motion and internal kinetic energy, namely,  
\begin{align}
    \label{eq:dt_u2_alpha}
    \frac{1}{2}\ddt (m_\alpha u_\alpha^2)
    &= 
    \textbf{u}_\alpha\cdot
    \textbf{g}m_\alpha
    + 
    \textbf{u}_\alpha\cdot
    \intS{\bm\sigma_1^0 \cdot \textbf{n}_2},\\
    \label{eq:dt_w2_alpha}
    \ddt (W_\alpha + \gamma s_\alpha)
    &= 
    \intS {\textbf{w}_1^0 \cdot \bm{\sigma}_1^0 \cdot \textbf{n}_2 }
    - \intO{ \bm{\sigma}_2^0 : \grad\textbf{u}_2^0 }
    \\
     \label{eq:dt_e_alpha}
    \ddt (m_\alpha e_\alpha )
    &= 
     \intO{ \bm{\sigma}_2^0 : \grad\textbf{u}_2^0  }
    -  \intS{\textbf{q}_1^0\cdot \textbf{n}_2 } 
\end{align}
respectively. 
The first equation is obtained by taking the dot product of \ref{eq:dt_p_alpha} with $\textbf{u}_\alpha$. 
The third equation is directly obtained using the local conservation of $e_k^0$ (\ref{eq:dt_rhoe_k}) and setting $f_2^0 = e_2^0$ in \ref{eq:dt_q_alpha_tot}.
The last equation is then obtained by subtracting the first and third equations to \ref{eq:dt_E_alpha_tot}. 
Note that in \citet{eq:dt_w2_alpha} the use of \ref{eq:dt_rhoI_uI2} makes appear explicitly the derivative of the surface energy $s_\alpha \gamma$. 
Note that under this form we see that the energy loss in the deformation represented by $W_p$ will be gathered in the surface energy which will in turn act as a source term in the internal kinetic energy motion.
The surface tension plays the role as a spring in the energy balance.   
From this set of equation we can easily see that the rate of dissipation terms $\intS{\bm{\sigma}_2^0 : \grad\textbf{u}_2^0}$ represent an energy sink in the equation of $W_\alpha$ while it is a source term in the internal energy equation. 
As it has been observed in the previous section, this terms convert the energy of internal motion to molecular agitation. 
However, the interplay between the center of mass  kinetic energy and the internal fluctuation is not obvious and has no common term with the heat and internal kinetic energy equation.
In fact, we will see that the transfer between these scales is archived thought the fluid phase pseudo turbulent energy. 


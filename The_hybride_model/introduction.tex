
\section{Introduction}

%Dispersed multiphase flows are ubiquitous in chemical engineering processes. 
%Examples include gas-solid flows in fluidized bed reactors, liquid-liquid flows in liquid-liquid extractors, and bubbly flows in flotation processes. 
%Developing a versatile model capable of capturing the various nature of the dispersed multiphase flow is therefore essential.
%In this work we aim to bridge the existing gaps in current models by providing a unified and adaptable framework for any kind of dispersed multiphase flow problems including interfaccial transport equation.

Dispersed multiphase flows are encountered across a wide range of chemical engineering applications. 
These include gas-solid interactions in fluidized bed reactors , liquid-liquid flows in extractors, and gas-liquid bubbly flows in flotation processes. 
These systems exhibit a wide range of scales, from the size of individual inclusions (as small as a few micrometers) to the size of the reactor (often exceeding one meter), making fully resolved simulations computationally impractical. 
As a result, the current engineering practice relies on averaged equations of motion for both the dispersed and continuous phases. 
Therefore, developing a robust set of averaged equations that accurately captures the complexities of dispersed multiphase flows is essential. 
In this study, we aim to overcome the limitations of existing models by proposing a comprehensive set of averaged equations that can be applied to various types of dispersed multiphase flows, including interfacial transport processes.

%Due to the two many scales encountered in this process ranging from the particle size (from let'say 10 micrometer) to the reactor size (larger thant one meter) it appears to be unfeasible to model the process into play using fully resolved simulations.
%The current engineering practice is to relay on averaged set of equation of motion for both the dipsersed and continuous phase equation.
%Therefore, it is essential to develop a rigorous set of averaged equations that can accurately represent the diverse behaviors of dispersed multiphase flows. 
%In this study, our goal is to address the limitations of existing models by proposing a comprehensive averaged set of equations that can be applied to various kind of dispersed multiphase flow, including interfacial transport equations. %framework 

%Most of the averaged two-fluid models relies on the framework proposed by \citep{drew1983mathematical} where both dispersed and continuous phase are treated in a similar fashion (see \citet{hu2021cfd} for instance).
%Despite its very versatility (this framarwork may be used for non dispersed flows such as stratified or slug flow), the phisical meaning of the closure terms are difficult to obtain \citep{drew1983mathematical} and the mathematical well posdeness of the whole system is sill a matter of debate \citep{panicker2018hyperbolicity,lhuillier2013}.
%Another treatment of the dispersed phase is possible and by treating properly the dispersed phase in a lagrangian framework.
%This framework takes it root from the kinetic theory of gases in which the motion of the particles is described by the single-particle distribution function which obeys Botlzmann equation \citep{chapman1990mathematical}. 
%This approach have been very succesfull in predicting the motion of dry granlaur material \citep{rao2008introduction} or gas-solid flows \citep{simonin1996}.
%The main problematic of this framework is that the fluid phase and the dispersed phase are treated using two disting formalism: phase averaged for the fluid phase and lagrangian averaging for the particle average.
%In a pioneering study \citet{buyevich1979flow} demonstrated that both averaging formalism can be linked, by performing a Taylor expansion of the closure terms around the center of mass of the particle.
%Then, this framework was revisited by \citet{lhuillier1992} and \citet{zhang1994averaged,zhang1994ensemble}. 
%The latter provided closures in the inviscid flow limit for the mass and momentum transport equation.
%Numerous example of averaged dispersed two-phase flows model have been proposed this last two decades.
%After that \citet{jackson1997locally,zhang1997momentum} derive the averaged conservation equations for a solid spherical particle suspension.
%They provided the closure in the stokes flow limits.  
%As started by \citep{zhang1994averaged,zhang1994ensemble} where they study spherical bubbles suspension.
 
The majority of averaged two-fluid models are based on the framework proposed by \citet{drew1983mathematical}, where both the dispersed and continuous phases are treated similarly (see, for example, \citet{hu2021cfd}). 
Despite its versatility-allowing applications beyond dispersed flows, such as stratified or slug flow-the physical interpretation of the closure terms remains difficult \citep{drew1983mathematical}, and the mathematical well-posedness of the entire system is still a matter of debate \citep{panicker2018hyperbolicity, lhuillier2013}.
An alternative approach treats the dispersed phase using a Lagrangian framework. 
This approach originates from kinetic theory of gases, where the motion of particles is described by a single-particle distribution function governed by the Boltzmann equation \citep{chapman1990mathematical}. 
It has proven particularly successful in predicting the dynamics of dry granular materials \citep{rao2008introduction} and gas-solid flows \citep{simonin1996}. 
However, the fluid phase and the dispersed phase are handled through two distinct formalisms: phase averaging for the fluid and Lagrangian averaging for the dispersed phase.
In a pioneering study, \citet{buyevich1979flow} demonstrated that these two averaging methods could be connected by performing a Taylor expansion of the closure terms around the particle’s center of mass. 
This "hybrid" formalism was later revisited by \citet{lhuillier1992ensemble}, and by \citet{zhang1994averaged, zhang1994ensemble}. The latter provided closures for the momentum transport equations in the inviscid flow limit. 
Subsequently, \citet{jackson1997locally} derived averaged conservation equations for a suspension of spherical solid particles, providing closures for the Stokes flow regime.

%Most of the previous authors working on the "hybrid" formalism considered solid particles \citet{buyevich1979flow,jackson1997locally}. 
%A notable exception is the work of \citet{zhang1994ensemble} who considered spherical bubble with time dependent radii and \citet{zhang1997momentum} who considered spherical drops.  %They all derived Lagrangian based equations for the dispersed phase considering point of mass spherical particles.
%But then how to account for different inclusion shapes, fluid internal motion and transport equations on the surface within these Lagrangian models ?
%Indeed, to the best of our knowledge, the surface properties such as surface tension and chemical concentration of surfactant over the surface, have not been taken into account into such averaged hybrid model. 
%While it is of major importance to predict the hydrodynamics bubbly flows.
%Although the lagrangian form of surface transport equations is well known \citet{lhuillier2000bilan} is application to define a general lagrangian quantities remains unknown.
%Therefore, a whole in one, hybrid model that encapsulate all the physics, meaning, surface properties, particles of arbitrary shape, higher moments equations is needed. 

Many previous studies that used the "hybrid" formalism have focused on solid particles \citep{buyevich1979flow,jackson1997locally}. 
Notable exceptions include \citet{zhang1994ensemble}, who investigated spherical bubbles with time-varying radii, and \citet{zhang1997momentum}, who examined spherical droplets. 
Despite these advancements, the question remains: how can different inclusion shapes, internal fluid motion, and surface transport equations be incorporated into such Lagrangian models? 
To the best of our knowledge, important surface properties-such as surface tension and the distribution of surfactants-have yet to be integrated into these averaged hybrid models. 
While the Lagrangian formulation of interfacial area equation is well established \citep{lhuillier2000bilan}, its application to define general Lagrangian quantities for a dispersed phase is still not fully explored. 
Thus, there is a need for a comprehensive hybrid model that encompasses all relevant physical aspects, including surface properties, arbitrarily shaped particles.%, and higher-order moment equations.


%Even though these hybrid models for fluid particles were already quite sophisticated it still misses some major points. 

% Additionally, a proper and general derivation of these models must be established in the most general case possible, i.e. for any kind particles and conservation laws. 



%Some author tried to model fluid particle within a Lagrangian approach and already answered part of these questions.  
%Among them \citet{lhuillier2000bilan} and \citet{morel2015mathematical,zaepffel2012multisize} applied the Lagrangian framework equations for fluid particle with mass transfer. 
%These model consist in establishing the conservation equation of an integrated Eulerian quantity $f$, within the volume of a particle denoted by $\Omega_\alpha$, namely  $\int_{\Omega_\alpha} f d\Omega$. 
% Likewise, the so-called first moments of a quantity $f$ is defined by $\int_{\Omega_\alpha} \textbf{r} f_k d\Omega$ with \textbf{r} the position from the center of mass to any point in the volume of the particle. 
% These integrals can be derived within time and yield the moments conservation equitation for any Lagrangian moment, , of the particle $\alpha$. 

Several authors have addressed the question of equivalence between Lagrangian (or particle-averaged) and phase-averaged equations in various contexts \citep{zhang1997momentum,lhuillier2000bilan,nott2011suspension}. 
In \citet[Appendix A]{zhang1997momentum}, the authors demonstrated that these two frameworks are equivalent for spherical inclusions when only considering the firt order moments. 
Later, \citet[Appendix A]{nott2011suspension} extended the proof of \citet{zhang1997momentum}, showing that the Lagrangian and phase-averaged equations are strictly equivalent for suspensions of solid spherical particles by considering an infinite number of higher-order terms. 
In addition, \citet{lhuillier2010multiphase} argued that the phase-averaged equations applied to the dispersed phase are, in fact, a Taylor series expansion of the particle-averaged moment equations.
Similarly, in the context of interfacial area balance equations, \citet{lhuillier2000bilan} reached a comparable conclusion when comparing the phase-averaged and particle-averaged area density equations for spherical particles. 
These studies focus on monodisperse suspensions of spherical particles and all arrive at the same conclusion: the particle-averaged equation is rigorously equivalent to the phase-averaged equation for the dispersed phase.
In this work, we extend these results by providing a general proof of the equivalence between the phase-averaged and particle-averaged frameworks for inclusions of arbitrary shapes, including interfacial transport effects.




%Another question that has been addressed by several authors \citep{zhang1997momentum,lhuillier2000bilan,nott2011suspension} is the one of the equivalence between Lagrangian or particle-averaged and phase-averaged equations. 
%In \citet[Appendix A]{zhang1997momentum} they provided a demonstration that both formalism are equivalent at first order of tthe Taylor expansion for spherical inclusions.%for the momentum equation of tspherical . 
%In a different context of nterfacial area balance equation \cite{lhuillier2000bilan}  reached a similar conclusion when comparing the phase-averaged area density and particle-averaged area density equations for spherical particles. 
%Next, \citet[Appendix A]{nott2011suspension} provided the proof that  Lagrangian or particle-averaged and phase-averaged equations were strictly equivalent for suspension of solid spherical particles even considering an infinite number of higher order terms. 
%In \citet{lhuillier2010multiphase} they state that the phase-averaged equation applied on the dispersed phase is in fact a Taylor series expansion of the particular-averaged moments equations. 
%In all these studies they focus  mono-disperse spherical particle suspension and all reached the same conclusion, i.e. the particular-averaged linear momentum equation is rigorously equivalent to the phase-averaged momentum equation for the dispersed phase. 
%Here, we provide a general proof of the equivalence between phase-averaged framework and particle-averaged framework for an arbitrary shaped inclusion with interfacal tarnsport. 
%\citet[Appendix A]{zhang1997momentum} provided evidences that the particle-averaged momentum equation is as legitimate as the phase-averaged momentum equation, which is consistent with \ref{eq:proof2}. 
%Additionally, \citet[Appendix A]{nott2011suspension} derived a similar expression than \ref{eq:proof2}, also in the case of the averaged momentum equation for suspension of solid spherical particles.
%This demonstration has been derived for the area density concentration of spherical particles. 
%however, they reach a compatible but different conclusion. 

% Finally, to gives a more specific insight of the use of such a model in the practical cases, we take the example the contaminated rising bubbly flow. 
% In particular, we demonstrate how to derive a model to predict the distribution of surfactant over the interface of bubbles, as it is of major importance to predict mass transfer and drag force correlation \citep{kentheswaran2022direct}.
% Additionally, we also show how to derive the orientation tensor equation in fibrous media. 

%In this article we derive the a general hybrid model for disperse two phase flow with interfacial transport. 
%We start in \ref{sec:two-fluid} to expose the two-fluid formulation and the distributional form of the interfcaial transport equation. 
%Then, in \ref{sec:Lagrangian} we propose a Lagrangian-based model to describe each inclusion fluid particles with arbitrary shapes and surface properties.
%In \ref{sec:averaged_eq} we provide a demonstration that the particle-averaged equation is rigorously equivalent to the phase-averaged equation for any kind of dispersed multiphase flow and conservation law. 
%From this demonstration we present the hybrid set of conservation equations consisting of one equation for the fluid phase and an arbitrary number of equations for the dispersed phase repereseting the conservation of moments.
%The findings obtained in the course of the investigation are discussed in \ref{sec:conclusion}.
In \ref{sec:two-fluid}, we begin by presenting the two-fluid formulation and the distributional form of the interfacial transport equation. 
Next, in \ref{sec:Lagrangian}, we introduce a Lagrangian-based model that describes individual fluid particles with arbitrary shapes and surface properties. 
In \ref{sec:averaged_eq}, we demonstrate that the particle-averaged equation is rigorously equivalent to the phase-averaged equation for any type of dispersed inclusions. 
Building on this demonstration, we introduce a hybrid set of conservation equations, consisting of one equation for the fluid phase and multiple equations for the dispersed phase, representing the conservation of moments. 
Finally, we discuss the findings of this investigation in \ref{sec:conclusion}.
%based on the local conservation laws presented 
%Then we present the Lagrangian balance on an arbitrary particle.  
%conclude that the particle-averaged equations are sufficient to describe every aspect of the flow since they compose the phase-averaged equation.%, which confirm the affirmation of \citet[Appendix A]{zhang1997momentum}. 
%With the use of this general framework we are now able to identify in which circumstance the specific properties of the particles such as, internal velocities, shape, and surface tension forces, come into play in the averaged model. 
%The organization of this manuscript is as follows. 
%It is then demonstrated how to perform an average onto these equations to obtain the classical two-fluid formulation of two phase flows. 
%By deriving these time varying integrated properties along with applying an averaging procedure on these equations, we are able to derive a Lagrangian model for fluid particles. 

\JL{TO DO.
\begin{itemize}
    \item faire une derniere passe en faisant attention aux notations (par exemple $\text q_\alpha$ qui ne doit pas etre en italique), à l'anglais, aux coquilles de tout type
    \item je n'ai pas relu ni l'annexe C ni l'annexe D. Mais dans ces dernieres il faudrait specifier ce que veut dire le signe $[]$ ou le remplacer par une notation plus usuelle.
\end{itemize}

}



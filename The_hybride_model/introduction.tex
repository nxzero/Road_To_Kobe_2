\tb{point out the non equivalence when there is mass transfer or so }

\tb{give a rigourous expression for the first moment }

\JL{Je ne trouve pas le titre tres parlant : the hybrid model parle peu pour des non specialistes et qu'entends tu par surface properties ? j'ai modifie, mais cela pourra sans doute changer encore.}

\section{Introduction}

Dispersed multiphase flows are ubiquitous in numerous engineering and scientific domains. 
Examples include gas-solid flows in fluidized bed reactors, oil-water emulsions in petroleum industry processes, and fiber suspension in composite manufacturing. 
Developing a versatile model capable of capturing the various nature of the dispersed multiphase flow is therefore essential.
In this work we aim to bridge the existing gaps in current models by providing a unified and adaptable framework for any kind of dispersed multiphase flow problems.

Numerous example of dispersed two-phase flows model have been proposed this last two decades.
As started by \citep{zhang1994averaged,zhang1994ensemble} where they study spherical bubbles suspension.
They provided closures in the inviscid flow limit for the mass and momentum transport equation. 
After that \citet{jackson1997locally,zhang1997momentum} derive the averaged conservation equations for a solid spherical particle suspension.
They provided the closure in the stokes flow limits.  
They all derived Lagrangian based equations for the dispersed phase considering point of mass spherical particles.
But then how to account for different particle's shapes or fluid internal motion within these Lagrangian models ?
Some author tried to model fluid particle within a Lagrangian approach and already answered part of these questions.  
Among them \citet{lhuillier2000bilan} and \citet{morel2015mathematical,zaepffel2012multisize} applied the Lagrangian framework equations for fluid particle with mass transfer. 
These model consist in establishing the conservation equation of an integrated Eulerian quantity $f$, within the volume of a particle denoted by $\Omega_\alpha$, namely  $\int_{\Omega_\alpha} f d\Omega$. 
% Likewise, the so-called first moments of a quantity $f$ is defined by $\int_{\Omega_\alpha} \textbf{r} f_k d\Omega$ with \textbf{r} the position from the center of mass to any point in the volume of the particle. 
% These integrals can be derived within time and yield the moments conservation equitation for any Lagrangian moment, , of the particle $\alpha$. 
By deriving these time varying integrated properties along with applying an averaging procedure on these equations, we are able to derive a Lagrangian model for fluid particles. 

Even though these hybrid models for fluid particles were already quite sophisticated it still misses some major points. 
Indeed, to the best of our knowledge, the surface properties such as surface tension and chemical concentration of surfactant over the surface, have not been taken into account into such averaged hybrid model. 
While it is of major importance to predict the hydrodynamics bubbly flows. 
% Additionally, a proper and general derivation of these models must be established in the most general case possible, i.e. for any kind particles and conservation laws. 
Therefore, a whole in one, hybrid model that encapsulate all the physics, meaning, surface properties, particles of arbitrary shape, higher moments equations is needed. 


Another question that has been addressed by several authors \citep{nott2011suspension,zhang1997momentum} is the one of the equivalence between particle-averaged and continuous-averaged equations. 
In \citet[Appendix A]{zhang1997momentum} they provided a demonstration that both formalism are equivalent at first order. 
Afterward \citet[Appendix A]{nott2011suspension} provided the proof that the formalism were strictly equivalent even considering an infinite number of higher order terms. 
In \citet{lhuillier2010multiphase} however, they reach a compatible but different conclusion. 
Indeed, they state that the phase-averaged equation applied on the dispersed phase is in fact a series expansion of the particular-averaged moments equations. 
This demonstration has been derived for the area density concentration of spherical particles. 
In all these studies they focus on mono-disperse spherical particle suspension. 
And they all reached the same conclusion, i.e. the particular-averaged linear momentum equation is rigorously equivalent to the phase-averaged momentum equation for the dispersed phase. 
We thus need to provide a clear proof of the equivalence between phase-averaged framework and particle-averaged framework in the most general case and for a general conservation equation. 


Based on the local conservation laws presented \ref{sec:two-fluid} and the Lagrangian balance on an arbitrary particle, presented \ref{sec:Lagrangian}, we derive the generalized hybrid model for disperse two phase flow. 
Then we provide a demonstration which proves that the argument of \citet{lhuillier2000bilan} on the equivalence is generalizable to any kind of dispersed multiphase flow and conservation law. 
From this demonstration we conclude that the particle-averaged equations are sufficient to describe every aspect of the flow since they compose the phase-averaged equation, which confirm the affirmation of \citet[Appendix A]{zhang1997momentum}. 
With the use of this general framework we are now able to identify in which circumstance the specific properties of the particles such as, internal velocities, shape, and surface tension forces, come into play in the averaged model. 
Finally, to gives a more specific insight of the use of such a model in the practical cases, we take the example the contaminated rising bubbly flow. 
In particular, we demonstrate how to derive a model to predict the distribution of surfactant over the interface of bubbles, as it is of major importance to predict mass transfer and drag force correlation \citep{kentheswaran2022direct}.
Additionally, we also show how to derive the orientation tensor equation in fibrous media. 



\JL{

Dans l'ensemble cela peut faire un bon papier car je pense qu'il y a quelques nouveautes par rapport a la litterature. Mais il va vraiment falloir marcher sur des oeufs dans la presentation de tout cela car beaucoup de resultats sont deja connu. Quelques comentaires generaux
\begin{itemize}
\item comme tu me l'avais dit ce n'est qu'un premier jet. Il n'empeche qu'il y a un enorme travail a faire sur la forme. Que ce soit l'anglais, mais aussi la maniere dont tu structures ta pensee. Meme pour un premier jet c'est necessaire pour que le lecteur soit en mesure de tout comprendre a ton propos est que l'objectif du travail soit claire ... clairement cela ne letait pas a ma premiere lecture et j'ai du m'arracher les cheveux sur certains passages. C'est l'axe sur lequel tu dois le plus progresser d'ici la fin de la thèse pour que tu deviennes plus autonomne la dessus.
\item un autre point qui me semble tres important à travailler. il y a des choses dans ce papier que tu maitrises bien, voir tres bien (derivation des equations moyennees, equivalence entre equations,...), d'autres que tu maitrises moins bien (interpretation physique des equations), et d'autres tres peu (interpretation physique des forces hydros, fermeture des equations pour des particules axi, ...). C'est tout a fait normal, c'est un sujet complique et l'interpretation physique des resultats demande de la maturite qui s'acquiere petit a petit. Ce qui m'emebte par contre c'est que tu presentes les resultats que tu maitrises et ce que tu maitrises moins bien de la meme maniere. Or dans un article scientifique il faut etre sur a 200\% de ce que l'on affirme. La moindre erreur va remettre en cause ta maitrise du sujet et te decredibilisera totalement. Ce que je t'invite a faire pour les prochaines fois, c'est de mettre un code couleur: noir quand tu es sur de toi, et bleu par exemple pour les parties a travailler pour lesquels tu penses ne pas tout maitrises. Cette auto-critique c'est egalement essentiel pour un scientifique. 
\item comme je te le disais c'est un papier interessant mais ou il faut etre extremement prudent (et modeste) quand à ce qu'il apporte par rapport a la litterature existante. Un grand nombre de resultats ont deja ete derives. Le principal interet je trouve est de proposer un cadre commun à toutes les etudes precedentes, ce qui en soit merite d'etre publie. Pour cela il faudra bien expliquer en quoi le transport de surface sert. 
\item Je pense que tu maitrises bien les papiers historiques de lequipe de Morel $\&$ Lhuillier. Par contre clairement il faut que tu lises ou relises les papiers suivants : 
\begin{itemize}
\item tous les papiers de Zhang et Prosperetti (1994a,1994b,1997). En particulier celui de 1997 qui contient beaucoup de matiere en commun avec ce que l'on discute. Sais tu egalement que l'analogie avec l'equqation de Rayleigh-PLesset est deja faite dans Zhang et Prosperetti (1994b) ?
\tb{
    \begin{itemize}
        \item \citet{zhang1997momentum} : Il parle effectivement du fait que l'on peut décrire la phase dispersé avec un nombre arbitraire de moments. aussi il démontre l'équivalence de la même manière que Nott masi uniquement a l'ordre 1. 
        \item Je pense que ce papier peut servir d'introduction dans le sens ou il pose les même question que nous (sur les moment d'order sup) mais n'y repond pas entièrement, ce que l'on fait. 
    \end{itemize}
}
    \item j'ai trouve un papier : "A Note on the Net Force and Moment on a Drop Due to Surface Forces" de Hesla et al. 1993, qui je pense discute d'une partie de l'influence de la tension de surface sur les equations des moments. Ce serait bien de verifier si on n'obtient des resultats analogues
    
\begin{itemize}
    \item 
    \tb{Dans ce papie il demontre que la force de tension de surface n'agit pas sur la quantité de mouvement et le torque même en présence de force de marangonie comme nous donc. 
    Mais nous avons une information en plus, c'est la partie symmetrique de l'équation du premier moments de al quantitée de mouvement, et la on voit l'impact des forces de tension de surface contrairement a lui. }
\end{itemize}
    \JL{\item concernant le tranport de $<pipj>$ le papier le plus claire il me semble est celui de Wang $\&$ Tucker 2008.}
    \tb{
        \item Oui c'est très clair, la difference avec nous c'est que l'on ne va pas plus loin que l'expression $\dot{\textbf{p}} = \omega \times \textbf{p}$ qui est trivial.
        Alors que il est possible exprimer $ \omega\times\textbf{p} $ en fonction des charatériqtique de l'ecoulements grace aux loi de faxen, ce qui donne l'equation de jeffrey dans le cas des fiberes/stokes, qui est une equation fermé de la cinématique 
        \item  dans notre approche on reste trivial sur la cinematique mais on fait apparaitre des term de fluctuation et en plus de ca on doit résoudre la dynamic 
        }
\end{itemize}
\end{itemize}
}
%tout les paiers de Zhang et Prosp, papier de Hesla, papier de Advani et tcucker ?, papier bulle ellispodiale ?(je m'en charge)

%\item mon principal probleme avec la version actuelle du papier est que l'on ne voit pas encore bien la plus value par rapport à la litterature existante.  En particulier comme on en discutait je pense qu'il faut insister sur lequivalence entre particle phase and solide phase averaging, plus le fait que tu derives tout un set dequations sans hypotheses.

\JL{
Remarques specifiques :
\begin{itemize}
\item la reference au livre de Morel nest pas complete ni celle de Gatignol d'ailleurs. Merci de bien verifier que les articles sont cites correctement

\tb{Fait}

\item trier les references par odre alphabétique dans le .bib permets de bien mieux s'y retrouver. Je t'avais deja fait la remarque ce serait bien de le faire ...

\tb{Fait}

bleblela

\item l'anglais est clairement a revoir tout comme la construction de certaines phrases. tu trouveras un certain nombre d'exemple dans le corps du texte. Tu emplois tres regulierement besides qui a ma connaissance ne s'emploie que rarement. Idem pour "This way" ...
\item tu introduis des subsubsection alors qu'il n 'y a pas de subsection. Ce n'est pas recommande.
\item merci de bien verifier que tes citations d'equations apparaissent et qu'elles pointent bien vers la bonne equation.
\item de maniere generale il faut etre plus precis dans ce que tu fais. Quand tu cites un papier il faut citer avec precision le bon ou les bons auteurs, quand tu ecris quelque chose il faut que tu sois sur de sa veracite et egalement ecrit de la maniere la plus limpide possible.
\item pour definir les quantités lagrangiennes cela pourrait etre interessant dutiliser une police ou une notation differente comme discute ensemble. Par exemple mathcal et/ou eventuellement de mettre ces quantites en majuscule.
\item comme suggerais dans un des mails de Lhuillier je trouve que la notation 1 et 2 pourrait etre rendue plus claire par l'utilisation de c (continuous) et dispersed. Apres c'est peut etre plus lourd en terme de notation.
\item je pense qu'il faut definir le theoreme de Reynolds que tu utilises un certain nombre de fois au cours du papier.\tb{Thermohydraulique des réacteurs}
\item en relisant le cours de Daniel Lhuillier, je m'apercois que beaucoup de resultats (et de notations) sont semblables. Ce serait peut etre bien de le citer non ?
\tb{Voir avec daniel }
\item j'ai trouve la partie sur les moyennes franchement peu claire : tu commencais par parler des moyennes volumiques pour finalement decrire egalement les autres types de moyennes etc ... J'ai reorganise. Par ailleurs tu mixais definition probabiliste des moyennes (nottament pour la fraction volumqie par exemple) et moyenne volumique. L'ideal pour perdre le lecteur ! Pour information un article n'est pas un chapitre de these et n'est pas oblige d'etre exhaustif. L'idee est plutot d'aller droit au but. Si tu uitlises la moyenne volumique tu la decris et point final.
\item en terme d'application : je pense que les deux applications que tu proposes sont bien : spheroide + bulle. Il faudrait aussi faire le lien avec le papier de Prosperetti sur les contraintes non symetrique.
\tb{le quel ? regarder Prosperetti}
\item il faut eviter les phrases du du type "this term is of great interest". Explique plutot la signification physique de ce dernier.
\item dans la section 5 : tu repars sur des notations probabilistiques ( $_{1|2}$ par exemple). Encore une fois un article n'est pas un chapitre de thèse. Si tu as decide de faire une moyenne volumique il ne FAUT PAS parler de probabilite.
\item de meme si tu peux tenter des choses, je ne suis pas contre mais dans un article il faut etre sur a 200\% de ce que l'on dit. "Note that the corresponding surface quantities appearing in \ref{eq:hybrid_avg_dt_dq_alpha}, would be the volume of the surface, namely $v_{I\alpha} = \int_{\Sigma_\alpha} e_I d\Sigma$, with $e_I$ the thickness of the surface." Cette derniere phrase ne veut rien dire ! La quantite equivalente a la moyenne volumique dans ce cas surfasique est simplement la surface.
\tb{Non pas d'accord dans  les deux quantité doivent avoire la même unité qd on fait la somme des equations de surafce et de volume donc le volume de la surafce est bien son épaisseur fois la surafec, de la même manière que l'on integre la mass surfacique pour obtenir la mass de la surface. tout ca vient du fait que l'on modélise l'interface comme une surface et non comme un volume }
\item concernant la partie application je pense qu'il faut etre honnete et dire que l'on va rester phenomenologique dans la fermeture des equations. Meme en regime de Stokes cela semble tres (tres) complique. On pourra egalement parler de l'application bulles ellipsoidales (je m'en chargerai). Comme cela c'est plus en accord avec le reste du papier (particules fluides).
\tb{Bubble ellispoidale avec mass ajouté ? ca pourrait etre interessant de faire le parrallel avec le papier de fox dont on a parlé l'autre fois}
\end{itemize}


}

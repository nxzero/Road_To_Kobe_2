\section{Introduction}


\tb{
\begin{itemize}
    \item Give a brief introduction on disperse multiphase flow and insist on the various existing nature :
    \begin{itemize}
        \item suspension of spherical, cylindrical solid particle.
        \item Suspension of deformable particle (blood cells in blood\ldots)
        \item Emulsion : water/oil
        \item Bubbly flows : air/water, air/concrete \ldots
        \item Regarding the physics we can have many mass transfer , compression , etc..
    \end{itemize}
    \item Therefore, our paper is of the main importance because it conciliates any of these case in one unique model. 
    Besides, mention why we insist on the specific application case given in outro. 
    \item Review the literature :
    \begin{itemize}
        \item  hard spheres \citet{batchelor1972sedimentation,jackson1997locally}
        \item  spherical bubble \citet{zhang1994averaged,zhang1994ensemble}
        \item  More complicated shape : (Batchelor)
        \item Work of \citet{morel2015mathematical} (morel, lhuiller and other) for deformable bubbles and surface transport. 
        \item Lagrangian models,  PBE and GPBE \citet{fox2023generalized} 
    \end{itemize}
    \item  Missing work : 
    \begin{itemize}
        \item A general derivation of the hybrid model for all higher moments equations and all conservation laws. (published in \citet{morel2015mathematical} and \citet{zaepffel2012multisize} in specific cases)
        \item Consideration of the surafce properties of the particle (in \citet{morel2015mathematical} they assume that there is no surface properties)
        \item Proof of the equivalence between the dispersed and continuous formalism of the dispersed phase in the most general case and for volumic and surface quantities : 
        \begin{itemize}
            \item 
            It has been treated by \citet{lhuillier2000bilan} in the restricted case of the air density conservation equation considering spherical particle.
            \item \citet{nott2011suspension} investigated this topic also, but their conclusion is not appropriate since they do not deal with the higher moments equations. thus their conclusion is somewhat incomplete. 
            \item And a part has been treated in \citet{zhang1997momentum}
        \end{itemize}
    \end{itemize}
    \item  For all those reason it is of the most important to provide a well posed model which is able to describe any kind of dispersed flows from the local conservation equation and to understand the differences and similarities between both formalism. 
    \item Therefore in this paper we re derived the hybrid model for dispersed two phase flow by considering surface and volumic properties of arbitrary shaped particle.
\end{itemize}
}

Dispersed multiphase flows are ubiquitous in numerous engineering and scientific domains. 
Examples include gas-solid flows in fluidized bed reactors, oil-water emulsions in petroleum industry processes, and fiber suspension in composite manufacturing. 
Developing a versatile model capable of capturing the various nature of the dispersed multiphase flow is therefore essential.
In this work we aim to bridge the existing gaps in current models by providing a unified and adaptable framework for any kind of dispersed multiphase flow problems.

Numerous example of dispersed two-phase flows model had been proposed this last two decades.
As started by \citep{zhang1994averaged,zhang1994ensemble} where they study spherical bubbles suspension.
They provided closures in the inviscid flow limit for the mass and momentum transport equation. 
After that \citet{jackson1997locally,zhang1997momentum} derive the averaged conservation equations for a solid spherical particles suspenison.
They provided the closure in the stokes flow limits.  
They all derived Lagrangian based equations for the dispersed phase considering point of mass spherical particles.
Then how to account for different particle's shape or fluid internal motion in these Lagrangian models ?
Some author tried to model fluid particle within a Lagrangian approach and already answered part of those questions.  
Among them \citet{lhuillier2000bilan} and \citet{morel2015mathematical,zaepffel2012multisize} applied the Lagrangian framework equations for fluid particle with mass transfer. 
These model consist in establishing the conservation equation of an integrated Eulerian quantity $f$, within the volume of a particle denoted by $\Omega_\alpha$, namely  $\int_{\Omega_\alpha} f d\Omega$. 
% Likewise, the so-called first moments of a quantity $f$ is defined by $\int_{\Omega_\alpha} \textbf{r} f_k d\Omega$ with \textbf{r} the position from the center of mass to any point in the volume of the particle. 
% These integrals can be derived within time and yield the moments conservation equitation for any Lagrangian moment, , of the particle $\alpha$. 
By applying an averaging procedure on these equations we can then derive the hybrid model for fluid particles. 

Even though these hybrid models for fluid particle were already quite sophisticated it still misses some major points. 
Indeed, a proper and general derivation of these models must be established in the most general case possible, i.e. any particle and conservation laws. 
Additionally, to the best of our knowledge, the surface properties such as surface tension and chemical concentration of surfactant, have never been taken into account into such a model. 
While it is of major importance in the hydrodynamics of particles suspension \citet{kentheswaran2022direct}. 
Therefore, we must provide a whole in one, hybrid model that encapsulate all the physics, meaning, surface properties, particles of arbitrary shape, higher moments equations. 


Another question that has been addressed by several authors \citep{nott2011suspension,zhang1997momentum} is the one of the equivalence between particle-averaged and continuous-averaged equations. 
In \citet[Appendix A]{zhang1997momentum} they provided a demonstration that both formalism are equivalent at first order. 
Afterward \citet[Appendix A]{nott2011suspension} the proof that the formalism were strictly equivalent even considering an infinite number higher order terms. 
In \citet{lhuillier2010multiphase} however, they reach a compatible but different conclusion. 
Indeed, they state that the phase-averaged equation applied on the dispersed phase is in fact a series expansion of the particular-averaged equations of different order. 
In all these studies they studied homogeneous spherical suspension of solid particles. 
And they both reached the same conclusion, i.e. the particular-averaged linear momentum equation is rigorously equivalent to the phase-averaged equation. 
This demonstration has been derived for the area density concentration of spherical particles. 
We thus need to provide a clear proof of the equivalence between phase-averaged framework and particle-averaged framework in the most general case for a general conservation equation. 


Based on the local conservation laws presented \ref{sec:two-fluid} and the Lagrangian balance on an arbitrary particle, presented \ref{sec:Lagrangian}, we derive the generalized hybrid model for disperse two phase flow. 
Then we provide a proof which generalize the argument of  \citet{lhuillier2000bilan} on the equivalence. 
From this demonstration we conclude that the particle-averaged equations were sufficient to describe every aspect of the flow since they compose the phase-averaged equation, which confirm the affirmation of \citet[Appendix A]{zhang1997momentum}. 
With the use of this general framework we are now able to identify in which circumstance the specific properties of the particles such as, internal velocities, shape, and surface tension forces, come into play in the averaged model. 
Finally, to gives a more specific insight of the use of such a model in the practical cases, we take the example the contaminated rising bubbly flow and derive the first moment of surface distribution of surfactant concentration on a surface. 
\tb{Specify more the example }
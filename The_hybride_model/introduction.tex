
\section{Introduction}

%Dispersed multiphase flows are ubiquitous in chemical engineering processes. 
%Examples include gas-solid flows in fluidized bed reactors, liquid-liquid flows in liquid-liquid extractors, and bubbly flows in flotation processes. 
%Developing a versatile model capable of capturing the various nature of the dispersed multiphase flow is therefore essential.
%In this work we aim to bridge the existing gaps in current models by providing a unified and adaptable framework for any kind of dispersed multiphase flow problems including interfaccial transport equation.

Dispersed multiphase flows are encountered across a wide range of chemical engineering applications. 
These include gas-solid interactions in fluidized bed reactors, liquid-liquid flows in extractors, and gas-liquid bubbly flows in flotation processes. 
These systems exhibit a wide range of scales, from the size of individual inclusions (as small as a few micrometers) to the size of the reactor (often exceeding one meter), making fully resolved simulations computationally impractical. 
As a result, the current engineering practice relies on averaged equations of motion for both the dispersed and continuous phases. 
Therefore, developing a robust set of averaged equations that accurately captures the complexities of dispersed multiphase flows is essential. 
In this study, we aim to overcome the limitations of existing models by proposing a comprehensive set of averaged equations that can be applied to various types of dispersed multiphase flows, including interfacial transport processes.

%Due to the two many scales encountered in this process ranging from the particle size (from let'say 10 micrometer) to the reactor size (larger thant one meter) it appears to be unfeasible to model the process into play using fully resolved simulations.
%The current engineering practice is to relay on averaged set of equation of motion for both the dipsersed and continuous phase equation.
%Therefore, it is essential to develop a rigorous set of averaged equations that can accurately represent the diverse behaviors of dispersed multiphase flows. 
%In this study, our goal is to address the limitations of existing models by proposing a comprehensive averaged set of equations that can be applied to various kind of dispersed multiphase flow, including interfacial transport equations. %framework 

Numerous example of dispersed two-phase flows model have been proposed this last two decades.
As started by \citep{zhang1994averaged,zhang1994ensemble} where they study spherical bubbles suspension.
They provided closures in the inviscid flow limit for the mass and momentum transport equation. 
After that \citet{jackson1997locally,zhang1997momentum} derive the averaged conservation equations for a solid spherical particle suspension.
They provided the closure in the stokes flow limits.  
They all derived Lagrangian based equations for the dispersed phase considering point of mass spherical particles.
But then how to account for different particle's shapes or fluid internal motion within these Lagrangian models ?
Some author tried to model fluid particle within a Lagrangian approach and already answered part of these questions.  
Among them \citet{lhuillier2000bilan} and \citet{morel2015mathematical,zaepffel2012multisize} applied the Lagrangian framework equations for fluid particle with mass transfer. 
These model consist in establishing the conservation equation of an integrated Eulerian quantity $f$, within the volume of a particle denoted by $\Omega_\alpha$, namely  $\int_{\Omega_\alpha} f d\Omega$. 
% Likewise, the so-called first moments of a quantity $f$ is defined by $\int_{\Omega_\alpha} \textbf{r} f_k d\Omega$ with \textbf{r} the position from the center of mass to any point in the volume of the particle. 
% These integrals can be derived within time and yield the moments conservation equitation for any Lagrangian moment, , of the particle $\alpha$. 
By deriving these time varying integrated properties along with applying an averaging procedure on these equations, we are able to derive a Lagrangian model for fluid particles. 

Even though these hybrid models for fluid particles were already quite sophisticated it still misses some major points. 
Indeed, to the best of our knowledge, the surface properties such as surface tension and chemical concentration of surfactant over the surface, have not been taken into account into such averaged hybrid model. 
While it is of major importance to predict the hydrodynamics bubbly flows. 
% Additionally, a proper and general derivation of these models must be established in the most general case possible, i.e. for any kind particles and conservation laws. 
Therefore, a whole in one, hybrid model that encapsulate all the physics, meaning, surface properties, particles of arbitrary shape, higher moments equations is needed. 


Another question that has been addressed by several authors \citep{nott2011suspension,zhang1997momentum} is the one of the equivalence between particle-averaged and continuous-averaged equations. 
In \citet[Appendix A]{zhang1997momentum} they provided a demonstration that both formalism are equivalent at first order. 
Afterward \citet[Appendix A]{nott2011suspension} provided the proof that the formalism were strictly equivalent even considering an infinite number of higher order terms. 
In \citet{lhuillier2010multiphase} however, they reach a compatible but different conclusion. 
Indeed, they state that the phase-averaged equation applied on the dispersed phase is in fact a series expansion of the particular-averaged moments equations. 
This demonstration has been derived for the area density concentration of spherical particles. 
In all these studies they focus on mono-disperse spherical particle suspension. 
And they all reached the same conclusion, i.e. the particular-averaged linear momentum equation is rigorously equivalent to the phase-averaged momentum equation for the dispersed phase. 
We thus need to provide a clear proof of the equivalence between phase-averaged framework and particle-averaged framework in the most general case and for a general conservation equation. 


Based on the local conservation laws presented \ref{sec:two-fluid} and the Lagrangian balance on an arbitrary particle, presented \ref{sec:Lagrangian}, we derive the generalized hybrid model for disperse two phase flow. 
Then we provide a demonstration which proves that the argument of \citet{lhuillier2000bilan} on the equivalence is generalizable to any kind of dispersed multiphase flow and conservation law. 
From this demonstration we conclude that the particle-averaged equations are sufficient to describe every aspect of the flow since they compose the phase-averaged equation, which confirm the affirmation of \citet[Appendix A]{zhang1997momentum}. 
With the use of this general framework we are now able to identify in which circumstance the specific properties of the particles such as, internal velocities, shape, and surface tension forces, come into play in the averaged model. 
% Finally, to gives a more specific insight of the use of such a model in the practical cases, we take the example the contaminated rising bubbly flow. 
% In particular, we demonstrate how to derive a model to predict the distribution of surfactant over the interface of bubbles, as it is of major importance to predict mass transfer and drag force correlation \citep{kentheswaran2022direct}.
% Additionally, we also show how to derive the orientation tensor equation in fibrous media. 


The organization of this manuscript is as follows. 
We start in \ref{sec:local_eq} to expose the generic formulation of the local scale governing equations as well as the so-called topological equations. 
It is then demonstrated how to perform an average onto these equations to obtain the classical two-fluid formulation of two phase flows. 

\JL{TO DO.
\begin{itemize}
    \item mettre en parenthèse les numeros dans les equations. Eviter aussi les repetitions de Equation ..., Equation ..., ... et remplacer par Equations () and () par exemple
    \item faire une derniere passe en faisant attention aux notations (par exemple $\text q_\alpha$ qui ne doit pas etre en italique), à l'anglais, aux coquilles de tout type
    \item je n'ai pas relu ni l'annexe C ni l'annexe D. Mais dans ces dernieres il faudrait specifier ce que veut dire le signe $[]$ ou le remplacer par une notation plus usuelle.
\end{itemize}

}


In \cite{lhuillier2000bilan} they reached a similar conclusion when comparing the phase-averaged area density and particle-averaged area density equations for spherical particles. 
\citet[Appendix A]{zhang1997momentum} provided evidences that the particle-averaged momentum equation is as legitimate as the phase-averaged momentum equation, which is consistent with \ref{eq:proof2}. 
Additionally, \citet[Appendix A]{nott2011suspension} derived a similar expression than \ref{eq:proof2}, also in the case of the averaged momentum equation for suspension of solid spherical particles.
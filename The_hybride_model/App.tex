
Now that we reached a clear understanding of the mathematical structures of the averaged two-phase flow equations we now expose the averaged set of equations which constitute the \textit{Hybrid model}. 
In this section we consider the simplifying assumption exposed in \ref{ap:hypothesis} for the interfaces equations. 
As mentioned in \ref{sec:two-fluid} we derive the mass, momentum and energy for the particles and continuous phase. 
Additionally, to describe the particle shape and inner velocity, one must consider the second moment of mass and first moment of momentum averaged equations. 
This, makes a total of 10 equations, 6 for the particle phase and 4 for the continuous phase.

To support the subsequent discussion, we provide the expressions of the closure terms for a dilute emulsion of spherical droplets. 
Even though the closures are expressed in the Stokes limit, note that the set of equations provided remains valid regardless of the flow regime.
We will consider a monodisperse suspension of droplets with radius $a$ and viscosity $\mu_d$. 
Although much of this information is already known, it is useful for the purposes of understanding. 
Additionally, significant implications regarding the energy equation arise due to the consideration of fluid particles, even in the limit of Stokes flows. 

\subsection{Continuous phase equations}

The equations for the carrier fluid are basically the same as in the classic two-fluid model derived in \ref{ap:two-fluid_model}, except for the exchange terms of the form $\avg{\delta_I \ldots }$, that must be reformulated in terms of particle-averaged quantities to be consistent with the particle-phase equations \citep{jackson1997locally,zhang1994averaged}. 
This is achieved through the use of \ref{eq:f_exp} which enables us to convert the exchange terms of the form $\avg{\delta_I \ldots }$ appearing in \ref{eq:avg_dt_chi_f} into a series expansion of particle phase quantities. 
For clarity, we only retain the first order term in the expansion of the interfacial terms. 
However, understand that the higher order moments are not necessarily negligible. 
The continuous phase-averaged mass, momentum and total energy equations yield, 
\begin{align}
    \label{eq:dt_hybrid_rho}
    &\pddt (\phi_f \rho_f)  
    + \div (
        \phi_f \rho_f\textbf{u}_f
    )
    = 
    0\\
    \label{eq:dt_hybrid_rhou_f}
    &\pddt (\phi_f \rho_f\textbf{u}_f)  
    + \div (
        \phi_f \rho_f\textbf{u}_f\textbf{u}_f
        + \bm{\sigma}_f^\text{eq}
    )
    = 
    \phi_f \rho_f \textbf{g} 
    - \pSavg{{\bm{\sigma}_f^0 \cdot \textbf{n}_d}}
    % +\div  \pSavg{{\textbf{r}\bm{\sigma}_f^0 \cdot \textbf{n}_d}}
    \\
    \label{eq:dt_hybrid_rhoE_f}
    &\pddt (\phi_f\rho_fE_f)  
    + \div (
        \phi_f\rho_fE_f\textbf{u}_f
        + \bm{q}_f^\text{eq}
        + \textbf{u}_f \cdot \bm{\sigma}_f^\text{eq}
        % - \textbf{u}_f^0 \cdot \bm{\sigma}_f^0 
        % + \textbf{q}_f^0
        )
    = 
    \phi_f \rho_f\textbf{u}_f \cdot \textbf{g} 
    - \textbf{u}_p \cdot \pSavg{{\bm{\sigma}_f^0 \cdot \textbf{n}_d}}\nonumber \\
    &- \pavg{ \textbf{u}_\alpha' \cdot \intS{  \bm{\sigma}_f^0 \cdot \textbf{n}_d}}
    - \pavg{ \intS{\textbf{w}_d^0 \cdot \bm{\sigma}_f^0 \cdot \textbf{n}_d}}
    + \pSavg{{\textbf{q}_f\cdot \textbf{n}_d}}
    % &\div [    
        % \textbf{u}_p \cdot \pSavg{{ \textbf{r}\bm{\sigma}_f^0 \cdot \textbf{n}_d}}
    % + \pavg{ \textbf{u}_\alpha' \cdot \intS{ \textbf{r} \bm{\sigma}_f^0 \cdot \textbf{n}_d}}
    % + \pavg{ \intS{\textbf{r}\textbf{w}_d^0 \cdot \bm{\sigma}_f^0 \cdot \textbf{n}_d}}
    % - \pavg{ \intS{\textbf{r}  \textbf{q}_f^0 \cdot \textbf{n}_d}}
    % ]
\end{align} 
where we defined the equivalent stress tensor $\bm{\sigma}_f^\text{eq}$ and energy flux $\textbf{q}^\text{eq}_f$ as,
\begin{align}
    \label{eq:sigma_eq_def}
    \bm{\sigma}_f^\text{eq}
    =& 
    \avg{\chi_f\rho_f\textbf{u}_f'\textbf{u}_f'}
    - \phi_f \bm{\sigma}_f%- n_p \textbf{M}_p
    - \pSavg{{\textbf{r}\bm{\sigma}_f^0 \cdot \textbf{n}_d}}\\
    \textbf{q}_f^\text{eq}
    =&\textbf{q}_f^\text{e} +\textbf{q}_f^\text{k}  \nonumber\\
    \textbf{q}_f^\text{e}
    =& \rho_f \avg{\chi_f \textbf{u}_f' e_f'} 
    + \phi_f\textbf{q}_f 
    +\pSavg{{\textbf{r}\textbf{q}_f^0 \cdot \textbf{n}_d}} 
    \nonumber\\
    \textbf{q}_f^\text{k}
    =& \rho_f \avg{\chi_f \textbf{u}_f' k_f} 
    - \avg{\chi_f \textbf{u}_f' \cdot \bm{\sigma}_f^0}
    + (\textbf{u}_f - \textbf{u}_p)\cdot
    \pSavg{{\textbf{r}\bm{\sigma}_f^0 \cdot \textbf{n}_d}}
    \nonumber\\\nonumber&
    - \pavg{ \textbf{u}_\alpha' \cdot \intS{ \textbf{r} \bm{\sigma}_f^0 \cdot \textbf{n}_d}}
    - \pavg{ \intS{\textbf{r}\textbf{w}_d^0 \cdot \bm{\sigma}_f^0 \cdot \textbf{n}_d}}
\end{align}
It is clear that those equations yield essentially the same as the previous set of equations presented in \ref{ap:two-fluid_model}.
The only difference is the presence of additional fluxes inside $\bm{\sigma}^\text{eq}_f$ and $\textbf{q}^\text{eq}_f$ due to the expansion of the interfacial terms. 

The averaged continuous-phase momentum balance \eqref{eq:dt_hybrid_rhou_f} was established long ago by numerous authors \citep{zhang1997momentum,jackson1997locally,drew1983mathematical,kataoka1986local}.  
% However, note that here we choose to expose a very general formulation.  
In our formulation the term $\pSavg{\bm{\sigma}_f^0 \cdot \textbf{n}_d}$ represents the total components of the interphase drag force, including the mean divergence of the fluid phase stress $\bm\sigma_f^0$. 
Likewise, $\pSavg{\textbf{r}\bm{\sigma}_f^0 \cdot \textbf{n}_d}$ is the total averaged first moment of force traction, which includes the mean fluid phase stress. 
Similar consideration hold for the second moment of the force traction. 
For purpose of understanding note that in the case of a dilute emulsion of identical droplets of radius $a$ these terms reads as
\begin{align}
    \pSavg{\bm{\sigma}_f^0\cdot \textbf{n}_d} &= 
    \phi_d \div\bm\Sigma_f
    + \frac{3\phi_d\mu_f}{2 a^2} 
    \left(\frac{3\lambda+2}{\lambda+1}\right) \textbf{u}_{f p} 
    + \frac{3\phi_d\mu_f}{4} \left(\frac{\lambda}{\lambda+1}\right)\grad^2\textbf{u}_f\\
    \label{eq:first_mom}
    \pavg{\intS{\textbf{r}\bm{\sigma}_f^0 \cdot \textbf{n}_d}} 
    &= \phi_d \bm\Sigma_f + 
    \frac{3}{5}\mu_f \phi_d \left(\frac{2+5\lambda}{1+\lambda}\right)
    \textbf{E}_f
        \\
        \pavg{\intS{(\bm{\sigma}_f^0 \cdot \textbf{n}_d)_ir_kr_l}} &=
        \frac{3\mu_f\phi_d}{10}\left(\frac{5\lambda+1}{\lambda+1}\right)u_{fp,i}\delta_{kl}
        + \frac{3\mu_f\phi_d}{5}\left(\frac{1}{\lambda+1}\right)(u_{fp,k}\delta_{il}+u_{fp,l}\delta_{ki})
\end{align}
Where $\bm\Sigma_f = -p_f \bm\delta+2\mu_f \textbf{E}_f$ with $\textbf{E}_f = \frac{1}{2}[\grad \textbf{u}_f+(\grad \textbf{u}_f)^\dagger]$.
Were we highlighted the fact that $2\phi_f \textbf{e}_f = \avg{\chi_f [\grad \textbf{u}_f^0+(\grad \textbf{u}_f^0)^\dagger]}$ is not the same as $2\phi_f \textbf{E}_f = 2 \mu_f [\grad \textbf{u}_f+(\grad \textbf{u}_f)^\dagger]$ for fluid particles.
Note that \citet{zhang1997momentum} decided to retrieve the continuous-phase stress $\bm\sigma_f$ to each of these terms instead of $\bm\Sigma_f$, leaving the remaining exchange terms to represent the stress resultant caused by the disturbance field around the particles (see \ref{ap:conditionally_navier_stokes}). 
This is why both formulations appear different, nevertheless they are indeed consistent.
The tensor $\pSavg{\textbf{r}\bm{\sigma}_f^0 \cdot \textbf{n}_d}$ is responsible for the famous Einstein viscosity correction \citep{guazzelli2011} which is valid in the stokes flow regime for solid spherical inclusion. 
Consequently, this term is of particular interest in the averaged momentum equations and is non-negligible in most of the flow conditions, if not all of them.
The second moment appearing in the series is $\pSavg{\textbf{rr}\bm{\sigma}_f^0 \cdot \textbf{n}_d}$. 
Although not displayed in \ref{eq:dt_hybrid_rhou_f} it is shown in \citet{jackson1997locally,zhang1997momentum} that to remain first order in the dispersed phase volume fraction $\phi_d$, one must keep this term in the momentum equation. 
Consequently, it is likely that these moments (at least the first two) are non-negligible in most of the situation. 
Even though this is known since \citet{jackson1997locally} and \citet{zhang1997momentum} it is surprising that most of the studies in the literature regarding Euler-Euler equations neglected these moments. 
In fact, most of the time they are not even mentioned, especially the second moment of the hydrodynamic force traction. 
Conversely, the Reynolds stress term $\avg{\chi_f \rho_f \textbf{u}_f'\textbf{u}_f'}$  is subject of extensive studies. 
This term represents it the stress related to the microscale agitation whether it is actual turbulence or particles induced turbulence. 

Now, let's discuss the continuous-phase averaged total energy balance \eqref{eq:dt_hybrid_rhoE_f}. 
Most of the terms have already been addressed in \ref{ap:two-fluid_model}, so for now, let's direct our attention to the exchange terms. 
On the right-hand side of \ref{eq:dt_hybrid_rhoE_f} we identify four exchange terms.
Indeed, after taking the Taylor expansion of the interfacial term $\avg{\delta_I (\textbf{u}^0_d \cdot \bm{\sigma}_f^0 \cdot \textbf{n}_d)}$ of \ref{eq:dt_avg_rhoE_k}, we used the following decomposition on each of the moments:
\begin{align}
    \label{eq:exergysource}
    \pavg{ \intS{\textbf{u}^0_d \cdot \bm{\sigma}_f^0 \cdot \textbf{n}_d}}
    &= 
    \textbf{u}_p \cdot \pSavg{{\bm{\sigma}_f^0 \cdot \textbf{n}_d}}
    + \pavg{ \textbf{u}_\alpha' \cdot \intS{  \bm{\sigma}_f^0 \cdot \textbf{n}_d}}
    + \pavg{ \intS{\textbf{w}_d^0 \cdot \bm{\sigma}_f^0 \cdot \textbf{n}_d}},\\
    \label{eq:exergysource2}
    \pavg{ \intS{\textbf{r}\textbf{u}^0_d \cdot \bm{\sigma}_f^0 \cdot \textbf{n}_d}}
   &= 
    \textbf{u}_p \cdot \pSavg{{\bm{\sigma}_f^0 \cdot \textbf{n}_d}}
    + \pavg{ \textbf{u}_\alpha' \cdot \intS{\textbf{r}  \bm{\sigma}_f^0 \cdot \textbf{n}_d}}
    + \pavg{ \intS{\textbf{r}\textbf{w}_d^0 \cdot \bm{\sigma}_f^0 \cdot \textbf{n}_d}},
\end{align}
where we have noticed that $\textbf{u}_d^0 = \textbf{u}_p + \textbf{u}_\alpha' +\textbf{w}_d^0$ according to \ref{eq:def_fluc_p} and to the definition of the \textit{inner velocity} of a particle $\textbf{w}_d^0$. 
In this form the contribution to the energy exchange is now clear. 
The first term on the right hands side of \ref{eq:exergysource} represents the work done by the mean particle-phase motion with the mean drag force.
The second term is the covariance term of the velocity of the particles with their respective drag forces.
In a relatively dilute suspension the drag force applied on a single particle is likely to be a function of its instantaneous velocity, thus in a general manner this term is non-negligible. 
The last term represents the work made by the local force traction on the particle surface with the velocity at the surface of the particles $\textbf{w}_d^0$.
Regarding \eqref{eq:exergysource2} same comments can be made except that these terms act as energy fluxes instead of sources. 
The relative importance of these three contribution depends highly on the particles' nature. 
To our knowledge, such a decomposition has not been seen in the literature except in \citep[Chapter 2]{scorsim2021particle} where they make similar consideration, but for solid spherical particles.
We recall that the stress integral $\pSavg{\bm{\sigma}_f^0 \cdot \textbf{n}_d}$ contains contact forces as well, making our model consistent with the latter study. 
Our formulation is therefor more general. 



The continuous phase averaged total energy can be further decomposed into three energy components (see \ref{ap:two-fluid_model}), that is,  
\begin{align}
    E_k = e_k + k_k + u_k^2/2
    \label{eq:E_def}
\end{align}
where $k_k$ is the pseudo-turbulent kinetic energy defined as, $\phi_k k_k = \frac{1}{2}\avg{\chi_k \textbf{u}_k'\cdot \textbf{u}_k'}$. 
To fully describe the averaged total energy one must add at least a supplementary equation either for $k_k$ or $e_k$. 
The derivation of these equations is explained in \ref{ap:two-fluid_model} in the context of the two-fluid formulation. 
However, \ref{eq:dt_avg_uk2}, \ref{eq:dt_avg_kk} and \ref{eq:dt_avg_ek} need to be reformulated consistently with the dispersed phase exchange term. 
Therefore, we reformulate the kinetic energy, pseudo turbulent energy and internal energy equations as
\begin{align}
    \pddt (\phi_f \rho_fu_f^2/2)  
    + \div (
        \phi_f \rho_f\textbf{u}_fu_f^2/2
        + \textbf{u}_f \cdot \bm{\sigma}_f^\text{eq}
    )
    = 
     \bm{\sigma}_f^\text{eq} : \grad \textbf{u}_f
    + \phi_f \rho_f \textbf{u}_f\cdot \textbf{g} 
    -  \textbf{u}_f\cdot 
        \pSavg{{\bm{\sigma}_f^0 \cdot \textbf{n}_d}} 
        \\
    \label{eq:dt_hybrid_k1}
    \pddt (\phi_f\rho_fk_f)  
    + \div (
        \phi_f\rho_fk_f\textbf{u}_f
        + \textbf{q}_f^\text{k} 
        )
    = 
    - \avg{\chi_f\bm{\sigma}_f^0 : \grad \textbf{u}_f^0}
    - \bm{\sigma}_f^\text{eq} : \grad \textbf{u}_f\nonumber
    - \pavg{ \textbf{u}_\alpha' \cdot \intS{  \bm{\sigma}_f^0 \cdot \textbf{n}_d}}\\
    + (\textbf{u}_f - \textbf{u}_p)\cdot \pSavg{{\bm{\sigma}_f^0 \cdot \textbf{n}_d}} 
    - \pavg{ \intS{\textbf{w}_d^0 \cdot \bm{\sigma}_f^0 \cdot \textbf{n}_d}} 
    \\
    \label{eq:dt_hybrid_e1}
    \pddt (\phi_f\rho_fe_f)  
    + \div (
        \phi_f \rho_fe_f\textbf{u}_f
        +
        \textbf{q}_f^\text{e} 
        )
    = 
    \avg{\chi_f\bm{\sigma}_f^0 : \grad \textbf{u}_f^0}
    + \pSavg{{\textbf{q}_f^0 \cdot \textbf{n}_d}} 
\end{align}
One can verify that summing these three equations yields \ref{eq:dt_hybrid_rhoE_f}. 
The averaged pseudo-turbulent energy balance \eqref{eq:dt_hybrid_k1} is consistent with those in former studies \citep[Chapter 7]{morel2015mathematical}\citep[Chapter 2]{scorsim2021particle}\citet{kataoka1989basic}. 
However, the decomposition of the exchange term is not present, and the expression of $\textbf{q}_f^k$, which collects the first moments of the work, has not been discussed in the literature in such generality.
The energy exchange between the macroscopic, microscopic, internal energy as well as the energy exchange between both phases, will be addressed later on.   

Let us focus on the last term of \ref{eq:exergysource}.
As mentioned above this is the source of pseudo turbulent energy $k_f$. 
As a first approximation one might consider that every particle in the flow posses the surface velocity of an isolated particle in stokes flow immersed in an arbitrary linear flow.
In this case the \textit{inner velocity} evaluated at the surface of the particle $\alpha$ can be written as
\begin{equation*}
    \textbf{w}_d^0 
    = \left(\frac{\lambda + \frac{1}{2}}{\lambda +1} - 1\right)
    \textbf{u}_{pf} 
    + 
    \frac{1}{2}\left(\frac{1}{\lambda +1}\right)
    \textbf{rr} \cdot \textbf{u}_{pf} 
    + \left[1-\frac{\lambda}{(\lambda + 1)}\right]\textbf{E}_f\cdot\textbf{r}
    -\frac{1}{a^2}\left(\frac{1}{\lambda +1 } \right) \textbf{r} \textbf{E}_f:\textbf{rr}. 
\end{equation*}
Injecting this expression into the Last term on the right-hand side of \ref{eq:exergysource} gives, 
\begin{multline}
    \pSavg{\textbf{w}_2^0 \cdot \bm{\sigma}_1^0\cdot\textbf{n}_2}
    =  
    \frac{1}{\lambda+1}\textbf{u}_{p f} \cdot \left[
        \left(\frac{\lambda +1 }{2}\right)\pSavg{ \bm{\sigma}_1^0\cdot\textbf{n}_2}
        % + \pavg{\textbf{u}_{\alpha}' \cdot \intS{ \bm{\sigma}_1^0\cdot\textbf{n}_2} }
        + \frac{1}{2a^2}
        \pSavg{\textbf{rr}\cdot \bm{\sigma}_1^0\cdot\textbf{n}_2}
    \right]
    \\
    % \left[
    %     \textbf{u}_{p f} \cdot
    %     +
        % \pavg{\textbf{u}_{\alpha}' \cdot \intS{\textbf{rr}\cdot \bm{\sigma}_1^0\cdot\textbf{n}_2}}
    % \right]
    + \frac{1}{\lambda + 1} \textbf{E}_{f} : \left[ 
         \pSavg{\textbf{r} \bm{\sigma}_1^0\cdot\textbf{n}_2}
         -\frac{1}{a^2} 
         \pSavg{ \textbf{rrr} \cdot \bm{\sigma}_1^0\cdot\textbf{n}_2}
         \right]
    \label{eq:energy_term}
\end{multline}
In this expression we clearly identify the zero, first and second order moments of the surface force traction provided by \ref{eq:first_mom}. 
Additionally, notice that we did not consider rotation of droplets consequently the mean fluid vorticity nor the particles angular velocity appears in this expression. 
Therefore, one can notice that taking the limit $\lambda \to \infty$ gives zero for \ref{eq:energy_term}. 
Which is consistent since $\textbf{w}_d^0 = 0$ for non-rotating solid particles. 
In light of \ref{eq:energy_term} the simple presence of a fluid particle in a linear flow generate pseudo turbulence due to its simple presence. 
Likewise, the relative translation between phases, induce motion on the fluid interface which produce work. 


\subsection{Dispersed phase equations}

% Now, we turn our attention to the particle phase equations. 
% \tb{Do i put the surface in or out}
% \subsubsection{Primary equations}

By applying the ensemble average on \ref{eq:dt_m_alpha}, \ref{eq:dt_p_alpha} and \ref{eq:dt_E_alpha_tot} we obtain the particle averaged mass, momentum and energy equation, namely, 
\begin{align}
    \label{eq:dt_hybrid_mp}
    \pddt \left(n_p m_p\right)
    + \div \left(n_pm_p\textbf{u}_p
    \right)
    = 
    0\\
    \label{eq:dt_hybrid_up}
    \pddt \left(n_p m_p \textbf{u}_p\right)
    + \div \left(n_p
    m_p \textbf{u}_p \textbf{u}_p 
    + \bm{\sigma}_p^\text{eq}
    \right)
    = 
    n_p m_p \textbf{g}
    + \pSavg{{\bm{\sigma}_f^0 \cdot \textbf{n}_d}},\\
    \label{eq:dt_hybrid_Ep}
    \pddt(m_p n_pE_p^\text{tot})
    + \div(m_pn_p E_p^\text{tot} \textbf{u}_p 
    + \textbf{q}_p^\text{eq} 
    + \textbf{u}_p \cdot \bm{\sigma}_p^\text{eq})
    =  n_p m_p \textbf{u}_p\cdot  \textbf{g}
    % +  n_p ( \textbf{u}'_f \cdot \bm{\sigma}_f^0 \cdot \textbf{n}_d)_p^\Sigma
    -  \pSavg{\textbf{q}_f^0 \cdot \textbf{n}_d}\nonumber\\
    + \textbf{u}_p \cdot\pSavg{{\bm{\sigma}_f^0 \cdot \textbf{n}_d}}
    + \pavg{\textbf{u}_\alpha' \cdot\intS{\bm{\sigma}_f^0 \cdot \textbf{n}_d}}
    + \pSavg{{\textbf{w}_d^0 \cdot\bm{\sigma}_f^0 \cdot \textbf{n}_d}}
\end{align}
where we have defined, 
\begin{align*}
    &\bm{\sigma}_p^\text{eq}
    =  m_p\pavg{\textbf{u}_\alpha'\textbf{u}_\alpha'}
    &\textbf{q}_p^\text{eq}
    =\textbf{q}_p^\text{e} 
    +\textbf{q}_p^\text{k}  
    +\textbf{q}_p^\text{w}  
    \\
    &\textbf{q}_f^\text{e}
    = m_p \pavg{\textbf{u}_\alpha' e_\alpha'} 
    &\textbf{q}_p^\text{k}
    = m_p \pavg{\textbf{u}_\alpha' k_\alpha} 
    \\
    &\textbf{q}_p^\text{w}
    = 
     \pavg{\textbf{u}_\alpha'W_\alpha'}
    + \pavg{\textbf{u}_\alpha' s_\alpha' \gamma}.
\end{align*}
We have introduced the internal kinetic energy with $n_pW_p = \pOavg{{\rho_d  (w_d^0)^2/2}}$. 
We recognize that these equations posses indeed the exchange terms corresponding to the fluid phase averaged equations but with opposite sign. 
However, note that in opposition to the fluid phase averaged equations, the first order moments do not appear inside the fluxes of the particles equations. 
Consequently, only the fluctuating quantities plays the role of dissipative fluxes. 
It is noteworthy to mentions that in the total drag force term, $ \pSavg{{\bm{\sigma}_f^0 \cdot \textbf{n}_d}}$ the divergence of a stress is hidden, teh latter represent particles-particles contact forces, \citet{jackson1997locally,zhang1997momentum}. 
One may argue that this is not consistent since this stress would also appear on the fluid phase momentum equation upon the development of the term $\pSavg{\bm{\sigma}_f^0\cdot\textbf{n}_d}$. 
However, this is made consistent if one notice that contact force stress is also present in the first moment $\pSavg{\textbf{r}\bm{\sigma}_f^0\cdot\textbf{n}_d}$ but with opposed sign. 
Likewise, note that in some recent models it is possible to expands the momentum exchange terms, as the sum of a \textit{binary force} and the divergence of a stress accounting for particles' long range interaction forces \citep{zhang2021ensemble,nott2011suspension}. 
In opposition to the contact stress, this long range interaction stress, appears on the particle and carrier fluid momentum conservation equation. 
Eventhrougth, the latter stress has been shown to be indispensable to ensure the hypertonicity of the two phase flow equations\citep{fox2020hyperbolic}, we choose to not explicitly display this kind of stresses for succinctness. 

% \subsubsection{Secondary equations}

The particle averaged total energy can be decomposed in the similar way than the continuous averaged total energy \ref{eq:E_def}. 
The decomposition is somewhat more involving than the continuous phase and reads as, 
\begin{equation*}
    n_p m_p E_p^\text{tot}(t) 
    = m_p n_p e_p 
    + n_p W_p
    + n_p s_p \gamma
    + m_p n_p k_p
    + m_p n_p (u_p)^2/2
    \label{eq:E_p_def}
\end{equation*}
The total energy of the particle phase is made of :
the mean particle's internal energy $e_p$; 
the averaged particle's internal kinetic energy $W_p$;
the averaged particle's surface energy $n_p s_p \gamma$;
the granular temperature $n_p k_p =\pavg{\textbf{u}_\alpha \cdot\textbf{u}_\alpha}/2$;
and the kinetic energy of the mean particle phase velocity. 
% The mean surface energy $n_p s_p \gamma$ is treated as a source terms in the following equations, that is why it doesn't appear in \ref{eq:E_p_def}.  
If one wish to solve for every component of the energy it is therefore needed to derive two supplementary equation. 
Applying the average procedure on \ref{eq:dt_e_alpha}, \ref{eq:dt_w2_alpha} and \ref{eq:dt_u2_alpha} one can derive an equation for the particle averaged internal energy, internal kinetic energy and mean kinetic energy, it yields, 
\begin{align}
    % &\pddt \left(n_p m_p u_p^2/ 2\right)
    % + \div \left(n_p
    % m_p u_p^2/ 2 \textbf{u}_p 
    % + \textbf{u}_p \cdot \bm{\sigma}_p^\text{eq}
    % \right)
    % = 
    % + \bm{\sigma}_p^\text{eq}  :\grad \textbf{u}_p
    % +  n_p v_p \textbf{u}_p \cdot 
    % \rho_d \textbf{g}
    % + n_p \textbf{u}_p \cdot (\bm{\sigma}_f^0 \cdot \textbf{n}_d)^\Sigma_p,\\
    \label{eq:dt_hybrid_u2p}
    \pddt \left(\pavg{m_\alpha u_\alpha^2/2}\right)
    + \div \left(\pavg{m_\alpha u_\alpha^2/2} \textbf{u}_p 
    + \textbf{q}^k_p
    + \textbf{u}_p \cdot \bm{\sigma}_p^\text{eq}
    \right)
    = 
    n_p m_p \textbf{u}_p \cdot
    \textbf{g}\nonumber\\  
    + \textbf{u}_p\cdot\pSavg{{\bm{\sigma}_f^0 \cdot \textbf{n}_d}}
    + \pavg{\textbf{u}_\alpha'\cdot\intS{\bm{\sigma}_f^0 \cdot \textbf{n}_d}}
    \\
    \label{eq:dt_hybrid_Wp}
    \pddt \left(n_p (W_p + s_p\gamma)\right)
    + \div 
    (n_p (W_p + \gamma s_p)
    \textbf{u}_p 
    +  \textbf{q}_p^\text{w}
    )
    = 
    - \pOavg{{\bm{\sigma}_d^0 : \grad\textbf{u}_d^0}}
    + \pSavg{{\textbf{w}_d^0 \cdot \bm{\sigma}_f^0 \cdot  \textbf{n}_d}}
    % - \pavg{\dot{ s_\alpha}}
    \\
    \pddt \left(n_p m_p e_p\right)
    + \div \left(n_p
    m_p e_p \textbf{u}_p 
    +  \textbf{q}_p^\text{e}
    \right)
    = 
    + \pOavg{{\bm{\sigma}_d^0 : \grad\textbf{u}_d^0}}
    - \pSavg{{\textbf{q}_f^0\cdot \textbf{n}_d}}
    \label{eq:dt_hybrid_ep}
\end{align}
The center of mass kinetic energy can be further decomposed such as $\pavg{u_\alpha^2}/2 = n_p k_p + n_p u_p^2/2$. 
Then, to derive an equation for $k_p$ one must retrieve to \ref{eq:dt_hybrid_u2p} the dot product of \ref{eq:dt_hybrid_up} with $\textbf{u}_p$, which eventually yields an equation for the mean kinetic energy and another for the granular temperature $k_p$, namely,
\begin{align}
    \label{eq:dt_hybrid_up2}
\pddt \left(n_p m_p u_p^2/ 2\right)
    + \div \left(n_p
    m_p u_p^2/ 2 \textbf{u}_p 
    + \textbf{u}_p \cdot \bm{\sigma}_p^\text{eq}
    \right)
    = 
    \bm{\sigma}_p^\text{eq}  :\grad \textbf{u}_p
    +  n_p m_p \textbf{u}_p \cdot 
     \textbf{g}
    + \textbf{u}_p \cdot \pSavg{{\bm{\sigma}_f^0 \cdot \textbf{n}_d}},\\
    \label{eq:dt_hybrid_kp}
    \pddt \left(n_p m_p k_p\right)
    + \div \left(n_p
    m_p k_p \textbf{u}_p 
    + \textbf{q}^k_p
    % + \textbf{u}_p \cdot \bm{\sigma}_p^\text{eq}
    \right)
    = 
    - \bm{\sigma}_p^\text{eq}  :\grad \textbf{u}_p
    + \pavg{\textbf{u}_\alpha'\cdot\intS{\bm{\sigma}_f^0 \cdot \textbf{n}_d}},
\end{align}
respectively.
Since equation \ref{eq:dt_hybrid_Wp}, \ref{eq:dt_hybrid_ep} and \ref{eq:dt_hybrid_u2p} are discussed in \ref{ap:particles_eq} let's focus on the granular temperature equation. 
The usual way to derive the granular temperature equations is by the use of kinetic theory, see \citet[Chapter 7 and 9]{rao2008introduction} equation (7.75). 
To bridge the usual formulation of the equation for $k_p$ with the kinetic theory and our model, we remark that the term $\pSavg {\bm{\sigma}_d^0 \cdot \textbf{n}_d}$ takes in account both hydrodynamic forces and particle interaction forces. 
Consequently, the second term on the right hands side of \ref{eq:dt_hybrid_kp} can be decomposed into a contribution due to particle-particle interactions and a contribution due to particle fluid interactions, the former is the dissipation term of see \citet[Chapter 7 and 9]{rao2008introduction} equation (7.75). 
Also, a term written as the divergence of a stress is in fact included in kinetic theory, it is supposed to account for fluxes of granular agitation due to particle-particle elastic interactions. 
This terms can be recovered from the exchange term $\pavg{\textbf{u}_\alpha'\cdot\intS{\bm{\sigma}_f^0 \cdot \textbf{n}_d}}$ with a similar procedure than the derivation of the contact stress tensor, see \citet{scorsim2021particle}. 
Consequently, if we consider only particles-particles interaction terms uch as in \citet{rao2008introduction} we obtain consistent results. 
Notice that we did not make any hypothesis so far, consequently, \ref{eq:dt_hybrid_kp} itself is valid regardless of the particles nature and concentration.
The hypothesis made in kinetic theory are in fact needed to derive the closure for the exchange term, $\pavg{\textbf{u}_\alpha'\cdot\intS{\bm{\sigma}_f^0 \cdot \textbf{n}_d}}$. 

\subsection{The energy exchanges}

One can verify that summing \ref{eq:dt_hybrid_ep}, \ref{eq:dt_hybrid_Wp} and \ref{eq:dt_hybrid_kp} and \ref{eq:dt_hybrid_up2} makes \ref{eq:dt_hybrid_Ep}.  
Under this form it is easy to observe the exchange terms which drive the energy transfer between each component of the total energy. 
Firstly, the source term $\bm{\sigma}_p^\text{Re} :\grad \textbf{u}_p$ appear in \ref{eq:dt_hybrid_up2} and \ref{eq:dt_hybrid_k1} with opposite sign. 
Consequently, macroscopic kinetic energy is transmitted to granular agitation through the macroscopic diffusion scalar : $\bm{\sigma}_p^\text{Re} :\grad \textbf{u}_p$. 
Then between \ref{eq:dt_hybrid_Wp} and \ref{eq:dt_hybrid_ep} we already observed that the source terms is the dissipation term,  $\pOavg{\bm{\sigma}_d^0:\grad \textbf{u}_d^0}$.
However, note that no common term is present between \ref{eq:dt_hybrid_kp} and \ref{eq:dt_hybrid_Wp} which implies that there is no direct transfer of energy between the center of mass velocity fluctuation quantified by $k_p$ and the internal velocity fluctuation energy $W_p$. 
However, notice that the transport equation for $k_f$, \ref{eq:dt_hybrid_k1}, contains the terms $\pavg{\textbf{u}_\alpha' \intS{\bm{\sigma}_f^0 \cdot \textbf{n}_d}}$ and $\pSavg{\textbf{w}_d^0 \cdot \bm{\sigma}_f^0 \cdot \textbf{n}_d}$ which are also present in \ref{eq:dt_hybrid_kp} and \ref{eq:dt_hybrid_Wp}. 
Consequently, the energy transfer from granular agitation $k_p$ and the internal kinetic energy $W_p$ is done through the fluid phase pseudo turbulent kinetic energy. 
To summarize this quite complicated energy cascade between both phases and the different scales we propose the following diagram, see \ref{fig:energy}. 
\begin{figure}[h!]
    \centering
    \tikzstyle{quadri}=[rectangle,draw]
    \begin{tikzpicture}[scale=1.2]
        \node[quadri,fill=gray!10] (u2) at (0,0){$(u_p)^2 / 2$};
        \node[quadri,fill=gray!10] (kp) at (4,0){$k_p$};
        \node[quadri,fill=gray!10] (Wp) at (8,0){$W_p +s_p\gamma$};
        \node[quadri,fill=gray!10] (ep) at (12,0){$e_p$};
        \node[quadri,fill=gray!10] (u12)at (0,-3){$\frac{\rho_f}{2}(u_f)^2$};
        \node[quadri,fill=gray!10] (k1) at (6,-3){$k_f$};
        \node[quadri,fill=gray!10] (e1) at (10,-3){$e_f$};
        \draw[->] (u2)--(kp)node[midway,above]{\footnotesize $\bm{\sigma}^\text{eq}_p:\grad \textbf{u}_f$};
        % \draw[<->,text width=2cm] (kp)--(u12) node[midway,left]{\footnotesize $+  n_p v_p \textbf{u}_p \cdot 
        % (\rho_d \textbf{g} - \grad p_f)
        % + n_p \textbf{u}_p \cdot \textbf{f}_{pm} - \textbf{F}_\text{pfp}$};
        \draw[<->] (k1)--(u12) node[midway,above]{\footnotesize $\bm{\sigma}^\text{eq}_f:\grad \textbf{u}_f$}node[midway,below,sloped]{\footnotesize $\textbf{u}_f\cdot\pSavg{\bm{\sigma}_f^0\cdot \textbf{n}_d} $};
        \draw[<->] (k1)--(e1) node[midway,below]{\footnotesize $\avg{\chi_f \bm{\sigma}_f^0 : \grad \textbf{u}_f^0}$};
        \draw[<->,sloped] (k1)--(kp) node[midway,above]{\footnotesize $\pavg{ \textbf{u}_\alpha'\cdot \intS{\bm{\sigma}_f^0\cdot\textbf{n}_d}}$};
        \draw[<->] (k1)--(u2) node[midway,below,sloped]{\footnotesize $\textbf{u}_p\cdot \pSavg{\bm{\sigma}_f^0 \cdot \textbf{n}_f}$};
        \draw[<->,sloped] (k1)--(Wp) node[midway,below]{\footnotesize $\pSavg{{\textbf{w}_d^0 \cdot \bm{\sigma}_f^0\cdot \textbf{n}_f}}$};
        % \draw[->] (kp)--(Wp)node[midway,above]{$(\textbf{u}_\alpha' \cdot \textbf{f}_\alpha')_p$};
        \draw[->] (Wp)--(ep)node[midway,above]{\footnotesize $\pOavg{\bm{\sigma}_d^0 : \grad \textbf{u}_d^0}$};
        \draw (e1)--(ep)node[midway,above,sloped]{\footnotesize $\pSavg{\textbf{q}_f^0 \cdot \textbf{n}_d}$};
    \end{tikzpicture}
    \caption{Energy exchange between the different components of energy in a dispersed two phase flow.
    Macroscopic kinetic energy of the particle phase, $u_p^2/2$, and of the carrier fluid $u_f^2/2$.
    $k_f$, Pseudo turbulent energy of the carrier fluid. 
    $k_p$, Pseudo turbulent energy of particle center of mass. 
     }
    \label{fig:energy}
\end{figure}
% Consequently, the energy gain due to internal dissipation stress $\pOavg{\bm{\sigma}_d^0:\grad \textbf{u}_d^0}$ comes from the internal velocity fluctuation equation. 
In the literature, it is said that the transfer terms between internal energy $e_p$ and the granular temperature $k_p$ is the \textit{dissipation rate} due to inelastic particle-particle collision present in \ref{eq:dt_hybrid_up2}, see for example \citet{fox2014multiphase,rao2008introduction}. 
However, in light of \ref{fig:energy} the energy gain due to  $\pOavg{\bm{\sigma}_d^0:\grad \textbf{u}_d^0}$ which is the \textit{dissipation rate} has no reason to be equal to the energy loss in \ref{eq:dt_hybrid_up2} represented by the term $\pavg{\textbf{u}_\alpha' \intS{\bm{\sigma}_f^0 \cdot \textbf{n}_d}}$. 
In fact some energy is first transmitted to the fluid phase $k_p$, then some of this energy is transmitted to the internal kinetic energy $W_p$, which will induce viscous dissipation within the particle. 
In short, the internal kinetic energy is transformed into internal energy but by no means the \textit{dissipation rate} $\pOavg{\bm{\sigma}_d^0:\grad \textbf{u}_d^0}$ makes the link between to the granular temperature $k_p$ and the internal energy of the particle phase $e_p$. 

\subsection{The first order momentum and mass equations}

As it is suggested in the previous section, the needs for higher moments equations arise if one of the closure terms present in the previous set of equation is highly dependent on one of the moments of the particles. 
In our case we suppose that the second order description of the averaged shape, i.e. $\textbf{M}_p$, and a first order description of velocity distribution, i.e. $\textbf{P}_p$,  is enough to express all closure terms. 
By applying the average operator on \ref{eq:dt_M_alpha},\ref{eq:dt_S_alpha} and \ref{eq:dt_mu_alpha}, one get the second order moment of mass, and first order moment of momentum symmetric and skew symmetric parts, namely, 
\begin{align}
    \pddt \left(n_p \textbf{M}_p\right)
    + \div \left(
        n_p \textbf{u}_p \textbf{M}_p
    + \textbf{M}_p^\text{Re}
    \right)
    &=
    n_p2  \textbf{S}_p
    \label{eq:dt_hybrid_Mp}
    \\
    % \label{eq:dt_hybrid_Pp}
    % \pddt \left(n_p \textbf{P}_p\right)
    % + \div \left(
    %     n_p \textbf{u}_p \textbf{P}_p
    % + \textbf{P}_p^\text{Re}
    % \right)
    % &=
    % % -n_p v_p p_f \textbf{I}
    % % + n_p \textbf{F}_p
    % \pSavg{
    %     \textbf{r} \bm{\sigma}_f^0 \cdot\textbf{n}_d
    % }
    % + \pOavg{
    %     \rho_d \textbf{w}_d^0  \textbf{w}_d^0 
    %     - \bm{\sigma}_d'
    % }
    % -  \pSavg{\gamma (\textbf{I} - \textbf{nn})},\\
\label{eq:dt_hybrid_Sp}
\pddt \left(n_p \textbf{S}_p\right)
+ \div \left(
    n_p \textbf{u}_p \textbf{S}_p
+ \textbf{S}_p^\text{Re}
\right)
&=
% -n_p v_p p_f \textbf{I}
\pSavg{(\textbf{r}\bm\sigma_f^0+\bm\sigma_f^0\textbf{r})\cdot \textbf{n}_d}
% n_p  \mathscr{S}_p^*
% + \pSavg{
%     \textbf{r} \bm{\sigma}_f^0 \cdot\textbf{n}_d
% }
&+ \pOavg{
    \rho_d \textbf{w}_d^0  \textbf{w}_d^0 
    - \bm{\sigma}_d
}\\
&-  \pSavg{\gamma (\bm\delta - \textbf{nn})},
\\
\label{eq:dt_hybrid_mup}
\pddt \left(n_p \bm{\mu}_p\right)
+ \div \left(
n_p \textbf{u}_p \bm{\mu}_p
+ \bm{\mu}_p^\text{Re}
\right)
&=
\pSavg{\textbf{r}\times(\bm\sigma_f^0\cdot \textbf{n}_d)}
\end{align}
respectively, where we have defined the fluctuaiton terms as $
 \textbf{M}_p^\text{Re}
 = \pavg{\textbf{M}_\alpha' \textbf{u}_\alpha'} $,  $ 
 \textbf{S}_p^\text{Re}
 = \pavg{\textbf{P}_\alpha' \textbf{u}_\alpha'}$ and $ 
 \bm{\mu}_p^\text{Re}
 = \pavg{\bm{\mu}_\alpha' \textbf{u}_\alpha'}
$.
Note many comments can be made about these equations as they have been already treated in \ref{sec:Lagrangian}. 
However, it is interesting to discuss the case of infinitely rigid particles where the internal velocity is defined by $\textbf{w}_d^0 = \bm\Omega_\alpha \cdot \textbf{r}$ and the particle internal stress is undefined. 
In this case \ref{eq:dt_hybrid_Mp} and \ref{eq:dt_hybrid_mup} might be used to solve for the orientation and angular velocity of the particles. 
Therefore, \ref{eq:dt_hybrid_Sp} seems to be redundant and is in practice never used. 
In fact this equation can be used to determine the averaged stress, $\pOavg{\bm{\sigma}_d}$ in terms of the particles' kinetic properties. 
Knowing the internal stress of the particles, and the constitutive law of the solid material, one is able to compute the averaged deformation of the particle. 
Once the deformation is obtained, one is able to validate, or not, the assumption of infinitely rigid particles made at the origin. 

At this stage it is interesting to have a look at the different closures in the situation of stokes flows. 
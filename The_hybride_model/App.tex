
Now that we reached a clear understanding of the mathematical structures of the averaged two-phase flow equations we now expose the averaged set of equations which constitute the \textit{Hybrid model}. 
In this section we consider the simplifying assumption exposed in \ref{ap:hypothesis} for the interfaces equations. 
As mentioned in \ref{sec:two-fluid} we derive the mass, momentum and energy for the particles and continuous phase. 
Additionally, to describe the particle shape and inner velocity, one must consider the second moment of mass and first moment of momentum averaged equations. 
This, makes a total of 10 equations, 6 for the particle phase and 4 for the continuous phase.

To support the subsequent discussion, we provide the expressions of the closure terms for a dilute emulsion of spherical droplets. 
We will consider a monodisperse suspension of droplets with radius $a$ and viscosity $\mu_d$. 
Although much of this information is already known, it is useful for the purposes of understanding. 
Additionally, significant implications regarding the energy equation arise due to the consideration of fluid particles, even in the limit of Stokes flows. 
Even though the closures are expressed in the Stokes limit, note that the set of equations provided remains valid regardless of the flow regime.

\subsection{Continuous phase equations}

The equations for the carrier fluid are basically the same as in the classic two-fluid model derived in \ref{ap:two-fluid_model}, except that the interfacial terms of the form $\avg{\delta_I \ldots }$ need to be reformulated.
Indeed, those terms must be reformulated in terms of particle-averaged quantities in order to be consistent with the particle-phase equations \citep{jackson1997locally,zhang1994averaged}. 
This is achieved through the use of \ref{eq:f_exp} which enables us to convert the exchange terms appearing in \ref{eq:avg_dt_chi_f} into a series expansion of particle phase quantities. 
For clarity, we only retain the first order terms of this expansion, however understand that the higher order moments are not necessarily negligible. 

\subsubsection*{Primary fluid phase equations}

The continuous phase averaged mass, momentum and total energy equations yield, 
\begin{align}
    \label{eq:dt_hybrid_rho}
    &\pddt (\phi_f \rho_f)  
    + \div (
        \phi_f \rho_f\textbf{u}_f
    )
    = 
    0,\\
    \label{eq:dt_hybrid_rhou_f}
    &\pddt (\phi_f \rho_f\textbf{u}_f)  
    + \div (
        \phi_f \rho_f\textbf{u}_f\textbf{u}_f
        + \bm{\sigma}_f^\text{eq}
    )
    = 
    \phi_f \rho_f \textbf{g} 
    - \pSavg{{\bm{\sigma}_f^0 \cdot \textbf{n}_d}},
    % +\div  \pSavg{{\textbf{r}\bm{\sigma}_f^0 \cdot \textbf{n}_d}}
    \\
    \label{eq:dt_hybrid_rhoE_f}
    &\pddt (\phi_f\rho_fE_f)  
    + \div (
        \phi_f\rho_fE_f\textbf{u}_f
        + \bm{q}_f^\text{eq}
        + \textbf{u}_f \cdot \bm{\sigma}_f^\text{eq}
        % - \textbf{u}_f^0 \cdot \bm{\sigma}_f^0 
        % + \textbf{q}_f^0
        )
    = 
    \phi_f \rho_f\textbf{u}_f \cdot \textbf{g} 
    - \textbf{u}_p \cdot \pSavg{{\bm{\sigma}_f^0 \cdot \textbf{n}_d}}\nonumber \\
    &- \pavg{ \textbf{u}_\alpha' \cdot \intS{  \bm{\sigma}_f^0 \cdot \textbf{n}_d}}
    - \pavg{ \intS{\textbf{w}_d^0 \cdot \bm{\sigma}_f^0 \cdot \textbf{n}_d}}
    + \pSavg{{\textbf{q}_f\cdot \textbf{n}_d}},
    % &\div [    
        % \textbf{u}_p \cdot \pSavg{{ \textbf{r}\bm{\sigma}_f^0 \cdot \textbf{n}_d}}
    % + \pavg{ \textbf{u}_\alpha' \cdot \intS{ \textbf{r} \bm{\sigma}_f^0 \cdot \textbf{n}_d}}
    % + \pavg{ \intS{\textbf{r}\textbf{w}_d^0 \cdot \bm{\sigma}_f^0 \cdot \textbf{n}_d}}
    % - \pavg{ \intS{\textbf{r}  \textbf{q}_f^0 \cdot \textbf{n}_d}}
    % ]
\end{align} 
respectively. 
Where we have introduced the equivalent stress tensor $\bm{\sigma}_f^\text{eq}$ and equivalent energy flux $\textbf{q}^\text{eq}_f$ as,
\begin{align}
    \label{eq:sigma_eq_def}
    \bm{\sigma}_f^\text{eq}
    =& 
    \avg{\chi_f\rho_f\textbf{u}_f'\textbf{u}_f'}
    - \phi_f \bm{\sigma}_f%- n_p \textbf{M}_p
    - \pSavg{{\textbf{r}\bm{\sigma}_f^0 \cdot \textbf{n}_d}}\\
    \textbf{q}_f^\text{eq}
    =&\textbf{q}_f^\text{e} +\textbf{q}_f^\text{k}  \nonumber\\
    \textbf{q}_f^\text{e}
    =& \rho_f \avg{\chi_f \textbf{u}_f' e_f'} 
    + \phi_f\textbf{q}_f 
    +\pSavg{{\textbf{r}\textbf{q}_f^0 \cdot \textbf{n}_d}} 
    \nonumber\\
    \textbf{q}_f^\text{k}
    =& \rho_f \avg{\chi_f \textbf{u}_f' k_f} 
    - \avg{\chi_f \textbf{u}_f' \cdot \bm{\sigma}_f^0}
    + (\textbf{u}_f - \textbf{u}_p)\cdot
    \pSavg{{\textbf{r}\bm{\sigma}_f^0 \cdot \textbf{n}_d}}
    \nonumber\\\nonumber&
    - \pavg{ \textbf{u}_\alpha' \cdot \intS{ \textbf{r} \bm{\sigma}_f^0 \cdot \textbf{n}_d}}
    - \pavg{ \intS{\textbf{r}\textbf{w}_d^0 \cdot \bm{\sigma}_f^0 \cdot \textbf{n}_d}}
\end{align}
It is clear that those equations yield essentially the same as the previous set of equations presented in \ref{ap:two-fluid_model}.
The only difference is the presence of additional terms inside $\bm{\sigma}^\text{eq}_f$ and $\textbf{q}^\text{eq}_f$ due to the expansion of the interfacial terms. 

The averaged continuous-phase momentum balance \eqref{eq:dt_hybrid_rhou_f} under its \textit{hybrid} form was established long ago by \citet{zhang1997momentum,jackson1997locally}.  
\ref{eq:dt_hybrid_rhou_f} is of course consistent with the formulation given by the named authors.
Let us now focus on the exchange terms, the other closures such as the Reynolds stress tensor $\avg{\chi_f\rho_f\textbf{u}_f'\textbf{u}_f'}$ are discussed in\ref{ap:two-fluid_model}.  
For purpose of understanding, we have re-derived in \ref{ap:Translating_sphere}, the closure terms of \ref{eq:dt_hybrid_rhou_f} for dilute suspension of spherical droplets. 
Specifically, we consider an isolated spherical non-rotating droplet of viscosity $\lambda \mu_f$ immersed in an arbitrary linear flow. 
It is found that the first three moment of the hydrodynamic forces are related to the mean fluid phase velocity field as, 
\begin{align}
    \pSavg{\bm{\sigma}_f^0\cdot \textbf{n}_d} &= 
    \phi_d \div\bm\sigma_f
    + \frac{3\phi_d\mu_f}{2 a^2} 
    \left(\frac{3\lambda+2}{\lambda+1}\right) \textbf{u}_{f p} 
    + \frac{3\phi_d\mu_f}{4} \left(\frac{\lambda}{\lambda+1}\right)\grad^2\textbf{u}_f\\
    \label{eq:first_mom}
    \pavg{\intS{\textbf{r}\bm{\sigma}_f^0 \cdot \textbf{n}_d}} 
    &= 
    \phi_d \bm\sigma_f + 
    \frac{3}{5}\mu_f \phi_d \left(\frac{2+5\lambda}{1+\lambda}\right)
    \textbf{E}_f
    \\
    \label{eq:second_mom}
        \pavg{\intS{(\bm{\sigma}_f^0 \cdot \textbf{n}_d)_ir_kr_l}} &=
        n_pv_p  \frac{a^2}{5} 3 [(\div \bm\sigma_f)\bm\delta]^\text{sym}
        + \frac{3\mu_f\phi_d}{2}\left(\frac{\lambda}{\lambda+1}\right)u_{fp,i}\delta_{kl}\\
        &+ \frac{3\mu_f\phi_d}{5}\left(\frac{1}{\lambda+1}\right)(u_{fp,i}\delta_{kl}+ u_{fp,k}\delta_{il}+u_{fp,l}\delta_{ki})\nonumber
\end{align}
where $\textbf{u}_{fp} = \textbf{u}_f - \textbf{u}_p$ and $\textbf{E}_f = \frac{1}{2}\left[\grad \textbf{u}_f + (\grad \textbf{u}_f)^\dagger\right]$. 
In our formulation the term $\pSavg{\bm{\sigma}_f^0 \cdot \textbf{n}_d}$ represents the total components of the interphase drag force.
Specifically, the first term is the contribution from the mean fluid phase stress $\bm\sigma_f$, the second term is the Hadamard-Rybczynski force and the last is the Faxen contribution. 
Likewise, $\pSavg{\textbf{r}\bm{\sigma}_f^0 \cdot \textbf{n}_d}$ is the total averaged first moment of force traction, which includes the mean fluid phase stress. 
Similar consideration hold for the second moment of the force traction. 
\ref{eq:first_mom} is in agreement with the finding of \citet[Appendix A]{zhang1997momentum}.
Note that the tensor $\pSavg{\textbf{r}\bm{\sigma}_f^0 \cdot \textbf{n}_d}$ is responsible for the famous Einstein viscosity correction which is valid in the stokes flow regime for solid spherical inclusion.
Therefore, this term is of upmost importance in the averaged momentum equations and is non-negligible in most of the flow conditions, if not all of them.
The second moment of the hydrodynamic forces $\pSavg{\textbf{rr}\bm{\sigma}_f^0 \cdot \textbf{n}_d}$, is shown to be non-negligible at first order in $\phi$ \citep{jackson1997locally,zhang1997momentum}. 
Thus, these moments are non-negligible in the Stokes and dilute regime. 
Consequently, it is likely that they are also non-negligible in most of the flow situation. 
Even though this is known since \citet{jackson1997locally} and \citet{zhang1997momentum} it is surprising that most of the studies in the literature regarding Euler-Euler equations neglected these moments. 


Now, let us discuss the continuous-phase averaged total energy balance \eqref{eq:dt_hybrid_rhoE_f}. 
Most of the terms have already been addressed in \ref{ap:two-fluid_model}, so for now, let's direct our attention to the exchange terms. 
On the right-hand side of \ref{eq:dt_hybrid_rhoE_f} we identify four exchange terms.
Indeed, after taking the Taylor expansion of the interfacial term $\avg{\delta_I (\textbf{u}^0_d \cdot \bm{\sigma}_f^0 \cdot \textbf{n}_d)}$ on the right-hand side of \ref{eq:dt_avg_rhoE_k}, we used the following decomposition on each of the moments:
\begin{align}
    \label{eq:exergysource}
    \pavg{ \intS{\textbf{u}^0_d \cdot \bm{\sigma}_f^0 \cdot \textbf{n}_d}}
    &= 
    \textbf{u}_p \cdot \pSavg{{\bm{\sigma}_f^0 \cdot \textbf{n}_d}}
    + \pavg{ \textbf{u}_\alpha' \cdot \intS{  \bm{\sigma}_f^0 \cdot \textbf{n}_d}}
    + \pavg{ \intS{\textbf{w}_d^0 \cdot \bm{\sigma}_f^0 \cdot \textbf{n}_d}},
%     \label{eq:exergysource2}
%     \pavg{ \intS{\textbf{r}\textbf{u}^0_d \cdot \bm{\sigma}_f^0 \cdot \textbf{n}_d}}
%    &= 
%     \textbf{u}_p \cdot \pSavg{{\bm{\sigma}_f^0 \cdot \textbf{n}_d}}
%     + \pavg{ \textbf{u}_\alpha' \cdot \intS{\textbf{r}  \bm{\sigma}_f^0 \cdot \textbf{n}_d}}
%     + \pavg{ \intS{\textbf{r}\textbf{w}_d^0 \cdot \bm{\sigma}_f^0 \cdot \textbf{n}_d}},
\end{align}
where we have noticed that $\textbf{u}_d^0 = \textbf{u}_p + \textbf{u}_\alpha' +\textbf{w}_d^0$ according to \ref{eq:def_fluc_p} and to the definition of the \textit{inner velocity} of a particle $\textbf{w}_d^0$. 
In this form the contribution to the kinetic energy exchange is now clear. 
The first term on the right hands side of \ref{eq:exergysource} represents the work done by the mean particle-phase motion with the mean drag force.
The second term is the covariance term of the velocity of the particles with their respective drag forces.
Note that in a dilute suspension the drag force applied on each particle is likely to be a function of its instantaneous velocity, such as in \ref{eq:first_mom}, thus in a general manner this term is non-negligible. 
The last term represents the work made by the local force traction on the particle surface with the velocity at the surface of the particles $\textbf{w}_d^0$.
Regarding the higher order moment of kinetic energy exchange same comments can be made except that these terms act as energy fluxes instead of sources. 
The relative importance of these three contribution depends highly on the particles' nature. 
To our knowledge, such a decomposition is not present in the literature except in \citep[Chapter 2]{scorsim2021particle} where they make similar consideration, but for solid spherical particles.
We recall that the stress integral $\pSavg{\bm{\sigma}_f^0 \cdot \textbf{n}_d}$ contains contact forces as well, making our model consistent with the latter study. 
Our formulation is therefor more general. 


\subsubsection*{Secondary energy equations}

The continuous phase averaged total energy can be further decomposed into three energy components (see \ref{ap:two-fluid_model}), that is,  
\begin{align}
    E_k = e_k + k_k + u_k^2/2
    \label{eq:E_def}
\end{align}
where $k_k$ is the pseudo-turbulent kinetic energy defined as, $\phi_k k_k = \frac{1}{2}\avg{\chi_k \textbf{u}_k'\cdot \textbf{u}_k'}$. 
Therefore, to fully describe the averaged total energy of the continuous phase one must add at least a supplementary equation, either for $k_k$ or $e_k$. 
The derivation of these equations is explained in \ref{ap:two-fluid_model} in the context of the two-fluid formulation (see \ref{eq:dt_avg_uk2}, \ref{eq:dt_avg_kk} and \ref{eq:dt_avg_ek}) . 
Under the hybrid formulation, the kinetic energy, pseudo turbulent energy and internal energy equations read as,
\begin{align}
    \pddt (\phi_f \rho_fu_f^2/2)  
    + \div (
        \phi_f \rho_f\textbf{u}_fu_f^2/2
        + \textbf{u}_f \cdot \bm{\sigma}_f^\text{eq}
    )
    = 
    \phi_f \rho_f \textbf{u}_f\cdot \textbf{g} 
    + \bm{\sigma}_f^\text{eq} : \grad \textbf{u}_f
    -  \textbf{u}_f\cdot 
        \pSavg{{\bm{\sigma}_f^0 \cdot \textbf{n}_d}},
        \\
    \label{eq:dt_hybrid_k1}
    \pddt (\phi_f\rho_fk_f)  
    + \div (
        \phi_f\rho_fk_f\textbf{u}_f
        + \textbf{q}_f^\text{k} 
        )
    = 
    - \avg{\chi_f\bm{\sigma}_f^0 : \grad \textbf{u}_f^0}
    - \bm{\sigma}_f^\text{eq} : \grad \textbf{u}_f\nonumber
    - \pavg{ \textbf{u}_\alpha' \cdot \intS{  \bm{\sigma}_f^0 \cdot \textbf{n}_d}}\\
    + (\textbf{u}_f - \textbf{u}_p)\cdot \pSavg{{\bm{\sigma}_f^0 \cdot \textbf{n}_d}} 
    - \pavg{ \intS{\textbf{w}_d^0 \cdot \bm{\sigma}_f^0 \cdot \textbf{n}_d}},
    \\
    \label{eq:dt_hybrid_e1}
    \pddt (\phi_f\rho_fe_f)  
    + \div (
        \phi_f \rho_fe_f\textbf{u}_f
        +
        \textbf{q}_f^\text{e} 
        )
    = 
    \avg{\chi_f\bm{\sigma}_f^0 : \grad \textbf{u}_f^0}
    + \pSavg{{\textbf{q}_f^0 \cdot \textbf{n}_d}},
\end{align}
respectively. 
One can verify that summing these three equations gives back \ref{eq:dt_hybrid_rhoE_f}. 
It is interesting to see that under this form the equivalent flux of the mean kinetic energy, $\textbf{u}_f \cdot \bm{\sigma}_f^\text{eq}$, is in fact equal to the mean velocity $\textbf{u}_f$ dotted with the equivalent stress of the averaged momentum equation $\bm{\sigma}_f^\text{eq}$. 
From \ref{eq:dt_hybrid_k1} we can infer that the pseudo turbulent energy $k_f$ in a dispersed two phase flow is generated by five different sources. 
The first term on the right-hand side of \ref{eq:dt_hybrid_k1} corresponds to the local scale energy dissipation that acts as a source term in the internal equation \eqref{eq:dt_hybrid_e1}. 
The second term corresponds to the macroscopic dissipation term. 
And the three other exchange terms correspond to the generation of work made by the particles through the three possible way described by \ref{eq:exergysource}. 
The averaged pseudo-turbulent energy balance \eqref{eq:dt_hybrid_k1} is consistent with those in former studies \citep[Chapter 7]{morel2015mathematical}\citep[Chapter 2]{scorsim2021particle}\citet{kataoka1989basic}. 
However, the decomposition of the exchange term is not present, and the expression of $\textbf{q}_f^k$, which collects the first moments of the work, has not been discussed in the literature in such generality.
The energy exchange between the macroscopic, microscopic, internal energy as well as the energy exchange between both phases, will be addressed later on.   

Let us focus on the last term of \ref{eq:exergysource}.
As mentioned above this term represents the source of pseudo turbulent energy in the fluid phase due to the presence of the particles. 
As a first approximation one might consider that every particle in the flow posses the surface velocity of an isolated particle in an arbitrary linear flow stokes flow with relative velocity  $\textbf{u}_{fp}$ and mean shear $\textbf{E}_f$.
In this case the \textit{inner velocity} evaluated at the surface of the particle $\alpha$ can be written as (see \ref{ap:Translating_sphere})
\begin{equation*}
    \textbf{w}_d^0 (a \textbf{n})
    = \left(\frac{\lambda + \frac{1}{2}}{\lambda +1} - 1\right)
    \textbf{u}_{fp} 
    + 
    \frac{1}{2}\left(\frac{1}{\lambda +1}\right)
    \textbf{rr} \cdot \textbf{u}_{fp} 
    + \left[1-\frac{\lambda}{(\lambda + 1)}\right]\textbf{E}_f\cdot\textbf{r}
    -\frac{1}{a^2}\left(\frac{1}{\lambda +1 } \right) \textbf{r} \textbf{E}_f:\textbf{rr}. 
\end{equation*}
The contributions of the velocity field factor of $\textbf{u}_{fp}$ represents the famous hill's vortex reticulation inside a spherical droplet. 
The second contribution is the motion generated due to a mean shear flow. 
Injecting this expression into the Last term on the right-hand side of \ref{eq:exergysource} gives, 
\begin{multline}
    \pSavg{\textbf{w}_2^0 \cdot \bm{\sigma}_1^0\cdot\textbf{n}_2}
    =  
    \frac{1}{\lambda+1}\textbf{u}_{fp} \cdot \left[
        \left(\frac{\lambda +1 }{2}\right)\pSavg{ \bm{\sigma}_1^0\cdot\textbf{n}_2}
        % + \pavg{\textbf{u}_{\alpha}' \cdot \intS{ \bm{\sigma}_1^0\cdot\textbf{n}_2} }
        + \frac{1}{2a^2}
        \pSavg{\textbf{rr}\cdot \bm{\sigma}_1^0\cdot\textbf{n}_2}
    \right]
    \\
    % \left[
    %     \textbf{u}_{p f} \cdot
    %     +
        % \pavg{\textbf{u}_{\alpha}' \cdot \intS{\textbf{rr}\cdot \bm{\sigma}_1^0\cdot\textbf{n}_2}}
    % \right]
    + \frac{1}{\lambda + 1} \textbf{E}_{f} : \left[ 
         \pSavg{\textbf{r} \bm{\sigma}_1^0\cdot\textbf{n}_2}
         -\frac{1}{a^2} 
         \pSavg{ \textbf{rrr} \cdot \bm{\sigma}_1^0\cdot\textbf{n}_2}
         \right]
    \label{eq:energy_term}
\end{multline}
In this expression we clearly identify the zero, first and second order moments of the surface force traction provided by \ref{eq:first_mom}. 
Additionally, notice that we did not consider rotation of droplets consequently the mean fluid vorticity nor the particles angular velocity appears in this expression. 
One can also notice that taking the limit $\lambda \to \infty$ gives zero for \ref{eq:energy_term}. 
Which is consistent since $\textbf{w}_d^0 = 0$ for non-rotating solid particles. 
Consequently, in light of \ref{eq:energy_term} the work generated due to the local motion at the surface of a spherical droplet, namely  $\pSavg{\textbf{w}_2^0 \cdot \bm{\sigma}_1^0\cdot\textbf{n}_2}$ is either due to its relative motion with the continuous phase  or due to its simple presence in a shear flow. 
Indeed, in both cases motion at the surface of the particle is observed and stresses is also generated, which in turns contribute the generation of pseudo turbulence. 
In fact if we add the contribution given by \ref{eq:energy_term} in  \ref{eq:dt_hybrid_k1} we observe that  the consideration of hill's vortexes add the coefficient $\frac{\lambda +\frac{1}{2}}{\lambda+1}$ in front of the drag force velocity terms in \ref{eq:dt_hybrid_k1}.
Besides, the first moments of surface traction forces appearing in the diffusive equivalent flux $\textbf{q}_1^k$ are also subject to these comments.  
Consequently, the consideration of hill's vortex end up to add a coefficient in front of this exchange term which varies from $1$ to $1/2$ for respectively, solid particles and bubbles.  
Same comments can be made regarding the consideration of the droplets internal motion to the higher moments present in the flux term $\textbf{q}^k_f$. 

As a matter of fact the consideration of the internal motion of particles such as hill's vortex have a very significant impact regarding the magnitude of the pseudo turbulent exchange terms, especially when one is considering bubbly flow. 
The physical explanation of the decrease of the coefficient in front of the exchange terms for bubbles, can be due to the facts that the fluid slip on the bubbles or droplet's surface induce less work exchange than if the fluid followed the particle's surface as it is the case for solid particles. 



\subsection{Dispersed phase equations}

% Now, we turn our attention to the particle phase equations. 
% \tb{Do i put the surface in or out}
% \subsubsection{Primary equations}

By applying the ensemble average on the particle phase equations, namely \ref{eq:dt_m_alpha}, \ref{eq:dt_p_alpha} and \ref{eq:dt_E_alpha_tot} we obtain the particle-phase averaged mass, momentum and energy equations, namely, 
\begin{align}
    \label{eq:dt_hybrid_mp}
    \pddt \left(n_p m_p\right)
    + \div \left(n_pm_p\textbf{u}_p
    \right)
    = 
    0\\
    \label{eq:dt_hybrid_up}
    \pddt \left(n_p m_p \textbf{u}_p\right)
    + \div \left(n_p
    m_p \textbf{u}_p \textbf{u}_p 
    + \bm{\sigma}_p^\text{eq}
    \right)
    = 
    n_p m_p \textbf{g}
    + \pSavg{{\bm{\sigma}_f^0 \cdot \textbf{n}_d}},\\
    \label{eq:dt_hybrid_Ep}
    \pddt(m_p n_pE_p^\text{tot})
    + \div(m_pn_p E_p^\text{tot} \textbf{u}_p 
    + \textbf{q}_p^\text{eq} 
    + \textbf{u}_p \cdot \bm{\sigma}_p^\text{eq})
    =  n_p m_p \textbf{u}_p\cdot  \textbf{g}
    % +  n_p ( \textbf{u}'_f \cdot \bm{\sigma}_f^0 \cdot \textbf{n}_d)_p^\Sigma
    -  \pSavg{\textbf{q}_f^0 \cdot \textbf{n}_d}\nonumber\\
    + \textbf{u}_p \cdot\pSavg{{\bm{\sigma}_f^0 \cdot \textbf{n}_d}}
    + \pavg{\textbf{u}_\alpha' \cdot\intS{\bm{\sigma}_f^0 \cdot \textbf{n}_d}}
    + \pSavg{{\textbf{w}_d^0 \cdot\bm{\sigma}_f^0 \cdot \textbf{n}_d}}
\end{align}
where we have defined, 
\begin{align*}
    &\bm{\sigma}_p^\text{eq}
    =  m_p\pavg{\textbf{u}_\alpha'\textbf{u}_\alpha'}
    &\textbf{q}_p^\text{eq}
    =\textbf{q}_p^\text{e} 
    +\textbf{q}_p^\text{k}  
    +\textbf{q}_p^\text{w}  
    \\
    &\textbf{q}_f^\text{e}
    = m_p \pavg{\textbf{u}_\alpha' e_\alpha'} 
    &\textbf{q}_p^\text{k}
    = m_p \pavg{\textbf{u}_\alpha' k_\alpha} 
    \\
    &\textbf{q}_p^\text{w}
    = 
     \pavg{\textbf{u}_\alpha'W_\alpha'}
    + \pavg{\textbf{u}_\alpha' s_\alpha' \gamma}.
\end{align*}
Where we have introduced the averaged internal kinetic energy with $n_pW_p = \pOavg{{\rho_d  (w_d^0)^2/2}}$. 
We recognize that these equations all posses the same exchange terms appearing in the fluid phase averaged equations but with opposite sign. 
However, note that in opposition to the fluid phase averaged equations, the first order moments do not appear inside the fluxes of the particles equations. 
Consequently, under this form only the fluctuating quantities plays the role of dissipative fluxes. 
However, it is noteworthy to mentions that for example the term, $ \pSavg{{\bm{\sigma}_f^0 \cdot \textbf{n}_d}}$ can be reformulated in certain situation a mean drag force term plus a divergence of a stress, the latter represent particles-particles contact forces, \citet{jackson1997locally,zhang1997momentum,nott2011suspension,zhang2021ensemble}. 
Likewise, in some recent models it is possible to expands the momentum exchange terms, as the sum of a \textit{binary force} and the divergence of a stress accounting for particles' long range interaction forces \citep{zhang2021ensemble,nott2011suspension}. 
In opposition to the contact stress this long range interaction stress, appears on the particle and carrier fluid momentum conservation equation. 
Even though, the latter stresses have been shown to be indispensable to ensure the hypertonicity of the two phase flow equations\citep{fox2020hyperbolic}, we choose to not explicitly display this stresses for succinctness. 

% \subsubsection{Secondary equations}

The particle-phase averaged total energy can also be decomposed into five different contributions, it yields 
\begin{equation*}
    n_p m_p E_p^\text{tot}(t) 
    = m_p n_p e_p 
    + n_p W_p
    + n_p s_p \gamma
    + m_p n_p k_p
    + m_p n_p (u_p)^2/2. 
    \label{eq:E_p_def}
\end{equation*}
Each of these terms represent: 
the mean particle's internal energy $e_p$; 
the averaged particle's internal kinetic energy $W_p$;
the averaged particle's surface energy $n_p s_p \gamma$;
the granular temperature $n_p k_p =\pavg{\textbf{u}_\alpha \cdot\textbf{u}_\alpha}/2$;
and the kinetic energy of the mean particle phase velocity. 
If one wish to solve for every component of the energy it is therefore needed to derive two supplementary equation. 
In \ref{ap:particles_eq} we have demonstrated how to derive the secondary equations for the energy of a single particle, see  \ref{eq:dt_e_alpha}, \ref{eq:dt_w2_alpha} and \ref{eq:dt_u2_alpha}. 
Thus, applying the average procedure on these equations one obtains, the particle averaged kinetic energy, internal kinetic energy and internal energy equations, namely,
\begin{align}
    % &\pddt \left(n_p m_p u_p^2/ 2\right)
    % + \div \left(n_p
    % m_p u_p^2/ 2 \textbf{u}_p 
    % + \textbf{u}_p \cdot \bm{\sigma}_p^\text{eq}
    % \right)
    % = 
    % + \bm{\sigma}_p^\text{eq}  :\grad \textbf{u}_p
    % +  n_p v_p \textbf{u}_p \cdot 
    % \rho_d \textbf{g}
    % + n_p \textbf{u}_p \cdot (\bm{\sigma}_f^0 \cdot \textbf{n}_d)^\Sigma_p,\\
    \label{eq:dt_hybrid_u2p}
    \pddt \left(\pavg{m_\alpha u_\alpha^2/2}\right)
    + \div \left(\pavg{m_\alpha u_\alpha^2/2} \textbf{u}_p 
    + \textbf{q}^k_p
    + \textbf{u}_p \cdot \bm{\sigma}_p^\text{eq}
    \right)
    = 
    n_p m_p \textbf{u}_p \cdot
    \textbf{g}\nonumber\\
    + \textbf{u}_p\cdot\pSavg{{\bm{\sigma}_f^0 \cdot \textbf{n}_d}}
    + \pavg{\textbf{u}_\alpha'\cdot\intS{\bm{\sigma}_f^0 \cdot \textbf{n}_d}}
    \\
    \label{eq:dt_hybrid_Wp}
    \pddt \left(n_p (W_p + s_p\gamma)\right)
    + \div 
    (n_p (W_p + \gamma s_p)
    \textbf{u}_p 
    +  \textbf{q}_p^\text{w}
    )
    = 
    - \pOavg{{\bm{\sigma}_d^0 : \grad\textbf{u}_d^0}}
    + \pSavg{{\textbf{w}_d^0 \cdot \bm{\sigma}_f^0 \cdot  \textbf{n}_d}}
    % - \pavg{\dot{ s_\alpha}}
    \\
    \pddt \left(n_p m_p e_p\right)
    + \div \left(n_p
    m_p e_p \textbf{u}_p 
    +  \textbf{q}_p^\text{e}
    \right)
    = 
    \pOavg{{\bm{\sigma}_d^0 : \grad\textbf{u}_d^0}}
    - \pSavg{{\textbf{q}_f^0\cdot \textbf{n}_d}}
    \label{eq:dt_hybrid_ep}
\end{align}
The center of mass kinetic energy can be further decomposed such as $\pavg{u_\alpha^2}/2 = n_p k_p + n_p u_p^2/2$. 
Then, to derive an equation for $k_p$ one must retrieve to \ref{eq:dt_hybrid_u2p} the dot product of \ref{eq:dt_hybrid_up} with $\textbf{u}_p$, which yields an equation for the mean kinetic energy and another for the granular temperature $k_p$, namely,
\begin{align}
    \label{eq:dt_hybrid_up2}
\pddt \left(n_p m_p u_p^2/ 2\right)
    + \div \left(n_p
    m_p u_p^2/ 2 \textbf{u}_p 
    + \textbf{u}_p \cdot \bm{\sigma}_p^\text{eq}
    \right)
    = 
    \bm{\sigma}_p^\text{eq}  :\grad \textbf{u}_p
    +  n_p m_p \textbf{u}_p \cdot 
     \textbf{g}
    + \textbf{u}_p \cdot \pSavg{{\bm{\sigma}_f^0 \cdot \textbf{n}_d}},\\
    \label{eq:dt_hybrid_kp}
    \pddt \left(n_p m_p k_p\right)
    + \div \left(n_p
    m_p k_p \textbf{u}_p 
    + \textbf{q}^k_p
    % + \textbf{u}_p \cdot \bm{\sigma}_p^\text{eq}
    \right)
    = 
    - \bm{\sigma}_p^\text{eq}  :\grad \textbf{u}_p
    + \pavg{\textbf{u}_\alpha'\cdot\intS{\bm{\sigma}_f^0 \cdot \textbf{n}_d}},
\end{align}
respectively.
\ref{eq:dt_hybrid_Wp}, \ref{eq:dt_hybrid_ep} and \ref{eq:dt_hybrid_up2} are discussed in \ref{ap:particles_eq} under a non-averaged form.
The only difference with the non averaged form is the presence of the equivalent fluxes, $\textbf{q}_p^e$, $\textbf{q}_p^w$ and $\bm\sigma^\text{eq}_p$ which contain the covariance tensors. 
Additionally, one can verify that summing \ref{eq:dt_hybrid_ep}, \ref{eq:dt_hybrid_Wp} and \ref{eq:dt_hybrid_kp} and \ref{eq:dt_hybrid_up2} makes \ref{eq:dt_hybrid_Ep}.  


Most of the closure terms present in these expressions also appear in the continuous phase. 
They have therefore already been discussed. 
The only exception is the particle phase dissipation term (first term on the right-hand side of \ref{eq:dt_hybrid_Wp}) which appears only in the particle-phase equations. 
It is interesting to mention that in the context of unreformable sphere in stokes flows the internal energy term as well as the internal dissipation can in fact be computed directly in terms of the continuous phase unknowns. 
The details of the calculation is given in \ref{ap:Translating_sphere} the result yields, 
\begin{align}
    \label{eq:stokes_Wp}
    W_p =  \frac{\rho_d \phi_d}{24 (\lambda +1)^2}
    (\textbf{u}_{pf} \cdot \textbf{u}_{pf})
    + \frac{\rho_d \phi_d}{30(\lambda+1)^2}
    \textbf{E}_f:\textbf{E}_f    \\
    \pOavg{\bm{\sigma}_2^0:\grad \textbf{u}_2^0}
    % = 2\mu_2 \intO{\textbf{e}_2^0: \textbf{e}_2^0 }
    = 
    \frac{3 \phi_d \mu_f \lambda}{(\lambda+1)^2}\textbf{E}_f:\textbf{E}_f
    + \frac{6 \phi_d \mu_f \lambda}{a^2(1+\lambda)^2}
    (\textbf{u}_{pf}\cdot \textbf{u}_{pf})
    \label{eq:diff_d}
\end{align}
From \ref{eq:stokes_Wp} we deduce that the relative motion between phases $\textbf{u}_{pf}$, as well as the mean gradient of the flow $\textbf{E}_f$ induce an inner circulation inside the droplets. 
As can be seen under this hypothesis the internal energy is not an unknown anymore, since it is entirely determined by the fluid phase unknown and $\phi_d$. 
Nevertheless, it is still non-zero at finite Reynolds number and therefor it must be considered in \ref{eq:E_p_def}. 
The dissipation term given by \ref{eq:diff_d} represents the energy dissipated into heat inside the particles. 
Due to the finite value of $\mu_d$ this term remains non-null. 
Since the motion inside the particles is directly determined by $\textbf{u}_{pf}$ and $\textbf{E}_f$ the dissipation rate equally. 


Now let us focus on the equation for granular temperature $k_p$ \eqref{eq:dt_hybrid_kp}. 
The usual way to derive the granular temperature equations is by the use of Louisville equations, see \citet[Chapter 7 and 9]{rao2008introduction} equation (7.75). 
To bridge the usual formulation of the equation for $k_p$ with the kinetic theory and our model, we remark that the term $\pSavg {\bm{\sigma}_d^0 \cdot \textbf{n}_d}$ takes in account both hydrodynamic forces and particle interaction forces. 
Consequently, the second term on the right hands side of \ref{eq:dt_hybrid_kp} can be decomposed into a contribution due to particle-particle interactions and a contribution due to particle fluid interactions, the former is the dissipation term of see \citet[Chapter 7 and 9]{rao2008introduction} equation (7.75). 
Also, a term written as the divergence of a stress is in fact included in kinetic theory, it is supposed to account for fluxes of granular agitation due to particle-particle elastic interactions. 
This terms can be recovered from the exchange term $\pavg{\textbf{u}_\alpha'\cdot\intS{\bm{\sigma}_f^0 \cdot \textbf{n}_d}}$ with a similar procedure than the derivation of the contact stress tensor, see \citet{scorsim2021particle}. 
Consequently, if we consider only particles-particles interaction term such as in \citet{rao2008introduction} we obtain consistent results. 
Notice that we did not make any hypothesis so far, consequently, \ref{eq:dt_hybrid_kp} itself is valid regardless of the particles nature and concentration.
The hypothesis made in kinetic theory are in fact needed to derive the closure for the exchange term, $\pavg{\textbf{u}_\alpha'\cdot\intS{\bm{\sigma}_f^0 \cdot \textbf{n}_d}}$. 


\subsection{The energy exchanges}

Under this form it is easy to observe the exchange terms which drive the energy transfer between each component of the total energy. 
Firstly, the source term $\bm{\sigma}_p^\text{Re} :\grad \textbf{u}_p$ appear in \ref{eq:dt_hybrid_up2} and \ref{eq:dt_hybrid_k1} with opposite sign. 
Consequently, macroscopic kinetic energy is transmitted to granular agitation through the macroscopic diffusion scalar : $\bm{\sigma}_p^\text{Re} :\grad \textbf{u}_p$. 
Then between \ref{eq:dt_hybrid_Wp} and \ref{eq:dt_hybrid_ep} we already observed that the source terms is the dissipation term,  $\pOavg{\bm{\sigma}_d^0:\grad \textbf{u}_d^0}$.
However, note that no common term is present between \ref{eq:dt_hybrid_kp} and \ref{eq:dt_hybrid_Wp} which implies that there is no direct transfer of energy between the center of mass velocity fluctuation quantified by $k_p$ and the internal velocity fluctuation energy $W_p$. 
However, notice that the transport equation for $k_f$, \ref{eq:dt_hybrid_k1}, contains the terms $\pavg{\textbf{u}_\alpha' \intS{\bm{\sigma}_f^0 \cdot \textbf{n}_d}}$ and $\pSavg{\textbf{w}_d^0 \cdot \bm{\sigma}_f^0 \cdot \textbf{n}_d}$ which are also present in \ref{eq:dt_hybrid_kp} and \ref{eq:dt_hybrid_Wp}. 
Consequently, the energy transfer from granular agitation $k_p$ and the internal kinetic energy $W_p$ is done through the fluid phase pseudo turbulent kinetic energy. 
To summarize this quite complicated energy cascade between both phases and the different scales we propose the following diagram, see \ref{fig:energy}. 
\begin{figure}[h!]
    \centering
    \tikzstyle{quadri}=[rectangle,draw]
    \begin{tikzpicture}[scale=1.2]
        \node[quadri,fill=gray!10] (u2) at (0,0){$(u_p)^2 / 2$};
        \node[quadri,fill=gray!10] (kp) at (4,0){$k_p$};
        \node[quadri,fill=gray!10] (Wp) at (8,0){$W_p +s_p\gamma$};
        \node[quadri,fill=gray!10] (ep) at (12,0){$e_p$};
        \node[quadri,fill=gray!10] (u12)at (0,-3){$\frac{\rho_f}{2}(u_f)^2$};
        \node[quadri,fill=gray!10] (k1) at (6,-3){$k_f$};
        \node[quadri,fill=gray!10] (e1) at (10,-3){$e_f$};
        \draw[->] (u2)--(kp)node[midway,above]{\footnotesize $\bm{\sigma}^\text{eq}_p:\grad \textbf{u}_f$};
        % \draw[<->,text width=2cm] (kp)--(u12) node[midway,left]{\footnotesize $+  n_p v_p \textbf{u}_p \cdot 
        % (\rho_d \textbf{g} - \grad p_f)
        % + n_p \textbf{u}_p \cdot \textbf{f}_{pm} - \textbf{F}_\text{pfp}$};
        \draw[<->] (k1)--(u12) node[midway,above]{\footnotesize $\bm{\sigma}^\text{eq}_f:\grad \textbf{u}_f$}node[midway,below,sloped]{\footnotesize $\textbf{u}_f\cdot\pSavg{\bm{\sigma}_f^0\cdot \textbf{n}_d} $};
        \draw[<->] (k1)--(e1) node[midway,below]{\footnotesize $\avg{\chi_f \bm{\sigma}_f^0 : \grad \textbf{u}_f^0}$};
        \draw[<->,sloped] (k1)--(kp) node[midway,above]{\footnotesize $\pavg{ \textbf{u}_\alpha'\cdot \intS{\bm{\sigma}_f^0\cdot\textbf{n}_d}}$};
        \draw[<->] (k1)--(u2) node[midway,below,sloped]{\footnotesize $\textbf{u}_p\cdot \pSavg{\bm{\sigma}_f^0 \cdot \textbf{n}_f}$};
        \draw[<->,sloped] (k1)--(Wp) node[midway,below]{\footnotesize $\pSavg{{\textbf{w}_d^0 \cdot \bm{\sigma}_f^0\cdot \textbf{n}_f}}$};
        % \draw[->] (kp)--(Wp)node[midway,above]{$(\textbf{u}_\alpha' \cdot \textbf{f}_\alpha')_p$};
        \draw[->] (Wp)--(ep)node[midway,above]{\footnotesize $\pOavg{\bm{\sigma}_d^0 : \grad \textbf{u}_d^0}$};
        \draw (e1)--(ep)node[midway,above,sloped]{\footnotesize $\pSavg{\textbf{q}_f^0 \cdot \textbf{n}_d}$};
    \end{tikzpicture}
    \caption{Energy exchange between the different components of energy in a dispersed two phase flow.
    Macroscopic kinetic energy of the particle phase, $u_p^2/2$, and of the carrier fluid $u_f^2/2$.
    $k_f$, Pseudo turbulent energy of the carrier fluid. 
    $k_p$, Pseudo turbulent energy of particle center of mass. 
     }
    \label{fig:energy}
\end{figure}
% Consequently, the energy gain due to internal dissipation stress $\pOavg{\bm{\sigma}_d^0:\grad \textbf{u}_d^0}$ comes from the internal velocity fluctuation equation. 
In the literature, it is said that the transfer terms between internal energy $e_p$ and the granular temperature $k_p$ is the \textit{dissipation rate} due to inelastic particle-particle collision present in \ref{eq:dt_hybrid_up2}, see for example \citet{fox2014multiphase,rao2008introduction}. 
However, in light of \ref{fig:energy} the energy gain due to  $\pOavg{\bm{\sigma}_d^0:\grad \textbf{u}_d^0}$ which is the \textit{dissipation rate} has no reason to be equal to the energy loss in \ref{eq:dt_hybrid_up2} represented by the term $\pavg{\textbf{u}_\alpha' \intS{\bm{\sigma}_f^0 \cdot \textbf{n}_d}}$. 
In fact some energy is first transmitted to the fluid phase $k_p$, then some of this energy is transmitted to the internal kinetic energy $W_p$, which will induce viscous dissipation within the particle. 
In short, the internal kinetic energy is transformed into internal energy but by no means the \textit{dissipation rate} $\pOavg{\bm{\sigma}_d^0:\grad \textbf{u}_d^0}$ makes the link between to the granular temperature $k_p$ and the internal energy of the particle phase $e_p$. 

\subsection{The first order momentum and mass equations}

As it is suggested in the previous section, the needs for higher moments equations arise if one of the closure terms present in the previous set of equation is highly dependent on one of the moments of the particles. 
In our case we suppose that the second order description of the averaged shape, i.e. $\textbf{M}_p$, and a first order description of velocity distribution, i.e. $\textbf{P}_p$,  is enough to express all closure terms. 
By applying the average operator on \ref{eq:dt_M_alpha},\ref{eq:dt_S_alpha} and \ref{eq:dt_mu_alpha}, one get the second order moment of mass, and first order moment of momentum symmetric and skew symmetric parts, namely, 
\begin{align}
    \pddt \left(n_p \textbf{M}_p\right)
    + \div \left(
        n_p \textbf{u}_p \textbf{M}_p
    + \textbf{M}_p^\text{Re}
    \right)
    &=
    n_p2  \textbf{S}_p
    \label{eq:dt_hybrid_Mp}\\
    \label{eq:dt_hybrid_mup}
    \pddt \left(n_p \bm{\mu}_p\right)
    + \div \left(
    n_p \textbf{u}_p \bm{\mu}_p
    + \bm{\mu}_p^\text{Re}
    \right)
    &=
    \pSavg{\textbf{r}\times(\bm\sigma_f^0\cdot \textbf{n}_d)}
    \\
    % \label{eq:dt_hybrid_Pp}
    % \pddt \left(n_p \textbf{P}_p\right)
    % + \div \left(
    %     n_p \textbf{u}_p \textbf{P}_p
    % + \textbf{P}_p^\text{Re}
    % \right)
    % &=
    % % -n_p v_p p_f \textbf{I}
    % % + n_p \textbf{F}_p
    % \pSavg{
    %     \textbf{r} \bm{\sigma}_f^0 \cdot\textbf{n}_d
    % }
    % + \pOavg{
    %     \rho_d \textbf{w}_d^0  \textbf{w}_d^0 
    %     - \bm{\sigma}_d'
    % }
    % -  \pSavg{\gamma (\textbf{I} - \textbf{nn})},\\
\label{eq:dt_hybrid_Sp}
\pddt \left(n_p \textbf{S}_p\right)
+ \div \left(
    n_p \textbf{u}_p \textbf{S}_p
+ \textbf{S}_p^\text{Re}
\right)
&=
% -n_p v_p p_f \textbf{I}
\pSavg{(\textbf{r}\bm\sigma_f^0+\bm\sigma_f^0\textbf{r})\cdot \textbf{n}_d}
% n_p  \mathscr{S}_p^*
% + \pSavg{
%     \textbf{r} \bm{\sigma}_f^0 \cdot\textbf{n}_d
% }
+ \pOavg{
    \rho_d \textbf{w}_d^0  \textbf{w}_d^0 
    - \bm{\sigma}_d
}\nonumber\\
&-  \pSavg{\gamma (\bm\delta - \textbf{nn})},
\end{align}
respectively, where we have defined the fluctuaiton terms as $
 \textbf{M}_p^\text{Re}
 = \pavg{\textbf{M}_\alpha' \textbf{u}_\alpha'} $,  $ 
 \textbf{S}_p^\text{Re}
 = \pavg{\textbf{P}_\alpha' \textbf{u}_\alpha'}$ and $ 
 \bm{\mu}_p^\text{Re}
 = \pavg{\bm{\mu}_\alpha' \textbf{u}_\alpha'}
$.
At this stage it is interesting to have a look at the different closures in the situation of stokes flows. 
Especially, the closure term in the stretching of momentum have the form 
\begin{align*}
    \pOavg{\rho_d \textbf{w}_d^0  \textbf{w}_d^0 }
    &= \frac{\rho_d \phi_d}{140(\lambda +1 )}\left(7\textbf{u}_{fp}\textbf{u}_{pf} + (\textbf{u}_{pf}\cdot \textbf{u}_{pf})\bm\delta\right)
    + \frac{\rho_d \phi_d a^2}{315 (\lambda + 1)^2}[(\textbf{E}_f : \textbf{E}_f)\bm\delta+15\textbf{E}_f\cdot \textbf{E}_f]\\
    \pOavg{\bm{\sigma}_d}
    &= \frac{6}{5}\phi \mu_f \frac{1}{1+\lambda} \textbf{E}_f
\end{align*}
This indicates a possible link between translation and deformation of the drops. 
Indeed, the translation of the drop induce inertial motion inside it which tends to deform the droplet. 
Regarding the surface tension contribution it is proportional to the mean deformation of the particle, in fact at first order in deformation $\pSavg{\gamma (\bm\delta - \textbf{nn})} \sim \frac{\gamma a^2}{\rho_d} \textbf{M}_p$ \citep{lhuillier1987phenomenology}. 

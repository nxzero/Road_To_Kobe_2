
Carrying out the method of \citet{drew1983mathematical,zhang1994averaged} and \citet{jackson1997locally},  we assign to the microscopic field $f(\textbf{y})$ defined over $\Omega$ the corresponding averaged field $\avg{f}(\textbf{x})$ also defined over $\Omega$.
Where $\avg{\ldots}$ is the averaging operator, $\textbf{y}$ is the spacial coordinate in the laboratory reference frame,
and $\textbf{x}$ macro scale coordinate at which we evaluate the average.
Thus, the \textbf{volume average} of the microscopic feilds $f$ is defined as \citep{nott2011suspension,jackson1997locally,marle1982macroscopic},
\begin{equation}
    \left<f\right>(\textbf{x}) = \int g(\textbf{x},\textbf{y}) f(\textbf{y})d \textbf{y},
    \label{eq:avg}
\end{equation}
where $g(\textbf{x},\textbf{y})$ is the smoothing or weighting function.
The first argument $\textbf{x}$, is location at which we take the average, and \textbf{y} is the variable of integration or the local coordinate vector.
The smoothing function $g(\textbf{x},\textbf{y})$ must follow two properties, the first one is normalization to unity
$\int g(\textbf{x},\textbf{y}) d\textbf{y} = 1$.
The second one is that $g$ vanish for all $\textbf{y}$ far form $\textbf{x}$, thus $\lim\limits_{|\textbf{r}| \to \infty} g(\textbf{x},\textbf{y}) = 0$ with $\textbf{r} = \textbf{x} - \textbf{y}$.
Also, the radius $l$, of the weighting function $g$, it is defined such that $\int_{|\textbf{r}|<l} g(\textbf{x},\textbf{y})d\textbf{y} = 1/2$.
Those characteristics ensure that the integral of \ref{eq:avg} is convergent and well normalized.
The volume average method is one among others such as the time average \citep{ishii2010thermo} and ensemble average \citep{zhang1994averaged}. 
However, all of these averaging technics remains equivalent \citep{jackson1997locally} as long as the separation of scale condition is respected, namely,
\begin{equation*}
    a \ll l \ll L,
\end{equation*}
where we recall that $a$ is the length scale of the particle, $l$ the length scale of the weighting function and $L$ the macroscopic length scale.
Similarly, we can use the statistical approach to carry out the average. 
We follow \citet{zhang2021ensemble} which extended the definition of \citet{batchelor1972sedimentation}. 
Let, $P(\mathscr{C},t)$ be the probability density function that describe the probability of finding the flow in the configuration $\mathscr{C}$ at time $t$, were $\mathscr{C} = (\lambda_1,\lambda_1,\lambda_1,\ldots\lambda_N )$ is the finite set of all the parameters describing the dispersed multiphase flows (such as the particles' velocity and position \ldots).  
In other words, we define $d\mathscr{P} = P(\mathscr{C},t)d\mathscr{C}$ as the probable number of particles in the incremental region of the particles' phase space $d\mathscr{C}$ around $\mathscr{C}$ at time $t$. 
It follows the definition of the ensemble average of an arbitrary property $f$,
\begin{equation}
    \avg{f}(\textbf{x},t)
    =\int f(\textbf{x},\mathscr{C},t) d\mathscr{P}. 
    \label{eq:ensemble_avg}
\end{equation}  
Besides, it is interesting to mention some mathematical properties shared by all the averaging operators. 
For two arbitrary Eulerian fields $f$ and $h$ we have,
\begin{align}
    &\avg{f+h} = \avg{f}+\avg{h}, 
    &\avg{\avg{f}h} = \avg{f}\avg{h}, \nonumber \\
    &\avg{\pddt f} 
    = \pddt\avg{f}, 
    &\avg{\nablabh f}
    = \nablab \avg{f}. 
    \label{eq:avg_properties}
\end{align}
The two first relations are called the Reynolds' rules, the $3^{th}$ one is the Leibniz' 
rule and the last one, the Gauss' rule \citep{drew1983mathematical}.
It is important to understand both method lead to the same results and thus are rigorously equivalent. 

Applying \ref{eq:avg} on \ref{eq:dt_chi_k_f_k} and \ref{eq:dt_delta_I_f_I} and considering the properties from \ref{eq:avg_properties} yields the general form of the averaged equations of multiphase flows, namely,
\begin{align}
    \pddt \avg{\chi_k f_k}
    &= \nablabh \cdot \avg{\chi_k \bm{\Phi}_k - \chi_k f_k \textbf{u}_k}
    + \avg{\chi_k \textbf{S}_k}
    + \avg{\delta_I\left[
        \bm{\Phi}_k
        + f_k
        \left(
            \textbf{u}_I
            - \textbf{u}_k
        \right)
    \right]
    \cdot \textbf{n}_k} ,
    \label{eq:avg_dt_chi_f}\\
    \pddt \avg{\delta_If_I}
    &= 
    \nablabh \cdot \avg{\delta_I \mathbf{\Phi}_{I||} - \delta_I f_I \textbf{u}_I}
    +\avg{\delta_I\textbf{S}_I} 
    - \avg{\delta_I \Jump{
    f_k (\textbf{u}_I - \textbf{u}_k)
    + \mathbf{\Phi}_k
    } }.
    \label{eq:avg_dt_delta_f}
\end{align}
Not many comments can be made on those equations, expect that we now have the averaged field as unknown instead of the locals fields. 
Together \ref{eq:avg_dt_delta_f}  and \ref{eq:avg_dt_delta_f} form the \textit{two-fluid} formulation of averaged multiphase flows problem. 
Besides, on can derive the so-called averaged \textbf{single-fluid}, indeed applying \ref{eq:avg} on \ref{eq:dt_f} directly gives us, 
\begin{equation}
    \pddt \avg{f}
    = \nablab \cdot \avg{\bm{\Phi} - f \textbf{u}}
    + \avg{\textbf{S}}.
    \label{eq:avg_dt_f}
\end{equation}
Again, the only difference with \ref{eq:dt_f} is that we now have the average of $\avg{f}$ as unknown instead of the local property $f$. 
It is interesting to notice that in Volume Of Fluid method we solve for the fields averaged on a cell volume, thus we actually discretize \ref{eq:avg_dt_f} and solve for the averaged quantities \citep{tryggvason2011direct} instead of directly solving the local transport equations \ref{eq:dt_f}, same goes for \ref{eq:avg_dt_delta_f} and \ref{eq:avg_dt_delta_f}.
Correspondingly, we are able to take the average of the particular fields equations \ref{eq:dt_dq_alpha} and \ref{eq:dt_dq_I_alpha}, yielding, 
\begin{align}
    \pddt \pavg{ q_\alpha}
    + \nablabh\cdot \pavg{ q_\alpha \textbf{u}_\alpha}
    &= \pavg{\int_{\Omega_\alpha} \textbf{S}_k d\Omega}\nonumber\\
    &+ \pavg{\int_{\Sigma_\alpha} \left[\bm{\Phi}_k + f_k (\textbf{u}_I-\textbf{u}_k) \right] \cdot \textbf{n}_k d\Sigma},
    \label{eq:avg_dt_dq_alpha}\\
    \pddt \pavg{ q_{I\alpha}}
    + \nablabh\cdot \pavg{ q_{I\alpha} \textbf{u}_\alpha}
    &= \pavg{\int_{\Sigma_\alpha} 
        \textbf{S}_I
    d\Sigma}\nonumber\\
    &- \pavg{\int_{\Sigma_\alpha} \Jump{
        f_k (\textbf{u}_I - \textbf{u}_k)
        + \mathbf{\Phi}_k
    }
    d\Sigma}.
    \label{eq:avg_dt_dq_I_alpha}
\end{align}
When considering a dispersed multiphase flows problem we use Euler-Euler method to solves the equations of the continuous and dispersed phase numerically. 
In those code we solve the equations for the cell averaged values, thus we should consider discretizing \ref{eq:avg_dt_dq_I_alpha} and \ref{eq:avg_dt_dq_alpha} while solving the dispersed phase equations.
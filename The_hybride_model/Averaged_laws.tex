In this study we use the statistical approach to derive the averaged equations of conservation. 
Specifically, we use the method described in \citet{zhang2021ensemble} which extended the ensemble average definition of \citet{batchelor1972sedimentation}. 
In the following we recall some properties of the ensemble average operator. 

\subsection{Ensemble phase average}

Let, $P(\mathscr{C},t)$ be the probability density function that describe the probability of finding the flow in the configuration $\mathscr{C}$ at time $t$, were $\mathscr{C} = (\lambda_1,\lambda_2,\lambda_3,\ldots)$ is a finite set of all the parameters describing the flow configuration. 
Then, we define $d\mathscr{P} = P(\mathscr{C},t)d\mathscr{C}$ as the probable number of particles in the incremental region of the particles' phase space $d\mathscr{C}$ around $\mathscr{C}$ at time $t$. 
It follows this definition, that the ensemble average of an arbitrary property $f$ defined on $\Omega$, yields,
\begin{equation}
    \avg{f}(\textbf{x},t)
    =\int f(\textbf{x},\mathscr{C},t) d\mathscr{P}. 
    \label{eq:avg}
\end{equation}  
It is interesting to mention some mathematical properties of the ensemble average operators. 
For two arbitrary Eulerian fields $f$ and $h$ we have,
\begin{align}
    &\avg{f+h} = \avg{f}+\avg{h}, 
    &\avg{\avg{f}h} = \avg{f}\avg{h}, \nonumber \\
    &\avg{\pddt f} 
    = \pddt\avg{f}, 
    &\avg{\nablabh f}
    = \nablab \avg{f}. 
    \label{eq:avg_properties}
\end{align}
The two first relations are called the Reynolds' rules, the $3^{th}$ one is the Leibniz' 
rule and the last one, the Gauss' rule \citep{drew1983mathematical}.
Additionally, for any phase quantity defined in $\Omega_k$ we introduce the definition, 
\begin{equation}
    \phi_k\kavg{f}(\textbf{x},t) = \avg{\chi_k f_k}
    \label{eq:1_avg}
\end{equation}
where, $\phi_k(\textbf{x},t) = \avg{\chi_k}$ is the volume fraction of the phase $k$.
Therefore, $\kavg{f}$ is the conditional average of the field $\chi_k f_k$ knowing that we are in the phase $k$.
Similarly, for the particle fields, we can introduce the particle phase average by,
\begin{equation}
     \pnavg{q_\alpha}(\textbf{x},t) = \avg{\delta_\alpha f_\alpha}
     \label{eq:p_avg}
\end{equation}
where, $n_p(\textbf{x},t) = \avg{\delta_\alpha}$ is the probable number of finding a particle center of mass at $\textbf{x}$. 
Thus, $\pnnavg{q_\alpha}$ is the conditional average of $q_\alpha$ on the set of particle. 


The ensemble average method is one among others, such as the volume average method\citep{jackson1997locally} or the time average
\citep{ishii2010thermo} and ensemble average \citep{zhang1994averaged}. 
However, all of these averaging technics remains equivalent \citep{jackson1997locally}. 

\subsection{Phase average and particle averaged equations}

Applying \ref{eq:avg} on \ref{eq:dt_chi_k_f_k} and \ref{eq:dt_delta_I_f_I} and considering the properties from \ref{eq:avg_properties} yields the general form of the averaged equations of multiphase flows, namely,
\begin{align}
    \pddt \avg{\chi_k f_k}
    &= \nablabh \cdot \avg{\chi_k \mathbf{\Phi}_k - \chi_k f_k \textbf{u}_k}
    + \avg{\chi_k \textbf{S}_k}
    + \avg{\delta_I\left[
        \mathbf{\Phi}_k
        + f_k
        \left(
            \textbf{u}_I
            - \textbf{u}_k
        \right)
    \right]
    \cdot \textbf{n}_k} ,
    \label{eq:avg_dt_chi_f}\\
    \pddt \avg{\delta_If_I}
    &= 
    \nablabh \cdot \avg{\delta_I \mathbf{\Phi}_{I||} - \delta_I f_I \textbf{u}_I}
    +\avg{\delta_I\textbf{S}_I} 
    - \avg{\delta_I \Jump{
    f_k (\textbf{u}_I - \textbf{u}_k)
    + \mathbf{\Phi}_k
    } }.
    \label{eq:avg_dt_delta_f}
\end{align}
Together \ref{eq:avg_dt_chi_f}  and \ref{eq:avg_dt_delta_f} form the \textit{two-fluid} formulation of averaged multiphase flows problem. 
One can derive the so-called averaged \textit{single-fluid} formulation by applying \ref{eq:avg} on \ref{eq:dt_f} which directly gives us, 
\begin{equation}
    \pddt \avg{f}
    = \nablab \cdot \avg{\mathbf{\Phi} - f \textbf{u}}
    + \avg{\textbf{S}}.
    \label{eq:avg_dt_f}
\end{equation}
It is interesting to notice that in Volume Of Fluid method we solve for the fields averaged on a cell volume, thus we actually discretize \ref{eq:avg_dt_f} and solve for the averaged quantities \citep{popinet2018numerical,tryggvason2011direct} instead of directly solving the local transport equations \ref{eq:dt_f}.

Correspondingly, we take the average of the particle fields equations by using \ref{eq:avg} on \ref{eq:dt_dq_alpha_tot} and \ref{eq:dt_dQ_alpha_tot}, which gives, 
\begin{multline}
    \pddt \avg{\delta_\alpha (q_\alpha+q_{I_\alpha})}
    + \nablabh \cdot \avg{\delta_\alpha\textbf{u}_\alpha(q_\alpha+q_{I_\alpha})}
    = \avg{\delta_\alpha\int_{\Omega_\alpha} \textbf{S}_2 d\Omega}\\
    + \avg{\delta_\alpha\int_{\Sigma_\alpha} \textbf{S}_I d\Sigma}
    + \avg{\delta_\alpha\int_{\Sigma_\alpha} \left[\mathbf{\Phi}_1 + f_1 (\textbf{u}_I-\textbf{u}_1) \right] \cdot \textbf{n}_2 d\Sigma,}
    \label{eq:avg_dt_dq_alpha_tot}
\end{multline}
\begin{multline}
    \pddt \avg{\delta_\alpha (\mathcal{Q}_\alpha+\mathcal{Q}_{I_\alpha})}
    + \nablabh \cdot \avg{\delta_\alpha\textbf{u}_\alpha(\mathcal{Q}_\alpha+\mathcal{Q}_{I_\alpha})}
    =\\ \avg{\delta_\alpha\int_{\Omega_\alpha} \left(
        \textbf{r} \textbf{S}_2         
        + f_2  \textbf{w}_2 
        - \mathbf{\Phi}_2
    \right) d\Omega}
    + \avg{\delta_\alpha\int_{\Sigma_\alpha} \left(
        \textbf{r}\textbf{S}_I
        + f_I \textbf{w}_I
        - \mathbf{\Phi}_{||}^I
    \right) d\Sigma}\\
    + \avg{\delta_\alpha\int_{\Sigma_\alpha} \textbf{r} \left[
        \mathbf{\Phi}_1
        + f_1 (\textbf{u}_I-\textbf{u}_1)
    \right]\cdot \textbf{n}_2  d\Sigma}.
    \label{eq:avg_dt_dQ_alpha_tot}
\end{multline}
From \ref{ap:Moments_equations} the derivation of the higher moment particle-average equations is straightforward. 
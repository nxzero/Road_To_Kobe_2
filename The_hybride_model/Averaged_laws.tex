In this study we use the statistical approach to derive the averaged equations of conservation. 
Specifically, we use the method described in \citet{zhang2021ensemble} which extended the ensemble average definition of \citet{batchelor1972sedimentation}. 
In the following we recall some properties of the ensemble average operator. 

\subsection{Ensemble average}

Let, $P(\mathscr{C},t)$ be the probability density function that describe the probability of finding the flow in the configuration $\mathscr{C}$ at time $t$, were $\mathscr{C} = (\lambda_1,\lambda_2,\lambda_3,\ldots)$ is a finite set of all the parameters describing the flow configuration. 
Then, we define $d\mathscr{P} = P(\mathscr{C},t)d\mathscr{C}$ as the probable number of particles in the incremental region of the particles' phase space $d\mathscr{C}$ around $\mathscr{C}$ at time $t$. 
It follows from this definition, that the ensemble average of an arbitrary local property $f^0$ defined on $\Omega$, yields,
\begin{equation}
    f(\textbf{x},t)
    = \avg{f^0}(\textbf{x},t)
    =\int f^0(\textbf{x},\mathscr{C},t) d\mathscr{P}. 
    \label{eq:avg}
\end{equation}  
This definition can be applied to Lagrangian properties as well by using the previous formulation, namely $\pavg{q_\alpha}(\textbf{x},t) = \int \delta(\textbf{x} - \textbf{x}_\alpha) q_\alpha(\mathscr{C}) d\mathscr{P}. $
It is interesting to mention some mathematical properties of the ensemble average operators. 
For two arbitrary Eulerian fields $f$ and $h$ we have,
\begin{align}
    &\avg{f^0+h^0} = f+h, 
    &\avg{\avg{f^0}h^0} = gh, \nonumber \\
    &\avg{\pddt f^0} 
    = \pddt f, 
    &\avg{\grad f^0}
    = \grad f. 
    \label{eq:avg_properties}
\end{align}
\todo{Either $P$ or $\Pi$ depend on time actually. on my whole work i haven't done that}

The two first relations are called the Reynolds' rules, the $3^{th}$ one is the Leibniz' 
rule and the last one, the Gauss' rule \citep{drew1983mathematical}.
Additionally, for any phase quantity defined in $\Omega_k$ we introduce the definition, 
\begin{equation}
    \phi_k f_k (\textbf{x},t) = \avg{\chi_k f_k^0}
    \label{eq:1_avg}
\end{equation}
where, $\phi_k(\textbf{x},t) = \avg{\chi_k}$ is the volume fraction of the phase $k$
and $f_k$ the conditional average of the field $\chi_k f_k^0$ on the phase $k$.
Similarly, for the particle fields, we can introduce the particle phase average by,
\begin{equation}
     n_p q_p(\textbf{x},t) = \avg{\delta_\alpha q_\alpha}
     \label{eq:p_avg}
\end{equation}
where, $n_p(\textbf{x},t) = \avg{\delta_\alpha}$ is the probable number of finding a particle center of mass at $\textbf{x}$
and $q_p$ is the conditional average of $q_\alpha$ on the set of particles. 
We also define the fluctuation of a particle-averaged and phase-averaged quantity by,
\begin{equation}
    q_\alpha' = q_\alpha - q_p
    \;\;\;\;\;\;\text{and}
    \;\;\;\;\;\;
    f_k' = f_k^0 - f_k,
    \label{eq:def_fluctu}
\end{equation}
respectively. 
These definitions lead to the following properties, $\avg{\chi_1 f'_k} = 0$ and $\avg{\delta_\alpha q_\alpha'} =0$. 
Consequently, the product $\avg{\chi_k f^0_kg^0_k}$ can be decomposed as $\avg{\chi_k f_k^0g_k^0}=\phi_k f_kg_k + \avg{\chi_k f'_kg'_k}$. 
These decompositions will play a crucial role in the upcoming section. 
\todo{more statistical / mathematical comments about covarience }

The ensemble average method is one among others, such as the volume average method\citep{jackson1997locally} or the time average
\citep{ishii2010thermo}.
It is shown that all of these averaging technics remains equivalent \citep{jackson1997locally}. 

\subsection{Phase average and particle averaged equations}

Applying \ref{eq:avg} on \ref{eq:dt_chi_k_f_k} and \ref{eq:dt_delta_I_f_I} and considering the properties from \ref{eq:avg_properties} yields the general form of the averaged equations of multiphase flows, namely,
\begin{align}
    \pddt \avg{\chi_k f_k^0}
    &= \div \avg{\chi_k \mathbf{\Phi}_k^0 - \chi_k f_k^0 \textbf{u}_k^0}
    + \avg{\chi_k s_k^0}
    + \avg{\delta_I\left[
        \mathbf{\Phi}_k^0
        + f_k^0
        \left(
            \textbf{u}_I^0
            - \textbf{u}_k^0
        \right)
    \right]
    \cdot \textbf{n}_k} ,
    \label{eq:avg_dt_chi_f}\\
    \pddt \avg{\delta_If_I^0}
    &= 
    \div \avg{\delta_I \mathbf{\Phi}_{I||}^0 - \delta_I f_I^0 \textbf{u}_I^0}
    +\avg{\delta_Is_I^0} 
    - \avg{\delta_I \Jump{
    f_k^0 (\textbf{u}_I^0 - \textbf{u}_k^0)
    + \mathbf{\Phi}_k^0
    } }.
    \label{eq:avg_dt_delta_f}
\end{align}
Together \ref{eq:avg_dt_chi_f}  and \ref{eq:avg_dt_delta_f} form the \textit{two-fluid} formulation of averaged multiphase flows problem. 
One can derive the so-called averaged \textit{single-fluid} formulation by applying \ref{eq:avg} on \ref{eq:dt_f} which directly gives, 
\begin{equation}
    \pddt f
    = \div (\mathbf{\Phi}^0 - f \textbf{u} - \avg{f'\textbf{u}'})
    + s.
    \label{eq:avg_dt_f}
\end{equation}
It is interesting to notice that in Volume Of Fluid method we solve for the fields averaged on a cell volume, thus we actually discretize \ref{eq:avg_dt_f} and solve for the averaged quantities \citep{popinet2018numerical,tryggvason2011direct} instead of directly solving the local transport equations \ref{eq:dt_f}.

Correspondingly, we take the average of the particle fields equations by using \ref{eq:avg} on \ref{eq:dt_dq_alpha_tot} and \ref{eq:dt_dQ_alpha_tot}, which gives, 
\begin{multline}
    \pddt \avg{\delta_\alpha  q_\alpha^\text{tot}}
    + \div \avg{\delta_\alpha\textbf{u}_\alpha q_\alpha^\text{tot}}
    = \avg{\delta_\alpha\int_{\Omega_\alpha} s_2^0 d\Omega}
    + \avg{\delta_\alpha\int_{\Sigma_\alpha} s_I^0 d\Sigma}\\
    + \avg{\delta_\alpha\int_{\Sigma_\alpha} \left[\mathbf{\Phi}_1^0 + f_1^0 (\textbf{u}_I^0-\textbf{u}_1^0) \right] \cdot \textbf{n}_2 d\Sigma,}
    \label{eq:avg_dt_dq_alpha_tot}
\end{multline}
\begin{multline}
    \pddt \avg{\delta_\alpha \mathcal{Q}_\alpha^\text{tot}}
    + \div \avg{\delta_\alpha\textbf{u}_\alpha\mathcal{Q}_\alpha^\text{tot}}
    =\avg{\delta_\alpha\int_{\Omega_\alpha} \left(
        \textbf{r} s_2^0         
        + f_2^0  \textbf{w}_2^0 
        - \mathbf{\Phi}_2^0
    \right) d\Omega}\\
    + \avg{\delta_\alpha\int_{\Sigma_\alpha} \left(
        \textbf{r}s_I^0
        + f_I^0 \textbf{w}_I^0
        - \mathbf{\Phi}_{I||}^0
    \right) d\Sigma}
    + \avg{\delta_\alpha\int_{\Sigma_\alpha} \textbf{r} \left[
        \mathbf{\Phi}_1^0
        + f_1^0 (\textbf{u}_I^0-\textbf{u}_1^0)
    \right]\cdot \textbf{n}_2  d\Sigma}.
    \label{eq:avg_dt_dQ_alpha_tot}
\end{multline}
In \ref{ap:Moments_equations} the derivation of the higher moment particle-averaged equations is provided. 
In this study,\ref{eq:avg_dt_chi_f} and \ref{eq:avg_dt_delta_f} are refereed to as the phase-averaged equations, while \ref{eq:avg_dt_dq_alpha_tot} and \ref{eq:avg_dt_dQ_alpha_tot} are denoted as the particle-averaged equation. 
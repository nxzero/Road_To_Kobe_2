
\section{The volume averaged equations of motion}

We start this section by presenting the most common technics of averages and their operators.
Therefore, we note $\left<f\right>(\textbf{x})$ the average of an arbitrary quantity $f(\textbf{y})$,
where $\textbf{y}$ is the spacial coordinate in the laboratory reference frame,
and $\textbf{x}$ macro scale coordinate at which we evaluate the average.
Therefore, we define the \textbf{volume average} operator such as,
\begin{equation}
    \left<f\right>(\textbf{x},t) = \int g(\textbf{x},\textbf{y}) f(\textbf{y},t)dV,
    \label{eq:avg}
\end{equation}
where $g(\textbf{x},\textbf{y})$ is the smoothing (or weighting) function introduced by
\citet{jackson1997locally,marle1982macroscopic}.
The first argument, $\textbf{x}$, is location at which we take the average, and \textbf{y} is the variable of integration or the local coordinate vector.
The smoothing function $g(\textbf{x},\textbf{y})$ must follow two properties, the first one is normalization to unity
$\int g(\textbf{x},\textbf{y}) dV = 1 \;\forall g$.
The second one is that $g$ vanish for all $\textbf{y}$ far form $\textbf{x}$, thus $\lim\limits_{|\textbf{r}| \to \infty} g(\textbf{x},\textbf{y}) = 0$ with $\textbf{r} = \textbf{x} - \textbf{y}$.
Also, the radius $R$, of the weighting function $g$, it is defined as $1/2 = \int_{|\textbf{r}|<R} g(\textbf{x},\textbf{y})dV$.
Those characteristics ensure that the integral of \ref{eq:avg} is convergent and well normalized.


Now we introduce the subclass operators, or conditional average operators.
The average of the quantity $f$ considering only the volume of the $k^{th}$ phase, will be defined, such that,
\begin{equation}
    \phi_k \kavg{f}
    = \avg{\chi_k f_k},
    \label{eq:avg_k_phase}
\end{equation}
where $\phi_k$ is the volume fraction of the phase $k$ at \textbf{x} and $\kavg{\ldots}$ is the conditional average operator on the phase $k$.
It can be obtained by substituting $f$ by $1$ in \ref{eq:avg_k_phase}, it reads,
\begin{equation*}
    \phi_k
    = \avg{\chi_k}
\end{equation*}

Having established averages operators over the volume of both phases independently, we now define one last conditional averaging operator.
Namely, the surface phase average operator,
\begin{equation}
    a_I\Iavg{f}
    = \avg{\delta_I f},
    \label{eq:avg_I_phase}
\end{equation}
where $a_I$ is the interfacial area concentration.
$a_I$ can be thought of the ratio between the surface of the interfaces over the volume where the interface are included.
It is defined by,
\begin{equation}
    a_I
    = \avg{\delta_I}.
\end{equation}

Now, we would like to emphasize that the quantities inside phase average, are implicitly defined inside the $k^{th}$ phase so that it avoid redundancies.
Therefore, $\kavg{f} = \kavg{f_k}= \avg{f_k\chi_k}$.
At the interface however it is not the case, indeed by definition, at the interface the quantities are defined in both phases.
Thus, we adopt the following convention, when averaging an equation on the volume of phase $k$, $\Iavg{f_k}$ will be the interfacial average of $f_k$, with $f_k$ being the quantity $f$ defined on the phase $k$.
However, $\Iavg{f}$ without the subscript $_k$, will refer to the interface average of the quantity $f$ defined on the neighboring phase of $k$.


\subsection{Average of an arbitrary conservation equation}

From the \ref{eq:two-fluid_global} and the phase average operator definition (\ref{eq:avg_k_phase}) we can show that the phase averaged conservation equation of an arbitrary quantity $f_k$ reads as,
\begin{align}
    \pddt \avg{\chi_k f_k}
    &= \nablabh \cdot \avg{\chi_k \bm{\Phi}_k - \chi_k f_k \textbf{u}_k}
    + \avg{\chi_k \textbf{S}_k}
    + \avg{\delta_I\left[
        \bm{\Phi}_k
        + f_k
        \left(
            \textbf{u}_I
            - \textbf{u}_k
        \right)
    \right]
    \cdot \textbf{n}_k} ,
    \label{eq:avg_dt_chif}\\
    \pddt \avg{\delta_If_I}
    &= 
    \nablabh \cdot \avg{\delta_I \mathbf{\Phi}_{I||} - \delta_I f_I \textbf{u}_I}
    +\avg{\delta_I\textbf{S}_I} 
    - \avg{\delta_I \Jump{
    f_k (\textbf{u}_I - \textbf{u}_k)
    + \mathbf{\Phi}_k
    } }
    \label{eq:avg_dt_deltaf}
\end{align}
So as the microscopic jump condition this equation maintains the consistency between the different averaged quantity in presence.
Similarly, the bulk or global averaged conservation equation can be obtained averaging \ref{eq:single-fluid_global} yielding,
\begin{equation*}
    \pddt \avg{f}
    = \nablab \cdot \avg{\bm{\Phi} - f \textbf{u}}
    + \avg{\textbf{S}}
    \label{eq:avg_global}
\end{equation*}


Likewise, we can take the average of the particular fields equations \ref{eq:dt_dq_alpha} and \ref{eq:dt_dq_I_alpha}, 
\begin{align}
    \pddt \pavg{ q_\alpha}
    + \nablabh \pavg{ q_\alpha \textbf{u}_\alpha}
    &= \pavg{\int_{\Omega_\alpha} \textbf{S}_k d\Omega}\nonumber\\
    &+ \pavg{\int_{\Sigma_\alpha} \left[\bm{\Phi}_k + f_k (\textbf{u}_I-\textbf{u}_k) \right] \cdot \textbf{n}_k d\Sigma},
    \label{eq:avg_dt_dq_alpha}\\
    \pddt \pavg{ q_{I\alpha}}
    + \nablabh \pavg{ q_{I\alpha} \textbf{u}_\alpha}
    &= \pavg{\int_{\Sigma_\alpha} 
        \textbf{S}_I
    d\Sigma}\nonumber\\
    &- \pavg{\int_{\Sigma_\alpha} \Jump{
        f_k (\textbf{u}_I - \textbf{u}_k)
        + \mathbf{\Phi}_k
    }
    d\Sigma}
    \label{eq:avg_dt_dq_I_alpha}
\end{align}
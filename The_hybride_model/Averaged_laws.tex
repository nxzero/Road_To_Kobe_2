\JL{je trouve cette section assez mal organise donc j'ai un peu reorganise : pour moi le plus simple est de presenter brievement les techniques de moyennes. De dire ensuite qu'on choisit ici la moyenne volumique et de la presenter ensuite.}
The volume average method is one among others such as the time average \citep{ishii2010thermo} and ensemble average \citep{zhang1994averaged}. 
However, all of these averaging technics remains equivalent \citep{jackson1997locally} as long as the separation of scale condition is respected, namely,
\begin{equation*}
    a \ll l \ll L,
\end{equation*}
where we recall that $a$ is the particle length scale, $l$ the length scale of the weighting function and $L$ the macroscopic length scale.

\JL{Tu ne peux pas avoir 2 pages de texte apres une section puis une sous section. J'ai donc ajoute une sous section.}
\subsection{Volume averaging}
Following  the strategy of \citet{jackson1997locally},  we assign to the microscopic field $f(\textbf{y})$ defined over $\Omega$ the corresponding averaged field $\avg{f}(\textbf{x})$. 
Where $\avg{\ldots}$ is the average operator, $\textbf{y}$ is the spacial coordinate in the laboratory reference frame,
and $\textbf{x}$ macro scale coordinate in the averaged phase space.
The volume average of the microscopic field $f$ is then defined as \citep{nott2011suspension,jackson1997locally,marle1982macroscopic},
\JL{le bon papier de marle a citer dans ce cadre est celui de 1967. je pense pas que Nott soit pertinent ici}
\begin{equation}
    \left<f\right>(\textbf{x}) = \int g(\textbf{x},\textbf{y}) f(\textbf{y})d \textbf{y},
    \label{eq:avg}
\end{equation}
where $g(\textbf{x},\textbf{y})$ is the smoothing or weighting function.
The first argument $\textbf{x}$, is location at which we take the average, and \textbf{y} is the variable of integration or the local coordinate vector.
The smoothing function $g(\textbf{x},\textbf{y})$ must follow two properties, the first one is normalization to unity
$\int g(\textbf{x},\textbf{y}) d\textbf{y} = 1$.
The second one is that $g$ vanish for all $\textbf{y}$ far form $\textbf{x}$, thus $\lim\limits_{|\textbf{r}| \to \infty} g(\textbf{x},\textbf{y}) = 0$ with $\textbf{r} = \textbf{x} - \textbf{y}$.
Also, the radius $l$, of the weighting function $g$, it is defined such that $\int_{|\textbf{r}|<l} g(\textbf{x},\textbf{y})d\textbf{y} = 1/2$.
Those characteristics ensure that the integral of \ref{eq:avg} is convergent and well normalized. \JL{je ne sais pas si c'est necessaire de decrire autant les proprietes de g. Tu peux juste dire que g doit satisfaires certaines hypotheses resumees dans les travaux de Jackson...}

Similarly, we can use the statistical approach to carry out the average. \JL{pourquoi parles tu de l'approche statistique pour moyennee ? ce n'est pas une revue choisi en une et c'est termine.}
% In order to derive the averaged equations we use the method of \citet{zhang2021ensemble} which extended the ensemble average definition of \citet{batchelor1972sedimentation}. 
% Let, $P(\mathscr{C},t)$ be the probability density function that describe the probability of finding the flow in the configuration $\mathscr{C}$ at time $t$, were $\mathscr{C} = (\lambda_1,\lambda_2,\lambda_3,\ldots)$ is a finite set of all the parameters describing the dispersed phase at time $t$. 
% Then, we define $d\mathscr{P} = P(\mathscr{C},t)d\mathscr{C}$ as the probable number of particles in the incremental region of the particles' phase space $d\mathscr{C}$ around $\mathscr{C}$ at time $t$. 
% It follows this definition, that the ensemble average of an arbitrary property $f$ defined over $\Omega$ yields,
% \begin{equation}
%     \avg{f}(\textbf{x},t)
%     =\int f(\textbf{x},\mathscr{C},t) d\mathscr{P}. 
%     \label{eq:ensemble_avg}
% \end{equation}  
Besides, it is interesting to mention some mathematical properties shared by all the averaging operators. 
For two arbitrary Eulerian fields $f$ and $h$ we have,
\begin{align}
    &\avg{f+h} = \avg{f}+\avg{h}, 
    &\avg{\avg{f}h} = \avg{f}\avg{h}, \nonumber \\
    &\avg{\pddt f} 
    = \pddt\avg{f}, 
    &\avg{\nablabh f}
    = \nablab \avg{f}. 
    \label{eq:avg_properties}
\end{align}
The two first relations are called the Reynolds' rules, the $3^{th}$ one is the Leibniz' 
rule and the last one, the Gauss' rule \citep{drew1983mathematical}.
It is important to understand both method lead to the same results and thus are rigorously equivalent. \JL{je n'avais jamais entendu parler du nom de ces regles. Pour moi l'ensemble de ces regles correspondent à l'operateur de Reynolds, mais est il necessaire de le specifier ?}


Additionally, for any phase quantity defined in $\Omega_1$ we introduce the definition, 
\begin{equation}
    \phi_1\oneavg{f}(\textbf{x},t) = \avg{\chi_1 f_1}
    \label{eq:1_avg}
\end{equation}
\JL{je trouve la notation $<>^1$ plus claire que $\oneavg{}$}
where, $\phi_1(\textbf{x},t) = \avg{\chi_1}$ is the probability of being in the phase $1$ at \textbf{x}, known as the volume fraction \JL{$\phi _1$ est  la fraction volumique c n'a rien a voir avec une probabilite}, and 
$\oneavg{f}$ is the average of the field $\chi_1 f_1$ knowing that $\textbf{x}\in \Omega_1$ \JL{idem $\oneavg{f}$ est la moyenne sur la phase 1, pas la peine de parler de probabilite}. 
Similarly, for the particle fields, we can introduce the particle phase average by,
\begin{equation}
     \pnavg{q_\alpha}(\textbf{x},t) = \avg{\delta_\alpha f_\alpha}
     \label{eq:p_avg}
\end{equation}
where, $n_p(\textbf{x},t) = \avg{\delta_\alpha}$ is the probable number of finding the center of mass of a particle inside the volume $d\textbf{x}$ around the position \textbf{x} and  at time $t$, known as the number density. 
And, $\pnnavg{q_\alpha}$ is the consitional average of $q_\alpha\delta_\alpha$ knowing a particle is present inside the volume $d\textbf{x}$ around the position \textbf{x} at time $t$. 
\JL{meme remarque que precedemment}

%\subsubsection*{Continuous and particle averaged equations}
\subsection{Phase average and particle averaged equations}
\JL{j'ai legerement modifie le titre. Je pense que phase average est plus parlant que continuous average}

Applying \ref{eq:avg} on \ref{eq:dt_chi_k_f_k} and \ref{eq:dt_delta_I_f_I} and considering the properties from \ref{eq:avg_properties} yields the general form of the averaged equations of multiphase flows, namely,
\begin{align}
    \pddt \avg{\chi_k f_k}
    &= \nablabh \cdot \avg{\chi_k \mathbf{\Phi}_k - \chi_k f_k \textbf{u}_k}
    + \avg{\chi_k \textbf{S}_k}
    + \avg{\delta_I\left[
        \mathbf{\Phi}_k
        + f_k
        \left(
            \textbf{u}_I
            - \textbf{u}_k
        \right)
    \right]
    \cdot \textbf{n}_k} ,
    \label{eq:avg_dt_chi_f}\\
    \pddt \avg{\delta_If_I}
    &= 
    \nablabh \cdot \avg{\delta_I \mathbf{\Phi}_{I||} - \delta_I f_I \textbf{u}_I}
    +\avg{\delta_I\textbf{S}_I} 
    - \avg{\delta_I \Jump{
    f_k (\textbf{u}_I - \textbf{u}_k)
    + \mathbf{\Phi}_k
    } }.
    \label{eq:avg_dt_delta_f}
\end{align}
Together \ref{eq:avg_dt_delta_f}  and \ref{eq:avg_dt_delta_f} \JL{erreur de numerotation} form the \textit{two-fluid} formulation of averaged multiphase flows problem. 
Besides, on can derive the so-called averaged \textbf{single-fluid} formulation by applying \ref{eq:avg} on \ref{eq:dt_f} which directly gives us, 
\begin{equation}
    \pddt \avg{f}
    = \nablab \cdot \avg{\mathbf{\Phi} - f \textbf{u}}
    + \avg{\textbf{S}}.
    \label{eq:avg_dt_f}
\end{equation}
It is interesting to notice that in Volume Of Fluid method we solve for the fields averaged on a cell volume, thus we actually discretize \ref{eq:avg_dt_f} and solve for the averaged quantities \citep{popinet2018numerical,tryggvason2011direct} instead of directly solving the local transport equations \ref{eq:dt_f}.

Correspondingly, we take the average of the particle fields equations by using \ref{eq:avg} on \ref{eq:dt_dq_alpha} and \ref{eq:dt_dq_I_alpha}, giving, 
\begin{multline}
    \pddt \pavg{ q_\alpha}
    + \nablabh\cdot \pavg{ q_\alpha \textbf{u}_\alpha}
    = \pavg{\int_{\Omega_\alpha} \textbf{S}_k d\Omega}\\
    + \pavg{\int_{\Sigma_\alpha} \left[\mathbf{\Phi}_k + f_k (\textbf{u}_I-\textbf{u}_k) \right] \cdot \textbf{n}_k d\Sigma},
    \label{eq:avg_dt_dq_alpha}
\end{multline}
\begin{multline}
    \pddt \pavg{ q_{I\alpha}}
    + \nablabh\cdot \pavg{ q_{I\alpha} \textbf{u}_\alpha}
    = \pavg{\int_{\Sigma_\alpha} 
    \textbf{S}_I
    d\Sigma}\\
    - \pavg{\int_{\Sigma_\alpha} \Jump{
        f_k (\textbf{u}_I - \textbf{u}_k)
        + \mathbf{\Phi}_k}
        d\Sigma}.
        \label{eq:avg_dt_dq_I_alpha}
\end{multline}
\JL{les phrases qui suivaient sur le Euler-Euler n'apportaient rien. Je les ai commentees.}
%When considering a dispersed multiphase flows problem we use Euler-Euler method to solves the equations of the continuous and dispersed phase numerically. 
%In those code we solve the equations for the cell averaged values, thus we should consider discretizing \ref{eq:avg_dt_dq_I_alpha} and \ref{eq:avg_dt_dq_alpha} while solving the dispersed phase equations.

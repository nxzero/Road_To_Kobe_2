\section{Singularity solution for capillary motion of a droplet}

At the leading order in droplet volume fraction, the closure problem is equivalent to that of an isolated droplet in an infinite medium \citet{hinch1977averaged}. 
Hence, we consider the problem of an isolated droplet, immersed in an arbitrary quadratic flow without the presence of fluid inertia. 
The disturbances pressure and velocity field are noted $\textbf{u}_{out}$, $\textbf{u}_{in}$, $p_{out}$ and $p_{in}$, for the velocity outside the droplet, the velocity inside the droplet, the pressure outside the droplet and the pressure inside the droplet, respectively. 
The corresponding ``undisturbed field'' are noted $\textbf{u}$ and $p$. 

In dimensionless form ($\textbf{u}_{out}$, $\textbf{u}_{in}$) are governed by the Stokes equations, namely, 
\begin{align}
    \div \textbf{u}_{in} = 0 
    && \div \textbf{u}_{out} = 0 \\
    -\grad p_{in} + \grad^2 \textbf{u}_{in}  = 0 
    &&-\grad p_{out} + \grad^2 \textbf{u}_{out}  = 0 
\end{align}
As these field are disturbance field they follow,
\begin{equation}
    \lim_{r\to \infty}(\textbf{u}_{out},p_{out}) = 0 
\end{equation}
far from the droplet centered at the origin ($\textbf{x} = 0$). 
At the surface of the droplet ($r = 1$) the continuity of velocity and the far field condition impose, 
\begin{align}
    \textbf{u}_{in} = \textbf{u}_{out}
    && 
    \textbf{u}_{in} \cdot \textbf{n}
    =
    - \textbf{u}\cdot \textbf{n}
    \label{eq:boundary_cdt_vel}
\end{align}
where it must be noted that both $\textbf{u}_{in}$ and $\textbf{u}$ are evaluated at the points on the surface of the droplet, hence $\textbf{u}$ may not always be a constant vector. 

Let us introduce the dimensionless stress tensor $\bm\sigma_{in/out} = -p_{in/out} \bm\delta + \grad \textbf{u}_{in/out} + ^\dagger \grad \textbf{u}_{in/out}$ and the dimensionless shear stress tensor, $\bm\tau_{in/out} = \grad \textbf{u}_{in/out} + ^\dagger \grad \textbf{u}_{in/out}$. 
The undisturbed shear stress reads  $\bm\tau = \grad \textbf{u} + ^\dagger \grad \textbf{u}$
With these notations the tangential shear stress jump reads at $r = 1$ reads as, 
\begin{equation}
    \mathbf{n}\cdot (\bm\tau_{out} - \lambda \bm\tau_{in}+\bm\tau -\lambda\bm\tau)\cdot (\bm\delta - \textbf{nn})
    % +
    % (1-\lambda)\mathbf{n}\cdot \bm\tau \cdot (\bm\delta - \textbf{nn})
    =Ma  (\bm\delta - \textbf{nn})\cdot \grad \gamma
    \label{eq:boundary_cdt_stress}
\end{equation}
where Ma is the Marangoni number defined as \citep{Kawakami_Vlahovska_2025} which represent the effect of non-uniform surface tension. 


In the most general case senario $\grad \gamma$ as well as $\textbf{u}$ are not constant vectors. 
However, in the disturbance field problem it is assumed that these quantities varies very slowly compared to the droplet radius, on which we evaluate these fields. 
Hence, we may introduce the following relationship,
\begin{align*}
    \textbf{u}(\textbf{n}) 
    &=  \textbf{u}|_{\textbf{x}=0}
    +  \textbf{r} \cdot  \grad\textbf{u}|_{\textbf{x}=0}
    +  \frac{1}{2}\textbf{rr} \cdot  \grad\grad\textbf{u}|_{\textbf{x}=0}
    + \ldots\\
     \bm\tau(\textbf{n}) 
    &=   \bm\tau|_{\textbf{x}=0}
    + \textbf{r} \cdot  \grad \bm\tau|_{\textbf{x}=0}
    + \frac{1}{2}\textbf{rr} \cdot  \grad\grad \bm\tau|_{\textbf{x}=0}
    + \ldots\\
    \grad \gamma (\textbf{n}) 
    &=  \grad \gamma|_{\textbf{x}=0}
    +  \textbf{r} \cdot  \grad\grad \gamma|_{\textbf{x}=0}
    +  \frac{1}{2}\textbf{rr} \cdot  \grad\grad\grad \gamma|_{\textbf{x}=0}. 
\end{align*}
\tb{Each gradient posses their own dimensionless number}
We deduce that the disturbance fields are linear with the constant tensor $\textbf{u}|_{\textbf{x}=0}$, $\grad \gamma|_{\textbf{x}=0}$ and their derivatives through the boundary conditions \ref{eq:boundary_cdt_stress} and \ref{eq:boundary_cdt_vel}. 
Hence, we deduce that $\textbf{u}_{in/out}$ and $p_{in/out}$  must be linear combination of spherical harmonics function proportional to $\textbf{u}|_{\textbf{x}=0}$, $\grad \gamma|_{\textbf{x}=0}$, and their derivatives. 


Using the standard procedure of spherical harmonics we can stipulate that \citep{leal2007advanced,raja2010inertial}, 
\begin{align*}
    p_{out}
    &=
    \{\mathcal{P}_{u\text -out}^{(1)} + \mathcal{P}_{u\text -out}^{(2)}\cdot \grad + \mathcal{P}_{u\text -out}^{(3)} :\grad\grad\}\cdot \textbf{u} 
    +
    \{\mathcal{P}_{\gamma\text -out}^{(1)} + \mathcal{P}_{\gamma\text -out}^{(2)}\cdot \grad + \mathcal{P}_{\gamma\text -out}^{(3)} :\grad\grad\} \cdot \grad \gamma\\
    (\textbf{u}_{out})_i
    &=
    \{\mathcal{U}_{u\text -out}^{(1)} + \mathcal{U}_{u\text -out}^{(2)}\cdot \grad + \mathcal{U}_{u\text -out}^{(3)} :\grad\grad\}\cdot \textbf{u} 
    +
    \{\mathcal{U}_{\gamma\text -out}^{(1)} + \mathcal{U}_{\gamma\text -out}^{(2)}\cdot \grad + \mathcal{U}_{\gamma\text -out}^{(3)} :\grad\grad\} \cdot \grad \gamma\\
\end{align*}
where $\mathcal{P}_{out}^{(n)},\mathcal{U}_{out}^{(n)}$ are tensor of order $n$ and $n+1$ respectively that are function of $\textbf{r}$ and the viscosity ratio $\lambda$. 
Because the pressure is an harmonics function the tensor $\mathcal{P}_{(.)}^{(n)}$  take the form, 
\begin{equation}
    \mathcal{P}_{(.)}^{(2)}
    =
    \left(\frac{1}{r}\right)\bm\delta
    + 
    \grad\grad \left(\frac{1}{r}\right) \\
    \mathcal{P}_{(.)}^{(3)}
    =
    \left(\frac{1}{r}\right)\bm\delta
    + 
    \grad\grad \left(\frac{1}{r}\right) \\
\end{equation}

Similar consideration can be applied to the interior velocity fields and pressure, yielding, 
\begin{align*}
    \begin{pmatrix}
        \textbf{u}_{out}\\
        p_{out}\\
        \textbf{u}_{in}\\
        p_{in}
    \end{pmatrix}
    =
    \begin{pmatrix}
        \mathcal{U}_{u\text -out}^{(1)} + \mathcal{U}_{u\text -out}^{(2)}\cdot \grad + \mathcal{U}_{u\text -out}^{(3)} :\grad\grad &
        \mathcal{U}_{\gamma\text -out}^{(1)} + \mathcal{U}_{\gamma\text -out}^{(2)}\cdot \grad + \mathcal{U}_{\gamma\text -out}^{(3)} :\grad\grad \\
        \mathcal{P}_{u\text -out}^{(1)} + \mathcal{P}_{u\text -out}^{(2)}\cdot \grad + \mathcal{P}_{u\text -out}^{(3)} :\grad\grad &
        \mathcal{P}_{\gamma\text -out}^{(1)} + \mathcal{P}_{\gamma\text -out}^{(2)}\cdot \grad + \mathcal{P}_{\gamma\text -out}^{(3)} :\grad\grad \\
        \mathcal{U}_{u\text-in}^{(1)} + \mathcal{U}_{u\text-in}^{(2)}\cdot \grad + \mathcal{U}_{u\text-in}^{(3)} :\grad\grad &
        \mathcal{U}_{\gamma\text-in}^{(1)} + \mathcal{U}_{\gamma\text-in}^{(2)}\cdot \grad + \mathcal{U}_{\gamma\text-in}^{(3)} :\grad\grad \\
        \mathcal{P}_{u\text-in}^{(1)} + \mathcal{P}_{u\text-in}^{(2)}\cdot \grad + \mathcal{P}_{u\text-in}^{(3)} :\grad\grad &
        \mathcal{P}_{\gamma\text-in}^{(1)} + \mathcal{P}_{\gamma\text-in}^{(2)}\cdot \grad + \mathcal{P}_{\gamma\text-in}^{(3)} :\grad\grad \\
    \end{pmatrix}
    \cdot 
    \begin{pmatrix}
        \textbf{u}\\
        \grad \gamma
    \end{pmatrix}
\end{align*}

The tensor $\mathcal{U}_{u\text -out}^{(1)},\mathcal{U}_{u\text -in}^{(1)},\mathcal{P}_{u\text -out}^{(1)}$  and $\mathcal{P}_{u\text -in}^{(1)}$ represent the effect of uniform relative translation on the disturbance fields, hence these expression to correspond to the Hadamard-Ribczynski solution \citep{pozrikidis1992boundary,kim2013microhydrodynamics}. 
The dependency with the mean gradient velocity $\grad \textbf{u}$ is defined through the tensor $\mathcal{U}_{u\text -out}^{(2)},\mathcal{U}_{u\text -in}^{(2)},\mathcal{P}_{u\text -out}^{(2)}$ and $\mathcal{P}_{u\text -in}^{(2)}$.
The solution of a drop in pure linear flow have been derived in \citet{rallison1978note,leal2007advanced,raja2010inertial}. 
Finally the disturbance field of a droplet in quadratic flow may be defined through the tensor $\mathcal{U}_{u\text -out}^{(3)},\mathcal{U}_{u\text -in}^{(3)},\mathcal{P}_{u\text -out}^{(3)}$ and $\mathcal{P}_{u\text -in}^{(3)}$.
The definition of the fomer tensor may be found in \citet{nadim1991motion}.


Regarding the effect of non-uniform surface tension on the disturbance fields, solution are less commun. 
For instance one might refer to \citet{Subramanian_1985} we they compute the disturbance fields due to a constant surface tension gradient, hence defining  $\mathcal{U}_{\gamma\text -out}^{(1)},\mathcal{U}_{\gamma\text -in}^{(1)},\mathcal{P}_{\gamma\text -out}^{(1)}$  and $\mathcal{P}_{\gamma\text -in}^{(1)}$. 
Then no-study could be found that express the disturbance fields, in term of $\grad\grad \gamma$ or higher order derivatives\footnote{In \citet[Appendix C]{raja2010inertial} 
They provide a solution which could correspond to $\mathcal{U}_{\gamma\text -out}^{(1)},\mathcal{U}_{\gamma\text -in}^{(1)},\mathcal{P}_{\gamma\text -out}^{(1)}$  and $\mathcal{P}_{\gamma\text -in}^{(1)}$ but only in the case $\grad \cdot \grad \gamma = 0$ since their tensor \textbf{B} is assumed traceless. 
} . 


Note that only the surface force at $r=1$ is used to compute the closure terms, hence we provide they explicite expression here,
\begin{align*}
    (\bm\sigma_{out}\cdot \textbf{n}|_{r=1})_{i}
    &=
    \{\mathcal{S}_{u\text -out}^{(1)} + \mathcal{S}_{u\text -out}^{(2)}\cdot \grad + \mathcal{S}_{u\text -out}^{(3)} :\grad\grad\}\cdot \textbf{u} 
    +
    \{\mathcal{S}_{\gamma\text -out}^{(1)} + \mathcal{S}_{\gamma\text -out}^{(2)}\cdot \grad + \mathcal{S}_{\gamma\text -out}^{(3)} :\grad\grad\} \cdot \grad \gamma\\
\end{align*}
with,
\begin{align*}
    \mathcal{S}^{(1)}_{u\text-out} 
    &= 3\frac{1}{\lambda + 1}x{}^{i}x{}^{j} 
    + \left(\frac{3}{2}\right)\frac{\lambda}{\lambda + 1}\delta^{ij} \\
    \mathcal{S}^{(2)}_{u\text-out} 
    &=
    8\frac{1}{\lambda + 1}x{}^{i}x{}^{j}x{}^{k} 
    + \left(\frac{1}{2}\right)\frac{1}{\lambda + 1}\left(3 \lambda - 2\right)(\delta{}^{ij}x{}^{k} +\delta{}^{ik}x{}^{j})\\
    \mathcal{S}^{(3)}_{u\text-out} 
    &=
    -\left(\frac{1}{4}\right)\frac{\lambda^{2} + 15 \lambda + 4}{\lambda^{2} + 5 \lambda + 4}\delta{}^{jk}x{}^{i}x{}^{l} 
    - \left(\frac{1}{8}\right)\frac{\lambda\left(3 \lambda - 8\right)}{\lambda^{2} + 5 \lambda + 4}\delta{}^{il}\delta{}^{jk} \\
    &+ \left(\frac{1}{8}\right)\frac{\left(17 \lambda^{2} + 20 \lambda - 32\right)}{\lambda^{2} + 5 \lambda + 4}\delta{}^{il}x{}^{j}x{}^{k} 
    + \left(\frac{1}{8}\right)\frac{\left(\lambda^{2} + 20 \lambda - 16\right)}{\lambda^{2} + 5 \lambda + 4}(\delta{}^{ij}x{}^{k}x{}^{l} +\delta{}^{ik}x{}^{j}x{}^{l} )\\
    % &+ \left(\frac{1}{8}\right)\frac{\left(\lambda^{2} + 20 \lambda - 16\right)}{\lambda^{2} + 5 \lambda + 4}
    &+ \left(\frac{25}{4}\right)\frac{1}{\lambda + 1}x{}^{i}x{}^{j}x{}^{k}x{}^{l}\\
    \mathcal{S}^{(1)}_{\gamma\text-out} 
    &=
    -\frac{2}{\lambda + 1}x{}^{i}x{}^{j} 
    + \frac{1}{\lambda + 1}\delta{}^{ij}\\
    \mathcal{S}^{(2)}_{\gamma\text-out} 
    &=
    -\left(\frac{8}{5}\right)\frac{1}{\lambda + 1}x{}^{i}x{}^{j}x{}^{k} 
    + \left(\frac{1}{2}\right)\frac{1}{\lambda + 1}(\delta{}^{ij}x{}^{k} +\delta{}^{ik}x{}^{j})
    + \left(\frac{1}{5}\right)\frac{1}{\lambda + 1}\delta{}^{jk}x{}^{i}\\
    \mathcal{S}^{(3)}_{\gamma\text-out} 
    &=
    -\left(\frac{5}{7}\right)\frac{1}{\lambda + 1}x{}^{i}x{}^{j}x{}^{k}x{}^{l} 
    +\left(\frac{1}{105}\right)\frac{1}{\lambda + 1}(\delta{}^{jk}x{}^{i}x{}^{l}+\delta{}^{jl}x{}^{i}x{}^{k}+\delta{}^{kl}x{}^{i}x{}^{j})\\ 
    % &+\left(\frac{1}{105}\right)\frac{1}{\lambda + 1} 
    % +\left(\frac{1}{105}\right)\frac{1}{\lambda + 1} \\
    &+\left(\frac{1}{6}\right)\frac{1}{\lambda + 1}(\delta{}^{ij}x{}^{k}x{}^{l} +\delta{}^{ik}x{}^{j}x{}^{l}+\delta{}^{il}x{}^{j}x{}^{k})
    % +\left(\frac{1}{6}\right)\frac{1}{\lambda + 1} \\
    % &+\left(\frac{1}{6}\right)\frac{1}{\lambda + 1}
\end{align*}


It will also be of interest to compute the surface velocity, 
% \begin{align*}
%     \mathcal{A}^{(1)}_{u\text-out} 
%     &=
%     - \frac{1}{\lambda + 1}\left(\lambda + \frac{1}{2}\right)(\delta^{ij}x{}^{p} + \delta^{jp}x{}^{i} )
%     - \frac{1}{\lambda + 1}x{}^{i}x{}^{j}x{}^{p}\\
%     \mathcal{A}^{(2)}_{u\text-out} 
%     &=
%     -2\frac{1}{\lambda + 1}x{}^{i}x{}^{j}x{}^{k}x{}^{p} 
%     - \left(\frac{1}{2}\right)\frac{\lambda}{\lambda + 1}
%     (\delta^{ij}x{}^{k}x{}^{p} 
%     + \delta^{ik}x{}^{j}x{}^{p} 
%     + \delta^{jp}x{}^{i}x{}^{k} 
%     + \delta^{kp}x{}^{i}x{}^{j})\\
%     \mathcal{A}^{(3)}_{u\text-out} 
%     &=
%     - \left(\frac{1}{8}\right)\frac{1}{\lambda^{2} + 5 \lambda + 4}\left(4 \lambda^{2} + 5 \lambda - 4\right)(\delta^{il}x{}^{j}x{}^{k}x{}^{p} 
%     + \delta^{lp}x{}^{i}x{}^{j}x{}^{k} )
%     - \left(\frac{5}{4}\right)\frac{1}{\lambda + 1}x{}^{i}x{}^{j}x{}^{k}x{}^{l}x{}^{p} \\
%     &- \left(\frac{5}{8}\right)\frac{\lambda}{\lambda^{2} + 5 \lambda + 4}(
%                 \delta^{ij}x{}^{k}x{}^{l}x{}^{p} 
%                 + \delta^{ik}x{}^{j}x{}^{l}x{}^{p} 
%                 + \delta^{jp}x{}^{i}x{}^{k}x{}^{l} 
%                 + \delta^{kp}x{}^{i}x{}^{j}x{}^{l}
%                 )\\
%     &- \left(\frac{5}{8}\right)\frac{\lambda}{\lambda^{2} + 5 \lambda + 4}
%     (\delta^{il}\delta^{jk}x{}^{p} 
%     + \delta^{jk}\delta^{lp}x{}^{i} )
%     + \left(\frac{5}{4}\right)\frac{\lambda}{\lambda^{2} + 5 \lambda + 4} \delta^{jk}x{}^{i}x{}^{l}x{}^{p}\\
%     \mathcal{A}^{(1)}_{\gamma\text-out} 
%     &=
%     - \left(\frac{1}{3}\right)\frac{1}{\lambda + 1}(\delta^{ij}x{}^{p} + \delta^{jp}x{}^{i}) 
%     + \left(\frac{2}{3}\right)\frac{1}{\lambda + 1}x{}^{i}x{}^{j}x{}^{p}\\
%     \mathcal{A}^{(2)}_{\gamma\text-out} 
%     &=
%     - \left(\frac{1}{10}\right)\frac{1}{\lambda + 1}(\delta^{ij}x{}^{k}x{}^{p} 
%         + \delta^{ik}x{}^{j}x{}^{p} 
%         + \delta^{jp}x{}^{i}x{}^{k} 
%         + \delta^{kp}x{}^{i}x{}^{j} )
%     + \left(\frac{2}{5}\right)\frac{1}{\lambda + 1}x{}^{i}x{}^{j}x{}^{k}x{}^{p}\\
%     \mathcal{A}^{(3)}_{\gamma\text-out} 
%     &=
%     - \left(\frac{1}{42}\right)\frac{1}{\lambda + 1}(\delta^{ij}x{}^{k}x{}^{l}x{}^{p} 
%             +\delta^{ik}x{}^{j}x{}^{l}x{}^{p} 
%             +\delta^{il}x{}^{j}x{}^{k}x{}^{p} 
%             +\delta^{jp}x{}^{i}x{}^{k}x{}^{l} 
%             +\delta^{kp}x{}^{i}x{}^{j}x{}^{l} 
%             +\delta^{lp}x{}^{i}x{}^{j}x{}^{k} )\\
%     &- \left(\frac{2}{315}\right)\frac{1}{\lambda + 1}(\delta^{ij}\delta^{kl}x{}^{p} 
%         +\delta^{ik}\delta^{jl}x{}^{p} 
%         +\delta^{il}\delta^{jk}x{}^{p} 
%         +\delta^{jk}\delta^{lp}x{}^{i} 
%         +\delta^{jl}\delta^{kp}x{}^{i} 
%         +\delta^{jp}\delta^{kl}x{}^{i} )\\
%     &+ \left(\frac{4}{315}\right)\frac{1}{\lambda + 1}(\delta^{jk}x{}^{i}x{}^{l}x{}^{p} 
%      + \delta^{jl}x{}^{i}x{}^{k}x{}^{p} 
%      + \delta^{kl}x{}^{i}x{}^{j}x{}^{p})
%     + \left(\frac{1}{7}\right)\frac{1}{\lambda + 1}x{}^{i}x{}^{j}x{}^{k}x{}^{l}x{}^{p} 
% \end{align*}
\begin{align*}
    \mathcal{U}_{u}^{(1)}
    &=
    -\frac{1}{\lambda + 1}\left(\lambda + \frac{1}{2}\right)\delta^{ij} - \left(\frac{1}{2}\right)\frac{1}{\lambda + 1}x{}^{i}x{}^{j}
    \\ \mathcal{U}_{u}^{(2)}
    &=
    -\frac{1}{\lambda + 1}x{}^{i}x{}^{j}x{}^{k}  - \left(\frac{1}{2}\right)\lambda\frac{1}{\lambda + 1}(\delta^{ij}x{}^{k} + \delta^{ik}x{}^{j})
    \\ \mathcal{U}_{u}^{(3)}
    &=
    - \left(\frac{5}{8}\right)\frac{1}{\lambda + 1}x{}^{i}x{}^{j}x{}^{k}x{}^{l} 
    - \left(\frac{1}{8}\right)\frac{1}{\lambda^{2} + 5 \lambda + 4}\left(4 \lambda^{2} + 5 \lambda - 4\right)\delta^{il}x{}^{j}x{}^{k} \\
    &- \left(\frac{5}{8}\right)\lambda\frac{1}{\lambda^{2} + 5 \lambda + 4}(\delta^{ij}x{}^{k}x{}^{l} + \delta^{ik}x{}^{j}x{}^{l} + \delta^{il}\delta^{jk} - \delta^{jk}x{}^{i}x{}^{l}) 
    \\ \mathcal{U}_{\gamma}^{(1)}
    &=
    -\left(\frac{1}{3}\right)\frac{1}{\lambda + 1}\delta^{ij} 
    + \left(\frac{1}{3}\right)\frac{1}{\lambda + 1}x{}^{i}x{}^{j}
    \\ \mathcal{U}_{\gamma}^{(2)}
    &=
    - \left(\frac{1}{10}\right)\frac{1}{\lambda + 1}(\delta^{ij}x{}^{k} + \delta^{ik}x{}^{j}) 
    + \left(\frac{1}{5}\right)\frac{1}{\lambda + 1}x{}^{i}x{}^{j}x{}^{k}
    \\ \mathcal{U}_{\gamma}^{(3)}
    &=
    -\left(\frac{1}{42}\right)\frac{1}{\lambda + 1}(\delta^{ij}x{}^{k}x{}^{l} 
     + \delta^{ik}x{}^{j}x{}^{l} 
     + \delta^{il}x{}^{j}x{}^{k} )
    + \left(\frac{1}{14}\right)\frac{1}{\lambda + 1}x{}^{i}x{}^{j}x{}^{k}x{}^{l} \\
    &+ \left(\frac{2}{315}\right)\frac{1}{\lambda + 1}(
    -\delta^{ij}\delta^{kl} 
    - \delta^{ik}\delta^{jl} 
    - \delta^{il}\delta^{jk} 
    + \delta^{jk}x{}^{i}x{}^{l} 
    + \delta^{jl}x{}^{i}x{}^{k} 
    + \delta^{kl}x{}^{i}x{}^{j})
\end{align*}

\subsection{Compute the force traction terms}

In the main text one needs to integrate the force traction of the disturbance field $\bm\sigma_{out}$ over the surface of the droplet to obtain the closure. 
Particularily we need to compute, these three integrals, 
\begin{align*}
    \intS{(\bm\sigma_{out}\cdot \textbf{n})_i}
    && \intS{x_m (\bm\sigma_{out}\cdot \textbf{n})_i}
    && \intS{x_n x_m (\bm\sigma_{out}\cdot \textbf{n})_i}
\end{align*}
which are the three first hydrodynamic moments of the force traction. 

To do so one needs to notice that by integrating over the unit sphere we obtain, 
\begin{align*}
    % \intS{x_i} &= 0\\
    \frac{1}{4\pi}\intS{\delta_{ij}} &=1\\
    \frac{3}{4\pi}\intS{x_ix_j} &= 1\\
    % \intS{x_ix_jx_k} &= 0 \\
    \frac{15}{4\pi}\intS{x_ix_jx_kx_l} &= (\delta_{ij}\delta_{kl} + \delta_{ik}\delta_{jl} + \delta_{il}\delta_{kj}) \\
    \frac{105}{4\pi}\intS{x_ix_jx_kx_lx_mx_n} 
    &= 
    \delta_{nm}(
        \delta_{ij}\delta_{kl} 
        + \delta_{ik}\delta_{jl} 
        + \delta_{il}\delta_{kj}
        ) \\
    &+ 
    \delta_{nl}(
        \delta_{ij}\delta_{km} 
        + \delta_{ik}\delta_{jm} 
        + \delta_{im}\delta_{kj}
        ) \\
    &+ 
    \delta_{nk}(
        \delta_{ij}\delta_{ml} 
        + \delta_{im}\delta_{jl} 
        + \delta_{il}\delta_{mj}
        ) \\
    &+ 
    \delta_{nj}(
        \delta_{im}\delta_{kl} 
        + \delta_{ik}\delta_{ml} 
        + \delta_{il}\delta_{km}
        ) \\
    &+ 
    \delta_{ni}(
        \delta_{mj}\delta_{kl} 
        + \delta_{mk}\delta_{jl} 
        + \delta_{ml}\delta_{kj}
        ) \\
\end{align*}
\tb{maybe the 105 is wrong}
Additionally all odd order tesnor $x_i$ or $x_ix_jx_k$, integrated over the surface of the sphere goes to zero. 
\begin{align*}
    \intS{\bm\sigma_{out}\cdot \textbf{n}} &
    =
    2\pi\frac{2+3\lambda}{1+\lambda}\textbf{u}_r
    + \pi \frac{\lambda}{\lambda +1} \grad^2 \textbf{u}
    +
    \frac{4\pi}{3}\frac{1}{\lambda +1}\grad \gamma
    + \frac{2\pi}{15(\lambda +1)}\grad^2(\grad\gamma)
    \\
    \intS{ \textbf{x}\bm\sigma_{out}\cdot \textbf{n}} &
    =
    \frac{2\pi(5\lambda +2)}{5(\lambda +1)}[\grad \textbf{u}+ (\grad \textbf{u})^\dagger]
    + \frac{12\pi}{25(\lambda +1)}\grad\grad \gamma
    - \frac{4\pi}{25(\lambda +1)}\bm\delta(\grad\cdot\grad) \gamma
    \\
    \intS{\textbf{xx}\bm\sigma_{out}\cdot \textbf{n}} &
    =
    \frac{4\pi}{5(\lambda +1)} (\textbf{u}_r \bm\delta + \bm\delta \textbf{u}_r)
    + \frac{2\pi(5\lambda +2)}{5(\lambda+1)}\bm\delta \textbf{u}_r
    - \frac{8\pi}{15(\lambda +1)} (\grad \gamma \bm\delta + \bm\delta \grad \gamma)
    + \frac{4\pi}{5(\lambda+1)}\bm\delta \grad \gamma
    \\
\end{align*}

It will also be useful to compute the droplet internal shear for the closure problem. 
Hence one needs the velocity at the droplet interface. 
\begin{align}
    \intO{\bm\tau_{in}}
    =
    \intO{[
        \grad \textbf{u}_{in} 
        + 
        (\grad \textbf{u}_{in})^\dagger
    ]}
    =
    \intS{(\textbf{n} \textbf{u}_{in} + \textbf{u}_{in} \textbf{n})}\\
    \intO{\bm\tau_{in}}
    =
    \intO{[
        \grad \textbf{u}_{in} 
        + 
        (\grad \textbf{u}_{in})^\dagger
    ]}
    =
    \intS{(\textbf{n} \textbf{u}_{in} + \textbf{u}_{in} \textbf{n})}
\end{align}

Thus, we may compute that gives,
\begin{align*}
    \intS{(\textbf{n} \textbf{u}_{in} + \textbf{u}_{in} \textbf{n})}
    &=
    -\frac{4\pi}{15}\frac{5\lambda +2}{\lambda+1}
    [\grad \textbf{u}+ (\grad \textbf{u})^\dagger]
    - \frac{8\pi}{25}\frac{1}{\lambda+1}
    \grad\grad \gamma
    + \frac{8\pi}{75}\frac{1}{\lambda+1}
    \bm\delta(\grad\cdot\grad) \gamma\\
    \intS{\textbf{r}(\textbf{n} \textbf{u}_{in} + \textbf{u}_{in} \textbf{n})}
    &=
    -\frac{2\pi}{15}\frac{10\lambda +7}{\lambda+1}
    (\bm\delta \textbf{u}_r + \textbf{u}_r \bm\delta)
    -\frac{4\pi}{15}\frac{1}{\lambda+1}\bm\delta \textbf{u}_r
    -\frac{4\pi}{15}\frac{1}{\lambda+1}
    (\bm\delta \grad \gamma + \grad \gamma \bm\delta)\\
    &+\frac{8\pi}{45}\frac{1}{\lambda+1}
    \bm\delta \grad \gamma
\end{align*}
\begin{align*}
    \intS{2\textbf{e}_{in}}
    &=
    -\frac{4\pi}{15}\frac{5\lambda +2}{\lambda+1}
    [\grad \textbf{u}+ (\grad \textbf{u})^\dagger]
    - \frac{8\pi}{25}\frac{1}{\lambda+1}
    \grad\grad \gamma
    + \frac{8\pi}{75}\frac{1}{\lambda+1}
    \bm\delta(\grad\cdot\grad) \gamma\\
    \intS{2\textbf{r}\textbf{e}_{in}}
    &=
    -\frac{2\pi}{15}\frac{20\lambda + 17}{\lambda+1}
    (\bm\delta \textbf{u}_r + \textbf{u}_r \bm\delta)
    -\frac{4\pi}{15}\frac{1}{\lambda+1}\bm\delta \textbf{u}_r
    -\frac{4\pi}{15}\frac{1}{\lambda+1}
    (\bm\delta \grad \gamma + \grad \gamma \bm\delta)\\
    &+\frac{8\pi}{45}\frac{1}{\lambda+1}
    \bm\delta \grad \gamma
\end{align*}

\subsubsection*{Deformation of the droplets}
The usual procedure to determine the droplet deformation is to use the normal stress balance. 
However to point out the meaning of the first- and second-moment of momentum here we introduce an alternative approaches based on Lagrangian balance laws. 

\begin{multline}
    \intS{ (\bm{\sigma}_\Gamma^0)_{ik}}
    +\intO{ (\bm{\sigma}_d^0)_{ik}}
    = 
    % \intO{ \rho_d 
    % (\textbf{w}_d^0\textbf{w}_d^0  )_{ik}
    % }
    % -\frac{1}{2}\left(\frac{d^2 \textbf{M}_\alpha}{dt^2} \right)_{ik}\\
    % +\frac{1}{2}\intO{ \left[
    %     (\textbf{b}_d^0)_i
    %     r_k 
    %     + (\textbf{b}_d^0)_k
    %     r_i
    % \right]}
    +
    \frac{1}{2}\intS{ \left[
        (\bm{\sigma}_f^0 \cdot \textbf{n})_i r_k
        + (\bm{\sigma}_f^0 \cdot \textbf{n})_k r_i
    \right]
    }
\end{multline}
\begin{multline}
    \intO{ r_{j}(\bm{\sigma}^0_d)_{ik}+r_{k}(\bm{\sigma}^0_d)_{ji}}
    +\intS{ r_{j}(\bm{\sigma}^0_\Gamma)_{ik}+r_{k}(\bm{\sigma}_\Gamma^0)_{ji}}
    = 
    % - \ddt\intO{ \rho_d (\textbf{u}_d^0)_i r_j r_k }
    % \\
    % + \intO{ \left[
    %     \rho_d (\textbf{u}^0_d\textbf{r}\textbf{w}_d^0)_{ijk} + \rho_d (\textbf{u}^0_d\textbf{r}\textbf{w}_d^0)_{kji}
    % \right]}
    +\intS{  r_{k}r_{j} (\bm{\sigma}_f^0\cdot\textbf{n})_i }
    + \intO{ r_{k}r_{j}  \rho_d (\textbf{b}_d^0)_i } 
    % \label{eq:dt_P2_alpha_bis}
\end{multline}
the mean part of the stress then read
\begin{equation}
    \ldots=\intO{  r_{k}r_{j} (\div \bm{\Sigma})_i }
    +(1-\lambda)\intS{  (\bm{\Sigma})_{ik} r_{j}  }
    +(1-\lambda)\intS{  (\bm{\Sigma})_{ij} r_{k}  }
\end{equation}
Using the trik of \citet{lhuillier1996contribution} we may re-write the second moment of moment with the symmetry relation  $B_{ijk} + B_{jik} - B_{kij}$  which yiel, 
Indeed, consider the 
\begin{align}
    \intO{2 r_{k}(\bm{\sigma}^0_d)_{ij}}
    +\intS{2r_{k}(\bm{\sigma}^0_\Gamma)_{ij}}
    &= 
    +\intS{
        r_{k}r_{j} (\bm{\sigma}_f^0\cdot\textbf{n})_i 
        + r_{k}r_{i} (\bm{\sigma}_f^0\cdot\textbf{n})_j 
        - r_{j}r_{i} (\bm{\sigma}_f^0\cdot\textbf{n})_k 
    }\\
    &
    + \intO{ 
        r_{k}r_{j}  \rho_d (\textbf{b}_d^0)_i 
        +r_{k}r_{i}  \rho_d (\textbf{b}_d^0)_j 
        -r_{j}r_{i}  \rho_d (\textbf{b}_d^0)_k 
        }
    % \label{eq:dt_P2_alpha_bis}
\end{align}


The traceless part (on the index $ij$) of this equation yields, 
\begin{align}
    \intO{2 r_{k}(\bm{\sigma}^0_d)_{ij}^{dev}}
    +\intS{2r_{k}(\bm{\sigma}^0_\Gamma)_{ij}^{dev}}
    &= 
    +\intS{
        r_{k}r_{j} (\bm{\sigma}_f^0\cdot\textbf{n})_i 
        + r_{k}r_{i} (\bm{\sigma}_f^0\cdot\textbf{n})_j 
        - r_{j}r_{i} (\bm{\sigma}_f^0\cdot\textbf{n})_k 
    }\\
    &-\frac{1}{3}\delta_{ij}\intS{
        r_{k}r_{l} (\bm{\sigma}_f^0\cdot\textbf{n})_l
        + r_{k}r_{l} (\bm{\sigma}_f^0\cdot\textbf{n})_l 
        - r_{l}r_{l} (\bm{\sigma}_f^0\cdot\textbf{n})_k 
    }\\
    &+\intO{  
        r_{k}r_{j} (\div \bm{\Sigma})_i 
        +r_{k}r_{i} (\div \bm{\Sigma})_j 
        -r_{j}r_{i} (\div \bm{\Sigma})_k 
        }\\
    &-\frac{1}{3}\delta_{ij}\intO{  
        r_{k}r_{l} (\div \bm{\Sigma})_l
        +r_{k}r_{l} (\div \bm{\Sigma})_l 
        -r_{l}r_{l} (\div \bm{\Sigma})_k 
        }\\
    &+\mu_f (1-\lambda)\intS{ 4\textbf{E}_{ij} r_{k}  }
    \\
    &
    + \rho_d \intO{ 
        r_{k}r_{j} (\textbf{b}_d^0)_i 
        +r_{k}r_{i} (\textbf{b}_d^0)_j 
        -r_{j}r_{i} (\textbf{b}_d^0)_k 
        }\\
    &-\rho_d \frac{1}{3}\delta_{ij} \intO{ 
        r_{k}r_{l} (\textbf{b}_d^0)_l 
        +r_{k}r_{l} (\textbf{b}_d^0)_l 
        -r_{l}r_{l} (\textbf{b}_d^0)_k 
        }
    % \label{eq:dt_P2_alpha_bis}
\end{align}


Because $\div\bm\Sigma = - \rho_f \textbf{g}$ one recover teh buoyancy forces for the remaining terms, but these cancel with the dependency of the secoond mom with the $\textbf{u}_r$ and thus contribute to nothing to the deformation. 

So in the end we are left with 
\begin{align}
    \intO{2 r_{k}(\bm{\sigma}')_{ij}^{dev}}
    +\intS{2r_{k}(\bm{\sigma}^0_\Gamma)_{ij}^{dev}}
    &= 
    +\intS{
        r_{k}r_{j} (\bm{\sigma}'_f\cdot\textbf{n})_i 
        + r_{k}r_{i} (\bm{\sigma}'_f\cdot\textbf{n})_j 
        - r_{j}r_{i} (\bm{\sigma}'_f\cdot\textbf{n})_k 
    }\\
    &-\frac{1}{3}\delta_{ij}\intS{
        r_{k}r_{l} (\bm{\sigma}'_f\cdot\textbf{n})_l
        + r_{k}r_{l} (\bm{\sigma}'_f\cdot\textbf{n})_l 
        - r_{l}r_{l} (\bm{\sigma}'_f\cdot\textbf{n})_k 
    }\\
    &+\mu_f (1-\lambda)\intS{ 4\textbf{E}_{ij} r_{k}  }
    \\
    &
    + (\rho_d-\rho_f) \intO{ 
        r_{k}r_{j} \textbf{g}_i 
        +r_{k}r_{i} \textbf{g}_j 
        -r_{j}r_{i} \textbf{g}_k 
        }\\
    &-(\rho_d-\rho_f) \frac{1}{3}\delta_{ij} \intO{ 
        r_{k}r_{l} \textbf{g}_l 
        +r_{k}r_{l} \textbf{g}_l 
        -r_{l}r_{l} \textbf{g}_k 
        }
    % \label{eq:dt_P2_alpha_bis}
\end{align}
\subsection{Volume jacobian}

let \textbf{M} be a point in the deformable particle then, 
\begin{equation}
    \textbf{M}
    =
    r(1+f(\varphi,\theta))\textbf{e}_r
    =
    r(1+f(\varphi,\theta))\begin{pmatrix}
        \cos\varphi \sin\theta\\
        \sin\varphi \sin\theta\\
        \cos\theta
    \end{pmatrix}
\end{equation}
let,
\begin{align}
    \textbf{e}_{\theta}
    =
    \frac{\partial \textbf{e}_r}{\partial \theta}
    = 
    \begin{pmatrix}
        \cos \varphi \cos\theta\\
        \sin \varphi \cos\theta\\
        -\sin\theta\\
    \end{pmatrix}
    &&
    \textbf{e}_{\varphi}
    =
    \frac{\partial \textbf{e}_r}{\partial \varphi}
    = 
    \begin{pmatrix}
        -\sin \varphi \sin\theta\\
        \cos \varphi \sin\theta\\
        0\\
    \end{pmatrix}\\
\end{align}

Hence, 
\begin{align}
    \frac{\partial \textbf{M}}{\partial r}
    &=
    (1+f)\textbf{e}_r
    \\
    \frac{\partial \textbf{M}}{\partial \varphi}
    &=
    r(f_\varphi \textbf{e}_r
    + (1 + f) \textbf{e}_{\varphi})
    \\
    \frac{\partial \textbf{M}}{\partial \theta}
    &=
    r(f_\theta \textbf{e}_r
    + (1 + f) \textbf{e}_{\theta})
\end{align}
The jacobian
\begin{align*}
    &|\frac{\partial \textbf{M}}{\partial (r,\varphi,\theta)}| =\\ 
    &\left|
    \begin{matrix}
        (1+f)\cos\varphi \sin\theta&
        r(f_\varphi \cos\varphi \sin\theta - (1 + f)\sin \varphi \sin\theta)&
        r(f_\theta  \cos\varphi \sin\theta + (1 + f)  \cos\varphi \cos\theta)\\
        (1+f)\sin\varphi \sin\theta&
        r(f_\varphi \sin\varphi \sin\theta + (1 + f) \cos \varphi \sin\theta)&
        r(f_\theta  \sin\varphi \sin\theta + (1 + f)  \sin\varphi \cos\theta)\\
        (1+f)\cos\theta&
        rf_\varphi \cos\theta &
        r(f_\theta \cos\theta - (1 + f)  \sin\theta)
    \end{matrix}
    \right|\\
\end{align*}
But also, 
\begin{align*}
    |\frac{\partial \textbf{M}}{\partial (r,\varphi,\theta)}|
    &=
    \left|
    \begin{matrix}
        \frac{\partial \textbf{M}}{\partial r}
        &\frac{\partial \textbf{M}}{\partial \varphi}
        &\frac{\partial \textbf{M}}{\partial \theta}
    \end{matrix}\right|
    =
    \frac{\partial \textbf{M}}{\partial r}
    \cdot
    \frac{\partial \textbf{M}}{\partial \varphi}
    \times 
    \frac{\partial \textbf{M}}{\partial \theta}\\
    &=
    r^2 (1+f)\textbf{e}_r\cdot
        (f_\varphi \textbf{e}_r + (1+f)\textbf{e}_\varphi)
        \times (f_\theta \textbf{e}_r + (1+f)\textbf{e}_\theta)\\
    &=
    r^2 (1+f)^2\textbf{e}_r\cdot
        ( f_\varphi \textbf{e}_r\times \textbf{e}_\theta + f_\theta \textbf{e}_\varphi \times  \textbf{e}_r  + (1+f)\textbf{e}_\varphi \times \textbf{e}_\theta)\\
        &= r^2 (1+f)^3 \sin\theta
\end{align*}

Because,
\begin{equation}
    \textbf{e}_r \cdot (\textbf{e}_\varphi \times \textbf{e}_\theta)=
    \sin\theta
\end{equation}
So that the final jacobian reads, 
\begin{equation}
    |\frac{\partial \textbf{M}}{\partial (r,\varphi,\theta)}|
    =
    (1+f)^3
    r^2 
    (\sin\theta)
\end{equation}


\subsection{Surface jacobian}

Let use the general parametrsiaition, 
\begin{equation}
    \textbf{M}(\theta,\varphi)
    =
    (1 + f(\theta,\varphi)) \textbf{e}_r
    =
    (1 + f(\theta,\varphi)) 
    \begin{pmatrix}
        \cos \varphi \sin\theta\\
        \sin \varphi \sin\theta\\
        \cos\theta\\
    \end{pmatrix}
\end{equation}
The vector in the basis reads, 
\begin{align}
    \frac{\partial \textbf{M}}{\partial \varphi}
    =
    f_\varphi \textbf{e}_r
    + (1 + f) \textbf{e}_{\varphi}
    \\
    \frac{\partial \textbf{M}}{\partial \theta}
    =
    f_\theta \textbf{e}_r
    + (1 + f) \textbf{e}_{\theta}
\end{align}
with,
\begin{align}
    \textbf{e}_{\theta}
    =
    \frac{\partial \textbf{e}_r}{\partial \theta}
    = 
    \begin{pmatrix}
        \cos \varphi \cos\theta\\
        \sin \varphi \cos\theta\\
        -\sin\theta\\
    \end{pmatrix}
    &&
    \textbf{e}_{\varphi}
    =
    \frac{\partial \textbf{e}_r}{\partial \varphi}
    = 
    \begin{pmatrix}
        -\sin \varphi \sin\theta\\
        \cos \varphi \sin\theta\\
        0 \\
    \end{pmatrix}\\
\end{align}
Hence the general formulation of the Jacobi determinant is the norm of the vector,
\begin{align}
    \textbf{J} = 
    \frac{\partial \textbf{M}}{\partial \varphi}
    \times\frac{\partial \textbf{M}}{\partial \theta}
    &= 
    (1 + f)[
        f_\varphi \textbf{e}_r \times \textbf{e}_{\theta}
        + f_\theta  \textbf{e}_{\varphi} \times \textbf{e}_r
        + (1 + f) \textbf{e}_{\varphi}\times \textbf{e}_{\theta}
    ]\\
\end{align}
\begin{align}
    \textbf{e}_r \times \textbf{e}_\theta
    =
    \begin{pmatrix}
        -\sin\varphi\\
        \cos\varphi\\
        0\\
    \end{pmatrix}
    &&
    \textbf{e}_\varphi \times \textbf{e}_r =
    \begin{pmatrix}
        \sin\theta\cos\theta \cos\varphi\\
        \sin\theta\cos\theta \sin\varphi\\
        -\sin^2\theta
    \end{pmatrix}
    &&
    \textbf{e}_\varphi \times \textbf{e}_\theta =
    \begin{pmatrix}
        -\sin^2\theta\cos\varphi\\
        -\sin^2\theta\sin\varphi\\
        -\cos\theta\sin\theta
    \end{pmatrix}
\end{align}

The absolute value of this tensor is the sqareroot of the scalar product 
\begin{multline}
    \textbf{J}\cdot \textbf{J}
    = 
    (1 + f)^2[
        f_\varphi \textbf{e}_r \times \textbf{e}_{\theta}
        + f_\theta  \textbf{e}_{\varphi} \times \textbf{e}_r
        + (1 + f) \textbf{e}_{\varphi}\times \textbf{e}_{\theta}
    ]
    \cdot [
        f_\varphi \textbf{e}_r \times \textbf{e}_{\theta}
        + f_\theta  \textbf{e}_{\varphi} \times \textbf{e}_r
        + (1 + f) \textbf{e}_{\varphi}\times \textbf{e}_{\theta}
    ]\\
    =
    (1 + f)^2[
        f_\varphi^2 (\textbf{e}_r \times \textbf{e}_{\theta})\cdot (\textbf{e}_r \times \textbf{e}_{\theta})
        + f_\varphi f_\theta (\textbf{e}_r \times \textbf{e}_{\theta})\cdot (\textbf{e}_{\varphi} \times \textbf{e}_r)
        + f_\varphi (1 + f) (\textbf{e}_r \times \textbf{e}_{\theta})\cdot (\textbf{e}_{\varphi}\times \textbf{e}_{\theta})
        \\
        + f_\theta f_\varphi  (\textbf{e}_{\varphi} \times \textbf{e}_r)\cdot (\textbf{e}_r \times \textbf{e}_{\theta})
        + f_\theta^2   (\textbf{e}_{\varphi} \times \textbf{e}_r)\cdot (\textbf{e}_{\varphi} \times \textbf{e}_r)
        + f_\theta (1+f)   (\textbf{e}_{\varphi} \times \textbf{e}_r)\cdot  (\textbf{e}_{\varphi}\times \textbf{e}_{\theta}) \\
        + (1 + f)f_\varphi (\textbf{e}_{\varphi}\times \textbf{e}_{\theta})\cdot (\textbf{e}_r \times \textbf{e}_{\theta})
        + (1 + f)f_\theta (\textbf{e}_{\varphi}\times \textbf{e}_{\theta})\cdot (\textbf{e}_{\varphi} \times \textbf{e}_r)
        + (1 + f)^2 (\textbf{e}_{\varphi}\times \textbf{e}_{\theta})\cdot(\textbf{e}_{\varphi}\times \textbf{e}_{\theta})
    ]
\end{multline}
Because, for any vector $\textbf{a}, \textbf{b}, \textbf{c}, \textbf{d}$ we have, 
\begin{equation}
    (\textbf{a}\times \textbf{b})
    \cdot (\textbf{c}\times \textbf{d})
    =
    (\textbf{a}\cdot \textbf{c})
    (\textbf{b}\cdot \textbf{d})
    -
    (\textbf{a}\cdot \textbf{d})
    (\textbf{b}\cdot \textbf{c})
\end{equation}
and that,
\begin{align}
    \textbf{e}_r\cdot \textbf{e}_r=1 
    &&\textbf{e}_\theta\cdot \textbf{e}_\theta=1 
    &&\textbf{e}_\varphi\cdot \textbf{e}_\varphi=\sin^2\theta
\end{align}
the last equaiton becomes, 
\begin{equation}
    \textbf{J}\cdot \textbf{J}
    =
    (1 + f)^2[
        f_\varphi^2 
        + f_\theta^2\sin^2\theta  
        + (1 + f)^2\sin^2\theta  
    ]
\end{equation}
And so the surface eara element becomes, 
\begin{equation}
    dS 
    =
    |\textbf{J}| d\theta d\varphi 
    =
    (1 + f)[
        f_\varphi^2 
        + \sin^2\theta f_\theta^2  
        + \sin^2\theta(1 + f)^2
    ]^{1/2}
    d\theta d \varphi
\end{equation}

\tb{VERYY WROUNG OR MAYBE I JUST FORGET THE sin t}
Now let consider  that $f(\theta , \varphi)\to Ca f(\theta , \varphi)$.
The above resuts reads 
\begin{equation}
    dS 
    =
    (1+2Ca f) 
    d\theta d\varphi
    + O(Ca^2d\theta\varphi)
\end{equation}
So that, 
\begin{equation}
    \int dS 
    =
    \int_0^{2\pi}
    \int_0^{\pi}
    (1+2Ca f) 
    d\theta d\varphi
    + O(Ca^2)
\end{equation}


According to the parametrsiaition the distance function of the surface reads,
\begin{equation}
    \FF = r - (1+Ca f) = 0
\end{equation}
hence, 
\begin{equation}
    \grad \FF = \textbf{e}_r - Ca \grad f
\end{equation}
while
\begin{equation}
    \frac{1}{|\grad \FF|} = (1 - 2 Ca  (\textbf{e}_r \cdot \grad) f)^{-1/2}
    \approx (1+ Ca (\textbf{e}_r \cdot \grad) f) +O(Ca^2)
\end{equation}
Thus,
\begin{equation}
    \textbf{n} 
    =
    \frac{\grad F}{|\grad F|}
    =
    \textbf{e}_r - Ca (\bm\delta - \textbf{e}_r \textbf{e}_r) \cdot \grad f
    =
    \textbf{e}_r - Ca \gradI f
\end{equation}

According to that results we may write, 
\begin{align*}
    \textbf{nn} 
    &=
    (\textbf{e}_r - Ca \gradI f)
    (\textbf{e}_r - Ca \gradI f)\\
    &=
    \textbf{e}_r \textbf{e}_r
    - Ca (\textbf{e}_r \gradI f + \gradI f \textbf{e}_r )
\end{align*}
Then it is clear that, 

\begin{align}
    (\frac{1}{3}\bm\delta - \textbf{nn}) dS
    &=
    [
    \frac{1}{3}\bm\delta
    - \textbf{e}_r \textbf{e}_r
    + Ca (\textbf{e}_r \gradI f + \gradI f \textbf{e}_r )
    ]
    (1 + 2 Ca f) d\theta d\varphi
    \\
    &=
    \left[
    \frac{1}{3}\bm\delta   
    - \textbf{e}_r \textbf{e}_r 
    + Ca [
        2f(\frac{1}{3}\bm\delta 
        - \textbf{e}_r \textbf{e}_r )
        + \textbf{e}_r \gradI f 
        + \gradI f \textbf{e}_r 
        ]
    \right]d\theta d\varphi
    \\
\end{align}


Assuming 
\begin{equation}
    f = H_{kl} e_ke_l
\end{equation}
\begin{equation}
    \int (\delta_{ij} - n_in_j) dS 
    =
    \frac{4\pi}{3}\frac{8}{5}
    H_{ij}
    - \frac{32\pi}{45}\delta_{ij} H_{kk}
\end{equation}
The volume of the particle is given by, 
\begin{equation}
    \intO{}=
    \int_{0<r<1}
    \int_{\theta = 0}^\pi
    \int_{\varphi = 0}^{2\pi}
    (1+3 Ca \textbf{H}:\textbf{e}_r\textbf{e}_r) r^2 \sin\theta dr d\theta d\varphi
    =
    \frac{4\pi}{3}\left(
        1
        +
        Ca \textbf{H}:\bm\delta 
    \right)
\end{equation}
From the conservation of volume we deduce that
\begin{equation}
    \textbf{H}:\bm\delta 
    =
    0 
\end{equation}


\begin{align}
    \textbf{M}_p
    &=
    \intO{\textbf{rr}}
    =
    \int{r^4 (1+ Ca f)^5 \sin\theta \textbf{e}_r \textbf{e}_r }
    =
    \frac{4\pi}{15}\bm\delta+
    \textbf{H}: 
    \int{r^4 (5 Ca \textbf{e}_r \textbf{e}_r) \textbf{e}_r \textbf{e}_r }\sin\theta dr d\theta d\varphi\\
    &=
    \frac{4\pi}{15}[\bm\delta+
    \textbf{H}_{kl}: (
        \bm\delta_{ij}
        \bm\delta_{kl}
        +\bm\delta_{ik}
        \bm\delta_{jl}
        +\bm\delta_{il}
        \bm\delta_{jk}
    )
    ]\\
    &=
    \frac{4\pi}{15}[\bm\delta+
    2 Ca \textbf{H}
    ]
\end{align}


\begin{align}
    \textbf{n} 
    &= \textbf{e}_r - Ca (\bm\delta - \textbf{e}_r\textbf{e}_r)\cdot \grad f
    = \textbf{e}_r - Ca  \gradI f\\
    \textbf{r}^{(n)} 
    &= \textbf{e}_r^{(n)} r^n (1+ Ca f)^n
    = \textbf{e}_r^{(n)} r^n (1+ n Ca f) + O(r^n Ca)\\
    \bm\delta/3 - \textbf{nn} 
    &=
    \bm\delta/3
    - (\textbf{e}_r - Ca \gradI f)
    (\textbf{e}_r - Ca \gradI f)
    =
    \bm\delta/3
    - \textbf{e}_r\textbf{e}_r
    + Ca [
        \textbf{e}_r \gradI f
        + ^\dagger \textbf{e}_r \gradI f
    ]\\
    \textbf{r}^{(n)}(\bm\delta/3 - \textbf{nn}) 
    &=
    \textbf{e}_r^{(n)}(\bm\delta/3
    - \textbf{e}_r\textbf{e}_r)
    + Ca  \textbf{e}_r^{(n)} n f (\bm\delta/3
    - \textbf{e}_r\textbf{e}_r)
    + Ca \textbf{e}_r^{(n)} [
        \textbf{e}_r \gradI f
        + ^\dagger \textbf{e}_r \gradI f
    ]
\end{align}

So that 
\begin{align}
    \textbf{r}^{(n)}(\bm\delta/3 - \textbf{nn}) dS
    &= 
    \textbf{r}^{(n)}(\bm\delta/3 - \textbf{nn}) dS^0 
    + 2 Ca f \textbf{e}^{(n)}_r(\bm\delta/3 - \textbf{e}_r \textbf{e}_r)  dS^0 \\
    &=
    \textbf{e}_r^{(n)}(\bm\delta/3
    - \textbf{e}_r\textbf{e}_r)
    + Ca  \textbf{e}_r^{(n)} (n+2) f (\bm\delta/3 - \textbf{e}_r\textbf{e}_r)
    + Ca \textbf{e}_r^{(n)} [
        \textbf{e}_r \gradI f
        + ^\dagger \textbf{e}_r \gradI f
    ] dS^0 \\
    &=
    \textbf{e}_r^{(n)}(\bm\delta/3
    - \textbf{e}_r\textbf{e}_r)
    + Ca \textbf{e}_r^{(n)}
    [(n+2) f (\bm\delta/3 - \textbf{e}_r\textbf{e}_r)
    + \textbf{e}_r \gradI f
    + ^\dagger \textbf{e}_r \gradI f
    ] dS^0 \\
\end{align}


So that,

\begin{equation}
    \intS{\gamma (\bm\delta - \textbf{nn})}
    =
    \gamma_0 \intS{ (\bm\delta - \textbf{nn})}
    + \grad\gamma \cdot \intS{\textbf{r} (\bm\delta - \textbf{nn})}
    + \grad\grad\gamma : \intS{\textbf{rr} (\bm\delta - \textbf{nn})}
\end{equation}
where, 
\begin{align*}
    \intS{ (\bm\delta/3 - \textbf{nn})_{ij}} &= \pi\frac{32}{15} (H_{ij}- H_{qq} \delta_{ij}/3)\\
    \intS{ (\bm\delta/3 - \textbf{nn})_{ij}r_k} &= \pi\frac{8}{7} (H_{ijk} - H_{iqq} \delta_{jk}/5- H_{qjq} \delta_{ik}/5- H_{qqk} \delta_{ij}/5)\\
    \intS{ (\bm\delta/3 - \textbf{nn})_{ij}r_kr_l} &= 
    \pi\frac{64}{105} H_{ijkl}^{dev} 
    + \pi\frac{16}{35} H_{ij}\delta_{kl}
    + \pi\frac{32}{315} H_{kl}\delta_{ij}
    + \pi\frac{16}{315} H_{qq}(\delta_{ik}\delta_{jl}+\delta_{il}\delta_{jk})\\
    &- \pi\frac{8}{15} (H_{ik}\delta_{jl}+ H_{jl}\delta_{ik}+H_{il}\delta_{jk}+H_{jk}\delta_{il})
    - \pi\frac{176}{945} H_{qq}\delta_{ij}\delta_{kl}\\
    &=\pi\frac{64}{105} H_{ijkl}^{dev} 
    + \pi\frac{16}{35} H_{ij}\delta_{kl}
    + \pi\frac{32}{315} H_{kl}\delta_{ij}
    + \pi\frac{16}{315} H_{qq}(\delta_{ik}\delta_{jl}+\delta_{il}\delta_{jk})\\
    &- \pi\frac{8}{15} (H_{ik}\delta_{jl}+ H_{jl}\delta_{ik}+H_{il}\delta_{jk}+H_{jk}\delta_{il})
    - \pi\frac{176}{945} H_{qq}\delta_{ij}\delta_{kl}\\
    &=\pi\frac{64}{105} H_{ijkl}^{dev} 
    + \pi\frac{16}{35} \{
        H_{ij}\delta_{kl}
        + \frac{2}{9} H_{kl}\delta_{ij}
        + \frac{1}{6} H_{qq}(\delta_{ik}\delta_{jl}+\delta_{il}\delta_{jk})\\
        &- \frac{1}{6} (H_{ik}\delta_{jl}+ H_{jl}\delta_{ik}+H_{il}\delta_{jk}+H_{jk}\delta_{il})
        - \pi\frac{11}{27} H_{qq}\delta_{ij}\delta_{kl}\}
\end{align*}
Or 
\begin{multline*}
    (\grad\grad \gamma )_{kl}
    \intS{ (\bm\delta/3 - \textbf{nn})_{ij}r_kr_l}
    =
    \pi\frac{64}{105} H_{ijkl}^{dev} (\grad\grad \gamma)_{kl}\\
    -\frac{\pi 16}{105} [
        (\grad\grad \gamma)_{ik} (\textbf{H}_p^{(2)})_{jk}
        + (\grad\grad \gamma)_{jk} (\textbf{H}_p^{(2)})_{ik}
        -\frac{1}{3}\grad^2 \gamma (\textbf{H}_p^{(2)})_{ij}
        -\frac{2}{3}(\grad\grad \gamma)_{kl} (\textbf{H}_p^{(2)})_{kl}\delta_{ij}
        ]
\end{multline*}

Also, 
\begin{multline}
    b_l \intS{ r_kr_l(\bm\delta/3 - \textbf{nn})_{ij}}
    =
    \pi\frac{64}{105}b_l H_{ijkl}
    - \frac{176}{945}b_k \delta_{ij} H_{kk}^2
    - \frac{8}{105}b_l (\delta_{jk} H_{il}+\delta_{ik} H_{jl}-\frac{4}{3}\delta_{ik} H_{jl})\\
    - \frac{8}{105}(b_i H_{jk}+ b_j\delta_{ik})
    +\frac{16}{315}(b_i H_{ll}\delta_{jk}+b_j H_{ll}\delta_{ik} ) 
    +\frac{16}{35}(b_k H_{ij}) 
\end{multline}

\subsubsection{Singularity solution and faxen laws}
Any flow produced by the motion of a body may be expressed in terms of free-space green function and its derivatives,
\begin{align}
    \mathcal{G}_{ij}
    &=
    r^{-3}(\delta_{ij} r^2 + x_ix_j)
    =
    (\delta_{ij} - x_i \partial_j )r^{-1}\\
    \mathcal{G}_{ij,k}
    &=
    \partial_k (\delta_{ij} - x_i \partial_j )r^{-1}
    =(\delta_{ij}\partial_k  
    - \delta_{ik} \partial_j  
    - x_i \partial_{jk} )r^{-1}\\
    \mathcal{G}_{ij,kl}
    &=(
        \delta_{ij} \partial_{kl} 
    - \delta_{ik} \partial_{jl}  
    - \delta_{il} \partial_{jk}
    - x_i \partial_{jkl}
    )r^{-1}\\
    \mathcal{G}_{ij,klm}
    &=(
        \delta_{ij} \partial_{klm} 
    - \delta_{ik} \partial_{jlm}  
    - \delta_{il} \partial_{jkm}
    - \delta_{im} \partial_{jkl}
    - x_i \partial_{jklm}
    )r^{-1}\\
    \mathcal{G}_{ij,klmn}
    &=(
    \delta_{ij} \partial_{klmn} 
    - \delta_{ik} \partial_{jlmn}  
    - \delta_{il} \partial_{jkmn}
    - \delta_{im} \partial_{jkln}
    - \delta_{in} \partial_{jklm}
    - x_i \partial_{jklmn}
    )r^{-1}\\
\end{align}
Note that the laplacien of teh green func is, 
\begin{align}
    \mathcal{G}_{ij,kl}
    &=(
    -  \partial_{ji}  
    - \partial_{ji}
    )r^{-1}
    = -2 \partial_{ji}r^{-1}\\
    \mathcal{G}_{ij,kll}
    &=(
        \delta_{ij} \partial_{kll} 
    - \delta_{ik} \partial_{jll}  
    - \delta_{il} \partial_{ijk}
    - \delta_{im} \partial_{ijk}
    - x_i \partial_{jkll}
    )r^{-1}\\
    \mathcal{G}_{ij,klmm}
    &=(
    - \delta_{im} \partial_{ijkl}
    - \delta_{in} \partial_{ijkl}
    )r^{-1}
    = -2 \partial_{ijkl}r^{-1}
\end{align}

Flow field of a droplet in quadratic flow may be expressed in term of spherical harmonics as, 
\begin{align}
    \mathcal{P}_{jkl}
    &=
    (C_0 \delta_{kl}\partial_j
    + C_1 \delta_{jl}\partial_k
    + C_2 \delta_{jk}\partial_l
    + C_3 \partial_{jkl})r^{-1}\\
    r \mathcal{U}_{ijkl}
    &=
    x_i 
    (C_0 \delta_{kl}\partial_j
    + C_1 \delta_{jl}\partial_k
    + C_2 \delta_{jk}\partial_l
    + C_3 \partial_{jkl})
    + C_4\delta_{ij}\delta_{kl}
    + C_5\delta_{ik}\delta_{jl}
    + C_6\delta_{il}\delta_{kj}\\
    &
    + C_7\delta_{ij}\partial_{kl}
    + C_8\delta_{ik}\partial_{jl}
    + C_9\delta_{il}\partial_{kj}
    + C_{10}\delta_{kl}\partial_{ij}
    + C_{11}\delta_{jl}\partial_{ik}
    + C_{12}\delta_{kj}\partial_{il}
    + C_{13}\partial_{ijkl}
\end{align}


If one need to find a flow in a $\grad\grad\grad \textbf{u} = \partial_{ijk}u_l$ field he must shear for the constant of the form of, 
\begin{align}
    \mathcal{P}_{jklm}
    &=
    (C_0 \delta_{kl}\partial_j
    + C_1 \delta_{jl}\partial_k
    + C_2 \delta_{jk}\partial_l
    + C_3 \partial_{jkl}
    )r^{-1}\\
    r \mathcal{U}_{ijkl}
    &=
    x_i 
    (C_0 \delta_{kl}\partial_j
    + C_1 \delta_{jl}\partial_k
    + C_2 \delta_{jk}\partial_l
    + C_3 \partial_{jkl})
    + C_4\delta_{ij}\delta_{kl}
    + C_5\delta_{ik}\delta_{jl}
    + C_6\delta_{il}\delta_{kj}\\
    &
    + C_7\delta_{ij}\partial_{kl}
    + C_8\delta_{ik}\partial_{jl}
    + C_9\delta_{il}\partial_{kj}
    + C_{10}\delta_{kl}\partial_{ij}
    + C_{11}\delta_{jl}\partial_{ik}
    + C_{12}\delta_{kj}\partial_{il}
    + C_{13}\partial_{ijkl}
\end{align}

\section{deformable droplet singularity solution}
\subsection{Physical problem}
The points $\textbf{r}_\Gamma(t)$ lying on the droplet surface or volume may be computed as,
\begin{equation}
    \textbf{r}_\Gamma = r [1 + f(\theta,\varphi)]\textbf{e}_r
\end{equation}


In dimensionless form ($\textbf{u}_{out}$, $\textbf{u}_{in}$) are governed by the Stokes equations, namely, 
\begin{align}
    \div \textbf{u}_{in} = 0 
    && \div \textbf{u}_{out} = 0 \\
    -\grad p_{in} + \mu_{in}\grad^2 \textbf{u}_{in}  = 0 
    &&-\grad p_{out} + \mu_{out}\grad^2 \textbf{u}_{out}  = 0 
\end{align}
As these field are disturbance field they follow,
\begin{equation}
    \lim_{r\to \infty}(\textbf{u}_{out},p_{out}) = 0 
\end{equation}
far from the droplet centered at the origin ($\textbf{x} = 0$). 
At the surface of the droplet ($r = 1 + f(\theta,\varphi)$) the continuity of velocity and the far field condition impose, 
\begin{align}
    \textbf{u}_{in} = \textbf{u}_{out}
    && 
    (\textbf{u}_{in}  + \textbf{u} - \textbf{u}_\Gamma)\cdot \textbf{n}
    = (\textbf{u}_{in}  + \textbf{u}_r)\cdot \textbf{n}
    = 0
    &&
    \text{ at }
    r = 1 + f(\theta,\varphi)
\end{align}
where it must be noted that both $\textbf{u}_{in}$ and $\textbf{u}$ are evaluated at the points on the surface of the droplet, hence $\textbf{u}$ may not always be a constant vector. 
$\textbf{u}_\Gamma$ is the interface of teh droplet, which may be replaced by the droplet center of mass for undeformable droplets, otherwise it is related to the moment of momentum and mass. 


Let us introduce the dimensionless stress tensor $\bm\sigma_{in/out} = -p_{in/out} \bm\delta + \mu_{in/out}[\grad \textbf{u}_{in/out} + ^\dagger \grad \textbf{u}_{in/out}]$ and the dimensionless shear stress tensor, $2\textbf{e}_{in/out} = \grad \textbf{u}_{in/out} + ^\dagger \grad \textbf{u}_{in/out}$. 
The undisturbed shear stress reads  $2\textbf{e} = \grad \textbf{u} + ^\dagger \grad \textbf{u}$
With these notations the tangential shear stress jump reads at $r = 1$ reads as, 
\begin{align}
    \mathbf{n}\cdot (\textbf{e}_{out} - \lambda \textbf{e}_{in}+\textbf{e} -\lambda\textbf{e})\cdot (\bm\delta - \textbf{nn})
    % +
    % (1-\lambda)\mathbf{n}\cdot \textbf{e} \cdot (\bm\delta - \textbf{nn})
    = (\bm\delta - \textbf{nn})\cdot \textbf{b}
    &&
    \text{ at }
    r = 1 + f(\theta,\varphi)
\end{align}
with, 
\begin{equation}
    \textbf{b}
    =
    \frac{\grad \gamma}{2\mu_f}
    % \approx (a\grad) Ca^{-1}. 
    % =
    % \frac{a \grad \gamma}{2 \mu_f |\textbf{u}_r|}
    % \approx (a\grad) Ca^{-1}. 
\end{equation}
The velocity \textbf{u} is imposed, the velocity $\textbf{u}_\Gamma$ is just the velocity of the interface which may be obtained by differetiating the point lying on the interface, namely, 
\begin{equation}
    \textbf{u}_\Gamma
    =
    \textbf{u}_\alpha
    +
    \pddt \textbf{r}_\Gamma
    =
    \textbf{u}_\alpha
    + 
    \textbf{e}_r\pddt f(\theta,\varphi,t)
\end{equation}

\subsection*{Dimensionless problem}
The relative velocity at the interfaces as well as the mean stresss may be expanded as a taylor series such that, 
\begin{align*}
    \textbf{u}_r(\textbf{n}) 
    &=  \textbf{u}_r|_{\textbf{x}=0}
    +  \textbf{r} \cdot  \grad\textbf{u}_r|_{\textbf{x}=0}
    + \ldots
    +  \frac{1}{n!}\textbf{r}^{(n)} \odot \grad^{(n)}\textbf{u}_r|_{\textbf{x}=0}
    \\
     \textbf{e}(\textbf{n}) 
    &=   \textbf{e}|_{\textbf{x}=0}
    + \textbf{r} \cdot  \grad \textbf{e}|_{\textbf{x}=0}
    % + \frac{1}{2}\textbf{rr} \cdot  \grad\grad \textbf{e}|_{\textbf{x}=0}
    + \ldots
    + \frac{1}{n!}\textbf{r}^{(n)} \odot  \grad^{(n)} \textbf{e}|_{\textbf{x}=0}
\end{align*}
Note that if the droplet is rotating or deforming it means that $\textbf{u}_\Gamma = \textbf{u}_\alpha + \textbf{r}\cdot \textbf{P}+ \ldots$ hence these boundaries includes the fact that droplets deform. 

Because of these relations one may be able to re-formulate the above problem in terms of generalized tensor such that, 
\begin{align}
    \textbf{u}_{out/in} &= \textbf{U}_{out/in}^{(n+2)}\odot \grad^{(n)}\textbf{u}_r\\
    % \textbf{u}_{in} &= \textbf{U}_{in}^{(n+2)}\odot \grad^{(n)}\textbf{u}_r\\
    p_{out/in} &= \mu_{out/in} \textbf{P}_{out/in}^{(n+1)}\odot \grad^{(n)}\textbf{u}_r\\
    % p_{in} &=  \mu_{in} \textbf{P}_{in}^{(n+1)}\odot \grad^{(n)}\textbf{u}_r\\
    \bm\sigma_{out/in}
    &=
    \mu_{out/in} 
    \bm\Sigma^{(n+3)}_{out/in}
    \odot \grad^{(n)}\textbf{u}_r
    =
    \mu_{out/in} 
    [
    \bm\delta\textbf{P}_{out/in}^{(n+1)}
    + 
    \grad \textbf{U}_{out/in}^{(n+2)}
    + 
    ^\dagger \grad \textbf{U}_{out/in}^{(n+2)}]\odot \grad^{(n)} \textbf{u}_r\\
    \textbf{e}_{out/in}
    &=
    2\textbf{E}^{(n+3)}_{out/in}
    \odot \grad^{(n)}\textbf{u}_r
    =
    [
    \grad \textbf{U}_{in}^{(n+2)}
    + 
    ^\dagger \grad \textbf{U}_{in}^{(n+2)}]\odot \grad^{(n)} \textbf{u}_r\\
    \textbf{b} &=
    \textbf{B}^{(n+1)}\odot \grad^{(n)}\textbf{u}
\end{align}
where $\odot$ represent the maximum contraction operators. 
Note that $\textbf{B} = \bm\delta$ means that $\textbf{b}$ is aligned with $\textbf{u}_r$. 
Hence the generalized tensor $\textbf{B}$ indicacted that one might choose a vector \textbf{b} such that it is in an other direction than $\textbf{u}_r$ which is physically unlikely.   
We follow the convention of \citet{brenner1963stokes} for the transpose operator$^\dagger$. 
The mean averaged stress can then be written, 
\begin{multline}
    \frac{2}{(n-1)!}\textbf{r}^\text{($n$-1)}\odot \grad^\text{($n$-1)}\textbf{e}
    =
    \frac{1}{(n-1)!}\textbf{r}^\text{($n$-1)}\odot \grad^\text{($n$-1)}[\grad \textbf{u}+ (\grad \textbf{u})^\dagger]\\
    =
    \frac{1}{(n-1)!}\textbf{r}^\text{($n$-1)}\odot [\grad^{(n)}\textbf{u} + \grad^{(n)}\textbf{u}^\dagger]
    =
    \frac{1}{(n-1)!}\textbf{r}^\text{($n$-1)}[\bm\delta\bm\delta + (\bm\delta\bm\delta)^\dagger] \odot \grad^{(n)}\textbf{u}
\end{multline}
In index notation the mean stress could be noted, 
\begin{multline}
    e_{ij}
    =
    \frac{1}{(n-1)!}(\textbf{r}^\text{($n$-1)})_{k_1k_2\ldots k_{n-1}} 
    [(\grad^{(n)}\textbf{u})_{k_1k_2\ldots k_{n-1}ij}  + (\grad^{(n)}\textbf{u})_{k_1k_2\ldots k_{n-1}ji}]\\
    =
    \frac{1}{(n-1)!}(\textbf{r}^\text{($n$-1)})_{k_1k_2\ldots k_{n-1}} 
    [\delta_{ik_n} \delta_{jk_{n+1}}+ \delta_{jk_n} \delta_{ik_{n+1}}]
    (\grad^{(n)}\textbf{u})_{k_1k_2\ldots k_{n-1}k_nk_{n+1}} 
\end{multline}
The interface is assumed already deformed, following the form, 
\begin{equation}
    f(\varphi,\theta,t) =  \sum_{n=2}^{\infty} \textbf{S}^{(n)}\odot \textbf{H}^{(n)}
\end{equation}
\begin{equation}
   \pddt  f(\varphi,\theta,t) =  \sum_{n=2}^{\infty} \textbf{S}^{(n)}\odot \dot{\textbf{H}}^{(n)}
\end{equation}

So that in general the equation to solve becomes, 
\begin{align*}
    \div \textbf{U}_{in/out}^{(n+2)} &= 0 \\
    \grad^2\textbf{U}_{in/out}^{(n+2)} &= \grad \textbf{P}_{in/out}^{(n+1)} \\
\end{align*}
with the boundary condition, 
\begin{align}
    \textbf{U}_{in}^{(n+2)} - \textbf{U}_{out}^{(n+2)}
    &=
    0
    &&\forall r = 1 + f(\theta,\varphi)
    \\
    \textbf{n}\cdot [\textbf{U}_{in}^{(n+2)} 
    +\frac{1}{n!}\bm\delta \textbf{r}^{(n)}] &= 0 
    &&\forall r = 1 + f(\theta,\varphi)\\
    \textbf{n}\cdot \{
        \textbf{E}_{out}^{(n+3)}
        -\lambda\textbf{E}_{in}^{(n+3)}
        +\frac{(1-\lambda)}{(n-1)!}
        \textbf{r}^{(n-1)}[\bm\delta\bm\delta + (\bm\delta\bm\delta)^\dagger]
        - \textbf{n}\textbf{B}^{(n+2)}
    \}\cdot (\bm\delta - \textbf{nn})
    &=
    0
    &&\forall r = 1 + f(\theta,\varphi)
\end{align}
note that we could factor by teh whole $\textbf{u}_r$ hence the velocity of deformation doesn't comes into account at this stage. 

\subsubsection*{Droplet deformation }

Because we cannot apply these boundary condition directly on the interface $r = 1 + f$ we assume $f \to \epsilon f$ where $\epsilon$ is small and $f$ of $O(1)$ and express any of the above funciton as a taylor expansion in terms of $\epsilon$ such that, any tensor \textbf{F} can be expressed as, 
\begin{equation}
    \textbf{F}
    = \ps{0}{\textbf{F}}
    + \epsilon \ps{1}{\textbf{F}}
    + O(\epsilon^2)
\end{equation}
Then, any function to be evaluated on the droplet interface might be written as, 
\begin{equation}
    \textbf{F}(r=1+\epsilon f)
    =
    \textbf{F}|_{r=1}
    + \epsilon f \frac{\partial \textbf{F}}{\partial r}|_{r=1}
    =
    \ps{0}{\textbf{F}}|_{r=1}
    + \epsilon \left[\ps{1}{\textbf{F}}
    + f \frac{\partial\ps{0}{\textbf{F}}}{\partial r}\right]_{r=1}
    + O(\epsilon)
\end{equation}
the position vector in perturbed coordinates may be approximated by, 
\begin{equation}
    \textbf{r}^{(n)} 
    = \textbf{e}_r^{(n)} r^n (1+\epsilon f)^n
    = \textbf{e}_r^{(n)} r^n (1+ n\epsilon f) + O(r^n \epsilon)
\end{equation}
where $\textbf{e}_r = \ps{0}{\textbf{n}}$ is the unit radial vector going outward the interface.
Not ethat the approximation is true only for small $r$ however. 

Likewise,
\begin{align}
    \textbf{n} 
    &= \textbf{e}_r - \epsilon (\bm\delta - \textbf{e}_r\textbf{e}_r)\cdot \grad f
    = \textbf{e}_r - \epsilon  \gradI f\\
    \bm\delta - \textbf{nn} 
    &=
    \bm\delta
    - (\textbf{e}_r - \epsilon \gradI f)
    (\textbf{e}_r - \epsilon \gradI f)
    =
    \bm\delta
    - \textbf{e}_r\textbf{e}_r
    + \epsilon [
        \textbf{e}_r \gradI f
        + ^\dagger \textbf{e}_r \gradI f
    ]\\
    \textbf{n}(\bm\delta - \textbf{nn}) 
    &=
    \textbf{e}_r(\bm\delta
    - \textbf{e}_r\textbf{e}_r)
    - \epsilon \gradI f (\bm\delta
    - \textbf{e}_r\textbf{e}_r)
    + \epsilon \textbf{e}_r [
        \textbf{e}_r \gradI f
        + ^\dagger \textbf{e}_r \gradI f
    ]
\end{align}

Expanding any of tehse terms and isolating any power of $\epsilon$ yields, at $O(\epsilon^0)$, the following system of equations, 
\begin{align*}
    \div \ps{0}{\textbf{U}}_{in/out}^{(n+2)} &= 0 \\
    \grad^2\ps{0}{\textbf{U}}_{in/out}^{(n+2)} &= \grad \ps{0}{\textbf{P}}_{in/out}^{(n+1)} \\
\end{align*}
with the boundary condition, 
\begin{align}
    \ps{0}{\textbf{U}}_{in}^{(n+2)} - \ps{0}{\textbf{U}}_{out}^{(n+2)}
    &=
    0
    &&\forall r = 1 
    \\
    \textbf{e}_r\cdot [\ps{0}{\textbf{U}}_{in}^{(n+2)} 
    +\frac{r^n}{n!}\bm\delta \textbf{e}_r^{(n)}] &= 0 
    &&\forall r = 1 \\
    \textbf{e}_r\cdot \{
        \ps{0}{\textbf{E}}_{out}^{(n+3)}
        -\lambda\ps{0}{\textbf{E}}_{in}^{(n+3)}
        +\frac{(1-\lambda)r^{n-1}}{(n-1)!}
        \textbf{e}_r^{(n-1)}[\bm\delta\bm\delta + (\bm\delta\bm\delta)^\dagger]
        - \textbf{e}_r\ps{0}{\textbf{B}}^{(n+2)}
    \}\cdot (\bm\delta - \textbf{e}_r \textbf{e}_r)
    &=
    0
    &&\forall r = 1
\end{align}


At the next order in $O(Ca)$ we obtain the following system, 
\begin{align*}
    \div \ps{1}{\textbf{U}}_{in/out}^{(n+2)} &= 0 \\
    \grad^2\ps{1}{\textbf{U}}_{in/out}^{(n+2)} &= \grad \ps{1}{\textbf{P}}_{in/out}^{(n+1)} \\
\end{align*}
with the boundary condition, 
\begin{align}
    \ps{1}{\textbf{U}}_{in}^{(n+2)} - \ps{1}{\textbf{U}}_{out}^{(n+2)}
    + f\frac{\partial}{\partial r} [\ps{0}{\textbf{U}}_{in}^{(n+2)} - \ps{0}{\textbf{U}}_{out}^{(n+2)}]
    &=
    0
    &&\forall r = 1 
    \\
    \textbf{e}_r\cdot [
    \ps{1}{\textbf{U}}_{in}^{(n+2)} 
    + f\frac{\partial}{\partial r}\ps{0}{\textbf{U}}_{in}^{(n+2)} 
    +\frac{r^n}{(n-1)!}\bm\delta \textbf{e}_r^{(n)} f ] 
    - \gradI f \cdot [\ps{0}{\textbf{U}}_{in}^{(n+2)} 
    +\frac{r^n}{n!}\bm\delta \textbf{e}_r^{(n)}] &= 0 
    &&\forall r = 1 
\end{align}
the last BC is even more involving it reads, 
\begin{multline}
    \textbf{e}_r\cdot \{
        \ps{1}{\textbf{E}}_{out}^{(n+3)}
        -\lambda\ps{1}{\textbf{E}}_{in}^{(n+3)}
        + f \frac{\partial}{\partial r}(\ps{0}{\textbf{E}}_{out}^{(n+3)}
        -\lambda\ps{0}{\textbf{E}}_{in}^{(n+3)})\\
        +\frac{(1-\lambda)r^{n-1}}{(n-2)!}
        f \textbf{e}_r^{(n-1)}[\bm\delta\bm\delta + (\bm\delta\bm\delta)^\dagger]
        - \ps{1}{\textbf{B}}^{(n+2)}
        - f \frac{\partial}{\partial r}\textbf{B}^{(n+2)}
    \}\cdot (\bm\delta - \textbf{e}_r \textbf{e}_r)\\
    + [\textbf{e}_r (\textbf{e}_r \gradI f + ^\dagger \textbf{e}_r \gradI f) - \gradI f (\bm\delta - \textbf{e}_r \textbf{e}_r) ] \\: \{
        \ps{0}{\textbf{E}}_{out}^{(n+3)}
        -\lambda\ps{0}{\textbf{E}}_{in}^{(n+3)}
        +\frac{(1-\lambda)r^{n-1}}{(n-1)!}
        \textbf{e}_r^{(n-1)}[\bm\delta\bm\delta + (\bm\delta\bm\delta)^\dagger]
        - \ps{0}{\textbf{B}}^{(n+2)}
    \} 
    =
    0
\end{multline}
solving this problem yields the final sol. 

One can then factor out by any order of deformation and find 
\begin{equation}
    \frac{1}{a\pi\mu_f }\intS{
        \bm\sigma\cdot \textbf{n}
    }
    =2 \frac{3\lambda +2}{\lambda+1}\textbf{u}_r
    + 
    Ca\frac{\left(9 \lambda^{2} + 11 \lambda + 20\right)}{5(\lambda + 1)^2}\left[
        \textbf{H}_p
        + \textbf{H}_p^\dagger
        -\frac{2}{3}(\textbf{H}_p:\bm\delta)\bm\delta
    \right]\cdot \textbf{u}_r
    + O(Ca^2)
\end{equation}
since 
\begin{equation}    
        \textbf{H}_p
        =
        \frac{19 \lambda + 16}{8 \left(\lambda + 1\right)}
        \left(\frac{a\mu_f}{\gamma Ca}\right)
        \textbf{E}_f
        =
        \frac{19 \lambda + 16}{8 \left(\lambda + 1\right)}
        \left(\frac{a}{U}\right)
        \textbf{E}_f
        =
        \frac{19 \lambda + 16}{8 \left(\lambda + 1\right)}
        \textbf{E}_f^*
\end{equation}
hence \textbf{H} is proportional to the dimentionless shear not the shear it self (which make sense because it has no unit). 
\begin{equation}
    \frac{1}{a\pi\mu_f }\intS{
        \bm\sigma\cdot \textbf{n}
    }
    =2 \frac{3\lambda +2}{\lambda+1}\textbf{u}_r
    + 
    Ca
    \frac{\left(9 \lambda^{2} + 11 \lambda + 20\right)(19 \lambda + 16)}{40(\lambda + 1)^3}
    \left[
        2\textbf{E}_f^*
        -\frac{2}{3}(\textbf{E}_f^*:\bm\delta)\bm\delta
    \right]\cdot \textbf{u}_r
\end{equation}
One may introduce the scales  $a \pi \mu_f \textbf{u}_r \to a \pi \mu_f U \textbf{u}^*_r$ hence; 
\begin{equation}
    \frac{1}{a\pi\mu_f U }\intS{
        \bm\sigma\cdot \textbf{n}
    }
    =2 \frac{3\lambda +2}{\lambda+1}\textbf{u}_r
    + 
    Ca
    \frac{\left(9 \lambda^{2} + 11 \lambda + 20\right)(19 \lambda + 16)}{40(\lambda + 1)^3}
    \left[
        2\textbf{E}_f^*
        -\frac{2}{3}(\textbf{E}_f^*:\bm\delta)\bm\delta
    \right]\cdot \textbf{u}_r^*
\end{equation}
Hence the coupling terms in the dimensioneless drag force (left-hand side) are of order $Ca$ ! 





\subsubsection{Perturbed solution of the Hybrid-model}

Remark that at this order in accuracy $\mathcal{O}(Ca^0)$ the linear momentum equations are not coupled with the dispersed phase moments ($\textbf{S}_p,\bm\mu_p$, and $\textbf{M}_p$).  
Consequently, in this first approach \ref{eq:dt_hybrid_Sp,eq:dt_hybrid_Mp,eq:dt_hybrid_mup} are not needed.
Hence, one can simply inject \ref{eq:drag_forces} to \ref{eq:mean_contributions}, into  \ref{eq:dt_hybrid_up} and \ref{eq:dt_uf} to obtain a closed form of the hybrid model, namely,  
\begin{align}
    \label{eq:first}
    \phi_f + \phi &= 1\\
    \textbf{u}_f &= \textbf{u}  + \phi \textbf{u}_r/\phi_f \approx \textbf{u} + \phi \textbf{u}_r\\
    \div \textbf{u} &= 0 \\
    (\pddt + \textbf{u} \cdot \grad)\phi
    &= \div (\phi\textbf{u}_r)\\
    \rho_d \phi (\pddt + \textbf{u}_p \cdot \grad)\textbf{u}_p
    &=
    \phi(\div \bm\Sigma
    + \rho_d  \textbf{g})
    + \div \bm\sigma_p^\text{eff}
    + \textbf{F}
    \\
    \phi_f \rho_f(\pddt + \textbf{u}_f  \cdot \grad) \textbf{u}_f
    % - \div \avg{\chi_f\rho_f \textbf{u}'\textbf{u}'}
    &= \phi_f 
    \left(\div \bm{\Sigma}
    + \rho_f \textbf{g}\right)
    + \div \bm\sigma^\text{eff}
    -\textbf{F}\\
    \label{eq:last}
\end{align}

\begin{align}
    \textbf{F}=&
    \phi
    \frac{\mu_f}{a^2}
    \frac{3(2+3\lambda)}{2(1+\lambda)}\textbf{u}_r
    + \phi\mu_f  \frac{3\lambda}{4(\lambda +1)} \grad^2 \textbf{u}\\
    \bm\sigma_p^\text{eff}
    =&
    -\rho_d C^1_{up}(\phi,\lambda) \textbf{u}_r \textbf{u}_r
    -\rho_d C^2_{up}(\phi,\lambda) (\textbf{u}_r \cdot \textbf{u}_r)\bm\delta\\
    % \bm\sigma_f^\text{eff}
    % =&
    % \bm\sigma_f^\text{eff-1}
    % + \mu_f a^2 \bm\sigma_f^\text{eff-2} \\
    \bm\sigma_f^\text{eff}
    =&
     \mu_f \phi \frac{5\lambda +2}{\lambda+1} \textbf{E}
    + \mu_f \frac{3\lambda}{4(\lambda+1)} [
    \grad(\phi \textbf{u}_r)
    + \grad(\phi \textbf{u}_r)^\dagger]
    - \mu_f \frac{3\lambda - 2}{4(\lambda+1)} \div(\phi \textbf{u}_r)  \bm\delta\nonumber\\
    &-\rho_f C^1_{uu}(\phi,\lambda)  \textbf{u}_r \textbf{u}_r
    -\rho_f C^2_{uu} (\phi,\lambda) (\textbf{u}_r \cdot \textbf{u}_r)\bm\delta
    \label{eq:sigma_feffff}
\end{align}

\subsubsection{We neglect the $O(Re)$ terms}
Or in an other form the total momentum ea,
\begin{align}
    \div \textbf{u}  &= 0\\
    (\pddt + \textbf{u}\cdot \grad) \phi &= \div(\textbf{w})\\
    0 &= 
    - \grad p_f 
    + \mu_f \grad^2\textbf{u}
    + (\rho_f +\phi \rho)\textbf{g}
    + \div (\bm\sigma^\text{eff}) \\
    % 0 &= 
    % - \grad^2 p_f  
    % + \rho \textbf{g}\cdot \grad \phi 
    % + \grad\grad : \bm\sigma^\text{eff}\\
    \bm\sigma^\text{eff}
    &=
     \mu_f \phi \frac{5\lambda +2}{\lambda+1} (\grad\textbf{u}+^\dagger \grad \textbf{u})
    + \mu_f \frac{3\lambda}{4(\lambda+1)} 
    \grad\textbf{w}
    + \mu_f \frac{ 2}{4(\lambda+1)} (\div\textbf{w} ) \bm\delta\\
    \textbf{w} &= - \phi \rho \frac{a^22(\lambda+1)}{\mu_f 3(3\lambda+2)}\textbf{g} - \frac{a^2\lambda}{2(3\lambda+2)} \phi \grad^2 \textbf{u} \\
\end{align} 
where $\rho = \rho_d - \rho_f$ and , 
\begin{align}
    - \phi \grad^2 p_f = O(\phi^2)\\
    \phi (-\grad p_f+\mu_f \grad^2 \textbf{u} )
    &=
    -\phi \textbf{g}\rho_f
    + O(\phi^2)\\
    \phi (-\grad\grad p_f+\mu_f \grad^2 \grad\textbf{u} )
    &=
    O(\phi^2)\\
    \phi \grad^4 \textbf{u}
    =
    O(\phi^2)
\end{align}  

Let assume a small perturbed state aroud ($\phi_0 = cst$ and $\textbf{u}_0 = 0$ ) such that, 
\begin{align*}
    \phi &= \phi_0 + \varphi\\
    \textbf{u} &= \textbf{u}_0 + \textbf{v}\\
    p &= p_0 + q\\
    \textbf{w} &= \textbf{w}_0 + \textbf{w}_1\\
    \bm\sigma^\text{eff} &= \bm\sigma^\text{eff}_0 + \bm\sigma^\text{eff}_1\\
\end{align*}
Hence the equilibrium varibale follows, 
\begin{align*}
    \textbf{u}_0\cdot \grad \phi_0  &= \div\textbf{w}_0\\
    0 &= 
    - \grad p_0 
    + (\rho_f +\phi_0 \rho)\textbf{g}
    + \div (\bm\sigma^\text{eff}_0) \\
    0 &= 
    - \grad^2 p_0  
    + \grad\grad : \bm\sigma^\text{eff}_0\\
    \bm\sigma^\text{eff}_0
    &=
    + \mu_f \frac{3\lambda}{4(\lambda+1)} 
    \grad\textbf{w}_0
    + \mu_f \frac{ 2}{4(\lambda+1)} (\div\textbf{w}_0)  \bm\delta\\
    \textbf{w}_0 &= - \phi_0 \rho \frac{a^22(\lambda+1)}{\mu_f 3(3\lambda+2)}\textbf{g}\\
\end{align*} 
Including the perturbed sol neglecting the non-linear termes and removing the stable eq gives
\begin{align*}
    \pddt \varphi &= \div \textbf{w}_1\\
    0 &= 
    - \grad  q
    + \mu_f \grad^2\textbf{v}
    + \varphi \rho \textbf{g}
    + \div \bm\sigma^\text{eff}_1 \\
    0 &= 
    - \grad^2 q
    + \rho \textbf{g}\cdot \grad \varphi 
    + \grad\grad : \bm\sigma^\text{eff}_1\\
    \bm\sigma^\text{eff}_1
    &=
     \mu_f \phi_0 \frac{5\lambda +2}{\lambda+1} (\grad\textbf{v}+^\dagger \grad \textbf{v})
    + \mu_f \frac{3\lambda}{4(\lambda+1)} 
    \grad  \textbf{w}_1
    + \mu_f \frac{ 2}{4(\lambda+1)} \div  \textbf{w}_1  \bm\delta\\
    \textbf{w}_1 &= -  \varphi \rho \frac{a^22(\lambda+1)}{\mu_f 3(3\lambda+2)}\textbf{g} - \frac{a^2\lambda}{2(3\lambda+2)} \phi_0 \grad^2 \textbf{v} 
\end{align*} 
of course we recall that $\div \textbf{v}= 0$. 
Hence the last two equation may be injected in the first three one to get three eq, for $\varphi, q, \textbf{v}$ namely, 
\begin{align*}
    \pddt \varphi &= - \rho \frac{a^22(\lambda+1)}{\mu_f 3(3\lambda+2)}\textbf{g}\cdot \grad \varphi \\
    0 &= 
    - \grad  q
    + \mu_f (1+\phi_0 \frac{5\lambda +2}{\lambda+1}) \grad^2\textbf{v}
    + \varphi \rho \textbf{g}
    -  \frac{a^2}{ 6(3\lambda+2)} \rho \textbf{g}\cdot (
    + 3\lambda \bm\delta \grad^2   
    +  2 \grad\grad  
    )\varphi\\
    0 &= 
    - \grad^2 q
    + \rho \textbf{g} \cdot(1 - \frac{a^2}{6}\grad^2) \grad \varphi \\
    \bm\sigma^\text{eff}_1
    &=
     \mu_f \phi_0 \frac{5\lambda +2}{\lambda+1} (\grad\textbf{v}+^\dagger \grad \textbf{v})\\
    &+ \mu_f \frac{3\lambda}{4(\lambda+1)}  (
    - \rho \frac{a^22(\lambda+1)}{\mu_f 3(3\lambda+2)}\textbf{g} \grad\varphi  
    - \frac{a^2\lambda}{2(3\lambda+2)} \phi_0 \grad^2 \grad\textbf{v} 
    )\\
    &+ \mu_f \frac{ 2}{4(\lambda+1)} (
        -   \rho \frac{a^22(\lambda+1)}{\mu_f 3(3\lambda+2)}\textbf{g} \cdot \grad\varphi  
    ) \bm\delta\\
    \textbf{w}_1 &= 
    -  \varphi \rho \frac{a^22(\lambda+1)}{\mu_f 3(3\lambda+2)}\textbf{g} 
    - \frac{a^2\lambda}{2(3\lambda+2)} \phi_0 \grad^2 \textbf{v} 
\end{align*} 

\paragraph*{Using fourier transform}
Let now consider the fourier transform, 
\begin{equation}
    \textbf{A}(\textbf{k},\omega)
    =
    \iiiint
    \textbf{A}(\textbf{x},t)
    e^{-i(\textbf{k}\cdot \textbf{x} + \omega t)}
    d\textbf{k}d\omega
\end{equation}
So the the above system of equaiton reads, 
\begin{align}
    \omega \varphi &= - \rho \frac{a^22(\lambda+1)}{\mu_f 3(3\lambda+2)}\textbf{g}\cdot \textbf{k} \varphi \\
    0 &= 
    i \textbf{k}  q
    - \mu_f (1+\phi_0 \frac{5\lambda +2}{\lambda+1}) k^2\textbf{v}
    + \varphi \rho \textbf{g}
    -  \frac{a^2}{ 6(3\lambda+2)} \rho \textbf{g}\cdot (
    - 3\lambda \bm\delta k^2   
    -  2 \textbf{kk}
    )\varphi\\
    0 &= 
    + k^2 q
    - \rho \textbf{g} \cdot(1 + \frac{a^2}{6}k^2) i\textbf{k} \varphi 
\end{align}
Or in matrix form, 
\begin{equation}
    \omega
    \begin{pmatrix}
        \varphi\\
        0\\
        0
    \end{pmatrix}
    =
    \begin{pmatrix}
        - \rho \frac{a^22(\lambda+1)}{\mu_f 3(3\lambda+2)}\textbf{g}\cdot \textbf{k}
        & 0 & 0 \\
        \rho \textbf{g}
        + \frac{a^2}{ 6(3\lambda+2)} \rho \textbf{g}\cdot (
        3\lambda \bm\delta k^2   
        +  2 \textbf{kk}
        )
        &
        - \mu_f (1+\phi_0 \frac{5\lambda +2}{\lambda+1}) k^2
        &
        i \textbf{k}  \\
        - \rho \textbf{g} \cdot(1 + \frac{a^2}{6}k^2) i\textbf{k}  
        &
        0
        &k^2 
    \end{pmatrix}
    \cdot 
    \begin{pmatrix}
        \varphi\\    
        \textbf{v}\\    
        q\\    
    \end{pmatrix}
\end{equation}
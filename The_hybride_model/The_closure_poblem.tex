\section{Introduction to the closure problem}
\label{ap:Closure_problem}

The closure terms in the above equation are the results of the ensemble average operator $\avg{\ldots}$. 
In all rigor, we cannot compute theoretically such an average since it necessitates knowing the distribution $P(\FF)$ and the exact expression of the local terms indicated by the notation $(\ldots)^0$. 
In the same spirit as in \citet{batchelor1972sedimentation,hinch1977averaged} and \citet{zhang1994averaged} we demonstrate here that it is therefore necessary to reformulate the closure term to remove the ensemble average procedure. 
We demonstrated in \ref{ap:Closure_problem} that any ensemble-averaged quantities can be reformulated as an integral of what we call \textit{conditionally-averaged quantities}. 
The expressions obtained are  consistent with \citet{batchelor1972sedimentation,hinch1977averaged} and \citet{zhang1994averaged} which also proposed conditional averages, but somewhat more general. 
In a second step, we demonstrate how to derive what we call the \textit{conditionally-averaged equations} that are needed to obtain the \textit{conditionally-averaged quantities}. 
For example, in \citet{kim1985modelling} they use the \textit{conditionally-averaged equations} to consider the effect of finite volume fraction on the drag force closure term, in fixed beds of spherical solid spheres. 
Again our derivation is directly inspired by the cited author, but the approach is generalized in many ways. 
For instance, in dilute and low Reynolds number assumption these equations correspond to the classic problem of the disturbance field generated by a translating droplet in an otherwise quiescent liquid flow. 
The real interest behind this demonstration is that it is kept general and offers many possibilities for model extension.
While more complicated considerations cannot be usually solved theoretically, these general formulations extend our current understanding of the closure problem. 


In \ref{sec:reformulation} and \ref{sec:the_disturbance_eq} we discuss the general approach to derive the closure problem, while in \ref{sec:application} we derive the closures terms of the momentum and energy equations. 
Since the derivation of the closure terms can be understood through physical arguments, readers who are less interested in the rigorous mathematical formulation of the closure problem, which can be quite involved, may skip directly to \ref{sec:application}.


\section{Reformulation of the closure terms}
\label{sec:reformulation}

Our final objective will be to compute our closures within the dilute Stokes flow regime for spherical particles of radius $a$. 
In this regime we expect that the closure terms will only be determined by the center of mass velocity of the droplets ($\textbf{u}_p$), their position in space, and the macroscopic properties of the continuous phase ($\textbf{u}_f, \grad \textbf{u}_f \ldots$). 
Indeed, as we will demonstrate in the following section, only these properties define the boundaries of the \textit{single-particle} conditionally-averaged equations, and are therefore sufficient to achieve closure of the problem  in the Stokes flow regime. 

Therefore, in the following, we express our closures in terms of averaged quantities, conditioned on the position of the center of mass of a \textit{test-particle} and its center of mass velocity. 
Note that for shape-dependent closure terms it would also be necessary to obtain the \textit{shape-conditioned} averaged fields, but this is out of the scope of this work.
  
\subsection{Interfacial terms}

In the equations presented in \ref{sec:application} certain closures, such as surface traction forces and surface heat fluxes, are expressed in the form of particle-averaged surface integrals. 
Here, we focus on reformulating these terms in terms of \textit{single-particle} conditionally-averaged quantities, which will be defined in the subsequent sections.

As the procedure is similar for all surface particle-averaged terms, let us take the example of the drag force term appearing in both averaged momentum equations (\ref{eq:dt_hybrid_rhou_f} and \ref{eq:dt_hybrid_up}). 
From the definition of the particle average we write,
\begin{align}
    \pSavg{\bm\sigma_f^0\cdot\textbf{n}}[\textbf{x},t]
    &= \avg{ \sum_{\alpha=1}^N \delta(\textbf{x}-\textbf{x}_\alpha[t; \FF])
    \int_{\Gamma_\alpha(t,\FF)}
    (\bm\sigma_f^0\cdot\textbf{n})[\textbf{y},t;\FF]
    d\Gamma[\textbf{y}] }
    \label{eq:first_step_reallay}
\end{align}
By writing this integral explicitly, we emphasize that the particle-averaged quantity (left-hand side of \ref{eq:first_step_reallay}) is evaluated at the point \textbf{x}, while the parameter $\textbf{y}$ is used for the integration over the particle's surface, which has its center of mass at \textbf{x} due to the presence of the Dirac delta. 
Thus, the notation $(\bm\sigma_f^0\cdot\textbf{n})[\textbf{y},t;\FF]$ means that we evaluate the local stress as well as the local normal $\textbf{n}$ to the particle, at \textbf{y}. 
We now enlarge the domain of integration from $\Gamma_\alpha$ (the surface of the particle $\alpha$) to $\mathbb{R}^3$ with the introduction of the interface indicator function of the particle $\alpha$, namely $\delta(|\textbf{y} - \textbf{x}_\alpha[t,\FF]| - a)$. 
It reads,
\begin{multline}
    \pSavg{\bm\sigma_f^0\cdot\textbf{n}}[\textbf{x},t]
    = \\
    \int_{\mathbb{R}^3}
    \avg{
     \sum_{\alpha=1}^N 
     \delta(\textbf{x}-\textbf{x}_\alpha[t \FF])
    \delta(|\textbf{y} - \textbf{x}_{\alpha}[t;\FF]|-a)
    (\bm\sigma_f^0\cdot\textbf{n})[\textbf{y},t;\FF]
    }
    d\textbf{y}. 
    \label{eq:first_step_drag}
\end{multline} 
Since the domain of integration $\mathbb{R}^3$ is now independent of $\FF$, we could substitute the integral and ensemble average operator. 
As mentioned above, we assume that our closure terms are entirely determined by the center of mass velocity of the particles and their position in space. 
Therefore, to include the condition on the particle velocity we introduce the relation 
\begin{equation}
    \int_{\mathbb{R}^3} \delta(\textbf{w} - \textbf{u}_\alpha[\FF,t]) d\textbf{w} = 1,
    \label{eq:Pw_normed}
\end{equation}
where $\textbf{w}$ is the test-particle center of mass velocity in the Eulerian space and $\textbf{u}_\alpha$ in the Lagrangian framework. 

Moreover, since the expression within the ensemble average in \ref{eq:first_step_drag} is identically zero when, $\textbf{x}_\alpha \neq \textbf{x}$, we may replace the interface indicator function such as   
\begin{equation}
    \delta(\textbf{x}-\textbf{x}_\alpha[t,\FF])\delta(|\textbf{y} - \textbf{x}_{\alpha}[t,\FF]|-a) = \delta(\textbf{x}-\textbf{x}_\alpha[t,\FF])\delta(|\textbf{y} - \textbf{x}|-a). 
    \label{eq:from_R3_to_S}
\end{equation}
Since the function $\delta(|\textbf{y} - \textbf{x}|-a)$ is not dependent on $\FF$ it can be taken out of the ensemble average operator, hence reducing the domain of integration from $\mathbb{R}^3$ to $|\textbf{y}-\textbf{x}| = a$ in \ref{eq:first_step_drag}. 
For deformable or non-spherical particles, this function may take a more complicated form, since the interface of a deformable or anisotropic particle is not entirely determinate by its center of mass position. 
It is clear that if the position of the interface is conditioned by the exact shape of the interface, then we can always express the interface indicator function as $\delta(f(\textbf{x}))$ where $f$ is a distance function of the shape that may depend on the particle's orientation (for anisotropic particles) or aspect ratio (for slightly deformable ones). 
Anyhow, this approach is generalizable to other kinds of particles, but in this work, we consider only spherical droplets. 

Injecting \ref{eq:Pw_normed} and \ref{eq:from_R3_to_S} in \ref{eq:first_step_drag} and using the last remark leads us to the relation 
\begin{equation}
    \pSavg{\bm\sigma_f^0\cdot\textbf{n}}[\textbf{x},t]
    =
    \int_{\mathbb{R}^3}
    P_1[\textbf{x},\textbf{w},t]
    \int_{|\textbf{x}-\textbf{y}|=a}
    (\bm\sigma_f^1 \cdot \textbf{n})[\textbf{y},\textbf{x},\textbf{w},t]
    d\textbf{y}
    d\textbf{w}
    \label{eq:conditionally_averaged}
\end{equation}
where we introduced the definitions, 
\begin{align}
    \bm\sigma^1_f[\textbf{y},\textbf{x},\textbf{w},t]
    % \phi_I^1[\textbf{y}|\textbf{x},\textbf{w},t] 
    % P_1[\textbf{x},\textbf{w},t]
    &= 
    \frac{1}{P_1}
    \avg{
    \sum_\alpha^N 
    \delta(\textbf{x} - \textbf{x}_\alpha[t,\FF])
    \delta(\textbf{w} - \textbf{u}_\alpha[t,\FF])
    % \delta(|\textbf{y} - \textbf{x}_{\alpha}[t,\FF]|-a)
    \bm\sigma_f^0[\textbf{y},t,\FF]
    },
    \label{eq:sigma_f_1}
    \\
    % \phi_I^1[\textbf{y}|\textbf{x},\textbf{w},t] 
    % P_1[\textbf{x},\textbf{w},t]\nonumber\\
    % = 
    % \avg{
    % \sum_\alpha^N 
    % \delta(\textbf{x} - \textbf{x}_\alpha[t,\FF])
    % \delta(\textbf{w} - \textbf{u}_\alpha[t,\FF])
    % \delta(|\textbf{y} - \textbf{x}_{\alpha}[t,\FF]|-a)
    % }\\
    P_1[\textbf{x},\textbf{w},t]
    &= 
    \avg{
    \sum_\alpha^N 
    \delta(\textbf{x} - \textbf{x}_\alpha[t,\FF])
    \delta(\textbf{w} - \textbf{u}_\alpha[t,\FF])
    }. 
\end{align}
% \tb{Since we reduced the surface int the stress isn't any more conditioned by the surface }
% Here $\bm\sigmais the \textit{single-particle} conditionally-averaged local stress of the continuous phase knowing that interface of the particle at $\textbf{x}$ is  present in \textbf{y} and that there is a particle at \textbf{x} with velocity \textbf{w}. 
With this definition, $\bm\sigma^1_f$ is the continuous phase stress, evaluated at $\textbf{y}$ and time $t$, averaged on every configuration where a particle is present at $\textbf{x}$ with velocity \textbf{w}, and that the point \textbf{y} is occupied by the surface of the test-particle.
% $\phi_I^1[\textbf{y}|\textbf{x},\textbf{w},t] $ is the probability of finding the interface of the particle at the location \textbf{y} knowing its center of mass is located at \textbf{x}. 
% For identical spherical particles we can write $\phi_I^1[\textbf{y}|\textbf{x},\textbf{w},t] = \delta(|\textbf{x} - \textbf{y}| -a)$.
% That is why the domain of integration over \textbf{y} in \ref{eq:conditionally_averaged} is reduced to $|\textbf{x} - \textbf{y}| < a$. 
$P_1[\textbf{x},\textbf{w},t]$ is the probability of finding a particle center of mass at \textbf{x} with velocity \textbf{w} at time $t$.
This distribution can be decomposed such that $P_1[\textbf{x},\textbf{w},t] = n_p[\textbf{x},t] P_1[\textbf{w}|\textbf{x},t]$ where $n_p$ is the number density evaluated at \textbf{x}, and $P_1$ the probability density of having a particle with velocity \textbf{w} knowing its center of mass location is \textbf{x}. 
Note that this distribution is normed 
\begin{equation*}
    \int_{\mathbb{R}^3} P_1[\textbf{w}|\textbf{x},t] d \textbf{w} = 1. 
\end{equation*}
Finally, note that in \ref{eq:conditionally_averaged} the droplet's shape and the position of its center of mass fully determine the normal vector $\textbf{n}$, allowing it to be taken outside the ensemble average. 

All quantities denoted with the superscript $^1$ refer to \textit{single-particle} conditionally-averaged quantities. 
These quantities are conditionally averaged based on the presence of a particle located at \textbf{x} with velocity \textbf{w}.
Additionally, in the following, we will use the shorthand,
\begin{equation}
    \delta_1[\textbf{x},\textbf{w},t,\FF]  \text{ for } \sum_\alpha \delta(\textbf{x} - \textbf{x}_\alpha[t,\FF]) \delta(\textbf{w} - \textbf{u}_\alpha[t,\FF]). 
\end{equation}  
Thus, we can define the \textit{single-particle} conditional-average of a local quantity $f^0$, as 
\begin{equation*}
    f^1[\textbf{y}|\textbf{x},\textbf{w},t] P_1[\textbf{x},\textbf{w},t] = \avg{\delta_1 f^0[\textbf{y},\FF,t]}.
\end{equation*}
Such that $f^1$ is the averaged value of $f^0$ at \textbf{y} over all configurations where a particle is present at \textbf{x} with velocity \textbf{w}. 
If $f_k$ is a quantity defined in the phase $k$ we may write 
\begin{equation*}
    f^1_k[\textbf{y}|\textbf{x},\textbf{w},t] \phi_k^1[\textbf{y}|\textbf{x},\textbf{w},t]  P_1 = \avg{\delta_1 \chi_k f^0_k [\textbf{y},\FF,t]}.
\end{equation*}
Such that $\phi_k^1$ is the probability of finding the phase $k$ at \textbf{y} knowing a particle is located at \textbf{w} with velocity \textbf{w}, and $f_k^1$ the conditional average of $f_k^0$. 
For example, we define the \textit{single-particle conditioned average} of the continuous phase velocity fields by, 
\begin{equation*}
    \textbf{u}_f^1 \phi_f^1 P_1
    = \avg{\delta_1 \chi_f \textbf{u}_f^0}
\end{equation*} 
where $\phi_f^1$ is the probability of finding the continuous phase at \textbf{y} knowing a particle is present at \textbf{x} with velocity \textbf{w}. 
$\textbf{u}_f^1$ is the averaged local velocity evaluated at \textbf{y} knowing the continuous phase is present at \textbf{y} with a particle at \textbf{x} having velocity \textbf{w}. 

The stress present in \ref{eq:conditionally_averaged} can now be expressed in terms of conditionally-averaged velocity and pressure fields. 
We use the constitutive law of Newtonian fluids, $\bm\sigma_f^0 = -p_f^0 \bm\delta + \mu_f [\pddy \textbf{u}_f^0+ (\pddy \textbf{u}_f^0)^\dagger]$, with $\pddy$ the gradient over the \textbf{y} coordinate.
Injecting this law into \ref{eq:sigma_f_1} we obtain directly 
\begin{equation}
    \bm\sigma_f^1
    = - p_f^1 \bm\delta
    +\mu_f [\pddy \textbf{u}_f^1+(\pddy \textbf{u}_f^1)^\dagger], 
    \label{eq:sigma_f_surf}
\end{equation}
where, we observed that $\delta_1$ is independent of \textbf{y}, meaning that it could be permuted with the gradient operator in \ref{eq:sigma_f_surf}. 
In conclusion, the ensemble averaged interphase drag force term can be expressed as the surface integral of the Newtonian stress defined based on the \textit{single-particle} conditional-averaged fields: $\textbf{u}_f^1$, and $p_f^1$. 

It must be understood that this definition \eqref{eq:sigma_f_surf} is only true because the stress is evaluated at the surface of the droplet, in the general case we have, 
\begin{equation}
    P_1 \phi_f^1 \bm\sigma_f^1
    =
    P_1 \phi_f^1 p_f^1
    +  \avg{\delta_1\chi_f \{ - p_f^0 \bm\delta + \mu_f [\pddy \textbf{u}_f^0+  (\pddy \textbf{u}_f^0)^\dagger]\}}. 
\end{equation}
From this expression, we can propose two formulations for the continuous phase stress,   
\begin{align}    
    \bm\sigma_f^1
    &= 
    - p_f^1 \bm\delta 
    + \mu_f [\pddy \textbf{u}_f^1+ (\pddy \textbf{u}_f^1)^\dagger ]
    - \frac{\mu_f }{P_1 \phi_f^1}\avg{\delta_1 \delta_\Gamma (\textbf{n}_d \textbf{u}_f''+  \textbf{u}_f'' \textbf{n}_d)},
    \label{eq:sigma_f_surf15}
    \\
    \bm\sigma_f^1
    &= 
    - p_f^1 \bm\delta 
    + \frac{\mu_f }{\phi_f^1} [\pddy \textbf{u}^1+ (\pddy \textbf{u}^1)^\dagger ]
    - \frac{\mu_f }{\phi_f^1} \phi_d^1 \textbf{e}_d^1. 
    \label{eq:sigma_f_surf2}
\end{align}
Where we have defined $\textbf{u}_f'' = \textbf{u}_f^0 - \textbf{u}_f^1$. 
Notice that the second term on the right-hand side of \ref{eq:sigma_f_surf2} vanishes for solid particles, because there are no shearing motions inside solid particles.
While the second term of \ref{eq:sigma_f_surf15} does not necessarily vanish. 
That is why the second expression is preferred for solid particles. 

% Additionally, multiplying \ref{eq:sigma_f_surf15} and \ref{eq:sigma_f_surf2} by $\phi_f^1$ and subtracting both expressions gives, 
% \begin{equation} 
%     \frac{1}{P_1}\avg{\delta_1 \delta_\Gamma (\textbf{n}_d \textbf{u}_f''+  \textbf{u}_f'' \textbf{n}_d)}
%     -  \phi_d^1 \textbf{e}_d^1
%     = 
%     [(\textbf{u}_f^1 - \textbf{u}_d^1)\pddy\phi_d^1+ \pddy \phi_d^1(\textbf{u}_f^1 - \textbf{u}_d^1) ]
%     - \phi_d^1 [\pddy\textbf{u}_d^1+  (\pddy \textbf{u}_d^1)^\dagger ]. 
% \end{equation}
% For ensemble averaged (not conditionally-averaged) quantities we can equally show that
% \begin{multline} 
%     \avg{\delta_\Gamma (\textbf{n}_d \textbf{u}_f'+  \textbf{u}_f' \textbf{n}_d)}
%     - \phi_d \textbf{e}_d
%     = 
%     +  [(\textbf{u}_f - \textbf{u}_d)\pddy\phi_d+ \pddy \phi_d(\textbf{u}_f - \textbf{u}_d) ]
%     - \phi_d [\pddy\textbf{u}_d+  (\pddy \textbf{u}_d)^\dagger ]
% \end{multline}
% Note that for solid particles this relation can directly lead us to a closure for the surface term on the left-hand side, namely
% \begin{multline} 
%     \avg{\delta_\Gamma (\textbf{n}_d \textbf{u}_f'+  \textbf{u}_f' \textbf{n}_d)}
%     = 
%     [(\textbf{u}_f - \textbf{u}_d)\pddy\phi_d+ \pddy \phi_d(\textbf{u}_f - \textbf{u}_d) ]
%     - \phi_d [\pddy\textbf{u}_d+  (\pddy \textbf{u}_d)^\dagger ]. 
%     \label{eq:closure_un_nu}
% \end{multline}


At the surface of the test-particle, the probability of finding the dispersed phase, i.e. finding another particle in contact with the reference particle, is identically null. 
Indeed, a thin film of continuous phase always separates the droplet's surface from its neighbors.
Therefore, we may write $\phi_d^1 = 0$ at $|\textbf{x}- \textbf{y}| =a$. 
Thus, when evaluated at the subsurface of the particle, we can use the relation $\textbf{u}_f^1 = \textbf{u}^1$ and $p_f^1 = p^1$. 
Consequently, to compute the surface stress of a particle, either the conditionally-averaged quantities of the continuous phase ($\textbf{u}_f^1, p_f^1$) or the bulk quantities ($\textbf{u}^1$, $p^1$) are required. 
Another consequence of this is that, the definitions given by, \ref{eq:sigma_f_surf15}, \ref{eq:sigma_f_surf2}, or \ref{eq:sigma_f_surf} are all consistent when evaluated at the points located on the surface of the droplet at \textbf{x}. 

\subsection{Mean fields and disturbance fields contribution}

As it is often done in the literature \citep{zhang1994ensemble,jackson2000,wang2021numerical,wang2024effect}, we would like to separate the contribution of the drag force into a contribution from the mean flow and pressure fields, and the one arising due to the disturbance velocity and pressure fields.  

\subsubsection{Definitions}

In the first place, we need to define what is a disturbance field.
Let us take the example of the conditioned velocity field, $\textbf{u}_f^1$, and its corresponding disturbance velocity field.
We first remark that,   
\begin{equation}
    \lim_{|\textbf{y}-\textbf{x}|\to\infty} 
    \textbf{u}_f^1[\textbf{y},\textbf{x},\textbf{w},t]
    =
    \textbf{u}_f[\textbf{y},t]. 
    \label{eq:lim_u_1}
\end{equation} 
This, relation means that the conditioned field, $\textbf{u}_f^1$, is equivalent to the semble-averaged velocity field $\textbf{u}_f$ when the particle at \textbf{x} is sufficiently far from the point where the velocity $\textbf{u}_f^1$ is evaluated (the point \textbf{y}). 
Note that this definition requires an infinitely large domain. 
This implies that the solutions obtained in the subsequent sections are restricted to infinitely large domains, devoid of boundary conditions.
Note that, since the particle-size scale $a$ is much smaller than the boundary length scale $L$, the boundaries can be considered as infinitely far from the point \textbf{x}, in which case \ref{eq:lim_u_1} is valid.
However, it is interesting to note that this will no longer be the case for other problems such as sediment transport for examples.
In this case, the boundary condition, i.e. the top of the particle bed and the ground, are at a distance of the same length scale as the particle-size. 

In light of \ref{eq:lim_u_1}, we define the disturbance velocity field as 
\begin{equation}
    \textbf{u}_f^{1d}
    =
    \textbf{u}_f^1 
    - 
    \textbf{u}_f. 
    \label{eq:def_u_1d}
\end{equation}
because it satisfies the definition, 
\begin{equation}
    \lim_{|\textbf{y}-\textbf{x}|\to\infty} 
    \textbf{u}_f^{1d}[\textbf{y},\textbf{x},\textbf{w},t]
    =
    \lim_{|\textbf{y}-\textbf{x}|\to\infty} 
    \{\textbf{u}_f^1[\textbf{y},\textbf{x},\textbf{w},t]
    - \textbf{u}_f[\textbf{y},t]\}
    = 0.
    \label{eq:lim_u_1d}
\end{equation} 
Thus, $\textbf{u}_f^{1d}$ tends to zero at large distances from the particle, this is thus consistent with the terminology used here, i.e. with the word ``disturbance''.
The definition \ref{eq:def_u_1d} can apply to any \textit{conditional-averaged} quantities $f^1$, we define
\begin{equation}
    \lim_{|\textbf{y}-\textbf{x}|\to\infty} 
    \{f_f^1[\textbf{y},\textbf{x},\textbf{w},t]
    - f_f[\textbf{y},t]\}
    =
    f_f^{\Delta}[\textbf{y},\textbf{x},\textbf{w},t]
    = 0.
\end{equation} 

\subsubsection{The force traction decomposition}

Using the decomposition $\bm\sigma_f^1 = \bm\sigma_f^{1d} + \bm\sigma_f$ in \ref{eq:conditionally_averaged} we finally introduce the decomposition of the drag force term as:  
\begin{align}
    \pSavg{\bm\sigma_f^0\cdot\textbf{n}}[\textbf{x},t]
    =
    n_p[\textbf{x},t]
    \int_{|\textbf{x}-\textbf{y}|=a}
    \bm\sigma_f[\textbf{y},t]
    \cdot \textbf{n}
    d\textbf{y}\\
    + 
    \int_{\mathbb{R}^3}
    P_1[\textbf{x},\textbf{w},t]
    \int_{|\textbf{x}-\textbf{y}|=a}
    \bm\sigma_f^{1d}[\textbf{y},\textbf{x},\textbf{w},t]
    \cdot \textbf{n}
    d\textbf{y}
    d\textbf{w}
    \label{eq:general_partition}
\end{align}
where the first term represents the contribution from the mean continuous phase stress, $\bm\sigma_f$, and the second term is the contribution from the disturbance fields stress $\bm\sigma_f^{1d}$. 
While this decomposition is arbitrary since $\bm\sigma_f^1$ could be partitioned into two other arbitrary tensors, it enables by definition, the separation of the mean flow contribution, $\bm\sigma_f$, from the drag induced by the local-scale disturbance fields, $\bm\sigma_f^{1d}$.
This decomposition is utilized by \citet[Chapter 2]{jackson2000} and \citet{zhang1997momentum,wang2021numerical,wang2024effect} for suspensions of solid spheres, although these authors do not explicitly provide the expression for the tensor $\bm\sigma_f^1$, which we aim to derive in the following sections. 

Note that $\bm\sigma_f$ is evaluated at $\textbf{y}$. 
Although $\bm\sigma_f$ is ensemble-averaged, we emphasize that it may still depend on the position in inhomogeneous flows. 
Consequently, to extract it from the integral, we recognize that $\bm\sigma_f[\textbf{y},t] = \bm\sigma_f[\textbf{x},t] + \textbf{r}\cdot \nabla\bm\sigma_f[\textbf{x},t] + \ldots$, where $\textbf{r} = \textbf{y} - \textbf{x}$. 
By retaining only the first three terms in the expansion, we can demonstrate that
\begin{equation}
    \pSavg{\bm\sigma_f^0\cdot\textbf{n}}
    =
    n_p v_p 
    \div\bm\sigma_f
    +
    \int_{\mathbb{R}^3}
    P_1
    \int_{|\textbf{x}-\textbf{y}|=a}
    \bm\sigma_f^{1d} \cdot \textbf{n}
    d\textbf{y}d\textbf{w}
    \label{eq:drag_final}
\end{equation}
Therefore, it is evident that the drag force term contains a component related to the divergence of the mean fluid-phase stress, in addition to the contribution from the disturbance fields. 
Similar arguments can be extended to the first two moments of the hydrodynamic force traction. 
This expression can be written as:
\begin{align}
    \pSavg{\textbf{r}\bm\sigma_f^0\cdot\textbf{n}}
    &=
    n_p v_p \bm\sigma_f
    +
    \int_{\mathbb{R}^3}
    P_1
    \int_{|\textbf{x}-\textbf{y}|=a}
    \textbf{r}\bm\sigma_f^{1d} \cdot \textbf{n}
    d\textbf{y}
    d\textbf{w}
    \label{eq:first_mom_general}
    \\
    \pSavg{\textbf{rr}\bm\sigma_f^0\cdot\textbf{n}}
    &=
    n_pv_p  \frac{a^2}{5} 3 [(\div \bm\sigma_f)\bm\delta]^\text{sym}
    +
    \int_{\mathbb{R}^3}
    P_1
    \int_{|\textbf{x}-\textbf{y}|=a}
    \textbf{rr}\bm\sigma_f^{1d} \cdot \textbf{n}
    d\textbf{y}
    d\textbf{w}
    \label{eq:second_mom_general}
\end{align}
where the operator $[\ldots]^\text{sym}$ returns the symmetric part of the arguments. 
It is important to note that the contribution from the mean stress in the second moment of the hydrodynamic force may become negligible for small $\phi_d$, as the term $a^2$ appears in this expression. 

According to the expressions \ref{eq:drag_final}, \ref{eq:first_mom_general}, and \ref{eq:second_mom_general}, we will need to compute the term $\bm\sigma^{1d}_f = \bm\sigma_f^1 + \bm\sigma_f$ where $\bm\sigma_f^1$ is given by \ref{eq:sigma_f_surf}. 
Regarding the mean continuous phase stress we recall that it can be written in two ways, namely 
\begin{align}
    \label{eq:mean_continuous_phase_stress}
    \bm\sigma_f
    &= - p_f \bm\delta 
    + \mu_f [\pddy \textbf{u}_f+ (\pddy \textbf{u}_f)^\dagger ]
    - \frac{\mu_f}{\phi_f}\avg{\delta_\Gamma (\textbf{n}_d \textbf{u}_f'+  \textbf{u}_f' \textbf{n}_d)}, \\
    \bm\sigma_f
    &= - p_f \bm\delta 
    + \frac{\mu_f}{\phi_f} [\pddy \textbf{u}+ (\pddy \textbf{u})^\dagger ]
    - \frac{\mu_f \phi_d}{\phi_f} \textbf{e}_d
    \label{eq:mean_continuous_phase_stress2}
\end{align}
Thus using \ref{eq:sigma_f_surf} and \ref{eq:mean_continuous_phase_stress} we find for the points on the particle's surface ($|\textbf{y}-\textbf{x}| = a$) that,
\begin{align}
    \bm\sigma_f^{1d}
    =
    - p_f^{1d} \bm\delta 
    + \mu_f [\pddy \textbf{u}_f^{1d}+ (\pddy \textbf{u}_f^{1d})^\dagger ]
    + \frac{\mu_f }{\phi_f}\avg{\delta_\Gamma (\textbf{n}_d \textbf{u}_f'+  \textbf{u}_f' \textbf{n}_d)}. 
    \label{eq:sigma_explict}
\end{align}
Thus, according to \ref{eq:sigma_explict}, the disturbance stress that is integrated over the particle surface in \ref{eq:drag_final} to \ref{eq:second_mom_general}, is not only the Newtonian stress-like contribution of the disturbance fields ($\textbf{u}_f^1$ and $p_f^1$), but also includes the contribution of the term $\avg{\delta_\Gamma (\textbf{n}_d \textbf{u}_f'+  \textbf{u}_f' \textbf{n}_d)}$. 
Thus, in the force traction closures: \ref{eq:drag_final} and \ref{eq:first_mom_general}, we will observe the appearance of the terms involving the divergence of $\avg{\delta_\Gamma (\textbf{n}_d \textbf{u}_f'+  \textbf{u}_f' \textbf{n}_d)}$ times $n_pv_p$, and  $n_pv_p \avg{\delta_\Gamma (\textbf{n}_d \textbf{u}_f'+  \textbf{u}_f' \textbf{n}_d)}$, respectively. 
These terms will ultimately cancel out their corresponding contributions in $\bm\sigma_f$ that appear on the left-hand side of \ref{eq:drag_final} and \ref{eq:first_mom_general}. 

To provide a better understanding for the following discussion we remark that for solid particles $\textbf{e}_d = 0$. 
Therefore, subtracting \ref{eq:mean_continuous_phase_stress} from \ref{eq:mean_continuous_phase_stress2}, gives directly, 
\begin{equation}
    \avg{\delta_\Gamma (\textbf{n}_d \textbf{u}_f'+  \textbf{u}_f' \textbf{n}_d)}
    = 
    (\textbf{u}_f - \textbf{u}_d)\pddy \phi_d + \pddy \phi_d (\textbf{u}_f - \textbf{u}_d)
    -  \phi_d [\pddy \textbf{u}_d+ (\pddy \textbf{u}_d)^\dagger ]. 
    \label{eq:closure_un_nu}
\end{equation} 
Therefore, at least for solid spheres, this term is non-zero as soon as there are non-negligible gradients of volume fraction and mean gradients of particle velocities. 


Thus, we conclude that the commonly used partition of the drag force, given by \ref{eq:general_partition}, does not constitute the best decomposition for this term since it requires adding the term  $\avg{\delta_\Gamma (\textbf{n}_d \textbf{u}_f'+  \textbf{u}_f' \textbf{n}_d)}$ to the classical Newtonian stresses in the second term of \ref{eq:general_partition} and subtracting it in the mean drag force term (first term of \ref{eq:general_partition}). 
This finding implies that, in the recent work of \citet{wang2021numerical, wang2024effect}, where this decomposition is employed for the drag force, we assert that they have actually computed the integral of the first two terms of \ref{eq:sigma_explict}, while neglecting the final term. 
Interestingly, \citet{wang2024effect} specifically investigates the effect of the volume fraction gradient ($\grad \phi_d$) on the drag force. 
In this context, it is clear from \eqref{eq:closure_un_nu} that the term $\avg{\delta_\Gamma (\textbf{n}_d \textbf{u}_f'+  \textbf{u}_f' \textbf{n}_d)}$ cannot be neglected. 
Only when the analysis is accurate to $\mathcal{O}(\phi_d)$ does this term vanish in \ref{eq:drag_final}, reducing \ref{eq:sigma_explict} to the Newtonian stress expression. 
Thus, the drag force computed in the DNS of \citet{wang2024effect} may not be the one defined in their momentum equations. 

Note that using \ref{eq:mean_continuous_phase_stress2} instead of \ref{eq:mean_continuous_phase_stress} in \ref{eq:sigma_explict} and switching $\textbf{u}_f^1$ and $\textbf{u}^1$ in \ref{eq:sigma_f_surf}, does not solve this inconsistency. 

\subsubsection{A better and simpler forces decomposition}

As the previous decomposition requires adding and subtracting $\avg{\delta_\Gamma (\textbf{n}_d \textbf{u}_f'+  \textbf{u}_f' \textbf{n}_d)}$ in each term of \ref{eq:general_partition}, we decide to avoid this overcomplicated operation and introduce, the partitioning 
\begin{equation}
    \bm\sigma_f^1 =
    \bm\Sigma_f + 
    \bm\Sigma_f^{1d}. 
    \label{eq:mean_Newtonian}
\end{equation}
Where the mean stress and disturbance stresses are defined as, 
\begin{align}
    \bm\Sigma_f^{1d}
    &=-p_f^{1d}\bm\delta + \mu_f^1 [\grad \textbf{u}^{1d}_f + (\grad \textbf{u}^{1d}_f)^\dagger], \\
    \bm\Sigma_f
    &=-p_f\bm\delta + \mu_f^1 [\grad \textbf{u}_f + (\grad \textbf{u}_f)^\dagger], 
\end{align}
respectively. 

Using the same methodology as in the previous manipulations we re-write the force traction closures such that, 
\begin{align}
    \pSavg{\bm\sigma_f^0\cdot\textbf{n}}
    &=
    n_p v_p 
    \div\bm\Sigma_f
    +
    \int_{\mathbb{R}^3}
    P_1
    \int_{|\textbf{x}-\textbf{y}|=a}
    \bm\Sigma_f^{1d} \cdot \textbf{n}
    d\textbf{y}d\textbf{w}
    \label{eq:drag_final2}\\
    \pSavg{\textbf{r}\bm\sigma_f^0\cdot\textbf{n}}
    &=
    n_p v_p \bm\Sigma_f
    +
    \int_{\mathbb{R}^3}
    P_1
    \int_{|\textbf{x}-\textbf{y}|=a}
    \textbf{r}\bm\Sigma_f^{1d} \cdot \textbf{n}
    d\textbf{y}
    d\textbf{w}
    \\
    \pSavg{\textbf{rr}\bm\sigma_f^0\cdot\textbf{n}}
    &=
    n_pv_p  \frac{a^2}{5} 3 [(\div \bm\Sigma_f)\bm\delta]^\text{sym}
    +
    \int_{\mathbb{R}^3}
    P_1
    \int_{|\textbf{x}-\textbf{y}|=a}
    \textbf{rr}\bm\Sigma_f^{1d} \cdot \textbf{n}
    d\textbf{y}
    d\textbf{w}
    \label{eq:second_mom_general2}
\end{align}
This decomposition, though perhaps less elegant, appears to be more physically meaningful. 
Indeed, unlike the previous approach, the mean contribution no longer depends on the mean relative motion through the term $\avg{\delta_\Gamma (\textbf{n}_d \textbf{u}_f'+  \textbf{u}_f' \textbf{n}_d)}$ in $\bm\sigma_f$ (see \ref{eq:closure_un_nu}), but solely on the mean fluid phase properties $p_f$ and $\textbf{u}_f$, while the local stress is a Newtonian-like stress.  

The main take-away of this section is that: (1)  the particle-averaged force traction terms can be computed based on the knowledge of $p_f^{1d}$ and $\textbf{u}_f^{1d}$. 
These fields can be obtained by solving the corresponding disturbance field equations. 
Notice that one may also compute the mixture properties $p^{1d}$ and $\textbf{u}^{1d}$ and then use the relation $p_f^{1d} = p^{1d} + \phi_d (p_f - p_d)$ or $\textbf{u}_f^{1d} = \textbf{u}^{1d} + \phi_d (\textbf{u}_f - \textbf{u}_d)$. 
And (2) the force decomposition often used \citep{jackson2000,zhang1997momentum,wang2021numerical,wang2024effect} seems inconsistent compared to the one computed in the cited studies.
This inconsistency is settled by re-defining the forces partition.  

\subsection{Particle phase volume terms}

Some closure terms such as the particle internal stress $\pOavg{\bm{\sigma}_2^0}$ or the particle internal dissipation term $\pOavg{\bm{\sigma}_2^0:\grad \textbf{u}_d^0}$ are particle-averaged volume integral of locals quantities. 
In this situation the reformulation is slightly different since we must consider volume and not surfaces of the particle nevertheless the approach is similar. 
For the particle internal stress we can write, 
\begin{equation}
    \pOavg{\bm\sigma_d^0}[\textbf{x},t]
    =
    \int_{\mathbb{R}^3}
    P_1[\textbf{x},\textbf{w}]
    \int_{|\textbf{x}-\textbf{y}|<a}
    \bm\sigma_d^1[\textbf{y},t;\textbf{x},\textbf{w}] 
    d\textbf{y}
    d\textbf{w}. 
    \label{eq:conditionally_averaged_vol}
\end{equation}
Assuming a Newtonian fluid for the particles, $\bm\sigma_d^0 = -p_d^0 \bm\delta + \mu_d [\nabla \textbf{u}_d^0 + (\nabla \textbf{u}_d^0)^\dagger]$, and given that within the region $|\textbf{x} - \textbf{y}| < a$, only the dispersed phase is present, allowing for the permutation between ensemble averages and derivatives, we can express this as:
\begin{equation}
    \bm\sigma_d^1  
    = 
    -p_d^1   \bm\delta
    + \mu_d  [\pddy \textbf{u}^1_d+(\pddy  \textbf{u}^1_d)^\dagger],
    \label{eq:dispersed_phase_stress}
\end{equation}
which is simply the expression of the Newtonian stress within the particle centered at \textbf{x}, based on the mean fields $p_d^1$ and $\textbf{u}_d^1$. 


\subsection{Continuous phase closures}

The closure terms of the form $\avg{\chi_f f_f^0}$ differ in their mathematical structure, as they represent an average over the continuous phase rather than the dispersed phase. 
Consequently, the reformulation method is slightly different and requires additional assumptions. Two examples of such terms are the Reynolds stress $\avg{\chi_f \textbf{u}_f'\textbf{u}_f'}$ and the fluid-phase dissipation $\avg{\chi_f \bm\sigma_f^0 : \nabla \textbf{u}_f^0}$, which appear in \ref{eq:dt_hybrid_rhou_f} and \ref{eq:dt_hybrid_k1}, respectively.  

We first notice that, 
\begin{equation}
    \frac{1}{N}\sum_\alpha^N
    \int_{\mathbb{R}^3}
    \int_{\mathbb{R}^3}
    \delta(\textbf{y}-\textbf{x}_\alpha[\FF,t])
    \delta(\textbf{w}-\textbf{u}_\alpha[\FF,t])
    d\textbf{x}
    d\textbf{w}
    = 1,
\end{equation}
where $N$ is the total number of particles in the flow. 
Using this relation one may re-formulate the ensemble average of a continuous phase quantity as 
\begin{equation}
    \phi_f f_f[\textbf{x},t]
    = 
    \frac{1}{N}
    \int_{\mathbb{R}^3}
    \int_{\mathbb{R}^3}
    f_f^1[\textbf{x},\textbf{y},\textbf{w},t] \phi_f^1[\textbf{x}|\textbf{y},\textbf{w},t]  P_1[\textbf{y},\textbf{w}] 
    d\textbf{y} 
    d\textbf{w}
    \label{eq:conditional_averaged_fluid}
\end{equation}
where,
\begin{equation*}
    f_f^1[\textbf{x},\textbf{y},\textbf{w},t] \phi_f^1[\textbf{x}|\textbf{y},\textbf{w},t]  P_1[\textbf{y},\textbf{w}]
    =     
    \avg{
    \sum_\alpha^N 
    \delta(\textbf{y}-\textbf{x}_\alpha[\FF,t])
     \delta(\textbf{w}-\textbf{u}_\alpha[\FF,t])
    (\chi_f
    f^0_f)[\textbf{x},t;\FF]
    }.
\end{equation*}
In this expression $f_f^1[\textbf{x},t;\textbf{y},\textbf{w}]$ is the average of the local quantity $f_f^0$ evaluated at $\textbf{x}$ and time $t$ conditionally on, the presence of the continuous phase at \textbf{x}, and a particle center of mass at $\textbf{y}$ with center of mass velocity $\textbf{w}$. 
Similarly, $\phi_f^1[\textbf{x},t;\textbf{y},\textbf{w}]$ is the fluid phase volume fraction at \textbf{x} and time $t$, conditionally on the presence of a particle at $\textbf{y}$ with center of mass velocity \textbf{w}. 
Notice that for $|\textbf{x} - \textbf{y}| < a$, $\phi_f^1[\textbf{y}|t,\textbf{x},\textbf{w}] = 0$ however at, 
$\lim_{|\textbf{x} - \textbf{y}| \to \infty} \phi_f^1 = \phi_f$. 
Notice that this derivation is consistent with (2.21) and (2.22) of \citet{zhang1994ensemble} with $K = 1$. 

This, formulation remains quite general and is valid regardless of the flow regime, however, the presence of the term $N$ makes this formulation unpractical. 
Indeed, $P_1 = n_p[\textbf{y},t] P_1[\textbf{w}|\textbf{y},t]$ and $n_p[\textbf{y},t] /N = V_\Omega$, where $V_\Omega$ is the volume of the whole domain. 
Remark that substituting this relation into \ref{eq:conditional_averaged_fluid} transforms the right-hand side of this relation to a volume average over $V_\Omega$ of a property evaluated at \textbf{x} on all possible particles positions in $V_\Omega$.  
Anyhow, \ref{eq:conditional_averaged_fluid} requires macroscopic information such as $N$ and $V_\Omega$, which we do not necessarily have if our goal is to compute general closure formulation. 
This is because, contrary to particle-averaged quantities, we could not consider a contribution per particle that is holds fixed at \textbf{x}, but the action of all particles on a given property of the fluid at \textbf{x}.
In other words, the integration variable is on the particle center of mass position in \ref{eq:conditional_averaged_fluid}, while in \ref{eq:conditionally_averaged} it is on the local non-averaged properties while the particle position remains fixed. 

Consequently, we adopt the approach proposed by \citet{batchelor1972sedimentation} and reformulate \ref{eq:conditional_averaged_fluid} based on the additivity assumption. 
Thus, we postulate that $f_f^0[\textbf{x},t;\FF]$ can be subdivided into $N$ contributions, namely:  
\begin{equation}
    f_f^0[\textbf{x},t;\FF]
    = 
    \sum_i^N
    f_{f_i}^0[\textbf{x},t;\FF]
    + f_{f_0}^0[\textbf{x},t;\FF]
\end{equation}
where $f_{f_i}^0$ is the disturbance fields produced by the particle $i$ on $f_f^0$ and $f_{f_0}^0$ is the undisturbed background flow. 
This implies that $f_{f}^0 = f_{f_0}^0$ in the absence of particle in the flow. 
Under this assumption, we can write, 
\begin{equation}
    \avg{\chi_f f_f^0}[\textbf{x},t]
    = 
    \int_{\mathbb{R}^3} 
    \avg{
        \sum_i^N 
    (\chi_f f_{f_i}^0)[\textbf{x}_i + \textbf{r},t,\FF] \delta(\textbf{x} - \textbf{x}_i[\FF,t] - \textbf{r})}d\textbf{r}
    +( \phi_f f_{f_0})[\textbf{x},t]
    \label{eq:first_step_additivity}
\end{equation}
Where $\phi_f f_{f_0}[\textbf{x},t]$ is the mean background flow, and were we have used a relation similar to \ref{eq:taylor_f_d}, to reformulate the first term of \ref{eq:first_step_additivity}. 
Then, we use the Taylor expansion, $\delta(\textbf{x} - \textbf{x}_i - \textbf{r}) =\delta(\textbf{x} - \textbf{x}_i) - \textbf{r}\cdot \grad\delta(\textbf{x} - \textbf{x}_i)+ \ldots$, on the first term on the right-hand side of \ref{eq:first_step_additivity} which gives,  
\begin{align}
    \avg{\chi_f f_f^0}[\textbf{x},t]
    = 
    \phi_f f_{f_0}[\textbf{x},t]
    + 
    \int_{\mathbb{R}^6} 
    (f_{f_p}^1\phi_f^1) [\textbf{y}|\textbf{x},\textbf{w},t] P_1[\textbf{w},\textbf{x}]
    d\textbf{r}
    d\textbf{w}\\
    + 
    \div 
    \int_{\mathbb{R}^6} 
    \textbf{r}
    (f_{f_p}^1\phi_f^1) [\textbf{y}|\textbf{x},\textbf{w},t] P_1[\textbf{w},\textbf{x}]
    d\textbf{r}
    d\textbf{w}
    + \ldots
    % + \grad^n 
    % \int_{\mathbb{R}^6} 
    % \mathcal{O}(\textbf{r}^n)
    % (f_{f_p}^1\phi_f^1) [\textbf{y}|\textbf{x},\textbf{w},t] P_1[\textbf{w},\textbf{x}]
    % d\textbf{r}
    % d\textbf{w}
    \label{eq:f_f_1_def}
\end{align}
with, 
\begin{equation}
    (f_{f_p}^1 \phi_f^1) [\textbf{y}|\textbf{x},\textbf{w},t] P_1[\textbf{w},\textbf{x}]
    = 
    \avg{
    \sum_i
    \chi_f f_{f_i}^0[\textbf{x}_i + \textbf{r},t;\FF] 
    \delta(\textbf{x} - \textbf{x}_i[\FF,t])
    \delta(\textbf{w} - \textbf{u}_i[\FF,t])
    }. 
\end{equation}
In this definition $f_{f_p}^1$ is the averaged value of the disturbance fields at $\textbf{x}+\textbf{r}$, produced by the particle at $\textbf{x}$, in opposition to $f_f^1$ \eqref{eq:conditional_averaged_fluid} which is the averaged value of $f_f^0$ evaluated at \textbf{x}, conditionally on the presence of an arbitrary particle at \textbf{y}.
Assuming a situation where there is no background flow (such as in the case of segmenting particle in an otherwise inert flow), a homogeneous situation, and in the dilute limit, such that $\phi_f^1 = \phi_f$ when $|\textbf{x}-\textbf{y}|>a$ and $\phi_f^1 =0$ when $|\textbf{x}-\textbf{y}|<a$, we obtain, 
\begin{equation}
    f_f[\textbf{x},t]
    = 
    \int_{\mathbb{R}^3} 
    P_1[\textbf{x},\textbf{w}] 
    \int_{|\textbf{x}-\textbf{y}| >a} 
    f_{f_p}^1[\textbf{x}+ \textbf{r}| \textbf{x}]
    d\textbf{r}
    d\textbf{w}
    + 
    \text{Error}
    \label{eq:Batchelor2}
\end{equation}
\begin{equation}
    \text{Error}
    = 
    \int{
    \mathcal{O}(|\textbf{r}| f_{f_p}^1  n_p / L)
    } d\textbf{r}. 
    \label{eq:error0}
\end{equation}
Note that in \ref{eq:error0} we have expressed explicitly the error generated due to the Taylor expansion of the Dirac delta: $\delta(\textbf{x} - \textbf{x}_i - \textbf{r})$. 
Indeed, at the leading order we find, $\delta(\textbf{x} - \textbf{x}_i - \textbf{r}) =\delta(\textbf{x} - \textbf{x}_i) +  \mathcal{O}(|\textbf{r}|/L)$, where $L$ is the typical length of the macroscopic flow variation.
Assuming that $\text{Error}= \mathcal(\phi_d^2)$ rather than \ref{eq:error0} in \ref{eq:Batchelor2}, we find that \ref{eq:Batchelor2} is exaclty equation (2.10) of \citet{batchelor1972sedimentation}. 
\citet{batchelor1972sedimentation} uses such a formula to compute the mean fluid phase velocity at a given point in the fluid, conditionally on the presence of a particle at a certain distance from this point. 
Consequently, \ref{eq:Batchelor2} is a generalization of Eq (2.10) of \citet{batchelor1972sedimentation}, in the sense that we give an explicit expression of the``Error'' term \eqref{eq:error0} based on mathematical arguments, rather than Batchelor's physical arguments. 
Additionally, \ref{eq:f_f_1_def} is a generalization of \ref{eq:Batchelor2} when the homogeneous hypothesis, as well as the dilute hypothesis, are not assumed. 



As discussed in \citet{batchelor1972sedimentation}, the first integral in \ref{eq:f_f_1_def} may diverge if the disturbance field $f_{f_p}^1$ is unbounded, and if it does not decay rapidly enough as $|\textbf{r}|$ approaches infinity. 
We think that, in cases where $f_{f_p}^1$ does not decay sufficiently fast as $|\textbf{r}|$ increases, the ``Error'' in \ref{eq:Batchelor2} also tends to infinity since we are integrating a term proportional to $f_{f_p}^1 |\textbf{r}|$ which is even more divergent for large $|\textbf{r}|$. 
Thus, we argue that Batchelor's original formula is not accurate at $\mathcal{O}(\phi_d^2)$, but rather at $\mathcal{O}(|\textbf{r}| f_{f_p}^1  n_p / L)$, making \ref{eq:Batchelor2} unable to produce physical results when $f_{f_p}^1$ does not decay rapidly since the ``Error'' also tends to infinity in these cases. 
In cases where the first integral on the right-hand side converges, but the "Error" does not, the results must still be considered unphysical, even though it is finite. 

Note that the wake of a spherical particle in an unbounded fluid in Stokes flow, yields a velocity field $\textbf{u}_{f_p}$ proportional to  $\sim 1/|\textbf{r}|$. 
Thus, such a velocity field is a good example of a situation where: the continuous phase properties $f_{f_p}^1$ is unbounded, the first integral and the ``Error'' in \ref{eq:Batchelor2} diverge. 
In such cases, Batchelor used the famous renormalization method to circumvent these difficulties. 
Note that these problems of divergent integral could be guessed well in advance since the Taylor expansion of $\delta(\textbf{x} - \textbf{x}_i - \textbf{r})$ for a vector $\textbf{r}$ that is arbitrarily large is not convergent. 
Nevertheless, such manipulation is required to demonstrate Batchelor's original formula and its generalization given by \eqref{eq:Batchelor2}. 


Additionally, we believe that in the general case, where $f_f$ is given by \ref{eq:f_f_1_def}, it is impossible to obtain meaningfull results with the latter formula. 
Indeed, \ref{eq:f_f_1_def} requires the use of the Taylor expansion, $\delta(\textbf{x} - \textbf{x}_i - \textbf{r}) =\delta(\textbf{x} - \textbf{x}_i) - \textbf{r}\cdot \grad\delta(\textbf{x} - \textbf{x}_i)+ \ldots + \mathcal{O}(\textbf{r}^n/L^n)$. 
Additionally, in the integrals of \ref{eq:f_f_1_def}, \textbf{r} is evaluated from the particle center to an infinitely large distance from it.
Thus, it is evident that for any unbounded $f_{f_p}^1$, when $|\textbf{r}| \to \infty$, there is always an arbitrary integer $n$ for which, $\int f_{f_p}^1 \phi^1_f \textbf{r}^n d\textbf{r} \to \infty$.
Thus, if one considers a sufficiently high order moment in \ref{eq:f_f_1_def}, he will end up including a divergent integral. 
Following the same argument we can show that the  ``Error'' term included due to the Taylor expansion, proportional to $\mathcal{O}(f_{f_p}^1 r^n /L^n)$ might diverge as well since $r$ goes to infinity and $L$ stays constants. 
Consequently, in the inhomogeneous situations, and for unbounded functions $f_{f_p}^1$, we state that \ref{eq:f_f_1_def} is nonphysical as it produces divergent integral and an infinite ``Error'' as well. 
 
In conclusion, \ref{eq:Batchelor2}  is meaningful only for fields that respect the following conditions: 
(1) The influence of the particles on the field $f_f^0$ must be additive, this is the case when ${f_f^0}$ follows the Stokes equations; 
(2) The closures must be derived in a homogeneous flow, such that the higher moments in \ref{eq:f_f_1_def} cancel exactly. 
And (3), the integral over $\mathbb{R}^3$ of the term $|\textbf{r}| f_{f_p}^1$ must be finite, such that \ref{eq:Batchelor2} remains finite. 
Even if \ref{eq:Batchelor2} is not ideal, we will be using this relation to compute the continuous phase closures since for instance, this is the only tool that we have. 
Note that a new method will be presented in \ref{chap:pseudoturbulence} where we use \textit{The Nearest particle statistics} \citep{zhang2021ensemble} to compute these kinds of ensemble-averaged terms, but without the need for such approximations.


\section{The single-particle ensemble averaged problem}
\label{sec:the_disturbance_eq}

Now that we linked the ensemble-averaged closures and the conditional averaged quantities, it is time to derive these conditional averaged quantities. 
Specifically, we want to find $\textbf{u}^1$ and $p^1$ or $\textbf{u}^{1d}$ and $p^{1d}$. 
To that end, we follow \citep{zhang1994averaged} and derive the \textit{single-particle} conditioned Navier-Stokes equations. 

We first recall the \textit{single-fluid} formulation of the mass and momentum at the local scale, it yields
\begin{align}
    \div \textbf{u}^0 = 0 \\
    \pddt (\rho^0\textbf{u}^0)
    + \div (\rho^0\textbf{u}^0\textbf{u}^0 - \bm\sigma^0)
    &= \rho^0 \textbf{g}
    \label{eq:dt_local}
\end{align}
where we recall that $\rho^0 = \chi_d \rho^d + \chi_f \rho^f$, $\textbf{u}^0 = \chi_f \textbf{u}_f + \chi_d \textbf{u}_d$ and that $\bm\sigma^0 = \bm\sigma_d + \bm\sigma_f + \bm\sigma_\Gamma$, with $\bm\sigma_{f,d}$ Newtonian stresses and $\bm\sigma_\Gamma = \gamma ( \bm\delta - \textbf{nn})$ is the surface tension contribution. 
Note that the boundary conditions at the surface of the droplets are implicitly included in the \textit{single-fluid} formulation \eqref{eq:dt_local}, however it will be useful later to recall them here, 
\begin{align}
    \label{eq:dt_rho_I3}
    \textbf{u}_f^0 = \textbf{u}_d^0 = \textbf{u}_\Gamma^0, \\
    \Jump{\bm{\sigma}_k^0} 
    =
    -\gamma\textbf{n}(\div \textbf{n}). 
    \label{eq:dt_rho_I2}
\end{align}

To obtain an equation for the disturbance velocity fields $\textbf{u}^{1d}$ we first notice that this field can be defined by the operation, 
\begin{equation}
    \avg{(\delta_1 - P_1) \textbf{u}^0}
    =
    P_1 \textbf{u}^1
    - P_1 \textbf{u}
    = P_1 \textbf{u}^{1d}. 
    \label{eq:first_step_u0}
\end{equation}
Note that $\textbf{u}^0$ and \ref{eq:dt_local} are evaluated at the point \textbf{x} while the Dirac function, $\delta_1(\textbf{y},\textbf{w},t,\FF) = \sum_i^N \delta(\textbf{x}_i-\textbf{y})\delta(\textbf{u}_i - \textbf{w})$ express the condition of having a particle at \textbf{y} with velocity \textbf{w}, and is therefore independent of \textbf{x}. 
From \ref{eq:first_step_u0}, we deduce that the momentum conservation equation for $\textbf{u}^{1d}$ is obtained by multiplying \ref{eq:dt_local} by $\delta_1 - P_1$ and averaging over all configurations. 
However, since $\delta_1$ is a function of time $t$, this operation will require a conservation equation for $\delta_1$ and $P_1$ as well.  

Taking the partial time derivative of $\delta_1$ yields directly, 
\begin{equation}
    \pddt\delta_1 
    + \pddy\cdot(\textbf{w}\delta_1)
    + \pddw\cdot(\textbf{a}_i\delta_1)
    = 0 
    \label{eq:dt_delta_1}
\end{equation}
where $\textbf{a}_i[\FF,t] = \pddt \textbf{u}_i[\FF,t]$ is the acceleration of the particle $i$ in the configuration $\FF$. 
Averaging this equation yields an equation for  $P_1[\textbf{x},\textbf{w},t]$ it gives, 
\begin{equation}
    \pddt P_1
    + \pddy\cdot[\textbf{w} (\delta_1 - P_1)]
    + \pddw\cdot[\delta_1 \textbf{a}_i - \textbf{a}_p P_1]
    = 0.
    \label{eq:dt_P_1}
\end{equation}
These two equations must be regarded as conservation equations of the one point statistics: $P_1$,  along its phase space formed by $\textbf{y},\textbf{w},t$. 
Note that integrating \ref{eq:dt_P_1} over $\textbf{w}$ yields a conservation equation for the number density $n_p[\textbf{y},t]$, because of \ref{eq:Pw_normed}.  

\subsection{The single-particle ensemble-averaged Navier-Stokes equations}

Multiplying \ref{eq:dt_local} by $(\delta_1 - P_1)$ and using \ref{eq:dt_delta_1},  \ref{eq:dt_P_1} yields the general form of the \textit{single-particle} conditionally-averaged \textit{single-fluid} formulation of the Navier-Stokes equations, namely,  
\begin{align}
    \div \avg{(\delta_1 - P_1) \textbf{u}^0}
    = 0 
    \label{eq:conditional_eqs_mass}
    \\
    \pddt \avg{(\delta_1 - P_1)\rho^0 \textbf{u}^0}
    + \div \avg{(\rho^0 \textbf{u}^0 \textbf{u}^0 - \bm\sigma^0 )(\delta_1 - P_1)} \nonumber \\ 
    + \pddy\cdot \avg{(\delta_1 - P_1)\rho^0 \textbf{u}^0 \textbf{w}}
    + \pddw\cdot \avg{(\delta_1\textbf{a}_i - P_1\textbf{a}_p)\rho^0 \textbf{u}^0}
    = \avg{(\delta_1 - P_1 ) \rho^0 \textbf{g}}
    \label{eq:conditional_eqs}
\end{align}
We can observe that the only differences with \ref{eq:conditional_eqs} and \ref{eq:conditional_eqs_mass} and their local counterpart is the presence of the factor $(\delta_1 - P_1)$ in front of all the terms and the additional advecting terms on the left-hand side of \ref{eq:conditional_eqs}. 
For purpose of understanding it is useful to reformulate the terms present in \ref{eq:conditional_eqs} and \ref{eq:conditional_eqs_mass}, they read
\begin{align}
    \label{eq:examples1}
    \avg{(\delta_1 - P_1) \textbf{u}^0}
    &= P_1 \textbf{u}^{1d},\\ 
    \avg{(\delta_1 - P_1)\rho^0 \textbf{g}}
    &= P_1 \rho^{1d} \textbf{g}\\
    \avg{(\delta_1 - P_1)\rho^0 \textbf{u}^0}
    &= P_1 (\rho^1 \textbf{u}_m^1 -\rho \textbf{u}_m)
    = P_1 (\rho^{1d} \textbf{u}_m^{1d} +\rho^{1d} \textbf{u}_m+\rho \textbf{u}_m^{1d}),\\
    \avg{(\delta_1 - P_1) \bm\sigma^0} 
    &= 
    P_1 \bm\sigma^{1d} 
    \label{eq:examples}
\end{align}
Where we have introduced $\rho^{1d} = \rho^1 - \rho$, which is the mean density of the mixture at \textbf{x} averaged on every configuration where a particle is present at \textbf{y}, minus the density of the whole mixture $\rho$ at \textbf{x}.  
We recall that $\textbf{u}_m = \avg{\textbf{u}^0\rho^0} / \rho$ is the Favre average of the velocity, and therefore $\textbf{u}^1_m = \avg{\textbf{u}^0\rho^0\delta_1} / (\rho^1P_1)$ is the Favre conditional average of the velocity conditioned by the presence of a particle at \textbf{y}. 
This represents the likelihood of finding the density of the mixture at \textbf{x} given that  a particle is at \textbf{y} minus the ensemble averaged mixture density.
For example, for all $|\textbf{x}- \textbf{y}| <a$  we obtain $\rho^{1d} = \rho_d - \rho_d\phi_d - \rho_f \phi_f = \phi_f (\rho_d - \rho_f)$ which when multiplied by \textbf{g} is the buoyancy.  

Although we did not explicitly write all the terms of \ref{eq:conditional_eqs} and \ref{eq:conditional_eqs_mass} we can already notice some interesting features. 
First, using \ref{eq:examples1},  \ref{eq:conditional_eqs_mass} can be written, $P_1 \div \textbf{u}^{1d} =0$, indicating that the averaged disturbance field around the partcile is divergence free.  
In \ref{eq:examples} we can observe the presence of the $\textbf{u}^{1d}$ and $\textbf{u}_m$. 
This, implies that there is a coupling between the disturbance fields $\textbf{u}^{1d}_m$, which is the local fields describing the flow around a particle (at \textbf{y}) and the mean flow variables such as the mean mixture velocity $\textbf{u}_m$.
Particularly, we can see that the total advective flux: $\avg{(\delta_1 - P_1)\rho^0 \textbf{u}^0}$,contains the product $\phi_f^{1d} \textbf{u}_m$, meaning that the ensemble averaged velocity fields $\textbf{u}_m$ will induces an additional momentum fluxes in the momentum equation of the disturbance velocity. 
As we will see in the following section, \ref{eq:conditional_eqs} is just a multiphase flow generalization of the Navier-Stokes equations written in the reference frame of an isolated particle.
Such equations are given in \citep{maxey1983equation}, where we indeed observe that the mean background ``undisturbed'' fluid velocity, is present in the advective term of the momentum equation. 

\subsection{The single-particle ensemble-averaged boundary conditions}


\ref{eq:conditional_eqs} and \ref{eq:conditional_eqs_mass} are completed by the following boundaries conditions far from the particle, 
\begin{align}
    \lim_{|\textbf{x}-\textbf{y}|\to\infty} 
    \textbf{u}^{1d}[\textbf{x},\textbf{w},\textbf{y},t] 
    = 
    \lim_{|\textbf{x}-\textbf{y}|\to\infty} 
    \textbf{u}^{1}[\textbf{x},\textbf{w},\textbf{y},t] 
    - \textbf{u}[\textbf{y},t] 
    = 0 \\
    \lim_{|\textbf{x}-\textbf{y}|\to\infty} 
    \phi_d^{1d}[\textbf{x},\textbf{w},\textbf{y},t] 
    = 
    \lim_{|\textbf{x}-\textbf{y}|\to\infty} 
    \phi_d^{1}[\textbf{x},\textbf{w},\textbf{y},t] 
    - \phi_d[\textbf{y},t] 
    = 0 \\
    \lim_{|\textbf{x}-\textbf{y}|\to\infty} 
    p^{1d}[\textbf{x},\textbf{w},\textbf{y},t] 
    = 
    \lim_{|\textbf{x}-\textbf{y}|\to\infty} 
    p^{1}[\textbf{x},\textbf{w},\textbf{y},t] 
    - p[\textbf{x},t] 
    = 0 
    \label{eq:boundary_at_infinity}
\end{align}
which basically state that the particle at \textbf{y} has no influence on the conditioned averaged field which is evaluated at \textbf{x} when $|\textbf{x}-\textbf{y}|$ is large enough. 

Note that the mean fields in \ref{eq:boundary_at_infinity} are evaluated at \textbf{x} not at the particle center \textbf{y}.
However, in the momentum equation \eqref{eq:conditional_eqs} and the boundary conditions \ref{eq:boundary_at_infinity} it will be more practical to consider these fields as constants as a function of \textbf{x}, so that they can be considered as input to our problem. 
Thus, one may replace these ensembles averaged terms by their expression at \textbf{x} using a Taylor expansion. 
At first order this yields the following boundaries condition for the conditional fields, 
\begin{align}
    % \lim_{|\textbf{x}-\textbf{y}|\to\infty} 
    \textbf{u}[\textbf{x},t] 
    &\approx \textbf{u}[\textbf{y},t] 
    + \textbf{r}\cdot \grad\textbf{u}[\textbf{y},t] 
    \label{eq:boundary_at_infinity23}
    + \ldots\\
    % \lim_{|\textbf{x}-\textbf{y}|\to\infty} 
    \phi_d[\textbf{x},t] 
    &\approx \phi_d[\textbf{y},t] 
    \label{eq:boundary_at_infinity22}
    + \textbf{r}\cdot \grad \phi_d[\textbf{y},t]  
    + \ldots\\
    % \lim_{|\textbf{x}-\textbf{y}|\to\infty} 
    p[\textbf{x},t] 
    &\approx p[\textbf{y},t] 
    + \textbf{r}\cdot  \grad p[\textbf{y},t] 
    + \ldots
    \label{eq:boundary_at_infinity2}
\end{align}
where $\textbf{r} = \textbf{x} - \textbf{y}$. 
Using \ref{eq:boundary_at_infinity2} in the boundary conditions \eqref{eq:boundary_at_infinity} we remark that they yield similar boundaries than the ones used in the problem of a moving particle immersed in an unbounded linear flow \citep{jackson1997locally,zhang1997momentum}. 
Nontheless,  \ref{eq:boundary_at_infinity} yield more general as they include a possible non-zero value for $\phi_d$. 
Thus, in non-dilute limit the particle-test cannot be considered as isolated, but immersed in an equivalent medium which has a particle volume fraction $\phi_d$ at infinity. 
Additionally, in \ref{eq:conditional_eqs}, $\phi_d[\textbf{x},t]$ appears within the expression of the mean density $\rho[\textbf{y},t]$. 
Thus, as witnessed by \ref{eq:boundary_at_infinity22}, we could also include the influence of $\phi_d$ and the mean gradient of particles concentration, $\grad \phi_d$, as input of our problem described by \eqref{eq:conditional_eqs} and \ref{eq:boundary_at_infinity}.


Additionally, the \textit{single-particle} conditionally averaged fields are all ensemble averaged on configuration where a spherical particle of radius $a$ is present at \textbf{y} with velocity \textbf{w}.
Therefore, the conditional velocity fields $\textbf{u}^1$ evaluated at any point on the surface of the particle, is given by 
\begin{equation}
    \textbf{n}\cdot \textbf{u}^1= \textbf{n} \cdot \textbf{w} \;\;\;\forall |\textbf{x} - \textbf{y}| = a. 
\end{equation}
Subtracting both side by $\textbf{u}[\textbf{x},t]$ and using \ref{eq:boundary_at_infinity2} yields
\begin{align}
    \textbf{n}\cdot\textbf{u}^{1d}
    = \textbf{n}\cdot(
    \textbf{w} 
    - \textbf{u}
    -\textbf{r} \cdot \grad\textbf{u} 
    + \ldots
    )\\
\end{align}
We recognize the classical velocity field boundary condition of a spherical droplet immersed in an arbitrary linear flow \citep{nadim}. 

The averaged stress and velocity jump conditions at the surface of the test-particle are obtained by conditionally ensemble averaging \ref{eq:dt_rho_I2} and \ref{eq:dt_rho_I3} and evaluating the resulting expressions at the points $|\textbf{x}-\textbf{y}| =a$. 
The conditional ensemble averaged stress at $|\textbf{x}-\textbf{y}| < a$ is given by \eqref{eq:stress}, and at the surface exterior of the test-particle by \ref{eq:sigma_f_surf}. 
Thus, applying the operation $\avg{\delta_1 \ldots}$ on \ref{eq:dt_rho_I2} and \ref{eq:dt_rho_I3}, and evaluating the expression at $|\textbf{x}-\textbf{y}| =a$, givesdirectly:  
\begin{align}
    \label{eq:dt_rho_I3}
    \textbf{u}_f^1 = \textbf{u}_d^1\\
    \Jump{\bm{\Sigma}_k^1} 
    =
    - \gamma\textbf{n}(\div \textbf{n}). 
    \label{eq:dt_rho_I2}
\end{align}
where we recall that $\bm{\Sigma}_k^1 = -p_k^1\bm\delta + \mu_k (\grad \textbf{u}_k^1 + (\grad \textbf{u}_k^1)^\dagger)$. 
Note that inside and at the surface exterior of the particle $\phi_d^1 = 1$ and $\phi_d^1 =0$, respectively, thus one may replace $p_k^1$ and $\textbf{u}_k^1$ by $p^1$ and $\textbf{u}^1$. 

At this stage the physical significance of the terms in \ref{eq:conditional_eqs} is not explicit. 
Nevertheless, we state that \ref{eq:conditional_eqs} corresponds to an equation for the fluid phase velocity disturbance fields, $\textbf{u}_f^{1d}$, and that \ref{eq:conditional_eqs_mass} is an equation for the disturbance fields of the fluid phase volume fraction, $\phi_f^{1d}$. 
In all generality these equations needs to be completed by a set of equations for the dispersed phases, which will be coupled with \ref{eq:conditional_eqs} and \ref{eq:conditional_eqs_mass} through the exchange term $\avg{(\delta_1 - P_1)\delta_\Gamma \bm\sigma_f \cdot \textbf{n}_d}$. 
However, as we see now, the restricted assumption made in this work allows us to neglect completely the particle phases conditionally-averaged equations. 

\tb{Does $\textbf{u}'$ depends on relative motion as suggest the BC ?? ? ?? }

\subsection{The finite particle Reynolds number dilute regime}
In the first place we consider the situation where the particle volume fraction is relatively small. 
More precisely we neglect all terms proportional to $\sim \phi_d^2$. 
To better understand what this implies notice that inside the volume of the particle, that is when $|\textbf{x}-\textbf{y}| < a$ we have $\phi_f^1 = 0$ since only the dispersed phase is presents in that area. 
However, when $|\textbf{y} - \textbf{x}| >a$ we are in the mixture phase and $\phi_f^1 \approx \phi_f$.
Therefore, neglecting the $\mathcal{O}(\phi_d^2)$ terms  implies assuming that $P_1 \phi_f \approx P_1$ and that $P_1^2 = 0$. 
Notice that $\phi_f^{1d} = -\phi_d^1 + \phi_d$ which means that $\phi_f^{1d} \sim \phi_d$ when $|\textbf{y} -\textbf{y}| > a$ therefore at  $\mathcal{O}(\phi_d^2)$ we have also $\phi_f^{1d} P_1= 0$. 
Additionally, since $\avg{\delta_\Gamma}\sim \phi_d$, we have $\avg{\delta_\Gamma(\delta_1-P_1)}\sim \phi_d^2$ if we exclude the point $\textbf{y}\in \{|\textbf{y}-\textbf{x}| = a\}$. 
Applying these considerations, we may rewrite the conditionally-averaged terms in \ref{eq:conditional_eqs} and \ref{eq:conditional_eqs_mass}  $\forall \textbf{y}\in \{|\textbf{y}-\textbf{x}| = a\}$ as, 
\begin{align*}
    \avg{(\delta_1 - P_1)\rho_f\chi_f}
    &=0 \\ 
    \avg{(\delta_1 - P_1)\rho_f\chi_f\textbf{u}^0_f}
    &= P_1 \rho_f \textbf{u}_f^{1d},\\
    % + \phi_f^{1d} \bm\sigma_f
    % + \phi_f^{1d} \bm\sigma_f^{1d} ]. 
    \avg{(\delta_1 - P_1)\rho^0\textbf{u}^0_f\textbf{u}^0_f}
    &=
    P_1\rho_f[
        \textbf{u}^{1d}_f\textbf{u}_f
        + \textbf{u}_f\textbf{u}_f^{1d}
        + \textbf{u}^{1d}_f\textbf{u}_f^{1d}
    ]
    + \avg{\delta_1\rho_f\chi_f\textbf{u}_f''\textbf{u}_f''}
    - P_1 \avg{\rho_f\chi_f\textbf{u}_f'\textbf{u}_f'}\\
    \avg{\chi_f \bm\sigma^0_f(\delta_1 - P_1)} 
    &= 
    P_1 \bm\sigma^{1d}_f 
    = 
    -P_1  p_f^{1d} 
    +P_1 \mu_f [\pddy \textbf{u}_f^{1d}+(\pddy \textbf{u}_f^{1d})^\dagger ]
\end{align*}

Using these approximations we can re-write \ref{eq:conditional_eqs} still for $|\textbf{y}-\textbf{x}| > a$, this reads,  
\begin{align}
    P_1 \pddy \cdot \textbf{u}^{1d}_f &= 0 \\
    \pddt (P_1 \rho_f \textbf{u}_f^{1d})
    + P_1 \pddy\cdot (
    \rho_f \textbf{u}^{1d}_f\textbf{u}_f^{1d} 
    + \textbf{u}_f\textbf{u}_f^{1d}
    + \textbf{u}^{1d}_f\textbf{u}_f
    + \bm\sigma^\text{Re}_f)\nonumber\\
    + \pddx\cdot (P_1 \textbf{u}_f^{1d}\rho_f\textbf{w}) 
    + \pddw\cdot \avg{(\delta_1 \textbf{a}_i- P_1 \textbf{a}_p)\rho^0\textbf{u}^0 }
    &= P_1 (
        \mu_f \pddy^2 \textbf{u}^{1d}_f  
        - \pddy p_f^{1d} 
    )
\end{align}
with, 
\begin{equation*}
    P_1\bm\sigma^{Re}_f
    = 
    % + \textbf{u}_f\textbf{u}_f^{1d}
    % + \textbf{u}^{1d}_f\textbf{u}_f^{1d}
    + \avg{\delta_1\rho_f\chi_f\textbf{u}_f''\textbf{u}_f''}
    - P_1 \avg{\rho_f\chi_f\textbf{u}_f'\textbf{u}_f'}
\end{equation*}
This is the conservative form of the dilute conditioned averaged Navier stokes equations. 
For reason that will be clear latter we would like to write the conservative form of this equation.
To do so we first notice that the averaged volume conservation equation of $\phi_f$ multiplied by $P_1$ gives, 
\begin{equation*}
    P_1\pddy \cdot \textbf{u}_f = 0.
\end{equation*}
Therefore, at $\mathcal{O}(\phi_d)$ the field $P_1\textbf{u}_f$ is divergence free, it does not imply that $\textbf{u}_f$. 
Additionally, in the dilute regime is it reasonable to assume that $\pddt P_1$ and $\pddx P_1$ are completely negligible compared to the other contributions. 
Alternatively, we may assume a complete homogeneous system such that $P_1(\textbf{x},\textbf{w},t) = P_1(\textbf{w})$. 
Under these assumptions one can write based on physical arguments that $\pddx \textbf{u}_f^{1d}[\textbf{y},\textbf{x},\textbf{w},t] = - \pddy \textbf{u}^{1d}[\textbf{y},\textbf{x},\textbf{w},t]$.
In that case, we reach the final form of the conditionally-averaged Navier-Stokes equations in the dilute regime  namely, 
\begin{align}
    \pddy \cdot \textbf{u}^{1d}_f &= 0 \\
    \rho_f \left[
        \pddt \textbf{u}_f^{1d}
        +  \textbf{u}^{1d}_f\cdot \pddy\textbf{u}_f^{1d} 
        +  \textbf{u}^{1d}_f\cdot \pddy\textbf{u}_f 
        +  (\textbf{u}_f - \textbf{w})\cdot \pddy\textbf{u}_f^{1d}
    \right]
    + \pddy \cdot \bm\sigma^\text{Re}_f
    &=
        \mu_f \pddy^2 \textbf{u}^{1d}_f  
        - \pddy p_f^{1d} 
    % + \pddw\cdot \avg{(\delta_1 - P_1)\rho^0\textbf{u}^0\textbf{a}_i}
    \label{eq:conditional_avg_eq_final}
\end{align}
with, 
\begin{equation*}
    \bm\sigma^{Re}_f
    =
    % + \textbf{u}_f\textbf{u}_f^{1d}
    % + \textbf{u}^{1d}_f\textbf{u}_f^{1d}
    + \frac{1}{P_1}\avg{\delta_1\rho_f\chi_f\textbf{u}_f''\textbf{u}_f''}
    - \avg{\rho_f\chi_f\textbf{u}_f'\textbf{u}_f'}
\end{equation*}
% and the boundary condition, 
% \begin{align*}
%     \lim_{|\textbf{y}-\textbf{x}|\to\infty} \textbf{u}_f^{1d}[\textbf{y}|\textbf{w},\textbf{x},t] = 0 \\
%     \textbf{u}_f^{1d} = \textbf{w} - \textbf{u}_f[\textbf{y},t]
%     = 
%     \textbf{w} 
%     - \textbf{u}_f|_{\textbf{y}=\textbf{x}}
%     -\textbf{r} \cdot \pddy\textbf{u}_f|_{\textbf{y}=\textbf{x}}
%     - \ldots
%     \;\;\; \forall \textbf{y} \in \{ |\textbf{y}-\textbf{x}| = a \}
%      \\
% \end{align*}
Notice that these equations are quite similar to the disturbance field equation produced by a translating particle (see for exemple equation (15) of \citet{maxey1983equation}). 
One might immediately recognize the equation (15) of \citep{maxey1983equation} which correspond to the equation for the disturbance field of an isolated translating droplets in an inertial frame. 
The only difference between equation (15) of \citep{maxey1983equation} and \ref{eq:conditional_avg_eq_final} is that we introduced the presence of an equivalent stress $\bm\sigma_f^{eq}$ related to local velocity fluctuation. 

\subsection{The stokes equation}

Noticing that the right-hand side of \ref{eq:conditional_avg_eq_final} is negligible in the Stokes flow regime. 
Thus, we finally obtain the well known system of equations describing the disturbance fields induced by an isolated particle translating in an arbitrary linear flow, that reads 
\begin{align}
    \pddy \cdot \textbf{u}^{1d}_f &= 0,  \\
    % \rho_f \left[
    %     \pddt \textbf{u}_f^{1d}
    %     +  \textbf{u}^{1d}_f\cdot \pddy\textbf{u}_f^{1d} 
    %     +  \textbf{u}^{1d}_f\cdot \pddy\textbf{u}_f 
    %     +  (\textbf{u}_f - \textbf{w})\cdot \pddy\textbf{u}_f^{1d}
    % \right]
    % + \pddy \cdot \bm\sigma^\text{Re}_f
    - \pddy p_f^{1d} 
    + \mu_f \pddy^2 \textbf{u}^{1d}_f  
    &= 0, 
    % + \pddw\cdot \avg{(\delta_1 - P_1)\rho^0\textbf{u}^0\textbf{a}_i}
    \label{eq:conditional_avg_eq_final_stokes}
\end{align}
with the boundary conditions, 
\begin{align*}
    \textbf{n}\cdot\textbf{u}^{1d}_f
    % = \textbf{n}\cdot\{\textbf{w} - \textbf{u}_f[\textbf{y},t]\}
    &= \textbf{n}\cdot\{
    \textbf{w} 
    - \textbf{u}_f[\textbf{x},t]
    -\textbf{r} \cdot \pddy\textbf{u}_f[\textbf{x},t] 
    % + \mathcal{O}(|\textbf{r}|^2)
    -\frac{1}{2}\textbf{rr} \cdot \pddy\pddy\textbf{u}_f[\textbf{x},t] + \ldots
    \},\\
    % \;\;\; \forall \textbf{y} \in \{ |\textbf{y}-\textbf{x}| = a \}. \\
    \textbf{u}^{1}_f
    &=\textbf{u}^{1}_d,\\
    [\bm\sigma^{1}_f - \bm\sigma^{1}_d ]\cdot\bm\delta_{||}
    &= 0, 
    % \;\;\; \forall \textbf{y} \in \{ |\textbf{y}-\textbf{x}| = a \}. 
\end{align*}
at on the surface of the particle, and \ref{eq:boundary_at_infinity}, at far distance from the particle. 
\tb{

\subsection{The finite volume fraction with no inertial effects}

Now we would like to study the effect of finite volume fraction on the conditionally-averaged equations. 
For purpose of simplicity we neglect all inertial terms as well as the acceleration of the particle. 
When considering the effect of non-negligible volume fraction it is easier to deal with the \textit{single-fluid} formulation of the equations. 
In this situation it can be shown that, 
\begin{align}
    P_1 \pddy \cdot \textbf{u}^{1d}
    % + \pddx\cdot (P_1 \rho_f \phi_f^{1d} \textbf{w})
    % + \pddw\cdot \avg{(\delta_1\textbf{a}_i - P_1\textbf{a}_p)\rho_f \chi_f }
    = 0 
    \label{eq:single_fluid_conditional_eqs_mass}
    \\
    % \pddt \avg{(\delta_1 - P_1)\rho_f \chi_f \textbf{u}^0_f}
     \pddy\cdot \avg{\bm\sigma^0 (\delta_1 - P_1)}
    % + \pddx\cdot \avg{(\delta_1 - P_1)\rho_f \chi_f \textbf{u}^0_f\textbf{w}}
    % + \pddw\cdot \avg{(\delta_1\textbf{a}_i - P_1\textbf{a}_p)\rho_f \chi_f \textbf{u}^0_f}
    = 0 
    \label{eq:single_fluid_conditional_eqs}
\end{align}
Notice that in this situation the disturbance velocity field $\textbf{u}_f^{1d}$ is not incompressible and follows a non-trivial transport equation.
In opposition to the bulk velocity fluid $\textbf{u}^{1d}$ which is divergence free according to \ref{eq:single_fluid_conditional_eqs_mass}.  

The  conditionally-averaged stress $\avg{\chi_f \bm\sigma^0_f (\delta_1 - P_1)}$ can be further written as, 
\begin{align*}
    \avg{\bm\sigma^0 (\delta_1 - P_1)}
    &= \avg{\chi_f \bm\sigma^0_f (\delta_1 - P_1)}
    + \avg{\chi_\Gamma \bm\sigma^0_\Gamma (\delta_1 - P_1)}
    + \avg{\chi_d \bm\sigma^0_d (\delta_1 - P_1)}\\
    &= 
    - P_1 [
        \phi_f p_f^{1d} 
        +\phi_f^{1d} p_f^{1d}
        +\phi_f^{1d} p_f 
    ]\bm\delta
    + P_1 \mu_f [\pddy \textbf{u}^{1d}+(\pddy \textbf{u}^{1d})^\dagger] \\
    &+ \avg{[\chi_d (\bm\sigma_d^0 - 2 \mu_f \textbf{e}^0_d ) + \chi_\Gamma \bm\sigma_\Gamma ]  (\delta_1 - P_1)}
\end{align*}
Additionally, the exchange term can be written such that,
\begin{multline*}
    \avg{[\chi_d (\bm\sigma_d^0) + \chi_\Gamma \bm\sigma_\Gamma ]  (\delta_1 - P_1)}\\
    \approx \avg{(\delta_1 - P_1)\delta(\textbf{x}_j - \textbf{y})\intS[j]{\textbf{r}\bm\sigma_f^0\cdot \textbf{n}}}
    -\pddy \cdot \avg{(\delta_1 - P_1)\delta(\textbf{x}_j - \textbf{y})\intS[j]{ \textbf{rr} \bm\sigma_f^0\cdot \textbf{n}}}
    + \ldots
\end{multline*}
where the summation over all the $j$ is implicit. 
Theoretically if we consider the point sufficiently far from $\textbf{x}$ we have $j\neq i$.
Thus it is the stresslet terms. 

It is clear that the above quantities shear the boundary condition, 
\begin{align*}
    \lim_{|\textbf{y}-\textbf{x}|\to\infty} 
    \textbf{u}^{1d}
    % [\textbf{y},\textbf{w},\textbf{x},t] 
    &= 
    \lim_{|\textbf{y}-\textbf{x}|\to\infty} 
    \textbf{u}^{1}
    % [\textbf{y},\textbf{w},\textbf{x},t] 
    - \textbf{u}
    = 0 
    \approx 
    \textbf{u}^{1}
    % [\textbf{y},\textbf{w},\textbf{x},t] 
    - \textbf{u} 
    -\textbf{r}\cdot\grad \textbf{u} 
    \\
    \lim_{|\textbf{y}-\textbf{x}|\to\infty} 
    \phi_f^{1d}
    % [\textbf{y},\textbf{w},\textbf{x},t] 
    &= 
    \lim_{|\textbf{y}-\textbf{x}|\to\infty} 
    \phi_f^{1}
    % [\textbf{y},\textbf{w},\textbf{x},t] 
    - \phi_f
    = 0
    \approx \phi_f^{1}
    % [\textbf{y},\textbf{w},\textbf{x},t] 
    - \phi_f
    - \textbf{r} \cdot \grad \phi_f
     \\
    \lim_{|\textbf{y}-\textbf{x}|\to\infty} 
    p^{1d}_f
    % [\textbf{y},\textbf{w},\textbf{x},t] 
    &= 
    \lim_{|\textbf{y}-\textbf{x}|\to\infty} 
    p^{1}_f
    % [\textbf{y},\textbf{w},\textbf{x},t] 
    - p_f
    = 0 \approx
    p^{1}_f
    % [\textbf{y},\textbf{w},\textbf{x},t] 
    - p_f 
    - \textbf{r}\cdot \grad p_f
    \label{eq:boundary_at_infinity}
\end{align*}
Where we have kept only the first-order  fields of the disturbance fields. 

The momentum equations is therefore, 
\begin{equation*}
    - P_1 \pddy(\phi_f p_f^{1d} + \phi_f^{1d} p_f^{1d} + \phi_f^{1d} p_f^{1d})
    + \mu_f P_1 \grad^2  \textbf{u}^{1d}
    % + \pddw\cdot \avg{(\delta_1\textbf{a}_i - P_1\textbf{a}_p)\rho_f \chi_f \textbf{u}^0_f}
    = - \avg{(\delta_1 - P_1)\delta_\Gamma \bm\sigma_f \cdot \textbf{n}_d}
\end{equation*}


}
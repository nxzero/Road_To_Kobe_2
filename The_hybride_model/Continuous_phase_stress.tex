
\subsection{The fluid phase equivalent stress}
\tb{This section can be shortened a lotsss }
% In this section we focus on the formulation of the averaged fluid phase equivalent stress tensor $\bm{\sigma}_f^\text{Re}$. 
For instance the stress appearing on the left hands side of the fluid phase momentum balance is of the form of \ref{eq:sigma_eq_def}. 
However, it is more convenient to express the equivalent stress as a Newtonian stress, plus a contribution arising due to the presence of the particles. 
In this objective we first reformulate $\bm{\sigma}_f$ by considering that the carrier fluid is Newtonian, therefore $\phi_f \bm{\sigma}_f = - \phi_f p_f + 2 \phi_f \mu_f \textbf{e}_f$ where $2 \phi_f \bm{e}_f = \avg{\chi_f  (\grad \textbf{u}_f^0 + (\grad \textbf{u}_f^0)^T)}$. 
Additionally, we state that the fluid strain is equal to the bulk strain $2\textbf{e} = \grad \textbf{u}+ (\grad \textbf{u})^T$, minus the particle averaged strain, i.e. $\phi_f \mu_f \textbf{e}_f = \mu_f\textbf{e} - \mu_f \phi_d \textbf{e}_d$. 
Under this form we clearly remark that $\phi_d \textbf{e}_d = 0$ for solid particles, recovering the expression $\phi_f \textbf{e}_f = \textbf{e}$ of \citet{jackson1997locally}. 
Upon developing $\phi_d \textbf{e}_d$ multipolar series, the equivalent stress of the fluid phase can be reformulated as, 
\begin{multline}
    \bm{\sigma}^\text{eq}_f = 
    \phi_f p_f \bm\delta 
    - 2\mu_f \textbf{e} 
    + \rho_f\avg{\chi_f\textbf{u}_f'\textbf{u}_f'} 
    + 2 \mu_f \pOavg{\textbf{e}_d^0}
    - \pSavg{\textbf{r}\bm{\sigma}_f^0\cdot \textbf{n}_d}
    \\
    + \div \left[
        \frac{1}{2} \pSavg{\textbf{rr}\bm{\sigma}_f^0\cdot \textbf{n}_d}
        - 2 \mu_f\pOavg{ \textbf{re}_d^0 }
        + \ldots
    \right]
    \label{eq:sigma_eq_0}
\end{multline} 
It is worth noting that $\textbf{e} = \grad \textbf{u} + (\grad \textbf{u})^\dagger$. 
Nevertheless, the bulk velocity \textbf{u} is not part of our unknown instead we solve for $\textbf{u}_f$, $\textbf{u}_p$, $\textbf{P}_p$ and eventually the higher moments. 
Therefore, in all rigor we must write 
\begin{equation}
    \textbf{e}
    = 
    \grad \textbf{U} + (\grad \textbf{U})^\dagger
    - \grad (\div (n_p \textbf{P}_p))
    - (\grad \div (n_p \textbf{P}_p))^\dagger
    + \ldots
    \label{eq:rate_of_strain}
\end{equation}
where $\textbf{U} = \phi_f \textbf{u}_f + n_p v_p \textbf{u}_p$ is equivalent to the bulk velocity \textbf{u} uniquely in an homogeneous medium. 

The fluid phase averaged stress is therefore composed of : 
(1) the pseudo turbulent contribution $\rho_f\avg{\chi_f  \textbf{u}_f' \textbf{u}_f'}$ which can be decomposed in an isotropic part $2 k_f = \avg{\chi_f \rho_f \textbf{u}_f' \textbf{u}_f'}:\bm\delta$ that contribute to the effective pressure, and a deviatoric part defined as $\avg{\chi_f \rho_f \textbf{u}_f' \textbf{u}_f'} - 2 k_f\bm\delta$. 
(2) the shear stress $2\mu_f \textbf{e}$ of the fluid phase \ref{eq:rate_of_strain}. 
(3) the particles internal shear $2\mu_f \pOavg{\textbf{e}_d^0}$
(4) the particle first moment of the hydrodynamic forces $\pSavg{\textbf{r}\bm{\sigma}_f^0\cdot \textbf{n}_d}$. 
(5) and the higher order moments of forces and internal shear.  

At this point if one want to write the fluid phase averaged stress as an equivalent Newtonian stress with effective pressure $p^{eff}$ and effective viscosity $\mu^{eff}$ he needs to express each of the closure terms mentioned above as a function of isotropic tensor which will contribute to the effective pressure, or as a linear function of  $\textbf{e}$ which will contribute to the effective viscosity. 
Note that this is not always possible, indeed, according to \ref{eq:second_mom} the second order moment of the hydrodynamic stress is a function of the relative velocity and not of the mean shear rate. 
Thus, in addition to the Newtonian behavior of the averaged fluid one must predict an effective viscosity dependent on the gradient of the viscosity. 

Once again it is useful to consider the stokes flow regime to provide a closed form of the fluid phese stresses. 
To that end notice first that the internal shear rate inside the particles, for isolated spherical droplets in an arbitrary linear flow  is written, 
\begin{align}
    \pOavg{\textbf{e}_d^0}
    = 
    \phi_d 
    \textbf{E}_f
    \frac{3}{5}\frac{1}{\lambda+1}
    \\
    \pOavg{\mu_f \textbf{e}_d^0\textbf{r} }
    = 
    - \frac{\phi_d\mu_f}{10(\lambda+1)}
    \left[
        (\bm\delta \textbf{u}_{fp})_{ijk}
        - \frac{3}{2}
        \left[
            (\bm\delta \textbf{u}_{fp})_{kij}
            + (\bm\delta \textbf{u}_{fp})_{jki}
        \right]
    \right]
    \label{eq:closur_e}
\end{align}
Notice that the second moment of momentum is present under the $\partial_k\partial_l$ operator in the moment of momentum equation, meaning that the skew-symmetic part of $\pavg{\intS{(\bm{\sigma}_f^0 \cdot \textbf{n}_d)_ir_kr_l}} $ and $\pSavg{{\mu(\textbf{e}_d^0)_{ik} r_l}}$ vanish in the momentum equation. 
Therefore, the second moment of surface traction force might be written, 
\begin{align*}
    \pavg{\intS{(\bm{\sigma}_f^0 \cdot \textbf{n}_d)_ir_k}} -
    2\pSavg{{\mu(\textbf{e}_d^0)_{ik}}} 
    = 
    \phi_d p_f\bm\delta
    - \frac{5\lambda +2}{\lambda +1}
    \textbf{e}_f \phi \mu_f
    \\
    \frac{1}{2}\pavg{\intS{(\bm{\sigma}_f^0 \cdot \textbf{n}_d)_ir_kr_l}} -
    2\pSavg{{\mu(\textbf{e}_d^0)_{ik} r_l}} 
    = 
    \frac{\mu_f\phi_d}{2(\lambda +1) }
    \left[
        \frac{3\lambda}{2} 
        u_{fp,i}\delta_{kl}
        +  u_{fp,l}\delta_{ki}
    \right]
\end{align*}
where we neglected the terms related to the mean fluid phase stress since it makes a contribution of $\mathcal{\phi^2}$\citet{jackson1997locally}.
Considering this relation together with \ref{eq:sigma_eq_0}, \ref{eq:closur_e}, \ref{eq:second_mom} and \ref{eq:first_mom} we can re-write the effective stress of the suspension as, 
\begin{align*}
    \bm{\sigma}^\text{eq}_{f,ik} =
    + \rho_f\avg{\chi_f\textbf{u}_f'\textbf{u}_f'}_{ik} 
    + p_f \bm\delta
    - 2 \mu_f \textbf{e}\left[
        1
        +\frac{\phi_d}{2}\left(
            \frac{5\lambda +2}{\lambda +1}
        \right)
    \right]
    + 
    \frac{\mu_f 3\lambda}{4(\lambda +1) }
    \left[
        \grad (\phi_d\textbf{u}_{fp,i})
        +  
        \frac{2}{3\lambda} 
        [\div (\phi_d\textbf{u}_{fp,l})]\bm\delta_{ki}
    \right]
\end{align*} 
This is in complete agreement with \citet[Appendix A]{zhang1997momentum}. 
In order to highlight that the stress tensor is symmetric we add and remove the term $ \grad \textbf{u}_{fp,i}^\dagger$ and notice that only the symmetric part in the indices $kl$ remain under the application of the double gradient to this expression and show that it gives, 
\begin{align}
    \bm{\sigma}^\text{eq}_{f,ik} =
    + \rho_f\avg{\chi_f\textbf{u}_f'\textbf{u}_f'}_{ik} 
    + p_f \bm\delta
    - 2\mu^\text{eff} \textbf{e}\\
    + 
    \mu_\text{U}^\text{eff}
    \left[
        \partial_k   (\phi_d\textbf{u}_{fp,i})
        + \partial_i (\phi_d\textbf{u}_{fp,k})
        + \frac{2-3\lambda}{3\lambda}  [\div (\phi_d\textbf{u}_{fp,l})]\bm\delta_{ki}
    \right]
    \label{eq:fluid_phase_stress}
\end{align} 
The equivalent viscosity of the fluid are given by 
\begin{align*}
    \mu^\text{eff}_\text{e} = \mu_f \left[
        1
        +\frac{\phi_d}{2}\left(
            \frac{5\lambda +2}{\lambda +1}
        \right)
    \right]\\
    \mu^\text{eff}_\text{U}
    = \mu_f\frac{ 3\lambda}{4(\lambda +1) }
\end{align*}
One can notice that $\mu^\text{eff}_\text{U}$ is the same as the faxen force coefficient. 
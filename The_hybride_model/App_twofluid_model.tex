\section{The two-fluid model}
\label{ap:two-fluid_model}

% \subsubsection{Volume equation}
Using the generic formulation \ref{eq:avg_dt_chi_f} and the local expression of the mass, momentum and total energy equation, i.e. : \ref{eq:dt_rho},\ref{eq:dt_rhou_k} and \ref{eq:dt_rhoE_k} we easily find the averaged form of the mass, momentum and total energy equation.
They read, 
\begin{align}
    \label{eq:dt_avg_rho}
    \pddt (\phi_k \rho_k)  
    + \div (
        \phi_k \rho_k\textbf{u}_k
    )
    &= 
    0\\
    \label{eq:dt_avg_rhou_k}
    \pddt (\phi_k \rho_k\textbf{u}_k)  
    + \div (
        \phi_k \rho_k\textbf{u}_k\textbf{u}_k
        + \bm{\sigma}_k^\text{eq}
    )
    &= 
    \phi_k \rho_k \textbf{g} 
    +  \avg{\delta_I \bm{\sigma}_k^0 \cdot \textbf{n}_k}\\
    \label{eq:dt_avg_rhoE_k}
    \pddt (\phi_k\rho_kE_k)  
    + \div (
        \phi_k\rho_kE_k\textbf{u}_k
        + \bm{q}_k^\text{eq}
        + \textbf{u}_k \cdot \bm{\sigma}_k^\text{eq}
        % - \textbf{u}_k^0 \cdot \bm{\sigma}_k^0 
        % + \textbf{q}_k^0
        )
    &= 
    \phi_k \rho_k\textbf{u}_k \cdot \textbf{g} 
    + \avg{\delta_I (\textbf{u}_k^0 \cdot \bm{\sigma}_k^0 - \textbf{q}_k^0)\cdot \textbf{n}_k}
\end{align} 
with the equivalent stress and heat flux defined as, 
\begin{align*}
    &\bm{\sigma}_k^\text{eq}
    = 
     \rho_k\avg{\chi_k \textbf{u}_k'\textbf{u}_k'}
      - \phi_k \bm{\sigma}_k,%- n_p \textbf{M}_p
    &\textbf{q}_k^\text{eq}
    =\textbf{q}_k^\text{e} +\textbf{q}_k^\text{k},  \\
    &\textbf{q}_k^\text{e}
    = \rho_k \avg{\chi_k\textbf{u}_k' e_k'} 
    + \phi_k\textbf{q}_k,
    &\textbf{q}_k^\text{k}
    = \rho_k \avg{\chi_k \textbf{u}_k' k_k} 
    - \avg{\chi_k \textbf{u}_k' \cdot \bm{\sigma}_k^0}.
\end{align*}
% The main differences between these equations and their microscale counterpart, is that : (1) we have introduced a pre factor $\phi_k$ in front of most of the terms
% (2) an additional stress is present, that is the covariance between the quantity to be conserved and the velocity. 
% (3) A new source term appear on the RHS accounting for the exchange across the phases. 

The terms $\avg{\chi_k \textbf{u}_k'\textbf{u}_k'}$ will be referred as the Reynolds stress or pseudo turbulent stress. 
It has a fundamental importance in the multiphase flow problem and as we see now its trace is related to the mean kinetic energy. 
Indeed, the phase averaged total energy can be further decompose into three energy components, that is,  
\begin{align}
    E_k = e_k + k_k + u_k^2/2
    \label{eq:E_def}
\end{align}
where $k_k$ is the pseudo-turbulent kinetic energy defined as, $\phi_k k_k = \frac{1}{2}\avg{\chi_k \textbf{u}_k'\cdot \textbf{u}_k'}$. 
Each of the components of the total energy represent the averaged agitation at different scales. 
From molecular agitation which quantified by $e_k$, to the macroscopic scale agitation or kinematic energy $u_k^2$, and in between we find the pseudo turbulent energy $k_k$ which represents the local scale velocity fluctuation. 
To fully describe the averaged total energy one must add at least a supplementary equation either for $k_k$ or $e_k$ assuming \ref{eq:dt_avg_rhoE_k} is solved. 
Using \ref{eq:dt_avg_rhou_k} dotted with $\textbf{u}_k$ yields an equation for the mean kinetic energy. 
Averaging \ref{eq:dt_rhoe_k} yields un equation for $e_k$.  
Then, subtracting these two equations to \ref{eq:dt_avg_rhoE_k} gives us an equation for $k_p$. 
The kinetic energy, pseudo turbulent and internal averaged equations read as, 
\begin{align}
    \label{eq:dt_avg_uk2}
    &\pddt (\phi_k \rho_ku_k^2/2)  
    + \div (
        \phi_k \rho_k\textbf{u}_ku_k^2/2
        + \textbf{u}_k \cdot \bm{\sigma}_k^\text{eq}
    )
    = 
    \bm{\sigma}_k^\text{eq} : \grad \textbf{u}_k
    + \phi_k \rho_k \textbf{u}_k\cdot \textbf{g} 
    +  \textbf{u}_k\cdot \avg{\delta_I \bm{\sigma}_k^0 \cdot \textbf{n}_k},\\
    \label{eq:dt_avg_kk}
    &\pddt (\phi_k\rho_kk_k)  
    + \div (
        \phi_k\rho_kk_k\textbf{u}_k
        + \textbf{q}_k^\text{k} 
        )
    = 
    - \avg{\chi_k\bm{\sigma}_k^0 : \grad \textbf{u}_k^0}
    - \bm{\sigma}_k^\text{eq} : \grad \textbf{u}_k
    + \avg{\delta_I \textbf{u}_k' \cdot \bm{\sigma}_k^0 \cdot \textbf{n}_k},\\
    \label{eq:dt_avg_ek}
    &\pddt (\phi_k\rho_ke_k)  
    + \div (
        \phi_k \rho_ke_k\textbf{u}_k
        +
        \textbf{q}_k^\text{e} 
        )
    = 
    \avg{\chi_k\bm{\sigma}_k^0 : \grad \textbf{u}_k^0}
    - \avg{\delta_I \textbf{q}_k^0 \cdot \textbf{n}_k},
\end{align}
respectively. 
This derivation is in agreement with \citet{morel2015mathematical}. 
Under this form the energy transfer across scale is clear. 
Indeed, the term $\bm{\sigma}_k^\text{eq} : \grad \textbf{u}_k$ act as a sink term in \ref{eq:dt_avg_uk2} and a source term in \ref{eq:dt_avg_kk}, while the averaged diffusive terms $\avg{\chi_k\bm{\sigma}_k^0 : \grad \textbf{u}_k^0}$ is a sink in \ref{eq:dt_avg_kk} and a source in \ref{eq:dt_avg_ek}. 
To determine the total energy only two of the four energy equations must be solved. 
In practice, it is useful to solve one equation for $k_k$ since it is useful for the Reynolds stress modeling since $\avg{\chi_k \textbf{u}_k^0 \textbf{u}_k^0}: \bm\delta = 2 k_k$, and another for $e_k$ since $u_k^2$ is already determined by the momentum equation. 

% \subsubsection{Averaged surface equations}
The volume averaged equations must be completed by an averaged surface transport equation.  
For this purpose we multiply \ref{eq:surface_tension}, \ref{eq:dt_rhoIe_I} and \ref{eq:dt_rhoI_uI3} by $\delta_I$ and apply the average operator.
By considering the topological equations we obtain the  momentum, kinetic and internal energy averaged surface equations, namely
\begin{align}
    \label{eq:dt_avg_uI}
    \avg{\delta_I \Jump{\bm{\sigma}^0_k}}
    &= -\gamma \div \avg{\delta_I (\bm\delta - \textbf{nn})},\\
    \label{eq:dt_avg_gamma}
    \avg{\delta_I \Jump{\textbf{u}_k^0 \cdot \bm{\sigma}^0_k}}
    &= - \gamma \left[
        \pddt \avg{\delta_I}
        +  \div \avg{\delta_I (\textbf{u}_{I}^0 \cdot \textbf{n})\textbf{n} },
    \right]\\
    \avg{\delta_I \Jump{\textbf{q}^0_k}}
    &= 0
\end{align}
respectively. 
The first equation represents the contribution of the surface tension stresses to the bulk momentum equation.
The second equation represents the amount of energy stored as surface deformation.
And the last equation witnesses of the continuity of heat flux through the interface. 

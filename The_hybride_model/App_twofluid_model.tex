\section{Averaged two-fluid model}
\label{ap:two-fluid_model}
Before introducing the \textit{hybrid model} which will be the subject of the next section, we expose in the first place the classic averaged two-fluid model. 

\subsection{Averaged volume equations}
Using the generic formulation \ref{eq:avg_dt_chi_f} and the local expression of the mass, momentum and total energy equation, i.e. : \ref{eq:dt_rho}, \ref{eq:dt_rhou_k} and \ref{eq:dt_rhoE_k} we easily find the averaged form of the mass, momentum and total energy equation.
They read, 
\begin{align}
    \label{eq:dt_avg_rho_k}
    \pddt (\phi_k \rho_k)  
    + \div (
        \phi_k \rho_k\textbf{u}_k
    )
    &= 
    0,\\
    \label{eq:dt_avg_rhou_k}
    \pddt (\phi_k \rho_k\textbf{u}_k)  
    + \div (
        \phi_k \rho_k\textbf{u}_k\textbf{u}_k
        + \bm{\sigma}_k^\text{eq}
    )
    &= 
    \phi_k \rho_k \textbf{g} 
    +  \avg{\delta_I \bm{\sigma}_k^0 \cdot \textbf{n}_k},\\
    \label{eq:dt_avg_rhoE_k}
    \pddt (\phi_k\rho_kE_k)  
    + \div (
        \phi_k\rho_kE_k\textbf{u}_k
        + \bm{q}_k^\text{eq}
        + \textbf{u}_k \cdot \bm{\sigma}_k^\text{eq}
        % - \textbf{u}_k^0 \cdot \bm{\sigma}_k^0 
        % + \textbf{q}_k^0
        )
    &= 
    \phi_k \rho_k\textbf{u}_k \cdot \textbf{g} 
    + \avg{\delta_I (\textbf{u}_k^0 \cdot \bm{\sigma}_k^0 - \textbf{q}_k^0)\cdot \textbf{n}_k},
\end{align} 
respectively. 
Where we have introduced the effective stress and heat fluxes namely, 
\begin{align*}
    &\bm{\sigma}_k^\text{eq}
    = 
     \rho_k\avg{\chi_k \textbf{u}_k'\textbf{u}_k'}
      - \phi_k \bm{\sigma}_k,%- n_p \textbf{M}_p
    &\textbf{q}_k^\text{eq}
    =\textbf{q}_k^\text{e} +\textbf{q}_k^\text{k},  \\
    &\textbf{q}_k^\text{e}
    = \rho_k \avg{\chi_k\textbf{u}_k' e_k'} 
    + \phi_k\textbf{q}_k,
    &\textbf{q}_k^\text{k}
    = \rho_k \avg{\chi_k \textbf{u}_k' k_k} 
    - \avg{\chi_k \textbf{u}_k' \cdot \bm{\sigma}_k^0}.
\end{align*}
The main differences between these equations and their microscale counterpart, is that : 
(1) we have introduced a pre factor $\phi_k$ in front of most the unknown  ($\textbf{u}_k$ and $E_k$), 
(2) the non-convective fluxes appear under the form of an effective tensor containing the covariance of the quantity to be conserved with the local velocity of the phase, 
and (3) a new source term appear on the RHS accounting for the exchange across the phases. 
Additionally, it is interesting to notice that the effective fluxes of the total energy is written as the mean work done by the effective stress of the momentum $\textbf{u}_k \cdot \bm{\sigma}_k^\text{eq}$ plus an effective heat flux $\bm{q}_k^\text{eq}$. 

The terms $\avg{\chi_k \textbf{u}_k'\textbf{u}_k'}$ will be referred as the Reynolds stress or pseudo turbulent stress. 
It has a fundamental importance in the multiphase flow problem as it appear in \ref{eq:dt_avg_rhou_k} and as we see now its trace is related to the mean kinetic energy. 
Indeed, the phase averaged total energy can be further decompose into three energy components, that is,  
\begin{align}
    E_k = e_k + k_k + u_k^2/2
    \label{eq:E_def2}
\end{align}
where $k_k$ is the pseudo-turbulent kinetic energy defined as, $\phi_k k_k = \frac{1}{2}\avg{\chi_k \textbf{u}_k'\cdot \textbf{u}_k'}$. 
Each of the components of the total energy represent the averaged agitation at different scales. 
From molecular agitation which quantified by $e_k$, to the macroscopic scale agitation or kinematic energy $u_k^2$, and in between we find the pseudo turbulent energy $k_k$ which represents the local scale velocity fluctuation. 
To fully describe the averaged total energy one must add at least a supplementary equation either for $k_k$ or $e_k$ assuming \ref{eq:dt_avg_rhoE_k} is solved. 
Using \ref{eq:dt_avg_rhou_k} dotted with $\textbf{u}_k$ yields an equation for the mean kinetic energy. 
Averaging \ref{eq:dt_rhoe_k} yields un equation for $e_k$.  
Then, subtracting these two equations to \ref{eq:dt_avg_rhoE_k} gives us an equation for $k_p$. 
The kinetic energy, pseudo turbulent and internal averaged equations are, 
\begin{align}
    \label{eq:dt_avg_uk2}
    &\pddt (\phi_k \rho_ku_k^2/2)  
    + \div (
        \phi_k \rho_k\textbf{u}_ku_k^2/2
        + \textbf{u}_k \cdot \bm{\sigma}_k^\text{eq}
    )
    = 
    \bm{\sigma}_k^\text{eq} : \grad \textbf{u}_k
    + \phi_k \rho_k \textbf{u}_k\cdot \textbf{g} 
    +  \textbf{u}_k\cdot \avg{\delta_I \bm{\sigma}_k^0 \cdot \textbf{n}_k},\\
    \label{eq:dt_avg_kk}
    &\pddt (\phi_k\rho_kk_k)  
    + \div (
        \phi_k\rho_kk_k\textbf{u}_k
        + \textbf{q}_k^\text{k} 
        )
    = 
    - \avg{\chi_k\bm{\sigma}_k^0 : \grad \textbf{u}_k^0}
    - \bm{\sigma}_k^\text{eq} : \grad \textbf{u}_k
    + \avg{\delta_I \textbf{u}_k' \cdot \bm{\sigma}_k^0 \cdot \textbf{n}_k},\\
    \label{eq:dt_avg_ek}
    &\pddt (\phi_k\rho_ke_k)  
    + \div (
        \phi_k \rho_ke_k\textbf{u}_k
        +
        \textbf{q}_k^\text{e} 
        )
    = 
    \avg{\chi_k\bm{\sigma}_k^0 : \grad \textbf{u}_k^0}
    - \avg{\delta_I \textbf{q}_k^0 \cdot \textbf{n}_k},
\end{align}
respectively. 
This derivation is in agreement with \citet{morel2015mathematical}. 
Under this form the energy transfer across scale is clear. 
Indeed, the term $\bm{\sigma}_k^\text{eq} : \grad \textbf{u}_k$ act as a sink term in \ref{eq:dt_avg_uk2} and a source term in \ref{eq:dt_avg_kk}, while the averaged diffusive terms $\avg{\chi_k\bm{\sigma}_k^0 : \grad \textbf{u}_k^0}$ is a sink in \ref{eq:dt_avg_kk} and a source in \ref{eq:dt_avg_ek}. 
To determine the total energy only two of the four energy equations must be solved. 
In practice, it is useful to solve one equation for $k_k$ since it is useful for the Reynolds stress modeling since $\avg{\chi_k \textbf{u}_k^0 \textbf{u}_k^0}: \bm\delta = 2 k_k$, and another for $e_k$ since $u_k^2$ is already determined by the momentum equation. 

\subsection{Averaged surface equations}
The volume averaged equations must be completed by an averaged surface transport equation.  
For this purpose we multiply \ref{eq:surface_tension}, \ref{eq:dt_rhoIe_I} and \ref{eq:dt_rhoI_uI3} by $\delta_I$ and apply the average operator.
By considering the topological equations we obtain the  momentum, kinetic and internal energy averaged surface equations, namely
\begin{align}
    \label{eq:dt_avg_uI}
    \avg{\delta_I \Jump{\bm{\sigma}^0_k}}
    &= -\gamma \div \avg{\delta_I (\bm\delta - \textbf{nn})},\\
    \label{eq:dt_avg_gamma}
    \avg{\delta_I \Jump{\textbf{u}_k^0 \cdot \bm{\sigma}^0_k}}
    &= - \gamma \left[
        \pddt \phi_I
        +  \div \avg{\delta_I (\textbf{u}_{I}^0 \cdot \textbf{n})\textbf{n} },
    \right]\\
    \avg{\delta_I \Jump{\textbf{q}^0_k}}
    &= 0
\end{align}
respectively. 
The first equation represents the contribution of the surface tension stresses to the bulk momentum equation.
The second equation represents the amount of energy stored as surface deformation.
And the last equation witnesses of the continuity of heat flux through the interface. 

\subsection{Averaged single fluid formulation}
Lastly, by summing the above formulations we can also derive equations for the bulk quantities. 
Indeed, summing \ref{eq:dt_avg_rho_k}, \ref{eq:dt_avg_rhou_k} and \ref{eq:dt_avg_rhoE_k} for $k=f,d$ with the surface equations one obtain the bulk conservation equations, 
\begin{align}
    \label{eq:dt_avg_rho}
    \pddt \rho 
    + \div (
         \rho \textbf{u}_m
    )
    &= 
    0,\\
    \label{eq:dt_avg_rhou}
    \pddt ( \rho\textbf{u}_m)  
    + \div (
         \rho\textbf{u}_m\textbf{u}_m
        + \bm{\sigma}^\text{eq}_m
    )
    &= 
     \rho \textbf{g} \\
    \label{eq:dt_avg_rhoE}
    \pddt (\rho E_m)  
    + \div (
        \rho E_m \textbf{u}_m
        + \bm{q}^\text{eq}_m
        + \textbf{u}_m \cdot \bm{\sigma}^\text{eq}_m
        % - \textbf{u}_k^0 \cdot \bm{\sigma}_k^0 
        % + \textbf{q}_k^0
        )
    &= 
     \rho \textbf{u}_m  \cdot \textbf{g} 
\end{align} 
respectively. 
Where we have introduced the effective stress and heat fluxes namely, 
\begin{align*}
    &\bm{\sigma}^\text{eq}_m
    = 
    \avg{ \rho^0\textbf{u}'_m\textbf{u}'_m}
      - \phi_f\bm\sigma_f
      - \phi_d\bm\sigma_d
      - \phi_I\bm\sigma_I,%- n_p \textbf{M}_p
    &\textbf{q}^\text{eq}_m
    =\textbf{q}^\text{e}_m +\textbf{q}^\text{k}_m,  \\
    &\textbf{q}^\text{e}_m
    = \avg{\rho^0\textbf{u}_m' e_m'} 
    + \phi\textbf{q}_m,
    &\textbf{q}^\text{k}_m
    = \avg{\rho^0 \textbf{u}_m' k_m} 
    - \avg{ \textbf{u}_m' \cdot \bm{\sigma}^0}.
\end{align*}
In these equations we have make use of the subscript $_m$ to denote the Favre average quantities, such that, 
\begin{align*}
    \rho \textbf{u}_m
    = \avg{\rho^0 \textbf{u}^0}
    &&
    \rho E_m
    = \avg{\rho^0 \textbf{u}^0}
    &&
    \rho e_m
    = \avg{\rho^0 \textbf{u}^0}
\end{align*}
Likewise, the fluctuating quantity are written as, 
\begin{equation}
    \textbf{u}_m'
    = \textbf{u}^0 - \textbf{u}_m.
\end{equation}
In \ref{eq:dt_avg_rhoE} we have defined the bulk total energy as the average of the sum of the local energy, namely, 
\begin{equation}
    \rho E_m = \avg{\sum_k \rho_k [e_k^0 + \frac{1}{2}(u_k)^2] 
    + \delta_I \gamma}
    = \rho( e_m +  k_m + \frac{1}{2}u_m^2)
\end{equation}
with, $e_m = \avg{\sum_k \rho_k\phi_k e_k + \phi_I \gamma}$ and $k = \avg{\rho^0 \textbf{u}'\cdot \textbf{u}'}/2$. 


These equations are particularly interesting from an experimental point of view. 
% Indeed, as in experiment we cannot distinguish between the details of both phase. 
For example in a Rheometer we are able to measure only for the \textit{Bulk stress} $\bm\sigma^\text{eq}_m$ and not $\bm\sigma^\text{eq}_f$ and $\bm\sigma^\text{eq}_d$ separately, as it measure the stress response from the mixture phases. 
Therefore, it is of interest to describe a bit more in details the \textit{Bulk stress}, which is the subject of the next section.
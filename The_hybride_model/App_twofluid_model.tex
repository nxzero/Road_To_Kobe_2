\section{The two-fluid model}
\label{ap:two-fluid_model}

% \subsubsection{Volume equation}
Using the generic formulation \ref{eq:avg_dt_chi_f} and the local expression of the mass, momentum and total energy equation, i.e. : \ref{eq:dt_rho},\ref{eq:dt_rhou_k} and \ref{eq:dt_rhoE_k} we easily find the averaged form of the mass, momentum and total energy equation.
They read, 
\begin{align}
    \label{eq:dt_avg_rho}
    \pddt (\phi_k \rho_k)  
    + \div (
        \phi_k \rho_k\textbf{u}_k
    )
    &= 
    0\\
    \label{eq:dt_avg_rhou_k}
    \pddt (\phi_k \rho_k\textbf{u}_k)  
    + \div (
        \phi_k \rho_k\textbf{u}_k\textbf{u}_k
        + \bm{\sigma}_k^\text{eq}
    )
    &= 
    \phi_k \rho_k \textbf{g} 
    +  \avg{\delta_I \bm{\sigma}_k^0 \cdot \textbf{n}_k}\\
    \label{eq:dt_avg_rhoE_k}
    \pddt (\phi_k\rho_kE_k)  
    + \div (
        \phi_k\rho_kE_k\textbf{u}_k
        + \bm{q}_k^\text{eq}
        + \textbf{u}_k \cdot \bm{\sigma}_k^\text{eq}
        % - \textbf{u}_k^0 \cdot \bm{\sigma}_k^0 
        % + \textbf{q}_k^0
        )
    &= 
    \phi_k \rho_k\textbf{u}_k \cdot \textbf{g} 
    + \avg{\delta_I (\textbf{u}_k^0 \cdot \bm{\sigma}_k^0 - \textbf{q}_k^0)\cdot \textbf{n}_k}
\end{align} 
with the equivalent stress and heat flux are defined as, 
\begin{align*}
    &\bm{\sigma}_k^\text{eq}
    = 
     \rho_k\avg{\chi_k \textbf{u}_k'\textbf{u}_k'}
      - \phi_k \bm{\sigma}_k,%- n_p \textbf{M}_p
    &\textbf{q}_k^\text{eq}
    =\textbf{q}_k^\text{e} +\textbf{q}_k^\text{k},  \\
    &\textbf{q}_k^\text{e}
    = \rho_k \avg{\chi_k\textbf{u}_k' e_k'} 
    + \phi_k\textbf{q}_k,
    &\textbf{q}_k^\text{k}
    = \rho_k \avg{\chi_k \textbf{u}_k' k_k} 
    - \avg{\chi_k \textbf{u}_k' \cdot \bm{\sigma}_k^0}.
\end{align*}
The main differences between these equations and their microscale counterpart, is that : (1) we have introduced a pre factor $\phi_k$ in front of most of the terms
(2) an additional stress is present, that is the covariance between the quantity to be conserved and the velocity. 
(3) A new source term appear on the RHS accounting for the exchange across the phases. 

The terms $\avg{\chi_k \textbf{u}_k'\textbf{u}_k'}$ will be referred as the Reynolds stress or pseudo turbulent stress. 
It has a fundamental importance in the multiphase flow problem and as we see now its trace is related to the mean kinetic energy. 
Indeed, the phase averaged total energy can be further decompose into three energy components, that is,  
\begin{align}
    E_k = e_k + k_k + u_k^2/2
    \label{eq:E_def}
\end{align}
where $k_1$ is the pseudo-turbulent kinetic energy defined as, $\phi_k k_k = \frac{1}{2}\avg{\chi_k \textbf{u}_k'\cdot \textbf{u}_k'}$. 
Each of the components of the total energy represent the averaged kinetic energy at different scales. 
From molecular kinetic energy which is $e_k$, to the macroscopic scale with the kinematic energy $u_k^2$, and in between we find the pseudo turbulent energy $k_k$. 
To fully describe the averaged total energy one must add a supplementary equation either for $k_k$ or $e_k$. 
In fact an equation for each of these components can be derived using the averaged momentum and total energy equation presented above. 
The kinetic energy, pseudo turbulent and internal averaged equations read, 
\begin{align}
    \label{eq:dt_avg_uk2}
    &\pddt (\phi_k \rho_ku_k^2/2)  
    + \div (
        \phi_k \rho_k\textbf{u}_ku_k^2/2
        + \textbf{u}_k \cdot \bm{\sigma}_k^\text{eq}
    )
    = 
    \bm{\sigma}_k^\text{eq} : \grad \textbf{u}_k
    + \phi_k \rho_k \textbf{u}_k\cdot \textbf{g} 
    +  \textbf{u}_k\cdot \avg{\delta_I \bm{\sigma}_k^0 \cdot \textbf{n}_k},\\
    \label{eq:dt_avg_kk}
    &\pddt (\phi_k\rho_kk_k)  
    + \div (
        \phi_k\rho_kk_k\textbf{u}_k
        + \textbf{q}_k^\text{k} 
        )
    = 
    - \avg{\chi_k\bm{\sigma}_k^0 : \grad \textbf{u}_k^0}
    - \bm{\sigma}_k^\text{eq} : \grad \textbf{u}_k
    + \avg{\delta_I \textbf{u}_k' \cdot \bm{\sigma}_k^0 \cdot \textbf{n}_k},\\
    \label{eq:dt_avg_ek}
    &\pddt (\phi_k\rho_ke_k)  
    + \div (
        \phi_k \rho_ke_k\textbf{u}_k
        +
        \textbf{q}_k^\text{e} 
        )
    = 
    \avg{\chi_k\bm{\sigma}_k^0 : \grad \textbf{u}_k^0}
    - \avg{\delta_I \textbf{q}_k^0 \cdot \textbf{n}_k},
\end{align}
respectively. 
This derivation is in agreement with \citet{morel2015mathematical}. 
Under this form the energy transfer across scale is clear. 
Indeed, the term $\bm{\sigma}_k^\text{eq} : \grad \textbf{u}_k$ act as a sink term in \ref{eq:dt_avg_uk2} and a source term in \ref{eq:dt_avg_kk}, while the averaged diffusive terms $\avg{\chi_k\bm{\sigma}_k^0 : \grad \textbf{u}_k^0}$ is a sink in \ref{eq:dt_avg_kk} and a source in \ref{eq:dt_avg_ek}. 
To determine the total energy only two of the four energy equations must be solved. 
In practice, it is useful to solve one equation for $k_1$ since it is related to the Reynolds stress appearing in the momentum equation through $\bm{\sigma}^\text{eq}_k$, with $\avg{\chi_k \textbf{u}_k^0 \textbf{u}_k^0}: \textbf{I} = 2 k_k$ and another for $e_1$ while $u_k^2$ is determinate with the averaged momentum equation. 


% \subsubsection{Averaged surface equations}
In light of \ref{eq:avg_dt_chi_f} and \ref{eq:avg_dt_delta_f} the volume averaged equations must be completed by an averaged surface transport equation.  
To this purpose we multiply \ref{eq:dt_rho_I}, \ref{eq:surface_tension}, \ref{eq:dt_rhoIe_I} and \ref{eq:dt_rhoI_uI2} by $\delta_I$ and apply the average operator.
By considering the topological equations it eventually gives us the mass, momentum, and total energy averaged conservation equation, 
\begin{align}
    \label{eq:dt_avg_rho_I}
    \avg{\delta_I \Jump{\rho_k (\textbf{u}_I - \textbf{u}_k^0)}}
    = 0,\\
    \label{eq:dt_avg_uI}
    \avg{\delta_I \Jump{\bm{\sigma}^0_k}}
    = - \div \avg{\delta_I (\textbf{I} - \textbf{nn}) \gamma},\\
    \label{eq:dt_avg_gamma}
    \avg{\delta_I \Jump{\textbf{u}_k^0 \cdot \bm{\sigma}^0_k - \textbf{q}^0_k}}
    = - \pddt \avg{\delta_I \gamma}
    - \div \avg{\delta_I \gamma (\textbf{u}_{I}^0 \cdot \textbf{n})\textbf{n} },
\end{align}
respectively. 
The first equation represents the condition of no mean mass transfer between phase.
The second equation represents the contribution of the surface tension stresses to the bulk momentum equation.
And the last equation represents the amount of energy stoked in the surface. 

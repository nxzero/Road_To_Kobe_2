Two distinct descriptions can be applied to the dispersed phase, while only one description is applicable to the fluid phase. 
In this section, we derive averaged equations for the dispersed phase using Lagrangian conservation laws. 
Following this, we  discuss the equivalence between the particle or lagrangian averaged equations for the dispersed phase and the averaged equations for the dispersed phase presented in \ref{sec:two-fluid}.


%Two different descriptions are possible for the dispersed phase while one is available for the fluid phase. 
%In this part we first derive averaged equations for the dispersed phase based on Lagrangian conservation laws. 
%Then we provide a complete discussion regarding the equivalence between the "lagrangian" averaged equations for the dispersed phase and the averaged equations governing the dispersed phase presented in \ref{sec:two-fluid}. 
%based on 
%the set of equations just derived and the averaged equations governing the dispersed phase presented in \ref{sec:two-fluid}. 

\subsection{Dispersed phase averaged equations}

In the preceding section, we have described the dispersed phase using a Lagrangian framework. 
However, to ensure consistency with the Eulerian conservation equations that describe the continuous phase, it is necessary to extend the Lagrangian equations to an Eulerian description. 
The approach presented here follows the methodology pioneered by \citep{lhuillier1992ensemble}.
%In the last section, we have described the dispersed phase within a Lagrangian framework.
%However, to be consistent with the Eulerian conservation equations used to describe the continuous phase, we need to extend the Lagrangian equations to an Eulerian description. 
%The strategy exposed here follow the approach pionnered by \citep{lhuillier1992ensemble}.
%In order to achieve this,
We introduce the function $\delta_\alpha$, which is defined as follows, 
\begin{align}
    \delta_\alpha(\textbf{x},\textbf{x}_\alpha(t,\FF)) 
    = \delta(\textbf{x}-\textbf{x}_\alpha(t,\FF)),
    \label{eq:delta_alpha}
\end{align}
where $\delta$ is the Dirac function.
Note that we explicitly note the arguments $(t,\FF)$ to highlight that the position of the particle $\alpha$ is a function of time and of the flow configuration $\FF$.
Applying the chain rule of derivative we may write the partial time derivative of $\delta_\alpha$ can be written as
\begin{equation}
\frac{\partial \delta_\alpha(\textbf{x},\textbf{x}_\alpha(t,\FF))}{\partial t} 
=  \frac{\partial \textbf{x}_\alpha}{\partial t} 
\cdot \frac{\partial \delta_\alpha}{\partial \textbf{x}_\alpha}(\textbf{x},\textbf{x}_\alpha(t,\FF)) .
\end{equation}
This leads to the following expression, 
\begin{equation}
    \pddt \delta_\alpha
    + \div (\textbf{u}_\alpha  \delta_\alpha)
    =0,
    \label{eq:dt_delta_alpha}
\end{equation}
where we used the identity, $\frac{\partial \delta_\alpha}{\partial \textbf{x}_\alpha}  = -\grad \delta_\alpha$ and the fact that $\textbf{u}_\alpha(t,\FF)$ is not a function of $\textbf{x}$. 
Equation \ref{eq:dt_delta_alpha} does not apply in scenarios where topological changes occur, such as break-up or coalescence events. 
In these cases, a source term can be introduced on the right-hand side of \ref{eq:dt_delta_alpha}, similar to the approach used in population balance equations, to account for the birth or death of particles.
%It should be noted that \ref{eq:dt_delta_alpha} is not applicable if changes in topology, such as break-up or coalescence events, occur.
%In such cases it is possible, as it is done in population balance equations, to include a source term on the RHS of \ref{eq:dt_delta_alpha} to account for particle birth or death. 
%Multiplying each Lagrangian quantities $q_\alpha$ by $\delta_\alpha$ yields the field $q_\alpha(t,\FF)\delta_\alpha(\textbf{x},t,\FF)$, which is defined over the entire domain $\Omega$.
%Likewise, for any derivative of Lagrangian quantities, such as $\ddt q_\alpha$, we define its corresponding Eulerian field by multiplying $\ddt q_\alpha$ with $\delta_\alpha$ and show that :
By multiplying each Lagrangian quantity $q_\alpha$​ by $\delta_\alpha$​, we obtain the field $q_\alpha(t,\FF)\delta_\alpha(\textbf{x},t,\FF)$, which is defined over the entire domain $\Omega$. 
Similarly, for any derivative of Lagrangian quantities, such as $\ddt q_\alpha$​, we define the corresponding Eulerian field by multiplying $\ddt q_\alpha$ with $\delta_\alpha$ 
%This can be expressed as
Given that $q_\alpha(t,\FF)$ and $\textbf{u}_\alpha(t,\FF)$ do not depend on \textbf{x}, and by using Equation \ref{eq:dt_delta_alpha}, we obtain
\begin{equation}
    \delta_\alpha \ddt q_\alpha
    = \pddt (\delta_\alpha q_\alpha)
    + \div (\delta_\alpha q_\alpha \textbf{u}_\alpha).
    \label{eq:dt_delta_alpha_q_alpha}
\end{equation}
%where we have used the fact that $q_\alpha(t,\FF)$ and $\textbf{u}_\alpha(t,\FF)$ are not function of \textbf{x}, and we made use of \ref{eq:dt_delta_alpha}.
%Now let us consider a domain containing $N$ particles.
%We define what we call the \textit{particle field} of a quantity $q_\alpha$, as the sum of the $\delta_\alpha q_\alpha$ over all particles in the flow, namely $\displaystyle\sum_{\alpha=0}^N \delta_\alpha q_\alpha$.
%Notice that \ref{eq:dt_delta_alpha_q_alpha} remains valid for a sum of fields since derivative and sum operators commute.
Consider a domain with $N$ particles. We define the \textit{particle field} of a quantity $q_\alpha$​ as the sum of $\delta_\alpha q_\alpha$ over all particles in the domain, represented by $\displaystyle\sum_{\alpha=0}^N \delta_\alpha q_\alpha$​. 
Note that the formula given by \ref{eq:dt_delta_alpha_q_alpha} holds true for the sum of fields, since the derivative and sum operators are commutative.

%In the objective of obtaining averaged equations for the dispersed phase, we introduce the average of $q_\alpha$ as  
To obtain the averaged equations for the dispersed phase, we define the particle average of $q_\alpha$​ as follows:
\begin{equation}
     n_p q_p(\textbf{x},t) = \avg{\sum_\alpha\delta_\alpha q_\alpha}
     \label{eq:p_avg}
\end{equation}
where, $n_p(\textbf{x},t) = \avg{\sum_\alpha \delta_\alpha}$ is the probability density of finding a particle center of mass in the infinitesimal volume $d\textbf{x}$ around \textbf{x}, and $q_p(\textbf{x},t)$ is the average of $q_\alpha$ conditionally on the presence of a particle at \textbf{x} and time $t$. 
To simplify the notations, we consider the shorthand,
\begin{equation*}
    \sum_\alpha \delta_\alpha \to \delta_p, 
\end{equation*}
such that $\pavg{q_\alpha}=\avg{\sum_\alpha \delta_\alpha q_\alpha}=n_pq_p$.
Note that we used the subscript $_p$ on $q_p$ to indicate that this is a particle averaged field based initially on Lagrangian quantities.  
Additionally, in light of \ref{eq:def_fluctu} we define the fluctuating part of a particle field $q_p$ as
\begin{equation}
    q_\alpha' = q_\alpha - q_p. 
    \label{eq:def_fluc_p}
\end{equation}

To obtain the particle phase averaged equations one multiply \ref{eq:dt_q_alpha_tot} and \ref{eq:dt_Q_alpha_tot} by $\delta_\alpha$ and apply the ensemble average (\ref{eq:avg}), this yields
\begin{align}
    \pddt (n_pQ_p)
    + \div (n_p Q_p \textbf{u}_p + \pavg{\textbf{u}_\alpha' Q_\alpha'})
    = \pOavg{ s_d^0 }
    + \pSavg{ s_I^0 }\nonumber\\
    + \pSavg{ \left[\mathbf{\Phi}_f^0 + f_f^0 (\textbf{u}_\Gamma^0-\textbf{u}_f^0) \right] \cdot \textbf{n}_d },
    \label{eq:avg_dt_dq_alpha_tot}\\
    \pddt (n_p\textbf{Q}_p^{(1)})
    + \div (n_p \textbf{Q}_p^{(1)} \textbf{u}_p + \pavg{\textbf{u}_\alpha' (Q_\alpha^{(1)})'})
    =\pOavg{ \left(
        \textbf{r} s_d^0         
        + f_d^0  \textbf{w}_d^0 
        - \mathbf{\Phi}_d^0
    \right) }\nonumber\\
    + \pSavg{ \left(
        \textbf{r}s_\Gamma^0
        + f_\Gamma^0 \textbf{w}_\Gamma^0
        - \mathbf{\Phi}_{\Gamma||}^0
    \right) }
    + \pSavg{ \textbf{r} \left[
        \mathbf{\Phi}_f^0
        + f_f^0 (\textbf{u}_\Gamma^0-\textbf{u}_f^0)
    \right]\cdot \textbf{n}_d  }.
    \label{eq:avg_dt_dQ_alpha_tot}
\end{align}
Notice that the only fluxes present in \ref{eq:avg_dt_dq_alpha_tot} and \ref{eq:avg_dt_dQ_alpha_tot} are the fluctuation tensors $\pavg{\textbf{u}_\alpha' q_\alpha'}$ and $\pavg{\textbf{u}_\alpha' \bm{Q}_\alpha'}$. 
Thus, the diffusive fluxes $\bm\Phi_d^0$ and $\bm\Phi_I^0$ do not play the role of macroscopic diffusive fluxes, as it is the case in \ref{eq:avg_dt_chi_f} and \ref{eq:avg_dt_delta_f}, but rather they act as source terms in the first moment and higher moment equations. 
This is the main structural differences between the Kinetic-like model (\ref{eq:avg_dt_dq_alpha_tot} and \ref{eq:avg_dt_dQ_alpha_tot}) and the two-phase flow model (\ref{eq:avg_dt_chi_f} and \ref{eq:avg_dt_delta_f}). 
The derivation of the higher moment particle-averaged equations is provided in \ref{ap:Moments_equations} \JL{separer l'annexe C en deux}\tb{pas trop d'accord de faire une annexe pour 1 equations}. 
In this study, \ref{eq:avg_dt_chi_f} and \ref{eq:avg_dt_delta_f} are referred to as the phase-averaged equations, while \ref{eq:avg_dt_dq_alpha_tot} and \ref{eq:avg_dt_dQ_alpha_tot} are called the particle-averaged equations. 

 




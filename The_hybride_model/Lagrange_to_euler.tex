The first objective of this section is to demonstrate how to derive averaged equations for the particle phase based on Lagrangian equilibrium laws. 
Then we provide a complete discussion regarding the equivalence between the set of equations just derived and the averaged equations governing the dispersed phase presented in \ref{sec:two-fluid}. 

\subsection{Particles phase averaged equations}

Up to this point, we have described the dispersed phase within a Lagrangian framework.
However, to be consistent with the Eulerian conservation equations used to describe the continuous phase, we need to extend the Lagrangian equations to an Eulerian description. 
In order to achieve this, we introduce the function $\delta_\alpha$, which is defined as follows, 
\begin{align}
    \delta_\alpha(\textbf{x},t) = \delta(\textbf{x}-\textbf{x}_\alpha(t,\FF)),
    \label{eq:delta_alpha}
\end{align}
where $\delta$ is the Dirac delta function.
Note that we explicitly note the arguments $(t,\FF)$ to highlight that the position of the particle $\alpha$ is a function of time and of the flow configuration $\FF$.
In the sense of generalized functions, the partial time derivative of $\delta_\alpha$ can be written $\frac{\delta_\alpha(\textbf{x},t,\FF)}{\partial t} =  \frac{\partial \textbf{x}_\alpha}{\partial t} \cdot \frac{\partial \delta_\alpha}{\partial \textbf{x}_\alpha} $.
This leads to the following expression, 
\begin{equation}
    \pddt \delta_\alpha
    + \div (\textbf{u}_\alpha  \delta_\alpha)
    =0,
    \label{eq:dt_delta_alpha}
\end{equation}
where we used the identity, $\frac{\partial \delta_\alpha}{\partial \textbf{x}_\alpha}  = -\grad \delta_\alpha$ and the fact that $\textbf{u}_\alpha(t,\FF)$ is not a function of $\textbf{x}$. 
Additionally, it should be noted that \ref{eq:dt_delta_alpha} is not applicable if changes in topology, such as break-up or coalescence events, occur.
In such cases it is possible, as it is done in population balance equations, to include a source term on the RHS of \ref{eq:dt_delta_alpha} to account for particle birth or death. 
Multiplying each Lagrangian quantities $q_\alpha$ by $\delta_\alpha$ yields the field $q_\alpha(t,\FF)\delta_\alpha(\textbf{x},t,\FF)$, which is defined over the entire domain $\Omega$.
Likewise, for any derivative of Lagrangian quantities, such as $\ddt q_\alpha$, we define its corresponding Eulerian field by multiplying $\ddt q_\alpha$ with $\delta_\alpha$ and show that :
\begin{equation}
    \delta_\alpha \ddt q_\alpha
    = \pddt (\delta_\alpha q_\alpha)
    + \div (\delta_\alpha q_\alpha \textbf{u}_\alpha)
    \label{eq:dt_delta_alpha_q_alpha}
\end{equation}
where we have utilized the fact that $q_\alpha(t,\FF)$ and $\textbf{u}_\alpha(t,\FF)$ are not function of \textbf{x}, and we made use of \ref{eq:dt_delta_alpha}.
Now let us consider a volume containing $N$ particles.
We define what we call the \textit{particle field} of a quantity $q_\alpha$, as the sum of the $\delta_\alpha q_\alpha$ over all particles in the flow, namely $\sum_{\alpha=0}^N \delta_\alpha q_\alpha$.
Notice that \ref{eq:dt_delta_alpha_q_alpha} remains valid for a sum of fields since derivative operators are linear.
To simplify the notations, we consider implicitly the summation over all particles included in $\Omega$ whenever a Lagrangian field denoted by $\delta_\alpha (\ldots)$ is present.

In the objective of obtaining coarse-grained level equations for the dispersed phase, we introduce the average of $q_\alpha$ as  
\begin{equation}
     n_p q_p(\textbf{x},t) = \avg{\delta_\alpha q_\alpha}
     \label{eq:p_avg}
\end{equation}
where, $n_p(\textbf{x},t) = \avg{\delta_\alpha}$ is the probability density of finding a particle center of mass in the infinitesimal volume $d\textbf{x}$ around \textbf{x}, and $q_p(\textbf{x},t)$ is the average of $q_\alpha$ conditionally on the presence of a particle at \textbf{x} and time $t$. 
Note that we used the subscript $_p$ on $q_p$ to indicate that this is an averaged field based initially on Lagrangian quantities.  
Additionally, in light of \ref{eq:def_fluctu} we define the fluctuating part of a particle field $q_p$ as
\begin{equation}
    q_\alpha' = q_\alpha - q_p. 
    \label{eq:def_fluc_p}
\end{equation}

To obtain the particle phase averaged equations one multiply \ref{eq:dt_q_alpha_tot} and \ref{eq:dt_Q_alpha_tot} by $\delta_\alpha$ and apply the ensemble average (\ref{eq:avg}), this yields
\begin{align}
    \pddt (n_pq_p)
    + \div (n_p q_p \textbf{u}_p + \bm\Phi_p^\text{eq})
    &= \pOavg{ s_d^0 }
    + \pSavg{ s_I^0 }
    + \pSavg{ \left[\mathbf{\Phi}_f^0 + f_f^0 (\textbf{u}_I^0-\textbf{u}_f^0) \right] \cdot \textbf{n}_d },
    \label{eq:avg_dt_dq_alpha_tot}\\
    \pddt (n_p\textbf{Q}_p)
    + \div (n_p \textbf{Q}_p \textbf{u}_p + \bm\Phi_p^\text{eq-2})
    &=\pOavg{ \left(
        \textbf{r} s_d^0         
        + f_d^0  \textbf{w}_d^0 
        - \mathbf{\Phi}_d^0
    \right) }
    + \pSavg{ \left(
        \textbf{r}s_I^0
        + f_I^0 \textbf{w}_I^0
        - \mathbf{\Phi}_{I||}^0
    \right) }\nonumber\\
    &+ \pSavg{ \textbf{r} \left[
        \mathbf{\Phi}_f^0
        + f_f^0 (\textbf{u}_I^0-\textbf{u}_f^0)
    \right]\cdot \textbf{n}_d  }.
    \label{eq:avg_dt_dQ_alpha_tot}
\end{align}
with, 
\begin{align*}
    \bm\Phi_p^\text{eq-1}= \pavg{\textbf{u}_\alpha' q_\alpha'},
    && \bm\Phi_p^\text{eq-1}= \pavg{\textbf{u}_\alpha' \bm{Q}_\alpha'}.
\end{align*}
Notice that the only fluxes present in \ref{eq:avg_dt_dq_alpha_tot} and \ref{eq:avg_dt_dQ_alpha_tot} are the fluctuation tensors $\pavg{\textbf{u}_\alpha' q_\alpha'}$ and $\pavg{\textbf{u}_\alpha' \bm{Q}_\alpha'}$. 
Thus, the diffusive fluxes $\bm\Phi_d^0$ and $\bm\Phi_I^0$ do not play the role of macroscopic diffusive fluxes, as it is the case in \ref{eq:avg_dt_chi_f} and \ref{eq:avg_dt_delta_f}, but rather they act as source terms in the first moment and higher moment equations. 
This is the main structural differences between the Kinetic-like model (\ref{eq:avg_dt_dq_alpha_tot} and \ref{eq:avg_dt_dQ_alpha_tot}) and the two-phase flow model (\ref{eq:avg_dt_chi_f} and \ref{eq:avg_dt_delta_f}). 
The derivation of the higher moment particle-averaged equations is provided in \ref{ap:Moments_equations}. 
In this study, \ref{eq:avg_dt_chi_f} and \ref{eq:avg_dt_delta_f} are referred to as the phase-averaged equations, while \ref{eq:avg_dt_dq_alpha_tot} and \ref{eq:avg_dt_dQ_alpha_tot} are called the particle-averaged equations. 

 




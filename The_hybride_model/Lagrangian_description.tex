
In this section, we present a Lagrangian-based model capable of describing the dispersed phase with an arbitrary order of accuracy.

\subsection{Fundamental properties}

At this stage, we define some fundamental properties associated to each particle labeled $\alpha$.
Following the strategy of \citet{lhuillier2009rheology,lhuillier1992volume,zaepffel2011modelisation} and \citet[Chapter 2]{morel2015mathematical}
we define the mass $m_\alpha$, position of center of mass $\mathbf{x}_\alpha$, and the momentum $\textbf{p}_\alpha$ of the particle $\alpha$, as
\begin{align}
    m_\alpha(t,\FF)
    = \intO{ \rho_d  }, 
    &&
    \textbf{x}_\alpha(t,\FF)
    = \frac{1}{m_\alpha(t,\FF) }\intO{ \rho_d \textbf{x} }, 
    &&\textbf{p}_\alpha(t,\FF) 
    = \intO{ \rho_d \textbf{u}_d^0 }.
    \label{eq:mass_pos}
    % \label{eq:momentum_energy}
\end{align}
$\Omega_\alpha(t,\FF)$ is the time-dependent domain occupied by the particle $\alpha$ (see \ref{fig:Scheme}). 
Subsequently, we define the velocity of the particle center of mass as
\begin{equation*}
\textbf{u}_\alpha = \frac{d \textbf{x}_\alpha}{dt}.
\end{equation*}
Replacing $\textbf{x}_\alpha$ by its definition (\ref{eq:mass_pos}) we obtain
\begin{equation*}
    \textbf{u}_\alpha = \frac{1}{m_\alpha}
    \frac{d}{dt} 
    \left(
        \intO{ \rho_d \textbf{x} }
    \right)
    - \frac{1}{m_\alpha^2} \frac{d}{dt} \left(\intO{ \rho_d } \right)
    \intO{ \rho_d \textbf{x} }.
\end{equation*}
%\tb{ A finaliser
Using the Reynolds transport theorem (\ref{eq:reynolds_transport}) for both terms in parentheses and making use of the conservation of mass (\ref{eq:dt_rho}) and the definition of $\textbf{x}_\alpha(t,\FF)$ in the last term, gives
\begin{equation}
    \textbf{u}_\alpha = 
    \frac{1}{m_\alpha}\intO{ \left[
        \pddt (\textbf{x}\rho_d ) + \div\left(\textbf{u}_d^0 \textbf{x} \rho_d\right) 
    \right]} \\
    + \frac{1}{m_\alpha}\intS{ \textbf{x} \rho_d(\textbf{u}_I   - \textbf{u}_d^0) \cdot \textbf{n}_d }
    -  \frac{\textbf{x}_\alpha}{m_\alpha}    \intS{ \rho_d(\textbf{u}_I   - \textbf{u}_d^0) \cdot \textbf{n}_d }
\end{equation}
Then by considering the mass conservation for the first term and noticing that $\grad \textbf{x} = \bm\delta$ with $\bm\delta$ the unit tensor, for the second term gives, 
\begin{equation}
    \textbf{u}_\alpha(t,\FF) = \frac{1}{m_\alpha(t,\FF)} \left(
        \textbf{p}_\alpha(t,\FF)
        +  \intS{\rho_d \textbf{r} (\textbf{u}_I^0 - \textbf{u}_d^0)\cdot \textbf{n}_d }
        \right),
        \label{eq:dt_y_alpha}
\end{equation}
where $\textbf{r}(\textbf{x},t) = \textbf{x} - \textbf{x}_\alpha(t)$. 
In \ref{eq:dt_y_alpha}, it can be observed that the first component of the velocity represents the linear momentum divided by the mass of the particle. 
This corresponds to the mass-averaged velocity over the volume of the particle.
The second term in \ref{eq:dt_y_alpha} arises from the contribution of anisotropic mass transfer across the surface of the particle. 
This mass transfer leads to the motion of the particle's center of mass, thereby contributing to the total velocity.
To illustrate this concept, let us consider a fixed drop with no momentum lying over a very hot plate.
In this scenario, we assume that the plate is sufficiently hot to induce evaporation, specifically on the bottom portion of the drop.
Hence, under the effect of an anisotropic evaporation flux one may expect the second term to be non-negligible.
Consequently, the center of mass of the drop has a non-zero velocity in the opposite direction of the plate, even though the momentum is assumed to be zero.
Note that \ref{eq:dt_y_alpha} generalized usual expression of the center of mass velocity whom neglect the second term.
In the following, for the sake of brevity we discard the dependency on $t$ and $\FF$ on the notations for all Lagrangian quantities denoted by the subscript $_\alpha$ and in particular $\Gamma_\alpha$ and $\Omega_\alpha$.
Nevertheless, the reader must understand that all Lagrangian quantities and integration domains subscribed by $_\alpha$ are time and configuration-dependent. 

The particle's internal relative motions or the \textit{inner velocity} is given by $\textbf{w}_d^0 = \textbf{u}_d^0 - \textbf{u}_\alpha$. 
Substituting the inner velocity in the momentum definition (\ref{eq:mass_pos}) yields
\begin{equation}
    \label{eq:momentum_definition_1}
    \textbf{p}_\alpha
    = m_\alpha \textbf{u}_\alpha
    + \int_{\Omega_\alpha} \rho_d \textbf{w}_d^0 d\Omega.
\end{equation}
Alternatively, from \eqref{eq:dt_y_alpha}, we obtain,
\begin{equation}
    \textbf{p}_\alpha
    =  m_\alpha \textbf{u}_\alpha
    - \int_{\Gamma_\alpha} \rho_d\textbf{r}(\textbf{u}_I^0 - \textbf{u}_d^0)\cdot \textbf{n}_d d\Sigma
    \label{eq:momentum_definition}
\end{equation}
Therefore, the momentum of a particle can be seen as a sum of the mean velocity plus the integral of the fluctuation (\ref{eq:momentum_definition_1}), with the latter being equivalent to minus the first moment of mass transfer term (\ref{eq:momentum_definition}).
Indeed, by identification we obtain : $\intO{ \rho_d \textbf{w}_d^0 } = - \intS{  \rho_d\textbf{r} (\textbf{u}_I^0 - \textbf{u}_d^0)\cdot \textbf{n}_d }$. 
Hence, the internal velocity fluctuations within a fluid particle do not contribute to the total linear momentum $\textbf{p}_\alpha$, as long as the anisotropic mass transfer is negligible.  
Only within this simplified context we can consider the classic relation $\textbf{p}_\alpha = m_\alpha \textbf{u}_\alpha$. 

\subsection{Conservation laws}

We assign to a particle indexed, $\alpha$, occupying the domain $\Omega_\alpha$ (see \ref{fig:Scheme}) an arbitrary Lagrangian property $q_\alpha$ defined by $q_\alpha  = \intO{ f_d^0}$.
Similarly, we define $q_{I\alpha} = \intS{ f_I^0}$ as an integrated surface property of the particle $\alpha$.

\subsubsection{Inside the volume}
To describe the evolution of any arbitrary Lagrangian quantity $q_\alpha$, we need to establish its time derivative.
Since $q_\alpha$ is an integral quantity with a time-dependent domain of integration, we apply the general Reynolds transport theorem for volume integral which gives for material domains (here the droplet volume),
\begin{equation}
    \ddt  \intO{f_d^0}
    = \intO{\left[ \pddt f_d^0 + \div\left(f_d^0\textbf{u}_d^0\right) \right]}\\
    + \intS{ f_d^0 (\textbf{u}_I^0-\textbf{u}_d^0)\cdot \textbf{n}_d }.
    \label{eq:reynolds_transport}
\end{equation}
By substituting the integrand of the first integral on the right-hand side (RHS) with \ref{eq:dt_f_k} we obtain the conservation law of the quantity $q_\alpha$, namely,  
\begin{equation}
    \ddt{q_\alpha}
    = \intO{ s_d^0 }
    + \intS{ \left[
        f_d^0 (\textbf{u}_I^0-\textbf{u}_d^0) 
        + \mathbf{\Phi}_d^0 
        \right] \cdot \textbf{n}_d }.
    \label{eq:dt_q_alpha}
\end{equation}
In \ref{eq:dt_q_alpha} we used the Gauss divergence theorem to show that
\begin{equation}
    \intO{\div \mathbf{\Phi}_d^0} = \intS{\mathbf{\Phi}_d^0 \cdot \textbf{n}_d}.
\end{equation}
The first term on the right-hand side of \ref{eq:dt_q_alpha} accounts for the total contribution of the source term $s_d^0$ to the particle $\alpha$,
while the second term is the surface integration of the exchange terms, which includes the phase transfer flux $f_d^0 (\textbf{u}_I^0-\textbf{u}_d^0)$ and the diffusive flux $\mathbf{\Phi}_d^0$. 


Let us consider the specific case of the momentum balance, i.e. $q_\alpha = \textbf{p}_\alpha$.
In this situation, \ref{eq:dt_q_alpha} reads
\begin{equation}
    \ddt  \textbf{p}_\alpha
    = \intO{ \rho_d\textbf{g} }
    + \intS{ 
        \left[
        f_d^0 (\textbf{u}_I^0-\textbf{u}_d^0)
         + \bm{\sigma}_d^0%\cdot\textbf{n}_d  
        %+ \mathbf{\Phi}_d^0 
        \right] 
        \cdot \textbf{n}_d },
\end{equation}
% first term reads as $\intO{ \rho_d\textbf{g} }$ 
The first term on the right-hand side represents the total weight acting on the particle $\alpha$, 
the second term represents the total source of momentum due to phase transfer, and it is expressed as, $\intS{ \rho_d \textbf{u}_d^0 (\textbf{u}_I^0-\textbf{u}_d^0)\cdot\textbf{n}_d }$,
and the last term $\intS{ \bm{\sigma}_d^0\cdot\textbf{n}_d }$, represents the resultant of the hydrodynamic forces acting on the surface of the particle.
It is important to notice that under this form, the exchange terms are expressed as integrals of dispersed phase fields denoted by the subscript $_d$.
Nevertheless, depending on the nature of the dispersed phase, these fields may not always be defined.
For rigid particles the stress within the particle $\bm{\sigma}_d^0$ is indeterminate \citep{guazzelli2011}.  
Hence, our objective is to express these exchange terms, in terms of the continuous phase field quantities instead of the dispersed phase fields, i.e. in terms of $\mathbf{\Phi}_f^0$ and $\textbf{u}_f^0$ rather than $\mathbf{\Phi}_d^0$ and $\textbf{u}_d^0$. 

\subsubsection{On the interfaces}
To address this issue in a general manner, let us derive the conservation equation for the integrated surface property $q_{I\alpha} = \intS{f_I^0}$.
To differentiate time-varying surface integrals within time, we make use of the general Leibniz rule, which states that for an arbitrary function $f_I^0$ defined on $\Gamma(t)$ we have the relation \citep{nadim1996concise}
\begin{equation}
    \ddt  \intS{f_I^0 }
    = \intS{ \left[
        \pddt f_I^0
        +   \gradI \cdot (\textbf{u}_I^0f_I^0)
    \right]}.
    \label{eq:surface_derivative}
\end{equation}
Substituting the right-hand side terms of \ref{eq:surface_derivative} with \ref{eq:dt_f_I}, gives,
\begin{equation}
    \ddt  q_{I\alpha}
    = \intS{ 
        s_I^0
    }
    - \intS{
 \Jump{
        f_k^0 (\textbf{u}_I^0 - \textbf{u}_k^0)
        + \mathbf{\Phi}_k^0
    }
    }.
    \label{eq:dt_q_I_alpha}
\end{equation}
We have used the surface divergence theorem applied to closed surfaces \citep{nadim1996concise}, it reads
\begin{equation}
    \intS{\gradI F}
    = 
    \intS{ F \textbf{n} (\div \textbf{n})},
    \label{eq:gauss_surface}
\end{equation} 
where $F$ is an arbitrary field.
This theorem demonstrates that any surface property parallel to the tangential plane of $\Gamma$, such as $\bm\Phi_{I||}$, satisfies the relation $\intS{\divI \bm\Phi_{I||}^0}
= 0$.
This explains why $\bm\Phi_{I||}$ does not appear in \ref{eq:dt_q_I_alpha}. 
\ref{eq:surface_derivative} can be interpreted as the conservation equation for the integrated surface property $f_I^0$, or as the jump condition of the $f^0_k$ integrated on the droplet surface. 
As discussed above we wish to get rid of $\mathbf{\Phi}_d^0$ in \ref{eq:dt_q_alpha}. 
To achieve this, we treat the particle's volume and surface as a unified entity and derive a conservation equation for $q_\alpha^\text{tot} = q_\alpha + q_{I\alpha}$. 
By summing \ref{eq:dt_q_alpha} and \ref{eq:dt_q_I_alpha} we directly obtain 
\begin{equation}
    \ddt  q_\alpha^\text{tot}
    = 
    \intO{ s_d^0 }
    + \intS{ s_I^0 }
    + \intS{ \left[
        f_f^0 (\textbf{u}_I^0-\textbf{u}_f^0) 
        + \mathbf{\Phi}_f^0 
        \right] \cdot \textbf{n}_d }. 
    \label{eq:dt_q_alpha_tot}
\end{equation}
This equation is the general form of the linear conservation law for the quantity $q_\alpha^\text{tot}$.
It applies to any particle immersed into a continuous phase following the local conservation, \ref{eq:dt_f_k} and \ref{eq:dt_f_I}.
We refer to this equation as the zeroth-order conservation equation or the linear conservation law for the particle $\alpha$.
We would like to highlight that due to the consideration of closed surface, the diffusive flux $\mathbf{\Phi}_{I||}^0$, plays no role at all in \ref{eq:dt_q_alpha_tot}.
Therefore, in the case of the linear momentum conservation law, the contribution of the momentum diffusive flux of surface noted, $\bm\sigma_{I||}^0$, will not contribute to the momentum balance of a particle.
% \begin{equation}
%     \ddt  \textbf{p}_\alpha^\text{tot}
%     = 
%     \intO{ \rho_d^0\textbf{g} }
%     + \intS{ \rho_I^0\textbf{g} }
%     + \intS{ 
%         \left[
%         f_d^0 (\textbf{u}_I^0-\textbf{u}_f^0)
%         + \bm{\sigma}_f^0
%         \right] 
%         \cdot \textbf{n}_d }. 
% \end{equation}
% In this case, note that $\textbf{p}_\alpha^\text{tot} = \intO{\rho^0_d \textbf{u}_d^0}+\intS{\rho^0_I \textbf{u}_I^0}$ is the momentum of the particle's volume and surface. 
% The latter might be negligible if the interface has a negligible weight. 
As a consequence, even in the presence of local Marangoni forces or surface viscous stresses, the resultant of the surface diffusive fluxes would still cancel out in the linear momentum balance.
This fact has already been demonstrated by \citet{hesla1993note} who showed that the surface tension force does not contribute to the linear and angular momentum balance. 
Here, we have provided the general proof that the interfacial diffusive flux $\mathbf{\Phi}_{I||}^0$, which is present at the local scale according to \ref{eq:dt_f_I}, does not contribute to the zeroth-order conservation law of a particle with a closed surface.
Of course, this holds only under the assumption that $\mathbf{\Phi}_{I||}^0$ stays parallel to the surface. 

For completeness, we exposed in \ref{ap:particles_eq} a clear derivation of the mass, momentum and total energy equations for a single particle.
The derivation takes place using the same hypothesis as it is exposed in \ref{ap:hypothesis}.
Especially, it is shown that the integration of the kinetic energy jump condition corresponds to the Lagrangian derivative of the particle surface, see \ref{eq:int_u_I2}. 

\subsection{Higher moment equations}

Because $f_d^0$ and $f_I^0$ are not always constant over the volumes and surfaces of the particles, it is interesting to introduce in the first place, the first moment of the quantities $f_d^0$ and $f_I^0$. 
They are defined as
\begin{align}
    &\textbf{Q}_\alpha 
    = \intO{ \textbf{r} f_d^0 },
    &\text{and}&
    &\textbf{Q}_{I\alpha}
    = \intS{ \textbf{r} f_I^0 },
    \label{eq:first_moment_definition}
\end{align}
where we recall that $\textbf{r} = \textbf{x} - \textbf{x}_\alpha$ is the distance between any point inside $\Omega_\alpha$ or $\Gamma_\alpha$, to the center of mass of the particle $\alpha$.
It is then possible to differentiate these moments with respect to time to obtain their conservation laws.
We use the Reynolds transport theorem (\ref{eq:reynolds_transport}) to describe the evolution of $\textbf{Q}_\alpha$ within time. 
It gives, 
\begin{equation*}
    \frac{d}{dt} \textbf{Q}_\alpha
      =  \intO{\left[
        \pddt(  f_d^0\textbf{r})
        + \div \left(  f_d^0 \textbf{r}\textbf{u}_d^0\right)
    \right]} 
    + \intS{  f_d^0 \textbf{r}  (\textbf{u}_I^0-\textbf{u}_d^0)\cdot \textbf{n}_d}
\end{equation*}
The first term on the right-hand side may be rewritten as
\begin{equation*}
\intO{ \left[
        \pddt(\textbf{r}  f_d^0)+ \div \left( \textbf{u}_d^0 \textbf{r} f^0_d\right) 
    \right]}
    = \intO{\textbf{r}\left[
        \pddt f_d^0
        + \div \left(f_d^0 \textbf{u}_d^0\right)
    \right] }
    + \intO{ f_d^0 \left[
        \pddt \textbf{r}
        +(\textbf{u}_d^0 \cdot \grad) \textbf{r}
    \right]}
\end{equation*}
Using \ref{eq:dt_f_k} for the first integral on the right-hand side, and considering the relation,
$  \pddt \textbf{r}
+ (\textbf{u}_d^0 \cdot \grad) \textbf{r}
= - \frac{d}{dt} \textbf{y}_\alpha  + \textbf{u}_d^0 
= \textbf{w}_d^0$,
for the second integral yields 
\begin{align}
    \frac{d}{dt} \textbf{Q}_\alpha
    % &= \intO{\textbf{r} \left[
    %      s_d^0  +  \div \bm\Phi_d^0
    % \right]}
    % +\intO{f_d^0  \textbf{w}_d }
    % + \int_{\Gamma_\alpha} \textbf{r}  f_d^0 (\textbf{u}_I^0-\textbf{u}_d^0)\cdot \textbf{n}_d  d\Sigma,\\
    = \intO{\left( 
        \textbf{r} s_d^0  
        + f_d^0  \textbf{w}_d 
        - \bm\Phi_d^0
    \right) }
    + \int_{\Gamma_\alpha} \textbf{r} \left[
        \bm\Phi_d^0
        + f_d^0 (\textbf{u}_I^0-\textbf{u}_d^0)
    \right]\cdot \textbf{n}_d  d\Sigma.
    \label{eq:dt_Q_alpha}
\end{align}
Where we have used the relation $\intO{\textbf{r}  \div \bm\Phi_d^0 }
= \intS{ \textbf{r} \bm\Phi_d^0 \cdot \textbf{n}_d }
- \intO{ \bm\Phi_d^0 }$. 
\ref{eq:dt_Q_alpha} is the first order moment conservation equation for the particle $\alpha$. 
Following the same procedure, and making use of \ref{eq:surface_derivative}, \ref{eq:gauss_surface} and \ref{eq:dt_f_I}, one can equally show that 
\begin{align}
    \ddt {\textbf{Q}_{I\alpha}}
    &= \intS{ \left(
        \textbf{r}s_I^0
        + f_I^0 \textbf{w}_I^0
        - \mathbf{\Phi}_{I||}^0
    \right) }
    - \intS{\textbf{r} 
    \Jump{\mathbf{\Phi}_k^0
        + f_k^0 (\textbf{u}_I^0 - \textbf{u}_k^0)
    }
    },
    \label{eq:dt_Q_I_alpha}
\end{align}
where $\textbf{w}_I^0 = \textbf{u}_{I||}^0 - \textbf{u}_\alpha$.
In \ref{eq:dt_Q_alpha}, we recognize the first moment of the source term $s_d^0$, the first moment of the diffusive flux term $\bm\Phi_d^0\cdot\textbf{n}_d$ and the first moment of phase exchange term, $f_d^0 (\textbf{u}_I^0-\textbf{u}_d^0)\cdot\textbf{n}_d$. 
Additionally, two supplementary terms appear in \ref{eq:dt_Q_alpha}, namely: the integral of the diffusive flux $\bm\Phi_d^0$, and a term related to the fluctuation of the internal velocity $f_d^0 \textbf{w}_d^0$.
Similar observations can be made for the first moment of surface equation \ref{eq:dt_Q_I_alpha}, as it shares similarities with \ref{eq:dt_Q_alpha}. 
In particular, it is worth noting the presence of the surface diffusive flux $\mathbf{\Phi}_{I||}^0$ in \ref{eq:dt_Q_I_alpha}.
This term will be further discussed in the following. 

For similar reason than the linear conservation equations, we sum \ref{eq:dt_Q_alpha} and \ref{eq:dt_Q_I_alpha} to expresses the conservation equation of the total first moment $\textbf{Q}_\alpha^\text{tot} = \textbf{Q}_\alpha + \textbf{Q}_{I\alpha}$, this yields 
\begin{multline}
    \ddt {\textbf{Q}_\alpha^\text{tot}}
    = \intO{ \left(
        \textbf{r} s_d^0         
        + f_d^0  \textbf{w}_d^0 
        - \mathbf{\Phi}_d^0
    \right) }
    + \intS{ \left(
        \textbf{r}s_I^0
        + f_I^0 \textbf{w}_I^0
        - \mathbf{\Phi}_{I||}^0
    \right) }
    + \intS{ \textbf{r} \left[
        \mathbf{\Phi}_f^0
        + f_f^0 (\textbf{u}_I^0-\textbf{u}_f^0)
    \right]\cdot \textbf{n}_d  }. 
    \label{eq:dt_Q_alpha_tot}
\end{multline}
Likewise, conservation laws can be derived for the $n^{th}$ order moments of volume and surface, i.e. for
\begin{align}
    \textbf{Q}_{\alpha n}
    = \intO{
         \underbrace{\textbf{rr}\ldots \textbf{rr}}_{n\text{ times}}
        f_d^0 },
        && \text{and} &&
    \textbf{Q}_{I\alpha n}
    = \intS{
         \underbrace{\textbf{rr}\ldots \textbf{rr}}_{n\text{ times}}
    f_I^0 },
    \label{eq:Q_n_definition}
\end{align} 
respectively. 
It can be shown that the derivative with time of $\textbf{Q}_{\alpha n}$ and $\textbf{Q}_{I\alpha n}$ do not involve any additional terms than in \ref{eq:dt_Q_alpha} and \ref{eq:dt_Q_I_alpha}, but rather just the $n^{th}$ order moments of the already presented terms.
We provide the full derivation of $\ddt{ \textbf{Q}_{\alpha n}}$ in \ref{ap:Moments_equations}.
In short, these higher order moments describe the distributions of the local quantities $f_d^0$ and $f_I^0$ inside the domain $\Omega_\alpha$ and $\Gamma_\alpha$, respectively.
Consequently, an infinite number of moments would be theoretically necessary to recover the fields $f_d^0$ and $f_I^0$ within $\Omega_\alpha$ and $\Gamma_\alpha$. 
Thus, one can reach an arbitrary order of accuracy upon the knowledge of an arbitrary number of moments for a given quantity.  

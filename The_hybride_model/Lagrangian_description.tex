Lagrangian-based modeling of the dispersed phase has been explored extensively in numerous studies \citep{buyevich1979flow, lhuillier1992volume, simonin1996, zhang1994averaged, zhang1994ensemble, zhang1997momentum, jackson1997locally, zaepffel2011modelisation}. However, these studies predominantly focus on solid particles \citep{buyevich1979flow, lhuillier1992volume, simonin1996, zhang1994averaged, jackson1997locally} or non-deformable spherical fluid inclusions \citep{zhang1994ensemble, zaepffel2011modelisation}. A notable exception is the work by \citep{zhang1994ensemble}, which considered spherical bubbles with varying radii, although this analysis was limited to constant surface tension and spherical shapes. In this work, we aim to address a more general scenario by considering fluid particles with arbitrary shapes and surface properties. Thus, we propose a Lagrangian-based model for each inclusion capable of describing the dispersed phase with arbitrary accuracy. %In the next section we first define the conservation laws 

 

\subsection{Lagrangian conservation laws}


%To describe the evolution of any arbitrary Lagrangian quantity $q_\alpha$, we need to establish its time derivative.

We assign to a particle indexed, $\alpha$, occupying the domain $\Omega_\alpha$ (see \ref{fig:Scheme}) an arbitrary Lagrangian property $q_\alpha$ defined by $q_\alpha  = \intO{ f_d^0}$. The Reynolds transport theorem for the (material) particle volume can be written as \citep{leal2007},
%Since $q_\alpha$ is an integral quantity with a time-dependent domain of integration, we apply the general Reynolds transport theorem on the material particle volume \citep{leal2007} which can be written as

\begin{equation}
   %\ddt  q_\alpha = 
   \ddt  \intO{f_d^0}
    = \intO{\pddt f_d^0 }\\
    + \intS{ f_d^0 \textbf{u}_I^0\cdot \textbf{n}_d }.
    \label{eq:reynolds_transport0}
\end{equation}
Using the divergence theorem the last equation may be rewritten as
 %for volume integral which gives for material domains, here the particle volume \citep{leal2007},
\begin{equation}
    \ddt  \intO{f_d^0}
    = \intO{\left[ \pddt f_d^0 + \div\left(f_d^0\textbf{u}_d^0\right) \right]}\\
    + \intS{ f_d^0 (\textbf{u}_I^0-\textbf{u}_d^0)\cdot \textbf{n}_d }.
    \label{eq:reynolds_transport}
\end{equation}
By substituting \ref{eq:dt_f_k} into \ref{eq:reynolds_transport} and using the divergence theorem we obtain the conservation law of the quantity $q_\alpha$ %into the integrand of the first integral on the right-hand side (RHS) with  we obtain the conservation law of the quantity $q_\alpha$, namely,  
\begin{equation}
    \frac{d q_\alpha}{d t}
    = \intO{ s_d^0 }
    + \intS{ \left[
        f_d^0 (\textbf{u}_I^0-\textbf{u}_d^0) 
        + \mathbf{\Phi}_d^0 
        \right] \cdot \textbf{n}_d }.
    \label{eq:dt_q_alpha}
\end{equation}
%In \ref{eq:dt_q_alpha} we used the Gauss divergence theorem to show that
%\begin{equation}
%    \intO{\div \mathbf{\Phi}_d^0} = \intS{\mathbf{\Phi}_d^0 \cdot \textbf{n}_d}.
%\end{equation}
The first term on the right-hand side of \ref{eq:dt_q_alpha} accounts for the total contribution of the source term $s_d^0$ to the particle $\alpha$,
while the second term is the surface integration of the phase exhange flux $f_d^0 (\textbf{u}_I^0-\textbf{u}_d^0)$ and the non-convective flux $\mathbf{\Phi}_d^0$.

Similarly, we define $q_{I\alpha} = \intS{ f_I^0}$ as an integrated surface property of the particle $\alpha$.
%To address this issue in a general manner, let us derive the conservation equation for the integrated surface property $q_{I\alpha} = \intS{f_I^0}$.
To differentiate time-varying surface integrals within time, we make use of the general Leibniz rule, which states that for an arbitrary function $f_I^0$ defined on $\Gamma(t)$ we have the relation \citep{nadim1996concise}
\begin{equation}
    \ddt  \intS{f_I^0 }
    = \intS{ \left[
        \pddt f_I^0
        +   \gradI \cdot (\textbf{u}_I^0f_I^0)
    \right]}.
    \label{eq:surface_derivative}
\end{equation}
Substituting the right-hand side terms of \ref{eq:surface_derivative} with \ref{eq:dt_f_I}, gives,
\begin{equation}
    \ddt  q_{I\alpha}
    = \intS{ 
        s_I^0
    }
    - \intS{
 \Jump{
        f_k^0 (\textbf{u}_I^0 - \textbf{u}_k^0)
        + \mathbf{\Phi}_k^0
    }
    }.
    \label{eq:dt_q_I_alpha}
\end{equation}
We have used the surface divergence theorem applied to closed surfaces \citep{nadim1996concise}, it reads
\begin{equation}
    \intS{\gradI F}
    = 
    \intS{ F \textbf{n} (\div \textbf{n})},
    \label{eq:gauss_surface}
\end{equation} 
where $F$ is an arbitrary field.
This theorem demonstrates that any surface property parallel to the tangential plane of $\Gamma$, such as $\bm\Phi_{I||}$, satisfies the relation $\intS{\divI \bm\Phi_{I||}^0}
= 0$.
This explains why $\bm\Phi_{I||}$ does not appear in \ref{eq:dt_q_I_alpha}. 
\ref{eq:surface_derivative} can be interpreted as the conservation equation for the integrated surface property $f_I^0$, or as the jump condition of the $f^0_k$ integrated on the droplet surface. 
As discussed above we wish to get rid of $\mathbf{\Phi}_d^0$ in \ref{eq:dt_q_alpha}. 
To achieve this, we treat the particle's volume and surface as a unified entity and derive a conservation equation for $q_\alpha^\text{tot} = q_\alpha + q_{I\alpha}$. 
By summing \ref{eq:dt_q_alpha} and \ref{eq:dt_q_I_alpha} we directly obtain 
\begin{equation}
    \ddt  q_\alpha^\text{tot}
    = 
    \intO{ s_d^0 }
    + \intS{ s_I^0 }
    + \intS{ \left[
        f_f^0 (\textbf{u}_I^0-\textbf{u}_f^0) 
        + \mathbf{\Phi}_f^0 
        \right] \cdot \textbf{n}_d }. 
    \label{eq:dt_q_alpha_tot}
\end{equation}
This equation is the general form of the linear conservation law for the quantity $q_\alpha^\text{tot}$.
It applies to any particle immersed into a continuous phase following the local conservation, \ref{eq:dt_f_k} and \ref{eq:dt_f_I}.
We refer to this equation as the zeroth-order conservation equation or the linear conservation law for the particle $\alpha$.
We would like to highlight that due to the consideration of closed surface, the diffusive flux $\mathbf{\Phi}_{I||}^0$, plays no role at all in \ref{eq:dt_q_alpha_tot}.
Therefore, in the case of the linear momentum conservation law, the contribution of the momentum diffusive flux of surface noted, $\bm\sigma_{I||}^0$, will not contribute to the momentum balance of a particle.
% \begin{equation}
%     \ddt  \textbf{p}_\alpha^\text{tot}
%     = 
%     \intO{ \rho_d^0\textbf{g} }
%     + \intS{ \rho_I^0\textbf{g} }
%     + \intS{ 
%         \left[
%         f_d^0 (\textbf{u}_I^0-\textbf{u}_f^0)
%         + \bm{\sigma}_f^0
%         \right] 
%         \cdot \textbf{n}_d }. 
% \end{equation}
% In this case, note that $\textbf{p}_\alpha^\text{tot} = \intO{\rho^0_d \textbf{u}_d^0}+\intS{\rho^0_I \textbf{u}_I^0}$ is the momentum of the particle's volume and surface. 
% The latter might be negligible if the interface has a negligible weight. 
As a consequence, even in the presence of local Marangoni forces or surface viscous stresses, the resultant of the surface diffusive fluxes would still cancel out in the linear momentum balance.
This fact has already been demonstrated by \citet{hesla1993note} who showed that the surface tension force does not contribute to the linear and angular momentum balance. 
Here, we have provided the general proof that the interfacial diffusive flux $\mathbf{\Phi}_{I||}^0$, which is present at the local scale according to \ref{eq:dt_f_I}, does not contribute to the zeroth-order conservation law of a particle with a closed surface.
Of course, this holds only under the assumption that $\mathbf{\Phi}_{I||}^0$ stays parallel to the surface.

%Lagrangian based modeling of the dispersed phase has been introduced in numerous paper \citet{buyevich1979flow,lhuillier1992volume,simonin1996,zhang1994averaged,zhang1994ensemble,zhang1997momentum,jackson1997locally,zaepffel2011modelisation}. However in almost all the previsous work only solid particles \citep{buyevich1979flow,lhuillier1992volume,simonin1996,zhang1994averaged,jackson1997locally} or non-deformable spherical fluid inclusion \citep{zhang1994ensemble,zaepffel2011modelisation} were considered. A notable exception is the work of \citep{zhang1994ensemble} where spherical bubble with varying radii were considered. However the analysis of Zhang was limited to constant surface tension and spherical shape. In this work we want to consider the most general case of fluid particles whith arbitrary shape and arbitrary surface properties. Hence we present a lagrangian-based model for each inclusion capable of describing the dispersed phase with an arbitrary order of accuracy. 

%the influence of the surface properties are neglected
%In this section, we present a Lagrangian-based model for eah inclusion capable of describing the dispersed phase with an arbitrary order of accuracy. 




\subsection{Higher order moment equations}

Because $f_d^0$ and $f_I^0$ are not always constant over the volumes and surfaces of the particles, it is interesting to introduce in the first place, the first moment of the quantities $f_d^0$ and $f_I^0$. 
They are defined as
\begin{align}
    &\textbf{Q}_\alpha 
    = \intO{ \textbf{r} f_d^0 },
    &\text{and}&
    &\textbf{Q}_{I\alpha}
    = \intS{ \textbf{r} f_I^0 },
    \label{eq:first_moment_definition}
\end{align}
where we recall that $\textbf{r} = \textbf{x} - \textbf{x}_\alpha$ is the distance between any point inside $\Omega_\alpha$ or $\Gamma_\alpha$, to the center of mass of the particle $\alpha$.
It is then possible to differentiate these moments with respect to time to obtain their conservation laws.
We use the Reynolds transport theorem (\ref{eq:reynolds_transport}) to describe the evolution of $\textbf{Q}_\alpha$ within time. 
It gives, 
\begin{equation*}
    \frac{d}{dt} \textbf{Q}_\alpha
      =  \intO{\left[
        \pddt(  f_d^0\textbf{r})
        + \div \left(  f_d^0 \textbf{r}\textbf{u}_d^0\right)
    \right]} 
    + \intS{  f_d^0 \textbf{r}  (\textbf{u}_I^0-\textbf{u}_d^0)\cdot \textbf{n}_d}
\end{equation*}
The first term on the right-hand side may be rewritten as
\begin{equation*}
\intO{ \left[
        \pddt(\textbf{r}  f_d^0)+ \div \left( \textbf{u}_d^0 \textbf{r} f^0_d\right) 
    \right]}
    = \intO{\textbf{r}\left[
        \pddt f_d^0
        + \div \left(f_d^0 \textbf{u}_d^0\right)
    \right] }
    + \intO{ f_d^0 \left[
        \pddt \textbf{r}
        +(\textbf{u}_d^0 \cdot \grad) \textbf{r}
    \right]}
\end{equation*}
Using \ref{eq:dt_f_k} for the first integral on the right-hand side, and considering the relation,
$  \pddt \textbf{r}
+ (\textbf{u}_d^0 \cdot \grad) \textbf{r}
= - \frac{d}{dt} \textbf{y}_\alpha  + \textbf{u}_d^0 
= \textbf{w}_d^0$,
for the second integral yields 
\begin{align}
    \frac{d}{dt} \textbf{Q}_\alpha
    % &= \intO{\textbf{r} \left[
    %      s_d^0  +  \div \bm\Phi_d^0
    % \right]}
    % +\intO{f_d^0  \textbf{w}_d }
    % + \int_{\Gamma_\alpha} \textbf{r}  f_d^0 (\textbf{u}_I^0-\textbf{u}_d^0)\cdot \textbf{n}_d  d\Sigma,\\
    = \intO{\left( 
        \textbf{r} s_d^0  
        + f_d^0  \textbf{w}_d 
        - \bm\Phi_d^0
    \right) }
    + \int_{\Gamma_\alpha} \textbf{r} \left[
        \bm\Phi_d^0
        + f_d^0 (\textbf{u}_I^0-\textbf{u}_d^0)
    \right]\cdot \textbf{n}_d  d\Sigma.
    \label{eq:dt_Q_alpha}
\end{align}
Where we have used the relation $\intO{\textbf{r}  \div \bm\Phi_d^0 }
= \intS{ \textbf{r} \bm\Phi_d^0 \cdot \textbf{n}_d }
- \intO{ \bm\Phi_d^0 }$. 
\ref{eq:dt_Q_alpha} is the first order moment conservation equation for the particle $\alpha$. 
Following the same procedure, and making use of \ref{eq:surface_derivative}, \ref{eq:gauss_surface} and \ref{eq:dt_f_I}, one can equally show that 
\begin{align}
    \ddt {\textbf{Q}_{I\alpha}}
    &= \intS{ \left(
        \textbf{r}s_I^0
        + f_I^0 \textbf{w}_I^0
        - \mathbf{\Phi}_{I||}^0
    \right) }
    - \intS{\textbf{r} 
    \Jump{\mathbf{\Phi}_k^0
        + f_k^0 (\textbf{u}_I^0 - \textbf{u}_k^0)
    }
    },
    \label{eq:dt_Q_I_alpha}
\end{align}
where $\textbf{w}_I^0 = \textbf{u}_{I||}^0 - \textbf{u}_\alpha$.
In \ref{eq:dt_Q_alpha}, we recognize the first moment of the source term $s_d^0$, the first moment of the diffusive flux term $\bm\Phi_d^0\cdot\textbf{n}_d$ and the first moment of phase exchange term, $f_d^0 (\textbf{u}_I^0-\textbf{u}_d^0)\cdot\textbf{n}_d$. 
Additionally, two supplementary terms appear in \ref{eq:dt_Q_alpha}, namely: the integral of the diffusive flux $\bm\Phi_d^0$, and a term related to the fluctuation of the internal velocity $f_d^0 \textbf{w}_d^0$.
Similar observations can be made for the first moment of surface equation \ref{eq:dt_Q_I_alpha}, as it shares similarities with \ref{eq:dt_Q_alpha}. 
In particular, it is worth noting the presence of the surface diffusive flux $\mathbf{\Phi}_{I||}^0$ in \ref{eq:dt_Q_I_alpha}.
This term will be further discussed in the following. 

For similar reason than the linear conservation equations, we sum \ref{eq:dt_Q_alpha} and \ref{eq:dt_Q_I_alpha} to expresses the conservation equation of the total first moment $\textbf{Q}_\alpha^\text{tot} = \textbf{Q}_\alpha + \textbf{Q}_{I\alpha}$, this yields 
\begin{multline}
    \ddt {\textbf{Q}_\alpha^\text{tot}}
    = \intO{ \left(
        \textbf{r} s_d^0         
        + f_d^0  \textbf{w}_d^0 
        - \mathbf{\Phi}_d^0
    \right) }
    + \intS{ \left(
        \textbf{r}s_I^0
        + f_I^0 \textbf{w}_I^0
        - \mathbf{\Phi}_{I||}^0
    \right) }
    + \intS{ \textbf{r} \left[
        \mathbf{\Phi}_f^0
        + f_f^0 (\textbf{u}_I^0-\textbf{u}_f^0)
    \right]\cdot \textbf{n}_d  }. 
    \label{eq:dt_Q_alpha_tot}
\end{multline}
Likewise, conservation laws can be derived for the $n^{th}$ order moments of volume and surface, i.e. for
\begin{align}
    \textbf{Q}_{\alpha n}
    = \intO{
         \underbrace{\textbf{rr}\ldots \textbf{rr}}_{n\text{ times}}
        f_d^0 },
        && \text{and} &&
    \textbf{Q}_{I\alpha n}
    = \intS{
         \underbrace{\textbf{rr}\ldots \textbf{rr}}_{n\text{ times}}
    f_I^0 },
    \label{eq:Q_n_definition}
\end{align} 
respectively. 
It can be shown that the derivative with time of $\textbf{Q}_{\alpha n}$ and $\textbf{Q}_{I\alpha n}$ do not involve any additional terms than in \ref{eq:dt_Q_alpha} and \ref{eq:dt_Q_I_alpha}, but rather just the $n^{th}$ order moments of the already presented terms.
We provide the full derivation of $\ddt{ \textbf{Q}_{\alpha n}}$ in \ref{ap:Moments_equations}.
In short, these higher order moments describe the distributions of the local quantities $f_d^0$ and $f_I^0$ inside the domain $\Omega_\alpha$ and $\Gamma_\alpha$, respectively.
Consequently, an infinite number of moments would be theoretically necessary to recover the fields $f_d^0$ and $f_I^0$ within $\Omega_\alpha$ and $\Gamma_\alpha$. 
Thus, one can reach an arbitrary order of accuracy upon the knowledge of an arbitrary number of moments for a given quantity.  

In general, the first moments $\textbf{Q}_{\alpha}$ and $\textbf{Q}_{I\alpha}$ hold significant importance when considering particles with high internal gradients, i.e. when $\grad f_d^0$ or $\gradI f_I^0$ are non-negligible at the scale of one particle. 

% \section{Lagrangian description of a single fluid particle}
\label{sec:dispersed-two-fluid}
While \ref{eq:dt_chi_k_f_k} and \ref{eq:dt_delta_I_f_I} describe multiphase-flow in the most general way, they do not take advantage of the topology of the dispersed phase. 
Originally, \citet{jackson1997locally,zhang1994averaged,batchelor1972sedimentation} derived averaged models with the consideration of the topology of the dispersed phase, solely for mono disperse solid spherical particle suspension. 
In the following we present a rigorous derivation of the Lagrangian models when considering arbitrary fluid particles.  

Following the strategy of \citet{paisant2014modelisation,zaepffel2012multisize,zaepffel2011modelisation,lhuillier2010multiphase} and \citet[Chapter 4]{morel2015mathematical}, 
we assign to a particle indexed, $\alpha$, occupying the domain $\Omega_\alpha \subset \Omega_2$, (see \ref{fig:Scheme}) a Lagrangian property $q_\alpha$.
We define $q_\alpha$ as the volumetric property of $f_k$ over the domain $\Omega_\alpha$.
Likewise, we define $q_{I\alpha}$ as being a surface property linked to the local property $f_I$ of the particle $\alpha$, yielding,
\begin{align*}
    &q_\alpha(t)
    = \int_{\Omega_\alpha(t)} f_k(\textbf{y},t) d\Omega,
    &q_{I\alpha}(t)
    = \int_{\Sigma_\alpha(t)} f_I(\textbf{y},t) d\Sigma.
    \label{eq:q_alpha}
\end{align*}
Since, all Lagrangian quantities depend solely on time we discard the argument $t$ in all variables indexed $\alpha$.

For any arbitrary Lagrangian quantity $q_\alpha$ we wish to define its evolution within time.
Then from the general Reynolds transport theorem we deduce that \citep{morel2015mathematical},
% \begin{equation*}
%     \ddt  q_\alpha
%     = \ddt \int_{\Omega_\alpha(t)} f_k d\Omega
%     = \int_{\Omega_\alpha(t)} \pddt f_kd\Omega
%     + \int_{\Sigma_\alpha(t)} f_k \textbf{u}_I \cdot \textbf{n}_k d\Sigma,
% \end{equation*}
\begin{equation}
    \ddt  q_\alpha
    = \int_{\Omega_\alpha(t)}\left[ \pddt f_k + \nablabh \cdot\left(f_k\textbf{u}_k\right) \right]d\Omega\\
    + \int_{\Sigma_\alpha(t)} f_k (\textbf{u}_I-\textbf{u}_k)\cdot \textbf{n}_k d\Sigma,
\end{equation}
where we clearly distinguish that the first term represent the volume integral of the material derivative, and the second term is the surface integral of the flux of $f_k$ across the phase.
By substituting the first term with the local conservation law from \ref{eq:dt_f_k}, it is straightforward to show that
\begin{equation}
    \ddt  q_\alpha
    = \int_{\Omega_\alpha} \textbf{S}_k d\Omega
    + \int_{\Sigma_\alpha} \left[\bm{\Phi}_k + f_k (\textbf{u}_I-\textbf{u}_k) \right] \cdot \textbf{n}_k d\Sigma,
    \label{eq:dt_q_alpha}
\end{equation}
where we used the divergence theorem to transform the non-conservative flux $\bm{\Phi}_k$ volumetric integral to a surface integral.
Likewise, for surface quantities it can be shown that \citep{bothe2022sharp,morel2015mathematical,stone1990simple}, 
\begin{equation}
    \ddt  q_{I\alpha}
    = \int_{\Sigma_\alpha} \left[
        \pddt f_I
        +   \nablabh_{||} \cdot (\textbf{u}_If_I)
    \right]d\Sigma.
    \label{eq:surface_derivative}
\end{equation}
Substituting the RHS of \ref{eq:surface_derivative} using \ref{eq:dt_f_I}, and by making use of the surface divergence theorem on closed surfaces \citep{kanwal1998generalized} on the non-convective term $\mathbf{\Phi}_I$, we can reformulate the RHS terms by, 
\begin{equation}
    \ddt  q_{I\alpha}
    = \int_{\Sigma_\alpha} 
        \textbf{S}_I
    d\Sigma
    - \int_{\Sigma_\alpha} \Jump{
        f_k (\textbf{u}_I - \textbf{u}_k)
        + \mathbf{\Phi}_k
    }
    d\Sigma.
    \label{eq:dt_q_I_alpha}
\end{equation}
We then obtained two Lagrangian balance with \ref{eq:dt_q_alpha} and \ref{eq:dt_q_I_alpha} for respectively the total volumetric and surface properties.
We will refer to these equations as the zeroth order equations, as it provide us solely with the resultant of a local quantity over the volume or the surface of a particle. 
It is interesting to notice that the non-convective term $\mathbf{\Phi}_I$ plays no role at all in \ref{eq:dt_q_I_alpha}.
This is a consequence of the closed surface assumption. 

Before diving in further details it is crucial to define some fundamental quantities of the particles $\alpha$.
We define the position vector of the center of mass of the particle and its momentum as respectively,
\begin{align}
    m_\alpha \textbf{y}_\alpha
    = \int_{\Omega_\alpha} \rho_k \textbf{y}_k d\Omega,
    &&
    \textbf{p}_\alpha 
    = \int_{\Omega_\alpha} \rho_k \textbf{u}_k d\Omega.
    \label{eq:position_and_momentum_def}
\end{align}
Additionally, we define the distance between any points inside $\Omega_\alpha$ and $\textbf{y}_\alpha$ by the vector \textbf{r}, such that, $\textbf{r}(\textbf{y},t) = \textbf{y} - \textbf{y}_\alpha(t)$.
Again, we can notice here, and it will be of major importance in the next derivations, that $\textbf{r}$ is function of space and time.
We then define the particle's center of mass velocity by the derivative within time of its position vector,
\begin{equation}
    \textbf{u}_\alpha = \ddt \textbf{y}_\alpha
    = \frac{1}{m_\alpha} \left(
        \textbf{p}_\alpha
        +  \int_{\Sigma_\alpha} \textbf{r} \rho_k (\textbf{u}_I - \textbf{u}_k)\cdot \textbf{n}_k d\Sigma
        \right),
        \label{eq:dt_y_alpha}
\end{equation}
where we made use of the Reynolds transport theorem and classical rules of derivation, see \ref{ap:velocity_definition}.
Notice that the first component of the RHS of the velocity is the linear momentum divided by the mass of the particle, it can be seen as the volume averaged velocity on $\Omega_\alpha$.
The second term is less intuitive, it results from the contribution of the anisotropic mass transfer across the surface of the particle.
In \citet{zaepffel2011modelisation}, \citet{paisant2014modelisation} and \citet{morel2015mathematical}, they state that the particle's center of mass velocity is $\textbf{u}_\alpha = \textbf{p}_\alpha / m_\alpha$ even though they are considering mass transfer.
It is indeed what we would expect in most of the cases, nevertheless this definition turns out to be not adapted in the presence of anisotropic mass transfer as denoted by \ref{eq:dt_y_alpha}.
Besides, another consequence of \ref{eq:dt_y_alpha} is that regardless of the particle's internal motions, the relevant velocity if we do not consider neglect mass transfer is $\textbf{u}_\alpha = \textbf{p}_\alpha / m_\alpha$.
Lastly, we define the particle's internal motions or the \textit{inner velocity} $\textbf{w}_k(\textbf{y},t)$, such that $\textbf{w}_k(\textbf{y},t) = \textbf{u}_k(\textbf{y}) - \textbf{u}_\alpha(t)$.
Using this velocity decomposition, the momentum definition \ref{eq:position_and_momentum_def}, and after manipulating \ref{eq:dt_y_alpha}, we obtain the following relation for the momentum of a particle,
\begin{equation}
    \textbf{p}_\alpha
    =  m_\alpha \textbf{u}_\alpha
    - \int_{\Sigma_\alpha} \textbf{r} \rho_k (\textbf{u}_I - \textbf{u}_k)\cdot \textbf{n}_k d\Sigma
    = m_\alpha \textbf{u}_\alpha
    + \int_{\Omega_\alpha} \rho_k \textbf{w}_k d\Omega,
    \label{eq:momentum_definition}
\end{equation}
Therefore, the momentum of a particle can be seen as a sum of the mean velocity plus the integral of the fluctuation, with the latter being equivalent to the anisotropic mass transfer term. 

Up to this point we described the particle properties as the volume or surface integral of any local properties $f_I$ or $f_k$.
To better describe the local properties' distribution within the particle's surface or volume we introduce the first moment or dipole of a particle. 
We define the first moment of any property $f_k$ and $f_I$ by respectively,
\begin{align}
    &\textbf{Q}_\alpha 
    = \int_{\Omega_\alpha} \textbf{r} f_k d\Omega,
    &\text{and}&
    &\textbf{Q}_{I\alpha}
    = \int_{\Sigma_\alpha} \textbf{r} f_I d\Sigma.
    \label{eq:first_moment_definition}
\end{align}
We wish to derive the time derivative of $\textbf{Q}_\alpha$ and $\textbf{Q}_{I\alpha}$. 
Considering \ref{eq:dt_f_k}, \ref{eq:dt_f_I} and the Reynolds theorem for volume and surface integral we can show equally that,
\begin{align}
    \ddt \textbf{Q}_\alpha
    &= \int_{\Omega_\alpha} \left( 
        \textbf{r} \textbf{S}_k 
        - \bm{\Phi}_k
        + f_k  \textbf{w}_k 
    \right) d\Omega
    + \int_{\Sigma_\alpha} \textbf{r} \left[
        \bm{\Phi}_k
        + f_k (\textbf{u}_I-\textbf{u}_k)
    \right]\cdot \textbf{n}_k  d\Sigma,
    \label{eq:dt_Q_alpha}\\
    \ddt \textbf{Q}_{I\alpha}
    &= \int_{\Sigma_\alpha} \left(
        \textbf{S}_I\textbf{r}
        - \mathbf{\Phi}_{||}^I
        + f_I \textbf{w}_I
    \right) d\Sigma
    - \int_{\Sigma_\alpha}\textbf{r} \Jump{
        f_k (\textbf{u}_I - \textbf{u}_k)
        + \mathbf{\Phi}_k
    }
    d\Sigma,
    \label{eq:dt_Q_I_alpha}
\end{align}
where $\textbf{w}_I = \textbf{u}_I - \textbf{u}_\alpha$.
Since the derivation of those equations are not trivial, we provide some details in \ref{ap:moment_derivative}. 
In \ref{eq:dt_Q_alpha} we recognize the first moment of the source term $\textbf{S}_k$, the first moment of the non-convective flux term $\mathbf{\Phi}_k\cdot\textbf{n}_k$ and the first moment of phase exchange term, $f_k (\textbf{u}_I-\textbf{u}_k)\cdot\textbf{n}_k$. 
Two supplementary terms appear in this equation, the integral of the non-convective flux $- \int \bm{\Phi}_k d\Omega$ and a term related to the fluctuation of the internal velocity $f_k \textbf{w}_k$.
For \ref{eq:dt_Q_I_alpha} similar remarks can be made as the equation yields comparable terms. 
It is worth mentioning that although the internal fluctuation and the non-convective term do not appear in \ref{eq:dt_q_alpha} and \ref{eq:dt_q_I_alpha} they are present in the first order balance \ref{eq:dt_Q_alpha} and \ref{eq:dt_Q_I_alpha}. 
Consequently, the non-convective surface flux and the fluctuating volume and surface velocities play a role on the properties' distribution (shape, moment of momentum\ldots) but not on the mean property (volume, momentum, energy\ldots). 

Furthermore, it can be shown that the derivative of an arbitrary order moments defined by,
\begin{align*}
    \textbf{Q}_\alpha^n
    = \int_{\Omega_\alpha} \underbrace{
        \textbf{r}\textbf{r}\ldots\textbf{r}
    }_{
        \text{n times}
    }
    f_k d\Omega,
    && \text{and} &&
    \textbf{Q}_{I\alpha}^n
    = \int_{\Sigma_\alpha}
        \textbf{r}\textbf{r}\ldots\textbf{r}
    f_I d\Sigma,
\end{align*} 
do not involve any additional terms, but rather just the $n^{th}$ order moments of the already present terms.
We provide the full derivation of $\textbf{Q}_\alpha^n$ in \ref{ap:Moments_equations}.
In short, these higher order moments describe the distributions of local quantities $f_k$ and $f_I$ inside the domain $\Omega_\alpha$.
As an example, the zeroth order moments are the summation of the distribution, the first order moments measure the mean of the distribution, the second order moments represent the standard deviations of the distribution and the third order moments hold the information of the skewness of the distribution. 
An infinity number of moment would be theoretically sufficient to recover the fields of $f_k$ within $\Omega_\alpha$. 

Up to now we described the dispersed phase within a Lagrangian framework, meaning that we derived, for each particle, a set of equations solely function of time.
However, to be coherent with the Eulerian conservation equations used to describe the continuous phase, we need to extend the Lagrangian equations to Eulerian equations. 
Meaning that we needs to generalize the Lagrangian quantities and Lagrangian derivative to Eulerian feilds. 
In order to accomplish this, we introduce the function $\delta_\alpha$ defined as, 
\begin{align}
    \delta_\alpha(\textbf{y},t) = \delta(\textbf{y}-\textbf{y}_\alpha(t)).
    \label{eq:delta_alpha}
\end{align}
where this function is valid on $\Omega$ and for all time $t$\citep[Chapter 2]{morel2015mathematical}. 
Then multiplying each Lagrangian quantities by \ref{eq:delta_alpha} yields the \textit{particular} field, $q_\alpha(t)\delta_\alpha(\textbf{y},t)$, which is valid through space and time.
By noticing that $\delta_\alpha(\textbf{x}_\alpha,t) = 1$ independently of $t$, it can be demonstrated that the convective derivative of the function $\delta_\alpha(\textbf{y},t)$ is nil and yields, 
\begin{equation}
    \pddt \delta_\alpha
    + \nablabh \cdot (\textbf{u}_\alpha  \delta_\alpha)
    =0,
    \label{eq:dt_delta_alpha}
\end{equation}
where we included $\textbf{u}_\alpha$ in the divergence operator since we recall that it is a function of time.
Besides, it must be said that \ref{eq:dt_delta_alpha} isn't valid if changes in topology such as break up or coalescence events.
Indeed, in those cases we must add a source terms at the RHS of \ref{eq:dt_delta_alpha} to account for particle birth or death. 
Similarly, for any derivative of Lagrangian quantity we define its related field, i.e. $\delta_\alpha \ddt q_\alpha$, and we show that,
\begin{equation}
    \delta_\alpha \ddt q_\alpha
    = \pddt (\delta_\alpha q_\alpha)
    + \nablabh \cdot (\delta_\alpha q_\alpha \textbf{u}_\alpha)
    \label{eq:dt_delta_alpha_q_alpha}
\end{equation}
where we use the fact that $q_\alpha(t)$ and $\textbf{u}_\alpha(t)$ were solely function of time and made use of \ref{eq:dt_delta_alpha}.
Now let's consider a volume containing $N$ particles, we define the \textit{particular} field of a given quantity, $q_\alpha$, as the sum of all the independent field, i.e. $\sum_\alpha \delta_\alpha q_\alpha$.
Notice that \ref{eq:dt_delta_alpha_q_alpha} remains valid for a sum of fields since derivative operators are linear.
However, as the summation notation can become quite cumbersome, we choose to consider the summation over all particles included in $\Omega$ implicitly whenever a Lagrangian property denoted by the subscript $_\alpha$ is present.
Then, Multiplying \ref{eq:dt_q_alpha} and \ref{eq:dt_q_I_alpha} by $\delta_\alpha$, and by considering \ref{eq:dt_delta_alpha_q_alpha} it is trivial to show that,
\begin{align}
    \pddt (\delta_\alpha q_\alpha)
    + \nablabh \cdot (\delta_\alpha q_\alpha \textbf{u}_\alpha)
    &= \delta_\alpha\int_{\Omega_\alpha} \textbf{S}_k d\Omega
    + \delta_\alpha\int_{\Sigma_\alpha} \left[\bm{\Phi}_k + f_k (\textbf{u}_I-\textbf{u}_k) \right] \cdot \textbf{n}_k d\Sigma,
    \label{eq:dt_dq_alpha}\\
    \pddt (\delta_\alpha q_{I\alpha})
    + \nablabh \cdot(\delta_\alpha q_{I\alpha} \textbf{u}_\alpha)
    &= \delta_\alpha\int_{\Sigma_\alpha} 
        \textbf{S}_I
    d\Sigma
    - \delta_\alpha\int_{\Sigma_\alpha} \Jump{
        f_k (\textbf{u}_I - \textbf{u}_k)
        + \mathbf{\Phi}_k
    }
    d\Sigma.
    \label{eq:dt_dq_I_alpha}
\end{align}
Where we empathize that we implicitly sum over all the particle $\alpha$ included in $\Omega$. 
Similar consideration can be applied to \ref{eq:dt_Q_alpha} and \ref{eq:dt_Q_I_alpha}.
Therefore, we obtained Eulerian equations of conservation of the particular field, for any Lagrangian property of the particles $q_\alpha$\;$q_{I\alpha}$, $Q_\alpha$ and $Q_{I\alpha}$ and any other higher moments. 

At this stage of the derivation there is two set of equation that can be used to solve for the dispersed phase. 
The first one is to use \ref{eq:dt_chi_k_f_k} and \ref{eq:dt_delta_I_f_I} with $k =2$.
The other way is to use \ref{eq:dt_dq_alpha} and \ref{eq:dt_dq_I_alpha} and possibly the higher moments equations.
Therefore, some comments are in order on the differences and compatibility of these two set of equations.
First, solving \ref{eq:dt_dq_alpha} provides us with a field $q_\alpha\delta_\alpha$ which contains the Lagrangian properties of the particles $q_\alpha$, which correspond to the volume integral of $f_k$ within the domains $\Omega_\alpha$.
While, in \ref{eq:dt_chi_k_f_k} we solve the equation for the field $f_k$ over $\omega$ therefore also within the volume of the particle contained in $\Omega_\alpha$.
Thus, from  \ref{eq:dt_f_k} to \ref{eq:avg_dt_dq_alpha} we lose the detailed description of the field $f_k$ within the particle, to recover solely the integrated value or the zeroth order moment of the distribution of $f_k$ in $\Omega_\alpha$, namely $q_\alpha$. 
In other words we reduced the $\alpha$'s PDF defined in each $\Omega_\alpha$ to the quantities $q_\alpha$, $Q_\alpha$ and the higher moments.  
The same comments can be made on the surface equations when comparing \ref{eq:avg_dt_dq_I_alpha} with \ref{eq:dt_f_I}. 
Therefore, \ref{eq:dt_dq_alpha} and \ref{eq:dt_dq_I_alpha} can be though as the averaged equations of respectively, \ref{eq:dt_chi_k_f_k} and  \ref{eq:dt_delta_I_f_I} where we recover only the averaged values of each particle. 
It is important to understand that in this sense, the average from \ref{eq:dt_chi_k_f_k} to \ref{eq:dt_dq_alpha} is carried out over the particle length scale.
Unlike the usual averaged technics that refer to the ones used to derive the classic averaged models \citep{jackson1997locally,zhang1994averaged}, which are the subject of the following section. 



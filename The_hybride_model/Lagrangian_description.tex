\section{Lagrangian equations for the dispersed phase}
\label{sec:Lagrangian}

Lagrangian-based modeling of the dispersed phase has been explored extensively in numerous studies \citep{buyevich1979flow, lhuillier1992volume, simonin1996, zhang1994averaged, zhang1994ensemble, zhang1997momentum, jackson1997locally, zaepffel2011modelisation}. 
However, these studies predominantly focus on solid particles \citep{buyevich1979flow, lhuillier1992volume, simonin1996, zhang1994averaged, jackson1997locally} or non-deformable spherical fluid inclusions \citep{zhang1994ensemble, zaepffel2011modelisation}. 
%A notable exception is the work by \citep{zhang1994ensemble}, which considered spherical bubbles with varying radii, although this analysis was limited to constant surface tension and spherical shapes. 
In this work, we aim to address a more general scenario by considering fluid particles with arbitrary shapes and surface properties. Thus, we propose a Lagrangian-based model for each inclusion capable of describing the dispersed phase with arbitrary accuracy. %In the next section we first define the conservation laws 

 \subsection{Fundamental properties of the dispersed phase}
 %Before defining the higher order moments we define some fundamental properties of the particles. 
We define the mass $m_\alpha$, the position of center of mass $\mathbf{x}_\alpha$ and the momentum $\textbf{p}_\alpha$ as
\begin{align}
    m_\alpha(t,\FF)
    &= \intOF{ \rho_d  }, 
    \\
    \textbf{x}_\alpha(t,\FF)
    &= \frac{1}{m_\alpha }\intOF{ \rho_d \textbf{x} }     \label{eq:mass_pos0},
    \\ \textbf{p}_\alpha(t,\FF)
    &= \intOF{ \rho_d \textbf{u}_d^0 }.
\end{align}
% $\Omega_\alpha(t,\FF)$ is the time-dependent domain occupied by the particle $\alpha$ (see \ref{fig:Scheme}). 
Note that while the mass of the interface might be considered in the center of mass calculation, in practice, the mass within the volume is typically much greater than the mass on the interface. 
Therefore, we have opted to use the more straightforward definition considering only the mass within the particle volume.
Subsequently, we define the velocity of the particle center of mass as
\begin{equation}
\textbf{u}_\alpha(t,\FF) = \frac{d \textbf{x}_\alpha}{dt}.
\label{eq:u_alpha}
\end{equation}
We also define the particle's internal relative motions or the \textit{inner velocity}  as $\textbf{w}_d^0 = \textbf{u}_d^0 - \textbf{u}_\alpha$. 
Similarly we define $\textbf{w}_I^0$ as $\textbf{w}_I^0 = \textbf{u}_{I||}^0 - \textbf{u}_\alpha$.
Replacing $\textbf{x}_\alpha$ by its definition (\ref{eq:mass_pos0}) in \ref{eq:u_alpha} we obtain
\begin{equation}
    \textbf{u}_\alpha(t,\FF) = \frac{1}{m_\alpha}
    \frac{d}{dt} 
    \left(
        \intOF{ \rho_d \textbf{x} }
    \right)
    - \frac{1}{m_\alpha^2} \frac{d}{dt} \left(\intOF{ \rho_d } \right)
    \intOF{ \rho_d \textbf{x} }.
\end{equation}
%\tb{ A finaliser
Using \ref{eq:reynolds_transport} for both terms in parentheses and the definition of $\textbf{x}_\alpha(t,\FF)$ in the last term gives
\begin{align}
    \textbf{u}_\alpha(t,\FF) &=  \frac{1}{m_\alpha}\intOF{
        \pddt (\textbf{x}\rho_d ) + \div\left(\textbf{u}_d^0 \textbf{x} \rho_d\right) 
    } 
    + \frac{1}{m_\alpha}\intSF{ \textbf{x} \rho_d(\textbf{u}_I - \textbf{u}_d^0) \cdot \textbf{n}_d } \nonumber\\
    & - \frac{1}{m_\alpha^2}\intOF{
        \pddt (\rho_d ) + \div\left(\textbf{u}_d^0 \rho_d\right) 
    } -  \frac{\textbf{x}_\alpha}{m_\alpha}    \intSF{ \rho_d(\textbf{u}_I   - \textbf{u}_d^0) \cdot \textbf{n}_d }
    \label{eq:u_alpha2}
\end{align}
Making use of the conservation of mass, the first term on the second line of \ref{eq:u_alpha2} vanishes. This yields,%(\ref{eq:dt_rho}) and the definition of $\textbf{x}_\alpha(t,\FF)$ in the last term, gives
\begin{multline}
    \textbf{u}_\alpha(t,\FF) = 
    \frac{1}{m_\alpha}\intOF{ \left[
        \pddt (\textbf{x}\rho_d ) + \div\left(\textbf{u}_d^0 \textbf{x} \rho_d\right) 
    \right]} \\
    + \frac{1}{m_\alpha}\intSF{ \textbf{x} \rho_d(\textbf{u}_\Gamma^0   - \textbf{u}_d^0) \cdot \textbf{n}_d }
    -  \frac{\textbf{x}_\alpha}{m_\alpha}    \intSF{ \rho_d(\textbf{u}_\Gamma^0   - \textbf{u}_d^0) \cdot \textbf{n}_d }
\end{multline}
Then by considering the mass conservation for the first term and noticing that $\grad \textbf{x} = \bm\delta$ gives%for the second term gives, 
\begin{equation}
    \textbf{u}_\alpha(t,\FF) = \frac{1}{m_\alpha(t,\FF)} \left(
        \textbf{p}_\alpha(t,\FF)
        +  \intSF{\rho_d \textbf{r} (\textbf{u}_\Gamma^0 - \textbf{u}_d^0)\cdot \textbf{n}_d }
        \right),
        \label{eq:dt_y_alpha}
\end{equation}
where $\textbf{r}(\textbf{x},t,\FF) = \textbf{x} - \textbf{x}_\alpha(t,\FF)$ is the distance between any point inside $\Omega_\alpha$, to the center of mass of the particle $\alpha$.
In \ref{eq:dt_y_alpha}, it can be observed that the first component of the velocity represents the linear momentum divided by the mass of the particle. 
This corresponds to the mass-averaged velocity over the volume of the particle.
The second term in \ref{eq:dt_y_alpha} arises from the contribution of anisotropic mass transfer across the surface of the particle. 
This mass transfer leads to the motion of the particle's center of mass, thereby contributing to the total velocity.
To illustrate this concept, let us consider a fixed drop with no momentum lying over a very hot plate.
In this scenario, we assume that the plate is sufficiently hot to induce evaporation, specifically on the bottom portion of the drop.
Hence, under the effect of an anisotropic evaporation flux one may expect the second term to be non-negligible.
Consequently, the center of mass of the drop has a non-zero velocity in the opposite direction of the plate, even though the momentum is assumed to be zero.
Note that \ref{eq:dt_y_alpha} generalized usual expression of the center of mass velocity whom neglect the second term.
In the following, for the sake of brevity we discard the dependency on $t$ and $\FF$ on the notations for all Lagrangian quantities denoted by the subscript $_\alpha$ and in particular $\Gamma_\alpha$ and $\Omega_\alpha$.
Nevertheless, the reader must understand that all Lagrangian quantities and integration domains subscribed by $_\alpha$ are time and configuration-dependent. 



\subsection{Lagrangian conservation laws}


%To describe the evolution of any arbitrary Lagrangian quantity $\text q_\alpha$, we need to establish its time derivative.

We assign to a particle indexed, $\alpha$, occupying the domain $\Omega_\alpha$ (see \ref{fig:Scheme}) an arbitrary Lagrangian property $\text q_\alpha$ defined by $\text q_\alpha  = \intO{ f_d^0}$. The Reynolds transport theorem for the (material) particle volume can be written as \citep{leal2007advanced},
%Since $\text q_\alpha$ is an integral quantity with a time-dependent domain of integration, we apply the general Reynolds transport theorem on the material particle volume \citep{leal2007} which can be written as

\begin{equation}
   %\ddt  \text q_\alpha = 
   \ddt  \intO{f_d^0}
    = \intO{\pddt f_d^0 }\\
    + \intS{ f_d^0 \textbf{u}_\Gamma^0\cdot \textbf{n}_d }.
    \label{eq:reynolds_transport0}
\end{equation}
Adding and subtracting $f_d^0 \textbf{u}_d^0\cdot \textbf{n}_d$ in the last integral of this equation and using the divergence theorem, yields
 %for volume integral which gives for material domains, here the particle volume \citep{leal2007},
\begin{equation}
    \ddt  \intO{f_d^0}
    = \intO{\left[ \pddt f_d^0 + \div\left(f_d^0\textbf{u}_d^0\right) \right]}\\
    + \intS{ f_d^0 (\textbf{u}_\Gamma^0-\textbf{u}_d^0)\cdot \textbf{n}_d }.
    \label{eq:reynolds_transport}
\end{equation}
By substituting \ref{eq:dt_f_k} into \ref{eq:reynolds_transport} and using the divergence theorem we obtain the conservation law of the quantity $\text q_\alpha$ %into the integrand of the first integral on the right-hand side (RHS) with  we obtain the conservation law of the quantity $\text q_\alpha$, namely,  
\begin{equation}
    \frac{d \text q_\alpha}{d t}
    = \intO{ s_d^0 }
    + \intS{ \left[
        f_d^0 (\textbf{u}_\Gamma^0-\textbf{u}_d^0) 
        + \mathbf{\Phi}_d^0 
        \right] \cdot \textbf{n}_d }.
    \label{eq:dt_q_alpha}
\end{equation}
%In \ref{eq:dt_q_alpha} we used the Gauss divergence theorem to show that
%\begin{equation}
%    \intO{\div \mathbf{\Phi}_d^0} = \intS{\mathbf{\Phi}_d^0 \cdot \textbf{n}_d}.
%\end{equation}
The first term on the right-hand side of \ref{eq:dt_q_alpha} accounts for the total contribution of the source term $s_d^0$ to the particle $\alpha$,
while the second term is the surface integration of the phase exhange flux $f_d^0 (\textbf{u}_\Gamma^0-\textbf{u}_d^0)$ and the non-convective flux $\mathbf{\Phi}_d^0$.

Similarly, we define $\text q_{\Gamma\alpha} = \intS{ f_\Gamma^0}$ as an integrated surface property of the particle $\alpha$.
%To address this issue in a general manner, let us derive the conservation equation for the integrated surface property $\text q_{\Gamma\alpha} = \intS{f_\Gamma^0}$.
To differentiate time-varying surface integrals with respect to time, we make use of the surface Reynolds transport theorem, which reads \citep[Appendix B]{morel2015mathematical} %which states that for an arbitrary function $f_\Gamma^0$ defined on $\Gamma(t)$ the following relation holds \citep{nadim1996concise}
\begin{equation}
    \ddt  \intS{f_\Gamma^0 }
    = \intS{ \left[
        \frac{D_\Gamma f_\Gamma^0}{Dt}+
        +   f_\Gamma^0\gradI \cdot \textbf{u}_\Gamma^0
    \right]}.
    \label{eq:surface_derivative}
\end{equation}
where $\frac{D_\Gamma}{Dt}  = \pddt + \textbf{u}_\Gamma^0\cdot \gradI $ is the material derivative operator on the surface of the particle. 
Upon expanding the acceleration term we obtain the expression,
\begin{equation}
    \ddt  \intS{f_\Gamma^0 }
    = \intS{ \left[
        \pddt f_\Gamma^0
        +   \gradI \cdot (\textbf{u}_\Gamma^0f_\Gamma^0)
    \right]}.
    \label{eq:surface_derivative}
\end{equation}
Then inserting \ref{eq:dt_f_I} into \ref{eq:surface_derivative} we obtain
%Substituting the right-hand side terms of \ref{eq:surface_derivative} with \ref{eq:dt_f_I}, gives,
\begin{equation}
    \ddt  \text q_{\Gamma\alpha}
    = \intS{ 
        \gradI \cdot \boldsymbol{\Phi}_{\Gamma||}^0
    }
    +\intS{ 
        s_\Gamma^0
    }
    - \intS{
 \Jump{
        f_k^0 (\textbf{u}_\Gamma^0 - \textbf{u}_k^0)
        + \mathbf{\Phi}_k^0
    }
    }.
    \label{eq:dt_q_I_alpha}
\end{equation}
The surface divergence theorem applied to closed surfaces  reads \citep{nadim1996concise}
\begin{equation}
    \intS{\gradI F}
    = 
    \intS{ F \cdot\textbf{n} (\div \textbf{n})},
    \label{eq:gauss_surface}
\end{equation} 
where $F$ is an arbitrary field.
This theorem demonstrates that any surface property parallel to the tangential plane of $\Gamma$, such as $\bm\Phi_{I||}$, fulfills the condition $\intS{\divI \bm\Phi_{I||}^0}
= 0$.
Therefore \ref{eq:dt_q_I_alpha} yields,%explains why $\bm\Phi_{I||}$ does not appear in \ref{eq:dt_q_I_alpha}. 

\begin{equation}
    \ddt  \text q_{\Gamma\alpha}
    = \intS{ 
        s_\Gamma^0
    }
    - \intS{
 \Jump{
        f_k^0 (\textbf{u}_\Gamma^0 - \textbf{u}_k^0)
        + \mathbf{\Phi}_k^0
    }
    }.
    \label{eq:dt_q_I_alpha}
\end{equation}
%\ref{eq:surface_derivative} can be interpreted as the conservation equation for the integrated surface property $f_\Gamma^0$, or as the jump condition of the $f^0_k$ integrated on the droplet surface. 
%As discussed above we wish to get rid of $\mathbf{\Phi}_d^0$ in \ref{eq:dt_q_alpha}. 
In \ref{eq:dt_q_alpha}, the non-convective flux within the dispersed phase appears on the right hand side. While this is not problematic for fluid particles for which the internal stress is defined, the stress within rigid particles is not defined \citep{batchelor1970stress}.
Hence to remove the internal stress from \ref{eq:dt_q_alpha}, we treat the particle's volume and surface as a single entity and derive a conservation equation for $\text Q_\alpha = \text q_\alpha + \text q_{\Gamma\alpha}$. 
By summing \ref{eq:dt_q_alpha} and \ref{eq:dt_q_I_alpha} we directly obtain 
\begin{equation}
    \ddt  \text Q_\alpha
    = 
    \intO{ s_d^0 }
    + \intS{ s_\Gamma^0 }
    + \intS{ \left[
        f_f^0 (\textbf{u}_\Gamma^0-\textbf{u}_f^0) 
        + \mathbf{\Phi}_f^0 
        \right] \cdot \textbf{n}_d }. 
    \label{eq:dt_q_alpha_tot}
\end{equation}
This equation is the general form of the linear conservation law for the quantity $\text Q_\alpha$.
It applies to any particle immersed in a continuous phase following the conservation laws given by \ref{eq:dt_f_k} and \ref{eq:dt_f_I}.
We refer to this equation as the zeroth-order conservation equation of $f_d^0$, or alternatively, the linear conservation law of $f_d^0$, for the particle $\alpha$.
We would like to highlight that due to the consideration of closed surface, the non-convective flux $\mathbf{\Phi}_{\Gamma||}^0$, does not appear in \ref{eq:dt_q_alpha_tot}.
Consequently, for the conservation of linear momentum, the surface stresses do not contribute to the momentum balance of a particle.
%Therefore, in the case of the linear momentum conservation law, the contribution of the surface non-convective flux will not contribute to the momentum balance of a particle.
% \begin{equation}
%     \ddt  \textbf{p}_\alpha^\text{tot}
%     = 
%     \intO{ \rho_d^0\textbf{g} }
%     + \intS{ \rho_I^0\textbf{g} }
%     + \intS{ 
%         \left[
%         f_d^0 (\textbf{u}_\Gamma^0-\textbf{u}_f^0)
%         + \bm{\sigma}_f^0
%         \right] 
%         \cdot \textbf{n}_d }. 
% \end{equation}
% In this case, note that $\textbf{p}_\alpha^\text{tot} = \intO{\rho^0_d \textbf{u}_d^0}+\intS{\rho^0_I \textbf{u}_\Gamma^0}$ is the momentum of the particle's volume and surface. 
% The latter might be negligible if the interface has a negligible weight. 
%As a result, even if local Marangoni forces or surface viscous stresses are present, their overall effect will cancel out in the balance of linear momentum.
%As a consequence, even in the presence of local Marangoni forces or surface viscous stresses, the resultant of this stress would cancel out in the linear momentum balance.
As a result, surface tension or surface viscous stresses do not directly affect the linear momentum balance \footnote{However, Marangoni effect can induce motion through the imbalance of tangential stress \citep{young1959}. Indeed the resulting imbalance alters the stress within the continuous phase, generating a hydrodynamic force.}. 
This property has already been demonstrated by \citet{hesla1993note} who showed that the surface tension force does not contribute to the linear and angular momentum balance. 
Here, we have provided the general proof that the interfacial non-convective flux $\mathbf{\Phi}_{\Gamma||}^0$, which is present at the local scale according to \ref{eq:dt_f_I}, does not contribute to the zeroth-order conservation law of a particle with a closed surface.
It is important to note that this conclusion is valid only if the non-convective flux is parallel to the interface, which is what we assumed here.
%Of course, this holds only under the assumption that the non-convective flux is parallel to the interface.%$\mathbf{\Phi}_{\Gamma||}^0$ is parallel to the surface.

%Lagrangian based modeling of the dispersed phase has been introduced in numerous paper \citet{buyevich1979flow,lhuillier1992volume,simonin1996,zhang1994averaged,zhang1994ensemble,zhang1997momentum,jackson1997locally,zaepffel2011modelisation}. However in almost all the previsous work only solid particles \citep{buyevich1979flow,lhuillier1992volume,simonin1996,zhang1994averaged,jackson1997locally} or non-deformable spherical fluid inclusion \citep{zhang1994ensemble,zaepffel2011modelisation} were considered. A notable exception is the work of \citep{zhang1994ensemble} where spherical bubble with varying radii were considered. However the analysis of Zhang was limited to constant surface tension and spherical shape. In this work we want to consider the most general case of fluid particles whith arbitrary shape and arbitrary surface properties. Hence we present a lagrangian-based model for each inclusion capable of describing the dispersed phase with an arbitrary order of accuracy. 

%the influence of the surface properties are neglected
%In this section, we present a Lagrangian-based model for eah inclusion capable of describing the dispersed phase with an arbitrary order of accuracy. 




\subsection{Higher order moment equations}
%At this stage, we define some fundamental properties associated to each particle labeled $\alpha$.
%Following the strategy of \citet{lhuillier2009rheology,lhuillier1992volume,zaepffel2011modelisation} and \citet[Chapter 2]{morel2015mathematical}
%we define the mass $m_\alpha$, position of center of mass $\mathbf{x}_\alpha$, the momentum $\textbf{p}_\alpha$ and the total energy $E_\alpha$ of the particle $\alpha$, as

Because $f_d^0$ and $f_\Gamma^0$ are not always constant over the volumes and surfaces of the particles, it is interesting to introduce the first moment of the quantities $f_d^0$ and $f_\Gamma^0$. 
They are defined as $\textbf{q}_\alpha^{(1)}     = \intO{ \textbf{r} f_d^0 }$ and $\textbf{q}_{\Gamma\alpha}^{(1)}    = \intS{ \textbf{r} f_\Gamma^0 }$.  
%, where
%\begin{align}
%    &\textbf{Q}_\alpha 
%    = \intO{ \textbf{r} f_d^0 },
%    &\text{and}&
%    &\textbf{Q}_{\Gamma\alpha}
%    = \intS{ \textbf{r} f_\Gamma^0 },
%    \label{eq:first_moment_definition}
%\end{align}
%$\textbf{r} = \textbf{x} - \textbf{x}_\alpha$ is the distance between any point inside $\Omega_\alpha$ or $\Gamma_\alpha$, to the center of mass of the particle $\alpha$ which is defined as
 %\begin{equation}
%    \textbf{x}_\alpha
%    = \frac{1}{m_\alpha }\intO{ \rho_d \textbf{x} }
% \end{equation}   
% In the previous equation, we have denoted $m_\alpha$ as the average mass of the inclusion, which is defined as
%\begin{equation}
%m_\alpha= \intO{ \rho_d  }.
%\end{equation}   
 %Note that while the mass of the interface might be considered in the center of mass calculation, in practice, the mass within the volume is typically much greater than the mass on the interface. 
 %Therefore, we have opted to use the more straightforward definition considering only the mass within the particle volume.
 %Note that we may have have taken into account in the center of mass definition the mass of the interface. 
 %However in practice except in the rare case of sop bubbles for instance the mass within the volume is much greater than the mass o,n the interface and we have choosen to keep the most classical definition.
In general, the first moments $\textbf{q}^{(1)}_{\alpha}$ and $\textbf{q}^{(1)}_{\Gamma\alpha}$ hold significant importance when considering particles with high internal gradients, i.e. when $\grad f_d^0$ or $\gradI f_\Gamma^0$ are non-negligible at the scale of one particle. 
Typically, when considering the motion of a fiber, one must consider the angular momentum balance \citep{guazzelli2011}, which corresponds to the antisymmetric part of the first moment of momentum.
It is possible to differentiate the moments $\textbf{q}^{(1)}_\alpha$ and $\textbf{q}^{(1)}_{\Gamma\alpha}$  with respect to time to obtain their conservation laws.
We use \ref{eq:reynolds_transport} to describe the evolution of $\textbf{q}^{(1)}_\alpha$ within time. This yields, 
\begin{equation}
    \frac{d}{dt} \textbf{q}^{(1)}_\alpha
      =  \intO{\left[
        \pddt(  f_d^0\textbf{r})
        + \div \left(  f_d^0 \textbf{r}\textbf{u}_d^0\right)
    \right]} 
    + \intS{  f_d^0 \textbf{r}  (\textbf{u}_\Gamma^0-\textbf{u}_d^0)\cdot \textbf{n}_d}.
\end{equation}
The first term on the right-hand side may be rewritten as
\begin{equation}
\intO{ \left[
        \pddt(\textbf{r}  f_d^0)+ \div \left( \textbf{u}_d^0 \textbf{r} f^0_d\right) 
    \right]}
    = \intO{\textbf{r}\left[
        \pddt f_d^0
        + \div \left(f_d^0 \textbf{u}_d^0\right)
    \right] }
    + \intO{ f_d^0 \left[
        \pddt \textbf{r}
        +(\textbf{u}_d^0 \cdot \grad) \textbf{r}
    \right]}
\end{equation}
Substituting \ref{eq:dt_f_k} in the first integral on the right-hand side, and considering the relation,
$  \pddt \textbf{r}
+ (\textbf{u}_d^0 \cdot \grad) \textbf{r}
= - \frac{d}{dt} \textbf{x}_\alpha  + \textbf{u}_d^0 
= \textbf{w}_d^0$,
for the second integral yields 
\begin{align}
    \frac{d}{dt} \textbf{q}^{(1)}_\alpha = \intO{\textbf{r} (s_d^0 +\div \bm\Phi_d^0)  }
    + \intO{ f_d^0\textbf{w}_d^0 }  + \intS{  f_d^0 \textbf{r}  (\textbf{u}_\Gamma^0-\textbf{u}_d^0)\cdot \textbf{n}_d}
\end{align}
Then using relation $\intO{\textbf{r}  \div \bm\Phi_d^0 }
= \intS{ \textbf{r} \bm\Phi_d^0 \cdot \textbf{n}_d }
- \intO{ \bm\Phi_d^0 }$ we get
\begin{align}
    \frac{d}{dt} \textbf{q}^{(1)}_\alpha
    % &= \intO{\textbf{r} \left[
    %      s_d^0  +  \div \bm\Phi_d^0
    % \right]}
    % +\intO{f_d^0  \textbf{w}_d }
    % + \int_{\Gamma_\alpha} \textbf{r}  f_d^0 (\textbf{u}_\Gamma^0-\textbf{u}_d^0)\cdot \textbf{n}_d  d\Sigma,\\
    = \intO{\left( 
        \textbf{r} s_d^0  
        + f_d^0  \textbf{w}_d^0
        - \bm\Phi_d^0
    \right) }
    + \int_{\Gamma_\alpha} \textbf{r} \left[
        \bm\Phi_d^0
        + f_d^0 (\textbf{u}_\Gamma^0-\textbf{u}_d^0)
    \right]\cdot \textbf{n}_d  d\Sigma.
    \label{eq:dt_Q_alpha}
\end{align}
 \ref{eq:dt_Q_alpha} is the first order moment conservation equation for the particle $\alpha$. 
 In \ref{eq:dt_Q_alpha}, we recognize the first moment of the source term $s_d^0$, the first moment of the non convective flux term $\bm\Phi_d^0\cdot\textbf{n}_d$ and the first moment of phase exchange term, $f_d^0 (\textbf{u}_\Gamma^0-\textbf{u}_d^0)\cdot\textbf{n}_d$. 
 Additionally, two supplementary terms appear in \ref{eq:dt_Q_alpha}, namely: the volume integral of the non convective flux $\bm\Phi_d^0$, and a term related to the fluctuation of the internal velocity $f_d^0 \textbf{w}_d^0$.

Following the same procedure, and making use of \ref{eq:dt_f_I}, \ref{eq:surface_derivative} and \ref{eq:gauss_surface} , one can equally show that 
\begin{align}
    \ddt {\textbf{q}^{(1)}_{\Gamma\alpha}}
    &= \intS{ \left(
        \textbf{r}s_\Gamma^0
        + f_\Gamma^0 \textbf{w}_\Gamma^0
        - \mathbf{\Phi}_{||\Gamma}^0
    \right) }
    - \intS{\textbf{r} 
    \Jump{\mathbf{\Phi}_k^0
        + f_k^0 (\textbf{u}_\Gamma^0 - \textbf{u}_k^0)
    }
    },
    \label{eq:dt_Q_I_alpha}
\end{align}
Observations similar to those for \ref{eq:dt_Q_alpha} can also be made for the first moment of surface equation \ref{eq:dt_Q_I_alpha}.
%Similar observations can be made for the first moment of surface equation \ref{eq:dt_Q_I_alpha}, as it shares similarities with \ref{eq:dt_Q_alpha}. 
In particular, it is worth noting the presence of the surface non-convective flux $\mathbf{\Phi}_{\Gamma||}^0$ in \ref{eq:dt_Q_I_alpha}.
%This term will be further discussed in the following. 

For similar reason than the linear conservation equations, we sum \ref{eq:dt_Q_alpha} and \ref{eq:dt_Q_I_alpha} to express the conservation equation of the total first moment $\textbf{Q}^{(1)}_\alpha = \textbf{q}^{(1)}_\alpha + \textbf{q}^{(1)}_{\Gamma\alpha}$, this yields 
\begin{multline}
    \ddt {\textbf{Q}^{(1)}_\alpha}
    = \intO{ \left(
        \textbf{r} s_d^0         
        + f_d^0  \textbf{w}_d^0 
        - \mathbf{\Phi}_d^0
    \right) }
    + \intS{ \left(
        \textbf{r}s_\Gamma^0
        + f_\Gamma^0 \textbf{w}_\Gamma^0
        - \mathbf{\Phi}_{\Gamma||}^0
    \right) } \\
    + \intS{ \textbf{r} \left[
        \mathbf{\Phi}_f^0
        + f_f^0 (\textbf{u}_\Gamma^0-\textbf{u}_f^0)
    \right]\cdot \textbf{n}_d  }. 
    \label{eq:dt_Q_alpha_tot}
\end{multline}
Likewise, conservation laws can be derived for the $n^{th}$ order moments of volume and surface, i.e. for
\begin{align}
    \textbf{q}^{(n)}_{\alpha}
    = \intO{
         \underbrace{\textbf{rr}\ldots \textbf{rr}}_{n\text{ times}}
        f_d^0 },
        && \text{and} &&
    \textbf{q}^{(n)}_{\Gamma\alpha}
    = \intS{
         \underbrace{\textbf{rr}\ldots \textbf{rr}}_{n\text{ times}}
    f_\Gamma^0 },
    \label{eq:Q_n_definition}
\end{align} 
respectively. The time derivatives of $\textbf{q}^{(n)}_{\alpha}$ and $\textbf{q}^{(n)}_{\Gamma\alpha}$ do not introduce any additional terms beyond those already present in equations \ref{eq:dt_Q_alpha} and \ref{eq:dt_Q_I_alpha}.
Instead, they only involve the $n^{th}$ order moments of the existing terms.
%It can be shown that the derivative with time of $\textbf{q}^{(1)}_{\alpha n}$ and $\textbf{q}^{(1)}_{\Gamma\alpha n}$ do not involve any additional terms than in \ref{eq:dt_Q_alpha} and \ref{eq:dt_Q_I_alpha}, but rather just the $n^{th}$ order moments of the already presented terms.
We provide the full derivation of $\ddt{ \textbf{q}^{(n)}_{\alpha}}$ in \ref{ap:Moments_equations}.
The higher-order moments characterize the distributions of the local quantities $f_d^0$ and $f_\Gamma^0$ within the regions $\Omega_\alpha$ and $\Gamma_\alpha$, respectively. 
To precisely reconstruct the fields $f_d^0$ and $f_\Gamma^0$ within $\Omega_\alpha$ and $\Gamma_\alpha$, an infinite number of moments would theoretically be required. 
Nevertheless, as pointed out by \citet[Appendix A]{zhang1997momentum}, when the degrees of freedom of the particles are limited (solid particles, spherical droplets in stokes flows\ldots), only a finite number of moments are necessary to reconstruct $f_d^0$ and $f_\Gamma^0$. 
%Therefore, with knowledge of an arbitrary number of moments for a given quantity, one can achieve an arbitrary level of accuracy.
%These higher order moments describe the distributions of the local quantities $f_d^0$ and $f_\Gamma^0$ inside the domain $\Omega_\alpha$ and $\Gamma_\alpha$, respectively.
%Consequently, an infinite number of moments would be theoretically necessary to recover the fields $f_d^0$ and $f_\Gamma^0$ within $\Omega_\alpha$ and $\Gamma_\alpha$. 
%Thus, one can reach an arbitrary order of accuracy upon the knowledge of an arbitrary number of moments for a given quantity.  

\subsection{Particle-averaged equations}

In the preceding subsections, we have described the dispersed phase using a Lagrangian framework. 
However, to ensure consistency with the Eulerian conservation equations that describe the continuous phase, it is necessary to extend the Lagrangian equations to an Eulerian description. 
The approach presented here follows the methodology pioneered by \citep{lhuillier1992ensemble}.
%In the last section, we have described the dispersed phase within a Lagrangian framework.
%However, to be consistent with the Eulerian conservation equations used to describe the continuous phase, we need to extend the Lagrangian equations to an Eulerian description. 
%The strategy exposed here follow the approach pionnered by \citep{lhuillier1992ensemble}.
%In order to achieve this,
We introduce the function $\delta_\alpha$, which is defined as follows, 
\begin{align}
    \delta_\alpha(\textbf{x},\textbf{x}_\alpha(t,\FF)) 
    = \delta(\textbf{x}-\textbf{x}_\alpha(t,\FF)),
    \label{eq:delta_alpha}
\end{align}
where $\delta$ is the Dirac function.
Note that we explicitly note the arguments $(t,\FF)$ to highlight that the position of the particle $\alpha$ is a function of time and of the flow configuration $\FF$.
Taking the time derivative on $\delta_\alpha$ and 
applying the chain rule yields \citep{lhuillier1998}%we may write the partial time derivative of $\delta_\alpha$ can be written as
%\begin{equation}
%\frac{\partial \delta_\alpha(\textbf{x},\textbf{x}_\alpha(t,\FF))}{\partial t} 
%=  \frac{\partial \textbf{x}_\alpha}{\partial t} 
%\cdot \frac{\partial \delta_\alpha}{\partial \textbf{x}_\alpha}(\textbf{x},\textbf{x}_\alpha(t,\FF)) .
%\end{equation}
%This leads to the following expression, 
\begin{equation}
    \pddt \delta_\alpha
    + \div (\textbf{u}_\alpha  \delta_\alpha)
    =0,
    \label{eq:dt_delta_alpha}
\end{equation}
where we used the identity, $\frac{\partial \delta_\alpha}{\partial \textbf{x}_\alpha}  = -\grad \delta_\alpha$ and the fact that $\textbf{u}_\alpha(t,\FF)$ is not a function of $\textbf{x}$. 
\ref{eq:dt_delta_alpha} does not apply in scenarios where topological changes occur, such as break-up or coalescence events. 
In these cases, a source term can be introduced on the right-hand side of \ref{eq:dt_delta_alpha}, similar to the approach used in population balance equations, to account for the birth or death of particles \citep{randolph2012theory}.
%It should be noted that \ref{eq:dt_delta_alpha} is not applicable if changes in topology, such as break-up or coalescence events, occur.
%In such cases it is possible, as it is done in population balance equations, to include a source term on the RHS of \ref{eq:dt_delta_alpha} to account for particle birth or death. 
%Multiplying each Lagrangian quantities $\text q_\alpha$ by $\delta_\alpha$ yields the field $\text q_\alpha(t,\FF)\delta_\alpha(\textbf{x},t,\FF)$, which is defined over the entire domain $\Omega$.
%Likewise, for any derivative of Lagrangian quantities, such as $\ddt \text q_\alpha$, we define its corresponding Eulerian field by multiplying $\ddt \text q_\alpha$ with $\delta_\alpha$ and show that :
%where we have used the fact that $\text q_\alpha(t,\FF)$ and $\textbf{u}_\alpha(t,\FF)$ are not function of \textbf{x}, and we made use of \ref{eq:dt_delta_alpha}.
%Now let us consider a domain containing $N$ particles.
%We define what we call the \textit{particle field} of a quantity $\text q_\alpha$, as the sum of the $\delta_\alpha \text q_\alpha$ over all particles in the flow, namely $\displaystyle\sum_{\alpha=0}^N \delta_\alpha \text q_\alpha$.
%Note that \ref{eq:dt_delta_alpha_q_alpha} remains valid for a sum of fields since derivative and sum operators commute.
Consider a domain containing $N$ particles. We define the \textit{particle field} for a quantity $\text q_\alpha$ as the sum of $\delta_\alpha \text q_\alpha$ over all particles within the domain, expressed by $\displaystyle\sum_{\alpha=0}^N \delta_\alpha \text q_\alpha$. 
Note that the formula given by \ref{eq:dt_delta_alpha} remains valid for the sum of such fields, since the operations of differentiation and summation commute.
%In the objective of obtaining averaged equations for the dispersed phase, we introduce the average of $\text q_\alpha$ as  
To obtain the averaged equations for the dispersed phase, we define the particle average of $\text q_\alpha$ as
\begin{equation}
     n_p \text q_p(\textbf{x},t) = \avg{\sum_\alpha\delta_\alpha \text q_\alpha},
     \label{eq:p_avg}
\end{equation}
where, $n_p(\textbf{x},t) = \avg{\sum_\alpha \delta_\alpha}$ is the probability density of finding a particle center of mass in the infinitesimal volume $d\textbf{x}$ around \textbf{x}, and $\text q_p(\textbf{x},t)$ is the average of $\text q_\alpha$ conditionally on the presence of a particle's center of mass at \textbf{x} and time $t$. 
To simplify the notations, we consider the shorthand \citep{lhuillier1998},
\begin{equation*}
    \sum_\alpha \delta_\alpha \to \delta_p, 
\end{equation*}
such that $\pavg{\text q_\alpha}=\avg{\sum_\alpha \delta_\alpha \text q_\alpha}=n_p\text q_p$.
Note that we used the subscript $_p$ on $\text q_p$ to denote that this represents a particle-averaged field, initially derived from Lagrangian quantities. 
%Furthermore, in view of equation   
Additionally, in light of \ref{eq:def_fluctu} we define the fluctuating part of a particle field $\text q_p$ as
\begin{equation}
    \text q_\alpha'(\textbf{x},t,\FF) = \text q_\alpha(\FF,t) - \text q_p(\textbf{x},t). 
    \label{eq:def_fluc_p}
\end{equation}

To derive the averaged equations for the particle phase, we begin by noting that multiplying a Lagrangian quantity $\text q_\alpha$ by $\delta_\alpha$ yields the corresponding Eulerian field  $\text q_\alpha(t,\FF)\delta_\alpha(\textbf{x},t,\FF)$, which is defined throughout the entire domain $\Omega$. 
Similarly, for any derivative of a Lagrangian quantity, such as $\ddt \text q_\alpha$, the corresponding Eulerian field is defined by multiplying $\ddt \text q_\alpha$ with $\delta_\alpha$.
%This can be expressed as
Given that $\text q_\alpha(t,\FF)$ and $\textbf{u}_\alpha(t,\FF)$ do not depend on \textbf{x}, and by using \ref{eq:dt_delta_alpha}, we obtain
\begin{equation}
    \delta_\alpha \ddt \text q_\alpha
    = \pddt (\delta_\alpha \text q_\alpha)
    + \div (\delta_\alpha \text q_\alpha \textbf{u}_\alpha).
    \label{eq:dt_delta_alpha_q_alpha}
\end{equation}
Multiplying \ref{eq:dt_q_alpha_tot} and \ref{eq:dt_Q_alpha_tot} by $\delta_\alpha$, using \ref{eq:dt_delta_alpha_q_alpha} and applying the ensemble average (\ref{eq:avg}), yields
\begin{align}
    \pddt (n_p\text Q_p)
    + \div (n_p \text Q_p \textbf{u}_p + \pavg{\textbf{u}_\alpha' \text Q_\alpha'})
    = \pOavg{ s_d^0 }
    + \pSavg{ s_I^0 }\nonumber\\
    + \pSavg{ \left[\mathbf{\Phi}_f^0 + f_f^0 (\textbf{u}_\Gamma^0-\textbf{u}_f^0) \right] \cdot \textbf{n}_d },
    \label{eq:avg_dt_dq_alpha_tot}\\
    \pddt (n_p\textbf{Q}_p^{(1)})
    + \div \left(n_p \textbf{Q}_p^{(1)} \textbf{u}_p + \pavg{\textbf{u}_\alpha' \textbf{Q}_\alpha^{(1)'}}\right)
    =\pOavg{ \left(
        \textbf{r} s_d^0         
        + f_d^0  \textbf{w}_d^0 
        - \mathbf{\Phi}_d^0
    \right) }\nonumber\\
    + \pSavg{ \left(
        \textbf{r}s_\Gamma^0
        + f_\Gamma^0 \textbf{w}_\Gamma^0
        - \mathbf{\Phi}_{\Gamma||}^0
    \right) }
    + \pSavg{ \textbf{r} \left[
        \mathbf{\Phi}_f^0
        + f_f^0 (\textbf{u}_\Gamma^0-\textbf{u}_f^0)
    \right]\cdot \textbf{n}_d  }.
    \label{eq:avg_dt_dQ_alpha_tot}
\end{align}
The derivation of the higher moment particle-averaged equations is provided in \ref{ap:Moments_equations}.
The only fluxes appearing in \ref{eq:avg_dt_dq_alpha_tot} and \ref{eq:avg_dt_dQ_alpha_tot} are the fluctuation tensors $\pavg{\textbf{u}_\alpha' \text q_\alpha'}$ and $\pavg{\textbf{u}_\alpha' \textbf{Q}_\alpha'}$. 
Therefore, the non-convective fluxes $\bm\Phi_d^0$ and $\bm\Phi_I^0$ do not play the role of macroscopic fluxes, as it is the case in \ref{eq:avg_dt_chi_f} and \ref{eq:avg_dt_delta_f}. Instead, they act as source terms in the first moment and higher moment equations. 
This distinction is the main structural differences between the Kinetic-like model (\ref{eq:avg_dt_dq_alpha_tot} and \ref{eq:avg_dt_dQ_alpha_tot}) and the two-phase flow model (\ref{eq:avg_dt_chi_f} and \ref{eq:avg_dt_delta_f}). 
In this study, \ref{eq:avg_dt_chi_f} and \ref{eq:avg_dt_delta_f} are referred to as the phase-averaged equations, while \ref{eq:avg_dt_dq_alpha_tot} and \ref{eq:avg_dt_dQ_alpha_tot} are called the particle-averaged equations. 


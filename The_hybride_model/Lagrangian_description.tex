
While \ref{eq:dt_chi_k_f_k} and \ref{eq:dt_delta_I_f_I} describe multiphase-flow in a general manner, they do not leverage the topology of the dispersed phase. 
Therefore, in this section, we present a Lagrangian-based model capable of describing the dispersed phase with an arbitrary order of accuracy.

\subsection{Fundamental properties}

At this stage, it is crucial to define some fundamental properties associated to each particle.
To begin, let us define the mass, position of center of mass and momentum of the particle $\alpha$, as respectively,
\begin{align}
    m_\alpha(t)
    = \int_{\Omega_\alpha(t)} \rho_2  d\Omega,
    &&
    \textbf{x}_\alpha(t)
    = \frac{1}{m_\alpha(t) }\int_{\Omega_\alpha(t)} \rho_2 \textbf{x} d\Omega,
    &&
    \textbf{p}_\alpha(t) 
    = \int_{\Omega_\alpha(t)} \rho_2 \textbf{u}_2 d\Omega,
    \label{eq:position_and_momentum_def}
\end{align}
with $\rho_2$ the density of the phase $2$ and $\Omega_\alpha$ the domain occupied by the particle $\alpha$ (see \ref{fig:Scheme}). 
Subsequently, we define the velocity of the particle's center of mass, denoted as $\textbf{u}_\alpha$ which is given by $\textbf{u}_\alpha = \ddt \textbf{x}_\alpha$. 
The derivation of $\ddt \textbf{x}_\alpha$ is straightforward but requires some algebra which are detailed in \ref{ap:velocity_definition}. 
The final expression reads,
\begin{equation}
    \textbf{u}_\alpha(t) = \frac{1}{m_\alpha(t)} \left(
        \textbf{p}_\alpha(t)
        +  \int_{\Sigma_\alpha(t)} \rho_2 \textbf{r} (\textbf{u}_I - \textbf{u}_2)\cdot \textbf{n}_2 d\Sigma
        \right),
        \label{eq:dt_y_alpha}
\end{equation}
where $\textbf{r}(\textbf{x},t) = \textbf{x} - \textbf{x}_\alpha(t)$. 
In Equation \ref{eq:dt_y_alpha}, it can be observed that the first component of the velocity represents the linear momentum divided by the mass of the particle. 
This corresponds to the mass-averaged velocity over the volume of the particle.
The second term in Equation \ref{eq:dt_y_alpha} arises from the contribution of anisotropic mass transfer across the surface of the particle. 
This mass transfer leads to the motion of the particle's center of mass, thereby contributing to the total velocity.
To illustrate this concept, let us consider a fixed drop with no momentum lying over a very hot plate.
In this scenario, we assume that the plate is sufficiently hot to induce evaporation, specifically on the bottom portion of the drop.
Hence, under the effect of an anisotropic evaporation flux one may expect the second term to be non-negligible.
Consequently, the center of mass of the drop has a non-zero velocity in the opposite direction of the plate, even though the momentum is assumed to be zero.
We can also consider the case of the nucleation of a bubble in water. 
In this case, although the particle momentum is null at all time the center of mass of the particle moves due to the growth of the particle. 
In both cases, we need to take into account the mass transfer term in \ref{eq:dt_y_alpha}, while the first term is negligible. 
Note that \ref{eq:dt_y_alpha} generalized previous expression given by \citet{morel2015mathematical} whom neglect the second term.
In the following we discard the time dependency notation for all Lagrangian quantities denoted by the subscript $_\alpha$ and also $\Sigma(t)$ and $\Omega_\alpha$.
Nevertheless, the reader must understand that all Lagrangian quantities and integration domains are function of time. 

The particle's internal relative motions or the \textit{inner velocity} is given by $\textbf{w}_2(\textbf{x},t) = \textbf{u}_2(\textbf{x}) - \textbf{u}_\alpha(t)$.
Thus, from its definition in \ref{eq:position_and_momentum_def}, we can rewrite the momentum as follows,
\begin{equation}
    \label{eq:momentum_definition_1}
    \textbf{p}_\alpha
    = m_\alpha \textbf{u}_\alpha
    + \int_{\Omega_\alpha} \rho_2 \textbf{w}_2 d\Omega.
\end{equation}
Alternatively, by manipulating \ref{eq:dt_y_alpha}, we obtain,
\begin{equation}
    \textbf{p}_\alpha
    =  m_\alpha \textbf{u}_\alpha
    - \int_{\Sigma_\alpha} \textbf{r} \rho_2 (\textbf{u}_I - \textbf{u}_2)\cdot \textbf{n}_2 d\Sigma
    \label{eq:momentum_definition}
\end{equation}
Therefore, the momentum of a particle can be seen as a sum of the mean velocity plus the integral of the fluctuation (\ref{eq:momentum_definition_1}), with the latter being equivalent to minus the anisotropic mass transfer term (\ref{eq:momentum_definition}).
Indeed, by identification we obtain : $\int_{\Omega_\alpha} \rho_2 \textbf{w}_2 d\Omega =\int_{\Sigma_\alpha} \textbf{r} \rho_2 (\textbf{u}_I - \textbf{u}_2)\cdot \textbf{n}_2 d\Sigma$. 
The essential aspect of this relation is that the internal velocity fluctuations within a fluid particle do not contribute to the total linear momentum $\textbf{p}_\alpha$, as long as the anisotropic mass transfer is negligible.  


\subsection{Conservation laws}
Following the strategy of \citet{zaepffel2011modelisation} and \citet[Chapter 2]{morel2015mathematical} we assign to a particle indexed, $\alpha$, occupying the domain $\Omega_\alpha$ (see \ref{fig:Scheme}) an arbitrary Lagrangian property $q_\alpha$ defined by $q_\alpha  = \int_{\Omega_\alpha} f_2(\textbf{x},t) d\Omega$.
Similarly, we define $q_{I\alpha} = \int_{\Sigma_\alpha} f_I(\textbf{x},t) d\Sigma$ as being an integrated surface property associated to the particle $\alpha$.


To describe the evolution of any arbitrary Lagrangian quantity $q_\alpha$, we need to establish its time derivative.
Since, $q_\alpha$ is an integral quantity with a time-dependent domain of integration, we apply the general Reynolds transport theorem for volume integral (exposed in \ref{ap:math}) to compute its time derivative \citep{morel2015mathematical}.
This yields the following expression :
\begin{equation}
    \ddt  q_\alpha
    = \int_{\Omega_\alpha}\left[ \pddt f_2 + \nablabh \cdot\left(f_2\textbf{u}_2\right) \right]d\Omega\\
    + \int_{\Sigma_\alpha} f_2 (\textbf{u}_I-\textbf{u}_2)\cdot \textbf{n}_2 d\Sigma.
\end{equation}
By substituting the integrand of the first integral on the right-hand side (RHS) with \ref{eq:dt_f_k} we obtain the conservation laws of the quantity $q_\alpha$, namely,  
\begin{equation}
    \ddt  q_\alpha
    = \int_{\Omega_\alpha} \textbf{S}_2 d\Omega
    + \int_{\Sigma_\alpha} \left[
        f_2 (\textbf{u}_I-\textbf{u}_2) 
        + \mathbf{\Phi}_2 
        \right] \cdot \textbf{n}_2 d\Sigma,
    \label{eq:dt_q_alpha}
\end{equation}
The first term on the RHS accounts for the total contribution of the source term $\textbf{S}_2$ to the particle $\alpha$.
While, The second term on the RHS is the surface integration of the exchange terms, which includes the phase transfer flux $f_2 (\textbf{u}_I-\textbf{u}_2)$ and the diffusive flux $\mathbf{\Phi}_2$. 
For clarity, let us consider the specific case of the momentum balance, i.e. when $q_\alpha = \textbf{p}_\alpha$.
In this situation, the first term reads as $\int_{\Omega_\alpha} \rho_2\textbf{g} d\Omega$ and represents the total weight acting on the particle $\alpha$. 
Likewise, the second term represents the total source of momentum due to phase transfer, and it is expressed as, $\int_{\Sigma_\alpha} \rho_2 \textbf{u}_2 (\textbf{u}_I-\textbf{u}_2)\cdot\textbf{n}_2 d\Sigma$. 
Lastly, $\int_{\Sigma_\alpha} \textbf{T}_2\cdot\textbf{n}_2 d\Sigma$ represents the resultant of the hydrodynamic forces acting on the surface of the particle.
It is important to notice that under this form, the exchange terms are expressed as integrals of dispersed phase fields denoted by the subscript $_2$.
Nevertheless, depending on the nature of the dispersed phase, these fields may not always be defined.
For infinitely rigid particles it is indeed the case since, the stress $\textbf{T}_2$ isn't defined.  
Hence, our objective is to express these exchange terms, in terms of the continuous phase field quantities instead of the dispersed phase field, i.e. in terms of $\mathbf{\Phi}_1$ and $\textbf{u}_1$ rather than $\mathbf{\Phi}_2$ and $\textbf{u}_2$. 

To address this issue, let us derive the conservation equation for the integrated surface property $q_{I\alpha}$.
To differentiate time-varying surface integrals within time, we can use the general Leibniz rule (see \ref{eq:Leibnitz}), to derive the following expression :
\begin{equation}
    \ddt  q_{I\alpha}
    = \int_{\Sigma_\alpha} \left[
        \pddt f_I
        +   \nablabh_{||} \cdot (\textbf{u}_If_I)
    \right]d\Sigma.
    \label{eq:surface_derivative}
\end{equation}
Substituting the RHS terms of \ref{eq:surface_derivative} using \ref{eq:dt_f_I}, and making use of the surface divergence theorem on closed surfaces (see \ref{eq:surf_div_theorem}), gives,
\begin{equation}
    \ddt  q_{I\alpha}
    = \int_{\Sigma_\alpha} 
        \textbf{S}_I
    d\Sigma
    - \int_{\Sigma_\alpha} \Jump{
        f_k (\textbf{u}_I - \textbf{u}_k)
        + \mathbf{\Phi}_k
    }
    d\Sigma.
    \label{eq:dt_q_I_alpha}
\end{equation}
This equation can be interpreted as the surface conservation equation for the integrated surface property $f_I$, or as the flux jump condition integrated on a closed surface. 

As discussed above we wish to get rid of $\mathbf{\Phi}_2$ in \ref{eq:dt_q_alpha}. To achieve this, we treat the particle's volume and surface as a unified entity and derive a conservation equation for $q_\alpha + q_{I\alpha}$. 
This is done by summing \ref{eq:dt_q_alpha} and \ref{eq:dt_q_I_alpha} which leads to, 
\begin{equation}
    \ddt  (q_\alpha+q_{I\alpha})
    = 
    \int_{\Omega_\alpha} \textbf{S}_2 d\Omega
    + \int_{\Sigma_\alpha} \textbf{S}_I d\Sigma
    + \int_{\Sigma_\alpha} \left[
        f_1 (\textbf{u}_I-\textbf{u}_1) 
        + \mathbf{\Phi}_1 
        \right] \cdot \textbf{n}_2 d\Sigma. 
    \label{eq:dt_q_alpha_tot}
\end{equation}
This equation is the general form of the linear conservation law of $\chi_2 f_2 + \delta_I f_I$ for the system consisting of the particle volume $\Omega_\alpha$, and its surface $\Sigma_\alpha$. It is applicable to any particle immersed into a continuous phase following the local conservation,\ref{eq:dt_f_k} and \ref{eq:dt_f_I}.
We refer to this equation as the zeroth-order conservation equation or the linear conservation law for the particle $\alpha$.

Assuming interface with a negligible area density ($\rho_I=0$), and no phase transfer, i.e. $\textbf{u}_I=\textbf{u}_2=\textbf{u}_1$, the linear momentum balance yields, 
\begin{equation}
    \ddt \textbf{p}_\alpha
    = 
    \rho_2\textbf{g}v_\alpha
    + \textbf{f}_\alpha
    \label{eq:dt_p_alpha}
\end{equation}
where $m_\alpha = \int_{\Omega_\alpha}\rho_2 d\Omega$ is the volume of the particle, and $\textbf{f}_\alpha = \int_{\Sigma_\alpha}  \textbf{T}_1 \cdot \textbf{n}_2 d\Sigma$ is the resultants of the hydrodynamic force. 
This model does not account for inter-particle interactions. 
However, it is possible to include manually such forces in equation \ref{eq:dt_p_alpha}.
Finally, we would like to highlight that  due to the consideration of closed surface, the diffusive flux $\mathbf{\Phi}_I$, plays no role at all in \ref{eq:dt_q_alpha_tot}.
Therefore, in the case of the linear momentum conservation law, the contribution of the surface tension forces exposed in \ref{eq:surface_tension}, do not contribute to the momentum balance in \ref{eq:dt_p_alpha}.
As a consequence, even in the presence of local Marangoni forces, the resultant of the local surface tension forces cancels out in the linear momentum balance.
This fact has already been demonstrated by \citet{hesla1993note} who showed that the surface tension force does not contribute to the linear and angular momentum balance. 
Here, we have provided the general proof that the interfacial diffusive flux $\mathbf{\Phi}_I$, which is present at the local scale according to \ref{eq:dt_f_I}, does not contribute to the zeroth-order conservation law of a particle with a closed surface.
This principle is therefore applicable to other conservation equations, such as the surface energy balance or the surface mass balance of constituents, where surface diffusive fluxes are also present \citep{bothe2022sharp,manikantan2020surfactant}. 

Nevertheless, it is known that surface tension forces impact the hydrodynamic of droplets and bubbles \citep{kentheswaran2022direct,pesci2018computational}. 
Therefore, if the diffusive flux of surface are not involved in the linear conservation law, it must appear at some point in the mathematical description of Lagrangian particles. 
To find out where this contribution arise we shall describe the particle with a higher level of accuracy. 
This is the purpose of the next section. 

\subsection{First order moment equations}

To better describe the local properties within the particles, we now introduce the first moment or the dipole of a particle.
We define the first moment of any properties $f_2$ and $f_I$ by respectively,
\begin{align}
    &\mathcal{Q}_\alpha 
    = \int_{\Omega_\alpha} \textbf{r} f_2 d\Omega,
    &\text{and}&
    &\mathcal{Q}_{I\alpha}
    = \int_{\Sigma_\alpha} \textbf{r} f_I d\Sigma,
    \label{eq:first_moment_definition}
\end{align}
where we recall that $\textbf{r} = \textbf{x} - \textbf{x}_\alpha$ is the distance between any point inside $\Omega_\alpha$ or $\Sigma_\alpha$, to the center of mass of the particle $\alpha$.
It is then possible to differentiate these moments with respect to time in order to obtain their conservation laws.
Indeed, considering \ref{eq:dt_f_k}, \ref{eq:dt_f_I} and applying the Leibniz rule for volume and surface integrals (see \ref{eq:Reynolds} and \ref{eq:Leibnitz} respectively), we can show equally that,
\begin{align}
    \ddt \mathcal{Q}_\alpha
    &= \int_{\Omega_\alpha} \left(
        \textbf{r} \textbf{S}_2         
        + f_2  \textbf{w}_2 
        - \mathbf{\Phi}_2
    \right) d\Omega,
    + \int_{\Sigma_\alpha} \textbf{r} \left[
        \mathbf{\Phi}_2
        + f_2 (\textbf{u}_I-\textbf{u}_2)
    \right]\cdot \textbf{n}_2  d\Sigma 
    \label{eq:dt_Q_alpha}\\
    \ddt \mathcal{Q}_{I\alpha}
    &= \int_{\Sigma_\alpha} \left(
        \textbf{r}\textbf{S}_I
        + f_I \textbf{w}_I
        - \mathbf{\Phi}_{||}^I
    \right) d\Sigma,
    - \int_{\Sigma_\alpha}\textbf{r} \Jump{\mathbf{\Phi}_k
        + f_k (\textbf{u}_I - \textbf{u}_k)
    }
    d\Sigma
    \label{eq:dt_Q_I_alpha}
\end{align}
where $\textbf{w}_I = \textbf{u}_I - \textbf{u}_\alpha$.
The detailed derivation of \ref{eq:dt_Q_alpha} is provided in \ref{ap:moment_derivative}.
The derivation of \ref{eq:dt_Q_I_alpha} follows a similar procedure. 
% \JL{je n'ai pas relu la derivation detaillee en annexe, ... je te fais confiance. par contre en annexe tu ne derive pas le premier moment interfacial. 
% J'imagine que la derivation est la meme encore faut il le preciser. 
% Egalement j'ai regorganise les elements dans les equations precedentes par signification physique. 
% D'ailleurs il y avait des differences dans les deux equations (la premiere $r S$, la seconde $S r$)... 
% Merci de faire attention a ce genre de detail. 
% J'avoue avoir du mal a comprendre l'interpretation physique de l'integrale de la contrainte dans le volume. 
% Comme on en discutait, par exemple pour une particule solide, celle integrale n'est pas determinee, donc il faudrait la remplacer par quelque chose que l'on connait non ? 
% \tb{Dans le cas ou les contrainte ne sont pas defini les degrées de liberté des particules solid font que cette contrainte ne peux ne pas etre prise en compte la partie symmetrique de cette formule 
% permet justement de remonter a la contrainte dans le cas ou elle ne serait pas defini. dans le cas des particue fluid cela a du sens parcontre c'est les contraintes interne qui s'oppose a la deformations}}
% \JL{
%  Enfin bon a discuter (pas forcement ici). 
%  je pense que c'est un point important. 
% }\tb{cela va etre discuter dans la partie ou on traitre du momentum non ?}
% \JL{
%  Par ailleurs dans quel cas l'integrale des fluctuations $w_2$ est elle nulle ? 
%  pour une particule solide est ce le cas ? 
%  j'imagine que oui ? 
%  J'imagine que tout cela est traite plus tard (dans la derniere section), mais ca me parait crucial de bien expliquer a quoi servent ces termes et dans quel cas ils sont nuls. 
%  \tb{dur a expliquer pour une quantité general }
%  }\JL{
%  Une maniere d'expliciter tout cela pourrait etre de separer j'imagine le premier moment (au moins pour la vitesse) en une partie symmetrique et une partie anti symmetrique pour bien differencier ce qui est lie a la vitesse angulaire et la deformation. 
%  \tb{dans ce cas il faudrait donner l'application du momentum maintenant ce qui change le plan}}
In \ref{eq:dt_Q_alpha}, we recognize the first moment of the source term $\textbf{S}_2$, the first moment of the diffusive flux term $\mathbf{\Phi}_2\cdot\textbf{n}_2$ and the first moment of phase exchange term, $f_2 (\textbf{u}_I-\textbf{u}_2)\cdot\textbf{n}_2$. 
Additionally, two supplementary terms appear in \ref{eq:dt_Q_alpha}, namely : the integral of the diffusive flux $\mathbf{\Phi}_2$, and a term related to the fluctuation of the internal velocity $f_2 \textbf{w}_2$.
Similar observations can be made for the fist moment of surface equation \ref{eq:dt_Q_I_alpha}, as it shares similarities with \ref{eq:dt_Q_alpha}. 
In particular, it is worth noting the presence of the surface diffusive flux $\mathbf{\Phi}_{I||}$ in \ref{eq:dt_Q_I_alpha}.
This term will be further discussed and analyzed in the following. 

For similar reason than the linear conservation equations, we sum \ref{eq:dt_Q_alpha} and \ref{eq:dt_Q_I_alpha} to expresses the conservation equation of the total first moment $\mathcal{Q}_\alpha + \mathcal{Q}_{I\alpha}$.
This leads to the following expression:
\begin{multline}
    \ddt (\mathcal{Q}_\alpha+\mathcal{Q}_{I\alpha})
    = \int_{\Omega_\alpha} \left(
        \textbf{r} \textbf{S}_2         
        + f_2  \textbf{w}_2 
        - \mathbf{\Phi}_2
    \right) d\Omega\\
    + \int_{\Sigma_\alpha} \left(
        \textbf{r}\textbf{S}_I
        + f_I \textbf{w}_I
        - \mathbf{\Phi}_{||}^I
    \right) d\Sigma
    + \int_{\Sigma_\alpha} \textbf{r} \left[
        \mathbf{\Phi}_1
        + f_1 (\textbf{u}_I-\textbf{u}_1)
    \right]\cdot \textbf{n}_2  d\Sigma.
    \label{eq:dt_Q_alpha_tot}
\end{multline}
Likewise, conservation laws can be derived for an arbitrary $n^{th}$ order moments of volume and surface, i.e. for
\begin{align}
    (\mathcal{Q}_\alpha)^n
    = \int_{\Omega_\alpha}
        \textbf{r}^n
        f_2 d\Omega,
        && \text{and} &&
    \mathcal{Q}_{I\alpha}^n
    = \int_{\Sigma_\alpha}
        \textbf{r}^n
    f_I d\Sigma,
    \label{eq:Q_n_definition}
\end{align} 
respectively, where $\textbf{r}^n$ is the shorthand for the tensor product $\textbf{r}^n = \underbrace{\textbf{rr}\ldots \textbf{rr}}_{n\text{ times}} $ with $n$ times itself. 
It can be shown that the derivative with time of do not involve any additional terms than in \ref{eq:dt_Q_alpha} and \ref{eq:dt_Q_I_alpha}, but rather just the $n^{th}$ order moments of the already presented terms.
We provide the full derivation of $\ddt (\mathcal{Q}_\alpha)^n$ in \ref{ap:Moments_equations}.
In short, these higher order moments describe the distributions of the local quantities $f_2$ and $f_I$ inside the domain $\Omega_\alpha$ and $\Sigma$ respectively.
Consequently, an infinite number of moments would be theoretically necessary to recover the fields of $f_2$ and $f_I$  within $\Omega_\alpha$ and $\Sigma$. 


At this stage it is difficult to interpret the physical meaning behind these moments equations. 
Therefore, to gain in understanding we now discuss the second order moment of mass and first order moment of momentum conservation equations. 
For clearly, in the following examples, we consider a negligible area density, i.e. $\rho_I=0$. 
Additionally, we assume no phase exchange, resulting in $\textbf{u}_I=\textbf{u}_1=\textbf{u}_2$. 

Following \ref{eq:Q_n_definition} we define the second-order moment of mass and the first-order moment of momentum as respectively,
\begin{equation}
    \mathcal{M}_\alpha 
    = \int_{\Omega_\alpha} \rho_2 \textbf{r} \textbf{r} d\Omega
    \;\;\;\text{and}\;\;\;
    \mathcal{P}_\alpha 
    = \int_{\Omega_\alpha} \rho_2 \textbf{r} \textbf{u}_2 d\Omega.
    \label{eq:first_moment_of_momentum_def}
\end{equation}
Note that $\mathcal{M}_\alpha$ is analogous to the inertia tensor $\mathcal{I}_\alpha$ in solid mechanics, and they are related through the expression, $\mathcal{I}_\alpha = \text{tr}(\mathcal{M}_\alpha)\textbf{I} - \mathcal{M}_\alpha$.
At constant density the tensor $\mathcal{M}_\alpha$ describes the volume distribution around the particle's center of mass and, consequently, the shape of the particle.
In order to provide a clearer physical interpretation to the moment of momentum tensor, we decompose $\mathcal{P}_\alpha$ into two distinct part, namely,
$\mathcal{P}_\alpha = \mathcal{S}_\alpha+\mathcal{T}_\alpha$ where $\mathcal{S}_\alpha$ represents the symmetric part and $\mathcal{T}_\alpha$ is the antisymmetric part of $\mathcal{P}_\alpha$.
The tensors $\mathcal{S}_\alpha$ and $\mathcal{T}_\alpha$ correspond respectively to the stretching and angular momentum of the particle $\alpha$. 
The tensor $\mathcal{S}_\alpha$ quantifies how fast and in which direction the particle get elongated, it represents the rate of stretching or deformation experienced by the particle.
The tensor $\mathcal{T}_\alpha$ is related to the angular momentum of the particle. 
In this study we use the pseudo vector $\mu_\alpha = \int_{\Omega_\alpha} \rho_2 \textbf{r} \times \textbf{u}_2 d\Omega$ to express this quantity. 
Indeed, both  $\mathcal{T}_\alpha$ and $\mu_\alpha$ represent the angular momentum and are related through $(\mu_\alpha)_i = \epsilon_{ijk} (\mathcal{P}_\alpha)_{jk}= \epsilon_{ijk} (\mathcal{T}_\alpha)_{jk}$, where $\epsilon$ is the third order alternating unit tensor. 
Lastly, we also introduce the scalar $\mathcal{D}_\alpha = \text{tr}(\mathcal{P}_\alpha) = \frac{1}{3}\int_{\Omega_\alpha} \rho_2 \textbf{r} \cdot \textbf{u}_2 d\Omega.$, which quantifies the rate at which the particle is being compressed.


Injecting, $f_2 = \rho_2$ in the second-order moment equation derived in \ref{ap:Moments_equations} gives:
\begin{equation}
    \ddt \mathcal{M}_\alpha=2\mathcal{S}_\alpha. 
    \label{eq:dt_M_alpha}
\end{equation}
From \ref{eq:dt_M_alpha} we deduce that the evolution of the distribution of mass of a particle is solely motivated by the stretching of momentum, denoted by $\mathcal{S}_\alpha$. 
Note that if the particle has a constant $\mathcal{M}_\alpha$ such as for solid spherical particle, the stretching of momentum is then null.
This argument is also valid for spherical fluid particles with inner velocity motion.  
Additionally, applying the trace operator on both sides of \ref{eq:dt_M_alpha}, yields the interesting relation : $\ddt \text{tr}(\mathcal{M}_\alpha)=2\mathcal{D}_\alpha$.
Since the tensor $\mathcal{M}_\alpha$ is symmetric, it can always be diagonalized. 
Therefore, we can state that $\text{tr}(\mathcal{M}_\alpha) = \lambda^\alpha_1(t)+\lambda^\alpha_2(t)+\lambda^\alpha_3(t)$, with $\lambda_1^\alpha$,$\lambda_2^\alpha$ and $\lambda_3^\alpha$, being the eigenvalues of $\mathcal{M}_\alpha$.
For unreformable particles it is evident that the eigenvalues are not function of time, therefore $\ddt \text{tr}(\mathcal{M}_\alpha)=0$.  
Consequently, $\mathcal{D}_\alpha$ has the notable property of being null whenever the particle shape remain constant, irrespective of the orientation.
% It could also be  demonstrated that the time derivative of the determinant of $\mathcal{M}_\alpha$ is null, i.e. $\ddt \text{det}(\mathcal{M}_\alpha)=0$, for any particle with constant mass. 


% \begin{figure}
%     \centering
%     \begin{tikzpicture}
%         % \draw[fill=gray] (-4,0) circle(1);
%         \draw[fill=gray] (0,0) circle(1);
%         \foreach \th in {0,30,60,90,120,150,180}{
%         \foreach \r in {0.2,0.5,0.8}{
%             \draw[->](\r*{cos(\th)},\r*{cos(\th)})--(\r*{cos(\th)},\r*{cos(\th)});
%         }
%         }
%         % \draw[fill=gray] (4,0) circle(1);
%     \end{tikzpicture}
%     \caption[short]{Representation of the internal velocity fields for three case with pur symmetric , antisymmetric and isotropic  moment of momentum }
% \end{figure}


The moment of momentum equation is derived injecting $\mathcal{Q}_\alpha = \mathcal{P}_\alpha$ in \ref{eq:dt_Q_alpha_tot}, it reads, 
\begin{equation}
    \ddt \mathcal{P}_\alpha
    = \int_{\Omega_\alpha} \left(
        \rho_2  \textbf{w}_2 \textbf{w}_2 
        - \mathbf{T}_2
    \right) d\Omega
    - \int_{\Sigma_\alpha} 
        \sigma \textbf{I}_{||}
    d\Sigma
    + \textbf{M}_\alpha
    \label{eq:dt_P_alpha}
\end{equation}
where $\textbf{M}_\alpha = \int_{\Sigma_\alpha} \textbf{r}\mathbf{T}_1\cdot \textbf{n}_2d\Sigma $ represents the first hydrodynamic moment on the particle $\alpha$.
The tensor $\textbf{M}_\alpha$ is usually decomposed into a symmetric and antisymmetric part, i.e. $M^\alpha_{ij} 
% - \frac{1}{3}M^\alpha_{kk}\delta_{ij} 
= S^\alpha_{ij}+T^\alpha_{ij}$, with,
\begin{align}
    \label{eq:M_decomposition}
    S^\alpha_{ij} 
    &= \frac{1}{2}  \int_{\Sigma_\alpha} \left[
        r_i(T_{jk}n_k)
        + (T_{ik}n_k)r_j
        \right]d\Sigma
    %     - \frac{\delta_{ij}}{3}\int_{\Sigma_\alpha} \left[
    %         r_l(T_{lk}n_k)
    % \right]d\Sigma
    \\
    T^\alpha_{ij}
    &= \frac{1}{2}  \int_{\Sigma_\alpha} \left[
        r_i(T_{jk}n_k)
        - (T_{ik}n_k)r_j
    \right]d\Sigma \nonumber
\end{align}
where $T_{ij}$ and $n_k$ represents the components of the tensors $\textbf{T}_1$ and $\textbf{n}_2$ respectively. 
The tensors $\textbf{S}_\alpha$ and $\textbf{T}_\alpha$ correspond, respectively, to the hydrodynamic stresslet and torque applied on the particle $\alpha$ \citep{guazzelli2011,kim2013microhydrodynamics}. 
We also introduce the hydrodynamical torque vector as $\textbf{t}_\alpha = \int_{\Sigma_\alpha} \textbf{r} \times (\mathbf{T}_1\cdot \textbf{n}_2) d\Sigma$ which is related to the torque tensor by $t^\alpha_i = \epsilon_{ikj} T^\alpha_{jk}$, where $\epsilon$ the third-order alternating tensor. 
Notice that in \ref{eq:dt_P_alpha} the first moment  $\int_{\Omega_\alpha} \textbf{rg} d\Omega$ canceled out since \textbf{g} is constant, i.e. $\int_{\Omega_\alpha} \textbf{rg} d\Omega =\textbf{g}\int_{\Omega_\alpha} \textbf{r} d\Omega=0$. 
Each of the other terms appearing in \ref{eq:dt_P_alpha} is discussed in further detail in the following.

The conservation equation of the angular momentum $\mu_\alpha$ is obtained by taking the double contracted product of \ref{eq:dt_P_alpha} with $\epsilon$, which gives the simple expression :
\begin{equation}
    \ddt\mu_\alpha
    =  
    \textbf{t}_\alpha.
    \label{eq:dt_mu_alpha}
\end{equation}
Notice that every terms on the RHS of \ref{eq:dt_P_alpha} vanish due to their symmetric nature apart from the first hydrodynamic moment $\textbf{M}_\alpha$.
Particularly, the surface tension terms do not appear in the angular momentum balance, which is consistent with the findings of \citet{hesla1993note}. 
As a consequence, the surface tension has no effect on the angular momentum. 
In the literature it is common to include the torque due to inter-particular interactions in the angular momentum balance, as it is done in \citet{jackson1997locally} and \citet{zhang1997momentum}.
Here, it can be manually included on the RHS of \ref{eq:dt_mu_alpha}. 


Taking the symmetric part of \ref{eq:dt_P_alpha}, yield an equation for the stretching of momentum, which can be written as,
\begin{equation}    
    \ddt \mathcal{S}_\alpha
    =  \int_{\Omega_\alpha} \left(
        \rho_2\textbf{w}_2 \textbf{w}_2
        - \mathbf{T}_2
        \right) d\Omega
        - \int_{\Sigma_\alpha} 
        \sigma \textbf{I}_{||}
        d\Sigma
        + \textbf{S}_\alpha.
    \label{eq:dt_S_alpha}
\end{equation}
On the RHS of \ref{eq:dt_S_alpha} we can identify several terms: 
the internal kinetic energy $\int \rho_2\textbf{w}_2\textbf{w}_2 d\Omega$; 
the integral of the particle internal stress $\int_{\Omega_\alpha} \mathbf{T}_2 d\Omega$; 
the integral of the surface stress $\int_{\Sigma_\alpha} \sigma \textbf{I}_{||} d\Sigma$; 
and the stresslet tensor, $\textbf{S}_\alpha$ introduced earlier.
Based on \ref{eq:dt_M_alpha} we can infer that the evolution of $\mathcal{M}_\alpha$ is driven by the internal kinetic energy and the stresslet.
However, it is being counteracted by surface tension forces and internal stresses which tend to oppose the deformation of the particle. 
Therefore, if the surface tension forces play no role in the linear and angular momentum equation, it does impact the stretching of momentum $\mathcal{S}_\alpha$.
As a consequence, the surface tension force impact the hydrodynamic behavior of a particle solely through its action on $\mathcal{S}_\alpha$, which is related to the shape of a particle through \ref{eq:dt_M_alpha}.
Another significant aspect of \ref{eq:dt_S_alpha} is that it can be interpreted as an equation for the integrated stress tensor within the volume of the particle.
This becomes particularly relevant when determining the total stress inside assumed infinitely rigid particles. 


Lastly, we take the trace of \ref{eq:dt_Q_alpha_tot}, which directly yields the scalar equation :
\begin{equation}
    \ddt \mathcal{D}_\alpha
    = \int_{\Omega_\alpha} \left(
        \rho_2 \textbf{w}_2 \cdot \textbf{w}_2
        - \mathbf{T}_2 : \textbf{I}
        \right) d\Omega
        - 2\int_{\Sigma_\alpha} \sigma d\Sigma
        + \text{tr}(\textbf{M}_\alpha)
    \label{eq:dt_D_alpha}
\end{equation}
which correspond to the isotropic work balance within the particle's volume and surface. 
As a matter of fact, the rate of compression of a particle, denoted by the scalar $\mathcal{D}_\alpha$ evolves according to : 
the internal kinetic energy, $\int_{\Omega_\alpha}\rho_2 \textbf{w}_2 \cdot \textbf{w}_2 d\Omega$;
the trace of the integral of the hydrodynamic stresses, $\int_{\Omega_\alpha} \text{tr}(\textbf{T}_2)d\Omega$; 
the surface energy $\int_{\Sigma_\alpha} \sigma d\Sigma$; 
and the trace of the hydrodynamic first moment, $\text{tr}(\textbf{M}_\alpha)$.
To provide a concrete insight of the physical implication of the above equation, we consider the following example :
we examine a single spherical fluid particle of radius $a$, immersed in a steady flow such that $\textbf{u} = 0$ on $\Omega$. 
In this situation, the stress tensor can be written as $\textbf{T}_k = \textbf{I} p_k$ for $k = 1, 2$ where $p_k$ is the pressure in the phase $k$. 
Therefore, applying these considerations to \ref{eq:dt_D_alpha} yields the relation, 
\begin{equation*}
    \smallavg{p_2}{\Omega_\alpha} 
    - \smallavg{p_1}{\Sigma_\alpha}
    =
    \frac{2}{a} \smallavg{\sigma}{\Sigma_\alpha}
    \label{eq:Laplace_law}
\end{equation*}
where,  $\smallavg{p_1}{\Sigma_\alpha}$ and  $\smallavg{\sigma}{\Sigma_\alpha}$ are the surface-averaged external pressure and surface tension coefficient respectively, and $\smallavg{p_2}{\Omega_\alpha}$ represent the volume-averaged internal pressures.
Under this form it is evident that \ref{eq:Laplace_law} represent the well-known Laplace's Law. 
Additionally, in light of \ref{eq:dt_M_alpha}, this equation can be interpreted as an equilibrium equation for the particle internal mass distribution, or moment of inertia, since $\ddt\text{tr}(\mathcal{M}_\alpha) = 2 \mathcal{D}_\alpha$. 
From this argument and \ref{eq:dt_D_alpha}, one can may derive the \textit{Rayleigh-Plesset} equation by considering compressible spherical particles with non-constant density $\rho_2$.
A demonstration of this derivation can be found in the class of \tb{CITER LE COURS DE Lhuillier}. 
By the mean of kinetic theory \citet{zhang1994averaged} also derived the \textit{Rayleigh-Plesset} equation under an equivalent but averaged form.


\subsection{From Lagrangian to Eulerian fields}
Up to this point, we have described the dispersed phase within a Lagrangian framework.
However, to be coherent with the Eulerian conservation equations used to describe the continuous phase, we need to extend the Lagrangian equations to an Eulerian models. 
In order to achieve this, we introduce the function $\delta_\alpha$, which is defined as follows, 
\begin{align}
    \delta_\alpha(\textbf{x},t) = \delta(\textbf{x}-\textbf{x}_\alpha(t)).
    \label{eq:delta_alpha}
\end{align}
where $\delta$ is the Dirac delta function.
By noticing that $\delta_\alpha(\textbf{x}_\alpha,t) = 1$ independently of the time $t$, it can be shown that the convective derivative of the function $\delta_\alpha(\textbf{x},t)$ results in the following expression, 
\begin{equation}
    \pddt \delta_\alpha
    + \nablabh \cdot (\textbf{u}_\alpha  \delta_\alpha)
    =0,
    \label{eq:dt_delta_alpha}
\end{equation}
Additionally, it should be noted that \ref{eq:dt_delta_alpha} is not applicable if changes in topology, such as break up or coalescence events, occur.
In such cases it is possible to include a source term on the RHS of \ref{eq:dt_delta_alpha} to account for particle birth or death. 
Multiplying each Lagrangian quantities by $\delta_\alpha$ yields the \textit{particle field} of a quantity $q_\alpha$, denoted as $q_\alpha(t)\delta_\alpha(\textbf{x},t)$, which is defined throughout space and time.
Likewise, for any derivative of Lagrangian quantities, such as $\ddt q_\alpha$, we define its corresponding Eulerian field by Multiplying $\ddt q_\alpha$ with $\delta_\alpha$ and show that :
\begin{equation}
    \delta_\alpha \ddt q_\alpha
    = \pddt (\delta_\alpha q_\alpha)
    + \nablabh \cdot (\delta_\alpha q_\alpha \textbf{u}_\alpha)
    \label{eq:dt_delta_alpha_q_alpha}
\end{equation}
where we have utilized the fact that $q_\alpha(t)$ and $\textbf{u}_\alpha(t)$ are solely functions of time, and we made use of \ref{eq:dt_delta_alpha}.
Additionally, let's consider a volume containing $N$ particles.
We can then define the particle-field of a given quantity $q_\alpha$ as the sum of all the independent field, i.e. $\sum_{\alpha=0}^N \delta_\alpha q_\alpha$.
Notice that \ref{eq:dt_delta_alpha_q_alpha} remains valid for a sum of fields since derivative operators are linear.
To simplify the notations, we consider implicitly the summation over all particles included in $\Omega$ whenever a Lagrangian field denoted by $\delta_\alpha (\ldots)$ is present.

Multiplying \ref{eq:dt_q_alpha_tot} and \ref{eq:dt_Q_alpha_tot} by $\delta_\alpha$, summing over all particles, and by considering \ref{eq:dt_delta_alpha_q_alpha}, it is straightforward to show that,
\begin{multline}
    \pddt (\delta_\alpha (q_\alpha+q_{I_\alpha}))
    + \nablabh \cdot (\delta_\alpha\textbf{u}_\alpha(q_\alpha+q_{I_\alpha}))
    = \delta_\alpha\int_{\Omega_\alpha} \textbf{S}_2 d\Omega\\
    + \delta_\alpha\int_{\Sigma_\alpha} \textbf{S}_I d\Sigma
    + \delta_\alpha\int_{\Sigma_\alpha} \left[\mathbf{\Phi}_1 + f_1 (\textbf{u}_I-\textbf{u}_1) \right] \cdot \textbf{n}_2 d\Sigma,
    \label{eq:dt_dq_alpha_tot}
\end{multline}
\begin{multline}
    \pddt (\delta_\alpha (\mathcal{Q}_\alpha+\mathcal{Q}_{I_\alpha}))
    + \nablabh \cdot (\delta_\alpha\textbf{u}_\alpha(\mathcal{Q}_\alpha+\mathcal{Q}_{I_\alpha}))
    = \delta_\alpha\int_{\Omega_\alpha} \left(
        \textbf{r} \textbf{S}_2         
        + f_2  \textbf{w}_2 
        - \mathbf{\Phi}_2
    \right) d\Omega\\
    + \delta_\alpha\int_{\Sigma_\alpha} \left(
        \textbf{r}\textbf{S}_I
        + f_I \textbf{w}_I
        - \mathbf{\Phi}_{||}^I
    \right) d\Sigma
    + \delta_\alpha\int_{\Sigma_\alpha} \textbf{r} \left[
        \mathbf{\Phi}_1
        + f_1 (\textbf{u}_I-\textbf{u}_1)
    \right]\cdot \textbf{n}_2  d\Sigma.
    \label{eq:dt_dQ_alpha_tot}
\end{multline}
Similar consideration can be applied to the higher order moments equations derived in \ref{ap:moment_derivative}.

At this stage, we obtained two sets of equations that can be used to describe the dispersed phase. 
The first set of equations is the global conservation laws, i.e. \ref{eq:dt_chi_k_f_k} with for $k=2$ and \ref{eq:dt_delta_I_f_I}. 
The other is the particle-fields equations, such as \ref{eq:dt_dq_alpha_tot} and potentially the higher moments equations.
Therefore, some comments are in order regarding the differences and compatibility of these two sets of equations.
Solving \ref{eq:dt_dq_alpha_tot} ideally provides us with a field $\delta_\alpha(q_\alpha+q_{I\alpha})$ which contains the Lagrangian properties $q_\alpha+q_{I\alpha}$.
Thus, it corresponds to the volume and surface integral of $f_2$ and $f_I$ on $\Omega_\alpha$ and $\Sigma_\alpha$ respectively.
While, in \ref{eq:dt_chi_k_f_k} we solve the equation for the complete field $f_2$ defined inside the domains $\Omega_\alpha$.  
Thus, from  \ref{eq:dt_f_k} to \ref{eq:avg_dt_dq_alpha_tot} we lose the detailed description of $f_2$ within the particles' domain.
Indeed, with \ref{eq:avg_dt_dq_alpha_tot}, we recover solely the integrated value of $f_2$ over the particles' volume and surface. 
Therefore, \ref{eq:dt_dq_alpha_tot} can be though as averaged equations of \ref{eq:dt_chi_k_f_k} and \ref{eq:dt_delta_I_f_I} since we recover only the integrated properties of each particle. 
It is important to understand that in this sense, the passage from \ref{eq:dt_chi_k_f_k} and \ref{eq:dt_delta_I_f_I} to \ref{eq:dt_dq_alpha_tot} is an average operation carried out on the particles' volume and surface.
This is different to the usual averaged technics that refer to the ones used to derive the classic averaged models such as in \citet{jackson1997locally} and \citet{zhang1994averaged}, which are the subject of the following section. 




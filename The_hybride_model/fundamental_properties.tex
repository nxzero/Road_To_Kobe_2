\subsection{Fundamental properties of a particle}

At this stage, we define some fundamental properties associated to each particle labeled $\alpha$.
Following the strategy of \citet{lhuillier2009rheology,lhuillier1992volume,zaepffel2011modelisation} and \citet[Chapter 2]{morel2015mathematical}
we define the mass $m_\alpha$, position of center of mass $\mathbf{x}_\alpha$, and the momentum $\textbf{p}_\alpha$ of the particle $\alpha$, as
\begin{align}
    m_\alpha(t,\FF)
    = \intO{ \rho_d  }, 
    &&
    \textbf{x}_\alpha(t,\FF)
    = \frac{1}{m_\alpha(t,\FF) }\intO{ \rho_d \textbf{x} }, 
    &&\textbf{p}_\alpha(t,\FF) 
    = \intO{ \rho_d \textbf{u}_d^0 }.
    \label{eq:mass_pos}
    % \label{eq:momentum_energy}
\end{align}
$\Omega_\alpha(t,\FF)$ is the time-dependent domain occupied by the particle $\alpha$ (see \ref{fig:Scheme}). 
Subsequently, we define the velocity of the particle center of mass as
\begin{equation*}
\textbf{u}_\alpha = \frac{d \textbf{x}_\alpha}{dt}.
\end{equation*}
Replacing $\textbf{x}_\alpha$ by its definition (\ref{eq:mass_pos}) we obtain
\begin{equation*}
    \textbf{u}_\alpha = \frac{1}{m_\alpha}
    \frac{d}{dt} 
    \left(
        \intO{ \rho_d \textbf{x} }
    \right)
    - \frac{1}{m_\alpha^2} \frac{d}{dt} \left(\intO{ \rho_d } \right)
    \intO{ \rho_d \textbf{x} }.
\end{equation*}
%\tb{ A finaliser
Using the Reynolds transport theorem (\ref{eq:reynolds_transport}) for both terms in parentheses and making use of the conservation of mass (\ref{eq:dt_rho}) and the definition of $\textbf{x}_\alpha(t,\FF)$ in the last term, gives
\begin{equation}
    \textbf{u}_\alpha = 
    \frac{1}{m_\alpha}\intO{ \left[
        \pddt (\textbf{x}\rho_d ) + \div\left(\textbf{u}_d^0 \textbf{x} \rho_d\right) 
    \right]} \\
    + \frac{1}{m_\alpha}\intS{ \textbf{x} \rho_d(\textbf{u}_I   - \textbf{u}_d^0) \cdot \textbf{n}_d }
    -  \frac{\textbf{x}_\alpha}{m_\alpha}    \intS{ \rho_d(\textbf{u}_I   - \textbf{u}_d^0) \cdot \textbf{n}_d }
\end{equation}
Then by considering the mass conservation for the first term and noticing that $\grad \textbf{x} = \bm\delta$ with $\bm\delta$ the unit tensor, for the second term gives, 
\begin{equation}
    \textbf{u}_\alpha(t,\FF) = \frac{1}{m_\alpha(t,\FF)} \left(
        \textbf{p}_\alpha(t,\FF)
        +  \intS{\rho_d \textbf{r} (\textbf{u}_I^0 - \textbf{u}_d^0)\cdot \textbf{n}_d }
        \right),
        \label{eq:dt_y_alpha}
\end{equation}
where $\textbf{r}(\textbf{x},t) = \textbf{x} - \textbf{x}_\alpha(t)$. 
In \ref{eq:dt_y_alpha}, it can be observed that the first component of the velocity represents the linear momentum divided by the mass of the particle. 
This corresponds to the mass-averaged velocity over the volume of the particle.
The second term in \ref{eq:dt_y_alpha} arises from the contribution of anisotropic mass transfer across the surface of the particle. 
This mass transfer leads to the motion of the particle's center of mass, thereby contributing to the total velocity.
To illustrate this concept, let us consider a fixed drop with no momentum lying over a very hot plate.
In this scenario, we assume that the plate is sufficiently hot to induce evaporation, specifically on the bottom portion of the drop.
Hence, under the effect of an anisotropic evaporation flux one may expect the second term to be non-negligible.
Consequently, the center of mass of the drop has a non-zero velocity in the opposite direction of the plate, even though the momentum is assumed to be zero.
Note that \ref{eq:dt_y_alpha} generalized usual expression of the center of mass velocity whom neglect the second term.
In the following, for the sake of brevity we discard the dependency on $t$ and $\FF$ on the notations for all Lagrangian quantities denoted by the subscript $_\alpha$ and in particular $\Gamma_\alpha$ and $\Omega_\alpha$.
Nevertheless, the reader must understand that all Lagrangian quantities and integration domains subscribed by $_\alpha$ are time and configuration-dependent. 

The particle's internal relative motions or the \textit{inner velocity} is given by $\textbf{w}_d^0 = \textbf{u}_d^0 - \textbf{u}_\alpha$. 
Substituting the inner velocity in the momentum definition (\ref{eq:mass_pos}) yields
\begin{equation}
    \label{eq:momentum_definition_1}
    \textbf{p}_\alpha
    = m_\alpha \textbf{u}_\alpha
    + \int_{\Omega_\alpha} \rho_d \textbf{w}_d^0 d\Omega.
\end{equation}
Alternatively, from \eqref{eq:dt_y_alpha}, we obtain,
\begin{equation}
    \textbf{p}_\alpha
    =  m_\alpha \textbf{u}_\alpha
    - \int_{\Gamma_\alpha} \rho_d\textbf{r}(\textbf{u}_I^0 - \textbf{u}_d^0)\cdot \textbf{n}_d d\Sigma
    \label{eq:momentum_definition}
\end{equation}
Therefore, the momentum of a particle can be seen as a sum of the mean velocity plus the integral of the fluctuation (\ref{eq:momentum_definition_1}), with the latter being equivalent to minus the first moment of mass transfer term (\ref{eq:momentum_definition}).
Indeed, by identification we obtain : $\intO{ \rho_d \textbf{w}_d^0 } = - \intS{  \rho_d\textbf{r} (\textbf{u}_I^0 - \textbf{u}_d^0)\cdot \textbf{n}_d }$. 
Hence, the internal velocity fluctuations within a fluid particle do not contribute to the total linear momentum $\textbf{p}_\alpha$, as long as the anisotropic mass transfer is negligible.  
Only within this simplified context we can consider the classic relation $\textbf{p}_\alpha = m_\alpha \textbf{u}_\alpha$. 

\subsection{Conservation laws}



\subsubsection{Inside the volume}

 


Let us consider the specific case of the momentum balance, i.e. $q_\alpha = \textbf{p}_\alpha$.
In this situation, \ref{eq:dt_q_alpha} reads
\begin{equation}
    \ddt  \textbf{p}_\alpha
    = \intO{ \rho_d\textbf{g} }
    + \intS{ 
        \left[
        f_d^0 (\textbf{u}_I^0-\textbf{u}_d^0)
         + \bm{\sigma}_d^0%\cdot\textbf{n}_d  
        %+ \mathbf{\Phi}_d^0 
        \right] 
        \cdot \textbf{n}_d },
\end{equation}
% first term reads as $\intO{ \rho_d\textbf{g} }$ 
The first term on the right-hand side represents the total weight acting on the particle $\alpha$, 
the second term represents the total source of momentum due to phase transfer, and it is expressed as, $\intS{ \rho_d \textbf{u}_d^0 (\textbf{u}_I^0-\textbf{u}_d^0)\cdot\textbf{n}_d }$,
and the last term $\intS{ \bm{\sigma}_d^0\cdot\textbf{n}_d }$, represents the resultant of the hydrodynamic forces acting on the surface of the particle.
It is important to notice that under this form, the exchange terms are expressed as integrals of dispersed phase fields denoted by the subscript $_d$.
Nevertheless, depending on the nature of the dispersed phase, these fields may not always be defined.
For rigid particles the stress within the particle $\bm{\sigma}_d^0$ is indeterminate \citep{guazzelli2011}.  
Hence, our objective is to express these exchange terms, in terms of the continuous phase field quantities instead of the dispersed phase fields, i.e. in terms of $\mathbf{\Phi}_f^0$ and $\textbf{u}_f^0$ rather than $\mathbf{\Phi}_d^0$ and $\textbf{u}_d^0$. 

%\subsubsection{On the interfaces}
 

%For completeness, we exposed in \ref{ap:particles_eq} a clear derivation of the mass, momentum and total energy equations for a single particle.
%The derivation takes place using the same hypothesis as it is exposed in \ref{ap:hypothesis}.
%Especially, it is shown that the integration of the kinetic energy jump condition corresponds to the Lagrangian derivative of the particle surface, see \ref{eq:int_u_I2}.

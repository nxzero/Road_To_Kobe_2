
\section{Singularity solution for spherical droplets in pure linear flows}
\label{ap:solution_singularity}
% \subsubsection*{A droplet in an arbitrary linear flows}

As the general solution of \ref{eq:conditional_avg_eq_final_stokes} might become quite complicated we neglect the quadratic and higher order contribution from the flow feild. 
Then, the solution for the disturbance velocity field within the droplets and fluid phase can be written \citep{leal2007advanced}, 
\begin{equation}
    \textbf{u}_f^1[\textbf{r}|\textbf{x},\textbf{w}]
    = 
    \mathcal{U}_f\cdot (\textbf{u}_f - \textbf{w})
    + \mathcal{E}_f: \textbf{E}_{f}
\end{equation}
\begin{equation}
    \textbf{u}_d^1[\textbf{r}|\textbf{x},\textbf{w}]
    = 
    \mathcal{U}_d\cdot (\textbf{u}_f - \textbf{w})
    + \mathcal{E}_d: \textbf{E}_{f}
\end{equation}
where the second and third order tensor $\mathcal{U}$ and $\mathcal{E}$ read as, 
\begin{align}
    \mathcal{U}_{ik,f} = 
    -\frac{1}{4}\left(\frac{3\lambda + 2}{\lambda +1}\right)
    \left(\frac{\delta_{ik}}{r} + \frac{r_ir_k}{r^3}\right) 
    - 
    \frac{1}{4}\left(\frac{\lambda}{\lambda +1}\right)
    \left(\frac{\delta_{ik}}{r^3} - \frac{3r_ir_k}{r^5}\right)  \\
    \mathcal{E}_{ijk,f}
    =
    %  \bm\delta\textbf{r}
    -\frac{\lambda}{(\lambda + 1)r^5} \bm\delta\textbf{r}
    -\left(\frac{5\lambda +2}{2(\lambda +1 )r^5} - \frac{5\lambda}{2(\lambda+1)r^7}\right) \textbf{rrr}
\end{align}
\begin{align}
    \mathcal{U}_{ik,d} = 
    -\frac{1}{2}\left(\frac{2\lambda +3}{\lambda +1}\right)\bm\delta
    +\frac{1}{2} (2r^2 \bm\delta - \textbf{rr})
    \left(\frac{1}{\lambda +1}\right)\\
    \mathcal{E}_{ijk,d}
    =
    \frac{5r^2 -3}{2(\lambda +1)} \textbf{r}\bm\delta
    - \frac{1}{\lambda+1}\textbf{rrr}
\end{align}
where the \textbf{r} are made dimensionless with the particle radius. 

The dimensionless pressure field which are also linear with the position reads as,
\begin{align*}
    p_f^1[\textbf{r}|\textbf{x},\textbf{w}]
    = 
    \mathcal{U}_f^p\cdot (\textbf{u}_f - \textbf{w})
    + \mathcal{E}_f^p: \textbf{E}_{f}\\
    p_d^1[\textbf{r}|\textbf{x},\textbf{w}]
    = 
    \mathcal{U}_d^p\cdot (\textbf{u}_f - \textbf{w})
    + \mathcal{E}_d^p: \textbf{E}_{f}
\end{align*}
where we have introduced, 
\begin{align*}
   \mathcal{U}_f^p = 
     \frac{1}{2}\left(\frac{3\lambda + 2}{\lambda +1}\right) \frac{r_j}{r^3}
   && \mathcal{U}_d^p = 
   - 5 \left(\frac{1}{\lambda +1}\right) r_j\\
   \mathcal{E}_f^p = - \frac{(5\lambda+2)}{(\lambda+1)r^5}\textbf{rr}
   &&\mathcal{E}_d^p = \frac{21\lambda}{2(\lambda+1)} \textbf{rr}
\end{align*}

% The rate of train field of the phase $k$ can be written in the dimensionless form as, 
% \begin{align*}
%    ( \textbf{e}_{k})_{ij} 
%     = 
%     (
%     \partial_j 
%     (\mathcal{U}_{k})_{ik}
%     + 
%     \partial_i 
%     (\mathcal{U}_{k})_{jk}
%     )
%     \textbf{U}_k
%     +
%     (
%     \partial_j 
%     (\mathcal{E}_{k})_{ikl}
%     + 
%     \partial_i 
%     (\mathcal{E}_{k})_{jkl}
%     )
%    ( \textbf{E}_{k})_{kl}
% \end{align*}
% Or, 
% \begin{align*}
%    ( \textbf{e}_{k})_{ij} 
%     = 
%     (\mathcal{U}_k^e)_{ijk}
%     \textbf{U}_k
%     +
%     (\mathcal{E}_k)_{ijkl}
%    ( \textbf{E}_{k})_{kl}
% \end{align*}
% with, 
% \begin{align*}
%     (\mathcal{U}_k^e)_{ijk}
%     = 
%     \frac{1}{2}(
%     \partial_j 
%     \mathcal{U}_{f,ik}
%     + 
%     \partial_i 
%     \mathcal{U}_{f,jk}
%     )\\
%     (\mathcal{E}_k^e)_{ijkl}
%     = 
%     \frac{1}{2}(
%     \partial_j 
%     \mathcal{E}_{f,ikl}
%     + 
%     \partial_i 
%     \mathcal{E}_{f,jkl}
%     )\\
% \end{align*}

% Finally, the stress tensor can also be written as, 
% Or, 
% \begin{align*}
%    ( \bm\sigma_{k})_{ij} 
%     &= 
%     [
%     - (\mathcal{U}_k^p)_{k}\delta_{ij}
%     + 2(\mathcal{U}_k^e)_{ijk}
%     \textbf{U}_k
%     ]
%     +
%     - (\mathcal{E}_k^p)_{kl}\delta_{ij}
%     + 2(\mathcal{E}_k)_{ijkl}
%    ( \textbf{E}_{k})_{kl} \\
%     &=(\mathcal{U}_k^\sigma)_{ijk}
%     \textbf{U}_k
%     +
%     (\mathcal{E}^\sigma_k)_{ijkl}
%    ( \textbf{E}_{k})_{kl} \\
% \end{align*}



Firstly, we give the expression of the particle's surface velocity which is used to derive \ref{eq:energy_term}. 
This is done by evaluating $\mathcal{U}$ and $\mathcal{E}$ at the particle surface which directly gives, 
\begin{align}
    \mathcal{U}_{ik} = 
    % \frac{1}{4}\left(\frac{3\lambda + 2}{\lambda +1}\right)
    % \left(\delta_{ik} + r_ir_k\right) 
    % + 
    % \frac{1}{4}\left(\frac{\lambda}{\lambda +1}\right)
    % \left(\delta_{ik} - 3r_ir_k\right)  \\
    \frac{1}{2}\left(\frac{2\lambda + 1}{\lambda +1}\right)
    \delta_{ik} 
    + 
    \frac{1}{2}\left(\frac{1}{\lambda +1}\right)
    r_ir_k,  \\
    \mathcal{E}_{ijk}
    = 
    % \left[1-
    \frac{\lambda}{(\lambda + 1)}
    % \right]
    \bm\delta\textbf{r}
    -\left(\frac{2}{2(\lambda +1 )} \right) \textbf{rrr}. 
\end{align}
Then, the velocity at the surface of a particle relative to the center of mass velocity \textbf{w}, reads as, 
\begin{multline}
    \textbf{w}_d^0 
    = \left(\frac{\lambda + \frac{1}{2}}{\lambda +1} - 1\right)
    (\textbf{u}_f - \textbf{w}) 
    + 
    \frac{1}{2}\left(\frac{1}{\lambda +1}\right)
    \textbf{rr} \cdot (\textbf{u}_f - \textbf{w})  \\
    + \left[1-\frac{\lambda}{(\lambda + 1)}\right]\textbf{E}_f\cdot\textbf{r}
    -\left(\frac{2}{2(\lambda +1 )} \right) \textbf{r} \textbf{E}_f:\textbf{rr}. 
\end{multline}


Now that we have an explicit solution for the internal flow and stress one can compute the terms appearing in the energy and moment of momentum balance equations. 
The integral involving the particle internal motions and stress in the equation exposed in \ref{sec:Lagrangian} are evaluated using the formulation \ref{eq:conditionally_averaged_vol} and yield, 
\begin{align*}
    \pOavg{\rho_2(\textbf{w}_2^0)_i (\textbf{w}_2^0)_j}
    &= \frac{\rho_d \phi_d}{140(\lambda +1 )^2}
    \left[
        7\textbf{u}_{fp}\textbf{u}_{pf} 
    + (\textbf{u}_{pf}\cdot \textbf{u}_{pf})\bm\delta
    + 7\pavg{\textbf{u}_\alpha'\textbf{u}_\alpha'}/n_p 
    + 2k_p \bm\delta
    \right]\\
    &+ \frac{\rho_d \phi_d a^2}{315 (\lambda + 1)^2}[(\textbf{E}_f : \textbf{E}_f)\bm\delta+15\textbf{E}_f\cdot \textbf{E}_f]
\\
    \pOavg{\rho_2(\textbf{w}_2^0)_k (\textbf{w}_2^0)_k}
    &=  \frac{\rho_d \phi_d}{12 (\lambda +1)^2}
    (\textbf{u}_{fp} \cdot \textbf{u}_{fp} + 2 k_p)
    + \frac{a^2 \rho_d \phi_d}{15(\lambda+1)^2}
    \textbf{E}_f:\textbf{E}_f    \\
    % \pOavg{\rho_2(\textbf{w}_2^0)_i (\textbf{w}_2^0)_j}^\text{dev}
    % &= \frac{m_\alpha}{20 (\lambda +1)^2}
    % ((\textbf{u}_{pf})_i(\textbf{u}_{pf})_j + (\textbf{u}_{pf}\cdot \textbf{u}_{pf})\delta_{ij})
% \\
    \pOavg{\bm\sigma_2^0}
    &= \frac{6}{5}\phi \mu_f \frac{1}{1+\lambda} \textbf{E}_f
\\
    \pOavg{\bm{\sigma}_2^0:\grad \textbf{u}_2^0}
    % &= 2\mu_2 \pOavg{\textbf{e}_2^0: \textbf{e}_2^0 }
    &= 
    % \frac{6 \mu_2 \phi_d}{a^2(1+\lambda)^2}
    % (\textbf{u}_{pf}\cdot \textbf{u}_{pf})
    \frac{6 \phi_d \mu_f \lambda}{a^2(1+\lambda)^2}
    (\textbf{u}_{fp}\cdot \textbf{u}_{fp} + 2k_p)
    + \frac{3 \phi_d \mu_f \lambda}{(\lambda+1)^2}\textbf{E}_f:\textbf{E}_f
\\
    % \pOavg{\textbf{r}\textbf{u}_2^0}
    % &= 0 \\
    \pOavg{\mu(\textbf{e}_2^0)_{ik} r_l} &=
    \frac{\phi_d\mu_1}{10}\left(\frac{1}{\lambda+1}\right)
    \left(
        2\delta_{ik}(\textbf{u}_{fp})_l
        -3\delta_{kl}(\textbf{u}_{fp})_i
        -3\delta_{il}(\textbf{u}_{fp})_k
    \right)
\end{align*}
Regarding the interface term they are evaluated using the formulation \ref{eq:conditionally_averaged} and reads, 
\begin{align}
    \pSavg{\bm{\sigma}_f^0\cdot \textbf{n}_d} &= 
    \phi_d \div\bm\Sigma_f
    + \frac{3\phi_d\mu_f}{2 a^2} 
    \left(\frac{3\lambda+2}{\lambda+1}\right) \textbf{u}_{f p} 
    + \frac{3\phi_d\mu_f}{4} \left(\frac{\lambda}{\lambda+1}\right)\grad^2\textbf{u}_f\\
    \pavg{\intS{\textbf{r}\bm{\sigma}_f^0 \cdot \textbf{n}_d}} 
    &= 
    \phi_d \bm\Sigma_f + 
    \frac{3}{5}\mu_f \phi_d \left(\frac{2+5\lambda}{1+\lambda}\right)
    \textbf{E}_f
    \\
        \pavg{\intS{(\bm{\sigma}_f^0 \cdot \textbf{n}_d)_ir_kr_l}} &=
        % \phi_d  \frac{a^2}{5} 3 [(\div \bm\sigma_f)\bm\delta]^\text{sym}
        + \frac{3\mu_f\phi_d}{2}\left(\frac{\lambda}{\lambda+1}\right)(\textbf{u}_{fp})_i\delta_{kl}\\
        &+ \frac{3\mu_f\phi_d}{5}\left(\frac{1}{\lambda+1}\right)((\textbf{u}_{fp})_i\delta_{kl}+ (\textbf{u}_{fp})_k\delta_{il}+(\textbf{u}_{fp})_l\delta_{ki})\nonumber
\end{align}
where the last term on the RHS of the first integral comes from the quadratic contribution of the flow. 
It is possible that such contribution arise for the other term but they are neglected here. 


The last kind of integral are the fluid phase closure terms, they are computed using \ref{eq:Batchelor2} and reads, 
\begin{align*}
    \avg{\chi_f \bm\sigma_f^0 :\grad\textbf{u}_f^0}
    &=
    \frac{3\mu_f \phi_d}{2a^2}
    \frac{(3\lambda^2 + 4\lambda +2)}{(\lambda + 1)^2}
    (\textbf{u}_{fp}\cdot \textbf{u}_{fp} + 2 k_p ) \\
    &+ 
    \frac{3\mu_f \phi_d}{5}
    \frac{(5\lambda^2 + 4\lambda +4)}{(\lambda + 1)^2}
    \textbf{E}_f:\textbf{E}_f
    +2 \phi_f \textbf{E}_f:\textbf{E}_f \\
    \avg{\chi_f \rho_f \textbf{u}_f' \textbf{u}_f'}
    &=
    \frac{\phi_d a^2 \rho_f}{105 (\lambda +1)^2 }\left[
        (129\lambda^2+108\lambda+24)\textbf{E}_f\cdot \textbf{E}_f
        + (20\lambda^2 +20\lambda + 6)
        (\textbf{E}_f : \textbf{E}_f)\bm\delta
    \right]\\
    &+\infty(\textbf{u}_{fp})\\
     \avg{\chi_f\rho_f \textbf{u}_f' k_f} &=  0\\
    \avg{\chi_f \textbf{u}_f' \cdot \bm{\sigma}_f^0} &=  0
\end{align*}

The second contribution of the Reynolds stress is infinite because \ref{eq:Batchelor2} is not exact. 
However, based on \citet{hill2001first} and \citet{zhang1994ensemble} we can expect that,  
\begin{equation*}
    \avg{\chi_f \rho_f \textbf{u}_f' \textbf{u}_f'}
    =
    C_1(\phi) [\textbf{u}_{fp} \textbf{u}_{fp}
    + \pavg{\textbf{u}_\alpha'\textbf{u}_\alpha'} ]
    + C_2(\phi) [\textbf{u}_{fp}\cdot \textbf{u}_{fp} + 2 n_p k_p]\bm\delta
\end{equation*}
Where $C_1$ and $C_2$ are two unknown constants. 
% \begin{equation*}
%     \int_{r>a} \mu_f \bm\sigma^0_f:\grad \textbf{u}_f  d\textbf{r}
%     = 
%     \frac{3\mu_f \phi_d}{2a^2}
%     \frac{(3\lambda^2 + 4\lambda +2)}{(\lambda + 1)^2}
%     (\textbf{u}_{fp}\cdot \textbf{u}_{fp})
% \end{equation*}

% \subsection*{A translating droplets}
% \label{ap:Translating_sphere}

% \tb{Consider using the famous reciprocal theorem  ?  ? ? ? }

% \tb{This is usefull to derive the faxen contribution }

% \tb{this is not usefull to derive the spetical contribution . . .}


% In this appendix we expose the velocity fields solution for the stokes flow past a spherical drop. 
% In a second step, we compute the form of the moment of momentum closure mentioned in \ref{sec:Lagrangian} in terms of the fluid properties and the drop fluid relative velocity.

% We consider a drop of radius $a$ translating with the velocity $\textbf{u}_\alpha$ in a stokes flow with undisturbed velocity $\textbf{u}_1$.
% The relative velocity between the droplet and the fluid is defined as $\textbf{u}_{\alpha 1}= \textbf{u}_\alpha - \textbf{u}_1(\textbf{x}_\alpha)$ where $\textbf{u}_1(\textbf{x}_\alpha)$ is the undisturbed averaged fluid velocity at the position $\textbf{x}_\alpha$. 
% In these conditions, the local fluid phase velocity $\textbf{u}_1^0$, the local particle interior velocity $\textbf{u}_2^1$ and the local stress fields $\bm\sigma_2^0$ within the particle phase, might be written as \citet{pozrikidis1992boundary}\footnote{The solution of this problem is derived in the above cited book p .207.  However, the author made a slight mistakes on the constant $\textbf{c}$ which we corrected here. }, 
% \begin{align*}
%     u_{1,i}^0
%     = \left(\frac{\delta_{ik}}{r} + \frac{r_ir_k}{r^3}\right)  g_k
%     + \left(-\frac{\delta_{ik}}{r^3} + \frac{3r_ir_k}{r^5}\right)  d_k\\
%     u_{2,i}^0
%     = c_i
%     + \left(2 r^2 \delta_{ik} - r_ir_k\right) e_k\\
%     % e_{2,ik}^0
%     % = \mu(
%     %     3 \delta_{ij} r_k 
%     %     + 3 \delta_{kj} r_i
%     %     -2 r_j \delta_{ki}
%     % )e_j 
%     \sigma_{2,ik}^0
%     = \mu 3(
%         - 4 \delta_{ik} r_j
%         + \delta_{ij} r_k
%         + r_i \delta_{kj}
%     )e_j 
% \end{align*}
% with, 
% \begin{align*}
%     &\textbf{g} = a\frac{1}{4}\left(\frac{3\lambda + 2}{\lambda +1}\right) \textbf{u}_{\alpha 1},
%     &\textbf{d} = -a^3\frac{1}{4}\left(\frac{\lambda}{\lambda +1}\right) \textbf{u}_{\alpha 1},\\
%     &\textbf{c} = \frac{1}{2}\left(\frac{2\lambda + 3}{\lambda +1}\right) \textbf{u}_{\alpha 1},
%     &\textbf{e} = -\frac{1}{a^2}\frac{1}{2}\left(\frac{1}{\lambda +1}\right)  \textbf{u}_{\alpha 1}.\\
% \end{align*}

% Now that we have an explicit solution for the internal flow and stress one can compute the terms appearing in the energy and moment of momentum balance equations. 
% The integral involving the particle internal motion in the equation exposed in \ref{sec:Lagrangian} are evaluated exhaustively and yield, 
% \begin{equation*}
%     \intO{\rho_2(\textbf{w}_2^0)_i (\textbf{w}_2^0)_j}
%     = \frac{m_\alpha}{140 (\lambda +1)^2}
%     (7 (\textbf{u}_{\alpha 1})_i(\textbf{u}_{\alpha 1})_j + (\textbf{u}_{\alpha 1}\cdot \textbf{u}_{\alpha 1})\delta_{ij})
% \end{equation*} 
% \begin{equation*}
%     \intO{\rho_2(\textbf{w}_2^0)_k (\textbf{w}_2^0)_k}
%     = \frac{m_\alpha}{14 (\lambda +1)^2}
%      (\textbf{u}_{\alpha 1})_k(\textbf{u}_{\alpha 1})_k
% \end{equation*} 
% \begin{equation*}
%     \intO{\rho_2(\textbf{w}_2^0)_i (\textbf{w}_2^0)_j}^\text{dev}
%     = \frac{m_\alpha}{20 (\lambda +1)^2}
%     ((\textbf{u}_{\alpha 1})_i(\textbf{u}_{\alpha 1})_j + (\textbf{u}_{\alpha 1}\cdot \textbf{u}_{\alpha 1})\delta_{ij})
% \end{equation*} 
% \begin{equation*}
%     \intO{\bm\sigma_2^0}
%     = 0 
% \end{equation*}
% \begin{equation*}
%     \intO{\bm{\sigma}_2^0:\grad \textbf{u}_2^0}
%     = 2\mu_2 \intO{\textbf{e}_2^0: \textbf{e}_2^0 }
%     = 
%     \frac{6 \mu_2 v_p}{a^2(1+\lambda)^2}
%     (\textbf{u}_{\alpha 1}\cdot \textbf{u}_{\alpha 1})
% \end{equation*}
% \begin{align*}
%     \intO{\textbf{r}\textbf{u}_2^0}
%     = 0 
% \end{align*}
% \begin{equation*}
%     \intS{\bm\sigma_1^0\cdot \textbf{n}_2}
%     = - \frac{3v_\alpha\mu_1}{2 a^2} 
%     \left(\frac{3\lambda+2}{\lambda+1}\right) 
%     \textbf{u}_{\alpha 1}
% \end{equation*}
% \begin{equation*}
%     \intS{\textbf{r}\bm\sigma_2^0\cdot \textbf{n}_2}
%     = 0 
% \end{equation*}
% \begin{equation*}
%     \intS{(\bm\sigma_2^0\cdot \textbf{n}_2)_i r_kr_l}
%     = \frac{3\mu_1\phi_2}{10}\left(\frac{5\lambda+2}{\lambda+1}\right)u_{1 \alpha,i}\delta_{kl}
%     + \frac{3\mu_1\phi_2}{5}\left(\frac{1}{\lambda+1}\right)(u_{1 \alpha,k}\delta_{il}+u_{1 \alpha,l}\delta_{ki})
% \end{equation*}
% \begin{equation*}
%     \intO{\mu(\textbf{e}_2^0)_{ik} r_l} =
%     \frac{\phi_2\mu_1}{10}\left(\frac{1}{\lambda+1}\right)
%     \left(
%         2\delta_{ik}u_{1 \alpha , l}
%         -3\delta_{kl}u_{1 \alpha , i}
%         -3\delta_{il}u_{1 \alpha , k}
%     \right)
% \end{equation*}
% \begin{equation*}
%     \int_{r>a} \mu_f \bm\sigma^0_f:\grad \textbf{u}_f  d\textbf{r}
%     = 
%     \frac{3\mu_f \phi_d}{2a^2}
%     \frac{(3\lambda^2 + 4\lambda +2)}{(\lambda + 1)^2}
%     (\textbf{u}_{fp}\cdot \textbf{u}_{fp})
% \end{equation*}
% We can note that now all these quantities are determined by the relative velocity.  
% While the hill vortex is a solution valid in potential and stokes flow, the solution for the exterior flow is limited to stoke flow. 
% Consequently, the zero, first and second moment of surface force are limited to stokes flow. 

% \subsection{A droplet in shear flow}

% We consider an infinite linear flow $\textbf{u}^\infty = \bm\Gamma\cdot \textbf{r} =(\bm\Omega + \textbf{E})\cdot \textbf{r} $ such that the undisturbed flow around the sphere is $\textbf{u}^\infty = \bm\Gamma\cdot\textbf{r}$.
% The disturbance flow in this case might be determined following the method outlined in \citep{leal2007advanced}, we obtain  :
% \begin{align*}
%     \textbf{u}^0_1
%     = \bm\Gamma\cdot\textbf{r}
%     -\frac{\lambda}{(\lambda + 1)r^5} \textbf{E}\cdot\textbf{r}
%     - \left(\frac{5\lambda +2}{2(\lambda +1 )r^5} - \frac{5\lambda}{2(\lambda+1)r^7}\right) \textbf{r}(\textbf{E}:\textbf{rr})\\
%     p_1^0 
%     = -\frac{(5\lambda+2)}{(\lambda+1)r^5}\textbf{E}:\textbf{rr}\\
% % \end{align*}
% % \begin{align*}
%     \textbf{u}^0_2
%     = \bm\Omega\cdot\textbf{r}
%     + \frac{5r^2- 3}{2(\lambda + 1)} 
%     \textbf{E}\cdot\textbf{r}
%     -\frac{1}{\lambda+1} \textbf{r}(\textbf{E}:\textbf{rr})\\
%     p_2^0 
%     = \frac{21\lambda}{2(\lambda+1)}
%     \textbf{E}:\textbf{rr}
% \end{align*}

% With that we are able to compute the following integrals,
% \begin{equation*}
%     \intO{\rho_2(\textbf{w}_2^0)_i (\textbf{w}_2^0)_j}
%     = \frac{4\pi}{15}\bm\Omega\cdot\bm\Omega
%     + \frac{4\pi}{945(\lambda+1)^2}
%     (2\textbf{E}:\textbf{EI} + 15 \textbf{E}\cdot \textbf{E})
% \end{equation*}
% \begin{equation*}
%     \intO{\rho_2(\textbf{w}_2^0)_k (\textbf{w}_2^0)_k}
%     = \frac{4\pi}{15}\bm\Omega:\bm\Omega
%     + \frac{4\pi}{45(\lambda+1)^2}
%     \textbf{E}:\textbf{E} 
% \end{equation*} 
% \begin{equation*}
%     \intO{\bm\sigma_2^0}
%     = \frac{8\pi}{5(\lambda+1)}
%     \textbf{E}
% \end{equation*}
% \begin{equation*}
%     \intO{\textbf{e}_2^0}
%     = \frac{4\pi}{5(\lambda+1)}
%     \textbf{E}
% \end{equation*}
% \begin{equation*}
%     \intO{\bm{\sigma}_2^0:\grad \textbf{u}_2^0}
%     = 2\mu_2 \intO{\textbf{e}_2^0: \textbf{e}_2^0 }
%     = 
%     \frac{4\pi}{(\lambda+1)^2}\textbf{E}:\textbf{E}
% \end{equation*}
% \begin{align*}
%     \intO{\textbf{r}\textbf{u}_2^0}
%     = 0 
% \end{align*}
% \begin{equation*}
%     \intS{\bm\sigma_1^0\cdot \textbf{n}_2}
%     = 0
% \end{equation*}
% \begin{equation*}
%     \intS{\textbf{r}\bm\sigma_1^0\cdot \textbf{n}_2}
%     = \frac{4\pi}{5}\frac{5\lambda+2}{\lambda+1}\textbf{E} 
% \end{equation*}
% \begin{equation*}
%     \intS{(\bm\sigma_2^0\cdot \textbf{n}_2)_i r_kr_l}
%     = 0
% \end{equation*}
% \begin{equation*}
%     \intO{\mu(\textbf{e}_2^0)_{ik} r_l} =
%     0
% \end{equation*}
% \begin{equation*}
%     \int_{r>a} \mu_f \bm\sigma^0_f:\grad \textbf{u}_f  d\textbf{r}
%     = 
%     \frac{3\mu_f \phi_d}{5}
%     \frac{(5\lambda^2 + 4\lambda +4)}{(\lambda + 1)^2}
%     \textbf{E}:\textbf{E}
% \end{equation*}
% \begin{equation*}
%     \int_{r>a} \rho_f \textbf{u}_f' \textbf{u}_f'  d\textbf{r}
%     = 
%     \frac{\phi_d a^2 \rho_f}{105 (\lambda +1)^2 }\left[
%         (129\lambda^2+108\lambda+24)\textbf{E}_f\cdot \textbf{E}_f
%         + (20\lambda^2 +20\lambda + 6)
%         (\textbf{E}_f : \textbf{E}_f)\bm\delta
%     \right]
% \end{equation*}
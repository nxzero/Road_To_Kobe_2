\tb{MAKE SMALL INTRODUCTION }
\section{Mass, momentum and total energy}
\label{ap:hypothesis}
In this section we re iterate the derivation presented from \ref{sec:local_eq} to \ref{sec:Lagrangian} applied to mass, momentum and energy conservation laws. 
We consider a mono-disperse emulsions with subject to buoyancy forces. 
The dispersed and continuous phases are considered Newtonian fluids defined by the constant viscosities $\mu_k$ and density $\rho_k$.


\subsection{Local conservation equations}

We now present the governing equations valid at the local scale within the bulk and on the interfaces. 

\subsubsection{In the volumes}
Within phase $k$, we note, $E_k^0 = e_k^0 + (u_k^0)^2/2$ the local total energy per unit of mass, where $e_k^0$ is the internal energy per unit of mass and $(u_k^0)^2/2$ is the kinetic energy per unit of mass.
All over the domain $\Omega_k$ the mass, momentum and total energy obey 
\begin{align}
    \label{eq:dt_rho}
    % \pddt \rho_k  
    % + 
    \div \textbf{u}_k^0
    &= 
    0,\\
    \label{eq:dt_rhou_k}
    \pddt (\rho_k\textbf{u}_k^0)  
    + \div (
        \rho_k\textbf{u}_k^0\textbf{u}_k^0
        - \bm{\sigma}_k^0 
    )
    &= 
    \rho_k \textbf{g},\\
    \label{eq:dt_rhoE_k}
    \pddt (\rho_kE_k^0)  
    + \div (
        \rho_kE_k^0\textbf{u}_k^0
        + \textbf{q}_k^0
        - \textbf{u}_k^0 \cdot \bm{\sigma}_k^0 
        )
    &= 
    \textbf{u}_k^0 \cdot \textbf{g}  \rho_k, 
\end{align} 
respectively. 
Where $\bm\sigma_k^0$ and $\textbf{q}_k^0$ represent the stress tensor and the thermal energy flux of phase $k$, respectively. 
The vector $\textbf{g}$ is the acceleration of gravity which is the only body force considered here.
We consider Newtonian fluids therefore $\bm\sigma_k^0 = -p_k^0\bm\delta + 2\mu_f \textbf{e}_k^0$ with $\textbf{e}_k^0$ is the local rate of strain of phase $k$, namely $\textbf{e}_k^0 = \frac{1}{2}[\grad \textbf{u}_k^0 + (\grad \textbf{u}_k^0)^\dagger]$. 
No heat source will be considered in this study. 
From the previous set of equations we can derive two independent equations, one for $e_k^0$ and a second for $(u_k^0)^2/2$, they read
\begin{align}
    \label{eq:dt_rhou_k2}
    \pddt [\rho_k(u_k^0)^2]  
    + \div [\rho_k(u_k^0)^2\textbf{u}_k^0/2 - \textbf{u}_k^0 \cdot \bm{\sigma}_k^0]
    &=
    \rho_k\textbf{u}_2^0 \cdot \textbf{g}  
    -  \bm{\sigma}_k^0 : \grad \textbf{u}_k^0,
    \\
    \label{eq:dt_rhoe_k}
    \pddt (\rho_ke_k^0)  
    + \div (
        \rho_ke_k^0\textbf{u}_k^0
        + \textbf{q}_k^0
        )
    &= 
    \bm{\sigma}_k^0 : \grad \textbf{u}_k^0,
\end{align} 
respectively. 
We can observe that the term, $\bm{\sigma}_k^0 : \grad \textbf{u}_k^0$,  appears with opposite signs in \ref{eq:dt_rhou_k2} and \ref{eq:dt_rhoe_k}.
This indicates that the amount of energy transferred from kinetic to internal energy is given by $\bm{\sigma}_k^0 : \grad \textbf{u}_k^0$.

\subsubsection{On interfaces}

On the interface $\Gamma$ the governing equations take the form of two dimensional conservation laws. 
They are often viewed as \textit{jump conditions} or as boundary conditions for  \ref{eq:dt_rho}, \ref{eq:dt_rhou_k} and \ref{eq:dt_rhoE_k}. 
In the most general case, the mass, momentum and energy surface equations read as \citep{ishii2010thermo,morel2015mathematical,bothe2022sharp}, 
\begin{align}
    \label{eq:dt_rhoI}
    \pddt \rho_I^0
    + \divI (\rho_I^0\textbf{u}_{I}^0)
    + \rho_I^0 (\textbf{u}_I^0 \cdot \textbf{n})(\div \textbf{n})
    = 
    -\Jump{
        \rho_k (\textbf{u}_I - \textbf{u}_k)
    }
    \\
    % \label{eq:dt_rhoIu_I}
    \pddt (\rho_I^0\textbf{u}_I^0)  
    + \divI (
        \rho_I^0\textbf{u}_{I}^0\textbf{u}_{I||}^0
        - \bm{\sigma}_{I||}^0)
        + \rho_I^0 \textbf{u}_I^0 (\textbf{u}_I^0 \cdot \textbf{n})(\div \textbf{n}) \nonumber\\
    = 
    \rho_I^0 \textbf{g} 
    - \Jump{
        \rho_k \textbf{u}_k (\textbf{u}_I - \textbf{u}_k)
        + \bm\sigma^0_k
    }
    \\
    \label{eq:dt_rhoIE_I}
    \pddt (\rho_I^0E_I^0)  
    + \divI (
        \rho_I^0 E_I^0\textbf{u}_{I||}^0
        - \textbf{u}_I^0 \cdot \bm{\sigma}_{I||}^0 
        + \textbf{q}_{I||}^0
        )
    + \rho_I^0E_I^0  (\textbf{u}_I \cdot \textbf{n})(\div \textbf{n})\nonumber\\
    = 
    \textbf{u}_I^0 \cdot \textbf{g}  \rho_I^0 
    - \Jump{\textbf{u}_k^0 \cdot \bm{\sigma}_k^0 - \textbf{q}_k^0
    + \rho_k E_k (\textbf{u}_I - \textbf{u}_k)
    },
\end{align} 
respectively.
Where $\rho_I^0$ is the interface density, $\rho_I^0\textbf{u}_I^0$ the interface momentum 
and $E_I^0 = e_I^0 + \frac{1}{2}(u_I^0)^2$ the total energy of the interface, with $e_I^0$ the internal energy of the interface.
$\bm{\sigma}_{I||}^0$ and $\textbf{q}_{I||}^0$ denote the non-convective fluxes of momentum and heat on the interface, respectively.


To simplify the analysis we make the following assumptions. We neglect the mass, the momentum and the kinetic energy of the interfaces so that,
\begin{align*}
    \rho_I \approx 0,
    &&
    \rho_I \textbf{u}_I^0 \approx 0,
    &&
    \rho_I E_I^0 \approx \rho_I e_I^0. 
\end{align*}
Additionally, we neglect all the dissipative processes at the interfaces such that
\begin{align*}
    \textbf{q}_{I||} = 0,
    &&
    \bm\sigma_{I||} = \gamma (\bm\delta  - \textbf{nn}),
\end{align*}
where $\gamma$ is surface tension coefficient. 
In this situation the surface equations, from \ref{eq:dt_rhoI} to \ref{eq:dt_rhoIE_I} reduce to, 
\begin{align}
    0&= 
    \Jump{
        \rho_k (\textbf{u}_I - \textbf{u}_k)
    }
    \\
    \divI [\gamma ( \bm\delta - \textbf{nn})]
    &= 
    \Jump{
        \rho_k \textbf{u}_k (\textbf{u}_I - \textbf{u}_k)
        + \bm\sigma^0_k
    }
    \\
    \pddt (\rho_I^0e_I^0)  
    + \divI [
        (\rho_I^0 e_I^0 - \gamma)
         \textbf{u}_{I||}^0
        ]
    + \rho_I^0e_I^0  (\textbf{u}_I \cdot \textbf{n})(\div \textbf{n})
    &= 
    -\Jump{\textbf{u}_k^0 \cdot \bm{\sigma}_k^0 - \textbf{q}_k^0
    + \rho_k E_k (\textbf{u}_I - \textbf{u}_k)
    }. 
\end{align} 
At this point notice that the coefficient $\gamma$ is not a constant. 
Lastly, if no mass transfer is allowed at the interface, together with the assumption on shape interfaces hypothesis this implies that $\textbf{u}_k^0 = \textbf{u}_I^0$ for $k = 1,2$ \citep[chapter 2]{tryggvason2011direct}. 
Additionally, if one makes the assumption of a constant surface tension coefficient, we have $\rho_I^0 e_I^0 = \gamma$ \citep{ishii2010thermo}.  
Under these restrictive assumptions we obtain the mass, momentum and total energy equations valid on the domain $\Gamma$, namely
\begin{align}
    \label{eq:dt_rho_I}
    \textbf{u}_k^0 = \textbf{u}_I^0, \\
    \Jump{\bm{\sigma}_k^0} 
    &=
    \divI\bm\sigma^0_{I||}
    =
    -\gamma\textbf{n}(\div \textbf{n}),
    \label{eq:surface_tension}\\
    \label{eq:dt_rhoI_EI}
    \Jump{\textbf{u}_k^0 \cdot \bm{\sigma}_k^0 - \textbf{q}_k^0}
    &=
    -\gamma\textbf{n}\cdot \textbf{u}_{I}^0(\div \textbf{n}).
\end{align}
% respectively. 
% The jump condition for the total energy can be separated into an expression for the kinetic and internal energy. 
Subtracting the dot product of \ref{eq:surface_tension} with $\textbf{u}_I^0$ to \ref{eq:dt_rhoI_EI}, gives 
\begin{align}
    \label{eq:dt_rhoI_uI3}
    \Jump{\textbf{u}_k^0 \cdot \bm{\sigma}_k^0}
    &=
    -\gamma\textbf{n}\cdot \textbf{u}_{I}^0(\div \textbf{n})\\
    \label{eq:dt_rhoIe_I}
    \Jump{ \textbf{q}_k^0}
    &= 
     0.
\end{align}
\ref{eq:dt_rhoI_uI3} and \ref{eq:dt_rhoIe_I} correspond to the interface kinetic energy and the internal interface energy jump condition, respectively. 
As witnesses by \ref{eq:dt_rhoI_uI3} the discontinuity of mechanical energy through the surface is equivalent to the work of the tension forces. 
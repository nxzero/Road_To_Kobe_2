We now treat the specific case of the mass, momentum and total energy conservation of an emulsion subject to buoyancy forces only. 
Thus, in this section we re iterate the derivation presented from \ref{sec:local_eq} to \ref{sec:Lagrangian}. 
Specifically, we first  present the local conservation laws, the two-fluid formulation and the Lagrangian conservation equations. 
The complete hybrid model will be presented in the subsequent section. 


\subsection{Local conservation equaitons}

Within phase $k$, we note $\rho_k$ the constant density of the fluid $k$, $\textbf{u}_k^0$ the local velocity and $E_k^0$ the local total energy per unit of mass.
All over the domain $\Omega_k$ the mass, momentum and total energy obey these conservation laws:
\begin{align}
    \label{eq:dt_rho}
    \pddt \rho_k^0  
    + \div (
        \rho_k^0\textbf{u}_k^0
    )
    &= 
    0,\\
    \label{eq:dt_rhou_k}
    \pddt (\rho_k^0\textbf{u}_k^0)  
    + \div (
        \rho_k^0\textbf{u}_k^0\textbf{u}_k^0
        - \bm{\sigma}_k^0 
    )
    &= 
    \rho_k^0 \textbf{g},\\
    \label{eq:dt_rhoE_k}
    \pddt (\rho_k^0E_k^0)  
    + \div (
        \rho_k^0E_k^0\textbf{u}_k^0
        + \textbf{q}_k^0
        - \textbf{u}_k^0 \cdot \bm{\sigma}_k^0 
        )
    &= 
    \textbf{u}_k^0 \cdot \textbf{g}  \rho_k^0, 
\end{align} 
respectively. 
Where $\bm\sigma_k^0$ and $\textbf{q}_k^0$ represent the stress tensor and the thermal energy flux of phase $k$, respectively. 
The vector $\textbf{g}$ is the acceleration of gravity which is the only body force considered here.
No heat source will be considered in this study.

The total energy of phase $k$ can be further decomposed as $E_k^0 = e_k^0 + (u_k^0)^2/2$, where $e_k^0$ is the internal energy per unit of mass representing molecular agitation and $(u_k^0)^2/2$ is the kinetic energy per unit of mass.
This decomposition, along with the previous set of equations, leads us to two independent equations, one for $e_k^0$ and a second for $(u_k^0)^2/2$, they read
\begin{align}
    \label{eq:dt_rhou_k2}
    \pddt [\rho_k^0(u_k^0)^2]  
    + \div [\rho_k^0(u_k^0)^2\textbf{u}_k^0/2 - \textbf{u}_k^0 \cdot \bm{\sigma}_k^0]
    &=
    \rho_k^0\textbf{u}_2^0 \cdot \textbf{g}  
    -  \bm{\sigma}_k^0 : \grad \textbf{u}_k^0,
    \\
    \label{eq:dt_rhoe_k}
    \pddt (\rho_k^0e_k^0)  
    + \div (
        \rho_k^0e_k^0\textbf{u}_k^0
        + \textbf{q}_k^0
        )
    &= 
    \bm{\sigma}_k^0 : \grad \textbf{u}_k^0,
\end{align} 
respectively. 
We can observe that the term, $\bm{\sigma}_k^0 : \grad \textbf{u}_k^0$,  appears with opposite signs in \ref{eq:dt_rhou_k2} and \ref{eq:dt_rhoe_k}.
This indicates that the amount of energy transformed from kinetic to internal energy is given by $\bm{\sigma}_k^0 : \grad \textbf{u}_k^0$.
Thus, in the case of Newtonian fluids, $\bm{\sigma}_f^0 : \grad \textbf{u}_f^0$ represents the energy dissipation due to viscous forces, which then acts as a heat source in \ref{eq:dt_rhoe_k}. 

% \subsection{On interfaces}

% On the interface $\Gamma$ the governing equations take the form of $2D$ conservation laws. 
% They are often viewed as \textit{jump conditions} or as boundary conditions for  \ref{eq:dt_rho}, \ref{eq:dt_rhou_k} and \ref{eq:dt_rhoE_k}. 
% In the most general case, the mass, momentum and energy surface equations read as \citep{ishii2010thermo,morel2015mathematical,bothe2022sharp}, 
% \begin{align}
%     \label{eq:dt_rhoI}
%     \pddt \rho_I^0
%     + \divI (\rho_I^0\textbf{u}_{I}^0)
%     + \rho_I^0 (\textbf{u}_I^0 \cdot \textbf{n})(\div \textbf{n})
%     &= 
%     -\Jump{
%         \rho_k^0 (\textbf{u}_I - \textbf{u}_k)
%     }
%     \\
%     \label{eq:dt_rhoIu_I}
%     \pddt (\rho_I^0\textbf{u}_I^0)  
%     + \divI (
%         \rho_I^0\textbf{u}_{I}^0\textbf{u}_{I||}^0
%         - \bm{\sigma}_{I||}^0)
%         + \rho_I^0 \textbf{u}_I^0 (\textbf{u}_I^0 \cdot \textbf{n})(\div \textbf{n})
%     &= 
%     \rho_I^0 \textbf{g}
%     - \Jump{
%         \rho_k^0 \textbf{u}_k (\textbf{u}_I - \textbf{u}_k)
%         + \bm\sigma^0_k
%     }
%     \\
%     \label{eq:dt_rhoIE_I}
%     \pddt (\rho_I^0E_I^0)  
%     + \divI (
%         \rho_I^0 E_I^0\textbf{u}_{I||}^0
%         - \textbf{u}_I^0 \cdot \bm{\sigma}_{I||}^0 
%         + \textbf{q}_{I||}^0
%         )
%     + \rho_I^0E_I^0  (\textbf{u}_I \cdot \textbf{n})(\div \textbf{n})
%     &= 
%     \textbf{u}_I^0 \cdot \textbf{g}  \rho_I^0 \nonumber\\
%     &- \Jump{\textbf{u}_k^0 \cdot \bm{\sigma}_k^0 - \textbf{q}_k^0
%     + \rho_k^0 E_k (\textbf{u}_I - \textbf{u}_k)
%     },
% \end{align} 
% respectively.
% Where $\rho_I^0$ is the interface density, $\rho_I^0\textbf{u}_I^0$ the interface momentum 
% and $E_I^0 = e_I^0 + \frac{1}{2}(u_I^0)^2$ the total energy of the interface, with $e_I^0$ the internal energy of the interface.
% $\bm{\sigma}_{I||}^0$ and $\textbf{q}_{I||}^0$ denote the momentum and heat fluxes on the interface, respectively.


% In \ref{ap:hypothesis} we expose the surface conservation equations considering no surface properties except the uniform surface tension effects, i.e. first term of \ref{eq:surface_fluxes} with $\gamma$ constant. 
% This yields a much simpler set of equations for the interfaces that will be used in the last section of the present paper (see \ref{sec:application}). 
% For further insights into the modeling of sharp interface thermodynamics a comprehensive review may be found in \cite{bothe2022sharp}. 

% \subsection{Simplifying hypothesis}

For the interfacial equations we consider that :
(1) The mass, momentum and kinetic energy per unit of surface can be neglected.
% , i.e. $\textbf{u}_I^0\rho_I = 0 $ and $(u_I^0)^2\rho_I = 0$. 
% Implying that the surface source terms are also neglected, here $\rho_I \textbf{u}_I^0\cdot \textbf{g} = 0$; 
(2) No mass transfer is allowed at the interface, together with the previous assumption this implies that $\textbf{u}_k^0 = \textbf{u}_I^0$ for $k = 1,2$. 
(3) All molecular diffusion fluxes can be neglected, i.e. the viscous stress at the interfaces as well as the surface heat flux can be neglected.
(4) The surface stress tensor is assumed surface isotropic, and is written $\bm{\sigma}_I^0  = \gamma (\bm\delta - \textbf{nn}) = \gamma \bm\delta_{||}$ where $\gamma$ is the constant surface tension coefficient.
% Under this hypothesis the internal energy and the surface tension coefficient are related through $\phi_I^0 e_I^0 = \gamma$ \cite{ishii2010thermo};
% When using a constant surface tension coefficient the interfacial mass, momentum and total energy balance equations reduce to the expressions,
Under these restrictive assumptions we obtain the following set of governing equations for the interfaces,
\begin{align}
    \label{eq:dt_rho_I}
    \textbf{u}_k^0 = \textbf{u}_I^0, \\
    \Jump{\bm{\sigma}_k^0} 
    &=
    \divI\bm\sigma^0_{I||}
    =
    -\gamma\textbf{n}(\div \textbf{n}),
    \label{eq:surface_tension}\\
    \label{eq:dt_rhoI_EI}
    \Jump{\textbf{u}_k^0 \cdot \bm{\sigma}_k^0 - \textbf{q}_k^0}
    &=
    -\gamma\textbf{n}\cdot \textbf{u}_{I}^0(\div \textbf{n}).
\end{align}
% respectively. 
% The jump condition for the total energy can be separated into an expression for the kinetic and internal energy. 
By taking the dot product of \ref{eq:surface_tension} with $\textbf{u}_I^0$ and subtracting this first expression to \ref{eq:dt_rhoI_EI}, gives us
\begin{align}
    \label{eq:dt_rhoI_uI3}
    \Jump{\textbf{u}_k^0 \cdot \bm{\sigma}_k^0}
    &=
    -\gamma\textbf{n}\cdot \textbf{u}_{I}^0(\div \textbf{n})\\
    \label{eq:dt_rhoIe_I}
    \Jump{ \textbf{q}_k^0}
    &= 
     0
\end{align}
which are the interface kinetic energy and the internal interface energy jump condition, respectively. 
As witnesses by \ref{eq:dt_rhoI_uI3} the discontinuity of mechanical energy through the surface is equivalent to the work of the tension forces. 

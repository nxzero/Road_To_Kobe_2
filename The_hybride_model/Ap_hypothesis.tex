\section{Derivation of the hybrid model}

\subsection{Simplifying hypothesis}
\label{ap:hypothesis}

The formulation given by \ref{eq:dt_rho_I},\ref{eq:dt_rhoIu_I}and \ref{eq:dt_rhoI} remains quite general and needs some further simplifications. 
Therefore, we consider that :
(1) The mass, momentum and kinetic energy per unit of surface can be neglected, i.e. $\textbf{u}_I^0\rho_I = 0 $ and $(u_I^0)^2\rho_I = 0$. 
Implying that the surface source terms are also neglected, here $\rho_I \textbf{u}_I^0\cdot \textbf{g} = 0$; 
(2) No mass transfer is allowed at the interface, together with the previous assumption this implies that $\textbf{u}_k^0 = \textbf{u}_I^0$ for $k = 1,2$. 
(3) All molecular diffusion fluxes can be neglected, i.e. the viscous stress at the interfaces as well as the surface heat flux can be neglected.
Therefore, the surface stress tensor is assumed isotropic on the surface, and is written $\bm{\sigma}_I^0  = \gamma (\bm\delta - \textbf{nn}) = \gamma \bm\delta_{||}$ where $\gamma$ is the constant surface tension coefficient.
Under this hypothesis the internal energy and the surface tension coefficient are related through $\phi_I^0 e_I^0 = \gamma$ \cite{ishii2010thermo};
% When using a constant surface tension coefficient the interfacial mass, momentum and total energy balance equations reduce to the expressions,
Under these restrictive assumptions we obtain the following set of governing equations for the interfaces,
\begin{align}
    \label{eq:dt_rho_I}
    \textbf{u}_k = \textbf{u}_I^0, \\
    \Jump{\bm{\sigma}_k^0} 
    &=
    \divI\bm\sigma^0_{I||}
    =
    -\gamma\textbf{n}(\div \textbf{n}),
    \label{eq:surface_tension}\\
    \label{eq:dt_rhoI_EI}
    \Jump{\textbf{u}_k^0 \cdot \bm{\sigma}_k^0 - \textbf{q}_k^0}
    &=
    -\gamma\textbf{n}\cdot \textbf{u}_{I}^0(\div \textbf{n})
\end{align}
respectively. 
% The jump condition for the total energy can be separated into an expression for the kinetic and internal energy. 
By taking the dot product of \ref{eq:surface_tension} with $\textbf{u}_I^0$ and subtracting this first expression to \ref{eq:dt_rhoI_EI}, gives us
\begin{align}
    \label{eq:dt_rhoI_uI3}
    \Jump{\textbf{u}_k^0 \cdot \bm{\sigma}_k^0}
    &=
    -\gamma\textbf{n}\cdot \textbf{u}_{I}^0(\div \textbf{n})\\
    \label{eq:dt_rhoIe_I}
    \Jump{ \textbf{q}_k^0}
    &= 
     0
\end{align}
which are the interface kinetic energy and the internal interface energy jump condition, respectively. 
As witnesses by \ref{eq:dt_rhoI_uI3} the discontinuity of mechanical energy through the surface is equivalent to the work of the tension forces. 

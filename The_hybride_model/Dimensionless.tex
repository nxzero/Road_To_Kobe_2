\subsection{Dimensionless analysis}

\subsubsection*{A single particle translating}
From the particle momentum equation in stokes regime we define a particle relaxation time. 
We consider the relation, 
\begin{equation*}
    \ddt {\intO{\rho_2 \textbf{u}_2}}
    = \intS{\bm\sigma_1^0\cdot \textbf{n}_2}
    + \intO{\rho_2\textbf{g}}
\end{equation*}
Or in the stokes regime, 
\begin{equation*}
    \ddt\intO{\textbf{u}_2}
    = 
    - 6 \pi \mu_1 a /m_\alpha \textbf{u}_{pf}
\end{equation*}
Assuming a constant fluid velocity, this equation gives, 
\begin{equation*}
    \textbf{u}_\alpha(t)
    = u_0 e^{- 6 \pi \mu_1 a /m_\alpha t}
\end{equation*}
where $u_0$ is the initial fluid velocity. 
From this trivial solution it is clear that the particle relaxation time in stokes regime is, 
\begin{equation*}
    \frac{1}{\tau_a}
    = \frac{- 6 \pi \mu_1 a }{m_\alpha}
    = - \frac{9}{2} \frac{\rho_1}{\rho_2} \frac{ \nu_1 }{a^2}
\end{equation*}
The Dimensionless form of this equation is, 
\begin{equation*}
    \frac{\rho_2  a^3 U_r}{3\tau_a}\ddt\intO{\textbf{u}_2}^*
    \sim
    - \mu_1 a U_r \textbf{u}_{pf}^*
\end{equation*}
By identification we can find,  
\begin{equation*}
    \frac{1}{\tau_a}
    \sim
    - \frac{\rho_1 \nu_1}{a^2 \rho_2}
    = 
    - \frac{\mu_1}{a^2 \rho_2}
\end{equation*}

Now if we look at the particle moment of momentum equation, 
\begin{equation}
    \ddt \intO{\rho_2 \textbf{r} \textbf{w}^0_2}
    = \intO{\rho_2  \textbf{w}_2^0 \textbf{w}_2^0 - \bm{\sigma}_2^0}
    - \intS{\sigma \textbf{I}_{||}}
    + \intS{ \textbf{r}\bm{\sigma}_1^0\cdot \textbf{n}_2}. 
\end{equation}
To gain scaling arguments we must first assume that the particle internal stress follows a Newtonian law. 

\subsubsection*{A droplet in pure shear}
Similarly, we assume that the first moment of surface traction force, $\intS{ \textbf{r}\bm{\sigma}_1^0\cdot \textbf{n}_2}$, is a function of the rate of strain of the ambiant fluid $\textbf{E}$ which has a norm $|\textbf{E}| = \gamma$ in $s^{-1}$. 
We also assume a mean internal strain in the particle to be also equal to $\gamma$. 
Therefore, in dimensionless form this equation gives,
\begin{equation}
    \frac{\rho_2 a^5 \gamma}{\tau_a}
    \ddt \intO{\textbf{r} \textbf{w}^0_2}^*
    = \rho_2 \gamma^2 a^5 \intO{\rho_2  \textbf{w}_2^0 \textbf{w}_2^0}^*
    - \mu_2 \gamma a^3 \intO{\bm{\sigma}_2^0}^*
    - \sigma a^2 \intS{ \textbf{I}_{||}}^*
    + a^3 \mu_1 \gamma \intS{ \textbf{r}\bm{\sigma}_1^0\cdot \textbf{n}_2}^*. 
\end{equation}
We're dividing both side by $a^3 \mu_1 \gamma$ since we guessed the form of the stress let, then it gives, 
\begin{equation}
    \frac{\rho_2 a^2 }{\mu_1 \tau_a}
    \ddt \intO{\textbf{r} \textbf{w}^0_2}^*
    - \frac{\rho_2 \gamma a^2}{\mu_1} \intO{\rho_2  \textbf{w}_2^0 \textbf{w}_2^0}^*
    = 
    - \frac{\mu_2}{\mu_1}  \intO{\bm{\sigma}_2^0}^*
    - \frac{\sigma }{a \mu_1 \gamma} \intS{ \textbf{I}_{||}}^*
    + \intS{ \textbf{r}\bm{\sigma}_1^0\cdot \textbf{n}_2}^*. 
\end{equation}
We can now identify each of these constant to a particular number of dimension especially we have, 
\begin{equation}
    \beta_\gamma
    \ddt \intO{\textbf{r} \textbf{w}^0_2}^*
    - Re_\gamma \intO{\rho_2  \textbf{w}_2^0 \textbf{w}_2^0}^*
    = 
    - \lambda  \intO{\bm{\sigma}_2^0}^*
    - \frac{1}{Ca_\gamma} \intS{ \textbf{I}_{||}}^*
    + \intS{ \textbf{r}\bm{\sigma}_1^0\cdot \textbf{n}_2}^*. 
\end{equation}
Where, $\beta_\gamma = \frac{\rho_2 a^2 }{\mu_1 \tau_a} $ is the frequency parameter in the shearing condition.  
The relaxation time of the drop deformation can be defined according to these many sources. 
It is either $\frac{1}{\tau_a} = \frac{\mu_1}{Ca_\gamma \rho_2 a^2}$ if the capilary forces are dominant, or $\frac{1}{\tau_a} = \frac{\mu_1}{\rho_2 a^2}$ if the hydrodynamic forces are dominant. 
Or even, $\frac{1}{\tau_a} = \lambda \frac{\mu_1}{\rho_2 a^2}$ if the dominant term is the particle internal stress. 
Finally, if the inertial forces are the dominant parameter the inertial scale is, 
$\frac{1}{\tau_a} = \frac{\mu_1}{\rho_2 a^2}Re_\gamma$. 
Anyhow, count a time $\tau_a$ and the particle should return to its rest state (except for the capillary motion which tends to make oscillate the droplet). 

More generally, if our problem depends on a relative translating velocity, $U_r$, such as rising deformable droplets, then the problem takes the form, 
\begin{equation}
    \beta
    \ddt \intO{\textbf{r} \textbf{w}^0_2}^*
    - Re \intO{\rho_2  \textbf{w}_2^0 \textbf{w}_2^0}^*
    = 
    - \lambda  \intO{\bm{\sigma}_2^0}^*
    - \frac{1}{Ca} \intS{ \textbf{I}_{||}}^*
    + \intS{ \textbf{r}\bm{\sigma}_1^0\cdot \textbf{n}_2}^*. 
\end{equation}
Where the new dimensionless number are defined substituting $a\gamma \rightarrow U_r$. 

The particle second moment of mass equation, under dimensionless form yields, 
\begin{equation*}
    \frac{a}{U_r \tau_a}\ddt 
    \intO{\textbf{rr}}^*
    = \intO{\textbf{rw}_2^0+\textbf{w}_2^0 \textbf{r}}^*,
\end{equation*}
Injecting this expression into the symmetric part equation of the moment of momentum equaiton yields,
\begin{equation}
    {\beta_2}
    \ddt^2  \intO{\textbf{r} \textbf{r}}^*
    - Re \intO{\rho_2  \textbf{w}_2^0 \textbf{w}_2^0}^*
    = 
    - \lambda  \intO{\bm{\sigma}_2^0}^*
    - \frac{1}{Ca} \intS{ \textbf{I}_{||}}^*
    + \intS{ (\textbf{r}\bm{\sigma}_1^0+\bm\sigma_2^0\textbf{r})\cdot \textbf{n}_2}^*. 
\end{equation}
where $\beta_2 = \frac{\rho_2 a^3}{\mu_1 U_r \tau_a^2}$
At the first order it can be shown that $\frac{1}{Ca} \intS{ \textbf{I}_{||}}^* \sim (\intO{\textbf{r} \textbf{r}}^* - \textbf{I})$ since the surface tension tensor is directly linked to the deviatoric part of the deformation tensor.  
For a capillary dominated regime we can therefore say that, 
\begin{equation*}
    \intO{\textbf{rr}}^*,_{ik}(t) 
    = k_{1,ij} \sin(\sqrt{\beta_2/Ca} t)
    + k_{2,ij} \sin(\sqrt{\beta_2/Ca} t)
\end{equation*}
where $k_1$ and $k_2$ are related to the initial position and velocity such as
$\intO{\textbf{rr}}^*(0) = k_1+k_2 $ and $\ddt\intO{\textbf{rr}}^*(0) = k_1(Ca \beta)^{-1/2}-k_2(Ca \beta)^{-1/2} $. 

Now, let's assume that internal viscous force are not negligible such that $\lambda = \mathcal{O}(1/Ca)$. 
Then it is possible to write for Newtonian fluid that, $\lambda  \intO{\bm{\sigma}_2^0}^*\sim \lambda  \frac{a}{U_r \tau_a}\ddt \intO{\textbf{rr}}^*$. 
In the end we obtain a PDE of the form 
\begin{equation}
    {\beta_2}
    \ddt^2  \intO{\textbf{r} \textbf{r}}^*
    % - Re \intO{\rho_2  \textbf{w}_2^0 \textbf{w}_2^0}^*
    + \frac{a}{U_r \tau_a}\ddt \intO{\textbf{rr}}^*
    + \frac{1}{Ca} \intO{ \textbf{rr}}^*
    = 
    % + \intS{ (\textbf{r}\bm{\sigma}_1^0+\bm\sigma_2^0\textbf{r})\cdot \textbf{n}_2}^*. 
\end{equation}
Under this form it is clear that the three coefficient might be linked to the mass, dumping and the stiffness of a mass spring model. 
The solution wil be of the form $e^{-ct}(\sin \omega t+ \cos \omega t)$. 
This already tells us that the droplet as its whole posses a viscoelastic behavior.
If the internal Reynolds number is non-negligible then a term which scale as $(\ddt \intO{\textbf{rr}}^*)^2  \intO{\textbf{rr}}^*$ will go into the equation. 


\begin{align*}
    \ddt \textbf{C}_{\alpha,ij}'
    = 
    \textbf{C}_{\alpha,ik}' \cdot \bm\Gamma_{\alpha,kj}
    +  \bm\Gamma_{\alpha,ki} \cdot \textbf{C}_{\alpha,jk}',
    + \bm\Gamma_{\alpha,ij}
    +  \bm\Gamma_{\alpha,ji}
    \\
    \ddt( 
    \textbf{C}_{\alpha,ik}' \cdot \bm\Gamma_{\alpha,kj}
    -  \bm\Gamma_{\alpha,ki} \cdot \textbf{C}_{\alpha,jk}')
    + \ddt (\bm\Gamma_{\alpha,kj} -  \bm\Gamma_{\alpha,ki})
    = 
    \frac{5}{a^2 \rho_2 v_\alpha} (\textbf{M}_{ij} - \textbf{M}_{ji})\\
    \ddt^2 \textbf{C}_{\alpha,ik}'
    + \frac{5\mu_2}{a^2\rho_2} (
        \ddt \textbf{C}_{\alpha,ij}'
        - \textbf{C}_{\alpha,ik}' \cdot \bm\Gamma_{\alpha,kj}
        - \bm\Gamma_{\alpha,ki} \cdot \textbf{C}_{\alpha,jk}'
        )
    + \frac{4 \gamma v_\alpha }{5 a} \textbf{C}_{\alpha,ij}'
    - \bm\Gamma_{\alpha,lj}\bm\Gamma_{\alpha,ki} \textbf{C}_{\alpha,kl}'
    - \bm\Gamma_{\alpha,kj}\bm\Gamma_{\alpha,ki} 
    \\
    = 
    +\frac{2 \gamma v_\alpha }{a} \textbf{I}_{ij} 
    +\frac{5}{a^2 \rho_2 v_\alpha} \frac{1}{2}(\textbf{M}_{ij}+\textbf{M}_{ji} + 2f_{\textbf{ww},ij} + 2f_{\bm\sigma,ij})
    % \label{eq:dt2_C}
\end{align*}
\begin{align*}
    \ddt \textbf{C}_{\alpha,ij}'
    = 
    \textbf{C}_{\alpha,ik}' \cdot \bm\Gamma_{\alpha,kj}
    +  \bm\Gamma_{\alpha,ki} \cdot \textbf{C}_{\alpha,jk}',
    + \bm\Gamma_{\alpha,ij}
    +  \bm\Gamma_{\alpha,ji}
    \\
    \ddt( 
    \textbf{C}_{\alpha,ik}' \cdot \bm\Gamma_{\alpha,kj}
    -  \bm\Gamma_{\alpha,ki} \cdot \textbf{C}_{\alpha,jk}')
    + \ddt (\bm\Gamma_{\alpha,kj} -  \bm\Gamma_{\alpha,ki})
    = 
    \frac{5}{a^2 \rho_2 v_\alpha} (\textbf{M}_{ij} - \textbf{M}_{ji})\\
    \ddt^2 \textbf{C}_{\alpha,ik}'
    + \frac{5\mu_2}{a^2\rho_2} (
        \ddt \textbf{C}_{\alpha,ij}'
        - \textbf{C}_{\alpha,ik}' \cdot \bm\Gamma_{\alpha,kj}
        - \bm\Gamma_{\alpha,ki} \cdot \textbf{C}_{\alpha,jk}'
        )
    + \frac{4 \gamma v_\alpha }{5 a} \textbf{C}_{\alpha,ij}'
    - \bm\Gamma_{\alpha,lj}\bm\Gamma_{\alpha,ki} \textbf{C}_{\alpha,kl}'
    - \bm\Gamma_{\alpha,kj}\bm\Gamma_{\alpha,ki} 
    \\
    = 
    +\frac{2 \gamma v_\alpha }{a} \textbf{I}_{ij} 
    +\frac{5}{a^2 \rho_2 v_\alpha} \frac{1}{2}(\textbf{M}_{ij}+\textbf{M}_{ji} + 2f_{\textbf{ww},ij} + 2f_{\bm\sigma,ij})
    \label{eq:dt2_C}
\end{align*}


RAW momentum eq :
\begin{multline}    
    \frac{m_\alpha a^2}{10}\ddt^2 \textbf{C}_\alpha
    =  \frac{m_\alpha a^2}{5}[
    \bm\Gamma_{\alpha,lj}\bm\Gamma_{\alpha,ki} \textbf{C}_{\alpha,kl} + f_\textbf{ww}]
    - \mu_2 v_\alpha [(\bm \Gamma_{p,ij}+\bm \Gamma_{p,ji})
    + f_{\bm{\sigma}}]\\
        - \frac{2 \gamma v_\alpha }{a} \textbf{I}_{ij} 
        + \frac{4 \gamma v_\alpha }{5 a} (\textbf{C}_{ij} - \textbf{I}_{ij})
        + \intS{(\textbf{r}\bm\sigma_2^0+ \bm\sigma_2^0\textbf{r})\cdot \textbf{n}}
\end{multline}
Assuming $\bm\Gamma_\alpha'  = \frac{a}{U}\bm\Gamma$ and that  $\intS{(\textbf{r}\bm\sigma_2^0+ \bm\sigma_2^0\textbf{r})\cdot \textbf{n}} \sim v_\alpha \mu \tau$ where $\tau = [s^{-1}]$ is the inverse timescale.  
\begin{multline}    
    \frac{m_\alpha a^2 \tau_a^2}{5}\ddt^2_* \textbf{C}_\alpha
    =  \frac{m_\alpha a^2 \tau^2}{5}[
    \bm\Gamma_{\alpha,lj}^* \bm\Gamma_{\alpha,ki}^* \textbf{C}_{\alpha,kl} + f_\textbf{ww}]
    - \tau \mu_2 v_\alpha [(\bm \Gamma_{p,ij}^*+\bm \Gamma_{p,ji}^*)
    + f_{\bm{\sigma}}]\\
        - \frac{2 \gamma v_\alpha }{a} \textbf{I}_{ij} 
        + \frac{4 \gamma v_\alpha }{5 a} (\textbf{C}_{ij} - \textbf{I}_{ij})
        + v_\alpha \mu \tau\intS{(\textbf{r}\bm\sigma_2^0+ \bm\sigma_2^0\textbf{r})\cdot \textbf{n}}^*
\end{multline}
Now multipliyng both side by, $5/(m_\alpha a^2)$ gives
\begin{multline}    
    \tau_a^2 \ddt^2_* \textbf{C}_\alpha
    =  \tau^2[
    \bm\Gamma_{\alpha,lj}^* \bm\Gamma_{\alpha,ki}^* \textbf{C}_{\alpha,kl} + f_\textbf{ww}]
    - \frac{5 \tau \mu_2}{a^2 \rho_2} [(\bm \Gamma_{p,ij}^*+\bm \Gamma_{p,ji}^*)
    + f_{\bm{\sigma}}]\\
        - \frac{10 \gamma }{a^3\rho_2} \textbf{I}_{ij} 
        + \frac{4 \gamma }{a^3 \rho_2} (\textbf{C}_{ij} - \textbf{I}_{ij})
        + \frac{5  \mu_1 \tau}{\rho_2a^2} 
        \intS{(\textbf{r}\bm\sigma_2^0+ \bm\sigma_2^0\textbf{r})\cdot \textbf{n}}^*
\end{multline}
It is in fact better to divide both side by $v_\alpha \mu_1 \tau$, 
\begin{multline}    
    \frac{\rho_2 a^2 \tau_a^2}{5\mu_1 \tau}\ddt^2_* \textbf{C}_\alpha
    =  \frac{\rho_2 a^2 \tau^2}{5 \mu_1 \tau}[
    \bm\Gamma_{\alpha,lj}^* \bm\Gamma_{\alpha,ki}^* \textbf{C}_{\alpha,kl} + f_\textbf{ww}]
    - \frac{\mu_2}{\mu_1}  [(\bm \Gamma_{p,ij}^*+\bm \Gamma_{p,ji}^*)
    + f_{\bm{\sigma}}]\\
        - \frac{2 \gamma }{\mu_1 \tau a} \textbf{I}_{ij} 
        + \frac{4 \gamma }{5 a \mu_1 \tau} (\textbf{C}_{ij} - \textbf{I}_{ij})
        + \intS{(\textbf{r}\bm\sigma_2^0+ \bm\sigma_2^0\textbf{r})\cdot \textbf{n}}^*
\end{multline}
Now we substitute the matrix $\textbf{C}_\alpha$, 
\begin{multline}    
    \frac{\rho_2 a^2 \tau_a^2}{5\mu_1 \tau}\ddt^2_* \textbf{C}_\alpha^*
    =  \frac{\rho_2 a^2 \tau^2}{5 \mu_1 \tau}[
    \bm\Gamma_{\alpha,lj}^* \bm\Gamma_{\alpha,ki}^* \textbf{C}_{\alpha,kl}^* 
    + \bm\Gamma_{\alpha,kj}^* \bm\Gamma_{\alpha,ki}^* 
    + f_\textbf{ww}]\\
    - \frac{\mu_2}{\mu_1}  [
        \frac{\tau_a}{\tau}\ddt \textbf{C}_{\alpha,ij}^*
        - \textbf{C}_{\alpha,ik}^* \cdot \bm\Gamma_{\alpha,kj}^*
        -  \bm\Gamma_{\alpha,ki}^* \cdot \textbf{C}_{\alpha,jk}^*,
    + f_{\bm{\sigma}}]\\
    + \frac{4 \gamma }{5 a \mu_1 \tau} \textbf{C}_{ij}^*
        - \frac{2 \gamma }{\mu_1 \tau a} \textbf{I}_{ij} 
        + \intS{(\textbf{r}\bm\sigma_2^0+ \bm\sigma_2^0\textbf{r})\cdot \textbf{n}}^*
\end{multline}

\paragraph*{INTERTIAL REGIME}
\begin{multline}    
    \frac{m_\alpha a^2 \tau_a^2}{5}\ddt^2_* \textbf{C}_\alpha
    =  \frac{m_\alpha a^2 \tau^2}{5}[
    \bm\Gamma_{\alpha,lj}^* \bm\Gamma_{\alpha,ki}^* \textbf{C}_{\alpha,kl} + f_\textbf{ww}]
    - \tau \mu_2 v_\alpha (\bm \Gamma_{p,ij}^*+\bm \Gamma_{p,ji}^*)
    + v_\alpha a^2 \rho_2 \tau^2 f_{\bm{\sigma}}\\
        - \frac{2 \gamma v_\alpha }{a} \textbf{I}_{ij} 
        + \frac{4 \gamma v_\alpha }{5 a} (\textbf{C}_{ij} - \textbf{I}_{ij})
        + v_\alpha \rho_1 a^2 \tau^2 \intS{(\textbf{r}\bm\sigma_2^0+ \bm\sigma_2^0\textbf{r})\cdot \textbf{n}}^*
\end{multline}

\begin{multline}    
    \frac{\zeta \beta^2}{5}\ddt^2_* \textbf{C}_\alpha
    =  \frac{\zeta \beta^2}{5}[
    \bm\Gamma_{\alpha,lj}^* \bm\Gamma_{\alpha,ki}^* \textbf{C}_{\alpha,kl} + f_\textbf{ww}]
    - \frac{\lambda}{Re} (\bm \Gamma_{p,ij}^*+\bm \Gamma_{p,ji}^*)
    +\zeta f_{\bm{\sigma}}\\
        - \frac{2 }{We} \textbf{I}_{ij} 
        + \frac{4  }{5 We} (\textbf{C}_{ij} - \textbf{I}_{ij})
        +  \intS{(\textbf{r}\bm\sigma_2^0+ \bm\sigma_2^0\textbf{r})\cdot \textbf{n}}^*
\end{multline}
\begin{multline}    
    \frac{\zeta \beta^2}{5}\ddt^2_* \textbf{C}_\alpha^*
    =  \frac{\zeta \beta^2}{5}[
        \bm\Gamma_{\alpha,lj}^* \bm\Gamma_{\alpha,ki}^* \textbf{C}_{\alpha,kl}^* 
        + 
        \bm\Gamma_{\alpha,kj}^* \bm\Gamma_{\alpha,ki}^* 
        + f_\textbf{ww}]
    - \frac{\lambda}{Re} (
        \beta\ddt_* \textbf{C}_{\alpha,ij}^*
    - 
    \textbf{C}_{\alpha,ik}^* \cdot \bm\Gamma_{\alpha,kj}^*
    +  \bm\Gamma_{\alpha,ki}^* \cdot \textbf{C}_{\alpha,jk}^*
    )
    \\
    +\zeta f_{\bm{\sigma}}
        - \frac{2 }{We} \textbf{I}_{ij} 
        + \frac{4  }{5 We} \textbf{C}_{\alpha,ij}^*
        +  \intS{(\textbf{r}\bm\sigma_2^0+ \bm\sigma_2^0\textbf{r})\cdot \textbf{n}}^*
\end{multline}
\begin{multline}    
    \frac{\zeta \beta^2}{5}\ddt^2_* \textbf{C}_\alpha^*
    + \frac{\lambda}{Re} (
        \beta\ddt_* \textbf{C}_{\alpha,ij}^*
    - 
    \textbf{C}_{\alpha,ik}^* \cdot \bm\Gamma_{\alpha,kj}^*
    +  \bm\Gamma_{\alpha,ki}^* \cdot \textbf{C}_{\alpha,jk}^*
    )
    - \frac{4  }{5 We} \textbf{C}_{\alpha,ij}^*\\
    =  \frac{\zeta \beta^2}{5}[
        \bm\Gamma_{\alpha,lj}^* \bm\Gamma_{\alpha,ki}^* \textbf{C}_{\alpha,kl}^* 
        + 
        \bm\Gamma_{\alpha,kj}^* \bm\Gamma_{\alpha,ki}^* 
        + f_\textbf{ww}]
        +\zeta f_{\bm{\sigma}}
        - \frac{2 }{We} \textbf{I}_{ij} 
        +  \intS{(\textbf{r}\bm\sigma_2^0+ \bm\sigma_2^0\textbf{r})\cdot \textbf{n}}^*
\end{multline}


If we rearrange the terms weobtain 

\begin{align*}    
    \frac{\zeta}{5}\left[
        \beta^2\ddt^2_* \textbf{C}_\alpha^*
        - \bm\Gamma_{\alpha,lj}^* \bm\Gamma_{\alpha,ki}^* \textbf{C}_{\alpha,kl}^* 
        - \bm\Gamma_{\alpha,kj}^* \bm\Gamma_{\alpha,ki}^* 
        - f_\textbf{ww}
    \right]\\
    + \frac{\lambda}{Re} \left[
        \beta\ddt_* \textbf{C}_{\alpha,ij}^*
    - 
    \textbf{C}_{\alpha,ik}^* \cdot \bm\Gamma_{\alpha,kj}^*
    +  \bm\Gamma_{\alpha,ki}^* \cdot \textbf{C}_{\alpha,jk}^*
    \right]\\
    - \frac{1}{We}\left[
        \frac{4}{5}\textbf{C}_{\alpha,ij}^*
        - 2\textbf{I}_{ij}
    \right]
    = 
    +  \intS{(\textbf{r}\bm\sigma_2^0+ \bm\sigma_2^0\textbf{r})\cdot \textbf{n}}^*
\end{align*}
\paragraph*{Low viscosity bubbles regime }
\begin{equation}    
    - \frac{4  }{5 We} \textbf{C}_{\alpha,ij}^*\\
    =  
    - \frac{2 }{We} \textbf{I}_{ij} 
    +  \intS{(\textbf{r}\bm\sigma_2^0+ \bm\sigma_2^0\textbf{r})\cdot \textbf{n}}^*
\end{equation}
As a matter of fact a bubble in water oscillate. 
Therefore $\intS{(\textbf{r}\bm\sigma_2^0+ \bm\sigma_2^0\textbf{r})\cdot \textbf{n}}^*\sim \ddt^2_* \textbf{C}^*_{\alpha,ij}$

\paragraph*{The invicid case : }
\begin{multline}    
    \frac{\zeta \beta^2}{5}\ddt^2_* \textbf{C}_\alpha^*
    % + \frac{\lambda}{Re} (
    %     \beta\ddt_* \textbf{C}_{\alpha,ij}^*
    % - 
    % \textbf{C}_{\alpha,ik}^* \cdot \bm\Gamma_{\alpha,kj}^*
    % +  \bm\Gamma_{\alpha,ki}^* \cdot \textbf{C}_{\alpha,jk}^*
    % )
    - \frac{4  }{5 We} \textbf{C}_{\alpha,ij}^*\\
    =  \frac{\zeta \beta^2}{5}[
        \bm\Gamma_{\alpha,lj}^* \bm\Gamma_{\alpha,ki}^* \textbf{C}_{\alpha,kl}^* 
        + 
        \bm\Gamma_{\alpha,kj}^* \bm\Gamma_{\alpha,ki}^* 
        + f_\textbf{ww}]
        +\zeta f_{\bm{\sigma}}
        - \frac{2 }{We} \textbf{I}_{ij} 
        +  \intS{(\textbf{r}\bm\sigma_2^0+ \bm\sigma_2^0\textbf{r})\cdot \textbf{n}}^*
\end{multline}

\paragraph*{The high Weber case}
\begin{multline}    
     \frac{\zeta \beta^2}{5}\ddt^2_* \textbf{C}_\alpha^*
    +   \frac{\lambda}{Re} (
        \beta\ddt_* \textbf{C}_{\alpha,ij}^*
    - 
    \textbf{C}_{\alpha,ik}^* \cdot \bm\Gamma_{\alpha,kj}^*
    +  \bm\Gamma_{\alpha,ki}^* \cdot \textbf{C}_{\alpha,jk}^*
    )\\
    % - \frac{4  }{5} \textbf{C}_{\alpha,ij}^*\\
    =   \frac{\zeta \beta^2}{5}[
        \bm\Gamma_{\alpha,lj}^* \bm\Gamma_{\alpha,ki}^* \textbf{C}_{\alpha,kl}^* 
        + 
        \bm\Gamma_{\alpha,kj}^* \bm\Gamma_{\alpha,ki}^* 
         +f_\textbf{ww}]
        +\zeta f_{\bm{\sigma}}
        +  \intS{(\textbf{r}\bm\sigma_2^0+ \bm\sigma_2^0\textbf{r})\cdot \textbf{n}}^*
\end{multline}
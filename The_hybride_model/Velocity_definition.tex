\section{Proof of the generalized form of the interfacial balance law}
\label{ap:interface_proof}
% In \citet[Appendix 2]{marle1982macroscopic} they demonstrated how to obtain \ref{eq:dt_delta_I_f_I} in the specific context of the mass, momentum and energy equations. 
For completeness and ease of understanding we give in this appendix the detailed derivation of \ref{eq:dt_delta_I_f_I}. 
Let us first introduce some important relations regarding surface properties. 
Consider the arbitrary tensor $ \textbf{F}_{I}$ defined on $\Gamma(t)$.
Let us take the surface divergence of $\textbf{F}_{I||} = \textbf{F}_I \cdot (\bm\delta -\textbf{nn})$, it yields,
\begin{align}
    \delta_I \divI \textbf{F}_{I||}
    &= 
    \div (\delta_I \textbf{F}_{I||})
    - \textbf{n}(\textbf{n}\cdot\grad)\cdot (\delta_I \textbf{F}_{I||})
    - \textbf{F}_{I}\cdot\gradI\delta_I \nonumber\\
    &= 
    \div (\delta_I \textbf{F}_{I||})
    - \textbf{n}(\textbf{n}\cdot\grad)\cdot (\delta_I \textbf{F}_{I||})
    - \textbf{F}_{I} \delta_I (\textbf{n}\cdot\grad)\textbf{n}
    % &= 
    % \div (\delta_I \textbf{F}_{I||})
    \label{eq:step1}
\end{align}
were we used the chain rule and the relation $\divI(\ldots) = \div(\ldots) - \textbf{n}(\textbf{n}\cdot \grad)\cdot(\ldots)$ for the first equality. 
The second equality is derived using \ref{eq:grad_delta_I} dotted with $(\bm\delta -\textbf{nn})$, which gives the relation 
\begin{equation*}
    (\bm\delta - \textbf{nn}) \cdot \grad\delta_I
    = \gradI \delta_I
    = 
    \delta_I (\textbf{n}\cdot \grad)\textbf{n}.
\end{equation*}
Expanding the second term of \ref{eq:step1} by noticing that $\textbf{F}_{I||} = (\bm\delta - \textbf{nn})\cdot \textbf{F}_I$, and using the expression $\textbf{n}(\textbf{n}\cdot\grad)\cdot (\bm\delta - \textbf{nn}) =  (\textbf{n}\cdot\grad)\textbf{n}$
% \begin{equation*}
%     - \textbf{n}(\textbf{n}\cdot\grad)\cdot (\delta_I \textbf{F}_{I||})
%     = 
%     - \delta_I \textbf{F}_{I}\textbf{n}(\textbf{n}\cdot\grad)\cdot (\bm\delta - \textbf{nn})
%     + (\bm\delta - \textbf{nn}) \cdot \textbf{n}(\textbf{n}\cdot\grad)(\delta_I \textbf{F}_{I})
%     = \textbf{F}_{I} \delta_I (\textbf{n}\cdot\grad)\textbf{n},
% \end{equation*}
 leads us to the useful relation :
\begin{equation}
    \delta_I \divI \textbf{F}_{I||}
    = 
    \div (\delta_I \textbf{F}_{I||}). 
    \label{eq:proof_1}
\end{equation}
% Making use of the same principles one can show the already common expression  
% \begin{equation}
%     \divI
%     \textbf{F}_{I}
%     = 
%     \divI
%     \textbf{F}_{I||}
%     +(\textbf{F}_I \cdot \textbf{n})  \div \textbf{n},
% \end{equation}
% which, multiplied with $\delta_I$ gives 
% \begin{equation}
%     \delta_I \divI
%     \textbf{F}_{I}
%     = 
%     \div
%     (\delta_I \textbf{F}_{I||})
%     +\delta_I (\textbf{F}_I \cdot \textbf{n})  \div \textbf{n}. 
%     \label{eq:proof_1}
% \end{equation}


To demonstrate that \ref{eq:dt_delta_I} and \ref{eq:dt_delta_I_f_I} are consistent we must prove the equality 
\begin{equation}
    \delta_I
    \left[ \pddt f_I^0 
    + f_I^0 (\textbf{u}_{I}^0\cdot \textbf{n})  (\div \textbf{n})
    +\divI
    (f_I^0 \textbf{u}_{I||}^0
    - \mathbf{\Phi}_{I||}^0 )
    \right]
    =
    \pddt (\delta_If_I^0) 
    +\div
    (\delta_If_I^0 \textbf{u}_I^0
        - \delta_I\mathbf{\Phi}_{I||}^0 ).
    \label{eq:to_prove}
\end{equation}
This is easily done  using \ref{eq:dt_f_I} on the two first terms on the left-hand side of \ref{eq:to_prove} which gives
\begin{equation*}
    \delta_I \pddt f_I^0 
    + f_I^0 \delta_I (\textbf{u}_I\cdot\textbf{n})(\div \textbf{n})
    = 
     \pddt (f_I^0 \delta_I)
    + \div(f_I^0  \delta_I \textbf{n}(\textbf{n}\cdot\textbf{u}_I^0)). 
    % - \delta_I (\textbf{n}\cdot\textbf{u}_I^0) (\textbf{n}\cdot \grad)f_I^0. 
\end{equation*}
Where it is assumed that $\delta_I (\textbf{n}\cdot\textbf{u}_I^0)(\textbf{n}\cdot\grad) f_I^0 = 0$ for reasons discussed in detailed in \citet{orlando2023evolution,estrada1985distributional}. 
Then, using the relation \ref{eq:proof_1} on the remaining terms on the left-hand side of \ref{eq:to_prove} directly proves \ref{eq:to_prove} and by extension \ref{eq:dt_delta_I_f_I}. 




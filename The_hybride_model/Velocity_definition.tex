\section{Derivation of the point velocity}
\label{ap:velocity_definition}
In this Appendix we derive the velocity of the center of mass of a particle $\alpha$. Consider a particle of center of mass $\textbf{y}_\alpha$ defined such as
\begin{equation*}
    m_\alpha(t) \textbf{y}_\alpha(t)
    = \int_{\Omega_\alpha} \rho_2 \textbf{y}_2 d\Omega,
\end{equation*}
where we empathize that both, $m_\alpha(t)$ and $\textbf{y}_\alpha(t)$ are function of time. 
The center of mass velocity is defined as the derivative of $\textbf{y}_\alpha(t)$ within time.
Yielding, 
\begin{align*}
    \ddt \textbf{y}_\alpha (t)
    &=
    \ddt \left(
        \frac{1}{m_\alpha} \int_{\Omega_\alpha} \rho_2 \textbf{y}_2 d\Omega
    \right)\\
    &= \frac{1}{m_\alpha}
    \ddt 
    \left(
        \int_{\Omega_\alpha} \rho_2 \textbf{y} d\Omega
    \right)
    - \frac{1}{m_\alpha^2} \ddt \int_{\Omega_\alpha} \rho_2 d\Omega \int_{\Omega_\alpha} \rho_2 \textbf{y}_2 d\Omega
    \\
    &= \frac{1}{m_\alpha}\int_{\Omega_\alpha} \left[
        \pddt (\rho_2 \textbf{y}) + \div\left(\rho_2 \textbf{y}\textbf{u}_2\right) 
    \right]d\Omega \\
    &+ \frac{1}{m_\alpha}\int_{\Sigma_\alpha} \textbf{y} \rho_2(\textbf{u}_I   - \textbf{u}_2) \cdot \textbf{n}_2 d \Sigma
    -  \frac{1}{m_\alpha^2} \int_{\Sigma_\alpha} \rho_2(\textbf{u}_I   - \textbf{u}_2) \cdot \textbf{n}_2 d\Sigma  \int_{\Omega_\alpha} \rho_2 \textbf{y}_2 d\Omega
    \\
    &= \frac{1}{m_\alpha}\int_{\Omega_\alpha} \textbf{y} \left[
    \pddt (\rho_2) + \div\left(\rho_2 \textbf{u}_2\right) 
    \right]d\Omega
    + \frac{1}{m_\alpha}\int_{\Omega_\alpha} \rho_2  \textbf{u}_2  \cdot \grad \textbf{y} d\Omega \\
    &+ \frac{1}{m_\alpha}\int_{\Sigma_\alpha} \textbf{y}_2 \rho_2 (\textbf{u}_I - \textbf{u}_2) \cdot \textbf{n}_2 d \Sigma
    - \frac{1}{m_\alpha}  \textbf{y}_\alpha \int_{\Sigma_\alpha} \rho_2(\textbf{u}_I   - \textbf{u}_2) \cdot \textbf{n}_2 d\Sigma
\end{align*}
By considering the mass conservation on a single fluid particle for the first term, noticing that $\grad \textbf{y} = \textbf{I}$ where $\textbf{I}$ is the identity tensor for the second term, and introducing $\mathbf{r} = \mathbf{y} - \mathbf{y}_\alpha$ for the last two terms gives, 
\begin{equation*}
    \textbf{u}_\alpha
    = \frac{1}{m_\alpha} \left(
        \int_{\Omega_\alpha} \rho_2 \textbf{u}_2 d\Omega
        +  \int_{\Sigma_\alpha} \rho_2 \textbf{r}  (\textbf{u}_I - \textbf{u}_2) \cdot \textbf{n}_2 d\Sigma
    \right).
    \label{eq:vel_def}
\end{equation*}

\section{Leibniz and Gauss divergence theorem for volume and surface integral}
\label{ap:math}

In this appendix we recall the form of the Leibnitz rules, Gauss integral rule and general Reynolds transport theorem. 
\subsubsection*{Volume integral over $\Omega_\alpha$}
For a surface integral over a closed surface the Gauss divergence theorem reads : 
\begin{equation}
    \int_{\Sigma_\alpha} \textbf{f} \cdot \textbf{n}_2 d\Sigma
    = \int_{\Omega_\alpha} \div \textbf{f}d\Omega
\end{equation}

To differentiate time varying integral we make use of the Reynolds transport theorem.
For a quantity $\textbf{f}$ of arbitrary tensorial order, it is defined  as follows in our notation, 
% \begin{equation}
%     \ddt \int_{\Omega_\alpha} \textbf{f} d\Omega
%     = \int_{\Omega_\alpha}\pddt \textbf{f}  d\Omega
%     + \int_{\Sigma_\alpha} \textbf{f} \textbf{u}_I \cdot \textbf{n}_2 d\Sigma,
% \end{equation}
% then by adding and subtracting  $\int_{\Sigma_\alpha} \textbf{f} \textbf{u}_2 \cdot \textbf{n}_2 d\Sigma$ on the RHS we arrive to the more practical expression,
\begin{equation}
    \ddt \int_{\Omega_\alpha} \textbf{f} d\Omega
    = \int_{\Omega_\alpha}\left[\pddt \textbf{f} + \div\left(\textbf{f}\textbf{u}_2\right) \right]d\Omega\\
    + \int_{\Sigma_\alpha} \textbf{f} (\textbf{u}_I-\textbf{u}_2)\cdot \textbf{n}_2 d\Sigma,
    \label{eq:Reynolds}
\end{equation}


\subsubsection*{Surface integral over $\Sigma_\alpha$}

In this work we are solely interested in closed surface topology. 
For a clear demonstration of the Reynolds transport and divergence theorem for interfaces we refer the reader to the work of \citet{nadim1996concise}. 
For closed surface integral the Gauss divergence theorem reads as :
\begin{equation}
    \int_{\Sigma_\alpha}  \divI \textbf{f} d\Sigma
    = 
    \int_{\Sigma_\alpha}  \textbf{f} \cdot \textbf{n}\divI\textbf{n} d\Sigma. 
    \label{eq:surf_div_theorem}
\end{equation}
Additional, The time derivative of any surface integral can then be obtained using Leibniz rule, which reads as  
\begin{equation}
    \ddt \int_{\Sigma_\alpha} \textbf{f} d\Sigma 
    = \int_{\Sigma_\alpha} \left[
        \pddt \textbf{f} 
        +   \divI (\textbf{u}_I\textbf{f})
    \right]d\Sigma,
    \label{eq:Leibnitz}
\end{equation}
The formulation of \ref{eq:Leibnitz} is sometime preferred as the partial time derivative exist only on the surface. 

\section{A detailed derivation of the moments conservation equaitons}
\label{ap:moment_derivative}
In this appendix we propose a detailed derivation of the moments equations. 
The first moment or dipoles of any property $q_\alpha$ can be defined as,
\begin{equation*}
    \mathcal{Q}_\alpha 
    = \int_{\Omega_\alpha} \textbf{r} f_2 d\Omega
\end{equation*}
We use the Reynolds transport theorem to describe the evolution of $\mathcal{Q}_\alpha$ within time. 
It gives, 
\begin{align*}
    \ddt \mathcal{Q}_\alpha
      &=  \int_{\Omega_\alpha} \left[
        \pddt(\textbf{r}  f_2)
        + \div \left(f \textbf{r} \textbf{u}_2\right)
    \right]d\Omega + \int_{\Sigma_\alpha} \textbf{r}  f_2  (\textbf{u}_I-\textbf{u}_2)\cdot \textbf{n}_2  d\Sigma  \nonumber \\
    &=  \int_{\Omega_\alpha} \textbf{r}\left[
        \pddt f_2
        + \div \left(f_2 \textbf{u}_2\right)
    \right] d\Omega
    + \int_{\Omega_\alpha} f_2 \left[
        \pddt \textbf{r}
        +\textbf{u}_2 \grad \textbf{r}
    \right]d\Omega\\
    &+ \int_{\Sigma_\alpha} \textbf{r}  f_2 (\textbf{u}_I-\textbf{u}_2)\cdot \textbf{n}_2  d\Sigma,
\end{align*}
Using \ref{eq:dt_f_k} for the first term, and considering the relation,
$  \pddt \textbf{r}
+ \textbf{u}_2 \cdot \grad \textbf{r}
= - \frac{d}{dt} \textbf{y}_\alpha  + \textbf{u}_2 \cdot \textbf{I}
= \textbf{w}_2$,
for the second yields the relation,
\begin{align*}
    \ddt \mathcal{Q}_\alpha
    &= \int_{\Omega_\alpha} \textbf{r} \left[
         \textbf{S}_2 +  \div \mathbf{\Phi}_2
    \right]d\Omega
    +\int_{\Omega_\alpha} f_2  \textbf{w}_2 d\Omega
    + \int_{\Sigma_\alpha} \textbf{r}  f_2 (\textbf{u}_I-\textbf{u}_2)\cdot \textbf{n}_2  d\Sigma,\\
    &= \int_{\Omega_\alpha} \left( 
        \textbf{r} \textbf{S}_2 
        - \mathbf{\Phi}_2
        + f_2  \textbf{w}_2 
    \right) d\Omega
    + \int_{\Sigma_\alpha} \textbf{r} \left[
        \mathbf{\Phi}_2
        + f_2 (\textbf{u}_I-\textbf{u}_2)
    \right]\cdot \textbf{n}_2  d\Sigma.
\end{align*}
To pass from the first line to the second lines we noticed that $\int_{\Omega_\alpha} \textbf{r}  \div \mathbf{\Phi}_2 d\Omega
= \int_{\Sigma_\alpha} \textbf{r} \mathbf{\Phi}_2 \cdot \textbf{n}_2 d\Sigma
- \int_{\Omega_\alpha} \mathbf{\Phi}_2 d\Omega$. 
% \JL{Dans le corps du texte tu notes $d\Sigma$ et $d\Omega$ les differentielles des surfaces et des volumes. Merci de faire la meme chose en annexe. Par ailleurs je trouve qu'il manque des explications pour passser de 78 a 79 (j'imagine que tu utilises la conservation du moment d'ordre 0) et pour passer de 79 a 80 ou tu dois faire une integration part partie. Encore faut il le preciser ...}


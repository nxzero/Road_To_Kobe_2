%\section{Derivation of the point velocity}
%\label{ap:velocity_definition}
%In this Appendix we derive the velocity of the center of mass of a particle $\alpha$. Consider a particle of center of mass $\textbf{y}_\alpha$ defined such as
%\begin{equation*}
%    m_\alpha(t) \textbf{y}_\alpha(t)
%    = \int_{\Omega_\alpha} \rho_2 \textbf{y}_2 d\Omega,
%\end{equation*}
%where we empathize that both, $m_\alpha(t)$ and $\textbf{y}_\alpha(t)$ are function of time. 
%The center of mass velocity is defined as the derivative of $\textbf{y}_\alpha(t)$ within time.
%Yielding, 
%\begin{align*}
%    \frac{d}{dt} \textbf{y}_\alpha (t)
%    &=
%    \frac{d}{dt} \left(
%        \frac{1}{m_\alpha} \int_{\Omega_\alpha} \rho_2 \textbf{y}_2 d\Omega
%    \right)\\
%    &= \frac{1}{m_\alpha}
%    \frac{d}{dt} 
%    \left(
%        \int_{\Omega_\alpha} \rho_2 \textbf{y} d\Omega
%    \right)
%    - \frac{1}{m_\alpha^2} \frac{d}{dt} \int_{\Omega_\alpha} \rho_2 d\Omega \int_{\Omega_\alpha} \rho_2 \textbf{y}_2 d\Omega
%    \\
%    &= \frac{1}{m_\alpha}\int_{\Omega_\alpha} \left[
%        \pddt (\rho_2 \textbf{y}) + \div\left(\rho_2 \textbf{y}\textbf{u}_2\right) 
%    \right]d\Omega \\
%    &+ \frac{1}{m_\alpha}\int_{\Sigma_\alpha} \textbf{y} \rho_2(\textbf{u}_I   - \textbf{u}_2) \cdot \textbf{n}_2 d \Sigma
%    -  \frac{1}{m_\alpha^2} \int_{\Sigma_\alpha} \rho_2(\textbf{u}_I   - \textbf{u}_2) \cdot \textbf{n}_2 d\Sigma  \int_{\Omega_\alpha} \rho_2 \textbf{y}_2 d\Omega
%    \\
%    &= \frac{1}{m_\alpha}\int_{\Omega_\alpha} \textbf{y} \left[
%    \pddt (\rho_2) + \div\left(\rho_2 \textbf{u}_2\right) 
%    \right]d\Omega
%    + \frac{1}{m_\alpha}\int_{\Omega_\alpha} \rho_2  \textbf{u}_2  \cdot \grad \textbf{y} d\Omega \\
%    &+ \frac{1}{m_\alpha}\int_{\Sigma_\alpha} \textbf{y}_2 \rho_2 (\textbf{u}_I - \textbf{u}_2) \cdot \textbf{n}_2 d \Sigma
%    - \frac{1}{m_\alpha}  \textbf{y}_\alpha \int_{\Sigma_\alpha} \rho_2(\textbf{u}_I   - \textbf{u}_2) \cdot \textbf{n}_2 d\Sigma
%\end{align*}
%By considering the mass conservation on a single fluid particle for the first term, noticing that $\grad \textbf{y} = \textbf{I}$ where $\textbf{I}$ is the identity tensor for the second term, and introducing $\mathbf{r} = \mathbf{y} - \mathbf{y}_\alpha$ for the last two terms gives, 
%\begin{equation*}
%    \textbf{u}_\alpha
%    = \frac{1}{m_\alpha} \left(
%        \int_{\Omega_\alpha} \rho_2 \textbf{u}_2 d\Omega
%        +  \int_{\Sigma_\alpha} \rho_2 \textbf{r}  (\textbf{u}_I - \textbf{u}_2) \cdot \textbf{n}_2 d\Sigma
%    \right).
%    \label{eq:vel_def}
%\end{equation*}

\section{Leibniz and Gauss divergence theorem for volume and surface integral}
\label{ap:math}
\tb{
\begin{itemize}
\item Le theoreme de Reynolds est un resultat pour un volume materiel (donc OK pr la premiere partie). Vu que l'on parle de Surface materiel je parlerais egalement de transport de Reynolds pr une surface materielle. D'ailleurs concernant cete derniere il me semble qu'on a de petites differences avec Nadim. Qu'en penses tu ?
\item c'est quoi $\mathbf{n}$
\item je ne comprends pas le theoreme A.3 : on passe d'une integrale surfacique a une integrale surfacique !
\end{itemize}
}


In this appendix we recall the form of the Leibnitz rules, Gauss integral rule and general Reynolds transport theorem. 
\subsubsection*{Volume integral over $\Omega_\alpha$}
For a surface integral over a closed surface the Gauss divergence theorem reads : 
\begin{equation}
    \int_{\Sigma_\alpha} \textbf{f} \cdot \textbf{n}_2 d\Sigma
    = \int_{\Omega_\alpha} \div \textbf{f}d\Omega
\end{equation}

To differentiate time varying integral we make use of the Reynolds transport theorem.
For a quantity $\textbf{f}$ of arbitrary tensorial order, it is defined  as follows in our notation, 
% \begin{equation}
%     \frac{d}{dt} \int_{\Omega_\alpha} \textbf{f} d\Omega
%     = \int_{\Omega_\alpha}\pddt \textbf{f}  d\Omega
%     + \int_{\Sigma_\alpha} \textbf{f} \textbf{u}_I \cdot \textbf{n}_2 d\Sigma,
% \end{equation}
% then by adding and subtracting  $\int_{\Sigma_\alpha} \textbf{f} \textbf{u}_2 \cdot \textbf{n}_2 d\Sigma$ on the RHS we arrive to the more practical expression,
\begin{equation}
    \frac{d}{dt} \int_{\Omega_\alpha} \textbf{f} d\Omega
    = \int_{\Omega_\alpha}\left[\pddt \textbf{f} + \div\left(\textbf{f}\textbf{u}_2\right) \right]d\Omega
    + \int_{\Sigma_\alpha} \textbf{f} (\textbf{u}_I-\textbf{u}_2)\cdot \textbf{n}_2 d\Sigma,
    \label{eq:Reynolds}
\end{equation}


\subsubsection*{Surface integral over $\Sigma_\alpha$}

In this work we are solely interested in closed surface topology. 
For a clear demonstration of the Reynolds transport and divergence theorem for interfaces we refer the reader to the work of \citet{nadim1996concise}. 
For closed surface integral the Gauss divergence theorem reads as :
\begin{equation}
    \int_{\Sigma_\alpha}  \divI \textbf{f} d\Sigma
    = 
    \int_{\Sigma_\alpha}  \textbf{f} \cdot \textbf{n}\divI\textbf{n} d\Sigma. 
    \label{eq:surf_div_theorem}
\end{equation}
Note that only the normal component of $\textbf{f}$ remain meaning that the integral any tangential tensor $\textbf f_{||}$ vanish. 
Additional, The time derivative of any surface integral can then be obtained using Leibniz rule, which reads as  
\begin{equation}
    \frac{d}{dt} \int_{\Sigma_\alpha} \textbf{f} d\Sigma 
    = \int_{\Sigma_\alpha} \left[
        \pddt \textbf{f} 
        +   \divI (\textbf{u}_I\textbf{f})
    \right]d\Sigma,
    \label{eq:Leibnitz}
\end{equation}
The formulation of \ref{eq:Leibnitz} is sometime preferred as the partial time derivative exist only on the surface. 




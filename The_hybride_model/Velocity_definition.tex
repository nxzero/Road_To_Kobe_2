\section{Topological equations for $\delta_\Gamma$}
\label{ap:delta_I}
In this appendix, we derive some useful relation related to $\delta_\Gamma$. In the sense of distribution \citep{appel2007}, 

\begin{equation}
<\grad \delta_\Gamma,\phi> = - <\delta_\Gamma,\grad \phi>.
\end{equation}
Hence,
\begin{equation}
<\grad \delta_\Gamma,\phi> = - \int_{\Gamma} \grad \phi d\Gamma.
\label{eq:1grad_delta_I}
\end{equation}
Since $\textbf{n}$ is a unit vector, we have $\grad  \textbf{n} \cdot \textbf{n} =\bm 0$ and%. Inserting this relation in \ref{eq:1grad_delta_I} we obtain
%\begin{equation}
$\grad \phi= \grad  (\phi \textbf{n}) \cdot \textbf{n},$
%\end{equation}
which yields 
\begin{equation}
<\grad \delta_\Gamma,\phi> = - \int_{\Gamma} \grad  (\phi \textbf{n}) \cdot \textbf{n} d\Gamma. 
\end{equation}
By applying the result of multiplying a distribution by a function we get \citep{appel2007} %By using the result of the product of a distribution by a function 
\begin{equation}
<\grad \delta_\Gamma,\phi> = <\grad  (\delta _I \textbf{n}) \cdot \textbf{n},\phi>.
\end{equation}
With a slight abuse of notation the preceding relation may be rewritten as
\begin{equation}
    \grad\delta_\Gamma 
    =   \grad  (\textbf{n} \delta_\Gamma) \cdot \textbf{n}
    \label{eq:grad_delta_I_app}
\end{equation}
Another relation may be obtained for the surface gradient operator of $\delta_\Gamma$. By definition,
\begin{equation}
  \gradI \delta_\Gamma  = \grad\delta_\Gamma - \textbf{nn}\cdot\grad\delta_\Gamma
\end{equation}
Since, $\textbf{nn}\cdot\grad\delta_\Gamma = \grad  (\delta_\Gamma\textbf{n})\cdot \textbf{n}-\grad \textbf{n}\cdot \textbf{n} \delta_\Gamma$ and using \ref{eq:grad_delta_I_app} we obtain
\begin{equation}
  \gradI \delta_\Gamma  = (\grad \textbf{n}\cdot \textbf{n}) \delta_\Gamma
\label{eq:gradI_deltaI}
\end{equation}


\section{Generalized form of the conservation equation on the interface}
\label{ap:interface_proof}
In this appendix we derive \ref{eq:dt_delta_I_f_I}. Multiplying \ref{eq:dt_f_I} by $\delta_\Gamma$ yields
\begin{equation}
    \delta_\Gamma
    \left[ \pddt f_\Gamma^0 
    + f_\Gamma^0 (\textbf{u}_\Gamma^0\cdot \textbf{n})  (\div \textbf{n})
    +\divI
    (f_\Gamma^0 \textbf{u}_{\Gamma||}^0
    - \mathbf{\Phi}_{\Gamma||}^0 )
    \right]
    = \delta_\Gamma s_I^0
    - \delta_\Gamma\Jump{
    f_k^0 (\textbf{u}_I^0 - \textbf{u}_k^0)
    + \mathbf{\Phi}_k^0}.
    \label{eq:delta_I_step1}
\end{equation}
We first focus on the first two terms on the left-hand side of \ref{eq:delta_I_step1}. Since, $\delta_\Gamma\pddt f_\Gamma^0 = \pddt (f_\Gamma^0\delta_\Gamma) - f_\Gamma^0\pddt\delta_\Gamma$ and using \ref{eq:dt_delta_I} yields
\begin{equation}
\delta_\Gamma
    \left[ \pddt f_\Gamma^0 
    + f_\Gamma^0 (\textbf{u}_\Gamma^0\cdot \textbf{n})  (\div \textbf{n}) \right] = \pddt (f_\Gamma^0\delta_\Gamma) + f_\Gamma^0\div [(\textbf{u}_I^0\cdot \textbf{n}) \textbf{n}\delta_\Gamma ].
\end{equation}
Since the gradient of $f_\Gamma^0$ lies in the plane parallel to the interface, we have  $\grad f_\Gamma^0 \cdot \textbf{n} = 0$. Consequently, the expression $f_\Gamma^0\div [(\textbf{u}_I^0\cdot \textbf{n}) \textbf{n}\delta_\Gamma ]$ simplifies to $\div [(\textbf{u}_I^0\cdot \textbf{n}) \textbf{n}f_\Gamma^0\delta_\Gamma ]$ which yields 
\begin{equation}
\delta_\Gamma
    \left[ \pddt f_\Gamma^0 
    + f_\Gamma^0 (\textbf{u}_\Gamma^0\cdot \textbf{n})  (\div \textbf{n}) \right] = \pddt (f_\Gamma^0\delta_\Gamma) + \div [(\textbf{u}_I^0\cdot \textbf{n}) \textbf{n}f_\Gamma^0\delta_\Gamma ].
\label{eq:delta_I_step1_2}
\end{equation}
%Using the relation, $f_\Gamma^0\div [(\textbf{u}_I^0\cdot \textbf{n}) \textbf{n}\delta_\Gamma ] = \div [(\textbf{u}_I^0\cdot \textbf{n}) \textbf{n}\delta_\Gamma f_\Gamma^0]$ since $\grad f_\Gamma^0 \cdot \textbf{n} = 0$
%Now, we focus on the third term on the left-hand side of \ref{eq:delta_I_step1}. Using the chain rule
Next, we turn our attention to the third term on the left-hand side of \ref{eq:delta_I_step1}. By applying the chain rule
\begin{equation}
    \delta_\Gamma \divI (f_\Gamma^0 \textbf{u}_{\Gamma||}^0
    - \mathbf{\Phi}_{\Gamma||}^0 ) %\textbf{F}_{I||}
    = 
    \div [\delta_\Gamma (f_\Gamma^0 \textbf{u}_{\Gamma||}^0
    - \mathbf{\Phi}_{\Gamma||}^0 )]
    - \textbf{n}(\textbf{n}\cdot\grad)\cdot [\delta_\Gamma (f_\Gamma^0 \textbf{u}_{\Gamma||}^0
    - \mathbf{\Phi}_{\Gamma||}^0 )]
    - (f_\Gamma^0 \textbf{u}_{\Gamma||}^0
    - \mathbf{\Phi}_{\Gamma||}^0 )\cdot\gradI\delta_\Gamma.
\label{eq:delta_I_step2}
% \nonumber\\
\end{equation}
Since $ (f_\Gamma^0 \textbf{u}_{\Gamma||}^0 - \mathbf{\Phi}_{\Gamma||}^0 )\cdot\gradI\delta_\Gamma  = (f_\Gamma^0 \textbf{u}_\Gamma^0 - \mathbf{\Phi}_{I}^0 )\cdot\gradI\delta_\Gamma$ and using \ref{eq:gradI_deltaI} we may rewrite the last term of \ref{eq:delta_I_step2} as $(f_\Gamma^0 \textbf{u}_\Gamma^0 - \mathbf{\Phi}_{I}^0 )\cdot(\grad \textbf{n}\cdot \textbf{n}) \delta_\Gamma$. This yields,
\begin{equation}
    \delta_\Gamma \divI (f_\Gamma^0 \textbf{u}_{\Gamma||}^0
    - \mathbf{\Phi}_{\Gamma||}^0 ) %\textbf{F}_{I||}
    = 
    \div [\delta_\Gamma (f_\Gamma^0 \textbf{u}_{\Gamma||}^0
    - \mathbf{\Phi}_{\Gamma||}^0 )]
    - \textbf{n}(\textbf{n}\cdot\grad)\cdot [\delta_\Gamma (f_\Gamma^0 \textbf{u}_{\Gamma||}^0
    - \mathbf{\Phi}_{\Gamma||}^0 )]
    - (f_\Gamma^0 \textbf{u}_\Gamma^0 - \mathbf{\Phi}_{I}^0 )\cdot(\grad \textbf{n}\cdot \textbf{n}) \delta_\Gamma.
\label{eq:delta_I_step3}
% \nonumber\\
\end{equation}
By noticing that $f_\Gamma^0 \textbf{u}_{\Gamma||}^0
    - \mathbf{\Phi}_{\Gamma||}^0 = (\bm\delta - \textbf{nn})\cdot (f_\Gamma^0 \textbf{u}_\Gamma^0
    - \mathbf{\Phi}_{I}^0 )$ we can expand the second term on the right hand side of \ref{eq:delta_I_step3} which yields
\begin{equation}
    - \textbf{n}(\textbf{n}\cdot\grad)\cdot [\delta_\Gamma (f_\Gamma^0 \textbf{u}_{\Gamma||}^0
    - \mathbf{\Phi}_{\Gamma||}^0 )]
    = (f_\Gamma^0 \textbf{u}_\Gamma^0 - \mathbf{\Phi}_{I}^0 )\cdot(\grad \textbf{n}\cdot \textbf{n}). \delta_\Gamma%(f_\Gamma^0 \textbf{u}_\Gamma^0
    %- \mathbf{\Phi}_{I}^0 ))%\delta_\Gamma(\textbf{n}\textbf{n}\cdot \grad(\bm\delta - \textbf{nn}))(f_\Gamma^0 \textbf{u}_\Gamma^0
    %- \mathbf{\Phi}_{I}^0 ))
\label{eq:delta_I_step4}
% \nonumber\\
\end{equation}
To derive the last equation, we used the following properties in indicial notation: $n_in_j(I_{ik}-n_in_k) =0$ and $n_in_j\partial_{j}(n_in_k)=\partial_{j}(n_k)n_j$. Inserting \ref{eq:delta_I_step4} in \ref{eq:delta_I_step3} yields
\begin{equation}
    \delta_\Gamma \divI (f_\Gamma^0 \textbf{u}_{\Gamma||}^0
    - \mathbf{\Phi}_{\Gamma||}^0 ) %\textbf{F}_{I||}
    = 
    \div [\delta_\Gamma (f_\Gamma^0 \textbf{u}_{\Gamma||}^0
    - \mathbf{\Phi}_{\Gamma||}^0 )]
\label{eq:delta_I_step5}
% \nonumber\\
\end{equation}
Inserting \ref{eq:delta_I_step1_2} and \ref{eq:delta_I_step5} in \ref{eq:delta_I_step1} gives

\begin{equation}
    \pddt (\delta_\Gamma f_\Gamma^0)  
    + \div (
        \delta_\Gamma f_\Gamma^0 \textbf{u}_I^0
        - \delta_\Gamma \mathbf{\Phi}_{\Gamma||}^0 
        )
    = 
    \delta_\Gamma s_I^0
    - \delta_\Gamma\Jump{
    f_k^0 (\textbf{u}_I^0 - \textbf{u}_k^0)
    + \mathbf{\Phi}_k^0},
\end{equation}
which is the final form of the conservation equation on the interface.

%Expanding the second term of \ref{eq:step1} by noticing that $\textbf{F}_{I||} = (\bm\delta - \textbf{nn})\cdot \textbf{F}_I$, and using the expression $\textbf{n}(\textbf{n}\cdot\grad)\cdot (\bm\delta - \textbf{nn}) =  (\textbf{n}\cdot\grad)\textbf{n}$

%\color{blue}
% In \citet[Appendix 2]{marle1982macroscopic} they demonstrated how to obtain \ref{eq:dt_delta_I_f_I} in the specific context of the mass, momentum and energy equations. 
%For completeness and ease of understanding we give in this appendix the detailed derivation of \ref{eq:dt_delta_I_f_I}. 
%Let us first introduce some important relations regarding surface properties. 
%Consider the arbitrary tensor $ \textbf{F}_{I}$ defined on $\Gamma(t)$.
%Let us take the surface divergence of $\textbf{F}_{I||} = \textbf{F}_I \cdot (\bm\delta -\textbf{nn})$, it yields,
%\begin{align}
%    \delta_\Gamma \divI \textbf{F}_{I||}
%    &= 
%    \div (\delta_\Gamma \textbf{F}_{I||})
%    - \textbf{n}(\textbf{n}\cdot\grad)\cdot (\delta_\Gamma \textbf{F}_{I||})
%    - \textbf{F}_{I}\cdot\gradI\delta_\Gamma \nonumber\\
%    &= 
%    \div (\delta_\Gamma \textbf{F}_{I||})
%    - \textbf{n}(\textbf{n}\cdot\grad)\cdot (\delta_\Gamma \textbf{F}_{I||})
%    - \textbf{F}_{I} \delta_\Gamma (\textbf{n}\cdot\grad)\textbf{n}
    % &= 
    % \div (\delta_\Gamma \textbf{F}_{I||})
%    \label{eq:step1}
%\end{align}
%were we used the chain rule and the relation $\divI(\ldots) = \div(\ldots) - \textbf{n}(\textbf{n}\cdot \grad)\cdot(\ldots)$ for the first equality. 
%The second equality is derived using \ref{eq:grad_delta_I} dotted with $(\bm\delta -\textbf{nn})$, which gives the relation 
%\begin{equation*}
%    (\bm\delta - \textbf{nn}) \cdot \grad\delta_\Gamma
%    = \gradI \delta_\Gamma
%    = 
%    \delta_\Gamma (\textbf{n}\cdot \grad)\textbf{n}.
%\end{equation*}
%Expanding the second term of \ref{eq:step1} by noticing that $\textbf{F}_{I||} = (\bm\delta - \textbf{nn})\cdot \textbf{F}_I$, and using the expression $\textbf{n}(\textbf{n}\cdot\grad)\cdot (\bm\delta - \textbf{nn}) =  (\textbf{n}\cdot\grad)\textbf{n}$
% \begin{equation*}
%     - \textbf{n}(\textbf{n}\cdot\grad)\cdot (\delta_\Gamma \textbf{F}_{I||})
%     = 
%     - \delta_\Gamma \textbf{F}_{I}\textbf{n}(\textbf{n}\cdot\grad)\cdot (\bm\delta - \textbf{nn})
%     + (\bm\delta - \textbf{nn}) \cdot \textbf{n}(\textbf{n}\cdot\grad)(\delta_\Gamma \textbf{F}_{I})
%     = \textbf{F}_{I} \delta_\Gamma (\textbf{n}\cdot\grad)\textbf{n},
% \end{equation*}
% leads us to the useful relation :
%\begin{equation}
%    \delta_\Gamma \divI \textbf{F}_{I||}
%    = 
%    \div (\delta_\Gamma \textbf{F}_{I||}). 
%    \label{eq:proof_1}
%\end{equation}
% Making use of the same principles one can show the already common expression  
% \begin{equation}
%     \divI
%     \textbf{F}_{I}
%     = 
%     \divI
%     \textbf{F}_{I||}
%     +(\textbf{F}_I \cdot \textbf{n})  \div \textbf{n},
% \end{equation}
% which, multiplied with $\delta_\Gamma$ gives 
% \begin{equation}
%     \delta_\Gamma \divI
%     \textbf{F}_{I}
%     = 
%     \div
%     (\delta_\Gamma \textbf{F}_{I||})
%     +\delta_\Gamma (\textbf{F}_I \cdot \textbf{n})  \div \textbf{n}. 
%     \label{eq:proof_1}
% \end{equation}


%To demonstrate that \ref{eq:dt_delta_I} and \ref{eq:dt_delta_I_f_I} are consistent we must prove the equality 
%\begin{equation}
%    \delta_\Gamma
%    \left[ \pddt f_\Gamma^0 
%    + f_\Gamma^0 (\textbf{u}_\Gamma^0\cdot \textbf{n})  (\div \textbf{n})
%    +\divI
%    (f_\Gamma^0 \textbf{u}_{\Gamma||}^0
%    - \mathbf{\Phi}_{\Gamma||}^0 )
%    \right]
%    =
%    \pddt (\delta_\Gamma f_\Gamma^0) 
%    +\div
%    (\delta_\Gamma f_\Gamma^0 \textbf{u}_I^0
%        - \delta_\Gamma\mathbf{\Phi}_{\Gamma||}^0 ).
%    \label{eq:to_prove}
%\end{equation}
%This is easily done  using \ref{eq:dt_f_I} on the two first terms on the left-hand side of \ref{eq:to_prove} which gives
%\begin{equation}
%    \delta_\Gamma \pddt f_\Gamma^0 
%    + f_\Gamma^0 \delta_\Gamma (\textbf{u}_I\cdot\textbf{n})(\div \textbf{n})
%    = 
%     \pddt (f_\Gamma^0 \delta_\Gamma)
%    + \div(f_\Gamma^0  \delta_\Gamma \textbf{n}(\textbf{n}\cdot\textbf{u}_I^0)). 
%    % - \delta_\Gamma (\textbf{n}\cdot\textbf{u}_I^0) (\textbf{n}\cdot \grad)f_\Gamma^0.
%    \label{eq:last_step} 
%\end{equation}
%Then, using the relation \ref{eq:proof_1} on the remaining terms on the left-hand side of \ref{eq:to_prove} directly proves \ref{eq:to_prove} and by extension \ref{eq:dt_delta_I_f_I}. 
%Note that to derive \ref{eq:last_step} we have assumed that $\delta_\Gamma (\textbf{n}\cdot\textbf{u}_I^0)(\textbf{n}\cdot\grad) f_\Gamma^0 = 0$. 
%In \citet{orlando2023evolution} it is discussed that the terms of the form $\delta_\Gamma (\textbf{n}\cdot\textbf{u}_I^0)(\textbf{n}\cdot\grad) f_\Gamma^0 $ are not necessarily zero.  
%However, we argue that if it was not the case, we wouldn't be able to derive \ref{eq:dt_f} which is by definition valid.
%In other worlds, the only way to remain consistent with \ref{eq:dt_f} is to assume that $\delta_\Gamma (\textbf{n}\cdot\textbf{u}_I^0)(\textbf{n}\cdot\grad) f_\Gamma^0 = 0$.

%Another way to interpret this results is to consider that $f_\Gamma^0$ is only function of two surface coordinate. 
%Therefore, taking the derivative along a third coordinate, that represent the distance to the surface, is zero since $f_\Gamma^0$ is not dependent on the latter coordinate. 
%(Eventhrougth i do not totally agrre with that because we need to suppose that $f_\Gamma^0$ is defined outside of the surface which isn't true)

%\color{black}

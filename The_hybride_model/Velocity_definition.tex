\section{Derivation of the point velocity}
\label{ap:velocity_definition}
In this Appendix we derive the velocity of the center of mass of any particle $\alpha$. Consider a particle of center of mass $\textbf{y}_\alpha$ defined such as
\begin{equation*}
    m_\alpha(t) \textbf{y}_\alpha(t)
    = \int_{\Omega_\alpha} \rho_k \textbf{y}_k d\Omega,
\end{equation*}
where we empathize that both, $m_\alpha(t)$ and $\textbf{y}_\alpha$ is function of time. 
The point velocity can be solely defined as the derivative of $\textbf{y}_\alpha$ within time.
Yielding, 
\begin{align*}
    \ddt \textbf{y}_\alpha (t)
    &=
    \ddt \left(
        \frac{1}{m_\alpha} \int_{\Omega_\alpha} \rho_k \textbf{y}_k d\Omega
    \right)\\
    &= \frac{1}{m_\alpha}
    \ddt 
    \left(
        \int_{\Omega_\alpha} \rho_k \textbf{y} d\Omega
    \right)
    - \frac{1}{m_\alpha^2} \ddt \int_{\Omega_\alpha} \rho_k d\Omega \int_{\Omega_\alpha} \rho_k \textbf{y}_k d\Omega
    \\
    &= \frac{1}{m_\alpha}\int_{\Omega_\alpha} \left[
        \pddt (\rho_k \textbf{y}) + \nablabh \cdot\left(\rho_k \textbf{y}\textbf{u}_k\right) 
    \right]d\Omega \\
    &+ \frac{1}{m_\alpha}\int_{\Sigma_\alpha} \textbf{y} \rho_k(\textbf{u}_I   - \textbf{u}_k) \cdot \textbf{n}_k d \Sigma
    -  \frac{1}{m_\alpha^2} \int_{\Sigma_\alpha} \rho_k(\textbf{u}_I   - \textbf{u}_k) \cdot \textbf{n}_k d\Sigma  \int_{\Omega_\alpha} \rho_k \textbf{y}_k d\Omega
    \\
    &= \frac{1}{m_\alpha}\int_{\Omega_\alpha} \textbf{y} \left[
    \pddt (\rho_k) + \nablabh \cdot\left(\rho_k \textbf{u}_k\right) 
    \right]d\Omega
    + \frac{1}{m_\alpha}\int_{\Omega_\alpha} \rho_k  \textbf{u}_k  \cdot \nablabh \textbf{y} d\Omega \\
    &+ \frac{1}{m_\alpha}\int_{\Sigma_\alpha} \textbf{y}_k \rho_k (\textbf{u}_I - \textbf{u}_k) \cdot \textbf{n}_k d \Sigma
    - \frac{1}{m_\alpha}  \textbf{y}_\alpha \int_{\Sigma_\alpha} \rho_k(\textbf{u}_I   - \textbf{u}_k) \cdot \textbf{n}_k d\Sigma
\end{align*}
By considering the mass conservation on a single fluid particle for the first term, noticing that $\nablabh \textbf{y} = \textbf{I}$ where $\textbf{I}$ is the identity tensor for the second term, and introducing $\mathbf{r} = \mathbf{y} - \mathbf{y}_\alpha$ in the third term gives, we obtain the following relation
\begin{equation*}
    \textbf{u}_\alpha
    = \frac{1}{m_\alpha} \left(
        \int_{\Omega_\alpha} \rho_k \textbf{u}_k d\Omega
        +  \int_{\Sigma_\alpha} \rho_k \textbf{r}  (\textbf{u}_I - \textbf{u}_k) \cdot \textbf{n}_k d\Sigma
    \right).
    % = \frac{1}{m_\alpha}  \left(
    %     \textbf{p}_\alpha
    % - \int_{\Omega_\alpha} \rho_k \textbf{w} d\Omega
    % \right)
\end{equation*}


\section{A detailed derivation of the moment derivative}
\label{ap:moment_derivative}
The first moment or dipoles of any property $q_\alpha$ can be defined as,
\begin{equation*}
    \textbf{Q}_\alpha 
    = \int_{V_\alpha} \textbf{r} f_k dV
\end{equation*}
As, before we use the Reynolds transport theorem to describe the evolution of any $\textbf{Q}_\alpha$ within time. 
Yielding,
\begin{align}
    \ddt \textbf{Q}_\alpha
    & = \ddt \int_{V_\alpha} \textbf{r} f_k dV  \\
      &=  \int_{V_\alpha} \left[
        \pddt(\textbf{r}  f_k)
        + \nablabh \cdot \left(f \textbf{r} \textbf{u}_k\right)
    \right]dV + \int_{S_\alpha} \textbf{r}  f_k  (\textbf{u}_I-\textbf{u}_k)\cdot \textbf{n}_k  dS  \nonumber \\
    &=  \int_{V_\alpha} \textbf{r}\left[
        \pddt f_k
        + \nablabh \cdot \left(f_k \textbf{u}_k\right)
    \right] dV
    + \int_{V_\alpha} f_k \left[
        \pddt \textbf{r}
        +\textbf{u}_k \nablabh \textbf{r}
    \right]dV\\
    &+ \int_{S_\alpha} \textbf{r}  f_k (\textbf{u}_I-\textbf{u}_k)\cdot \textbf{n}_k  dS,
\end{align}
Using \ref{eq:general_conservation} \JL{pas la bonne equation citee} for the first term, and considering the relation,
$  \pddt \textbf{r}
+ \textbf{u}_k \cdot \nablabh \textbf{r}
= - \frac{d}{dt} \textbf{y}_\alpha  + \textbf{u}_k \cdot \textbf{I}
= \textbf{w}_k$,
where $\textbf{I}$ is the identity tensor. 
Applying this relation for the second term, we get, 
\begin{align}
    \ddt \textbf{Q}_\alpha
    &= \int_{V_\alpha} \textbf{r} \left[
         \textbf{S}_k +  \nablabh \cdot \mathbf{\Phi}_k
    \right]dV
    +\int_{V_\alpha} f_k  \textbf{w}_k dV
    + \int_{S_\alpha} \textbf{r}  f_k (\textbf{u}_I-\textbf{u}_k)\cdot \textbf{n}_k  dS,\\
    &= \int_{V_\alpha} \left( 
        \textbf{r} \textbf{S}_k 
        - \mathbf{\Phi}_k
        + f_k  \textbf{w}_k 
    \right) dV
    + \int_{S_\alpha} \textbf{r} \left[
        \mathbf{\Phi}_k
        + f_k (\textbf{u}_I-\textbf{u}_k)
    \right]\cdot \textbf{n}_k  dS.
\end{align}
\JL{Dans le corps du texte tu notes $d\Sigma$ et $d\Omega$ les differentielles des surfaces et des volumes. Merci de faire la meme chose en annexe. Par ailleurs je trouve qu'il manque des explications pour passser de 78 a 79 (j'imagine que tu utilises la conservation du moment d'ordre 0) et pour passer de 79 a 80 ou tu dois faire une integration part partie. Encore faut il le preciser ...}


\subsection{The system of equation}


The fluid phase equation will reads essentially the same as it was presented in the first part, however the interfacial term need to be expanded into particle averaged quantity. 
Therefore, the fluid phase conservation may be written in the most general form as, 
\begin{align}
    \pddt (\phi_1 f_1)
    +\div (\phi_1 f_1 \textbf{u}_1  - \bm\Phi_1^\text{eq})
    &= 
    \phi_1 s_1
    - \pSavg{\delta_I\left[
        \bm{\Phi}_1^0
        + f_1^0
        \left(
            \textbf{u}_I^0
            - \textbf{u}_1^0
        \right)
    \right]
    \cdot \textbf{n}_2} ,
    \label{eq:dt_f_1_hybrid}
\end{align}
where $\bm\Phi_1^\text{eq}$ is the averaged scale equivalent non-convective flux in the hybrid form, namely 
\begin{align}
    \bm\Phi_1^\text{eq}
    = 
    - \avg{\chi_1 f_1' \textbf{u}_1'}
    + \phi_1 \bm\Phi_1
    + \pSavg{\textbf{r} \left[
        \bm{\Phi}_1^0
        + f_1^0 (\textbf{u}_I^0-\textbf{u}_1^0)
    \right]\cdot \textbf{n}_2}\\
    - \frac{1}{2}\div\left\{
        \pSavg{\textbf{rr} \left[
            \bm{\Phi}_1^0
            + f_1^0 (\textbf{u}_I^0-\textbf{u}_1^0)
            \right]\cdot \textbf{n}_2}
            +\ldots
        \right\}
\end{align}
is the equivalent non-convective flux of the macro-scale. 
This equivalent non-convective flux will always contain a covariance term, the average of the local scale non-convective flux and the first and possibly higher moment of the phase exchange term. 
The particle phases equations might be re-written as, 
\begin{align}
    \label{eq:dt_q_p_hybrid}
    \pddt (n_p  q_p^\text{tot})
    + \div (n_p\textbf{u}_p q_p^\text{tot} + \bm\Phi_p^\text{eq})
    &= \pOavg{ s_2^0 }
    + \pSavg{ s_I^0 }\\
    &+ \pSavg{ \left[\bm{\Phi}_1^0 + f_1^0 (\textbf{u}_I^0-\textbf{u}_1^0) \right] \cdot \textbf{n}_2 ,}
    \\
    \label{eq:dt_Q_p_hybrid}
    \pddt (n_p  \bm{Q}_p^\text{tot})
    + \div (n_p \textbf{u}_p\bm{Q}_p^\text{tot}
    + \bm\Phi_p^\text{eq-1})
    &=\pOavg{ \left(
        \textbf{r} s_2^0         
        + f_2^0  \textbf{w}_2^0 
        - \bm{\Phi}_2^0
    \right) }
    + \pSavg{ \left(
        \textbf{r}s_I^0
        + f_I^0 \textbf{w}_I^0
        - \bm{\Phi}_{I||}^0
    \right) }\nonumber\\
    &+ \pSavg{ \textbf{r} \left[
        \bm{\Phi}_1^0
        + f_1^0 (\textbf{u}_I^0-\textbf{u}_1^0)
    \right]\cdot \textbf{n}_2  }.
\end{align}
Where we have defined the equivalent non-convective flux of the particle phase as, 
\begin{align*}
    \bm\Phi_p^\text{eq-1}= \pavg{\textbf{u}_\alpha' q_\alpha'}
    && \bm\Phi_p^\text{eq-1}= \pavg{\textbf{u}_\alpha' \bm{Q}_\alpha'}
\end{align*}
Finally, to ensure conservation of volume we must add an additional equaiton which is, 
\begin{equation}
    \phi_1 + \phi_2 = 1
\end{equation}
Or in the hybrid form, 
\begin{equation}
    \phi_1 + n_p v_p +\frac{1}{2} \grad\grad : (n_p \textbf{V}_p) + \ldots = 1
\end{equation}
where $n_p\textbf{V}_p = \pOavg{\textbf{rr}}$. 


In the end depending on the degree of freedom of the particle phase one might need between one and an infinite number of equation for the particle phase. 
An example is that of solid particle, as the internal solid motion is linear, only the zeroth and the first moment of momentum are sufficient to wholly describe the internal motion of a solid particle. 
Regarding the fluid phase each quantity require only one equation. 
This equation will share the same source terms as the particle phase equations. 
In conclusion, to obtain a consistent model, one must have the same number of source terms in the fluid phase equation as the number of equation in the particle phase. 
Only in that way, one is able to reach an equilibrium between the internal particles'moments and the fluid phase stress. 



As remarked by \citet{jackson1997locally} for the angular momentum equations of solid spherical particles, and here in a more general case :\ref{eq:avg_dt_chi_f} and \ref{eq:avg_dt_dq_alpha_tot} may seem to be not coupled with the higher order moments equations, i.e. \ref{eq:avg_dt_dQ_alpha_tot}. 
Indeed, the first order moment $\textbf{Q}_p^\text{tot}$ do not appear explicitly in either \ref{eq:avg_dt_chi_f} or \ref{eq:avg_dt_dq_alpha_tot}.
However, the exchange and source terms 
appearing in both \ref{eq:avg_dt_chi_f} and \ref{eq:avg_dt_dQ_alpha_tot} might depend on the higher order moments of the particles.
For example, in the momentum equation of the continuous phase, i.e. the ensemble average of \ref{eq:dt_rhou_k}, the exchange term corresponds to the averaged drag force, namely $\pSavg{\bm{\sigma}_1^0\cdot \textbf{n}_2}$. 
It is clear that $\pSavg{\bm{\sigma}_1^0\cdot \textbf{n}_2}$ has a strong dependency with $\textbf{u}_p$,$\textbf{P}_p$ and $\textbf{M}_p$ since the drag force is a function of both the particle's kinematic and its shape. 
Therefore, \ref{eq:avg_dt_dQ_alpha_tot} is linked to \ref{eq:avg_dt_chi_f} and \ref{eq:avg_dt_dq_alpha_tot} solely through the dependence of the exchange and source terms with the properties of the particles, e.g. $q_p,\textbf{Q}_p$ and possibly the higher order moments. 
This reasoning applies for the energy equation and all kinds of conservation equation. 
Ultimately, the significance of the higher moments equations can be evaluated based on the dependency of the closure terms present in \ref{eq:avg_dt_chi_f} with the properties of the particle : $q_p, \textbf{Q}_p$ and potentially the higher-order moments of the particles.


Eventually, a closed model can be reached by expressing each of the terms on the right-hand side of \ref{eq:dt_f_1_hybrid}, \ref{eq:dt_q_p_hybrid} and \ref{eq:dt_Q_p_hybrid}, i.e. every term within the $\avg{\ldots}$ operator in terms of the unknown or gradient of the unknown.
If we consider $N$ closures terms of interest that is, 
\begin{align*}
    \avg{X_1} &=  f(\phi_1,n_p,v_p,\textbf{V}_p; f_1,q_p,\textbf{Q}_p)\\
    &\vdots\\
    \avg{X_N} &=  f(\phi_1,n_p,v_p,\textbf{V}_p; f_1,q_p,\textbf{Q}_p). \\
\end{align*}
These closed expressions are in all rigor obtained by solving the single-particle conditionally averaged equations \citet{hinch1977averaged,zhang1994ensemble}.
In practice, it is more convenient to consider experimental or numerical measurements.  


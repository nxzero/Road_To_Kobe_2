\subsection{The system of equation}
Now that we reached a clear understanding of the mathematical structures of the averaged two phase flow equation we start to expose the averaged set of equations which constitute the \textit{Hybrid model}. 
As, mentioned in \ref{sec:two-fluid} we consider the mass, momentum and energy for the particles and continuous phase. 
Additionally, to describe higher degree of freedom of the particle shape and momentum, one must consider the second moment of mass and first moment of momentum averaged equations. 
This, makes a total of 10 equations, 6 for the particle phase and 4 for the continuous phase.
This system involves numerous equations and closures terms, it is therefore important to consider in a second step a routine to simplify this system by the consideration of the physical particle degree of freedom.
This; is the subject of the next section.    


The fluid phase equation will reads essentially the same as it was presented in the first part, however the interfacial term need to be expanded into particle averaged quantity. 
Therefore, the fluid phase conservation may be written in the most general form as, 
\begin{align}
    \pddt (\phi_1 f_1)
    +\div (\phi_1 f_1 \textbf{u}_1  - \bm\Phi_1^\text{eq})
    &= 
    \phi_1 s_1
    - \pSavg{\delta_I\left[
        \mathbf{\Phi}_k^0
        + f_k^0
        \left(
            \textbf{u}_I^0
            - \textbf{u}_k^0
        \right)
    \right]
    \cdot \textbf{n}_k} ,
\end{align}
where, 
\begin{align}
    \bm\Phi_1^\text{eq}
    = 
    - \avg{\chi_1 f_k' \textbf{u}_k'}
    + \phi_1 \bm\Phi_1
    + \pSavg{\textbf{r} \left[
        \mathbf{\Phi}_1^0
        + f_1^0 (\textbf{u}_I^0-\textbf{u}_1^0)
    \right]\cdot \textbf{n}_2}\\
    - \frac{1}{2}\div\left\{
        \pSavg{\textbf{rr} \left[
            \mathbf{\Phi}_1^0
            + f_1^0 (\textbf{u}_I^0-\textbf{u}_1^0)
            \right]\cdot \textbf{n}_2}
            +\ldots
        \right\}
\end{align}
is the equivalent non-convective flux of the macro-scale. 
This equivalent non-convective flux will always contain a covariance term, the average of the local scale non-convective flux and the first and possibly higher moment of the phase exchange term. 

The particle phases equations might be re-written as, 
\begin{align}
    \pddt (n_p  q_p^\text{tot})
    + \div (n_p\textbf{u}_p q_p^\text{tot} + \bm\Phi_p^\text{eq})
    &= \pOavg{ s_2^0 }
    + \pSavg{ s_I^0 }
    + \pSavg{ \left[\mathbf{\Phi}_1^0 + f_1^0 (\textbf{u}_I^0-\textbf{u}_1^0) \right] \cdot \textbf{n}_2 ,}
    \\
    \pddt (n_p  \mathcal{Q}_p^\text{tot})
    + \div (n_p \textbf{u}_p\mathcal{Q}_p^\text{tot}
    + \bm\Phi_p^\text{eq-1})
    &=\pOavg{ \left(
        \textbf{r} s_2^0         
        + f_2^0  \textbf{w}_2^0 
        - \mathbf{\Phi}_2^0
    \right) }
    + \pSavg{ \left(
        \textbf{r}s_I^0
        + f_I^0 \textbf{w}_I^0
        - \mathbf{\Phi}_{I||}^0
    \right) }\nonumber\\
    &+ \pSavg{ \textbf{r} \left[
        \mathbf{\Phi}_1^0
        + f_1^0 (\textbf{u}_I^0-\textbf{u}_1^0)
    \right]\cdot \textbf{n}_2  }.
\end{align}
Where we have defined the equivalent non-convective flux of the particle phase as, 
\begin{align*}
    \bm\Phi_p^\text{eq-1}= \pavg{\textbf{u}_\alpha' q_\alpha'}
    && \bm\Phi_p^\text{eq-1}= \pavg{\textbf{u}_\alpha' \mathcal{Q}_\alpha'}
\end{align*}
Finally, to ensure conservation of volume we must add an additional equaiton which is, 
\begin{equation}
    \phi_1 + \phi_2 = 1
\end{equation}
Or in the hybrid form, 
\begin{equation}
    \phi_1 + \pOavg{} +\frac{1}{2} \grad\grad : \pOavg{\textbf{rr}} + \ldots = 1
\end{equation}



In the end depending on the degree of freedom of the particle phase one might need between one and an infinite number of equation for the particle phase. 
An example is that of solid particle, as the internal solid motion is linear, only the zeroth and the first moment of momentum are sufficient to wholly describe the internal motion of a solid particle. 

Regarding the fluid phase each quantity require only one equation. 
This equation will share the same source terms as the particle phase equations. 


In conclusion, to obtain a consistent model, one must have the same number of source terms in the fluid phase equation as the number of equation in the particle phase. 
Only in that way, one is able to reach an equilibrium between the internal particles'moments and the fluid phase stress. 



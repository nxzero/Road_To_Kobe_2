\begin{itemize}
    \item No mass transfer
    \item Interface with no weight 
    \item contante phase density
\end{itemize}

Now that we obtained a clear picture of the averaged two phases flows equations, let us introduce the averaged hybrid model for arbitrary two-phase flows. 
First we introduce the equation of the continuous phase by setting $k = 1$ in \ref{eq:avg_dt_chi_f} and by using \ref{eq:f_exp} for the interfacial term it yields,
\begin{multline}
    \pddt \avg{\chi_1 f_1}
    = \nablabh \cdot \avg{\chi_1 \bm{\Phi}_1 - \chi_1 f_1 \textbf{u}_1}
    - \pavg{\int_{\Sigma_\alpha}\left[
        \mathbf{\Phi}_1
        + f_1
        \left(
            \textbf{u}_I
            - \textbf{u}_1
        \right)
    \right]
    \cdot \textbf{n}_2d\Sigma} \\
    + \avg{\chi_1 \textbf{S}_1}
    +  \nablab \cdot \pavg{\int_{\Sigma_\alpha} \textbf{r}\left[
        \mathbf{\Phi}_1
        + f_1
        \left(
            \textbf{u}_I
            - \textbf{u}_1
        \right)
    \right]
    \cdot \textbf{n}_2d\Sigma} 
    \label{eq:hybrid_avg_dt_chif}
\end{multline}
Notice that the sign of the last two terms are opposite, indeed we switched the normal vector direction in both integral from $\textbf{n}_1$ to $-\textbf{n}_2$. 
This way this term is expressed as the projection on the outward normal vector of the particles, which is the conventional way to express these terms. 
Notice that we neglected the higher order moments of the interfacial term impling that we generate an error of $\mathcal{O}\left(a^2/L^2\right)$ \citep{jackson1997locally}. 

For the dispersed phase we consider the particle and the interfaces as a whole.
Consequently, the dispersed phase equations can be obtained by summing \ref{eq:avg_dt_dq_alpha} and \ref{eq:avg_dt_dq_I_alpha}, which directly gives us, 
\begin{multline}
    \pddt \pavg{ (q_\alpha+q_{I\alpha})}
    + \nablabh \cdot\pavg{\textbf{u}_\alpha (q_\alpha+q_{I\alpha}) }
    = \pavg{\int_{\Omega_\alpha} \textbf{S}_2 d\Omega}\\
    + \pavg{\int_{\Sigma_\alpha} \textbf{S}_I d\Sigma}
    + \pavg{\int_{\Sigma_\alpha} \left[\bm{\Phi}_1 + f_1 (\textbf{u}_I-\textbf{u}_1) \right] \cdot \textbf{n}_2 d\Sigma},
    \label{eq:hybrid_avg_dt_dq_alpha}\\
\end{multline}
Additionally, we can eventually add a second order equations obtained by averaging \ref{eq:dt_Q_alpha} and \ref{eq:dt_Q_I_alpha},
\begin{multline}
    \pddt \pavg{(\textbf{Q}_\alpha+\textbf{Q}_{I\alpha})}
    + \nablabh\cdot \pavg{  \textbf{u}_\alpha (\textbf{Q}_\alpha+\textbf{Q}_{I\alpha})}
    =\\ \pavg{\int_{\Omega_\alpha} \left[
        \textbf{r}\textbf{S}_2
        - \mathbf{\Phi}_2
        + f_k\textbf{w}_k
        \right] d\Omega}
    + \pavg{\int_{\Sigma_\alpha} \left[
        \textbf{r}\textbf{S}_I
        - \mathbf{\Phi}_I
        + f_I\textbf{w}_I
    \right] d\Sigma}\\
    + \pavg{\int_{\Sigma_\alpha} \textbf{r}\left[\bm{\Phi}_1 + f_1 (\textbf{u}_I-\textbf{u}_1) \right] \cdot \textbf{n}_2 d\Sigma}\\
    \label{eq:hybrid_avg_dt_dQ_alpha}\\
\end{multline}
We can notice that formulated this way the interfacial terms in \ref{eq:hybrid_avg_dt_chif} correspond to the ones appearing in \ref{eq:hybrid_avg_dt_dq_alpha} and \ref{eq:hybrid_avg_dt_dQ_alpha}.

\subsection{Mass and momentum conservation without mass transfer}

From the beginning we stayed general, let's now study the specific case of momentum and mass conservation with no mass transfer at a first order description of the dispersed phase. 
For the mass conservation the variable of the problem yields,
\begin{align*}
    f_1 = \rho_1           
    && f_2 =  \rho_2    
    && f_I =  0
    && q_\alpha &= \int_{\Omega_\alpha} \rho_2  d\Omega = m_\alpha
    && q_{I\alpha} = 0\\
    \mathbf{\Phi}_1 = 0    
    &&     \mathbf{\Phi}_2  =  0         
    &&     \mathbf{\Phi}_I  =  0         
    && \textbf{Q}_\alpha &= \int_{\Omega_\alpha} \textbf{r} \rho_2  d\Omega = 0
    && \textbf{Q}_{I\alpha} = 0\\
    \textbf{S}_1 = 0       
    &&      \textbf{S}_2    =  0         
    &&      \textbf{S}_I    =  0         
    && \textbf{Q}^2_\alpha &= \int_{\Omega_\alpha} \textbf{rr} \rho_2  d\Omega = \mathcal{M}_\alpha
    && \textbf{Q}^2_{I\alpha} = 0
\end{align*} 
where we introduced the second order moments for a reason that will be clearer in the next few paragraph. 
Another important consideration made due to the assumption of no mass transfer is that $\textbf{u}_1 - \textbf{u}_I =\textbf{u}_2 - \textbf{u}_I  =0 $ on $\Sigma(t)$.
By setting those value into \ref{eq:hybrid_avg_dt_chif}, \ref{eq:hybrid_avg_dt_dq_alpha} and \ref{eq:hybrid_avg_dt_dQ_alpha} we obtain respectively the mass conservation equation for the continuous phase, the  zeroth order moment of the mass conservation for the dispersed phase and the second order moment conservation equation of the particle mass distribution, namely,
\begin{equation}
    \pddt \avg{\chi_1 \rho_1}
    + \nablabh \cdot \avg{\chi_1 \rho_1 \textbf{u}_1}
    = 0,
    \label{eq:hybrid_avg_dt_chi_m}
\end{equation}
\begin{equation}
    \pddt \pavg{ m_\alpha}
    + \nablabh \cdot\pavg{ m_\alpha \textbf{u}_\alpha}
    = 0,
    \label{eq:hybrid_avg_dt_dm_alpha}
\end{equation}
\begin{equation}
    \pavg{\int_{\Omega_\alpha} 
        \rho_k\textbf{w}_k
         d\Omega}
    = 0.
    \label{eq:hybrid_avg_dt_dM_alpha}\\
\end{equation}
where we can notice that the mass conservation of the first moment of the mass actually provides us with the condition of the null velocities fluctuation discussed in \ref{sec:dispersed-two-fluid}.

Regarding the momentum conservation equations. 
It can be shown that we obtain the following variables, 
\begin{align*}
    f_1 = \rho_1 \textbf{u}_1          
    && f_2 =  \rho_2 \textbf{u}_2
    && f_I =  0
    && \textbf{q}_\alpha &= \int_{\Omega_\alpha} \rho_2 \textbf{u}_2  d\Omega = \textbf{p}_\alpha
    && \textbf{q}_{I\alpha} = 0\\
    \mathbf{\Phi}_1 = \textbf{T}_1
    &&     \mathbf{\Phi}_2  =  \textbf{T}_2
    &&     \mathbf{\Phi}_I  =  \sigma \textbf{I}_{||}        
    && \textbf{Q}_\alpha &= \int_{\Omega_\alpha} \rho_2 \textbf{r} \textbf{u}_2 d\Omega = \mathcal{P}_\alpha
    && \textbf{Q}_{I\alpha} = 0\\
    \textbf{S}_1 = \textbf{g}       
    &&      \textbf{S}_2    =  \textbf{g}         
    &&      \textbf{S}_I    =  0         
    && \textbf{Q}^2_\alpha &= \int_{\Omega_\alpha} \rho_2 \textbf{rru}_2   d\Omega = \mathcal{P}_\alpha^2
    && \textbf{Q}^2_{I\alpha} = 0
\end{align*} 

Following the mass derivation we obtain the momentum equations such as, 
\begin{multline}
    \pddt \avg{\chi_1 \rho_1\textbf{u}_1}
    = \nablabh \cdot \avg{\chi_1 \textbf{T}_1 - \chi_1 \rho_1\textbf{u}_1 \textbf{u}_1}
    + \avg{\chi_1 \textbf{g}}\\
    - \pavg{\int_{\Sigma_\alpha}
        \mathbf{T}_1
    \cdot \textbf{n}_2d\Sigma} 
    +  \nablab \cdot \pavg{\int_{\Sigma_\alpha} \textbf{r}
        \mathbf{T}_1
    \cdot \textbf{n}_2d\Sigma} 
    \label{eq:hybrid_avg_dt_chif}
\end{multline}

For the particular averaged equation we use the momentum formulation $\textbf{p}_\alpha = \textbf{m}_\alpha \textbf{u}_\alpha$ in \ref{eq:dt_q_alpha} and divide by the mass $m_\alpha$, then by averaging we obtain the following euquation,
\begin{equation}
    \pddt \pavg{m_\alpha\textbf{u}_\alpha}
    + \nablabh \cdot\pavg{m_\alpha \textbf{u}_\alpha \textbf{u}_\alpha }
    = \pavg{m_\alpha \textbf{g}}
    + \pavg{\int_{\Sigma_\alpha} \textbf{T}_1 \cdot \textbf{n}_2 d\Sigma},
    \label{eq:hybrid_avg_dt_dp_alpha}\\
\end{equation}

\begin{multline}
    \pddt \pavg{\mathcal{P}_\alpha}
    + \nablabh\cdot \pavg{  \textbf{u}_\alpha \mathcal{P}_\alpha}
    = \pavg{\int_{\Omega_\alpha} \left(
        \rho_2 \textbf{w}_2 \textbf{w}_2
        - \mathbf{T}_2
        \right) d\Omega}\\
        + \pavg{\int_{\Sigma_\alpha} \textbf{r}\textbf{T}_1 \cdot \textbf{n}_2d\Sigma}
    - \pavg{\int_{\Sigma_\alpha} \sigma \textbf{I}_{||} d\Sigma}
    \label{eq:hybrid_avg_dt_dP_alpha}
\end{multline}
where the first moment of the body forces canceled due o the fact that $\textbf{g}$ is constant. 
The fluctuation term reduce to ww because of the consideration of no mass transfer. 

\ref{eq:hybrid_avg_dt_dP_alpha} gives a dynamical equation for the moment of momentum of the particle, but it is also possible to derive a kinematic equation. 
Indeed, the second moment of mass equation gives, 
\begin{equation}
    \pddt \pavg{\mathcal{M}_\alpha}
    + \nablab \cdot \pavg{\mathcal{M}_\alpha \textbf{u}_\alpha}
    = \pavg{(\mathcal{P}_\alpha + \mathcal{P}_\alpha^T)}
    \label{eq:avg_p_G_alpha}
\end{equation}
\tb{disscus the link with the inertial tensor}
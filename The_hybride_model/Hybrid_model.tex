
\subsection{The equivalence between continuous and particular averaged equations.}


It is evident that both method of averaging seen in the previous parts lead to different terms in the equations. 
This part aim to point out the compatibility of both set of equations as they are solved together, namely, the equations produced by the continuous averaged operator defined by \ref{eq:avg_k_global}, and the one defined by the particular averaged formulation \ref{eq:q_avg_p_global}. 
As derivation operators are linear, the latter comparison is equivalent to evaluate the differences between the continuous averaged terms, $\kavg{f_k}$ and $\Iavg{f_k}$, against the particular averaged terms, namely $\pnavg{q_\alpha}$.

To start with, it is well known that the weighting function $g(\textbf{x},\textbf{y})$ in the volume average method can be expressed as a Taylor series expansion around any particles center of mass $\textbf{y}_\alpha$ \citep{jackson1997locally},
\begin{equation}
    g(\textbf{x},\textbf{y})
    = g(\textbf{x},\textbf{y}_\alpha)
    - \textbf{r} \cdot \nablab g(\textbf{x},\textbf{y}_\alpha)
    + \frac{1}{2} \textbf{r}\textbf{r} : \nablab\nablab g(\textbf{x},\textbf{y}_\alpha)
    + \ldots
    \label{eq:g_exp}
\end{equation} 
where, we used the relation $\nablabh g_\alpha = - \nablab g_\alpha$, to make appear the global gradient operators. 
This relation is the starting point to carry out the comparison between both averaging methods. 
Indeed, from this Taylor expansion it can be shown that any phase and interface averaged quantities can be expressed as a Taylor expansion series of particular averaged quantities. 
We can show that any continuous averaged volumetric and surface quantities follows these relations
\begin{align*}
    \avg{\chi_kf_k} 
    &=  \pavg{q_\alpha}
        - \nablab \cdot  
        \pavg{\textbf{Q}_\alpha}        
        + \frac{1}{2} \nablab\nablab : \pavg{\textbf{Q}_\alpha^2}
        + \ldots  \\
    \avg{\delta_I f_k} 
    &=  \pavg{q_{I\alpha}}        
        - \nablab \cdot \pavg{\textbf{Q}_{I\alpha}}
        + \frac{1}{2} \nablab\nablab : \pavg{\textbf{Q}_{I\alpha}^{2}}
        + \ldots  
\end{align*}      
where, we have used the definition of the particular, volume and interface average, together with the \ref{eq:g_exp}. 


\citet{nott2011suspension} demonstrated that the continuous averaged momentum equation, for mono disperse suspension of solid spheres, were strictly equivalent to the particular averaged momentum equation.
While they didn't provide many details on the derivation of this equivalence, they limited their study to mono disperse suspension of solid spherical particles.
Originally, this work pointed out that no term expressed as the divergence of a stress appear in the particular averaged momentum balance.
However, from their arguments, the dispersed phase momentum equation must possess a non-convective terms.
Indeed, since we observe particle migration in suspensions of solid particles \citep{guazzelli2011}, a term express as the divergence of a stress must appear inside the momentum equation even at low inertia.
That is the main argument for the proof of the existence of the so called, \textit{particle-fluid-particle} stress.
Anyhow, our motivation is to extend this equivalence to the whole system of equation of the dispersed two-phase flows model.

Based on the derivation of \citet{nott2011suspension}, in \ref{ap:exp}, we extended their theory to any kind of conservation laws and particles nature and demonstrated, as they did for mono disperse solid particles suspensions, the equivalence between both formalism.
To be brief, in \ref{ap:exp} we demonstrate that the continuous phase averaged non-convective term of \ref{eq:avg_k_global}, i.e.  $\kavg{\mathbf{\Phi}}$, can  be expressed as a multipole expansion.
Likewise, the other terms of \ref{eq:avg_k_global} can also be expanded in such a way.
It turns out that terms of the former expansion cancel out the terms of the latter expansion except for the zeroth order moments of these expansions.
Therefore, the phase average balance, \ref{eq:avg_k_global}, is \textbf{rigorously equivalent} to the corresponding particular averaged laws, i.e. \ref{eq:q_alpha_dt_avg}.
Nevertheless, this property is true if and only if, there is indeed a non-convective term present in the equation, i.e. if $\mathbf{\Phi} \neq \textbf{0}$. Because as stated above it is the expansion of the convective term that cancels out the others terms' expansion.
Therefore, as an example, the phase averaged mass conservation equation, and the particular averaged mass balance are not equivalent.
The same holds for the surface conservation laws, as pointed out by \citet{lhuillier2000bilan}.


To prove in which way those equation are equivalent we start by taking the Taylor expansion of each terms of \ref{eq:dt_dq_alpha} 
\begin{align}    
    0 &= 
    - \pddt \pavg{q_\alpha} +  \nablab \cdot  \partial_t\pavg{\textbf{Q}_\alpha} \ldots\nonumber\\
    &+ \nablab \cdot \pavg{\int_{\Omega_\alpha}\left(\mathbf{\Phi}_k - f_k \textbf{u}_k \right)d\Omega}
    -\nablab\nablab : \pavg{\int_{\Omega_\alpha}\textbf{r}\left(\mathbf{\Phi}_k - f_k \textbf{u}_k \right)d\Omega}
    \ldots\nonumber\\
    &+ \pavg{ \int_{\Omega_\alpha} \textbf{S}_k d\Omega}
    - \nablab \cdot \pavg{ \int_{\Omega_\alpha} \textbf{r}\textbf{S}_k d\Omega}
    \ldots\nonumber\\
    &+ \pavg{\int_{\Sigma_\alpha}\left[
        \mathbf{\Phi}_k
        + f_k
        \left(
            \textbf{u}_I
            - \textbf{u}_k
        \right)
    \right]
    \cdot \textbf{n}_kd\Sigma} \ldots\nonumber\\
    &-  \nablab \cdot \pavg{\int_{\Sigma_\alpha} \textbf{r}\left[
        \mathbf{\Phi}_k
        + f_k
        \left(
            \textbf{u}_I
            - \textbf{u}_k
        \right)
    \right]
    \cdot \textbf{n}_kd\Sigma} \ldots
\end{align}
where we have ignored the term of second order or higher. 
By using the relation : $\int_{\Omega_\alpha} f_k \textbf{u}_k d\Omega = q_\alpha\textbf{u}_\alpha  + \int_{\Omega_\alpha} f_k \textbf{w}_k d\Omega$
together with $\int_{\Omega_\alpha} \textbf{r} \textbf{u}_k f_k d\Omega = Q_\alpha\textbf{u}_\alpha  + \int_{\Omega_\alpha}\textbf{r} f_k \textbf{w}_k d\Omega$ and by  rearranging each terms of the equations we get, 
\begin{align}    
    0 = 
    &- \pavg{\ddt q_\alpha}
    + \pavg{ \int_{\Omega_\alpha} \textbf{S}_k d\Omega}
    + \pavg{\int_{\Sigma_\alpha}\left[
        \mathbf{\Phi}_k
        + f_k
        \left(
            \textbf{u}_I
            - \textbf{u}_k
        \right)
    \right]
    \cdot \textbf{n}_kd\Sigma} \nonumber\\
    &-  \nablab \cdot  \left[
        - \pavg{\ddt \textbf{Q}_\alpha} 
         + \pavg{ \int_{\Omega_\alpha} \left(
            \textbf{r}\textbf{S}_k - \mathbf{\Phi}_k + \textbf{w}_kf_k 
         \right)d\Omega}
    \right.
    \nonumber\\
    &\left. 
         + \pavg{\int_{\Sigma_\alpha} \textbf{r}\left[
        \mathbf{\Phi}_k
        + f_k
        \left(
            \textbf{u}_I
            - \textbf{u}_k
        \right)
        \right]
        \cdot \textbf{n}_kd\Sigma} 
    \right]\nonumber\\
    &+  \frac{1}{2}\nablab\nablab :
         \pavg{ \int_{\Omega_\alpha} 2\textbf{r}\left(\mathbf{\Phi}_k + \textbf{w}_kf_k 
         \right)d\Omega}
         + \ldots
\end{align}
We recognize clearly that each terms of the first line make the \ref{eq:avg_dt_dq_alpha}, and the terms of line two to three make the average of \ref{eq:dt_Q_alpha}
Similar consideration can be made for the surfaces equations, indeed we also derived this equivalence in \tb{APPENDIX}. 
 in agreement with \cite{lhuillier2000bilan} which reach similar conclusion for the area density transport of spherical particle. 

Therefore, in this way \ref{eq:avg_dt_chif} and \ref{eq:avg_dt_deltaf} are completely equivalent to their respective equation at first order, namely \ref{eq:avg_dt_dq_alpha} and \ref{eq:avg_dt_dq_I_alpha}. 

Let's write \ref{eq:avg_dt_chif} as $E(\chi_k f_k) = 0 $ and each moment equation as $M(q_\alpha) = 0$ and $M(Q_\alpha) =0$, then we just showed that, $C(\chi_k f_k) = M(q_\alpha) - \nablab \cdot M(Q_\alpha) \ldots = 0$. 


\subsection{First order description of dispersed two phase flows}

\subsubsection{Continuous phase conservation equation}
\begin{multline}
    \pddt \avg{\chi_1 f_1}
    = \nablabh \cdot \avg{\chi_1 \bm{\Phi}_1 - \chi_1 f_1 \textbf{u}_1}
    + \avg{\chi_1 \textbf{S}_1}
    - \pavg{\int_{\Sigma_\alpha}\left[
        \mathbf{\Phi}_1
        + f_1
        \left(
            \textbf{u}_I
            - \textbf{u}_1
        \right)
    \right]
    \cdot \textbf{n}_2d\Sigma} \\
    +  \nablab \cdot \pavg{\int_{\Sigma_\alpha} \textbf{r}\left[
        \mathbf{\Phi}_1
        + f_1
        \left(
            \textbf{u}_I
            - \textbf{u}_1
        \right)
    \right]
    \cdot \textbf{n}_2d\Sigma} \ldots
    \label{eq:hybrid_avg_dt_chif}
\end{multline}
\subsubsection{Dispersed phase}
If we consider the dispersed phase as the system : particle + interfaces, then write : 
\begin{multline}
    \pddt \pavg{ (q_\alpha+q_{I\alpha})}
    + \nablabh \cdot\pavg{ (q_\alpha+q_{I\alpha}) \textbf{u}_\alpha}
    = \pavg{\int_{\Omega_\alpha} \textbf{S}_2 d\Omega}\\
    + \pavg{\int_{\Sigma_\alpha} \textbf{S}_I d\Sigma}
    + \pavg{\int_{\Sigma_\alpha} \left[\bm{\Phi}_1 + f_1 (\textbf{u}_I-\textbf{u}_1) \right] \cdot \textbf{n}_2 d\Sigma},
    \label{eq:hybrid_avg_dt_dq_alpha}\\
\end{multline}
We can eventually add a second order equations,
\begin{multline}
    \pddt \pavg{(Q_\alpha+Q_{I\alpha})}
    + \nablabh\cdot \pavg{  \textbf{u}_\alpha (Q_\alpha+Q_{I\alpha})}
    =\\ \pavg{\int_{\Omega_\alpha} \left[
        \textbf{r}\textbf{S}_2
        - \mathbf{\Phi}_2
        + f_k\textbf{w}_k
        \right] d\Omega}
    + \pavg{\int_{\Sigma_\alpha} \left[
        \textbf{r}\textbf{S}_I
        - \mathbf{\Phi}_I
        + f_I\textbf{w}_I
    \right] d\Sigma}\\
    + \pavg{\int_{\Sigma_\alpha} \textbf{r}\left[\bm{\Phi}_1 + f_1 (\textbf{u}_I-\textbf{u}_1) \right] \cdot \textbf{n}_2 d\Sigma}\\
    \label{eq:hybrid_avg_dt_dQ_alpha}\\
\end{multline}
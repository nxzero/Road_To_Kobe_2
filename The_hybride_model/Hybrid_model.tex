\subsection{The first order hybrid model}

Now that we reached a clear understanding of the averaged models, we introduce the Hybrid model for dispersed two phase flows. 

As it has been shown in earlier studies \citep{jackson2000dynamics,chu2016flux} it is more physical to express the non-convective flux appearing in \ref{eq:avg_dt_chi_f} with the volume fraction outside the divergence operator. 
These considerations finally leads us to the general form of the continuous phase equation : 
\begin{equation}
    \pddt (\phi_1 f_1)
    + \div (\phi_1 f_1\textbf{u}_1
    + \phi_1 \bm{\Phi}^\text{Re}_1
    - n_p \textbf{M}_p )
    = 
    \phi_1(s_1  +\div \mathbf{\Phi}_1)
    - n_p \textbf{F}_p
    \label{eq:hybrid_avg_dt_chif}
\end{equation}
Where we introduced the following definition, 
\begin{align*}
    \textbf{F}_\alpha 
    &= 
    \int_{\Sigma_\alpha}
    \left(
        \mathbf{\Phi}_1^0 
        - \mathbf{\Phi}_1
    \right)  
    \cdot \textbf{n}_2d\Sigma
    + 
    \int_{\Sigma_\alpha}f_1^0
    \left(
        \textbf{u}_I^0
        - \textbf{u}_1^0
    \right)
    \cdot \textbf{n}_2d\Sigma,\\
    \textbf{M}_\alpha 
    &= \int_{\Sigma_\alpha} \textbf{r}
        \left(\mathbf{\Phi}_1^0- \mathbf{\Phi}_1\right)\cdot \textbf{n}_2d\Sigma
        + \int_{\Sigma_\alpha} \textbf{r}f_1^0
        \left(
            \textbf{u}_I^0
            - \textbf{u}_1^0
        \right)
    \cdot \textbf{n}_2d\Sigma\\
    \textbf{f}^\text{Re}_1
    &= \oneavg{f_1' \textbf{u}_1'} 
\end{align*}
where we neglected the second order moment or higher moments. 

Regarding the dispersed phase, we consider the zeroth and first moment equation (\ref{eq:avg_dt_dq_alpha_tot} and \ref{eq:avg_dt_dQ_alpha_tot}) as  a part of the hybrid model.
Under this form the couplings terms in \ref{eq:hybrid_avg_dt_chif} do not correspond to the ones appearing in both, \ref{eq:avg_dt_dq_alpha_tot} and \ref{eq:avg_dt_dQ_alpha_tot} which is essential for ensuring a consistent model. 
To fix this problem we reformulated both exchange terms following \citet{zhang1997momentum} such that it correspond to $\mathbf{F}_\alpha$ and $\textbf{M}_\alpha$. 
Then we may rewrite the momentum and moment of momentum equations (\ref{eq:avg_dt_dq_alpha_tot} and \ref{eq:avg_dt_dQ_alpha_tot}) as : 
\begin{multline}
    \pddt \left(n_p q_p^\text{tot}\right)
    + \div \left(n_p\textbf{u}_p
    q_p^\text{tot} + 
    \textbf{q}_p^\text{Re}
    \right)
    = 
    n_p v_p  \div \mathbf{\Phi}_1 
    + n_p \textbf{F}_p,
    + \pnavg{\int_{\Omega_\alpha} s_2 d\Omega}
    + \pnavg{\int_{\Sigma_\alpha} s_I d\Sigma}
    \label{eq:hybrid_avg_dt_q_alpha}
\end{multline}
\begin{multline}
    \pddt \left(n_p \mathcal{Q}_p^\text{tot}\right)
    + \div \left(
        n_p \textbf{u}_p \mathcal{Q}_p^\text{tot}
    + \mathcal{Q}_p^\text{Re}
    \right)
    =
    n_p v_p \mathbf{\Phi}_1 
    + n_p \textbf{M}_p\\
    \pnavg{\int_{\Omega_\alpha} \left(
        \textbf{r} s_2^0         
        + f_2^0  \textbf{w}_2^0 
        - \mathbf{\Phi}_2^0
        \right) d\Omega}
        + \pnavg{\int_{\Sigma_\alpha} \left(
        \textbf{r}s_I^0
        + f_I^0 \textbf{w}_I^0
        - \mathbf{\Phi}_{I||}^0
    \right) d\Sigma},
    \label{eq:hybrid_avg_dt_Q_alpha}
\end{multline}
respectively. 
Where  we introduced the following notation
\begin{align*}
    \textbf{q}_p^\text{Re}
    &= 
    \pnavg{\textbf{u}_\alpha'(q_\alpha'+q_{I_\alpha}')} \\
    \mathcal{Q}_p^\text{Re}
    &= 
    + \pnavg{\textbf{u}_\alpha'(\mathcal{Q}_\alpha'+\mathcal{Q}_{I_\alpha}')'}
\end{align*}

Then, \ref{eq:hybrid_avg_dt_chif}, \ref{eq:hybrid_avg_dt_q_alpha} and \ref{eq:hybrid_avg_dt_Q_alpha}, are the most general form of a first-order accurate hybrid model for arbitrary particle. 
Lastly, it is worth noting that by incorporating more terms in the expansion of \ref{eq:avg_dt_chi_f} and higher order moments equations, one can reach an arbitrary order of accuracy, as stated by \citet{zhang1997momentum}. 

As remarked by \citet{jackson1997locally} for the angular momentum equations of solid spherical particles, and here in a more general case :\ref{eq:hybrid_avg_dt_chif} and \ref{eq:avg_dt_dq_alpha_tot} may seem to be not coupled with the higher order moments equations, i.e. \ref{eq:avg_dt_dQ_alpha_tot}. 
Indeed, the first order moment $\mathcal{Q}_\alpha$ do not appear explicitly in either \ref{eq:hybrid_avg_dt_chif} or \ref{eq:avg_dt_dq_alpha_tot}.
However, the exchange terms 
$\textbf{F}_p$ 
and 
$\textbf{M}_p$ 
appearing in both \ref{eq:hybrid_avg_dt_chif} and \ref{eq:avg_dt_dq_alpha_tot} might depend on the higher order moments of the particles.
As an example, in the momentum equation of the continuous phase, i.e. the ensemble average of \ref{eq:dt_rhou_k}, the exchange term corresponds to the averaged drag force, namely $\textbf{f}_p$. 
It is clear that $\textbf{f}_p$ has a strong dependency with $\textbf{u}_p$ and $\mathcal{M}_p$ since the drag force is a function of both the particle's velocity and its shape. 
Therefore, \ref{eq:avg_dt_dQ_alpha_tot} is linked to \ref{eq:hybrid_avg_dt_chif} and \ref{eq:avg_dt_dq_alpha_tot} solely through the dependence between the exchange terms and the properties of the particles, e.g. $q_p$, $\mathcal{Q}^n_p$ and possibly the higher order moments. 
Consequently, the significance of the higher moments equations can be evaluated based on the dependency of the exchange terms in \ref{eq:hybrid_avg_dt_chif} with the moments,  $\mathcal{Q}_p$ and potentially the higher-order moments of the particles. 

\subsection{The minimal set of equations for a mono-disperse suspension}
We now expose the minimal set of equation for a mono disperse suspension of arbitrary particle. 

\subsubsection{Continuous phase}

We start by the continuous phase. 
Applying the above methodology, we express the mass, momentum and total energy of the continuous phase by, 

\subsubsection{The momentum exchange term}

As mentioned in the previous section all the exchange term must be expanded in a Taylor series to correspond to the particle phase terms. 
Here we focus on the drag force term which will be a bit special
\tb{The PFP must be introduced here the contact forces maybe before}


\subsubsection{A brief comments on the averaged stress}

Before exposing the hybrid model it is important to begin with the formulation of the fluid phase averaged stress.
Indeed, this formulation will determine the shape of the averaged model.
For instance the stress appearing on the left hands side of the fluid phase momentum balance is of the form, 
$\phi_1 (\rho_1\kavg{\textbf{u}_1'\textbf{u}_1'} - \bm{\tau}_1) - \mathscr{F}_p n_p$. 
Developing,  $\bm{\tau}_1\phi_1$  gives for Newtonian fluids,
\begin{equation*}
    \phi_1 \phi_1
    = \mu_1 \textbf{e} - \mu_1 \phi_2 \textbf{e}_2
\end{equation*}
where the last term is sometime referred as the extra deformation tenors and, one can find closure in \citet{ishii2010thermo}. 

However, here we argue that is term is rather part of the definition of the stresslet.
Indeed, in stokes theory we have the following definition,  




\subsubsection{Dispersed phase}


\subsection{Second order equations for higher accuracy}
Since the energy of the particle phase must be completed we can add those shapes equations




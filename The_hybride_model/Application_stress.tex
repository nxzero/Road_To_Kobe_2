\section{The equivalent stresses of an emulsion}

We consider a multiphase flow with no-mass transfer, i.e. $\textbf{u}_k=\textbf{u}_I$.
The interfaces have negligible weight and no interfacial viscosity. 

\subsection{Momentum equation}
The  continuous and particle averaged momentum equation of the continuous and dispersed phase now reads as, 
\begin{align*}
    \pddt (\rho_1\phi_1 \textbf{u}_1 )
    + \div (\rho_1\phi_1 \textbf{u}_1  \textbf{u}_1
    + \bm{\sigma}^\text{eff}_1 
    - \phi_1 \bm{\sigma}_1)
    &= 
    \phi_1 \textbf{b}_1
    - n_p \textbf{f}_p \\
    \pddt (n_p m_p \textbf{u}_p)
    + \div (
        n_p m_p \textbf{u}_p\textbf{u}_p
        + \bm{\sigma}^\text{eff}_p
        )
    &= 
     n_p v_p 
      \textbf{b}_2
    + n_p \textbf{f}_p ,
\end{align*}
Alternatively, they can be written as :
\begin{align*}
    \pddt (\rho_1\phi_1 \textbf{u}_1 )
    + \div (\rho_1\phi_1 \textbf{u}_1  \textbf{u}_1
    + \bm{\sigma}^\text{eff}_1 )
    &= 
    \phi_1( \textbf{b}_1  +\div  \bm{\sigma}_1)
    - n_p \textbf{f}_p \\
    \pddt (n_p m_p \textbf{u}_p)
    + \div (
        n_p m_p \textbf{u}_p\textbf{u}_p
        + \bm{\sigma}^\text{eff}_p
        )
    &= 
     n_p v_p (
      \textbf{b}_2
    + \div \bm{\sigma}_1 )
    + n_p \textbf{f}_p ,
\end{align*}
With the following definition, 
\begin{align*}
    \bm{\sigma}^\text{eff}_1
    &=\rho_1 \phi_1 \oneavg{\textbf{u}_1'\textbf{u}_1'}
    - n_p \textbf{M}_p 
    \\
    \bm{\sigma}^\text{eff}_2
    &= m_p \pnavg{ \textbf{u}_\alpha'\textbf{u}_\alpha'} 
\end{align*}
With the exchange terms function of teh relative or local stress $\bm{\sigma}_1'=\bm{\sigma}_1^0 - \bm{\sigma}_1$. 

Now we start by deriving each stress considering Newtonian fluid for the dispersed phase. 
\begin{align*}
    \bm{\sigma}_1 
    &= - p_1 \textbf{I}
    + \frac{\mu_1 }{\phi_1} \textbf{e}
    - \frac{\lambda \mu_2 \phi_2}{\phi_1} \textbf{e}_2\\
    \bm{\sigma}_1 
    &= - \left(p_1 + \frac{\lambda }{\phi_1}\phi_2 p_2\right) \textbf{I}
    + \frac{\mu_1}{\phi_1} \textbf{e}
    - \frac{\lambda }{\phi_1} \phi_2 \bm{\sigma}_2 \\
    \phi_1 \bm{\sigma}_1 
    &= - (\phi_1 p_1+ \lambda \phi_2 p_2) \textbf{I}
    + \mu_1 \textbf{e}
    - \lambda \phi_2 \bm{\sigma}_2\\
\end{align*}
Injecting this last equation onto the fluid phase formula gives, 
\begin{align*}
    \pddt (\rho_1\phi_1 \textbf{u}_1 )
    + \div (\rho_1\phi_1 \textbf{u}_1  \textbf{u}_1
    + \bm{\sigma}^\text{Re}_1 
    + \phi_1 p_1 \textbf{I}
    - \mu_1 \textbf{e}
    + \lambda \phi_2 \mu_2 \bm{e}_2)
    &= 
    \phi_1 \textbf{b}_1
    - \avg{\delta_I \bm{\sigma}_1 \cdot \textbf{n}_2} \\
\end{align*}
We can substitue the particle phase stress with it's respective momentum equation,
\begin{equation*}    
\div (\phi_2 \mu_2 \bm{e}_2)
=  
\grad (\phi_2 p_2 \textbf{I})
-\pddt (\phi_2\rho_2 \textbf{u}_2)
- \div (\phi_2\rho_2 \textbf{u}_2 \textbf{u}_2)
- \phi_2 \textbf{b}_2 
- \avg{\delta_I
    \bm{\sigma}_1^0
\cdot \textbf{n}_2}
- \div (\phi_I \bm{\sigma}_I) ,
\end{equation*}
Which gives, 
\begin{multline*}
    \pddt (\rho_1\phi_1 \textbf{u}_1 -\lambda \phi_2 \rho_2 \textbf{u}_2  )
    + \div (
        \rho_1\phi_1 \textbf{u}_1  \textbf{u}_1
        - \lambda\rho_2\phi_2 \textbf{u}_2  \textbf{u}_2
    + \bm{\sigma}^\text{Re}_1 - \lambda \bm{\sigma}^\text{Re}_2 \\
    + (\phi_1 p_1 + \phi_2 \lambda p_2) \textbf{I}
    - \mu_1 \textbf{e})
    = 
    \phi_1  \textbf{b}_1
    + \lambda \phi_2  \textbf{b}_2
    - (1-\lambda)\avg{\delta_I \bm{\sigma}_1 \cdot \textbf{n}_2}
    + \div (\phi_I \bm{\sigma}_I) \\
\end{multline*}


A more easy way to model these,
\begin{multline}
    \int_{\Omega_\alpha} 
    (\bm{\sigma}_2^0)_{ik}
    d\Omega
    = 
    - \int_{\Sigma_\alpha} 
    (\bm{\sigma}_I)_{ik}
    d\Sigma
    +  \int_{\Omega_\alpha} \rho_2 
    (\textbf{w}_2^0\textbf{w}_2^0  )_{ik}
    d\Omega
    -\ddt \int_{\Omega_\alpha} r_i (\textbf{u}^0_2)_k \Omega
    +\int_{\Sigma_\alpha} 
     r_i (\bm{\sigma}_1^0 \cdot \textbf{n}_2)_{k}
    d\Sigma
\end{multline}
Besides using the general formula \ref{eq:dt_Q_n} for $n = 2$ we obtained the relation, 
\begin{multline}
    \int_{\Omega_\alpha} (r_{j}(\bm{\sigma}^0_2)_{ki}+r_{k}(\bm{\sigma}^0_2)_{ji})d\Omega
    +\int_{\Sigma_\alpha} (r_{j}(\bm{\sigma}^0_I)_{ki}+r_{k}(\bm{\sigma}_I^0)_{ji})d\Sigma
    = \\
    - \ddt\int_{\Omega_\alpha} \rho_2 (\textbf{u}_2^0)_i r_j r_k d\Omega
    + \int_{\Omega_\alpha} \rho_2 (r_{j} (\textbf{w}_2^0)_k (\textbf{u}^0_2)_i + r_k (\textbf{w}_2^0)_j (\textbf{u}^0_2)_i)d\Omega\\
    +\int_{\Sigma_\alpha}  r_{k}r_{j} (\bm{\sigma}_1^0)_{il} (\textbf{n}_2)_l d\Sigma
    + \int_{\Omega_\alpha} r_{k}r_{j}  \rho_2 d\Omega g_i
\end{multline}
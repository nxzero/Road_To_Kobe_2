\section{Arbitrary order equivalence}
\label{ap:equivalence}
In this appendix, we provide a general proof of \ref{eq:scheme_equivalence}. 
Let's begin by re-writing the phase averaged equation:
\begin{equation}
        \pddt \avg{\chi_d f_d^0}
        = \div \avg{\chi_d \bm\Phi_d^0 - \chi_d f_d^0 \textbf{u}_d^0}
        + \avg{\chi_d s_d^0}
        + \avg{\delta_\Gamma\left[
            \bm\Phi_d^0
            + f_d^0
            \left(
                \textbf{u}_\Gamma^0
                - \textbf{u}_d^0
            \right)
        \right]
        \cdot \textbf{n}_d}.
        \label{eq:dt_f_d_O}
\end{equation}
Our objective here is to demonstrate how \ref{eq:dt_f_d_O} is related to the Lagrangian moments' equation, given by \ref{eq:dt_q_n}. 
The first step is to expand each term of \ref{eq:dt_f_d_O} using the relation \ref{eq:f_exp} which gives directly,
\begin{align*}
        0 &=
        - \pddt \expo{f_d^0} \\
        &+\div \expo{(\bm\Phi_d^0  - f_d^0 \textbf{u}_d^0)}\\
        &+ \expo{ s_d^0}\\
        &+ \expoS{\left[
            \bm\Phi_d^0
            + f_d^0
            \left(
                \textbf{u}_\Gamma^0
                - \textbf{u}_d^0
            \right)
        \right]
        \cdot \textbf{n}_d} \\
\end{align*}
The third term can be reformulated using the decomposition : $\textbf{u}_d^0 = \textbf{u}_\alpha + \textbf{w}_d^0$, which gives,
\begin{multline}
    \expo{f_d^0 \textbf{u}_d^0}\\
    =     \expoU{f_d^0 }\\
    +     \expo{f_d^0 \textbf{w}_d^0}
\end{multline}
Injecting this formulation in the former equation yields,
\begin{align}
    & \pddt \expo{f_d^0} \\
    &+ \div \expoU{f_d^0}\\
    &= \div \expo{(\bm\Phi_d^0 - f_d^0 \textbf{w}_d^0)}\\
    &+ \expo{ s_d^0}\\
    &+ \expoS{\left[
        \bm\Phi_d^0
        + f_d^0 
        \left(
            \textbf{u}_I^0
            - \textbf{u}_d^0
        \right)
    \right]
    \cdot \textbf{n}_d} \\
    \label{eq:nearly_done}
\end{align}
Then, note that the first term on the right-hand side can be re-written when evaluated at the order $n-1$, as follows,
\begin{multline*}
    \div \expo[{(n-1)}]{(\bm\Phi_d^0 - f_d^0 \textbf{w}_d^0)}
    = \\
    \frac{(-1)^{n}}{{(n)}!} \partialp{1}{n}  n \pOavg{ \pri{1}{n-1}(f_d^0 \textbf{w}_d^0 - \bm\Phi_d^0)_{i_{n}} }
\end{multline*} 
Upon using that expression in \ref{eq:nearly_done} we can factor out the gradient operators, i.e. the $\frac{(-1)^n}{n!} \partialp{1}{n}$, which gives, 
\begin{multline}
    0 = \frac{(-1)^n}{n!}
    \partialp{1}{n}
    \left[
        - \partial_t
        \pavg{(\textbf{q}_\alpha^{(n)})_{i_1\ldots i_n}}
        - \div \pavg{\textbf{u}_\alpha (\textbf{q}_\alpha^{(n)})_{i_1\ldots i_n}}
    \right.\\\left.
        +n\pavg{\int_{\Omega_\alpha} \pri{1}{n-1} (f_d^0 \textbf{w}_d^0-\bm\Phi_d^0) d\Omega}
        +\pavg{\int_{\Omega_\alpha} \pri{1}{n} s_d^0 d\Omega}
        \right.\\\left.
        +\pavg{\int_{\Omega_\alpha} \pri{1}{n} \left[
            \bm\Phi_d^0
            + f_d^0
            \left(
                \textbf{u}_I^0
                - \textbf{u}_d^0
            \right)
        \right]
        \cdot \textbf{n}_d d\Omega}
    \right].
    \label{eq:exp_f_d_O}
\end{multline}
At this stage, one might immediately recognize \ref{eq:dt_q_n} in the square bracket. 
However, note that the third term of \ref{eq:exp_f_d_O} differs from the second term of \ref{eq:dt_q_n}. 
Indeed, 
\begin{equation}
    n\pavg{\int_{\Omega_\alpha} \pri{1}{n-1} ( f_d^0 \textbf{w}_d^0-\bm\Phi_d^0)_{i_n} d\Omega}
    \neq
    \sum_{e=1}^{n} 
    \avg{
        \intO{
        \prod^{n}_{\substack{ m=1 \\m \neq e}} r_{i_m} (f_d^0 \textbf{w}_d^0  - \bm\Phi_d^0)_{i_e}
        }
    }. 
    \label{eq:ineq_which_does_not_make_sens}
\end{equation}
Even if the above inequality holds as it is, it must be understood that what matter in \ref{eq:exp_f_d_O} is the gradient of that term. 
Additionally, the contraction of one of these terms with the gradient operator $\partialp{1}{n}$ makes the skew-symmetric part of these tensors (in the indices $i_1\ldots i_n$) having a vanishing contribution to the expression.
% Moreover, if one apply the operator $\partialp{1}{n}$ on each side of \ref{eq:ineq_which_does_not_make_sens} he eventually finds that, 
Thus, one might apply the operator $\partialp{1}{n}$ on each side of \ref{eq:ineq_which_does_not_make_sens} and by noticing that the skew-symmetric part of the left-hand side of \ref{eq:ineq_which_does_not_make_sens} vanish, this leads to, 
\begin{equation*}
    \partialp{1}{n}\left[
        n\pavg{\int_{\Omega_\alpha} \pri{1}{n-1} ( f_d^0 \textbf{w}_d^0-\bm\Phi_d^0)_{i_n} d\Omega}
        \right]
    =
    \partialp{1}{n}\left[
    \sum_{e=1}^{n} 
    \avg{
        \intO{
        \prod^{n}_{\substack{ m=1 \\m \neq e}} r_{i_m} (f_d^0 \textbf{w}_d^0  - \bm\Phi_d^0)_{i_e}
        }
    }
    \right]. 
    \label{eq:it_make_sens_again}
\end{equation*}
Injecting this last equality in \ref{eq:exp_f_d_O} and using \ref{eq:dt_Qgamma_n} to reformulate the interfacial term gives directly, 
\begin{multline}
    0 = \frac{(-1)^n}{n!}
    \partialp{1}{n}
    \left[
        - \pddt \pavg{(\textbf{Q}_\alpha^{(n)})_{i_1\ldots i_n}^\alpha}
        - \div  \pavg{\textbf{u}_\alpha (\textbf{Q}_\alpha^{(n)})_{i_1\ldots i_n}^\alpha}
        = \phantom{\pOavg{\pri{}{n}}}\right.\\\left.
            +\sum_{e=1}^{n} 
        \pOavg{
            \prod^{n}_{\substack{ m=1 \\m \neq e}} r_{i_m} (f_d^0\textbf{w}_d^0  - \bm\Phi_d^0)_{i_e}
        }
        + \pOavg{ \pri{1}{n} (\textbf{s}_d^0)_k }\right.\\\left.
        +     
        \sum_{e=1}^{n} 
        \pSavg{
            \prod^{n}_{\substack{ m=1 \\m \neq e}} r_{i_m} (f_\Gamma^0\textbf{w}_\Gamma^0 - \bm\Phi_{||\Gamma}^0)_{i_e}
        }
        + \pSavg{ \pri{1}{n} (\textbf{s}_\Gamma^0)_k } \right.\\\left.
        +\pSavg{ \pri{1}{n} ([\bm\Phi_f^0 + \textbf{f}_f^0 \left(\textbf{u}_\Gamma^0 - \textbf{u}_f^0\right)]\cdot \textbf{n}_d)_k }. 
    \right],
\end{multline}
which proves \ref{eq:scheme_equivalence}.

\subsection{The averaged conservation equations}
\label{sec:avg_def}
In this study, we employ the ensemble average technique to establish the averaged conservation equations. 
This method is just one of several averaging approaches, including the volume average method \citep{jackson1997locally} and time averaging \citep{ishii2010thermo}. 
Although these techniques differ, they ultimately produce the same set of averaged equations, as evidenced by comparing volume averaging theory with ensemble averaging theory \citep{jackson1997locally,zhang1997momentum}. However, for slightly non-uniform suspensions, ensemble averaging is better suited for deriving the averaged equations because it does not require assumptions about the characteristic length scale of the spatial filter \citep{lhuillier1992ensemble}.
%Despite their differences, all these techniques yield the same set of averaged equations as demonstrated by comparing volume averaging theory to ensemble averaging theory \citep{jackson1997locally,zhang1997momentum}. However, for slightly non-uniform suspension, ensemble avergaing is konw to pe a more porwerfull to derive the set of averaged equationin comparison to volume averaging, since it does not require any assumption on the chata=racteris legnth scale iof the spatial filter \citep{lhuillier1992ensemble}.%with respet  the ensemble averaging is known to provide 
In the following we recall some properties of the ensemble average operator. 
Let, $P(\FF)$ be the probability density function that describes the probability of finding the flow in the configuration $\FF$. 
We note $d\PP = P(\FF) d\FF$ the probable number of flows located in the incremental region of the phase space $d\FF$ around the point $\FF$. 
It follows from this definition, that the ensemble average of an arbitrary local property $f^0(\textbf{x},t;\FF)$ defined on the whole space $\Omega$, is,
\begin{equation}
    f(\textbf{x},t)
    = \avg{f^0}(\textbf{x},t)
    =\int f^0(\textbf{x},t;\FF) d\mathscr{P}. 
    \label{eq:avg}
\end{equation}  
Note that we dropped the super script $^0$ on $f(\textbf{x},t)$ to indicate that this is an averaged quantity. 
The macroscopic variables are averaged over all $\FF$, and therefore depend only on $\textbf{x}$ and $t$.
Thus, we omit the arguments of the averaged fields, as this notation eliminates any potential ambiguity. 
The ensemble average operator is assumed to satisfy the following properties \citep{drew1983mathematical}
\begin{align}
    \avg{f^0+h^0} = f+h, 
    && \avg{\avg{f^0}h^0} = fh, \nonumber\\
    \avg{\pddt f^0} 
    = \pddt f,  
    &&\avg{\grad f^0}
    = \grad f. 
    \label{eq:avg_properties}
\end{align}
were $f^0$ and $h^0$ are two arbitrary Eulerian fields defined over $\Omega$. 
The first two relations are called the Reynolds' rules, the third one is the Leibniz' rule and the last one, is the Gauss' rule \citep{drew1983mathematical}.
Additionally, for any phase quantity defined in $\Omega_k$ we introduce the definition, 
\begin{equation}
    \phi_k f_k (\textbf{x},t) = \avg{\chi_k f_k^0},
    \label{eq:1_avg}
\end{equation}
where $\phi_k(\textbf{x},t) = \avg{\chi_k}$ is the probability of finding the phase $k$ at the location \textbf{x} and time $t$.
And $f_k$ is the average of the field $f_k^0$ conditioned on the presence of the phase $k$ in the configuration $\FF$ at $\textbf{x}$ and time $t$.
Equally, for interface quantities we have 
\begin{equation}
    \phi_\Gamma f_\Gamma (\textbf{x},t) = \avg{\delta_\Gamma f_\Gamma^0},
\end{equation}
with $\phi_\Gamma = \avg{\delta_\Gamma}$ the interface area per unit volume, also called the specific area at the point \textbf{x} at time $t$. 
Here, $f_\Gamma$ is the average of $f^0_\Gamma$ conditioned on the presence of an interface in the configuration $\FF$ at $\textbf{x}$ and time $t$. 
Additionally, we define the field of fluctuation of a given quantity around its mean as,
\begin{align}
    f'(\textbf{x},t,\FF) = f^0(\textbf{x},t,\FF) - f(\textbf{x},t).
    \label{eq:def_fluctu}
\end{align}
This relation applies to phase averaged quantities such that $f'_k = f^0_k - f_k$ and $f'_\Gamma = f^0_\Gamma - f_\Gamma$. 


Applying the ensemble average on \ref{eq:dt_chi_k_f_k} and \ref{eq:dt_delta_I_f_I} and considering the properties from \ref{eq:avg_properties} to \ref{eq:def_fluctu}, yields the general form of the averaged equations of multiphase flows, namely,
\begin{align}
    \pddt (\phi_k f_k)
    +\div (\phi_k f_k \textbf{u}_k + \mathbf{\Phi}_k^\text{eq})
    &= 
    \phi_k s_k
    + \avg{\delta_\Gamma\left[
        \mathbf{\Phi}_k^0
        + f_k^0
        \left(
            \textbf{u}_\Gamma^0
            - \textbf{u}_k^0
        \right)
    \right]
    \cdot \textbf{n}_k},
    \label{eq:avg_dt_chi_f}\\
    \pddt (\phi_\Gamma f_\Gamma)
    +\div (\phi_\Gamma f_\Gamma \textbf{u}_\Gamma+ \mathbf{\Phi}_\Gamma^\text{eq})
    &= 
    \phi_\Gamma s_\Gamma
    - \avg{\delta_\Gamma 
    \Jump{
    \mathbf{\Phi}_k^0
    + f_k^0 (\textbf{u}_\Gamma^0 - \textbf{u}_k^0)
    } 
     },
    \label{eq:avg_dt_delta_f}
\end{align}
with, 
\begin{align}
    \mathbf{\Phi}_k^\text{eq}
    = \avg{\chi_k f_k' \textbf{u}_k'}
    - \phi_k \bm\Phi_k,
    &&
    \mathbf{\Phi}_\Gamma^\text{eq}
    = \avg{\delta_\Gamma f_\Gamma' \textbf{u}_\Gamma'}
    - \phi_\Gamma \bm\Phi_\Gamma. 
\end{align}
These equations are to be solved for the averaged field $f_f,f_d$, and $f_\Gamma$.
The main differences between these equations and their microscale counterparts (\ref{eq:dt_f_k} and \ref{eq:dt_f_I}) are:
(1) The unknowns are now averaged quantities,
(2) factors $\phi_k$ and $\phi_\Gamma$ are introduced in front of all the terms, and
(3) the additional terms $\avg{\chi_k f_k' \textbf{u}_k'}$ and $\avg{\delta_\Gamma f_\Gamma' \textbf{u}_\Gamma'}$ appear, representing the covariance between the conserved quantity ($f_k$ or $f_\Gamma$) and the local velocities.  
%For a complete understanding, we derived the mass, momentum, and energy averaged equations in \ref{ap:two-fluid_model}. 
%These are derived considering the simplifying hypothesis exposed in \ref{ap:hypothesis}. 
%In addition, \ref{ap:two-fluid_model} presents how to derive the secondary averaged equations of the averaged energy $E_k$, i.e. the equation for the mean internal energy $e_k$, the pseudo turbulent energy $k_k = \frac{1}{2\phi_k}\avg{\chi_k (u'_k)^2}$, and the averaged kinetic energy $(u_k)^2/2$.  


It is important to highlight that the two-fluid model fails to adequately distinguish between the two phases, as evidenced by the \textit{symmetry} $k = f$ and $d$ in the aforementioned equations. This symmetry does not hold physically because the dispersed phase possesses a distinct topological nature compared to the continuous phase. 
In a dispersed two-phase flow system, the closure terms are typically expressed as functions of the Lagrangian properties of the particles \citep{jackson2000}. In contrast, the current system of equations yields continuously averaged quantities, which are not directly connected to the Lagrangian properties.
%More importantly, in a dispersed two-phase flow system the closure terms are most often expressed as a function of the Lagrangian properties of the particles \citep{jackson2000} whereas the present system of equation provides us with continuously averaged quantities which are not directly linked to the lagrangian properties.
%Specifically, the mean drag force or torque term in the averaged momentum equation is expressed as a function of the center of mass linear and angular velocity  of the particles. 
%Whereas this system of equation provides us with the phase averaged velocity of the whole phase not with no consideration for the particles properties.  
Therefore, in the subsequent section, we will introduce a kinetic-based model specifically devoted to the dispersed phase. 
As illustrated below, the equations governing the dispersed phase are more comprehensive as they bear a resemblance to the equations governing a single particle.


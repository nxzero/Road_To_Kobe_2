In this study we use the statistical approach to derive the averaged equations of conservation. 
Specifically, we use the method described in \citet{zhang2021ensemble} which extended the ensemble average definition of \citet{batchelor1972sedimentation}. 
In the following we recall some properties of the ensemble average operator. 

\subsection{Ensemble average}

Let, $P(\mathscr{C},t)$ be the probability density function that describe the probability of finding the flow in the configuration $\mathscr{C}$ at time $t$, were $\mathscr{C} = (\lambda_1,\lambda_2,\lambda_3,\ldots)$ is a finite set of all the parameters describing the flow configuration. 
Then, we define $d\mathscr{P} = P(\mathscr{C},t)d\mathscr{C}$ as the probable number of particles in the incremental region of the particles' phase space $d\mathscr{C}$ around $\mathscr{C}$ at time $t$. 
It follows from this definition, that the ensemble average of an arbitrary local property $f^0$ defined on $\Omega$, yields,
\begin{equation}
    f(\textbf{x},t)
    = \avg{f^0}(\textbf{x},t)
    =\int f^0(\textbf{x},\mathscr{C},t) d\mathscr{P}. 
    \label{eq:avg}
\end{equation}  
This definition can be applied to Lagrangian properties as well by using the previous formulation, namely $\pavg{q_\alpha}(\textbf{x},t) = \int \delta(\textbf{x} - \textbf{x}_\alpha) q_\alpha(\mathscr{C}) d\mathscr{P}. $
It is interesting to mention some mathematical properties of the ensemble average operators. 
For two arbitrary Eulerian fields $f$ and $h$ we have,
\begin{align}
    &\avg{f^0+h^0} = f+h, 
    &\avg{\avg{f^0}h^0} = gh, \nonumber \\
    &\avg{\pddt f^0} 
    = \pddt f, 
    &\avg{\grad f^0}
    = \grad f. 
    \label{eq:avg_properties}
\end{align}
\todo{Either $P$ or $\Pi$ depend on time actually. on my whole work i haven't done that}

The two first relations are called the Reynolds' rules, the $3^{th}$ one is the Leibniz' 
rule and the last one, the Gauss' rule \citep{drew1983mathematical}.
Additionally, for any phase quantity defined in $\Omega_k$ we introduce the definition, 
\begin{equation}
    \phi_k f_k (\textbf{x},t) = \avg{\chi_k f_k^0}
    \label{eq:1_avg}
\end{equation}
where, $\phi_k(\textbf{x},t) = \avg{\chi_k}$ is the volume fraction of the phase $k$
and $f_k$ the conditional average of the field $\chi_k f_k^0$ on the phase $k$.
Similarly, for the particle fields, we can introduce the particle phase average by,
\begin{equation}
     n_p q_p(\textbf{x},t) = \avg{\delta_\alpha q_\alpha}
     \label{eq:p_avg}
\end{equation}
where, $n_p(\textbf{x},t) = \avg{\delta_\alpha}$ is the probable number of finding a particle center of mass at $\textbf{x}$
and $q_p$ is the conditional average of $q_\alpha$ on the set of particles. 
We also define the fluctuation of a particle-averaged and phase-averaged quantity by,
\begin{equation}
    q_\alpha' = q_\alpha - q_p
    \;\;\;\;\;\;\text{and}
    \;\;\;\;\;\;
    f_k' = f_k^0 - f_k,
    \label{eq:def_fluctu}
\end{equation}
respectively. 
These definitions lead to the following properties, $\avg{\chi_1 f'_k} = 0$ and $\avg{\delta_\alpha q_\alpha'} =0$. 
Consequently, the product $\avg{\chi_k f^0_kg^0_k}$ can be decomposed as $\avg{\chi_k f_k^0g_k^0}=\phi_k f_kg_k + \avg{\chi_k f'_kg'_k}$. 
These decompositions will play a crucial role in the upcoming section. 
\todo{more statistical / mathematical comments about covarience }

The ensemble average method is one among others, such as the volume average method\citep{jackson1997locally} or the time average
\citep{ishii2010thermo}.
It is shown that all of these averaging technics remains equivalent \citep{jackson1997locally}. 


\subsection{The averaged model}

Using the generic formulation \ref{eq:hybrid_avg_dt_chif} and the local expression of the mass, momentum and total energy expression, i.e. : \ref{eq:dt_rho},\ref{eq:dt_rhou_k} and \ref{eq:dt_rhoE_k} we easily find the averaged form of these equations as, 

\begin{align}
    \label{eq:dt_avg_rho}
    \pddt (\phi_k \rho_k)  
    + \div (
        \phi_k \rho_k\textbf{u}_k
    )
    &= 
    0\\
    \label{eq:dt_avg_rhou_k}
    \pddt (\phi_k \rho_k\textbf{u}_k)  
    + \div (
        \phi_k \rho_k\textbf{u}_k\textbf{u}_k
        - \bm{\sigma}_k^\text{eq}
    )
    &= 
    \phi_k \rho_k \textbf{g} 
    +  \avg{\delta_I \bm{\sigma}_k^0 \cdot \textbf{n}_k}\\
    \label{eq:dt_avg_rhoE_k}
    \pddt (\phi_k\rho_kE_k)  
    + \div (
        \phi_k\rho_kE_k\textbf{u}_k
        + \bm{q}_k^\text{eq}
        - \textbf{u}_k \cdot \bm{\sigma}_k^\text{eq}
        % - \textbf{u}_k^0 \cdot \bm{\sigma}_k^0 
        % + \textbf{q}_k^0
        )
    &= 
    \phi_k \rho_k\textbf{u}_k \cdot \textbf{g} 
    + \avg{\delta_I (\textbf{u}_k^0 \cdot \bm{\sigma}_k^0 - \textbf{q}_k^0)\cdot \textbf{n}_k}
\end{align} 
\todo{Is this formulation usefull for the NRJ equaiton ? }
where we have defined, 
\begin{align*}
    &\bm{\sigma}_k^\text{eq}
    = \phi_k (
        \bm{\sigma}_k%- n_p \textbf{M}_p
        - \rho_k\kavg{\textbf{u}_k'\textbf{u}_k'})  
    &\textbf{q}_k^\text{eq}
    =\textbf{q}_k^\text{e} +\textbf{q}_k^\text{k}  \\
    &\textbf{q}_k^\text{e}
    = \phi_k\rho_k \kavg{\textbf{u}_k' e_k'} 
    + \phi_k\textbf{q}_k 
    &\textbf{q}_k^\text{k}
    = \phi_k\rho_k \kavg{\textbf{u}_k' k_k} 
    - \phi_k\kavg{\textbf{u}_k' \cdot \bm{\sigma}_k^0}
\end{align*}
Note that the phase averaged energy equation can be further decompose following, 
\begin{align*}
    E_1 = e_1 + k_1 + u_k^2/2
\end{align*}
where $K_1$ is the pseudo-turbulent kinetic energy defined such as, $\phi_1 k_1 = \avg{\chi_1 (u_1')^2/2}$. 
The Macroscopic kinetic energy equation can be obtain by taking the dot product with $\textbf{u}_k$. 
\begin{align}
    \pddt (\phi_k \rho_ku_k^2/2)  
    + \div (
        \phi_k \rho_k\textbf{u}_ku_k^2/2
        - \textbf{u}_k \cdot \bm{\sigma}_k^\text{eq}
    )
    &= 
    - \bm{\sigma}_k^\text{eq} : \grad \textbf{u}_k
    + \phi_k \rho_k \textbf{u}_k\cdot \textbf{g} 
    +  \textbf{u}_k\cdot \avg{\delta_I \bm{\sigma}_k^0 \cdot \textbf{n}_k}\\
    \pddt (\phi_k\rho_kk_k)  
    + \div (
        \phi_k\rho_kk_k\textbf{u}_k
        + \textbf{q}_k^\text{k} 
        )
    &= 
    - \avg{\chi_k\bm{\sigma}_k^0 : \grad \textbf{u}_k^0}
    + \bm{\sigma}_k^\text{eq} : \grad \textbf{u}_k
    + \avg{\delta_I \textbf{u}_k' \cdot \bm{\sigma}_k^0 \cdot \textbf{n}_k}\\
    \pddt (\phi_k\rho_ke_k)  
    + \div (
        \phi_k \rho_ke_k\textbf{u}_k
        +
        \textbf{q}_k^\text{e} 
        )
    &= 
    \avg{\chi_k\bm{\sigma}_k^0 : \grad \textbf{u}_k^0}
    - \avg{\delta_I \textbf{q}_k^0 \cdot \textbf{n}_k} 
\end{align}


make appear the PFP and first order moment.  
Transport equation for $e_{ij}e_{ij}$ regarder pope turbulent flow section (1.4) chapter 6 Aussi le livre de morel + le cluzeau. 

These equations must be completed by the averaged interfacial transport equations, namely,
\begin{align}
    \label{eq:dt_uI}
    - \div \avg{\delta_I \textbf{I}_{||} \gamma}
    = \avg{\delta_I \Jump{\bm{\sigma}^0_k}}\\
    \label{eq:dt_gamma}
    \pddt \avg{\delta_I \gamma}
    - \div \avg{\delta_I \gamma (\textbf{u}_{I}^0 \cdot \textbf{n})\textbf{n} }
    = \avg{\delta_I \Jump{\textbf{u}_k^0 \cdot \bm{\sigma}^0_k + \textbf{q}_k^0}}\\
\end{align}
multiplying \ref{eq:dt_uI} by $\textbf{u}_I$ gives, 
\begin{align}
    - \textbf{u}_I \cdot \avg{\delta_I  \gamma \textbf{n} (\div \textbf{n})}
    = \avg{\delta_I \Jump{\textbf{u}_I \cdot \bm{\sigma}^0_k}}\\
\end{align}
Therefore, taking the difference of this equation wit the previous one yields a fluctuation term thus both equation cannot be isolated  ??? 
\tb{must be isolated before but it works}



\tb{\subsection{Some comments on the two-fluid model}

At the coarse-grained level, the two-fluid model is made of four main equations
(one mass, one momentum and two energy equations) for each of the two phases
(or the two components) of the mixture. To be operational that 8-equations
model must be presented in closed form, i.e. it must involve not more (and
not less) than eight unknowns. We will not insist on that closure issue and
instead will focus on some weaknesses of the two-fluid model concerning the
description of the particulate phase. At first it must be noted that the two-fluid
model does not really distinguish between the two phases, as witnessed by the
”symmetry” k = 1 vs k = 2 of the equations presented above. That symmetry
is not physically tenable and some questions arise : is the particulate phase
able to sustain a heat flux q 2 throughout the suspension without collisions or
permanent contacts between the particles ? Is the stress σ 2 really able to have
a role in the momentum balance of the particles ? One can also wonder why the
two-fluid model has no consideration for the angular momentum of the particles
? For all these reasons we will develop in the following section a kinetic model
specially devoted to the particulate phase of the mixture.
}
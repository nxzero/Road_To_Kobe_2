\subsection{The averaged conservation equations}
In this study, we employ the ensemble average technique to establish the averaged conservation equations. 
This method is just one of several averaging approaches, including the volume average method \citep{jackson1997locally} and time averaging \citep{ishii2010thermo}. 
Despite their differences, all these techniques yield the same set of averaged equations \citep{jackson1997locally,zhang1997momentum}.
In the following we recall some properties of the ensemble average operator. 
Let, $P(\FF)$ be the probability density function that describe the probability of finding the flow in the configuration $\FF$. 
It follows from this definition, that the ensemble average of an arbitrary local property $f^0(\textbf{x},t;\FF)$ defined on the whole space $\Omega$, is,
\begin{equation}
    f(\textbf{x},t)
    = \avg{f^0}(\textbf{x},t)
    =\int f^0(\textbf{x},t;\FF) d\mathscr{P}. 
    \label{eq:avg}
\end{equation}  
Note that we dropped the super script $^0$ on $f$ to indicate that this an averaged quantity. 
The ensemble average operator is assumed to satisfy the following properties \citep{drew1983mathematical}
\begin{align}
    \avg{f^0+h^0} &= f+h, \\ 
    \avg{\avg{f^0}h^0} &= fh, \\
    \avg{\pddt f^0} 
    &= \pddt f, \\ 
    \avg{\grad f^0}
    &= \grad f. 
    \label{eq:avg_properties}
\end{align}
were $f$ and $h$ are two arbitrary Eulerian fields. The first two relations are called the Reynolds' rules, the third one is the Leibniz' rule and the last one, the Gauss' rule \citep{drew1983mathematical}.
Additionally, for any phase quantity defined in $\Omega_k$ we introduce the definition, 
\begin{equation}
    \phi_k f_k (\textbf{x},t) = \avg{\chi_k f_k^0}
    \label{eq:1_avg}
\end{equation}
where, $\phi_k(\textbf{x},t) = \avg{\chi_k}$ is the volume fraction of the phase $k$
and $f_k(\textbf{x},t)$ the average of the field $f_k^0$ conditioned on the presence of the phase $k$ at $\textbf{x}$ and time $t$.
Equally, for interfaces quantities we have 
\begin{equation}
    \phi_I f_I (\textbf{x},t) = \avg{\delta_I f_I^0}
\end{equation}
with $\phi_I = \avg{\delta_I}$ the interfacial indicator function and $f_I$ the average of $f^0(\textbf{x},t)$ conditioned on the presence of an interface at $\textbf{x}$ and time $t$. 
The fluctuation of a phase-averaged quantity around its mean is defined by,
\begin{equation}
    f_k' = f_k^0 - f_k.
    \label{eq:def_fluctu}
\end{equation}
These definitions lead to the following properties, $\avg{\chi_k f'_k} = 0$. 
For example, the product $\avg{\chi_k f^0_kg^0_k}$ can be decomposed as $\avg{\chi_k f_k^0g_k^0}=\phi_k f_kg_k + \avg{\chi_k f'_kg'_k}$. 
This decomposition will play a crucial role in the upcoming section. 


Applying the ensemble average (\ref{eq:avg}) on \ref{eq:dt_chi_k_f_k} and \ref{eq:dt_delta_I_f_I} and considering the properties from \ref{eq:avg_properties} yields the general form of the averaged equations of multiphase flows, namely,
\begin{align}
    \pddt (\phi_k f_k)
    +\div (\phi_k f_k \textbf{u}_k - \mathbf{\Phi}_k^\text{eq})
    &= 
    \phi_k s_k
    + \avg{\delta_I\left[
        \mathbf{\Phi}_k^0
        + f_k^0
        \left(
            \textbf{u}_I^0
            - \textbf{u}_k^0
        \right)
    \right]
    \cdot \textbf{n}_k} ,
    \label{eq:avg_dt_chi_f}\\
    \pddt (\phi_I f_I)
    +\div (\phi_I f_I \textbf{u}_I- \mathbf{\Phi}_{I}^\text{eq})
    &= 
    \phi_I s_I
    - \avg{\delta_I 
    \Jump{
    f_k^0 (\textbf{u}_I^0 - \textbf{u}_k^0)
    + \mathbf{\Phi}_k^0
    } 
     }.
    \label{eq:avg_dt_delta_f}
\end{align}
where $\mathbf{\Phi}_{I}^\text{eq}$ and $\mathbf{\Phi}_{I}^\text{eq}$ are the equivalent non-convective fluxes defined as 
\begin{align*}
    \mathbf{\Phi}_k^\text{eq}
    = \avg{\chi_k f_k' \textbf{u}_k'}
    - \phi_k \bm\Phi_k
    &&
    \mathbf{\Phi}_{I}^\text{eq}
    = \avg{\delta_I f_I' \textbf{u}_I'}
    - \phi_I \bm\Phi_I
\end{align*}
% \begin{align}
%     \pddt \avg{\chi_k f_k^0}
%     +\div \avg{\chi_k f_k^0 \textbf{u}_k^0 - \chi_k \mathbf{\Phi}_k^0}
%     &= 
%     \avg{\chi_k s_k^0}
%     + \avg{\delta_I\left[
%         \mathbf{\Phi}_k^0
%         + f_k^0
%         \left(
%             \textbf{u}_I^0
%             - \textbf{u}_k^0
%         \right)
%     \right]
%     \cdot \textbf{n}_k} ,
%     \label{eq:avg_dt_chi_f}\\
%     \pddt \avg{\delta_If_I^0}
%     +\div \avg{\delta_I f_I^0 \textbf{u}_I^0-\delta_I \mathbf{\Phi}_{I||}^0 }
%     &= 
%     \avg{\delta_Is_I^0} 
%     - \avg{\delta_I 
%     \Jump{
%     f_k^0 (\textbf{u}_I^0 - \textbf{u}_k^0)
%     + \mathbf{\Phi}_k^0
%     } 
%      }.
%     \label{eq:avg_dt_delta_f}
% \end{align}
These equations are to be solved for the averaged field $\phi_k,\phi_I,f_k$ and $f_I$. 

For complete understanding the application to the mass, momentum and energy conservation equation is derived in \ref{ap:two-fluid_model}. 
Especially we demonstrate how to derive the secondary equation of energy. 

It is important to highlight that the two-fluid model fails to adequately distinguish between the two phases, as evidenced by the \textit{symmetry} $k = 1$ and $2$ in the aforementioned equations. This symmetry does not hold physically because the dispersed phase possesses a distinct topological nature compared to the continuous phase. Therefore, in the subsequent section, we will introduce a kinetic model specifically devoted to the dispersed phase. As illustrated below, the equations governing the dispersed phase are more comprehensive as they bear a resemblance to the equations governing a single particle.


\tb{Insiste strongly on the fact that closures terms must be expressed in terms of particles property not phasic average thus lagrangian equations are primordial}
\section{Averaged conservation equations}
\subsection{Ensemble average}
\tb{
\begin{itemize}
\item j'ai simplifie la discussion au debut sur la méthode de moyenne. Vu que l'on ne rentre pas de dans le details j'ai essaye d'etre le plus concis possible.
\item le fait que l'operateur de moyenne commute avec les derives spatiales et temporelles ne semble pas etre triviale. Peut etre faudrait il specifier que c'est parceque les quantitées f et h sont definies dans les deux phases. A discuter ensemble
\item est ce que l'on besoin de rentrer dans la definition de moyenne conditionnee. ne pourrait on pas ecrie $<\chi_k>=\phi_k$ ?
%\item Dans l'enumerotation 
\end{itemize}
}
In this study, we employ the ensemble average technique to establish the averaged conservation equations. This method is just one of several averaging approaches, including the volume average method \citep{jackson1997locally} and time averaging \citep{ishii2010thermo}. Despite their differences, all these techniques yield the same set of averaged equations \citep{jackson1997locally,zhang1997momentum}.
%In this study we use the ensemble average to derive the averaged equations of conservation. The ensemble average method is one averaging method among others, such as the volume average method\citep{jackson1997locally} or the time average \citep{ishii2010thermo}.
%All of these averaging technics leads to the same set of averaged equations \citep{jackson1997locally,zhang1997momentum}. 
%statistical approach to derive the averaged equations of conservation. 
%Specifically, we use the method described in \citet{zhang2021ensemble} which extended the ensemble average definition of \citet{batchelor1972sedimentation}. 
In the following we recall some properties of the ensemble average operator. 
Let, $P(\FF)$ be the probability density function that describe the probability of finding the flow in the configuration $\FF$. %were $\FF = (\lambda_1,\lambda_2,\lambda_3,\ldots)$ is a set of all the parameters describing the initial flow configuration of both phases.
% \footnote{We assume that the flow can be described by a finite number of parameters related to both phase.}. 
%Then, we define $d\mathscr{P} = P(\FF)d\FF$ as the probable number of flows in the incremental region of the particles' phase space $d\FF$ around $\FF$. 
It follows from this definition, that the ensemble average of an arbitrary local property $f^0(\textbf{x},t;\FF)$ defined on the whole space $\Omega$, is,
\begin{equation}
    f(\textbf{x},t)
    = \avg{f^0}(\textbf{x},t)
    =\int f^0(\textbf{x},t;\FF) d\mathscr{P}. 
    \label{eq:avg}
\end{equation}  
Note that we dropped the super script $^0$ on $f$ to indicate that this an averaged quantity. 
% This definition can be applied to Lagrangian properties as well by using the previous formulation, namely $\pavg{q_\alpha}(\textbf{x},t) = \int \delta(\textbf{x} - \textbf{x}_\alpha) q_\alpha(\FF) d\mathscr{P}.$
%It is interesting to mention some mathematical properties of the ensemble average operators. 
%For two arbitrary Eulerian fields $f$ and $h$ we have \citep{drew1983mathematical},
The ensemble average operator is assumed to satisfy the following properties \citep{drew1983mathematical}
\begin{align}
    \avg{f^0+h^0} &= f+h, \\ 
    \avg{\avg{f^0}h^0} &= fh, \\
    \avg{\pddt f^0} 
    &= \pddt f, \\ 
    \avg{\grad f^0}
    &= \grad f. 
    \label{eq:avg_properties}
\end{align}
were $f$ and $h$ are two arbitrary Eulerian fields. The first two relations are called the Reynolds' rules, the third one is the Leibniz' rule and the last one, the Gauss' rule \citep{drew1983mathematical}.
Additionally, for any phase quantity defined in $\Omega_k$ we introduce the definition, 
\begin{equation}
    \phi_k f_k (\textbf{x},t) = \avg{\chi_k f_k^0}
    \label{eq:1_avg}
\end{equation}
where, $\phi_k(\textbf{x},t) = \avg{\chi_k}$ is the volume fraction of the phase $k$
and $f_k(\textbf{x},t)$ the average of the field $f_k^0$ conditioned on the presence of the phase $k$ at $\textbf{x}$ and time $t$.
Equally, for interfaces quantities we have 
\begin{equation}
    \phi_I f_I (\textbf{x},t) = \avg{\delta_I f_I^0}
\end{equation}
with $\phi_I = \avg{\delta_I}$ the interfacial indicator function and $f_I$ the average of $f^0(\textbf{x},t)$ conditioned on the presence of an interface at $\textbf{x}$ and time $t$. 
The fluctuation of a phase-averaged quantity around its mean is defined by,
\begin{equation}
    % q_\alpha' = q_\alpha - q_p
    % \;\;\;\;\;\;\text{and}
    % \;\;\;\;\;\;
    f_k' = f_k^0 - f_k.
    \label{eq:def_fluctu}
\end{equation}
These definitions lead to the following properties, $\avg{\chi_k f'_k} = 0$. 
For example, the product $\avg{\chi_k f^0_kg^0_k}$ can be decomposed as $\avg{\chi_k f_k^0g_k^0}=\phi_k f_kg_k + \avg{\chi_k f'_kg'_k}$. 
This decomposition will play a crucial role in the upcoming section. 

\subsection{The averaged conservation equations}
\tb{
\begin{itemize}
\item il y avait une coquille pour l'equation de transport surfacique ($\div$ au lieu de $\divI$ )
\item il va falloir bcp plus detaille les equations 3.9 et 3.10 en mettant que $\avg{\chi_k f_k^0} = \phi_ k f_k $, ... et en detaillant les closure.
\item ce serait bien de conclure sur le modele moyennee a deux fluides. Je n'ai pas l'impression qu'il y ait de nouveautés par rapport a la litterature, mais encore faut il le dire. A discuter ensemble.
\item pour l'instant j'ai enleve la discussion sur le modele a un fluide
\item j'ai fusionne cette sous section avec la suivante (que j'ai en partie reecrit car c'était un copié / collé du cours de Daniel).
\end{itemize}
}
%Now we proceed to the derivation of the averaged equations %by first introducing the general formulation of the two-fluid averaged equations %then we expose the mass, momentum and energy equation for the volumetric quantities, and finally we present the averaged jump condition. 

%\subsubsection{Generic formulation}
Applying the ensemble average (\ref{eq:avg}) on \ref{eq:dt_chi_k_f_k} and \ref{eq:dt_delta_I_f_I} and considering the properties from \ref{eq:avg_properties} yields the general form of the averaged equations of multiphase flows, namely,
\begin{align}
    \pddt \avg{\chi_k f_k^0}
    +\div \avg{\chi_k f_k^0 \textbf{u}_k^0 - \chi_k \mathbf{\Phi}_k^0}
    &= 
    \avg{\chi_k s_k^0}
    + \avg{\delta_I\left[
        \mathbf{\Phi}_k^0
        + f_k^0
        \left(
            \textbf{u}_I^0
            - \textbf{u}_k^0
        \right)
    \right]
    \cdot \textbf{n}_k} ,
    \label{eq:avg_dt_chi_f}\\
    \pddt \avg{\delta_If_I^0}
    +\divI \avg{\delta_I f_I^0 \textbf{u}_I^0-\delta_I \mathbf{\Phi}_{I||}^0 }
    &= 
    \avg{\delta_Is_I^0} 
    - \avg{\delta_I 
    \sum_k \left[
    f_k^0 (\textbf{u}_I^0 - \textbf{u}_k^0)
    + \mathbf{\Phi}_k^0
    \right] \cdot \textbf{n}_k 
    %\Jump{
    %f_k^0 (\textbf{u}_I^0 - \textbf{u}_k^0)
    %+ \mathbf{\Phi}_k^0
    %} 
     }.
    \label{eq:avg_dt_delta_f}
\end{align}
%Together, \ref{eq:avg_dt_chi_f} for $k=1,2$  and \ref{eq:avg_dt_delta_f}, form the \textit{two-fluid} formulation of averaged multiphase flows problem. 
%One can derive the so-called averaged \textit{single-fluid} formulation by applying the average on \ref{eq:dt_f} or summing the above equations, in both cases it gives, 
%\begin{equation}
%    \pddt f
%    + \div (f \textbf{u} + \avg{f'\textbf{u}'} - \mathbf{\Phi})
%    = 
%    s.
%    \label{eq:avg_dt_f}
%\end{equation}
%With these definitions it must be understood that $f = \sum_k \phi_k f_k + f_I \phi_I$, $\bm\Phi = \sum_k \phi_k \bm\Phi_k + \bm\Phi_I \phi_I$ and so on. 


%\subsection{Some comments on the two-fluid model}

%At the coarse-grained level, the two-fluid model is made of four main equations
%(one mass, one momentum and two energy equations) for each of the two phases
%(or the two components) of the mixture. 
%These equations are completed with the four averaged jump condition.  
%To be operational that 8-equations
%model must be presented in closed form, i.e. it must involve not more (and
%not less) than eight unknowns. 
%We will not insist on that closure issue and
%instead will focus on some weaknesses of the two-fluid model concerning the
%description of the dispersed phase. 
It is important to highlight that the two-fluid model fails to adequately distinguish between the two phases, as evidenced by the \textit{symmetry} $k = 1$ and $2$ in the aforementioned equations. This symmetry does not hold physically because the dispersed phase possesses a distinct topological nature compared to the continuous phase. Therefore, in the subsequent section, we will introduce a kinetic model specifically devoted to the dispersed phase. As illustrated below, the equations governing the dispersed phase are more comprehensive as they bear a resemblance to the equations governing a single particle.

%It must be noted that the two-fluid model does not really distinguish the two phases, as witnessed by the \textit{symmetry} $k = 1$ and $2$ of the equations presented above. That symmetry is not physically tenable as the dispersed phase have a different topological nature from the conitnuous phase. For this reason we will develop in the following section a kinetic model specially devoted to the dispersed phase. As demontrated below the equations governing the dispersed phase are more comprehensive because they bear some resemblance with the equations governing a single particle.

%and some questions arise. The equations governing the disperse phase are more comprehensive because
%they bear some resemblance with the equations governing a single particle

%In particular if we consider the averaged momentum equations $f_k^0 = \rho_k \mathbf{u}_k^0$, we may observe that 
%: is the particulate phase
%able to sustain a heat flux $\textbf{q}_2$ throughout the suspension without collisions or permanent contacts between the particles ? Is the stress $\bm{\sigma}_2$ and surface tension stress $\bm{\sigma}_I$ really able to have a role in the momentum balance of the particles ? 
%One can also wonder why the two-fluid model has no consideration for the angular momentum of the particles ? 



\section{Arbitrary order moments equation}
\label{ap:Moments_equations}

Let's define the arbitrary moment of the tensor $f$, by, 
\begin{equation*}
    Q_{i_1\ldots i_n}
    = \int_{V_\alpha} 
    \pri{1}{n} f dV
\end{equation*}
Then,
\begin{multline*}
    \ddt Q_{i_1\ldots i_n}
    =\int_{V_\alpha} \left[ \partial_t \left(\pri{1}{n}f\right) 
    + \partial_k \left(u_k \pri{1}{n}f\right) \right]dV\\
    +\int_{S_\alpha} \pri{1}{n} f \left(u^I_k - u_k\right) n_k dS.
\end{multline*}
Using the product rule on the derivatives yields, 
\begin{multline*}
    \ddt Q_{i_1\ldots i_n}
    =\int_{V_\alpha} f \left[ \partial_t \left(\pri{1}{n}\right) 
    + u_k \partial_k \left( \pri{1}{n}\right) \right]dV\\
    +\int_{V_\alpha} \pri{1}{n} \left[ \partial_t \left(f\right) 
    +  \partial_k \left(u_k f \right) \right]dV\\
    +\int_{S_\alpha} \pri{1}{n} f \left(u^I_k - u_k\right) n_k dS.
\end{multline*}
From similar arguments as before one can easily show that, 
\begin{multline*}
    \ddt Q_{i_1\ldots i_n}
    = \sum_{e=1}^{n} \int_{V_\alpha} f  \prod^{n}_{\substack{ m=1 \\   m \neq e}} r_{i_m} w_{i_e}dV
    +\int_{V_\alpha} \pri{1}{n} \nablabh\cdot\mathbf{\Phi} dV\\
    + \int_{V_\alpha} \pri{1}{n} \textbf{S} dV
    +\int_{S_\alpha} \pri{1}{n} f \left(u^I_k - u_k\right) n_k dS.
\end{multline*}
The second term can be reformulated such as,
\begin{align*}
    \int_{V_\alpha} \pri{1}{n} \nablabh\cdot\mathbf{\Phi} dV
    &= \int_{S_\alpha} \nablabh \cdot \left(\pri{1}{n} \mathbf{\Phi} \right)dV
    - \int_{V_\alpha} \mathbf{\Phi} \cdot \nablabh \left(\pri{1}{n} \right)dV\\
    &= \int_{S_\alpha} \pri{1}{n} \mathbf{\Phi} \cdot \textbf{n}dS
    -\sum_{e=1}^{n} \int_{V_\alpha} \mathbf{\Phi}  \prod^{n}_{\substack{ m=1 \\m \neq e}} r_{i_m}  dV
\end{align*}
Including this relation into the former equation yields, 
\begin{multline*}
    \ddt Q_{i_1\ldots i_n}
    = \sum_{e=1}^{n} \int_{V_\alpha} \prod^{n}_{\substack{ m=1 \\   m \neq e}} r_{i_m} (w_{i_e}f  - \mathbf{\Phi}_{i_e})dV
    +\int_{S_\alpha} \pri{1}{n} \mathbf{\Phi} \cdot \textbf{n}dS\\
    + \int_{V_\alpha} \pri{1}{n} \textbf{S} dV
    +\int_{S_\alpha} \pri{1}{n} f \left(u^I_k - u_k\right) n_k dS.
\end{multline*}
Then it is possible from this equation to carry out a particular average but also to get the local scale equations. 
Indeed, if we consider $V_\alpha$ as being a fixed control volume the above equality can be rewritten such as, 
\begin{multline*}
    \pddt \pavg{Q_{i_1\ldots i_n}^\alpha}
    + \partial_k  \pavg{Q_{i_1\ldots i_n}^\alpha u_k^\alpha}
    = \sum_{e=1}^{n} \pavg{\int_{V_\alpha} \prod^{n}_{\substack{ m=1 \\   m \neq e}} r_{i_m} (w_{i_e}f  - \mathbf{\Phi}_{i_e})dV}\\
    + \pavg{\int_{V_\alpha} \pri{1}{n} \textbf{S} dV}
    + \pavg{\int_{\Omega_\alpha} \pri{1}{n} \left[
            \mathbf{\Phi}_k
            + f_k
            \left(
                \textbf{u}_I
                - \textbf{u}_k
            \right)
        \right]
        \cdot \textbf{n}_kd\Omega}
\end{multline*}
% \begin{multline}
%     \pddt \left(\pri{1}{n}f\right)
%     + \nablabh \cdot \left(\pri{1}{n}f \textbf{u}\right)
%     = n  \pri{1}{n-1}  (w_{i_n}f  - \Phi)\\
%     + \pri{1}{n} \textbf{S} 
%     + \nablabh \cdot \left( \pri{1}{n} \mathbf{\Phi} \right)
%     \label{ap:eq:dt_Q_alpha_n}
% \end{multline}$
\subsection*{Arbitrary order equivalence}
\begin{align}    
    0 &= 
    - \pddt \pavg{q_\alpha} +  \nablab \cdot  \partial_t\pavg{\textbf{Q}_\alpha} \ldots \nonumber\\
    &+ \nablab \cdot \pavg{\int_{\Omega_\alpha}\left(\mathbf{\Phi}_k - f_k \textbf{u}_k \right)d\Omega}
    -\nablab\nablab : \pavg{\int_{\Omega_\alpha}\textbf{r}\left(\mathbf{\Phi}_k - f_k \textbf{u}_k \right)d\Omega}
    \ldots\nonumber\\
    &+ \pavg{ \int_{\Omega_\alpha} \textbf{S}_k d\Omega}
    - \nablab \cdot \pavg{ \int_{\Omega_\alpha} \textbf{r}\textbf{S}_k d\Omega}
    \ldots\nonumber\\
    &+ \pavg{\int_{\Sigma_\alpha}\left[
        \mathbf{\Phi}_k
        + f_k
        \left(
            \textbf{u}_I
            - \textbf{u}_k
        \right)
    \right]
    \cdot \textbf{n}_kd\Sigma} \ldots\nonumber\\
    &-  \nablab \cdot \pavg{\int_{\Sigma_\alpha} \textbf{r}\left[
        \mathbf{\Phi}_k
        + f_k
        \left(
            \textbf{u}_I
            - \textbf{u}_k
        \right)
    \right]
    \cdot \textbf{n}_kd\Sigma} \ldots
\end{align}
We assume an infinite summation on $l$
\begin{equation}
        \pddt \avg{\chi_k f_k}
        = \nablabh \cdot \avg{\chi_k \mathbf{\Phi}_k - \chi_k f_k \textbf{u}_k}
        + \avg{\chi_k \textbf{S}_k}
        + \avg{\delta_I\left[
            \mathbf{\Phi}_k
            + f_k
            \left(
                \textbf{u}_I
                - \textbf{u}_k
            \right)
        \right]
        \cdot \textbf{n}_k} 
\end{equation}
expangin each term gives,
\begin{align}
        0 &=
        - \pddt \expo{f_k} \\
        &+\nablabh \cdot \expo{(\mathbf{\Phi}_k - f_k \textbf{u}_k)}\\
        &+ \expo{ \textbf{S}_k}\\
        &+ \expoS{\left[
            \mathbf{\Phi}_k
            + f_k
            \left(
                \textbf{u}_I
                - \textbf{u}_k
            \right)
        \right]
        \cdot \textbf{n}_k} \\
\end{align}
considering, 
\begin{multline}
    \expo{f_k \textbf{u}_k}\\
    =     \expoU{f_k }\\
    +     \expo{f_k \textbf{w}_k}
\end{multline}
Then by factorizing the above equation we obtain : 
\begin{align}
    & \pddt \expo{f_k} \\
    &+ \nablabh \cdot \expoU{f_k}\\
    &= \nablabh \cdot \expo{(\mathbf{\Phi}_k - f_k \textbf{w}_k)}\\
    &+ \expo{ \textbf{S}_k}\\
    &+ \expoS{\left[
        \mathbf{\Phi}_k
        + f_k
        \left(
            \textbf{u}_I
            - \textbf{u}_k
        \right)
    \right]
    \cdot \textbf{n}_k} \\
\end{align}
where we can factorize such as, 
\begin{multline}
    0 = \frac{(-1)^n}{n!}
    \partialp{1}{n}
    \left[
        - \partial_t
        \pavg{\int_{\Omega_\alpha} \pri{1}{n}f_k d\Omega}
        - \nablab \cdot \pavg{\textbf{u}_\alpha \int_{\Omega_\alpha} \pri{1}{n}f_k d\Omega}
    \right.\\\left.
        +\pavg{\int_{\Omega_\alpha} \pri{1}{n-1} (\mathbf{\Phi}_k - f_k \textbf{w}_k) d\Omega}
        +\pavg{\int_{\Omega_\alpha} \pri{1}{n} \textbf{S}_k d\Omega}
        \right.\\\left.
        +\pavg{\int_{\Omega_\alpha} \pri{1}{n} \left[
            \mathbf{\Phi}_k
            + f_k
            \left(
                \textbf{u}_I
                - \textbf{u}_k
            \right)
        \right]
        \cdot \textbf{n}_kd\Omega}
    \right]
\end{multline}
where the third term is undefined for $n=0$ meaning that it must not be conisdered
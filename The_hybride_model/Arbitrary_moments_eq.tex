
\section{Arbitrary order moments equations}
\label{ap:Moments_equations}
In this appendix, we extend the Lagrangian conservation laws to an arbitrary order moment equation. 
Let us first define the arbitrary moment of the Eulerian field $f_d^0$, in index notation it reads, 
\begin{equation*}
    (\textbf{q}_\alpha^{(n)})_{i_1\ldots i_n}
    = \intO{
    \pri{1}{n} f_d^0 
    }, 
\end{equation*}
where we recall that $r_{i_m} = x_{i_m} - (\textbf{x}_\alpha)_{i_m}$. 
Applying the Reynolds transport theorem \eqref{eq:reynolds_transport} we show that :
\begin{multline}
    \ddt {(\textbf{q}_\alpha^{(n)})_{i_1\ldots i_n}}
    =\intO{
        \left[ \partial_t \left(f_d^0 \pri{1}{n}\right) 
    + \div \left(\textbf{u}_d^0 f_d^0 \pri{1}{n}\right) \right]
    }\\
    +\intS{ \pri{1}{n} f_d^0 \left(\textbf{u}_\Gamma^0 - \textbf{u}_d^0\right)\cdot \textbf{n}_d }. 
\end{multline}
Applying the product rule to the terms inside the first integral results in the following expression: 
\begin{multline}
    \ddt {(\textbf{q}_\alpha^{(n)})_{i_1\ldots i_n}}
    =\intO{ 
        f_d^0 \left[ \partial_t \left(\pri{1}{n}\right) 
        + (\textbf{u}_d^0\cdot \grad) \left( \pri{1}{n}\right) \right]
    }\\
    +\intO{ 
        \pri{1}{n} 
        \left[ \partial_t f_d^0
    +  \div \left(\textbf{u}_d^0 f_d^0 \right) \right]
    }
    +\intS{ \pri{1}{n} f_d^0 \left(\textbf{u}_\Gamma^0 - \textbf{u}_d^0\right)\cdot \textbf{n}_d }. 
\end{multline}
By applying the product rule, once again, to the terms within the first integral on, noting that $\pddt \textbf{r}_{i_e} = (\textbf{w}_d^0)_{i_e}$, and utilizing the conservation equation \ref{eq:dt_f_k} on the second integral, leads us to the relation: 
\begin{multline}
    \ddt {(\textbf{q}_\alpha^{(n)})_{i_1\ldots i_n}}
    = \sum_{e=1}^{n} \intO{ 
        f_d^0 \prod^{n}_{\substack{ m=1 \\   m \neq e}} r_{i_m} (\textbf{w}_d^0)_{i_e}
        }
    +\intO{\pri{1}{n} (\div\bm{\Phi}_d^0)}\\
    + \intO{ \pri{1}{n} s_d^0}
    +\intS{ \pri{1}{n} f_d^0 \left(\textbf{u}_\Gamma^0 - \textbf{u}_d^0\right)\cdot \textbf{n}_d }.
\end{multline}
The second term on the right-hand side of this equation can be reformulated as,
\begin{align*}
    \intO{ \pri{1}{n} (\div\bm\Phi_d^0) }
    &= \intO{ \div \left(\pri{1}{n} \bm\Phi_d^0 \right)}
    - \intO{ \bm\Phi_d^0 \cdot \grad \left(\pri{1}{n} \right)}\\
    &= \intS{ \pri{1}{n} (\bm\Phi_d^0 \cdot \textbf{n}_d)}
    -\sum_{e=1}^{n} 
    \intO{ (\bm\Phi_d^0)_{i_e}  \prod^{n}_{\substack{ m=1 \\m \neq e}} r_{i_m}  }
\end{align*}
Including this relation into the former equation yields, 
\begin{multline}
    \ddt {(\textbf{q}_\alpha^{(n)})_{i_1\ldots i_n}}
    = \sum_{e=1}^{n} 
    \intO{
        \prod^{n}_{\substack{ m=1 \\m \neq e}} r_{i_m} (f_d^0 \textbf{w}_d^0  - \bm\Phi_d^0)_{i_e}
    }
    % +\intS{ \pri{1}{n} (\bm\Phi_d^0 \cdot \textbf{n}_d)}\\
    + \intO{ \pri{1}{n} s_d^0 }\\
    +\intS{ \pri{1}{n} [\bm\Phi_d^0 + f_d^0 \left(\textbf{u}_\Gamma^0 - \textbf{u}_d^0\right)]\cdot \textbf{n}_d }.
    \label{eq:dt_q_n}
\end{multline}
This represents the final form of the Lagrangian conservation equation for the $n^{th}$ order moment of the quantity $f_d^0$ within the particle. 
When $f_d^0$ is a scalar quantity, this expression indicates that $\ddt \textbf{q}_\alpha^{(n)}$ is entirely symmetric since it involves only sums of products of the $r_{i_n}$ times a scalar quantity. 
Regarding the seemingly non-symmetric terms involving $\textbf{w}_d^0$ and $\bm\Phi_d^0$, it is important to note that, due to the summation operator's presence, these terms also become symmetric. 
% Note that if $f_d^0$ is a vector quantity, let say of index $k$ we obtain, 
% \begin{multline}
%     \ddt {(\textbf{q}_\alpha^{(n)})_{k i_1\ldots i_n}}
%     = \sum_{e=1}^{n} 
%     \intO{
%         \prod^{n}_{\substack{ m=1 \\m \neq e}} r_{i_m} [\textbf{f}_d^0\textbf{w}_d^0  - \bm\Phi_d^0]_{ki_e}
%     }
%     + \intO{ \pri{1}{n} (\textbf{s}_d^0)_k }\\
%     +\intS{ \pri{1}{n} ([\bm\Phi_d^0 + \textbf{f}_d^0 \left(\textbf{u}_\Gamma^0 - \textbf{u}_d^0\right)]\cdot \textbf{n}_d)_k }.
% \end{multline}
% As an example this relation can be directly applied with $f_d^0 = \rho_d \textbf{u}_d^0$ to obtain the moment of momentum conservation. 


Regarding the surface property conservation equations, the derivation is similar and will not be displayed here. 
The result yields, 
\begin{multline}
    \ddt {(\textbf{q}_{\alpha\Gamma}^{(n)})_{i_1\ldots i_n}}
    = \sum_{e=1}^{n} 
    \intS{
        \prod^{n}_{\substack{ m=1 \\m \neq e}} r_{i_m} (f_\Gamma^0\textbf{w}_\Gamma^0 - \bm\Phi_{||\Gamma}^0)_{i_e}
    }
    + \intS{ \pri{1}{n} (\textbf{s}_\Gamma^0)_k }
    \\
    +\intS{ \pri{1}{n} \Jump{\bm\Phi_k^0 + f_k^0 \left(\textbf{u}_\Gamma^0 - \textbf{u}_k^0\right)\cdot \textbf{n}_d}}.
    \label{eq:dt_Qgamma_n}
\end{multline}
where we have defined, 
\begin{equation*}
    (\textbf{q}_{\alpha\Gamma}^{(n)})_{i_1\ldots i_n}
    = \intS{
    \pri{1}{n} f_I^0 
    }. 
\end{equation*}

Summing \ref{eq:dt_q_n} and \ref{eq:dt_Qgamma_n} one obtains the equation for the total $n^{th}$ order moment, $\textbf{Q}_{\alpha}^{(n)} = \textbf{q}_{\alpha\Gamma}^{(n)}+\textbf{q}_{\alpha}^{(n)}$, namely, 
\begin{multline}
    \ddt {(\textbf{Q}_{\alpha}^{(n)})_{i_1\ldots i_n}}
    = 
    \sum_{e=1}^{n} 
    \intO{
        \prod^{n}_{\substack{ m=1 \\m \neq e}} r_{i_m} (f_d^0\textbf{w}_d^0  - \bm\Phi_d^0)_{i_e}
    }
    + \intO{ \pri{1}{n} (\textbf{s}_d^0)_k }\\
    +     
    \sum_{e=1}^{n} 
    \intS{
        \prod^{n}_{\substack{ m=1 \\m \neq e}} r_{i_m} (f_\Gamma^0\textbf{w}_\Gamma^0 - \bm\Phi_{||\Gamma}^0)_{i_e}
    }
    + \intS{ \pri{1}{n} (\textbf{s}_\Gamma^0)_k }
    \\
    +\intS{ \pri{1}{n} ([\bm\Phi_f^0 + \textbf{f}_f^0 \left(\textbf{u}_\Gamma^0 - \textbf{u}_f^0\right)]\cdot \textbf{n}_d)_k }. 
    \label{eq:dt_Q_n}
\end{multline}


% The symmetric part of $(\textbf{q}_\alpha^{(n)})_{i_0 i_1\ldots i_n}$ is, 
% \begin{equation*}
%     (\textbf{q}_\alpha^{(n)})_{(i_0 i_1\ldots i_p \ldots i_n )}
% = \frac{1}{n+1}
% \sum_{p=0}^{n} (\textbf{q}_\alpha^{(n)})_{i_p (i_1\ldots i_0\ldots i_n)}
% \end{equation*}
% where the parenthesis indicates the symmetric index, and it must be understood that this is permutation of the indices.  
% Therefore, the fully symmetric part of the preceding momentum balance can be obtained by summing every permutation of the index $k$ with all other index and dividing by $n$, namely,
% \begin{multline}
%     \ddt {(\textbf{q}_\alpha^{(n)})_{(i_0 i_1\ldots i_n) }}
%     = \frac{1}{n+1}
%     \sum_{p=0}^{n}
%     \sum_{\substack{ e=0 \\   e \neq i_p}}^{n} \int_{\Omega_\alpha} 
%     \prod^{n}_{\substack{ m=0 \\   m \neq e}} r_{i_m} (w_{i_e}f_{i_p}  - \bm\Phi_{i_p i_e})d\Omega\\
%     +\frac{1}{n+1}
%     \sum_{p=0}^{n}
%     \int_{\Sigma_\alpha} \prod^{n}_{\substack{ m=0 \\   m \neq i_p}} r_{i_m}
%     (\bm\Phi \cdot \textbf{n})_{i_p}d\Sigma
%     + \int_{\Omega_\alpha} 
%     \prod^{n}_{\substack{ m=0 \\   m \neq i_p}} r_{i_m}
%     \textbf{S}_{i_p} d\Omega
% \end{multline}
% It appears that this equation is the fully symmetric parts of the moments equations. 
% The skew symmetric parts will be written, 
% \begin{multline}
%     \frac{d}{dt} (
%     (\textbf{q}_\alpha^{(n)})_{i_0 i_1\ldots i_n} 
%     - (\textbf{q}_\alpha^{(n)})_{(i_0 i_1\ldots i_n) }
%     )
%     = 
%     \sum_{e=1}^{n} \int_{\Omega_\alpha} \prod^{n}_{\substack{ m=1 \\   m \neq e}} r_{i_m} (w_{i_e}f_{i_0}  - \bm\Phi_{i_0 i_e})d\Omega
%     -
%     \frac{1}{n+1}
%     \sum_{p=0}^{n}
%     \sum_{\substack{ e=0 \\   e \neq i_p}}^{n} \int_{\Omega_\alpha} 
%     \prod^{n}_{\substack{ m=0 \\   m \neq e}} r_{i_m} (w_{i_e}f_{i_p}  - \bm\Phi_{i_p i_e})d\Omega\\
%     +\int_{\Sigma_\alpha} \pri{1}{n} (\bm\Phi \cdot \textbf{n})_{i_0}d\Sigma
%     -
%     \frac{1}{n+1}
%     \sum_{p=0}^{n}
%     \int_{\Sigma_\alpha} \prod^{n}_{\substack{ m=0 \\   m \neq i_p}} r_{i_m}
%     (\bm\Phi \cdot \textbf{n})_{i_p}d\Sigma
%     + \int_{\Omega_\alpha} \pri{1}{n} \textbf{S}_{i_0} d\Omega
%     -
%     \int_{\Omega_\alpha} 
%     \prod^{n}_{\substack{ m=0 \\   m \neq i_p}} r_{i_m}
%     \textbf{S}_{i_p} d\Omega
% \end{multline}
% It is known that the non-convective fluxes vanish at the order one of this equation. 
% We would like to make appear this property explicitly. 
% \begin{multline*}
%     \sum_{e=1}^{n} \int_{\Omega_\alpha} \prod^{n}_{\substack{ m=1 \\   m \neq e}} r_{i_m} \bm\Phi_{i_0 i_e} d\Omega
%     -
%     \frac{1}{n+1}
%     \sum_{p=0}^{n}
%     \sum_{\substack{ e=0 \\   e \neq i_p}}^{n} \int_{\Omega_\alpha} 
%     \prod^{n}_{\substack{ m=0 \\   m \neq e}} r_{i_m}  \bm\Phi_{i_p i_e}d\Omega\\
%     =
%     \sum_{e=1}^{n} \int_{\Omega_\alpha} \prod^{n}_{\substack{ m=1 \\   m \neq e}} r_{i_m} \bm\Phi_{i_0 i_e}d\Omega
%     -
%     \frac{1}{n+1}
%     \sum_{p=0}^{n}
%     \sum_{\substack{ e=0 \\   e \neq i_p}}^{n} \int_{\Omega_\alpha} 
%     \prod^{n}_{\substack{ m=0 \\   m \neq e}} r_{i_m}  \bm\Phi_{i_p i_e}d\Omega\\
%     =
%     \frac{- 1}{n+1}
%     \sum_{p=1}^{n}
%     \sum_{\substack{ e=1 \\   e \neq i_p}}^{n} \int_{\Omega_\alpha} 
%     \prod^{n}_{\substack{ m=1 \\   m \neq e}} r_{i_m}  \bm\Phi_{i_p i_e}d\Omega
% \end{multline*}
% Which makes a non-vanishing parts for the integral of the stress. 
% Instead, we rather derive the moments' equation antisymmetric in the indices $i_e$ $i_0$ by subtracting the permuted equation
% \begin{multline}
%     \ddt{ (\textbf{q}_\alpha^{(n)})_{i_0 i_1\ldots i_n }}
%     = \sum_{e=1}^{n} \int_{\Omega_\alpha} \prod^{n}_{\substack{ m=1 \\   m \neq e}} r_{i_m} (w_{i_e}f_{i_0}  - \bm\Phi_{i_0 i_e})d\Omega
%     +\int_{\Sigma_\alpha} \pri{1}{n} (\bm\Phi \cdot \textbf{n})_{i_0}d\Sigma
%     + \int_{\Omega_\alpha} \pri{1}{n} \textbf{S}_{i_0} d\Omega
% \end{multline}

% As an example we give the two first order moments for particles without mass transfer: 
% If $n=1$ : 
% \begin{equation}
%     \ddt{ (\textbf{q}_\alpha^{(n)})_{i_1}}
%     = \int_{\Omega_\alpha} (w_{i_1}f  - \bm\Phi_{i_1})d\Omega
%     +\int_{\Sigma_\alpha} r_{i_1}\bm\Phi \cdot \textbf{n}d\Sigma
%     + \int_{\Omega_\alpha}r_{i_1} \textbf{S} d\Omega
% \end{equation}
% and for $n=2$ : 
% \begin{multline}
%     \label{eq:moment_n2}
%     \ddt {(\textbf{q}_\alpha^{(n)})_{i_1 i_2}}
%     = 
%     \int_{\Omega_\alpha} r_{i_2} (w_{i_1}f  - \bm\Phi_{i_1})d\Omega
%     +\int_{\Omega_\alpha} r_{i_1} (w_{i_2}f  - \bm\Phi_{i_2})d\Omega
%     +\int_{\Sigma_\alpha}  r_{i_1}r_{i_2} \bm\Phi \cdot \textbf{n}d\Sigma\\
%     + \int_{\Omega_\alpha} r_{i_1}r_{i_2}  \textbf{S} d\Omega
% \end{multline}
% For the momentum equation we obtain : 
% \begin{equation}
%     \ddt{ \mathcal{P}_{ij}}
%     = \int_{\Omega_\alpha} (w_{i}w_j \rho_2  - \bm{\sigma}_{ij})d\Omega
%     +\int_{\Sigma_\alpha} r_{i} \sigma_{jk} \cdot n_k d\Sigma
%     + \int_{\Omega_\alpha}r_{i} \rho_d g_j d\Omega
% \end{equation}
% \begin{multline}
%     \ddt{ \mathcal{P}_{i j k}}
%     = 
%     \int_{\Omega_\alpha} r_{j} (w_{i} w_k\rho_2 - \sigma_{ik})d\Omega
%     +\int_{\Omega_\alpha} r_{i} (w_{j} w_k\rho_2 - \sigma_{jk})d\Omega
%     +\int_{\Sigma_\alpha}  r_{i}r_{j} \sigma_{kl} n_l d\Sigma\\
%     + \int_{\Omega_\alpha} r_{i}r_{j}  \rho_2 g_k d\Omega
%     \label{eq:second_momoent_of_momentum}
% \end{multline}

% \section{Averaged moments equations}
By averaging \ref{eq:dt_Q_n}, we obtain the particle averaged equation for $\pavg{\textbf{Q}_\alpha^{(n)}} = n_p\textbf{Q}_p^{(n)} $, it yields,
% it is possible from this equation to carry out a particle-average, which directly yield the $n^{th}$ order moment equation : 
\begin{multline}
    \pddt \pavg{(\textbf{Q}_\alpha^{(n)})_{i_1\ldots i_n}^\alpha}
    + \div  \pavg{\textbf{u}_\alpha (\textbf{Q}_\alpha^{(n)})_{i_1\ldots i_n}^\alpha}
    = \sum_{e=1}^{n} 
    \pOavg{
        \prod^{n}_{\substack{ m=1 \\m \neq e}} r_{i_m} (f_d^0\textbf{w}_d^0  - \bm\Phi_d^0)_{i_e}
    }\\
    + \pOavg{ \pri{1}{n} (\textbf{s}_d^0)_k }
    +     
    \sum_{e=1}^{n} 
    \pSavg{
        \prod^{n}_{\substack{ m=1 \\m \neq e}} r_{i_m} (f_\Gamma^0\textbf{w}_\Gamma^0 - \bm\Phi_{||\Gamma}^0)_{i_e}
    }\\
    + \pSavg{ \pri{1}{n} (\textbf{s}_\Gamma^0)_k }
    +\pSavg{ \pri{1}{n} ([\bm\Phi_f^0 + \textbf{f}_f^0 \left(\textbf{u}_\Gamma^0 - \textbf{u}_f^0\right)]\cdot \textbf{n}_d)_k }. 
\end{multline}



\section{Arbitrary order moments equation}
\label{ap:Moments_equations}

Let's define the arbitrary moment of the tensor $f$, by, 
\begin{equation*}
    Q_{i_1\ldots i_n}
    = \int_{V_\alpha} 
    \pri{1}{n} f dV
\end{equation*}
Then,
\begin{multline*}
    \ddt Q_{i_1\ldots i_n}
    =\int_{V_\alpha} \left[ \partial_t \left(\pri{1}{n}f\right) 
    + \partial_k \left(u_k \pri{1}{n}f\right) \right]dV\\
    +\int_{S_\alpha} \pri{1}{n} f \left(u^I_k - u_k\right) n_k dS.
\end{multline*}
Using the product rule on the derivatives yields, 
\begin{multline*}
    \ddt Q_{i_1\ldots i_n}
    =\int_{V_\alpha} f \left[ \partial_t \left(\pri{1}{n}\right) 
    + u_k \partial_k \left( \pri{1}{n}\right) \right]dV\\
    +\int_{V_\alpha} \pri{1}{n} \left[ \partial_t \left(f\right) 
    +  \partial_k \left(u_k f \right) \right]dV\\
    +\int_{S_\alpha} \pri{1}{n} f \left(u^I_k - u_k\right) n_k dS.
\end{multline*}
From similar arguments as before one can easily show that, 
\begin{multline*}
    \ddt Q_{i_1\ldots i_n}
    = \sum_{e=1}^{n} \int_{V_\alpha} f  \prod^{n}_{\substack{ m=1 \\   m \neq e}} r_{i_m} w_{i_e}dV
    +\int_{V_\alpha} \pri{1}{n} \nablabh\cdot\mathbf{\Phi} dV\\
    + \int_{V_\alpha} \pri{1}{n} \textbf{S} dV
    +\int_{S_\alpha} \pri{1}{n} f \left(u^I_k - u_k\right) n_k dS.
\end{multline*}
The second term can be reformulated such as,
\begin{align*}
    \int_{V_\alpha} \pri{1}{n} \nablabh\cdot\mathbf{\Phi} dV
    &= \int_{S_\alpha} \nablabh \cdot \left(\pri{1}{n} \mathbf{\Phi} \right)dV
    - \int_{V_\alpha} \mathbf{\Phi} \cdot \nablabh \left(\pri{1}{n} \right)dV\\
    &= \int_{S_\alpha} \pri{1}{n} \mathbf{\Phi} \cdot \textbf{n}dS
    -\sum_{e=1}^{n} \int_{V_\alpha} \mathbf{\Phi}  \prod^{n}_{\substack{ m=1 \\m \neq e}} r_{i_m}  dV
\end{align*}
Including this relation into the former equation yields, 
\begin{multline*}
    \ddt Q_{i_1\ldots i_n}
    = \sum_{e=1}^{n} \int_{V_\alpha} \prod^{n}_{\substack{ m=1 \\   m \neq e}} r_{i_m} (w_{i_e}f  - \Phi)dV
    +\int_{S_\alpha} \pri{1}{n} \mathbf{\Phi} \cdot \textbf{n}dS\\
    + \int_{V_\alpha} \pri{1}{n} \textbf{S} dV
    +\int_{S_\alpha} \pri{1}{n} f \left(u^I_k - u_k\right) n_k dS.
\end{multline*}
Then it is possible from this equation to carry out a particular average but also to get the local scale equations. 
Indeed, if we consider $V_\alpha$ as being a fixed control volume the above equality can be rewritten such as, 
\begin{multline}
    \pddt \left(\pri{1}{n}f\right)
    + \nablabh \cdot \left(\pri{1}{n}f \textbf{u}\right)
    = n  \pri{1}{n-1}  (w_{i_n}f  - \Phi)\\
    + \pri{1}{n} \textbf{S} 
    + \nablabh \cdot \left( \pri{1}{n} \mathbf{\Phi} \right)
    \label{ap:eq:dt_Q_alpha_n}
\end{multline}


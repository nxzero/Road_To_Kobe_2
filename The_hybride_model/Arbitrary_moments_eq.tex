
\section{Arbitrary order moments equation}
\label{ap:Moments_equations}
In this appendix we extend the Lagrangian conservation laws to an arbitrary order moment equation. 
Let's first define the arbitrary moment of the Eulerian field $f_d^0$, in indices notation as, 
\begin{equation*}
    [\textbf{q}_\alpha^{(n)}]_{i_1\ldots i_n}
    = \intO{
    \pri{1}{n} f_d^0 
    }
\end{equation*}
we recall here that $r_{i_m} = x_{i_m} - (\textbf{x}_\alpha)_{i_m}$. 
Then by using the Reynolds transport theorem \ref{eq:reynolds_transport} we can equally show that :
\begin{equation}
    \ddt {[\textbf{q}_\alpha^{(n)}]_{i_1\ldots i_n}}
    =\intO{
        \left[ \partial_t \left(\pri{1}{n}f_d^0\right) 
    + \partial_k \left(u_k \pri{1}{n}f_d^0\right) \right]
    }
    +\intS{ \pri{1}{n} f_d^0 \left(\textbf{u}_\Gamma^0 - \textbf{u}_d^0\right)\cdot \textbf{n}_d }. 
\end{equation}
Using the product rule on the first integral derivatives yields the expression, 
\begin{multline*}
    \ddt {[\textbf{q}_\alpha^{(n)}]_{i_1\ldots i_n}}
    =\intO{ 
        f_d^0 \left[ \partial_t \left(\pri{1}{n}\right) 
        + (\textbf{u}_d^0\cdot \grad) \left( \pri{1}{n}\right) \right]
    }\\
    +\intO{ 
        \pri{1}{n} 
        \left[ \partial_t f_d^0
    +  \div \left(\textbf{u}_d^0 f_d^0 \right) \right]
    }
    +\intS{ \pri{1}{n} f_d^0 \left(\textbf{u}_\Gamma^0 - \textbf{u}_d^0\right)\cdot \textbf{n}_d }. 
\end{multline*}
Using the product rule of derivatives on the first integral, noticing that $\pddt r_{i_e} = w_{i_e}$, and using the conservation equation \ref{eq:dt_f_k} on the second integral leads us to, 
\begin{multline*}
    \ddt {[\textbf{q}_\alpha^{(n)}]_{i_1\ldots i_n}}
    = \sum_{e=1}^{n} \intO{ 
        f_d^0 \prod^{n}_{\substack{ m=1 \\   m \neq e}} r_{i_m} (\textbf{w}_d^0)_{i_e}
        }
    +\intO{\pri{1}{n} \div\bm{\Phi}_d^0}
    + \intO{ \pri{1}{n} s_d^0}\\
    +\intS{ \pri{1}{n} f_d^0 \left(\textbf{u}_\Gamma^0 - \textbf{u}_d^0\right)\cdot \textbf{n}_d }.
\end{multline*}
The second term of this equation can be reformulated following,
\begin{align*}
    \intO{ \pri{1}{n} \div\bm\Phi_d^0 }
    &= \intO{ \div \left(\pri{1}{n} \bm\Phi_d^0 \right)}
    - \intO{ \bm\Phi_d^0 \cdot \grad \left(\pri{1}{n} \right)}\\
    &= \intS{ \pri{1}{n} (\bm\Phi_d^0 \cdot \textbf{n}_d)}
    -\sum_{e=1}^{n} 
    \intO{ (\bm\Phi_d^0)_{i_e}  \prod^{n}_{\substack{ m=1 \\m \neq e}} r_{i_m}  }
\end{align*}
Including this relation into the former equation yields, 
\begin{multline}
    \ddt {[\textbf{q}_\alpha^{(n)}]_{i_1\ldots i_n}}
    = \sum_{e=1}^{n} 
    \intO{
        \prod^{n}_{\substack{ m=1 \\m \neq e}} r_{i_m} [f_d^0 \textbf{w}_d^0  - \bm\Phi_d^0]_{i_e}
    }
    % +\intS{ \pri{1}{n} (\bm\Phi_d^0 \cdot \textbf{n}_d)}\\
    + \intO{ \pri{1}{n} s_d^0 }\\
    +\intS{ \pri{1}{n} [\bm\Phi_d^0 + f_d^0 \left(\textbf{u}_\Gamma^0 - \textbf{u}_d^0\right)]\cdot \textbf{n}_d }.
    \label{eq:dt_q_n}
\end{multline}
which is the final form of the Lagrangian conservation for the $n^{th}$ order moment of the quantity $f_d^0$ inside the particle. 
For scalar quantity this expression shows that $\ddt \textbf{q}_\alpha^{(n)}$ is entirely symmetric since it involve only product of the $r_{i_n}$ with scalar quantity. 
Regarding the seemingly non-symmetric terms involving $\textbf{w}_d^0$ and $\bm\Phi_d^0$ one can notice that due to the presence of the summation these terms turns out to be symmetric. 
% Notice that if $f_d^0$ is a vector quantity, let say of index $k$ we obtain, 
% \begin{multline}
%     \ddt {[\textbf{q}_\alpha^{(n)}]_{k i_1\ldots i_n}}
%     = \sum_{e=1}^{n} 
%     \intO{
%         \prod^{n}_{\substack{ m=1 \\m \neq e}} r_{i_m} [\textbf{f}_d^0\textbf{w}_d^0  - \bm\Phi_d^0]_{ki_e}
%     }
%     + \intO{ \pri{1}{n} (\textbf{s}_d^0)_k }\\
%     +\intS{ \pri{1}{n} ([\bm\Phi_d^0 + \textbf{f}_d^0 \left(\textbf{u}_\Gamma^0 - \textbf{u}_d^0\right)]\cdot \textbf{n}_d)_k }.
% \end{multline}
% As an example this relation can be directly applied with $f_d^0 = \rho_d \textbf{u}_d^0$ to obtain the moment of momentum conservation. 


Regarding the surface property conservation equations the derivation is similar and will not be displayed here. 
The results yield, 
\begin{multline}
    \ddt {[\textbf{q}_{\alpha\Gamma}^{(n)}]_{i_1\ldots i_n}}
    = \sum_{e=1}^{n} 
    \intS{
        \prod^{n}_{\substack{ m=1 \\m \neq e}} r_{i_m} [f_\Gamma^0\textbf{w}_\Gamma^0 - \bm\Phi_{||\Gamma}^0]_{i_e}
    }
    + \intS{ \pri{1}{n} (\textbf{s}_\Gamma^0)_k }
    \\
    +\intS{ \pri{1}{n} \Jump{\bm\Phi_k^0 + f_k^0 \left(\textbf{u}_\Gamma^0 - \textbf{u}_k^0\right)\cdot \textbf{n}_d}}.
    \label{eq:dt_Qgamma_n}
\end{multline}
From the two conservation laws above 
Summing the equation for $\textbf{q}_{\alpha\Gamma}^{(n)}$ and $\textbf{q}_{\alpha\Gamma}^{(n)}$ one obtain the equation for the total $n^{th}$ order moment,  namely, 
\begin{multline}
    \ddt {[\textbf{Q}_{\alpha\Gamma}^{(n)}]_{i_1\ldots i_n}}
    = 
    \sum_{e=1}^{n} 
    \intO{
        \prod^{n}_{\substack{ m=1 \\m \neq e}} r_{i_m} [f_d^0\textbf{w}_d^0  - \bm\Phi_d^0]_{i_e}
    }
    + \intO{ \pri{1}{n} (\textbf{s}_d^0)_k }\\
    +     
    \sum_{e=1}^{n} 
    \intS{
        \prod^{n}_{\substack{ m=1 \\m \neq e}} r_{i_m} [f_\Gamma^0\textbf{w}_\Gamma^0 - \bm\Phi_{||\Gamma}^0]_{i_e}
    }
    + \intS{ \pri{1}{n} (\textbf{s}_\Gamma^0)_k }
    \\
    +\intS{ \pri{1}{n} ([\bm\Phi_f^0 + \textbf{f}_f^0 \left(\textbf{u}_\Gamma^0 - \textbf{u}_f^0\right)]\cdot \textbf{n}_d)_k }. 
    \label{eq:dt_Q_n}
\end{multline}


% The symmetric part of $[\textbf{q}_\alpha^{(n)}]_{i_0 i_1\ldots i_n}$ is, 
% \begin{equation*}
%     [\textbf{q}_\alpha^{(n)}]_{(i_0 i_1\ldots i_p \ldots i_n )}
% = \frac{1}{n+1}
% \sum_{p=0}^{n} [\textbf{q}_\alpha^{(n)}]_{i_p (i_1\ldots i_0\ldots i_n)}
% \end{equation*}
% where the parenthesis indicates the symmetric index, and it must be understood that this is permutation of the indices.  
% Therefore, the fully symmetric part of the preceding momentum balance can be obtained by summing every permutation of the index $k$ with all other index and dividing by $n$, namely,
% \begin{multline}
%     \ddt {[\textbf{q}_\alpha^{(n)}]_{(i_0 i_1\ldots i_n) }}
%     = \frac{1}{n+1}
%     \sum_{p=0}^{n}
%     \sum_{\substack{ e=0 \\   e \neq i_p}}^{n} \int_{\Omega_\alpha} 
%     \prod^{n}_{\substack{ m=0 \\   m \neq e}} r_{i_m} (w_{i_e}f_{i_p}  - \bm\Phi_{i_p i_e})d\Omega\\
%     +\frac{1}{n+1}
%     \sum_{p=0}^{n}
%     \int_{\Sigma_\alpha} \prod^{n}_{\substack{ m=0 \\   m \neq i_p}} r_{i_m}
%     (\bm\Phi \cdot \textbf{n})_{i_p}d\Sigma
%     + \int_{\Omega_\alpha} 
%     \prod^{n}_{\substack{ m=0 \\   m \neq i_p}} r_{i_m}
%     \textbf{S}_{i_p} d\Omega
% \end{multline}
% It appears that this equation is the fully symmetric parts of the moments equations. 
% The skew symmetric parts will be written, 
% \begin{multline}
%     \frac{d}{dt} (
%     [\textbf{q}_\alpha^{(n)}]_{i_0 i_1\ldots i_n} 
%     - [\textbf{q}_\alpha^{(n)}]_{(i_0 i_1\ldots i_n) }
%     )
%     = 
%     \sum_{e=1}^{n} \int_{\Omega_\alpha} \prod^{n}_{\substack{ m=1 \\   m \neq e}} r_{i_m} (w_{i_e}f_{i_0}  - \bm\Phi_{i_0 i_e})d\Omega
%     -
%     \frac{1}{n+1}
%     \sum_{p=0}^{n}
%     \sum_{\substack{ e=0 \\   e \neq i_p}}^{n} \int_{\Omega_\alpha} 
%     \prod^{n}_{\substack{ m=0 \\   m \neq e}} r_{i_m} (w_{i_e}f_{i_p}  - \bm\Phi_{i_p i_e})d\Omega\\
%     +\int_{\Sigma_\alpha} \pri{1}{n} (\bm\Phi \cdot \textbf{n})_{i_0}d\Sigma
%     -
%     \frac{1}{n+1}
%     \sum_{p=0}^{n}
%     \int_{\Sigma_\alpha} \prod^{n}_{\substack{ m=0 \\   m \neq i_p}} r_{i_m}
%     (\bm\Phi \cdot \textbf{n})_{i_p}d\Sigma
%     + \int_{\Omega_\alpha} \pri{1}{n} \textbf{S}_{i_0} d\Omega
%     -
%     \int_{\Omega_\alpha} 
%     \prod^{n}_{\substack{ m=0 \\   m \neq i_p}} r_{i_m}
%     \textbf{S}_{i_p} d\Omega
% \end{multline}
% It is known that the non-convective fluxes vanish at the order one of this equation. 
% We would like to make appear this property explicitly. 
% \begin{multline*}
%     \sum_{e=1}^{n} \int_{\Omega_\alpha} \prod^{n}_{\substack{ m=1 \\   m \neq e}} r_{i_m} \bm\Phi_{i_0 i_e} d\Omega
%     -
%     \frac{1}{n+1}
%     \sum_{p=0}^{n}
%     \sum_{\substack{ e=0 \\   e \neq i_p}}^{n} \int_{\Omega_\alpha} 
%     \prod^{n}_{\substack{ m=0 \\   m \neq e}} r_{i_m}  \bm\Phi_{i_p i_e}d\Omega\\
%     =
%     \sum_{e=1}^{n} \int_{\Omega_\alpha} \prod^{n}_{\substack{ m=1 \\   m \neq e}} r_{i_m} \bm\Phi_{i_0 i_e}d\Omega
%     -
%     \frac{1}{n+1}
%     \sum_{p=0}^{n}
%     \sum_{\substack{ e=0 \\   e \neq i_p}}^{n} \int_{\Omega_\alpha} 
%     \prod^{n}_{\substack{ m=0 \\   m \neq e}} r_{i_m}  \bm\Phi_{i_p i_e}d\Omega\\
%     =
%     \frac{- 1}{n+1}
%     \sum_{p=1}^{n}
%     \sum_{\substack{ e=1 \\   e \neq i_p}}^{n} \int_{\Omega_\alpha} 
%     \prod^{n}_{\substack{ m=1 \\   m \neq e}} r_{i_m}  \bm\Phi_{i_p i_e}d\Omega
% \end{multline*}
% Which makes a non-vanishing parts for the integral of the stress. 
% Instead, we rather derive the moments' equation antisymmetric in the indices $i_e$ $i_0$ by subtracting the permuted equation
% \begin{multline}
%     \ddt{ [\textbf{q}_\alpha^{(n)}]_{i_0 i_1\ldots i_n }}
%     = \sum_{e=1}^{n} \int_{\Omega_\alpha} \prod^{n}_{\substack{ m=1 \\   m \neq e}} r_{i_m} (w_{i_e}f_{i_0}  - \bm\Phi_{i_0 i_e})d\Omega
%     +\int_{\Sigma_\alpha} \pri{1}{n} (\bm\Phi \cdot \textbf{n})_{i_0}d\Sigma
%     + \int_{\Omega_\alpha} \pri{1}{n} \textbf{S}_{i_0} d\Omega
% \end{multline}

% As an example we give the two first order moments for particles without mass transfer: 
% If $n=1$ : 
% \begin{equation}
%     \ddt{ [\textbf{q}_\alpha^{(n)}]_{i_1}}
%     = \int_{\Omega_\alpha} (w_{i_1}f  - \bm\Phi_{i_1})d\Omega
%     +\int_{\Sigma_\alpha} r_{i_1}\bm\Phi \cdot \textbf{n}d\Sigma
%     + \int_{\Omega_\alpha}r_{i_1} \textbf{S} d\Omega
% \end{equation}
% and for $n=2$ : 
% \begin{multline}
%     \label{eq:moment_n2}
%     \ddt {[\textbf{q}_\alpha^{(n)}]_{i_1 i_2}}
%     = 
%     \int_{\Omega_\alpha} r_{i_2} (w_{i_1}f  - \bm\Phi_{i_1})d\Omega
%     +\int_{\Omega_\alpha} r_{i_1} (w_{i_2}f  - \bm\Phi_{i_2})d\Omega
%     +\int_{\Sigma_\alpha}  r_{i_1}r_{i_2} \bm\Phi \cdot \textbf{n}d\Sigma\\
%     + \int_{\Omega_\alpha} r_{i_1}r_{i_2}  \textbf{S} d\Omega
% \end{multline}
% For the momentum equation we obtain : 
% \begin{equation}
%     \ddt{ \mathcal{P}_{ij}}
%     = \int_{\Omega_\alpha} (w_{i}w_j \rho_2  - \bm{\sigma}_{ij})d\Omega
%     +\int_{\Sigma_\alpha} r_{i} \sigma_{jk} \cdot n_k d\Sigma
%     + \int_{\Omega_\alpha}r_{i} \rho_d g_j d\Omega
% \end{equation}
% \begin{multline}
%     \ddt{ \mathcal{P}_{i j k}}
%     = 
%     \int_{\Omega_\alpha} r_{j} (w_{i} w_k\rho_2 - \sigma_{ik})d\Omega
%     +\int_{\Omega_\alpha} r_{i} (w_{j} w_k\rho_2 - \sigma_{jk})d\Omega
%     +\int_{\Sigma_\alpha}  r_{i}r_{j} \sigma_{kl} n_l d\Sigma\\
%     + \int_{\Omega_\alpha} r_{i}r_{j}  \rho_2 g_k d\Omega
%     \label{eq:second_momoent_of_momentum}
% \end{multline}

% \section{Averaged moments equations}
Then, we obtain the particle averaged equation for $\pavg{\textbf{Q}_\alpha^{(n)}}$ by averaging \ref{eq:dt_Q_n},
% it is possible from this equation to carry out a particle-average, which directly yield the $n^{th}$ order moment equation : 
\begin{multline*}
    \pddt \pavg{[\textbf{Q}_\alpha^{(n)}]_{i_1\ldots i_n}^\alpha}
    + \div  \pavg{\textbf{u}_\alpha [\textbf{Q}_\alpha^{(n)}]_{i_1\ldots i_n}^\alpha}
    = \sum_{e=1}^{n} 
    \pOavg{
        \prod^{n}_{\substack{ m=1 \\m \neq e}} r_{i_m} [f_d^0\textbf{w}_d^0  - \bm\Phi_d^0]_{i_e}
    }\\
    + \pOavg{ \pri{1}{n} (\textbf{s}_d^0)_k }
    +     
    \sum_{e=1}^{n} 
    \pSavg{
        \prod^{n}_{\substack{ m=1 \\m \neq e}} r_{i_m} [f_\Gamma^0\textbf{w}_\Gamma^0 - \bm\Phi_{||\Gamma}^0]_{i_e}
    }
    + \pSavg{ \pri{1}{n} (\textbf{s}_\Gamma^0)_k }\\
    +\pSavg{ \pri{1}{n} ([\bm\Phi_f^0 + \textbf{f}_f^0 \left(\textbf{u}_\Gamma^0 - \textbf{u}_f^0\right)]\cdot \textbf{n}_d)_k }. 
\end{multline*}

\section{Arbitrary order equivalence}
\label{sec:demo}
In this appendix we provide a general proof of \ref{eq:scheme_equivalence} between particle-averaged and phase-avergaed equation for the dispersed phase. 
Let's begin by re-writing the phase averaged equation
\begin{equation}
        \pddt \avg{\chi_d f_d^0}
        = \div \avg{\chi_d \bm\Phi_d^0 - \chi_d f_d^0 \textbf{u}_d^0}
        + \avg{\chi_d s_d^0}
        + \avg{\delta_\Gamma\left[
            \bm\Phi_d^0
            + f_d^0
            \left(
                \textbf{u}_\Gamma^0
                - \textbf{u}_d^0
            \right)
        \right]
        \cdot \textbf{n}_d} 
        \label{eq:dt_f_d_O}
\end{equation}
Our objective here is to demonstrate in the first place how \ref{eq:dt_f_d_O} is related to the Lagrangian moments equations given by \ref{eq:dt_q_n}. 
The first step is to expand each term of \ref{eq:dt_f_d_O} using the relation \ref{eq:f_exp} which gives directly,
\begin{align*}
        0 &=
        - \pddt \expo{f_d^0} \\
        &+\div \expo{(\bm\Phi_d^0  - f_d^0 \textbf{u}_d^0)}\\
        &+ \expo{ s_d^0}\\
        &+ \expoS{\left[
            \bm\Phi_d^0
            + f_d^0
            \left(
                \textbf{u}_\Gamma^0
                - \textbf{u}_d^0
            \right)
        \right]
        \cdot \textbf{n}_d} \\
\end{align*}
The third term can be reformulated using the decomposition : $\textbf{u}_d^0 = \textbf{u}_\alpha + \textbf{w}_d^0$, which gives,
\begin{multline}
    \expo{f_d^0 \textbf{u}_d^0}\\
    =     \expoU{f_d^0 }\\
    +     \expo{f_d^0 \textbf{w}_d^0}
\end{multline}
Injecting this formulation in the former equation yields,
\begin{align}
    & \pddt \expo{f_d^0} \\
    &+ \div \expoU{f_d^0}\\
    &= \div \expo{(\bm\Phi_d^0 - f_d^0 \textbf{w}_d^0)}\\
    &+ \expo{ s_d^0}\\
    &+ \expoS{\left[
        \bm\Phi_d^0
        + f_d^0 
        \left(
            \textbf{u}_I^0
            - \textbf{u}_d^0
        \right)
    \right]
    \cdot \textbf{n}_d} \\
    \label{eq:nearly_done}
\end{align}
Then, notice that the first term on the right-hand side can be re-written, when evaluated at the order $n-1$, as follows,
\begin{multline*}
    \div \expo[(n-1)]{(\bm\Phi_d^0 - f_d^0 \textbf{w}_d^0)}
    = \\
    \frac{(-1)^{n}}{{(n)}!} \partialp{1}{n}  n \pOavg{ \pri{1}{n-1}(f_d^0 \textbf{w}_d^0 - \bm\Phi_d^0)_{i_{n}} }
\end{multline*} 
Upon using that expression in \ref{eq:nearly_done} we can factor out the gradient operators, i.e. the $\frac{(-1)^n}{n!} \partialp{1}{n}$, which gives, 
\begin{multline}
    0 = \frac{(-1)^n}{n!}
    \partialp{1}{n}
    \left[
        - \partial_t
        \pavg{[\textbf{q}_\alpha^{(n)}]_{i_1\ldots i_n}}
        - \div \pavg{\textbf{u}_\alpha [\textbf{q}_\alpha^{(n)}]_{i_1\ldots i_n}}
    \right.\\\left.
        +n\pavg{\int_{\Omega_\alpha} \pri{1}{n-1} (f_d^0 \textbf{w}_d^0-\bm\Phi_d^0) d\Omega}
        +\pavg{\int_{\Omega_\alpha} \pri{1}{n} s_d^0 d\Omega}
        \right.\\\left.
        +\pavg{\int_{\Omega_\alpha} \pri{1}{n} \left[
            \bm\Phi_d^0
            + f_d^0
            \left(
                \textbf{u}_I^0
                - \textbf{u}_d^0
            \right)
        \right]
        \cdot \textbf{n}_d d\Omega}
    \right].
    \label{eq:exp_f_d_O}
\end{multline}
At this stage one might immediately recognize \ref{eq:dt_q_n} in the square bracket, however notice that the third term of \ref{eq:exp_f_d_O} differ with the second term of \ref{eq:dt_q_n}. 
Indeed, 
\begin{equation*}
    n\pavg{\int_{\Omega_\alpha} \pri{1}{n-1} ( f_d^0 \textbf{w}_d^0-\bm\Phi_d^0)_{i_n} d\Omega}
    \neq
    \sum_{e=1}^{n} 
    \avg{
        \intO{
        \prod^{n}_{\substack{ m=1 \\m \neq e}} r_{i_m} [f_d^0 \textbf{w}_d^0  - \bm\Phi_d^0]_{i_e}
        }
    }. 
    \label{eq:ineq_which_does_not_make_sens}
\end{equation*}
Even if the above inequality holds as it is, it must be understood that what's matter in \ref{eq:exp_f_d_O} is the gradient of that term. 
Additionally, the contraction of either of these term with the gradient operator $\partialp{1}{n}$ makes the skew-symmetric part of these tensors (in the indices $i_1\ldots i_n$) having no contribution to the expression.
Moreover, if one apply the operator $\partialp{1}{n}$ on each side of \ref{eq:ineq_which_does_not_make_sens} he eventually finds that, 
Thus, one might apply the operator $\partialp{1}{n}$ on each side of \ref{eq:ineq_which_does_not_make_sens} and notice that the skew-symmetric part of the LHS term of \ref{eq:ineq_which_does_not_make_sens} vanish leading to, 
\begin{equation*}
    \partialp{1}{n}\left[
        n\pavg{\int_{\Omega_\alpha} \pri{1}{n-1} ( f_d^0 \textbf{w}_d^0-\bm\Phi_d^0)_{i_n} d\Omega}
        \right]
    =
    \partialp{1}{n}\left[
    \sum_{e=1}^{n} 
    \avg{
        \intO{
        \prod^{n}_{\substack{ m=1 \\m \neq e}} r_{i_m} [f_d^0 \textbf{w}_d^0  - \bm\Phi_d^0]_{i_e}
        }
    }
    \right]. 
    \label{eq:it_make_sens_again}
\end{equation*}
Injecting this last equality in \ref{eq:exp_f_d_O} and using \ref{eq:dt_Qgamma_n} to reformulate the exchange term gives directly, 
\begin{multline}
    0 = \frac{(-1)^n}{n!}
    \partialp{1}{n}
    \left[
        - \pddt \pavg{[\textbf{Q}_\alpha^{(n)}]_{i_1\ldots i_n}^\alpha}
        - \div  \pavg{\textbf{u}_\alpha [\textbf{Q}_\alpha^{(n)}]_{i_1\ldots i_n}^\alpha}
        =+\sum_{e=1}^{n} 
        \pOavg{
            \prod^{n}_{\substack{ m=1 \\m \neq e}} r_{i_m} [f_d^0\textbf{w}_d^0  - \bm\Phi_d^0]_{i_e}
        }\right.\\\left.
        + \pOavg{ \pri{1}{n} (\textbf{s}_d^0)_k }
        +     
        \sum_{e=1}^{n} 
        \pSavg{
            \prod^{n}_{\substack{ m=1 \\m \neq e}} r_{i_m} [f_\Gamma^0\textbf{w}_\Gamma^0 - \bm\Phi_{||\Gamma}^0]_{i_e}
        }
        + \pSavg{ \pri{1}{n} (\textbf{s}_\Gamma^0)_k } \right.\\\left.
        +\pSavg{ \pri{1}{n} ([\bm\Phi_f^0 + \textbf{f}_f^0 \left(\textbf{u}_\Gamma^0 - \textbf{u}_f^0\right)]\cdot \textbf{n}_d)_k }. 
    \right],
\end{multline}
which finally proves \ref{eq:scheme_equivalence} in all of its generality.
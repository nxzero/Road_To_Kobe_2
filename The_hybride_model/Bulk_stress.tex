

\subsection{The bulk stress in dispersed multiphase flow}


Now that the architecture of the averaged dispersed multiphase flow equation is clarified, we would like to present the expression of the bulk stress tensor in a suspension of inertial particles subject to an arbitrary local body force field, $\textbf{b}^0$.
Firstly, is important to recall the definition of the \textit{bulk stress}. 
We define the \textit{bulk stress} tensor as a force applied on the fluid and on the particles phase, having the form $\div \bm{\Sigma}$, which added to the total external force $\textbf{B}$, balance exactly the material derivative of the mixture momentum : $\frac{D \rho \textbf{u}}{Dt}$. 
In this definition $\textbf{B}$ cannot be decomposed into a vector and a divergence of a tensor, in which case the latter would just contribute to $\bm{\Sigma}$.
Equally, we insist on the fact that the momentum considered is $\rho\textbf{u}$ which is the bulk momentum. 

We first expose the averaged mixture momentum and angular momentum equation easily derived from \ref{eq:dt_avg_f}, 
\begin{align}
    \pddt (\rho u_i)
    + \partial (\rho u_iu_k
    + \sigma_{ik}^\text{eq})
    = b_i\\
    \epsilon_{ijk} \sigma_{jk}
    = 0 
    \label{eq:momentum_bulk}
\end{align}
In the momentum equation we have defined, $\sigma_{ik}^\text{eq} = \avg{\rho\textbf{u}'\textbf{u}'}
- \avg{\chi_1\bm{\sigma}_1^0}-\avg{\chi_2\bm{\sigma}_2^0} - \avg{\delta_I \bm{\sigma}_I^0}$. 
Additionally, in the averaged angular momentum equation we have assumed that no-body torque exist at the local scale making the second equality equal to $0$ \citet{leal2007advanced} and the averaged mixture stress $\bm{\sigma}$ a symmetric quantity. 
However, note that $\bm{\sigma}$ is not exactly equal to the \textit{bulk stress} tensor $\bm{\Sigma}$ since $\textbf{b}$ can be expressed as a divergence of a stress.
Indeed, have defined $\textbf{b} = \textbf{B} + \div  \pMOavg{\textbf{b}_2^0}$ where $\textbf{B} = \phi_1 \textbf{b}_1 +  \pOavg{\textbf{b}_2^0 }$ and  $\pMOavg{\textbf{r}\textbf{b}_2^0}$ is defined accordingly to the previous definition. 
It follows the definition of the \textit{bulk stress} : 
\begin{equation}
    \bm{\Sigma}
    = 
    \avg{\rho\textbf{u}'\textbf{u}'}
    - \avg{\chi_1\bm{\sigma}_1^0}
    - \avg{\chi_2\bm{\sigma}_2^0} 
    - \avg{\delta_I \bm{\sigma}_I^0}
    - \pMOavg{\textbf{r}\textbf{b}_2^0}
\end{equation}
which proves already, in the absence of particles moments of the body forces $\pMOavg{\textbf{r}\textbf{b}_2^0}$, the antisymmetric part of the suspension stress is null, in agreement with \citet{dolata2020heterogeneous}.
% This skew symmetric part can be written in vector form as, 
% \begin{equation*}
%     \epsilon_{ijk}\textbf{T}_{jk}
%     = 
%     -\epsilon_{ijk} \pOavg{r_kb_j}
%     -\epsilon_{ijk}\frac{1}{2}\partial_l \pOavg{r_lr_kb_j}
%     = 0  
% \end{equation*}
We recall that the carrier fluid is a Newtonian fluid, therefore we may express the fluid phase stress as, 
\begin{equation}
    \phi_1 \sigma_{1,jk}
    = -p_1 \delta_{jk}
    + \mu_1 e_{jk}
    - \mu_1 \phi_2 e_{2,jk}. 
\end{equation} 
Additionally, we use the methodology of \citep{lhuillier1992volume,lhuillier1996contribution} to re express the averaged particle stress terms. 
The divergence of the particle phase stress may be expressed using \ref{eq:f_exp}, 
\begin{align}
    \label{eq:exp_sigma22}
    \partial_k \avg{\chi_2 {\sigma}_2^0}_{ik}
    &=  \partial_k\pOavg{ {\sigma}_{2,ik}^0}
    -\frac{1}{2} \partial_k\partial_j
    \pOavg{ r_j{\sigma}^0_{2,ik} + r_k\sigma^0_{2,ij}}
    + \ldots  \\
    \label{eq:exp_sigmaI2}
    \partial_k \avg{\delta_I {\sigma}^0_I}_{ik} 
    &=  \partial_k\pSavg{ {\sigma}_{I,ik}^0 }
        -\frac{1}{2} \partial_k\partial_j \pSavg{ r_j {\sigma}_{I,ik}^0+r_k {\sigma}_{I,ij}^0 }
        + \ldots  
\end{align}
Note that the heterogeneous terms must remain symmetric in the index $kj$ due to the double contraction with the operator $\partial_k\partial_j$, thus only the symmetric part in $jk$ remain and the terms such as, $\pOavg{ r_j{\sigma}^0_{2,ik} - r_k\sigma^0_{2,ij}}$ vanish. 
Upon the use of the moment of momentum equation of the first and second order we can easily derive these expressions, 
\begin{align}
    \intS{ (\bm{\sigma}_I)_{ik}}
    +\intO{ (\bm{\sigma}_2^0)_{ik}}
    = 
    \intO{ \rho_2 
    (\textbf{w}_2^0\textbf{w}_2^0  )_{ik}
    }
    -\ddt \intO{ r_i (\textbf{u}^0_2)_k }
    +\intS{ 
        b_{i}
        r_k 
    }
    +\intS{ 
     r_i (\bm{\sigma}_1^0 \cdot \textbf{n}_2)_{k}
    }\\
    \intO{ r_{j}(\bm{\sigma}^0_2)_{ik}+r_{k}(\bm{\sigma}^0_2)_{ji}}
    +\intS{ r_{j}(\bm{\sigma}^0_I)_{ik}+r_{k}(\bm{\sigma}_I^0)_{ji}}
    = 
    - \ddt\intO{ \rho_2 (\textbf{u}_2^0)_i r_j r_k }\nonumber\\
    + \intO{ \rho_2 (r_{j} (\textbf{w}_2^0)_k (\textbf{u}^0_2)_i + r_k (\textbf{w}_2^0)_j (\textbf{u}^0_2)_i)}
    +\intS{  r_{k}r_{j} (\bm{\sigma}_1^0)_{il} (\textbf{n}_2)_l }
    + \intO{ r_{k}r_{j}  \rho_2 b_i } 
    \label{eq:dt_P2_alpha}
\end{align}
It is evident that by using an arbitrary order of moment of momentum equation one can substitute any volume integral of the particle stress appearing in the expansion \ref{eq:exp_sigma22}. 
% By consideration of symmetric of the local stress, it is evident that the skew symmetric part of the moment of momentum will not have any dynamical role thus we can retrieve the average of \ref{eq:dt_mu_alpha} to the first relation. 
% To extract the skew symmetric part we start by writing the permutation of these equations with $ik$ yielding,  
% \begin{align}
%     \intS{ 
%     (\bm{\sigma}_I)_{ki}
%     }
%     +\intO{ 
%     (\bm{\sigma}_2^0)_{ki}
%     }
%     = 
%     \intO{ \rho_2 
%     (\textbf{w}_2^0\textbf{w}_2^0  )_{ki}
%     }
%     -\ddt \intO{ r_k (\textbf{u}^0_2)_i }
%     +\intS{ b_{k}r_i }
%     +\intS{ r_k (\bm{\sigma}_1^0 \cdot \textbf{n}_2)_{i}}\\
%     \intO{ r_{j}(\bm{\sigma}^0_2)_{ki}+r_{i}(\bm{\sigma}^0_2)_{jk}}
%     +\intS{ r_{j}(\bm{\sigma}^0_I)_{ki}+r_{i}(\bm{\sigma}_I^0)_{jk}}
%     = 
%     - \ddt\intO{ \rho_2 (\textbf{u}_2^0)_k r_j r_i }\nonumber\\
%     + \intO{ \rho_2 (r_{j} (\textbf{w}_2^0)_i (\textbf{u}^0_2)_k + r_i (\textbf{w}_2^0)_j (\textbf{u}^0_2)_k)}
%     +\intS{  r_{i}r_{j} (\bm{\sigma}_1^0)_{kl} (\textbf{n}_2)_l }
%     + \intO{ r_{i}r_{j}  \rho_2 b_k } \\
%     \intO{ r_{i}(\bm{\sigma}^0_2)_{jk}+r_{k}(\bm{\sigma}^0_2)_{ij}}
%     +\intS{ r_{i}(\bm{\sigma}^0_I)_{jk}+r_{k}(\bm{\sigma}_I^0)_{ij}}
%     = 
%     - \ddt\intO{ \rho_2 (\textbf{u}_2^0)_j r_i r_k }\nonumber\\
%     + \intO{ \rho_2 (r_{i} (\textbf{w}_2^0)_k (\textbf{u}^0_2)_j 
%     + r_k (\textbf{w}_2^0)_i (\textbf{u}^0_2)_j)}
%     +\intS{  r_{k}r_{i} (\bm{\sigma}_1^0)_{jl} (\textbf{n}_2)_l }
%     + \intO{ r_{k}r_{i}  \rho_2 b_j } 
% \end{align}
% Acknowledgement of the symmetrical nature of $\bm{\sigma}_2^0$ and $\bm{\sigma}_I^0$ gives the following antisymmetrical balance equations, 
% \begin{align}
%     0
%     = 
%     -\ddt \intO{ r_i (\textbf{u}^0_2)_k -r_k (\textbf{u}^0_2)_i }
%     +\intS{ b_{i}r_k -b_{k}r_i }
%     +\intS{r_i (\bm{\sigma}_1^0 \cdot \textbf{n}_2)_{k} - r_k (\bm{\sigma}_1^0 \cdot \textbf{n}_2)_{i}}\\
%     \intO{ r_{j}(\bm{\sigma}^0_2)_{ik}
%             -r_{i}(\bm{\sigma}^0_2)_{jk}}
%     +\intS{ r_{j}(\bm{\sigma}^0_I)_{ik}
%            - r_{i}(\bm{\sigma}_I^0)_{jk}}
%     = 
%     - \ddt\intO{ \rho_2 (\textbf{u}_2^0)_i r_j r_k -  \rho_2 (\textbf{u}_2^0)_j r_i r_k }\nonumber\\
%     + \intO{ \rho_2 (r_{j} (\textbf{w}_2^0)_k (\textbf{u}^0_2)_i + r_k (\textbf{w}_2^0)_j (\textbf{u}^0_2)_i)}
%     - \intO{ \rho_2 (r_{i} (\textbf{w}_2^0)_k (\textbf{u}^0_2)_j + r_k (\textbf{w}_2^0)_i (\textbf{u}^0_2)_j)}\\
%     +\intS{  r_{k}r_{j} (\bm{\sigma}_1^0)_{il} (\textbf{n}_2)_l 
%     -r_{k}r_{i} (\bm{\sigma}_1^0)_{jl} (\textbf{n}_2)_l }
%     + \intO{ r_{k}r_{j}  \rho_2 b_i
%     - r_{k}r_{i}  \rho_2 b_j } 
% \end{align}
% The last equation need to be added to the second permutation which gives, 
% \begin{align*}
% \intO{ r_{j}(\bm{\sigma}^0_2)_{ik}}
% +\intS{ r_{j}(\bm{\sigma}^0_I)_{ik}}
% = 
% - \ddt\intO{ \rho_2 (\textbf{u}_2^0)_k r_j r_i 
% + \rho_2 (\textbf{u}_2^0)_i r_j r_k 
% -  \rho_2 (\textbf{u}_2^0)_j r_i r_k }\nonumber\\
% + \intO{ \rho_2 (r_{j} (\textbf{w}_2^0)_k (\textbf{u}^0_2)_i + r_k (\textbf{w}_2^0)_j (\textbf{u}^0_2)_i)}
% - \intO{ \rho_2 (r_{i} (\textbf{w}_2^0)_k (\textbf{u}^0_2)_j + r_k (\textbf{w}_2^0)_i (\textbf{u}^0_2)_j)}\\
% - \intO{ \rho_2 (r_{j} (\textbf{w}_2^0)_i (\textbf{u}^0_2)_k + r_i (\textbf{w}_2^0)_j (\textbf{u}^0_2)_k)}\\
% +\intS{  r_{k}r_{j} (\bm{\sigma}_1^0)_{il} (\textbf{n}_2)_l 
% +r_{j}r_{i} (\bm{\sigma}_1^0)_{kl} (\textbf{n}_2)_l 
% -r_{k}r_{i} (\bm{\sigma}_1^0)_{jl} (\textbf{n}_2)_l }
% + \intO{ r_{k}r_{j}  \rho_2 b_i
% + r_{j}r_{i}  \rho_2 b_k 
% - r_{k}r_{i}  \rho_2 b_j 
% } 
% \end{align*}
In, addition one must notice that the particle angular momentum balance equation doesn't involve the integral of the particle local stress and has therefore, no dynamical role in the equivalent stress expression. 
Making use of these remarks we obtain this general formula for the suspension stress,  
\begin{multline}
    \bm{\Sigma}
    = \avg{\rho\textbf{u}'\textbf{u}'}_{ik}
    + \phi_1 p_1 \delta_{ik}
    - \mu_1 e_{ik}
    % + \mu_1 \phi_2 e_{2,ik}. 
    - \pOavg{ \rho_2 (\textbf{w}_2^0\textbf{w}_2^0  )_{ik}}
    + \pavg{\ddt {\mathcal{S}_{ik}} }\\
    - \pSavg{ b_{i}r_k - b_{k}r_i }
    - \pSavg{ r_i (\bm{\sigma}_1^0 \cdot \textbf{n}_2)_{k}
    + r_k (\bm{\sigma}_1^0 \cdot \textbf{n}_2)_{i}}
    + \mu_1 \pOavg{e_2^0}_{ik}
    + \frac{1}{2} \div\bm{\Sigma}_1
    \label{eq:eq_stress}
\end{multline}
with the inhomogeneous stress gathered in $\bm{\Sigma}_1$, namely,
\begin{multline}
    \bm{\Sigma}_1
    = 
    - \pavg{\ddt\intO{ \rho_2 (\textbf{u}_2^0)_i r_j r_k }}
    + \pOavg{ \rho_2 (r_{j} (\textbf{w}_2^0)_k (\textbf{u}^0_2)_i + r_k (\textbf{w}_2^0)_j (\textbf{u}^0_2)_i)}\nonumber\\
    +\pSavg{  r_{k}r_{j} (\bm{\sigma}_1^0)_{il} (\textbf{n}_2)_l }
    - \mu_1 2 \pOavg{\textbf{r} \textbf{e}_2^0}_{jik}
\end{multline}
According to \ref{eq:scheme_equivalence}, expanding each component related to the dispersed phase in \ref{eq:momentum_bulk} one would see appear each moment of momentum equations under the divergence operator.
However, to stay consistent with the definition of the bulk stress tensor $\bm{\Sigma}$, we must keep the advecting term on the LHS of \ref{eq:momentum_bulk} unchanged, this is however not the case of the averaged body force term $\textbf{b}$ which allowed us to cancel all the body forces terms with the expansion of $\textbf{T}$.  

One of the major question in suspension dynamic raised by several authors, is the evaluation of the bulk stress or equivalent stress tensor of a suspension, see \citep{prosperetti2006stress, batchelor1970stress,zhang1997momentum,nadim1996concise} and more recently \citet{dolata2020heterogeneous}. 
The answer to this question is given in the general case of the generic averaged mixture equation. 
In light of \ref{eq:eq_stress} we have demonstrated how to express in a routine manner the bulk stress of the particle phase. 
And doing so without making appear explicitly the particles  internal stress. 
This, conclusion deserve several comments regarding previous studies. 
In  \citet{jackson1997locally},  the volume averaged momentum balance (equation (38) of \citet{jackson1997locally}) they make appear the higher moment of velocity of the particles as closure terms, these are hidden in $\pOavg{\textbf{w}_2^0 \textbf{w}_2^0}$.
However, in \citet{jackson1997locally} they did not remove the angular momentum to the stress yieldings a slightly different term. 
What we have shown here is that these higher moments of the particles phase such has the particles rotations have no dynamical significance in the mixture equations. 
Therefore, equation (38) of \citet{jackson1997locally} can be further simplified to the fluid and first order particle averaged equations. 
Equally, in the momentum mixture equation derived by \citet{zhang1997momentum} (equation (8.2)), they make appear explicitly the higher moment of acceleration and the higher moments of velocity in their equivalent stress. 
These terms must therefore simplify. 
In fact as, it could be supposed in their appendix these moments equally cancel. 
In agreement with \citet{dolata2021faxen} which also found that the only remaining part of the stress were solely the fluid phase exchange terms upon the calculation of the body forces moments. 
Similar, comments can be made on the study of \citet{prosperetti2006stress}. 
This also explain why \citet{nadim1996concise} found out that the interfacial terms of the surface tension and viscous interfacial forces play no direct role in the equivalent stress of the emulsion.
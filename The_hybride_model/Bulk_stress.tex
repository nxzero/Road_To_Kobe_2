

\section{The bulk stress in dispersed multiphase flow}
\label{sec:symetric_stress}


One of the major questions in suspension dynamic raised by several authors, is the evaluation of the bulk stress or equivalent stress tensor of a suspension, see \citep{batchelor1970stress, prosperetti2006stress,zhang1997momentum,nadim1996concise} and more recently \citet{dolata2020heterogeneous}. 
Specifically, we seek to express the bulk stress in the suspension in terms of particle-averaged quantities. 
Only then it is possible to derive closures for the stress. 
Therefore, in this subsection we derive an expression for the bulk stress of an emulsion in terms of the Lagrangian particle quantities derived in the two previous sections. 
For the sake of generality in this section we consider that the flow is subjected to an arbitrary body force $\textbf{b}$. 

\subsection{The bulk stress formulation}

Before proceeding further, it is useful to clarify what we mean exactly by the \textit{bulk stress}.
The \textit{bulk stress} tensor is the force per unit of surface applied on the fluid and on the particles phases, having the form $\div \bm{\sigma}^\text{eq}_m$, which added to the total external force, balances exactly the material derivative of the mixture momentum, namely: $\pddt (\rho \textbf{u}_m) + \div (\rho \textbf{u}_m\textbf{u}_m)$. 
This definition of the bulk stress implies that the external body force term $\textbf{b}$ cannot be decomposed into a vector plus a divergence of a tensor, in which case the latter would just contribute to $\bm{\sigma}^\text{eq}$.
Thus, in this definition $\textbf{b}$ must be reformulated since $\textbf{b} = \phi_f \textbf{b}_f+\phi_d \textbf{b}_d$ and that the term $\phi_d \textbf{b}_d$ can be decomposed into a Taylor series according to \ref{eq:f_exp}. 
Therefore, the momentum equation is obtained by setting $\rho \textbf{g} = \textbf{b}$ in \ref{eq:dt_avg_rhou_m} and by reformulating the body force term accordingly we obtain,
\begin{align}
    \label{eq:momentum_bulk}
    \pddt (\rho (\textbf{u}_m)_i)
    + \partial_j  [\rho (\textbf{u}_m \textbf{u}_m)_{ij}
    + (\bm\sigma_{m}^\text{eq})_{ij}]
    &= \left[\phi_f \textbf{b}_f + \pOavg{\textbf{b}}\right]_i
    % \epsilon_{ijk} \sigma_{jk}
    % &= 0 
    % \label{eq:angular_momentum_bulk}
\end{align}
% Indeed, in the averaged angular momentum equation we have assumed that no-body torque exist at the local scale making the skew-symmetric part of $\sigma_{jk}^0$ equal to $0$ \citet{leal2007advanced}. 
% Taking the average of the bulk momentum and angular momentum equation gives directly, 
Where we have defined the effective stress of the suspension as,
\begin{multline}
    \bm{\sigma}_m^\text{ed}
    = 
    \avg{\rho^0\textbf{u}_m'\textbf{u}'_m}
    + \phi_fp_f\bm\delta
    - 2\mu_f\textbf{e}
    - \phi_d(\bm{\sigma}_d - 2\mu_f \textbf{e}_d)
    - \phi_\Gamma \bm{\sigma}_\Gamma\\
    + \pOavg{\textbf{r}\textbf{b}_d^0}
    -\frac{1}{2} \div \pOavg{\textbf{rr}\textbf{b}_d^0} + \ldots
    \label{eq:sigma_bulk}
\end{multline}
% where we have used the relation 
The first term corresponds to the Reynolds stress, and the second third and fourth terms are obtained by noticing that 
$
    \phi_f \bm\sigma_f
    = -\phi_f p_f \bm\delta
    + 2\mu_f \textbf{e}
    - 2 \mu_f \phi_d \textbf{e}_d 
$
where $\phi_f p_f$ is the mean fluid pressure and $\textbf{e} = \grad \textbf{u} + ^\dagger(\grad \textbf{u})$ the averaged strain rate of the suspension. 
% The last two terms of \ref{eq:sigma_bulk} have been obtained by expanding the body force term $\phi_d \textbf{b}_d$ originally present on the right-hand side of \ref{eq:momentum_bulk} in a Taylor expansion according to \ref{eq:f_exp}. 
On a side note, the averaged rate of strain, \textbf{e}, present in \ref{eq:sigma_bulk} involves the averaged bulk velocity \textbf{u}. 
However, the momentum equation's purpose is to be solved for the Favre averaged velocity fields $\textbf{u}_m$. 
This means that two velocity fields are present in this equation meaning that a supplementary equation is needed to recover $\textbf{u}$ from $\textbf{u}_m$ and inversely. 

We considered that the local angular momentum balance follows  $\epsilon_{ijk} \sigma_{jk}^0 = 0$. 
The interface stress also follows this condition since $\bm\sigma_I^0 = \gamma (\bm\delta - \textbf{nn})$ which is by definition symmetric.  
Due to the linearity of the ensemble average operator we deduce that the averaged stress also respects $\epsilon_{ijk} \sigma_{jk} =0$.
Thus, we can already conclude that the first five terms on the right-hand side of \ref{eq:sigma_bulk} are by definition symmetric since they are just averaged quantities of tensors which are by definition symmetric. 
Consequently, the only possible skew-symmetric contribution to the suspension stress must arise from the last two terms on the right-hand side of \ref{eq:sigma_bulk}.  
The discussion regarding the symmetry of $\bm\sigma^{eq}_m$ will be addressed in more detail at the end of this section. 

% The first moment contribution to the skew symmetric part is given by 
% \begin{equation}
%     \pOavg{\textbf{r}\textbf{b}_d^0 - \textbf{b}_d^0 \textbf{r}}. 
%     \label{eq:body_torque}
% \end{equation}
% This term corresponds to the averaged torque generated by the body force field $\textbf{b}^0$ on the particles. 
% Thus, all body force fields generating body torque will induce self toque in the suspension. 

% For the higher moment of body force the reasoning is slightly different.
% We use a methodology similar to \citep{lhuillier1992volume,lhuillier1996contribution} to re express the second and higher moment of the body force.  
% For convenience let us note 
% \begin{equation}
%     B_{ijk}
%     = \pOavg{\textbf{rr}\textbf{b}_d^0} - \frac{1}{3}\div \pOavg{\textbf{rrr}\textbf{b}_d^0} + \ldots
% \end{equation}
% % which represent the second plus the divergence of the higher order moment of the body forces. 
% Since this tensor is the sum of the second plus the higher moment it is not symmetric on any index. 

% We recall that the carrier fluid is a Newtonian fluid, therefore we may express the fluid phase stress as, 
Let us now present the bulk stress formulation in the \textit{hybrid} form, meaning in terms of particle-average quantities.  
The divergence of the dispersed phase stresses present in \ref{eq:momentum_bulk} through \ref{eq:sigma_bulk} may be expressed using \ref{eq:f_exp}, it gives
\begin{align}
    \label{eq:exp_e2}
    \partial_k (\phi_d \textbf{e}_d)_{ik} 
    &=  \partial_k\pSavg{ (\textbf e_d^0)_{ik} }
        -\frac{1}{2} \partial_k\partial_j \pSavg{ r_j (\textbf e_d^0)_{ik} +r_k (\textbf e_d^0)_{ij} }
        + \ldots  \\
    \label{eq:exp_sigma22}
    \partial_k (\phi_d \bm\sigma_d)_{ik}
    &=  \partial_k\pOavg{ (\bm\sigma_d^0)_{ik}}
    -\frac{1}{2} \partial_k\partial_j
    \pOavg{ r_j(\bm\sigma^0_d)_{ik} + r_k(\bm\sigma^0_d)_{ij}}
    + \ldots  \\
    \label{eq:exp_sigmaI2}
    \partial_k (\phi_\Gamma \bm\sigma_I)_{ik} 
    &=  \partial_k\pSavg{ (\bm\sigma_I^0)_{ik} }
        -\frac{1}{2} \partial_k\partial_j \pSavg{ r_j (\bm\sigma_I^0)_{ik} +r_k (\bm\sigma_I^0)_{ij} }
        + \ldots  
\end{align}
Note that we reformulated the second terms on the right-hand side of \ref{eq:exp_e2},\ref{eq:exp_sigma22} and \ref{eq:exp_sigmaI2} by retaining only their symmetric part since these terms must remain symmetric in the index $k$,$j$ due to the double contraction with the operator $\partial_k\partial_j$. 
Now we can use the first moment of momentum conservation \eqref{eq:dt_S_alpha} and the second moment of momentum equation (derived in \ref{ap:Moments_equations}, see \eqref{eq:second_momoent_of_momentum}) to reformulate the particle internal stresses, this yields,  
\begin{multline}
    \intS{ (\bm{\sigma}_\Gamma^0)_{ik}}
    +\intO{ (\bm{\sigma}_d^0)_{ik}}
    = 
    \intO{ \rho_d 
    (\textbf{w}_d^0\textbf{w}_d^0  )_{ik}
    }
    -\frac{1}{2}\left(\frac{d^2 \textbf{M}_\alpha}{dt^2} \right)_{ik}\\
    +\frac{1}{2}\intO{ \left[
        (\textbf{b}_d^0)_i
        r_k 
        + (\textbf{b}_d^0)_k
        r_i
    \right]}
    +
    \frac{1}{2}\intS{ \left[
        (\bm{\sigma}_1^0 \cdot \textbf{n}_d)_i r_k
        + (\bm{\sigma}_1^0 \cdot \textbf{n}_d)_k r_i
    \right]
    }
    \label{eq:dt_P1_alpha_bis}
\end{multline}
\begin{multline}
    \intO{ r_{j}(\bm{\sigma}^0_d)_{ik}+r_{k}(\bm{\sigma}^0_d)_{ji}}
    +\intS{ r_{j}(\bm{\sigma}^0_I)_{ik}+r_{k}(\bm{\sigma}_\Gamma^0)_{ji}}
    = 
    - \ddt\intO{ \rho_d (\textbf{u}_d^0)_i r_j r_k }
    \\
    + \intO{ \left[
        \rho_d (\textbf{u}^0_d\textbf{r}\textbf{w}_d^0)_{ijk} + \rho_d (\textbf{u}^0_d\textbf{r}\textbf{w}_d^0)_{kji}
    \right]}
    +\intS{  r_{k}r_{j} (\bm{\sigma}_1^0\cdot\textbf{n}_d)_i }
    + \intO{ r_{k}r_{j}  \rho_d (\textbf{b}_d^0)_i } 
    \label{eq:dt_P2_alpha_bis}
\end{multline}
By using an arbitrary order of moment of momentum equation (derived in \ref{ap:Moments_equations}) one can substitute any volume integral of the particle stress appearing in the expansion \ref{eq:exp_sigma22} into particles' kinematic properties plus the hydrodynamic moment of the fluid phase. 
Substituting \ref{eq:dt_P2_alpha_bis} and \ref{eq:dt_P1_alpha_bis} into \ref{eq:sigma_bulk} yields, 
\begin{multline}
    (\bm{\sigma}^\text{eq}_m)_{ik}
    = 
    \underbrace{
        % \left[
        \phi_f p_f 
        % + \frac{1}{3}\pOavg{\textbf{r}\cdot\bm{\sigma}_f^0 \cdot \textbf{n}_d} 
    % \right]
    \delta_{ik}
    - \mu_f e_{ik} 
    }_\text{Newtonian contribution}
    + \underbrace{\avg{\rho^0 \textbf{u}'_m\textbf{u}'_m}_{ik}}_\text{Bulk Reynolds stress}
    % + \mu_f \phi_2 e_{2,ik}. 
    + \underbrace{\epsilon_{ikj} \frac{1}{2}\pSavg{ (\textbf{b}_d^0 \times \textbf{r})_j}}_\text{Particles body torque}\\
    - \frac{1}{2}\underbrace{\pSavg{\left[
        (\bm{\sigma}_f^0 \cdot \textbf{n}_d)_kr_i  
        + (\bm{\sigma}_f^0 \cdot \textbf{n}_d)_i r_k
        % - \frac{2}{3}(\textbf{r}\cdot\bm{\sigma}_f^0 \cdot \textbf{n}_d)\delta_{ik}
    \right]}
    + 2 \mu_f \pOavg{(\textbf{e}_d^0)_{ik}}}_\text{Averaged Stresslet}\\
    - \underbrace{\pOavg{ \rho_d (\textbf{w}_d^0\textbf{w}_d^0  )_{ik}}
    + \pavg{\frac{d^2 \textbf{M}_{ik}}{dt^2}  }}_\text{particles inertia}
    + \underbrace{\frac{1}{2} (\div\bm{\Sigma})_{ik}}_\text{Inhomogeneous stresses}
    \label{eq:stress_bulk_explicit}
\end{multline}
with,
\begin{multline}
    \bm{\Sigma}
    = 
    - \pavg{\ddt\intO{ \rho_d (\textbf{u}_d^0)_i r_j r_k }}
    + \pOavg{ 
        \rho_d \left[
        (\textbf{u}^0_d\textbf{r}\textbf{w}_d^0)_{ijk} +  (\textbf{u}^0_d\textbf{r}\textbf{w}_d^0)_{kji}
    \right]
    }\\
    +\pSavg{  (\bm{\sigma}_f^0 \cdot \textbf{n}_d)_i r_{j}  r_{k}  }
    - 2 \mu_f \pOavg{[( \textbf{e}_d^0 \textbf{r})_{ikj}
    + ( \textbf{e}_d^0 \textbf{r})_{ijk}]} + \div[\ldots]
    \label{eq:Sigma_inhomo}
\end{multline}
We can note that the body force terms present in \ref{eq:sigma_bulk} and \ref{eq:dt_P2_alpha_bis} canceled each other in \ref{eq:Sigma_inhomo}. 
Under this form the different contributions to the suspension stress are explicit. 
The first contribution is the averaged continuous phase Newtonian stress. 
The second term is the  contribution from the total phase fluctuation to the suspension stress, including the particle internal fluctuation. 
The third term is the skew-symmetric part of the first moment of the body forces, in this study $\textbf{b}^0 = \rho^0 \textbf{g}$ thus this term vanishes for any kind of particles. 
The symmetric part of the hydrodynamic stresses together with the particle's internal shear rate form what is called: the Stresslet \citep{pozrikidis1992boundary}. 
Especially, we see in the next section that this term is related to the Einstein equivalent viscosity (see \ref{eq:fluid_phase_stress}). 
The first two terms on the third line of \ref{eq:stress_bulk_explicit} are the particle inertia contribution to the suspension stresses. 
This includes the second-order derivative of the particle shape represented by $\textbf{M}_\alpha$ and the internal particle motions. 
As discussed in the previous section this term is solely related to the inertia induced by change of orientation or deformation of the particles. 
Finally, the last term of the expression represents the contribution to the stress from all the higher moments related to the particles, it includes the moments related to the internal velocity or external hydrodynamic forces.  
Notice that since this term appears under the divergence operator it is non-zero only for Inhomogeneous suspension, therefore we call it the \textit{Inhomogeneous stresses}. 
The formulation of the bulk stress given by \ref{eq:stress_bulk_explicit} is formally equivalent to equation (8) and (12) of \citet{lhuillier1996contribution} and equation (8.2) of \citet{zhang1997momentum}.
The equivalence with \citet{zhang1997momentum} is not obvious as they use different decompositions for most of the terms. 
Although the relation \ref{eq:stress_bulk_explicit} is not new, the derivation presented here is original since the physical significance of all terms is explicit because it is derived in terms of particle-averaged quantities.

    
% \subsection{Additional simplifications}


% This inconsistency is easily fixed by noticing that, $\rho \textbf{u}_m = \sum_k \rho_k \phi_k \textbf{u}_k$, which implies that, 
% \begin{equation}
%     \textbf{u}=\rho \textbf{u}_m\sum_k \frac{1}{\rho_k \phi_k}
% \end{equation}
% Thus, we can recover the velocity field \textbf{u} needed in the expression of \textbf{e} through $\textbf{u}_m$ and the volume fractions. 

% The only term in \ref{eq:stress_bulk_explicit} that is not expressed in terms of particle averaged quantities is the \textit{Reynolds stress} tensor, $\avg{\rho^0 \textbf{u}'_m\textbf{u}'_m}$. 
% Indeed, the local field $\textbf{u}_m' = \textbf{u}^0 - \textbf{u}_m$ involves the local velocity fields within the particle.
% From an experimental point of view, this quantity is difficult if not impossible to quantify.  
% Therefore, it is interesting to reformulate this \textit{bulk Reynolds stress} in terms of continuous fluid phase averaged Reynolds stress and a stress related to particle averaged properties. 
% This is easily done by noticing that, 
% \begin{equation*}
%     \avg{\rho^0 \textbf{u}'_m\textbf{u}'_m}
%     = 
%     \avg{\chi_f \rho_f \textbf{u}_f'\textbf{u}_f'}
%     + \phi_f \rho_f \textbf{u}_f\textbf{u}_f
%     - \rho \textbf{u}_m\textbf{u}_m
%     + \avg{\chi_d \rho_d \textbf{u}_d^0\textbf{u}_d^0}
%     \label{eq:u_mu_m}
% \end{equation*}
% We can notice that $\avg{\chi_f \rho_f \textbf{u}_f'\textbf{u}_f'}$ is the fluid phase Reynolds stress introduced in \ref{eq:dt_avg_rhou_k}. 
% Regarding the last term of \ref{eq:u_mu_m} it can be reformulated as, 
% \begin{equation*}
%     \avg{\chi_d \rho_d \textbf{u}_d^0\textbf{u}_d^0}
%     = 
%     \pavg{ m_\alpha \textbf{u}_\alpha' \textbf{u}_\alpha' }
%     + n_p m_p \textbf{u}_p\textbf{u}_p
%     + \pOavg{\rho_d \textbf{w}_d^0 \textbf{w}_d^0}
%     - \div [\ldots]
% \end{equation*}
% where we neglected the higher moments terms. 
% Consequently, in \ref{eq:stress_bulk_explicit}, one can use the relation 
% \begin{equation*}
%     \avg{\rho^0 \textbf{u}'_m\textbf{u}'_m}
%     - \pOavg{\rho_d \textbf{w}_d^0 \textbf{w}_d^0}
%     \approx 
%     \avg{\chi_f \rho_f \textbf{u}_f'\textbf{u}_f'}
%     + \pavg{ m_\alpha \textbf{u}_\alpha'\textbf{u}_\alpha'}
%     + \phi_f \rho_f \textbf{u}_f\textbf{u}_f
%     + n_p m_p \textbf{u}_p\textbf{u}_p
%     - \rho \textbf{u}_m\textbf{u}_m
% \end{equation*}
% to simplify the formulas. 
% Using this formulation in \ref{eq:stress_bulk_explicit} one notices that the only quantity needed to express the bulk stress is the particle phase averaged quantities and $\textbf{u}_m$. 

\subsection{The symmetry of the bulk stress}

We now discuss the symmetry properties of $\bm\sigma^\text{eq}_m$ as it has been a subject of controversy over the past 20 years.
\citet{prosperetti2006stress} found that the equivalent stress of a non-inertial suspension of spherical particles possess a skew-symmetric contribution (excluding the torque or the body torque term). 
However, more recently \citet{zhou2020lamb} and \citet{dolata2020heterogeneous} claimed that the only skew-symmetric part present in the equivalent stress tensor \eqref{eq:stress_bulk_explicit} is the body torque term $\intO{\textbf{b}_d^0 \times \textbf{r}}$ in opposition to the conclusion of \citep{prosperetti2006stress}.
In support to the conclusion of  \citet{dolata2020heterogeneous} limited to non-inertial suspension, we propose here to revisit their proof extended to inertial suspension. 
\citet{lhuillier1996contribution} reached the same conclusion as \citet{dolata2020heterogeneous} but years before them, i.e. he concluded that the only skew-symmetric part present in the equivalent stress tensor is the body torque. 
Since it seems that the article by \citet{lhuillier1996contribution} went unnoticed we propose here to re-demonstrate his proof as this will introduce useful relations for the next section. 

From \ref{eq:sigma_bulk} it is clear that a skew-symmetric contribution arises from the body torque term $\intO{\textbf{b}_d^0 \times \textbf{r}}$. 
On the other hand, another contribution might arise due to the inhomogeneous stress tensor $\div \bm\Sigma_1$. 
So the challenge here is to prove that  $\div \bm\Sigma_1$ is either symmetric or skew-symmetric. 
Notice that due to the higher moments present in $\Sigma_{ijk}$ this tensor is arbitrary. 
Additionally, in the bulk moment of momentum conservation \eqref{eq:momentum_bulk} the tensor $\Sigma_{ijk}$ appears under the operator $\partial_j \partial_k$, therefore only the term $\partial_j \partial_k \Sigma_{ijk}$ is of physical significance. 
Thus, one can demonstrate that \citep{lhuillier1996contribution}
\begin{equation}
    \partial_j \partial_k \Sigma_{ijk}
    = \partial_j \partial_k \Sigma_{i(jk)}
    =
    \partial_j \partial_k \left[
        \Sigma_{i(jk)}
        + \Sigma_{j(ik)}
        - \Sigma_{k(ij)}
    \right],
    \label{eq:sym_proof}
\end{equation}
where $\Sigma_{i(jk)} = \frac{1}{2}[\Sigma_{ijk} + \Sigma_{ikj}]$ represents the symmetric part of $\Sigma_{ijk}$ over the index $jk$, as indicated by the parenthesis. 
The first equality is made possible by noting that $\partial_j \partial_k [\Sigma_{ijk} - \Sigma_{ikj}] = 0$.
The second equality is obtained by adding and subtracting $\partial_j \partial_k \Sigma_{j(ik)}$ from the expression and recognizing that $\partial_j \partial_k \Sigma_{j(ik)} = \partial_j \partial_k \Sigma_{k(ij)}$ since $j$ and $k$ are dummy indices. 
In \ref{eq:sym_proof} the first two terms $\partial_k(\Sigma_{i(jk)} + \Sigma_{j(ik)})$ form a symmetric stress tensor over the index $i$ and $j$, while the last term $\partial_k\Sigma_{k(ij)}$ is symmetric over $i$ and $j$ as indicated by the parenthesis. 
Thus, according to \ref{eq:sym_proof} only the symmetric part of $\bm\Sigma$ is significant when applying the double divergence operator $\grad\grad$. 
Therefore, while $\bm\Sigma$ might possess a skew-symmetric part, it vanishes under the application of the double divergence operator. 
Consequently, in the momentum equation only the symmetric part of the second and higher moments of body force are of physical significance.
Note that we could use \ref{eq:sym_proof} to reformulate $\div\bm\Sigma$ in \ref{eq:stress_bulk_explicit} to write explicitly the symmetric contribution. 
Therefore, for the situation where there is no local body torque, i.e. with symmetric local stresses,  the only contribution to the skew-symmetric part of the bulk stress is the body torque.
Thus, in the situation where the only body force is gravity, the bulk stress is symmetric even for inertial non-homogeneous suspension of arbitrary particles. 

Our conclusion might be in contradiction with the recent study of \cite{wolgemuth2023continuum}.
Indeed, they demonstrated that the effective stress tensor of a suspension of solid spherical particles could possess a skew-symmetric part. 
Among the terms in the effective stress (Equation (2.18) of \cite{wolgemuth2023continuum}) they obtained a term of the form $\grad \textbf{F}_0$ where they defined $\textbf{F}_0$ as the averaged body force on the particles. 
Still with our notation the effective stress of \citet{wolgemuth2023continuum} might be written, 
\begin{equation*}
    (\bm\sigma^\text{eq}_m)_{ij}
    = \partial_j(\phi F_i),
\end{equation*}
which can be reformulated as 
\begin{equation*}
    (\bm\sigma^\text{eq}_m)_{ij}
    = \partial_k(\phi F_i \delta_{jk}). 
\end{equation*}
Therefore, according to \ref{eq:sym_proof} the skew-symmetric part of $\partial_k(\phi F_i \delta_{jk})$ does not play a role in the momentum equation due to the divergence operator. 
Note that the term $\grad \textbf{F}$ in \citet{wolgemuth2023continuum} correspond to the second moment of the hydrodynamic force traction on the particle surface present in \ref{eq:Sigma_inhomo}. 



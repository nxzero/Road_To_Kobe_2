\section{Singularity solution for droplets with surface tension gradient}
\label{ap:singularity_solution}
At the leading order in droplet volume fraction, the closure problem is equivalent to that of an isolated droplet in an infinite medium \citet{hinch1977averaged}. 
Hence, we consider the problem of an isolated droplet, immersed in an arbitrary quadratic flow without the presence of fluid inertia. 
The disturbances pressure and velocity field are noted $\textbf{u}_{out}$, $\textbf{u}_{in}$, $p_{out}$ and $p_{in}$, for the velocity outside the droplet, the velocity inside the droplet, the pressure outside the droplet and the pressure inside the droplet, respectively. 
The corresponding ``undisturbed field'' are noted $\textbf{u}$ and $p_f$, corresponding to the ensemble averaged mixture velocity and averaged pressure of the continuous phase in the main text. 

Under these hypotheses, in dimensionless form, $\textbf{u}_{out}$, $\textbf{u}_{in}$, $p_i$ and $p_o$ are governed by the Stokes equations, namely, 
\begin{align}
    \div \textbf{u}_{in} &= 0 
    & \div \textbf{u}_{out} &= 0 \\
     \grad^2 \textbf{u}_{in}  &= \grad p_{in} 
    & \grad^2 \textbf{u}_{out} &= \grad p_{out} 
\end{align}
At the surface of the droplet ($r=1$) the condition of continuity of velocity and continuity of the normal stresses read, 
\begin{align}
    \label{eq:vel_jump}
    \textbf{u}_{out} - \textbf{u}_{in} &= 0\\
    % (\textbf{u}_{in}  + \textbf{u} - \textbf{u}_\Gamma)\cdot \textbf{n}
    % = 
    \label{eq:vel_jump2}
    (\textbf{u}_{in}  + \textbf{u}_r)\cdot \textbf{n}
    &= 0\\
    \mathbf{n}\cdot (\textbf{e}_{out} - \lambda \textbf{e}_{in}+\textbf{E} -\lambda\textbf{E} )\cdot (\bm\delta - \textbf{nn})
    &= 
    \textbf{b}\cdot (\bm\delta - \textbf{nn})
    \label{eq:stress_jump}
\end{align}
where $\textbf{u}_r = \textbf{u} - \textbf{u}_p$ corresponds to the velocity of the mixture with respect to the droplet mean velocity. 
We recall that $\textbf{e}_{in/out} = (\grad \textbf{u}_{in/out} + ^\dagger \grad \textbf{u}_{in/out})/2 $ and $\textbf{E} = (\grad \textbf{u} + ^\dagger \grad \textbf{u})/2$, which correspond to the rate of strain inside or around or droplet, and to the ensemble averaged rate of strain, respectively. 
In \ref{eq:stress_jump} the vector \textbf{b} correspond to the dimensionless tangential stress jump across the interface and reads, 
\begin{equation}
    \textbf{b}
    =
    \frac{a \grad \gamma}{2 \mu_f U}
    % \approx (a\grad) Ca^{-1}. 
\end{equation}
Far from the droplet (centered at $\textbf{r}=\textbf{0}$) the velocity and pressure fields satisfy, 
\begin{equation}
    \lim_{r\to \infty}(\textbf{u}_{out},p_{out}) = 0. 
\end{equation}

In general $\textbf{b}$ as well as $\textbf{u}$ are not constant vectors and vary across space as a function of \textbf{r}. 
However, it can be assumed that these quantities are slowly varying functions of space at the scale of a droplet radius, hence we may introduce the simplifications 
\begin{align*}
    \textbf{u}(\textbf{n}) 
    &=  \textbf{u}_r|_{\textbf{r}=0}
    +  \textbf{r} \cdot  \grad\textbf{u}|_{\textbf{r}=0}
    +  \frac{1}{2}\textbf{rr} \cdot  \grad\grad\textbf{u}|_{\textbf{r}=0}
    + \ldots\\
     \textbf{E}(\textbf{n}) 
    &=   \textbf{E}|_{\textbf{r}=0}
    + \textbf{r} \cdot  \grad \textbf{E}|_{\textbf{r}=0}
    + \frac{1}{2}\textbf{rr} \cdot  \grad\grad \textbf{E}|_{\textbf{r}=0}
    + \ldots\\
     \textbf{b}(\textbf{n}) 
    &=   \textbf{b}|_{\textbf{r}=0}
    + \textbf{r} \cdot  \grad \textbf{b}|_{\textbf{r}=0}
    + \frac{1}{2}\textbf{rr} \cdot  \grad\grad \textbf{b}|_{\textbf{r}=0}
    + \ldots
\end{align*}
Note that because $\textbf{b} \sim \grad \gamma$ the tensor: $\grad \textbf{b}$ and $\grad\grad \textbf{b}$ are symmetric tensors. 
Injecting, these expressions into \ref{eq:vel_jump,eq:vel_jump2,eq:stress_jump} one deduce that the velocity and pressure fields are linearly related to $\textbf{u}|_{\textbf{r}=0}$, $\textbf{b}|_{\textbf{r}=0}$, and to the gradients of these vectors. 

According to the linearity of the Stokes equations we deduce that $\textbf{u}_{i/o}$ and $p_{i/o}$  must be linear combination of spherical harmonics proportional to $\textbf{u}|_{\textbf{x}=0}$, $\grad \gamma|_{\textbf{x}=0}$, and their derivatives.
Hence, we can write that the disturbance velocity and pressure fields are given by, 
\begin{align*}
    \begin{pmatrix}
        \textbf{u}_{o}\\
        p_{o}\\
        \textbf{u}_{i}\\
        p_{i}
    \end{pmatrix}
    =
    \begin{pmatrix}
        \textbf{U}_{o}^{(1)} + \textbf{U}_{o}^{(2)}\cdot \grad + \textbf{U}_{o}^{(3)} :\grad\grad &
        \textbf{U}_{o}^\text{(b-1)} + \textbf{U}_{o}^\text{(b-2)}\cdot \grad + \textbf{U}_{o}^\text{(b-3)} :\grad\grad \\
        \textbf{P}_{o}^{(1)} + \textbf{P}_{o}^{(2)}\cdot \grad + \textbf{P}_{o}^{(3)} :\grad\grad &
        \textbf{P}_{o}^\text{(b-1)} + \textbf{P}_{o}^\text{(b-2)}\cdot \grad + \textbf{P}_{o}^\text{(b-3)} :\grad\grad \\
        \textbf{U}_{i}^{(1)} + \textbf{U}_{i}^{(2)}\cdot \grad + \textbf{U}_{i}^{(3)} :\grad\grad &
        \textbf{U}_{i}^\text{(b-1)} + \textbf{U}_{i}^\text{(b-2)}\cdot \grad + \textbf{U}_{i}^\text{(b-3)} :\grad\grad \\
        \textbf{P}_{i}^{(1)} + \textbf{P}_{i}^{(2)}\cdot \grad + \textbf{P}_{i}^{(3)} :\grad\grad &
        \textbf{P}_{i}^\text{(b-1)} + \textbf{P}_{i}^\text{(b-2)}\cdot \grad + \textbf{P}_{i}^\text{(b-3)} :\grad\grad \\
    \end{pmatrix}
    \cdot 
    \begin{pmatrix}
        \textbf{u}_r\\
        \textbf{b}
    \end{pmatrix}
\end{align*}
The tensors $\textbf{U}^{(n)}_{o/i}$ (resp. $\textbf{P}^{(n)}_{o/i}$) are $(n+1)^{th}$ (resp. $n^{th}$) order tensors, solely function of \textbf{r} and the viscosity ratio $\lambda$. 
The tensors $\textbf{U}_{o/i}^{(1)}$ and $\textbf{P}_{o/i}^{(1)}$ represent the functional form of the famous Hadamard-Ribczynski solution \citep{pozrikidis1992boundary,kim2013microhydrodynamics}. 
The dependency of $\textbf{u}_{i/o}$ and $p_{i/o}$ with the mean gradient velocity $\grad \textbf{u}$ is given by the tensor $\textbf{U}_{o/i}^{(2)}$ and $\textbf{P}_{o}^{(2)}$.
The disturbance velocity and pressure fields of a drop immersed in pure linear flow can be found in \citet{rallison1978note,leal2007advanced,raja2010inertial}. 
Finally, the disturbance field of a droplet immersed in a pure quadratic flow is defined through the tensor $\textbf{U}_{o/i}^{(3)}$ and $\textbf{P}_{o/i}^{(3)}$.
The definition of the former tensors may be found in \citet{nadim1991motion}.




The tensors $\textbf{U}^{(b-n)}_{o/i}$ (resp. $\textbf{P}^{(b-n)}_{o/i}$) are $(n+1)^{th}$ (resp. $n^{th}$) order tensors, solely function of \textbf{r} and the viscosity ratio $\lambda$ as well. 
The disturbance fields induced due to a constant, linear, or quadratic stress jump at the interface of the droplet is less common in the literature. 
For instance one might refer to \citet{Subramanian_1985,leal2007advanced} to compute the disturbance fields due to a constant surface tension gradient, hence defining the expression of  $\textbf{U}_{o/i}^{(b-1)}$ and $\textbf{P}_{o/i}^{(b-1)}$. 
Then to express the disturbance fields in terms of $\grad \textbf{b}$ or higher order derivatives one might refer to the solution given in \citet[Appendix C]{raja2010inertial}.
Without loss of generality we consider only the symmetric traceless part of $(\grad \textbf{b}_{jk})$ in the following. 

% Based on the classic method of spherical harmonic solution \citep[chapter 8]{leal2007advanced} we could derive the expression for the disturbance fields due to an imposed $\grad \textbf{b}$, it reads,
% \begin{align}
%     (\textbf{U}_o^{(b-2)})_{ijk}
%     &=
%     -\frac{1}{10(\lambda+1)r^5}(\delta_{ij} x_k+\delta_{ik} x_j)
%     -\frac{3r^2-5}{10(\lambda+1)r^7}x_ix_jx_k
%     +\frac{r^2-1}{10(\lambda+1)r^5}\delta_{jk}x_i\\
%     (\textbf{P}_o^{(b-2)})_{jk}
%     &=
%     -\frac{3}{5(\lambda+1)r^5}x_jx_k
%     +\frac{1}{5(\lambda+1)r^3}\delta_{jk}\\
%     (\textbf{U}_i^{(b-2)})_{ijk}
%     &=
%     -\frac{5r^2-3}{20(\lambda+1)}(\delta_{ij} x_k+\delta_{ik} x_j)
%     +\frac{1}{5(\lambda+1)}x_ix_jx_k
%     +\frac{r^2-1}{10(\lambda+1)}\delta_{jk}x_i\\
%     (\textbf{P}_i^{(b-2)})_{jk}
%     &=
%     -\frac{21}{10(\lambda+1)}x_jx_k
%     +\frac{7r^2}{10(\lambda+1)r^3}\delta_{jk}
% \end{align}

Hence, based on these expressions and the solutions  provided in the literature, one can compute the force traction at the surface of the droplet ($\bm\sigma_{out}\cdot \textbf{n}$), as well as the internal shear rate ($\textbf{e}_{in}$), and obtains the following expressions, 
\begin{align}
    \intS{\bm\sigma_{out}\cdot \textbf{n}} &
    =
    2\pi\frac{2+3\lambda}{1+\lambda}\textbf{u}_r
    + \pi \frac{\lambda}{\lambda +1} \grad^2 \textbf{u}
    +
    \frac{4\pi}{3}\frac{1}{\lambda +1}\textbf{b}
    + \frac{2\pi}{15(\lambda +1)}\grad^2\textbf{b}
    \\
    \intS{ \textbf{x}\bm\sigma_{out}\cdot \textbf{n}} &
    =
    \frac{2\pi(5\lambda +2)}{5(\lambda +1)}[\grad \textbf{u}+ (\grad \textbf{u})^\dagger]
    + \frac{12\pi}{25(\lambda +1)} \grad \textbf{b} 
    \\
    \intS{\textbf{xx}\bm\sigma_{out}\cdot \textbf{n}} &
    =
    \frac{4\pi}{5(\lambda +1)} (\textbf{u}_r \bm\delta + ^\dagger\textbf{u}_r \bm\delta)
    + \frac{2\pi(5\lambda +2)}{5(\lambda+1)}\bm\delta \textbf{u}_r
    - \frac{8\pi}{15(\lambda +1)} ( \textbf{b}\bm\delta + ^\dagger\textbf{b}\bm\delta )
    + \frac{4\pi}{5(\lambda+1)}\bm\delta \textbf{b}
\end{align}
and, 
\begin{align}
    \intO{2\textbf{e}_{in}}
    &=
    -\frac{4\pi}{15}\frac{5\lambda +2}{\lambda+1}
    [\grad \textbf{u}+ (\grad \textbf{u})^\dagger]
    - \frac{8\pi}{25}\frac{1}{\lambda+1}
    \grad\textbf{b}
    % + \frac{8\pi}{75}\frac{1}{\lambda+1}
    % \bm\delta\div \textbf{b}
    \\
    \intO{2\textbf{r}\textbf{e}_{in}}
    &=
    \frac{2\pi}{5(\lambda+1)}
    (\bm\delta \textbf{u}_r +  \bm\delta\textbf{u}_r^\dagger)
    -\frac{4\pi}{15(\lambda+1)}\textbf{u}_r\bm\delta 
    -\frac{4\pi}{15}\frac{1}{\lambda+1}
    (\bm\delta \textbf{b} +  \bm\delta\textbf{b}^\dagger)
    +\frac{8\pi}{45}\frac{1}{\lambda+1}
    \textbf{b}\bm\delta. 
\end{align}

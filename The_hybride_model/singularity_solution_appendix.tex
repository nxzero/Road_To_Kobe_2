\section{Singularity solution for droplets with non-constant surface tension gradient}
\label{ap:singularity_solution}
At the leading order in droplet volume fraction, the closure problem is equivalent to that of an isolated droplet in an infinite medium \citet{hinch1977averaged}. 
Hence, we consider the problem of an isolated droplet, immersed in an arbitrary quadratic flow without the presence of fluid inertia. 
The disturbances pressure and velocity field are noted $\textbf{u}_{out}$, $\textbf{u}_{in}$, $p_{out}$ and $p_{in}$, for the velocity outside the droplet, the velocity inside the droplet, the pressure outside the droplet and the pressure inside the droplet, respectively. 
The corresponding ``undisturbed field'' are noted $\textbf{u}$ and $p_f$, corresponding to the ensemble averaged mixture velocity and averaged pressure of the continuous phase in the main text. 


Under these hypotheses, in dimensionless form, $\textbf{u}_{out}$, $\textbf{u}_{in}$, $p_i$ and $p_o$ are governed by the Stokes equations, namely, 
\begin{align}
    \div \textbf{u}_{in} &= 0 
    & \div \textbf{u}_{out} &= 0 \\
     \grad^2 \textbf{u}_{in}  &= \grad p_{in} 
    & \grad^2 \textbf{u}_{out} &= \grad p_{out} 
\end{align}
At the surface of the droplet ($r=1$) the continuity of velocity and the continuity of the normal stresses imposes, 
\begin{align}
    \textbf{u}_{out} - \textbf{u}_{in} &= 0\\
    % (\textbf{u}_{in}  + \textbf{u} - \textbf{u}_\Gamma)\cdot \textbf{n}
    % = 
    (\textbf{u}_{in}  + \textbf{u}_r)\cdot \textbf{n}
    &= 0\\
    \mathbf{n}\cdot (\textbf{e}_{out} - \lambda \textbf{e}_{in}+\textbf{E} -\lambda\textbf{E} - \textbf{b})\cdot (\bm\delta - \textbf{nn})
    &= 0 %(\bm\delta - \textbf{nn})\cdot 
\end{align}
where $\textbf{u}_r = \textbf{u} - \textbf{u}_p$ corresponds to the velocity of the mixture with respect to the droplet mean velocity. 
We recall that $\textbf{e}_{in/out} = (\grad \textbf{u}_{in/out} + ^\dagger \grad \textbf{u}_{in/out})/2 $ and $\textbf{E} = (\grad \textbf{u} + ^\dagger \grad \textbf{u})/2$, which correspond to the rate of strain inside or around or droplet, and to the ensemble averaged rate of strain, respectively. 
Far from the center of the test droplet (at $\textbf{r}=0$) the velocity and pressure fields satisfy, 
\begin{equation}
    \lim_{r\to \infty}(\textbf{u}_{out},p_{out}) = 0. 
\end{equation}

Let us introduce the dimensionless stress tensor $\bm\sigma_{in/out} = -p_{in/out} \bm\delta + \mu_{in/out}[\grad \textbf{u}_{in/out} + ^\dagger \grad \textbf{u}_{in/out}]$ and the dimensionless shear stress tensor, $2\textbf{e}_{in/out} = \grad \textbf{u}_{in/out} + ^\dagger \grad \textbf{u}_{in/out}$. 
The undisturbed shear stress reads  $2\textbf{e} = \grad \textbf{u} + ^\dagger \grad \textbf{u}$
With these notations the tangential shear stress jump reads at $r = 1$ reads as, 

with, 
\begin{equation}
    \textbf{b}
    =
    \frac{\grad \gamma}{2\mu_f}
    % \approx (a\grad) Ca^{-1}. 
    % =
    % \frac{a \grad \gamma}{2 \mu_f |\textbf{u}_r|}
    % \approx (a\grad) Ca^{-1}. 
\end{equation}
The velocity \textbf{u} is imposed, the velocity $\textbf{u}_\Gamma$ is just the velocity of the interface which may be obtained by differetiating the point lying on the interface, namely, 
\begin{equation}
    \textbf{u}_\Gamma
    =
    \textbf{u}_\alpha
    +
    \pddt \textbf{r}_\Gamma
    =
    \textbf{u}_\alpha
    + 
    \textbf{e}_r\pddt f(\theta,\varphi,t)
\end{equation}

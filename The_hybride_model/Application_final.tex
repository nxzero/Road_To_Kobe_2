
\subsection{Application to spherical droplets}


\begin{itemize}
    \item The simplified moments' equation.
    \item The simplified internal energy due to fluctuation ? 
    \item The simplified dissipation term 
    \item Closure terms (Drag stresslet etc...)
\end{itemize}

\subsubsection{The Reynolds stress due to wake}
The fluctuation term $\avg{\chi_1 \textbf{u}_1'\textbf{u}_1'}$ can be reformulated. 
It must have an isotropic component and a component slightly higher in teh direction of the relative velocity.  
First of all the trace can be written down with, 
\begin{equation*}
    k_1=
    \frac{1}{2}\textbf{I}:\oneavg{\textbf{u}_1'\textbf{u}_1'}
\end{equation*}
The deviatoric part is then, 
\begin{equation*}
    (\oneavg{\textbf{u}_1'\textbf{u}_1'})^\text{dev}=
    \oneavg{\textbf{u}_1'\textbf{u}_1'}
    - \frac{2}{3}k_1\textbf{I}
\end{equation*}
If we assume the form, 
\begin{equation*}
    \oneavg{\textbf{u}_1'\textbf{u}_1'}
    = 
    \textbf{U}
    \textbf{U}
    k^{||}_1
    + 
    \left[
        \textbf{I} (\textbf{U}\cdot \textbf{U})
    -
    \textbf{U}
    \textbf{U}
    \right]
    k^{\bot}_1
    = 
    \textbf{U}
    \textbf{U}
    (k^{||}_1 - k^{\bot}_1)
    + \textbf{I} (\textbf{U}\cdot\textbf{U}) k^\bot_1
\end{equation*}
where $\textbf{U} = (\textbf{u}_p - \textbf{u}_f)$. 
Since the velocity fluctuation is higher in the flow direction $k^\bot_1 < k_1 < k^{||}_1$. 
The last expression is the one usually used. 
However, we would like to see appear $k_1$ in this expression since we solve an equation for $k_1$.  
The ultimate goal is to factor out $k_1$ from the above expression thus notice that :
\begin{align*}
    k_1 =
    \frac{1}{2} \oneavg{\textbf{u}_1'\textbf{u}_1'} : \textbf{I}
    &= 
    \frac{1}{2} \textbf{U}
    \textbf{U}
    (k^{||}_1 - k^{\bot}_1) : \textbf{I}
    + \frac{1}{2} \textbf{I} (\textbf{U}\cdot\textbf{U}) k^\bot_{||} : \textbf{I}\\
    &= 
    \frac{1}{2} 
    \textbf{U}\cdot 
    \textbf{U}
    (k^{||}_1 - k^{\bot}_1)
    + \frac{3}{2}  (\textbf{U}\cdot\textbf{U}) k^\bot_1 \\
    &= 
    \frac{1}{2} 
    (\textbf{U}\cdot 
    \textbf{U})[
        k^{||}_1 
        + 2 k^\bot_1
    ]\\
    &= 
    (\textbf{U}\cdot 
    \textbf{U})[
        \frac{1}{2}k^{||}_1 
        + k^\bot_1
    ]
\end{align*}
Let's add and subtract $\frac{1}{2}\textbf{I}(\textbf{U}\cdot\textbf{U})(k^{||}_1)$ to the previous expression of the Reynolds stress, 
\begin{align*}
    \oneavg{\textbf{u}_1'\textbf{u}_1'}
    &= 
    \textbf{U}
    \textbf{U}
    (k^{||}_1 - k^{\bot}_1)
    + \textbf{I} 
    (\textbf{U}\cdot\textbf{U}) k^\bot_1
    + \frac{1}{2}\textbf{I} 
    (\textbf{U}\cdot\textbf{U}) (k^{||}_1)
    - \frac{1}{2}\textbf{I} 
    (\textbf{U}\cdot\textbf{U}) (k^{||}_1)\\
    &= 
    \textbf{U}
    \textbf{U}
    (k^{||}_1 - k^{\bot}_1)
    - \frac{1}{2}\textbf{I} 
    (\textbf{U}\cdot\textbf{U}) k^{||}_1
    + \textbf{I} 
    (\textbf{U}\cdot\textbf{U}) k^*_1
    \\
\end{align*}
where $ k^*_1 = k_1 /(\textbf{U}\cdot \textbf{U})$ is the dimensionless granular temperature.  

To introduce a simpler notation we pose $k^{||}_1 = (C_1 +1/3) 2k_1^*$ and $k^{\bot}_1 = (C_2+1/3) 2k_1^*$ so that in the isotope scenario,  when $k^{||}_1 = \frac{2}{3} k^*_1$ or $k^{||}_1 =\frac{2}{3} k^*_1$ the constant $C_1$ and $C_2$ equal zero. 
\begin{align*}
    \oneavg{\textbf{u}_1'\textbf{u}_1'}
    &= 
    \textbf{U}
    \textbf{U}
    ((C_1 +1/3)2k_1^* -( C_2+1/3) 2 k_1^*)
    - \frac{1}{2}\textbf{I} 
    (\textbf{U}\cdot\textbf{U})  (C_1+1/3) 2k_1^*
    + \textbf{I} 
    (\textbf{U}\cdot\textbf{U}) k^*_1\\
    &= k^*_1 \left[
        \textbf{U}
        \textbf{U}
        (C_1  - C_2 )
        + \textbf{I} 
        (\textbf{U}\cdot\textbf{U})  (C_1+2/3) 
    \right]
\end{align*}
Notice that when, $C_1=C_2= 0$ we recover $\oneavg{\textbf{u}_1'\textbf{u}_1'}=\frac{2}{3} \textbf{I}(\textbf{U}\cdot \textbf{U}) k^*_1$

\subsection{The particle Reynolds stress for stoke stain dynamic particles}
\subsection{The equivalent stress}
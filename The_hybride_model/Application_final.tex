
\subsection{Application to spherical droplets}
In this section we consider the case of spherical buoyant droplets and observe how the internal motions due to the fluid nature of the particle phase impact the averaged equation. 
For simplicity, we consider that the droplets have a negligible angular velocity such that they internal motion are solely related to their center of mass velocities. 
Additionally, we consider spherical droplets in all circonstencies. 
It must be understood that we assume a hill vortex shape above the non-inertial regime \cite{dandy1989} which holds true for spherical particles. 
Consequently, the first moment of momentum and second moment of mass equation can be entirely disregarded. 


\subsubsection{The droplets internal motions}

Consider a droplet translating in a carrier fluid with uniform velocity.
The inner flow can be reasonably represented by hill vortex as it is a solution valid in the viscous regime as well as in the potential flow limit.  
However, the force traction on the interface of the droplets  is not assumed since we are not considering stokes or potential flow limit. 
From the singularity method it is possible to derive an analytical formula for the inner motion of the fluid inside the drop, see \citet[Chapter 7.]{pozrikidis1992boundary}. 
Indeed, the relative inner motion inside a spherical drop of viscosity ratio $\lambda = \mu_2/\mu_1$ translating inside a carrier fluid can be expressed as,
\begin{equation}
    \textbf{w}_2^0(\textbf{x}_\alpha + \textbf{r}) = 
    \frac{1}{2}\frac{1}{1+\lambda} \textbf{u}_{\alpha f}\cdot \left[
        (2\lambda+3)\textbf{I}
        - 2 \textbf{r}\cdot \textbf{r} \textbf{I}/a^2
        + \textbf{rr}/a^2
    \right] - \textbf{u}_{\alpha f}\\
    \label{eq:wa_def}
\end{equation}
with $\textbf{u}_{\alpha f} = \textbf{u}_\alpha - \textbf{u}_1|_\alpha$, we evaluate the averaged fluid velocity at the location of the instantaneous position of the particle center of mass. 
% \begin{multline*}
%     (\textbf{w}_2^0 \textbf{w}_2^0)_{ij}
%     = 
%     c^{2} U_i U_j 
%     + 4 c e r^{2} U_i U_j 
%     - c e U_i U_l x_j x_l 
%     - c e U_j U_k x_i x_k \\
%     + 4 e^{2} r^{4} U_i U_j 
%     - 2 e^{2} r^{2} U_i U_l x_j x_l 
%     - 2 e^{2} r^{2} U_j U_k x_i x_k 
%     + e^{2} U_k U_l x_i x_j x_k x_l
% \end{multline*}
% Noticing that, 
% \begin{align*}
%     \int_\text{Sphere} x_ix_j d\Omega = \frac{4\pi}{15}a^5 \delta_{ij},\\
%     \int_\text{Sphere} x_ix_jx_kx_l d\Omega = \frac{4\pi}{7* 15}a^7 
%     (\delta_{ij}\delta_{kl}+\delta_{ik}\delta_{jl}+\delta_{il}\delta_{jk}),\\
% \end{align*}
% \begin{multline*}
    % \intO{\textbf{w}_2^0 \textbf{w}_2^0} 
    % = 
    % c^{2} U_i U_j \frac{4 \pi}{3}  a^3
    % + 4 c e U_i U_j \frac{4 \pi}{5} a^5
    % - c e U_i U_l \delta_{jl} \frac{4 \pi}{15} a^5
    % - c e U_j U_k \delta_{ik} \frac{4 \pi}{15} a^5 \\
    % + 4 e^{2}  U_i U_j \frac{4\pi}{7} a^7 
    % - 2 e^{2}  U_i U_l \frac{4\pi}{7* 3} a^7\delta_{jl} 
    % - 2 e^{2}  U_j U_k  \frac{4\pi}{7* 3} a^7\delta_{ik} 
    % + e^{2} U_k U_l \frac{4\pi}{7* 15}a^7 
    % (\delta_{ij}\delta_{kl}+\delta_{ik}\delta_{jl}+\delta_{il}\delta_{jk})\\
    % = \frac{4 \pi}{3}  a^3 U_iU_j\left(
        % c^{2} 
        % + 10 c e \frac{ a^2}{5}
        % - c e  \frac{a^2}{5}
        % - c e  \frac{a^2}{5} 
        % + 42 e^{2} \frac{a^4}{7 * 5}
        % - 2 e^{2} \frac{a^4}{7} 
        % - 2 e^{2}  \frac{a^4}{7} 
        % + e^{2} \frac{a^4}{7* 5}a^7 
        % + e^{2} \frac{a^4}{7* 5}a^7 
%     \right)
%     + e^{2} U_k U_k \delta_{ij} \frac{4\pi}{7* 15}a^7 \\
% \end{multline*}
% Consideringt this expression for the velocity gives. 
% \begin{multline*}
%     \intO{\textbf{w}_2^0 \cdot \textbf{w}_2^0} 
%     % = \frac{4 \pi}{3}  a^3 U_kU_k\left(
%     %     c^{2} 
%     %     + 2 c e a^2
%     %     + 9 e^{2} \frac{a^4}{7}
%     % \right)
%     =\frac{v_\alpha}{14 \left(\lambda +1\right)^2} (\textbf{u}_\alpha - \textbf{u}_1)\cdot (\textbf{u}_\alpha - \textbf{u}_1)\\
% \end{multline*}
% At this stage we already see that the energy related to the internal motion are small compared to the macroscopic energy, At most $\sim 1/14$.  
% From that fact we can evaluate the internal mean kinetic energies yielding,
Notice that from this analytical expression, the internal kinetic energy of the particle can be directly computed as, 
\begin{align*}
    &n_p W_p 
    % =
    % \pOavg{\rho_2 \textbf{w}_2^0 \cdot \textbf{w}_2^0/2} 
    % =\frac{n_pm_p}{14 \left(\lambda +1\right)^2} \left[
    %     \frac{1}{2}(\textbf{u}_p - \textbf{u}_1)\cdot (\textbf{u}_p - \textbf{u}_1)
    %     +k_p
    % \right]
    =\frac{n_pm_p}{14 \left(\lambda +1\right)^2} 
    \left[
        |\textbf{u}_{p f}|^2/2+k_p
    \right],
\end{align*}
where $\textbf{u}_{pf} = \textbf{u}_p - \textbf{u}_1|_\alpha$. 
Therefore, the internal energies' equation for the particles phase do not need to be solved since it is directly related to the averaged particle phase relative velocity and the granular temperature $k_p$, which are both solved through the momentum equations and granular temperature \ref{eq:dt_hybrid_kp}.
By assuming the internal flow defined by \ref{eq:wa_def} we discard any effect of particles collision, which might affect the flow. 
Thus, this is valid as long as particles interaction can be disregarded placing our problem in the dilute limit. 

Assuming the internal motion of the droplets can also help us find out an expression for the inner dissipation rate term appearing in the particle internal energy equations (\ref{eq:dt_hybrid_ep}). 
Indeed, by the use of \ref{eq:wa_def} we can also compute analytically the diffusion source term within the particles. 
Namely, 
\begin{equation*}
    \pOavg{\bm{\sigma}_2^0:\grad \textbf{u}_2^0}
    = \frac{1}{3}\pi a  \mu_2
    \frac{n_p}{(1+\lambda)^2}
    (|\textbf{u}_{p f}|^2 /2  
    +  k_p)
\end{equation*}
This, term is the contribution of the hill's vortex flows inside the particles to the generation of heat in the particle phase. 
However, it is reasonable to say that collisions effects will have an impact on the droplets internal motion and thus on the dissipation rate and the previous exchange terms. 
In fact particles collision will affect every exchange terms derived here. 
Nevertheless, due to the complexity of the problem it will not be discussed further in this work.
It is interesting to remark that the inner energy of the droplet, and the heat transfer term both are proportional to $|\textbf{u}_{p f}|^2 /2  
+  k_p$ which can be a starting point to build a model. 

\subsubsection*{The surface work exchange terms.}

In the fluid phase pseudo turbulent energy equations : \ref{eq:dt_hybrid_k1} the exchange terms $\pSavg{\textbf{w}_2^0 \cdot \bm{\sigma}_1^0\cdot\textbf{n}_2}$ and  $\pSavg{\textbf{rw}_2^0 \cdot \bm{\sigma}_1^0\cdot\textbf{n}_2}$ involve explicitely the inner velocity of the drops.
Thus, we assume that the internal motion has the form of \ref{eq:wa_def} however the external stress and velocity fields are considered unknown.
Therefore, we only suppose the form of the relative internal velocity evaluated at the surface points which yields : $\textbf{w}_2^0|_\Sigma =
\frac{1}{2}\frac{1}{1+\lambda} \textbf{u}_{\alpha f}\cdot \left[
    (2\lambda+1)\textbf{I}
    + \textbf{n}_2\textbf{n}_2
\right] - \textbf{u}_{\alpha f}$. 
Upon substituting this expression in the exchange term one obtain,
\begin{multline*}
    \pSavg{\textbf{w}_2^0 \cdot \bm{\sigma}_1^0\cdot\textbf{n}_2}
    =  
    \left(\frac{2\lambda + 1}{2\lambda+2} - 1\right)\left[
        \textbf{u}_{p f}\cdot\pSavg{ \bm{\sigma}_1^0\cdot\textbf{n}_2}
        + \pavg{\textbf{u}_{\alpha}' \cdot \intS{ \bm{\sigma}_1^0\cdot\textbf{n}_2} }
    \right]
    + \\
    + \frac{1}{2\lambda+2}\left[
        \textbf{u}_{p f} \cdot\pSavg{\textbf{n}_2\textbf{n}_2\cdot \bm{\sigma}_1^0\cdot\textbf{n}_2}
        +
        \pavg{\textbf{u}_{\alpha}' \cdot \intS{\textbf{n}_2\textbf{n}_2\cdot \bm{\sigma}_1^0\cdot\textbf{n}_2}}
    \right]
\end{multline*}
Notice that for solid particles all the terms completely vanish.
This come from the fact that we did not consider rotational motion as in the previous example. 
Injecting this expression in to the pseudo turbulent energy equation, gives,
\begin{multline}
    \label{eq:dt_hybrid_k1_new}
    \pddt (\phi_1\rho_1k_1)  
    + \div (
        \phi_1\rho_1k_1\textbf{u}_1
        + \textbf{q}_1^\text{k} 
        )
    = 
    - \avg{\chi_1\bm{\sigma}_1^0 : \grad \textbf{u}_1^0}
    - \bm{\sigma}_1^\text{eq} : \grad \textbf{u}_1\\
    - \frac{2\lambda +1}{2\lambda +2} 
    \left[
        \textbf{u}_{pf}
        \cdot \pSavg{\bm{\sigma}_1^0 \cdot \textbf{n}_2}
        + \pavg{ \textbf{u}_\alpha' \cdot \intS{  \bm{\sigma}_1^0 \cdot \textbf{n}_2}}
    \right]\\
    - \frac{1}{2\lambda+2}\left[
        \textbf{u}_{p f} \cdot\pSavg{\textbf{n}_2\textbf{n}_2\cdot \bm{\sigma}_1^0\cdot\textbf{n}_2}
        +
        \pavg{\textbf{u}_{\alpha}' \cdot \intS{\textbf{n}_2\textbf{n}_2\cdot \bm{\sigma}_1^0\cdot\textbf{n}_2}}
    \right]
\end{multline}
\begin{multline*}
    \textbf{q}_1^\text{k}
    = \rho_1 \avg{\chi_1 \textbf{u}_1' k_1} 
    - \avg{\chi_1 \textbf{u}_1' \cdot \bm{\sigma}_1^0}
    - \frac{2\lambda +1}{2\lambda +2} 
    \left[
        \textbf{u}_{pf}
        \cdot \pSavg{\textbf{r}\bm{\sigma}_1^0 \cdot \textbf{n}_2}
        + \pavg{ \textbf{u}_\alpha' \cdot \intS{ \textbf{r} \bm{\sigma}_1^0 \cdot \textbf{n}_2}}
    \right]\\
    - \frac{1}{2\lambda+2}\left[
        \textbf{u}_{p f} \cdot\pSavg{\textbf{n}_2\textbf{n}_2\cdot \textbf{r}\bm{\sigma}_1^0\cdot\textbf{n}_2}
        +
        \pavg{\textbf{u}_{\alpha}' \cdot \intS{\textbf{n}_2\textbf{n}_2\cdot \textbf{r}\bm{\sigma}_1^0\cdot\textbf{n}_2}}
    \right]
\end{multline*}
% Regarding the diffusive terms it will be treated in a future study. 
If we compare this equation to the more general expression from \ref{eq:dt_hybrid_k1} we can see that the consideration of hill's vortexes add the coefficient $\frac{\lambda +\frac{1}{2}}{\lambda+1}$ in front of the drag force velocity correlation terms, and includes two terms related to the second higher moments surface traction. 
Besides, the first moments of surface traction forces appearing in the diffusive equivalent flux $\textbf{q}_1^k$ is also subject to these comments.  
By scaling arguments, it is fair to say that the term  $\textbf{u}_{pf} \cdot \pSavg{\textbf{r}\bm{\sigma}_1^0 \cdot \textbf{n}_2}$ is the greatest among all the terms on the RHS of \ref{eq:dt_hybrid_k1_new}. 
As a matter of facts the consideration of hill's vortex end up to add a coefficient in front of this exchange term which varies from $1$ to $1/2$ for respectively, solid particles and bubbles.  
Therefore, the consideration of hill's vortex have a very significant impact regarding the magnitude of the pseudo turbulent exchange terms when one is considering bubbly flow. 
The physical explanation of the decrease of the coefficient in front of the exchange terms for bubbles, can be due to the facts that the fluid slip on the bubbles or droplet's surface which induce less work exchange than if the fluid followed the particle's surface as it is the case for solid particles. 


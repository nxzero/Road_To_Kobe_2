The system consists of $n+1$ equations for $n+1$ unknowns, specifically $f_f$, $\textbf{Q}_p^{(1)}$\ldots$\textbf{Q}_p^{(n)}$ \footnote{Mote that $\phi_f$ and $n_p$ can also be treated as unknowns by adding corresponding equations, achieved by setting $f_f=1$ and $Q_\alpha^{(n)} =1$ in \ref{eq:avg_hybrid_dt_chi_f} and \ref{eq:avg_hybrid_q}.}.
The closure problem then involves deriving explicit expressions for all terms of the form $\avg{\ldots}$ in terms of the problem's unknowns, i.e. 
$f_f$, $\text Q_\alpha$, $\text Q_\alpha^{(1)}$ \ldots $ \text Q_\alpha^{(n)}$.  
Once these expressions are determined, the equations can be solved. 
However the closure problem appears to be a terrible problem, which have only be tackle for very limited cases. In the following we review  
It appears that despite many work on the closure problem in  very few theoretical results exist taking intot acount the whole terms appearing in ... 
In the Stokes flow regime and for very dilute flow \citep{jackson1997locally,zhang1997momentum} derived the momentum closure for spherical droplets and spherical solid prticles. 
Even in this regime the averaged fluid phase exhbibit a non-nwetonian behaviour a discuter
A priori cela ne vient que du moment d'ordre 2. Le cpt non-newtonien vient de la differente de vitesse

la dependance du terme de fermture au momoment sod'rde superieru est non trivial. Je pense qu'il y a un couplage ave lequation de f. PLus eventuellement closure (magnus ou fibre via pp)


The question that we would like to address now is the following : in which cases the higher moments $\textbf{Q}_p^{(n)}$ is needed in this system of equations ?
In fact there is two reason for which the moment $\textbf{Q}_p^{(n)}$ could be needed, these are : 
(1) because $\textbf{Q}_p^{(n)}$ could be an information that we seek to obtain as it is.
For instance think of the orientation of fibers in a flow, which correspond to the second moment of the distribution of mass of the particles.  
This information might be the goal of the whole study, as one seek to understand how the orientation of fiber evolve in the flow for industrial purposes. 
(2) because $\textbf{Q}_p^{(n)}$ is essential to compute one of the closure in the above equations. 
Again, the particle orientation is a good example, as even if the user isn't particularily interested in that information, it might be relevant, or even essential to have this information to model accurately the closure terms. 
Another example might be the particle angular velocity, which corresponds to the first order moment of momentum. 
This information is often not the final objective of the study, nevertheless it is essential to well model the closure terms of \textit{The hybrid model}, especially the effective stress of the suspenison, or the inter-phase drag force. 
Thus, we can conclude that the higher moments $\textbf{Q}_p^{(n)}$ is needed uniquely if the closure terms are highly dependent on this moment, or if the information given by $\textbf{Q}_p^{(n)}$ is the final objective of the study. 

+ rajouter le cas des surfactants pour illustrer l'influence des moments arbitraires de surface (dans discussion du dossier closure hybrid)


 %for the unknowns $f_f$,  $Q_\alpha$, $\textbf{Q}_p^{(1)}$\ldots$\textbf{Q}_p^{(n)}$. 


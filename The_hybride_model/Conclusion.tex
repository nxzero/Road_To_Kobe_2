We provided a well-developed hybrid model for dispersed two phase flow made of \ref{eq:avg_dt_chi_f} for the continuous phase, and the moment equations \ref{eq:avg_dt_dq_alpha_tot} and \ref{eq:avg_dt_dQ_alpha_tot} for the particle phase. 
It is derived in the most general way based on volume and surface governing equations valid at the local scale, namely \ref{eq:dt_f_I} and \ref{eq:dt_f_k}. 
After averaging the equations that govern the dispersed phase using two distinct frameworks, namely, the phase-averaged and particle-averaged equations, we demonstrated the equivalence between both formalism by carrying out a series expansion. 
In light of \ref{eq:scheme_equivalence} we reached the major conclusion of that work, i.e.  the phase-averaged equation for the dispersed phase, is a series expansion of the particle averaged equations.  
As assumed by \citet{zhang1997momentum} it is possible to describe a particle with an arbitrary order of accuracy by deriving the moments equations of a particle. 
In this work we provided a general form of these equations together with a clear physical explanation. 
Acknowledging this fact we concluded that  any physical phenomenon could be modeled with the moments equations since they constitute the phase-averaged equation, i.e. \ref{eq:avg_dt_chi_f} for $k =2$. 
Overall, the main advancement of this model is its ability to incorporate the effects of the particles' surface and volume properties, and provide a framework for deriving particle-average equations tailored to any type of problem.
In addition, to the first order conservation laws we also derived the higher order moments equations. 

To illustrate our point all along the derivation of the hybrid model we treat the case of the momentum conservation equation. 
It is shown that the surface tension play no role on the linear and angular momentum, but it does  affect so-called stretching of momentum of a particle. 
In general any surface diffusive flux are not involved in the linear moment\ref{eq:avg_dt_dq_alpha_tot} but play as a source term in the first order moment balance equations. 

To give a more physical insight on these moments equations we end thie work by proposing some examples. 
First we showed and discus how from these moments equations we can find back already demonstrated laws, such as the Rayleigh-Pesslet equations or the orientation tensor equation in fiber suspension. 
Then, we propose a new case were we use the first moment equation of the surfactant distribution to derive a transport equation of the center of mass of surfactant along the surface of the particle. 

The main draw back of this work is that inter particles interactions are not included naturally in these models. 
It would be interested in a future studies to show how pair particles statistics can be unpacked from the phase-averaged equation. 

\tb{Discus the closure terms}
\tb{Insist on the flexibility of this model, it can be apply to any kind of particle nature, liquid, deformable  solid, solid etc..}
\tb{Do a bullet list conclusion}
\tb{INSIST On the fact that all physial phenom is included in the moments equation. 
This is teh cause of the equivalence priciple}

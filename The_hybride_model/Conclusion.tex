\section{Conclusion}
\label{sec:conclusion}

\JL{dans la discussion on va commencer par parler des effets Marangoni en disant que prendre en compte des surfactants est sans doute tres difficile}
\JL{par contre on peut deja discuter en ordre de grandeur des effets Marangoni thermique. a discuter}
\JL{ensuite parles des Reynolds finis}
The closure problem presents a great challenge, and has only been addressed in very specific cases. 
In what follows, we will review the limited attempts made to close the momentum equation in viscous dominated flows. 
Despite the extensive work on this topic, only a few theoretical results exist that fully account for the whole closure problem. 
In the regime of Stokes flow and for very dilute suspensions, \citet{jackson1997locally} and \citet{zhang1997momentum} independently derived the momentum closure for spherical solid particles, with \citet{zhang1997momentum} \tb{and the above section } extending the work to spherical drops. 
The inclusion of the symmetric part of the first moment of momentum, namely the stresslet, leads to Einstein's viscosity formula. 
Even in this regime, the averaged fluid phase exhibits non-Newtonian behavior, induced by the relative motion of the particles due to the incorporation of quadratic terms in the velocity field far from the particles.
There have been very few efforts to extend these results to finite Reynolds number regimes. \citet{stone2001inertial} demonstrated that, in the case of general linear flow, the effect of finite inertia is to produce normal stresses, leading to non-Newtonian behavior, while the effective viscosity remains unchanged with respect to Einstein's original formula. 
This prediction was extended to the case of drops by \citet{raja2010inertial}. 
In a work in preparation we investigate the effects of drop translation at finite Reynolds numbers on the stresslet. 


 % However the closure problem appears to be a terrible problem, which have only be tackle for very limited cases. 
% In the following we review the few attemps to close the momentum equation. 
% It appears that despite many work on the closure problem in  very few theoretical results exist taking intot acount the whole closure problem.%terms appearing in \ref{eq:avg_hybrid_q}.
% In the Stokes flow regime and for very dilute flow \citet{jackson1997locally,zhang1997momentum} derived  independently the momentum closure for spherical solid prticles although \citep{zhang1997momentum} generalize it for drops.  
% The inclusion of the symmetric part of the first moment of momentum \textit{i.e.} the stresslet leads to the Einstein’s viscosity.
% Even in this regime the averaged fluid phase exhbibit a non-newtonian induced by the relative motion of the particle due to the inclusion of quadractic like term in the velocity field seen by the particle.%Faxen terms into the stress.
%Very few attempts have been made so far to enlarge this to the finite Reynolds number limit. \citep{stone2001inertial} demontrates in the case of a general linear flow that the effect of finite inertia is to procduce normal stress (leading to a non-Newtonian behaviour) while the effective viscosity is unchanged from Einstein’s formula.
%This previson was recently extended to the case of drop by \citet{raja2010inertial}.
%IN a work in preparation we consider the effect of translation of the drop in finite Reynolds on stresslet.
%la dependance du terme de fermture au momoment sod'rde superieru est non trivial. Je pense qu'il y a un couplage ave lequation de f. PLus eventuellement closure (magnus ou fibre via pp)



%The question that we would like to address now is the following : in which cases the higher moments $\textbf{Q}_p^{(n)}$ is needed in this system of equations ?
%The last question we now aim to address is: in what circumstances is the inclusion of higher moments, such as $\textbf{Q}_p^{(n)}$, necessary within this system of equations?
%There are two primary reasons why the moment $\textbf{Q}_p^{(n)}$ may be required:
%In fact there is two reason for which the moment $\textbf{Q}_p^{(n)}$ could be needed, these are : 
%(1) because $\textbf{Q}_p^{(n)}$ could be an information that we seek to obtain as it is.
%For instance think of the orientation of fibers in a flow, which correspond to the second moment of the distribution of mass of the particles.  
%This information might be the goal of the whole study, as one seek to understand how the orientation of fiber evolve in the flow for industrial purposes such as composite. 
%\begin{enumerate}
%\item First when $\textbf{Q}_p^{(n)}$ provides critical information sought in the analysis. 
%For example, consider the case of fiber orientation in a flow, which corresponds to the second moment of the particle mass distribution. 
%This information might be the central objective of the study, especially when the goal is to understand how fiber orientation evolves in industrial applications like composite materials \citep{advani1987use}.

%\item Second $\textbf{Q}_p^{(n)}$is essential for accurately modeling closure terms. 
%Again, fiber orientation serves as a relevant example. 
%Indeed in dilute flows of axisymmetric fibers within the Stokes regime the force acting on the fiber (which represents the closure term in the momentum equation) is dependent on the fiber orientation \citep{kim2013microhydrodynamics}. %, which in turn is directly linked to the second-order moment of the mass distribution.
Another example directly relevant to the focus of this article involves the transport of surfactants. 
Consider a dilute flow of spherical bubbles rising in a liquid contaminated by soluble surfactants. 
%As shown in Figure \ref{fig
%}, 
When a bubble ascends through the flow, surfactants are transported along the bubble surface, leading to the formation of a stagnant cap on the side opposite to the bubble direction of motion \citep{cuenot1997effects}.
The surfactant uneven concentration field on the interface, gives rise to a spatially variable surface tension coefficient. 
Furthermore, this variation in surface tension induces additional Marangoni stresses. %, as illustrated in Figure \ref{fig
%} and mathematically described by Equation \ref{eq
These Marangoni stresses alter the flow around the bubble surface, significantly affecting the drag force experienced by the rising bubble \citep{cuenot1997effects,pesci2018computational}. 
This influence is governed by the surfactant concentration and its distribution, which can be characterized by the moments of the interfacial surfactant concentration.
%These Marangoni stresses modify the flow around the bubble surface which directly influences both the drag force experienced by the rising bubble \citep{cuenot1997effects,pesci2018computational}, as dictated by the surfactant concentration and its distribution on the surface characterized by the zeroth and first order moment distribution of interfacial contrcentration of surfactants.


%Consequently, the change in surface tension not only affects the local hydrodynamics but also alters the shape of the particle, as governed by Equations \ref{eq
%} and \ref{eq
%}. 

%In addition, recent work by \citet{kentheswaran2022direct} has shown that the mean concentration and distribution of surfactants on bubble surfaces can significantly impact the mass transfer rate between the dispersed and continuous phases.
%even if the specific orientation isn’t of direct interest, it might be indispensable for ensuring the accuracy of the closure models.
%because $\textbf{Q}_p^{(n)}$ is essential to compute one of the closure in the above equations. 
%Again, the particle orientation is a good example, as even if the user isn't particularily interested in that information, it might be relevant, or even essential to have this information to model accurately the closure terms. 
%Another example directly related to the main purpose of this article is the cases of surfactant transports
%Let's consider a mono-disperse rising bubbly flow without phase transfer made of spherical particles of radius $a$, that is contaminated by soluble surfactants.
%On the other hand, as represented on \ref{fig:contaminated_bubbles}, during the rise of a bubble in the flow, the surfactants are transported along the droplets surface, which in turns create a stagnant cap in the opposite direction of the droplet velocity.
%According to \ref{eq:sigma_def} the possibly inhomogeneous concentration field, $c_I$, generates a spatially non-constant surface tension coefficient $\sigma$.
%Additionally, remark that the gradient of the surface tension generates additional Marangoni forces as represented in \ref{fig:contaminated_bubbles} and expressed by \ref{eq:surface_tension}.
%These additional surface tension forces alter the flow behavior in the vicinity of the droplet's surface.
%Therefore, this alteration in surface tension impacts the local hydrodynamic, but also induce a change in the shape of the particle through both, \ref{eq:dt_S_alpha} and \ref{eq:dt_M_alpha}.
%As a result, the drag force and drift velocity experienced by a rising bubble or droplet are directly influenced by the surfactant concentration and its distribution over the surface \citep{pesci2018computational}.
%Moreover, a recent study by \citet{kentheswaran2022direct} have demonstrated that the mean concentration and distribution of surfactants on the bubbles' surface also have a significant impact on the mass transfer rate between the dispersed and continuous phases.
%Another example might be the particle angular velocity, which corresponds to the first order moment of momentum. 
%It is well known, that a rotating particle immersed in a uniform flow will  
%This information is often not the final objective of the study, nevertheless it is essential to well model the closure terms of \textit{The hybrid model}, especially the effective stress of the suspenison, or the inter-phase drag force. 
%+ rajouter le cas des surfactants pour illustrer l'influence des moments arbitraires de surface (dans discussion du dossier closure hybrid)
%<<<<<<< HEAD
\tb{\item
    Note that in the continuous phase conservation equation \eqref{eq:avg_hybrid_dt_chi_f}, in the total volume conservation \ref{eq:volume_conservation}, and in the bulk velocity formulation \ref{eq:velocity_conservation}, an infinite number of moments are involved. 
    Hence, the last reason why one may need to obtain such higher moments is because it is required directly by \ref{eq:avg_hybrid_dt_chi_f}, \ref{eq:volume_conservation} and \ref{eq:velocity_conservation}. 
    As discussed above the moments of hydrodynamic momentum exchanges are crucial to describe the suspension rheology \eqref{eq:avg_hybrid_dt_chi_f}, but also note how the second moment of volume and momentum in \ref{eq:volume_conservation} and \ref{eq:velocity_conservation} are relevant for the stability and hyperbolicity of the system of equations \citep{prosperetti1995finite}. 
}
\end{enumerate}
%=======
%\item \tb{
%    Note that in the continuous phase conservation equation \eqref{eq:avg_hybrid_dt_chi_f}, in the total volume conservation \ref{eq:volume_conservation}, and in the bulk velocity formulation \ref{eq:velocity_conservation}, an infinite number of moments are involved. 
%    Hence, the last reason why one may need to obtain such higher moments is because it is required directly by \ref{eq:avg_hybrid_dt_chi_f}, \ref{eq:volume_conservation} and \ref{eq:velocity_conservation}. 
%    As discussed above the moments of hydrodynamic momentum exchanges are crucial to describe the suspension rheology \eqref{eq:avg_hybrid_dt_chi_f}, but also note how the second moment of volume and momentum in \ref{eq:volume_conservation} and \ref{eq:velocity_conservation} are relevant for the stability and hyperbolicity of the system of equations \citep{prosperetti1995finite}. 
%}
%\end{enumerate}
%>>>>>>> 4731d90 (JLP: hybrid)
%Thus, we can conclude that the higher moments $\textbf{Q}_p^{(n)}$ is needed uniquely if the closure terms are highly dependent on this moment, or if the information given by $\textbf{Q}_p^{(n)}$ is the final objective of the study. 
%In conclusion, the higher-order moments, $\textbf{Q}_p^{(n)}$, are specifically required if the closure terms are significantly influenced by this moment or if the primary goal of the study is to analyze the information provided by $\textbf{Q}_p^{(n)}$, \tb{or when they appear explicitly in the conservation law and turns out being non-negligible}.
 %for the unknowns $f_f$,  $Q_\alpha$, $\textbf{Q}_p^{(1)}$\ldots$\textbf{Q}_p^{(n)}$. 

 
 %A priori cela ne vient que du moment d'ordre 2. Le cpt non-newtonien vient de la differente de vitesse
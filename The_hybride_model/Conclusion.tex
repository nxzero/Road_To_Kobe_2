\section{Conclusion}
\label{sec:conclusion}




In this work we tried to expose a self consistent and complete expse on the averaged equaitons and how to derive the closure for complex dispersed phases. 
To illustrate our purpose we studied the example of viscous-dominated suspension of droplet, including non-uniform surface tension effects. 
We showed how surface tension gradients induced momentum source, known as Marangoni drift, but also it induces non-Newtonian behavior in the momentum equaitons. 
For pedagogical purposes we have determined how to determine the shape of the droplets based on the momentum equations. 
An obvious perspective of this work is to use the Lagrangian-averaged equations of surfaces to properly derive the transport equations of surfactant concentration (or temperature) to relate the gradients of surface tension to the averaged hydrodynamic quantities.  


On a more pragmatic note, there have been very few efforts to extend these results to finite Reynolds number regimes. \citet{stone2001inertial} demonstrated that, in the case of general linear flow, the effect of finite inertia is to produce normal stresses, leading to non-Newtonian behavior, while the effective viscosity remains unchanged with respect to Einstein's original formula. 
This prediction was extended to the case of drops by \citet{raja2010inertial}. 
However, in these works they only consider neutrally buoyant particles, hence neglecting the effects of relative motion on the suspension rheology. 

So what if little inertial effects were to come into accountin buoyancy driven flows ?
To provide a first glimpse of how inertial effects could impact the suspension rheology and droplet deformation, we need to find at least the first order correction in $\mathcal{O}(Re)$ of the closure terms in \ref{eq:dt_hybrid_Sp} and \ref{eq:dt_uf2}. 
Based on symmetry arguments (similar to those used for \ref{eq:Reynolds_stress_functional_form}) one may state that the stresslet is given by, 
\begin{equation}
    \pSavg{\textbf{r}\bm\sigma^*_f\cdot \textbf{n}}
    -\pOavg{2\mu_f\textbf{e}^*_f}
    =
    Re \phi 
    \left[
       C_s^1 \textbf{u}_{r}\textbf{u}_{r} 
    +  C_s^2 (\textbf{u}_{r}\cdot \textbf{u}_{r})\bm\delta
    \right]
    + O(Re^2,\phi^2),
    \\
\end{equation}
at the first order in Reynolds number. 
We conclude that at finite Reynolds number relative motions impact the shape balance of the particle, at least through this term (in agreement with \citet{taylor1964deformation}) and the counterpart is that it induces stresses in the suspension proportional to $\sim \textbf{u}_r\textbf{u}_r$ in addition to the contribution of the Reynolds stress \eqref{eq:Reynolds_stress_functional_form}. 
In a work in preparation we investigate the effects of drop translation at finite Reynolds numbers on the stresslet, notably we provide the exact values of the constant $C_s^{1}$ and $C_s^{2}$ (see also \citet{fintzi2025}). 
 


%\end{enumerate}
%=======
%\item \tb{
%    Note that in the continuous phase conservation equation \eqref{eq:avg_hybrid_dt_chi_f}, in the total volume conservation \ref{eq:volume_conservation}, and in the bulk velocity formulation \ref{eq:velocity_conservation}, an infinite number of moments are involved. 
%    Hence, the last reason why one may need to obtain such higher moments is because it is required directly by \ref{eq:avg_hybrid_dt_chi_f}, \ref{eq:volume_conservation} and \ref{eq:velocity_conservation}. 
%    As discussed above the moments of hydrodynamic momentum exchanges are crucial to describe the suspension rheology \eqref{eq:avg_hybrid_dt_chi_f}, but also note how the second moment of volume and momentum in \ref{eq:volume_conservation} and \ref{eq:velocity_conservation} are relevant for the stability and hyperbolicity of the system of equations \citep{prosperetti1995finite}. 
%}
%\end{enumerate}
%>>>>>>> 4731d90 (JLP: hybrid)
%Thus, we can conclude that the higher moments $\textbf{Q}_p^{(n)}$ is needed uniquely if the closure terms are highly dependent on this moment, or if the information given by $\textbf{Q}_p^{(n)}$ is the final objective of the study. 
%In conclusion, the higher-order moments, $\textbf{Q}_p^{(n)}$, are specifically required if the closure terms are significantly influenced by this moment or if the primary goal of the study is to analyze the information provided by $\textbf{Q}_p^{(n)}$, \tb{or when they appear explicitly in the conservation law and turns out being non-negligible}.
 %for the unknowns $f_f$,  $Q_\alpha$, $\textbf{Q}_p^{(1)}$\ldots$\textbf{Q}_p^{(n)}$. 

 
 %A priori cela ne vient que du moment d'ordre 2. Le cpt non-newtonien vient de la differente de vitesse
In this work we provided a well-developed hybrid model for dispersed two phase flow made of fluid particles.
The main contribution of the work is the derivation of a generalized formulation of what is called a \textit{Hybrid model} for dispersed two phase flows. 
This model it is derived in the most general way incorporating volume and surface property of the particles.

After averaging the equations that govern the dispersed phase using two distinct frameworks, namely, the phase-average and the particle-average, we demonstrated the equivalence between both formalism. 
In light of \ref{eq:scheme_equivalence} we reached the major conclusion of this work, i.e.  the phase-averaged equation for the dispersed phase, is a series expansion of the particle-averaged equations.  
As expected by several authors \citep{zhang1997momentum,nott2011suspension} it is therefore possible to describe a particle with an arbitrary order of accuracy by using the moments equations of a particle. 
In this work we provided the general form of these equations together with a clear physical explanation of the moment equations. 
This statement is only true if one assume that the Taylor series used in the process are convergent within the particles domains. 
% Acknowledging this fact we concluded that  any physical phenomenon could be modeled with the moments equations since they constitute the phase-averaged equation, i.e. \ref{eq:avg_dt_chi_f} with $k =2$. 
The final system of equation, from \ref{eq:avg_hybrid_dt_chi_f} to \ref{eq:avg_hybrid_q_1} is constituted of a conservation equation for the continuous phase averaged quantity $f_f$, and $n$ dispersed phase equations for the moments $\textbf{q}_p^{(n)}$. 
The equations for the moments $\textbf{q}_p^{(n)}$ posses each a well-defined physical sense that are related to the moment of the distribution of the local quantity $\chi_df_d^0 + \delta_\Gamma f_\Gamma^0$ inside the particles. 
Then, we clarified the role of the non-convective fluxes on the dispersed phase properties. 

Overall, the main advancement of this model is its ability to incorporate the effects of the particles' surface properties within the Lagrangian balance laws of arbitrary order, and provides a framework for deriving particle-average equations tailored to any type of problem.

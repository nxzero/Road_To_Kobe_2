\section{Conclusion}
\label{sec:conclusion}




In this work, we aimed to present a complete exposition of the averaged equations and the derivation of closures for complex dispersed phases. 
The general derivation carried from \ref{sec:local_eq} to \ref{sec:equivalence} explains in detail the structure of the system of averaged equations under the so-called hybrid formulation.  
To illustrate our approach, in \ref{sec:closure} and \ref{sec:averaged_surface} we study the example of a dilute viscous-dominated suspension of droplets, including non-uniform surface tension effects. 
We demonstrated how surface tension gradients induce a momentum source, which is responsible for the Marangoni drift of the dispersed phase, and how they also lead to non-Newtonian behavior in the momentum equations. 
Then, we determined how to calculate the averaged shape of the droplets based on the moment of momentum equations. 
An obvious perspective of this work is to use the surfaces Lagrangian-averaged equations to properly derive the transport equations for surfactant concentration (or temperature field), thereby relating surface tension gradients to physical quantities.  


On another note, there have been very few efforts to extend these results to finite Reynolds number regimes, despite the fact that most flows of interest exhibit inertial characteristics. 
\citet{stone2001inertial} demonstrated that, in the case of general linear flow, the effect of finite inertia is to produce normal stresses, leading to non-Newtonian behavior, while the effective viscosity remains unchanged with respect to Einstein's original formula. 
This prediction was extended to the case of drops by \citet{raja2010inertial}. 
However, in these works they only consider neutrally buoyant particles, hence neglecting the effects of relative motion on the suspension rheology. 
To provide a first glimpse of how inertial effects could impact the suspension rheology and droplet deformation, we need to find at least the first order correction in $\mathcal{O}(Re)$ of the closure terms in \ref{eq:dt_hybrid_Sp} and \ref{eq:dt_uf2}. 
Based on symmetry arguments (similar to those used for \ref{eq:Reynolds_stress_functional_form}) one may state that the stresslet is given by, 
\begin{equation}
    \pSavg{\textbf{r}\bm\sigma^*_f\cdot \textbf{n}}
    -\pOavg{2\mu_f\textbf{e}^*_f}
    =
    Re \phi 
    \left[
       C_s^1 \textbf{u}_{r}\textbf{u}_{r} 
    +  C_s^2 (\textbf{u}_{r}\cdot \textbf{u}_{r})\bm\delta
    \right]
    + O(Re^2,\phi^2),
    \label{eq:final}
\end{equation}
at the first order in Reynolds number. 
We conclude that, at finite Reynolds numbers, relative motions induce stresses in the suspension proportional to $\sim \textbf{u}_r\textbf{u}_r$, in addition to the contribution of the Reynolds stress \eqref{eq:Reynolds_stress_functional_form}. 
The counterpart is that these relative motions also induce deformation of the droplets, as shown in \eqref{eq:dt_hybrid_Sp} (in agreement with \citet{taylor1964deformation}).
In a work currently in preparation, we investigate the effects of droplet translation at finite Reynolds numbers on the stresslet. 
Notably, we provide the exact values of the constants $C_s^{1}$ and $C_s^{2}$ of \ref{eq:final} (see also \citet{fintzi2025}). 
 

%Alhtough, the problem considered in the present work differs form the one encountered in processes by considreding a fixed surface tensio gradient rather than solving a transport scalr transport equation modifying locally the ssurface tension
%It is interesting to undertsnad in which cases gradient of surface tension effects becomes significant. 
%First we may note that the effect of surface tension gradient are null when the viscosity ratio is high, ie when the droplet are much more viscous thant the surrounding fluid.
%Then one may expect marangoni flows to much important for bubble than for droplets.
%In the following we will consider only the case of bubbles $\lambda =0$.
%Now we consider the case of a rising bubble in a fluid at rest with an imposed constant gradient or surface tension or equivalently an imposed gradient of species at infinity and linear relationship between the surface tension and the concentration of species.
%Equilibrating the force due to surface tension gradient to the gravity force one get and assuming moderate value of the viscosity ratio we get,  $\nabla \gamma  \sim (\rho_d -\rho) g a$ 
%Then we assume $\nabla \gamma \sim \Delta  \gamma / L$ where $\Delta  \gamma$ is the scales variation of surface tension and $L$ the lenthscale over which varies the surface tension varies. 
%Typically for an aqueous emulision ($\rho = 1000 $kg m$^{-3}$, $\mu =10^{-3}$Pa s$^{-1}$) for which $\rho_d -\rho \approx 100 $kg m$^{-3}$, typical variation of surface tension of order $\Delta  \gamma \approx 10^{-2}$N m$^{-1}$ and a typical radius $a \approx 10 \mu m$, Marangoni effects are comparable in magnitude to the weight of particle if $L \approx $1m.
%Hence we may expect significant effect of Marangoni force on the dynamics of very small bubbles even if the size over which the surface tension varies is of order of the meter.
%Now we consider the case of drop embedded in shear flow with an imposed second order spatial derivative of surface tension.
%Balancing in the effectvive stress the shear indiced viscosity term to the term orinating ftom surface tension gradient we obtain $\nabla \nabla \gamma \sim \mu \dot \gamma/a$ where $\dot \gamma$ is the typical shear rate.
%For a typical shear rate of order $\dot \gamma \approx 1 $s$^{-1}$, we get $L =1$ cm. 
%Then for processes with a local variation of surface tension of order of a few centimeter, the effective stress generated by Marangoni stress will be comparable in magnitude to that of the shear induced effective viscosity.








%\end{enumerate}
%=======
%\item \tb{
%    Note that in the continuous phase conservation equation \eqref{eq:avg_hybrid_dt_chi_f}, in the total volume conservation \ref{eq:volume_conservation}, and in the bulk velocity formulation \ref{eq:velocity_conservation}, an infinite number of moments are involved. 
%    Hence, the last reason why one may need to obtain such higher moments is because it is required directly by \ref{eq:avg_hybrid_dt_chi_f}, \ref{eq:volume_conservation} and \ref{eq:velocity_conservation}. 
%    As discussed above the moments of hydrodynamic momentum exchanges are crucial to describe the suspension rheology \eqref{eq:avg_hybrid_dt_chi_f}, but also note how the second moment of volume and momentum in \ref{eq:volume_conservation} and \ref{eq:velocity_conservation} are relevant for the stability and hyperbolicity of the system of equations \citep{prosperetti1995finite}. 
%}
%\end{enumerate}
%>>>>>>> 4731d90 (JLP: hybrid)
%Thus, we can conclude that the higher moments $\textbf{Q}_p^{(n)}$ is needed uniquely if the closure terms are highly dependent on this moment, or if the information given by $\textbf{Q}_p^{(n)}$ is the final objective of the study. 
%In conclusion, the higher-order moments, $\textbf{Q}_p^{(n)}$, are specifically required if the closure terms are significantly influenced by this moment or if the primary goal of the study is to analyze the information provided by $\textbf{Q}_p^{(n)}$, \tb{or when they appear explicitly in the conservation law and turns out being non-negligible}.
 %for the unknowns $f_f$,  $Q_\alpha$, $\textbf{Q}_p^{(1)}$\ldots$\textbf{Q}_p^{(n)}$. 

 
 %A priori cela ne vient que du moment d'ordre 2. Le cpt non-newtonien vient de la differente de vitesse
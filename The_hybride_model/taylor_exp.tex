%\section{Proof of the multipole expansion}
\section{Expansion of phase and surface quantities as series of moments}%Expansion of phase and surface quantities in terms of moments series}%Moment expansions of phase and surface quantities}
\label{app:expansion}
% \subsection{Nicolas proof}
% Let us first prove the equality, 
% \begin{equation}
%     (f^0_d \chi_\alpha)[\textbf{x}]
%     =
%     (f^0_d \chi_\alpha * \delta)[\textbf{x}]
%     =
%     \int_{\mathbb{R}^3} 
%      (f^0_d \chi_\alpha)[\textbf{x}_\alpha + \textbf{r}]\delta(\textbf{x} - \textbf{x}_\alpha - \textbf{r}) 
%      d\textbf{r},
%     \label{ap:eq:taylor_f_d}
% \end{equation}
% in the distributional sense, by applying the left-hand side and right-hand side on a test function $\phi$.
% We have used the notation $*$ to represent the convolution product. 
% The left-hand side of \ref{ap:eq:taylor_f_d} on a test function gives, 
% \begin{equation}
%     < (f^0_d \chi_\alpha)[\textbf{x}], \phi[\textbf{x}]>
%     = \int_{\mathbb{R}^3}
%     (f^0_d \chi_\alpha)[\textbf{x}] \phi[\textbf{x}]
%     d\textbf{x}
%     = \intO{
%         f^0_d [\textbf{x}] \phi[\textbf{x}]
%         }
% \end{equation}
% The left-hand side of \ref{ap:eq:taylor_f_d} on a test function yields, 
% \begin{align}
%     <   (f^0_d \chi_\alpha *\delta)[\textbf{x}]
%     , \phi[\textbf{x}]>
%     &= 
%     \int_{\mathbb{R}^3}
%     (f^0_d \chi_\alpha)[\textbf{x}_\alpha + \textbf{r}]
%     \int_{\mathbb{R}^3} 
%      \delta(\textbf{x} - \textbf{x}_\alpha - \textbf{r}) 
%      \phi[\textbf{x}]
%      d\textbf{x}
%      d\textbf{r}\\
%     &= 
%     \int_{\mathbb{R}^3}
%     (f^0_d \chi_\alpha)[\textbf{x}_\alpha + \textbf{r}]
%      \phi[\textbf{x}_\alpha + \textbf{r}]
%      d\textbf{r}
%     \\
%     &=\intO{
%         f^0_d [\textbf{x}_\alpha + \textbf{r}]
%         \phi[\textbf{x}_\alpha + \textbf{r}]
%     }
%     \label{eq:second_equality}
% \end{align}
% In both case we end up integrating $f_d^0$ over the particle volume so the equality \ref{ap:eq:taylor_f_d} is true. 


% Now let us prove the validity of \ref{eq:fd_asympt} in the distributional sense. 
% We start from \ref{eq:second_equality} and Taylor expand $\phi[\textbf{x}_\alpha +\textbf{r}]$ around the particle center, which is allowed since $\phi$ is smooth. 
% Injecting $\phi[\textbf{x}_\alpha +\textbf{r}] = \phi[\textbf{x}_\alpha] + \textbf{r} \cdot \grad \phi|_{\textbf{x}_\alpha} + \ldots$ in \ref{eq:second_equality} gives, 
% \begin{align}
%     <   (f^0_d \chi_\alpha *\delta)[\textbf{x}]
%     , \phi[\textbf{x}]> 
%     &= 
%     % \intO{
%     %     f^0_d [\textbf{x}_\alpha + \textbf{r}]
%     %     \phi[\textbf{x}_\alpha + \textbf{r}]
%     % }\\
%     % &= 
%     \phi[\textbf{x}_\alpha]
%     \intO{
%     f^0_d[\textbf{x}_\alpha + \textbf{r}]
%      }
%     + 
%     \grad\phi|_{\textbf{x}_\alpha}
%     \cdot 
%     \intO{
%     \textbf{r}
%     f^0_d[\textbf{x}_\alpha + \textbf{r}]
%      }
%     + \ldots
%     \label{eq:third_step}
% \end{align}
% Note that since the Dirac delta $\delta_\alpha = \delta(\textbf{x}- \textbf{x}_\alpha)$ is a unit of convolution we might as well note that, 
% \begin{align*}
%     \phi[\textbf{x}_\alpha]
%     &= \int_{\mathbb{R}^3} 
%     \delta(\textbf{x} - \textbf{x}_\alpha)
%     \phi[\textbf{x}]
%     d\textbf{x}
%     =  <\delta_\alpha, \phi[\textbf{x}]>
%     \\
%     \grad\phi|_{\textbf{x}_\alpha}
%     &= 
%     \int_{\mathbb{R}^3} 
%     \delta(\textbf{x} - \textbf{x}_\alpha)
%     \grad\phi|_{\textbf{x}}
%     d\textbf{x}
%     = <
%     \delta_\alpha,
%     \grad\phi|_{\textbf{x}}
%     >
%     = <
%     \grad\delta_\alpha|_{\textbf{x}},
%     \phi
%     >
% \end{align*}
% where the last relation is true since $\phi$ is a well-defined test-function \citet{appel2007}.
% Substituting this last expression into \ref{eq:third_step} yields the relation, 
% \begin{align}
%     <   (f^0_d \chi_\alpha *\delta)[\textbf{x}]
%     , \phi[\textbf{x}]> 
%     &= 
%     <\delta_\alpha
%     \intO{
%         f^0_d[\textbf{x}_\alpha + \textbf{r}]
%     }
%     , \phi[\textbf{x}]>
%     + < \grad\delta_\alpha
%     \cdot 
%     \intO{
%     \textbf{r}
%     f^0_d[\textbf{x}_\alpha + \textbf{r}]
%      }
%      , \phi[\textbf{x}]>
%     + \ldots
%     \label{eq:last_step}
% \end{align}

% Acknowledging \ref{ap:eq:taylor_f_d}, and \ref{eq:last_step} we finally arrive at the result, 
% \begin{align}
%     <   (f^0_d \chi_\alpha)[\textbf{x}]
%     , \phi[\textbf{x}]> 
%     = 
%     <\delta_\alpha
%     \intO{
%         f^0_d
%     }
%     , \phi[\textbf{x}]>
%     + < \grad\delta_\alpha
%     \cdot 
%     \intO{
%     \textbf{r}
%     f^0_d
%      }
%      , \phi[\textbf{x}]>
%     + \ldots
%     \label{eq:last2_step}
% \end{align}
% With a slight abuse of notation we might deduce the relations, 
% \begin{equation}
%     (f^0_d \chi_\alpha)[\textbf{x}]
%     = 
%     \delta_\alpha
%     \intO{
%         f^0_d
%     }
%     + \grad\delta_\alpha
%     \cdot 
%     \intO{
%     \textbf{r}
%     f^0_d
%      }
%     + \ldots
% \end{equation}
% \begin{equation}
%     \delta(\textbf{x}- \textbf{x}_\alpha - \textbf{r})
%     = 
%     \delta(\textbf{x}- \textbf{x}_\alpha)
%     - \textbf{r}\cdot \grad\delta(\textbf{x}- \textbf{x}_\alpha)
%     + \ldots
% \end{equation}

%\subsection{Nicolas proof V2}

We first provide a concise proof of the following relation
\begin{equation}
    \delta(\textbf{x} - \textbf{x}_\alpha - \textbf{r})
    = \delta(\textbf{x} - \textbf{x}_\alpha)
    - \textbf{r}\cdot\grad \delta(\textbf{x} - \textbf{x}_\alpha)
    + \frac{1}{2}\textbf{rr}:\grad\grad\delta(\textbf{x} - \textbf{x}_\alpha) 
    - \ldots
    % + \ldots
\label{eq:exp_delta}
\end{equation}
which is widely used in the literature (see for example \citet{zhang2023evolution}). 
We begin by applying the Dirac delta function $\delta(\textbf{x} - \textbf{x}_\alpha - \textbf{r})$, to a test function $\varphi(\textbf{x})$.
This results in the following expression,
\begin{align*}
    < \delta(\textbf{x} - \textbf{x}_\alpha - \textbf{r}), \varphi(\textbf{x})> 
    =
    \varphi(\textbf{x}_\alpha + \textbf{r}).  
\end{align*}
By applying the Taylor expansion to the test function $\varphi(\textbf{x}_\alpha + \textbf{r})$ around the point $\textbf{r} = 0$ , we obtain the following expression
\begin{equation}
    < \delta(\textbf{x} - \textbf{x}_\alpha - \textbf{r}), \varphi(\textbf{x})> 
    =
    \varphi(\textbf{x}_\alpha) 
    + \textbf{r} \cdot \grad \varphi|_{\textbf{x}_\alpha}
    + \frac{1}{2}\textbf{r}\textbf{r} : \grad\grad \varphi|_{\textbf{x}_\alpha}
    + \ldots
    \label{eq:first_step}
\end{equation}
By the definition of the Dirac delta function and its derivative \citep{appel2007}, we have  %$\delta_\alpha = \delta(\textbf{x}-\textbf{x}_\alpha)$ we can write \citep{appel2007}, 
%\begin{align}
$  <\delta_\alpha, \varphi> = \varphi(\textbf{x}_\alpha)$, % \\
$<\grad\delta_\alpha, \varphi> = -\grad\varphi|_{\textbf{x}_\alpha} $,
$...$  
%\end{align}
Substituting these definitions into \ref{eq:first_step} , we obtain in the distributional sense \ref{eq:exp_delta}. 
%\begin{align}
%    \delta(\textbf{x} - \textbf{x}_\alpha - \textbf{r})
%    =
%    \delta_\alpha - \textbf{r}\cdot \grad \delta_\alpha + \ldots
%    \label{eq:exp_delta2}
%\end{align}
%which is exactly 
%the relation given by \ref{eq:exp_delta}. 

Let us now turn our attention to the demonstration of \ref{eq:fd_asympt0}. 
We can show that, 
\begin{align}
    <(f_d^0\chi_\alpha)(\textbf{x}),\varphi(\textbf{x})>
    &= <(f_d^0\chi_\alpha)(\textbf{x}_\alpha+\textbf{r}),\varphi(\textbf{x}_\alpha+\textbf{r})> \label{eq:first_equality}\\
    &= 
    <(f_d^0\chi_\alpha)(\textbf{x}_\alpha + \textbf{r}) ,<\delta(\textbf{x} - \textbf{x}_\alpha - \textbf{r}), \varphi(\textbf{x})>>.
    \label{eq:second_equality}
    %\\
    %&= 
    %<(f_d^0\chi_d)(\textbf{x}_\alpha + \textbf{r}) \delta(\textbf{x} - \textbf{x}_\alpha - \textbf{r}), \varphi(\textbf{x})>
\end{align}
The first equality follows directly from the change of variables $\textbf{x}=\textbf{x}_\alpha + \textbf{r}$. 
The second equality holds by the definition of the Dirac delta functions and of the convolution product in the distributional sense \citep{appel2007}.
%Lastly, the third equality corresponds to the definition of the convolution product in the distributional sense \citep{appel2007}.
%obtained by noticing that the inner application of \ref{eq:second_equality} is along $\textbf{x}$ while the outer along the variable $\textbf{r}$. 
Substituting \ref{eq:exp_delta}  into the previous equation gives, 
% \begin{equation}
%    <(f_d^0\chi_d)(\textbf{x}),\varphi(\textbf{x})>
%    = 
%    <\delta_\alpha (f_d^0\chi_d)(\textbf{x}_\alpha + \textbf{r}),\varphi(\textbf{x})>
%     - <\div\{\delta_\alpha (\textbf{r} f_d^0\chi_d)(\textbf{x}_\alpha + \textbf{r}) \}, \varphi(\textbf{x})>
%    + \ldots
% \end{equation}
% from which we deduce the relation,
\begin{equation}
    f^0_d \chi_\alpha
    = 
    \delta_\alpha
    \intO{
        f^0_d
    }
    - \div\left(    
    \delta_\alpha
    \intO{
    \textbf{r}
    f^0_d}
    \right)
    + \frac{1}{2}\grad\grad :\left(\delta_\alpha\intO{\textbf{rr} f^0_d}\right)
    - \ldots
\end{equation}
The previous proof can be easily extended to surface quantities by following the same steps as in \ref{eq:first_equality} and \ref{eq:second_equality},  leading to
\begin{equation} 
    <f_\Gamma^0 \delta_{\Gamma\alpha},\varphi(\textbf{x})> = <(f_\Gamma^0 \delta_{\Gamma\alpha})(\textbf{x}_\alpha + \textbf{r}) ,<\delta(\textbf{x} - \textbf{x}_\alpha - \textbf{r}), \varphi(\textbf{x})>>,
\end{equation}
from which we obtain
\begin{equation} 
f_\Gamma^0 \delta_{\Gamma\alpha} 
=\delta_\alpha\intS{f^0_\Gamma}
- \div\left(\delta_\alpha\intS{\textbf{r} f^0_\Gamma}\right)
+ \frac{1}{2}\grad\grad :\left(\delta_\alpha\intS{\textbf{rr} f^0_\Gamma}\right)
-\ldots 
\end{equation}

%which directly prooves \ref{eq:taylor_f_d} when we sum over all particles. 


% \subsection{Jean-Lou proof}
% We first make the following change of variable $\textbf{x} = \textbf{x} ' - \textbf{x}_\alpha$. Then,  

% \begin{align}
% f^0_d(\mathbf{x}) \chi_\alpha (\mathbf{x}) &= f^0_d(\textbf{x} ' - \textbf{x}_\alpha) \chi_\alpha (\textbf{x} ' - \textbf{x}_\alpha) \\
%                                &= (f^0_d(\textbf{x}') \chi_\alpha (\textbf{x}'))*\delta(\textbf{x} ' - \textbf{x}_\alpha)\\
%                                &=<   (f^0_d(\textbf{x}') \chi_\alpha(\textbf{x}'),< \delta(\textbf{y} '-\textbf{x}_\alpha),\phi(\textbf{x'}+\textbf{y'})>>\\
%                                &= <   (f^0_d(\textbf{x}') \chi_\alpha(\textbf{x}'), \phi(\textbf{x'}+\textbf{x}_\alpha)> \\
%                                &= \int_{\mathbb{R}^3} f^0_d(\textbf{x}') \chi_\alpha(\textbf{x}')\phi(\textbf{x'}+\textbf{x}_\alpha)d\textbf{x'} \\
%                                &= \int_{V_\alpha} f^0_d(\textbf{x}')\phi(\textbf{x'}+\textbf{x}_\alpha)d\textbf{x'}
% \end{align}


% Then performing a Taylor expansion, $\phi(\textbf{x}_\alpha +\textbf{x}') = \phi(\textbf{x}_\alpha) + \textbf{x}' \cdot \grad \phi|_{\textbf{x}_\alpha} + \ldots$ we get

% \begin{align}
%     f^0_d(\mathbf{x}) \chi_\alpha (\mathbf{x}) &= \phi(\textbf{x}_\alpha)\int_{V_\alpha} f^0_d(\textbf{x}')d\textbf{x'} +   \grad \phi|_{\textbf{x}_\alpha}\cdot\int_{V_\alpha} \textbf{x}'f^0_d(\textbf{x}')d\textbf{x'}+\ldots
% \end{align}

% By definition
% \begin{align*}
%     \phi(\textbf{x}_\alpha)&
%     =  <\delta_\alpha, \phi>
%     \\
%     \grad\phi|_{\textbf{x}_\alpha}
%     &= 
%      -<
%     \grad\delta_\alpha,
%     \phi
%     >
% \end{align*}
% Then we get 
% \begin{align}
%     f^0_d(\mathbf{x}) \chi_\alpha (\mathbf{x}) &= \delta_\alpha\int_{V_\alpha} f^0_d(\textbf{x}')d\textbf{x'} -   \grad\delta_\alpha\cdot\int_{V_\alpha} \textbf{x}'f^0_d(\textbf{x}')d\textbf{x'}+\ldots
% \end{align}
% Then since the volume integral are independent of space variable one get
% \begin{align}
%     f^0_d(\mathbf{x}) \chi_\alpha (\mathbf{x}) &= \delta_\alpha\int_{V_\alpha} f^0_d(\textbf{x}')d\textbf{x'} -   \grad \cdot \left(\delta_\alpha\int_{V_\alpha} \textbf{x}'f^0_d(\textbf{x}')d\textbf{x'}\right)+\ldots
% \end{align}
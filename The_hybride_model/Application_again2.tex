\Huge{ \tb{Brouillion}}\normalsize

\section{Clear}
\subsection{Continuous phase}
Here we try to make appear only the gradient of pressure. 
We consider Newtonian fluid such that $\bm{\sigma}_k^0 = -p^0_k \textbf{I} + \bm{\tau}_k^0$. 

\begin{align}
    \label{eq:dt_&vg_rho}
    \pddt (\phi_1 \rho_1)  
    + \div (
        \phi_1 \rho_1\textbf{u}_1
    )
    &= 
    0\\
    \label{eq:dt_&vg_rhou_1}
    \pddt (\phi_1 \rho_1\textbf{u}_1)  
    + \div (
        \phi_1 \rho_1\textbf{u}_1\textbf{u}_1
        - \bm{\sigma}_1^\text{eq}
    )
    &= 
    \phi_1  (\rho_1\textbf{g} - \grad p_1 )
    -  \avg{\delta_I \bm{\sigma}_1' \cdot \textbf{n}_2}\\
    \label{eq:dt_&vg_rhoE_1}
    \pddt (\phi_1\rho_1E_1)  
    + \div (
        \phi_1\rho_1E_1\textbf{u}_1
        + \bm{q}_1^\text{eq}
        - \textbf{u}_1 \cdot \bm{\sigma}_1^\text{eq}
        % - \textbf{u}_1^0 \cdot \bm{\sigma}_1^0 
        % + \textbf{q}_1^0
        )
    &= 
    \phi_1 \textbf{u}_1 \cdot (\rho_1\textbf{g}
    - \grad p_1 
    )
    - \phi_1 p_1 \div \textbf{u}_1\\
    &- \avg{\delta_I (\textbf{u}_1' \cdot \bm{\sigma}_1^0 
    - \textbf{q}_1^0)\cdot \textbf{n}_2}
    - \textbf{u}_1 \cdot \avg{\delta_I ( \bm{\sigma}_1')\cdot \textbf{n}_2}
\end{align} 
\begin{align*}
    &\bm{\sigma}_1^\text{eq}
    = \phi_1 (
        \bm{\tau}_1%- n_p \textbf{M}_p
        - \rho_1\kavg{\textbf{u}_1'\textbf{u}_1'})  
    &\textbf{q}_1^\text{eq}
    =\textbf{q}_1^\text{e} 
    +\textbf{q}_1^\text{k}  \\
    &\textbf{q}_1^\text{e}
    = \phi_1\rho_1 \kavg{\textbf{u}_1' e_1'} 
    + \phi_1\textbf{q}_1 
    &\textbf{q}_1^\text{k}
    = \phi_1\rho_1 \kavg{\textbf{u}_1' k_1} 
    - \phi_1\kavg{\textbf{u}_1' \cdot \bm{\sigma}_1^0}
\end{align*}
In the above definition we wrote $\bm{\sigma}_1' =\bm{\sigma}_1^0 - p_1 \textbf{I}$. And we use the expressions, 
\begin{align*}
    \avg{\delta_I (\bm{\sigma}_1^0 ) \textbf{n}_2} - p_1 \grad \phi_1
    = 
    \avg{\delta_I (\bm{\sigma}_1^0 + p_1)\cdot \textbf{n}_2}
    = 
    \avg{\delta_I \bm{\sigma}_1'\cdot \textbf{n}_2}
    \\
    \avg{\delta_I (\textbf{u}_1^0 \cdot\bm{\sigma}_1^0 )} - \textbf{u}_1p_1\cdot \grad \phi_1
    % = \avg{\delta_I (\textbf{u}_1 \cdot \bm{\sigma}_1' + \textbf{u}_1' \cdot \bm{\sigma}_1^0 )\cdot \textbf{n}_2}
    = \textbf{u}_1 \cdot \avg{\delta_I \bm{\sigma}_1'\cdot \textbf{n}_2}
    + \avg{\delta_I (\textbf{u}_1' \cdot \bm{\sigma}_1^0 )\cdot \textbf{n}_2}
\end{align*}
The interfacial terms in the momentum equation can be reformulated as,
\begin{align*}
    \avg{\delta_I \bm{\sigma}_1' \cdot \textbf{n}_2}
    = n_p \textbf{f}_\text{pm} + \div (n_p \mathcal{F}_\text{pfp} - n_p \mathcal{F}_p)
\end{align*}
where $\mathcal{P}_\text{pfp}$ is the particle-fluid -particles interaction and $\mathcal{F}_p$ is the first hydrodynamical moments.
Regarding the energy equation transfer term it can be expressed as that,  
\begin{align*}
    \avg{\delta_I (\textbf{u}_1' \cdot \bm{\sigma}_1^0)\cdot \textbf{n}_2}
    = n_p \textbf{c}_\text{pm} + \div (n_p \mathcal{C}_\text{pfp} - \mathcal{C}_p)\\
    \avg{\delta_I \textbf{q}_1 \cdot \textbf{n}_2}
    = n_p \textbf{e}_\text{pm} + \div (n_p \mathcal{E}_\text{pfp} - \mathcal{E}_p)
\end{align*}
where we use similar definition as the drag force. 
With this consideration the above equations becomes, 

\begin{align}
    \pddt (\phi_1 \rho_1)  
    + \div (
        \phi_1 \rho_1\textbf{u}_1
    )
    &= 
    0\\
    \pddt (\phi_1 \rho_1\textbf{u}_1)  
    + \div (
        \phi_1 \rho_1\textbf{u}_1\textbf{u}_1
        - \bm{\sigma}_1^\text{eq}
    )
    &= 
    \phi_1  (\rho_1\textbf{g} - \grad p_1 )
    -  \textbf{f}_\text{pm}\\
    \pddt (\phi_1\rho_1E_1)  
    + \div (
        \phi_1\rho_1E_1\textbf{u}_1
        + \bm{q}_1^\text{eq}
        - \textbf{u}_1 \cdot \bm{\sigma}_1^\text{eq}
        % - \textbf{u}_1^0 \cdot \bm{\sigma}_1^0 
        % + \textbf{q}_1^0
        )
    &= 
    n_p (\mathcal{F}_\text{pfp} - \mathcal{F}_p):\grad \textbf{u}_1
    + \phi_1 \textbf{u}_1 \cdot (\rho_1\textbf{g}
    - \grad p_1 
    )\nonumber \\ 
    &- \phi_1 p_1 \div \textbf{u}_1
    - n_p \textbf{c}_\text{pm}
    + n_p \textbf{e}_\text{pm}
    - \textbf{u}_1 \cdot \textbf{f}_\text{pm}
\end{align} 
\begin{align*}
    &\bm{\sigma}_1^\text{eq}
    = \phi_1 (
        \bm{\tau}_1%- n_p \textbf{M}_p
        - \rho_1\kavg{\textbf{u}_1'\textbf{u}_1'}) 
        - n_p  (\mathcal{F}_{pfp} - \mathcal{F}_\text{p})
    &\textbf{q}_1^\text{eq}
    =\textbf{q}_1^\text{e} 
    +\textbf{q}_1^\text{k}  \\
    &\textbf{q}_1^\text{e}
    = \phi_1\rho_1 \kavg{\textbf{u}_1' e_1'} 
    + \phi_1\textbf{q}_1 
    - n_p (\mathcal{E}_\text{pfp} - \mathcal{E}_p)
    &\textbf{q}_1^\text{k}
    = \phi_1\rho_1 \kavg{\textbf{u}_1' k_1} 
    - \phi_1\kavg{\textbf{u}_1' \cdot \bm{\sigma}_1^0}
    + n_p (\mathcal{C}_\text{pfp} - \mathcal{C}_p)
\end{align*}
It is interesting to notice that the fluide-particle-fluide particle stress and the first moment (actually just the stress let) act as dissipative term in the equation of the total energy. 
In a recent study of \citet{boniou2023shock} they indeed include such term in the dissipation term. 
They claim that $\mathcal{F}_\text{pfp} : \grad \textbf{u}_1 $ is the work done by the pfp interaction on the particles.
to reduce their volume fraction. 
However, they omit the Stresslet and other probably important terms. 
The total averaged energy can be express as, 
\begin{equation*}
    E_1 = e_1 + k_1 + \frac{1}{2}u^2_1 
\end{equation*}
Taking the dot product of $\textbf{u}_1$ with the momentum equation one obtain an equation for $u^2_1$, it reads, 
\begin{align}
    \pddt (\phi_1 \rho_1u_1^2/2)  
    + \div (
        \phi_1 \rho_1\textbf{u}_1u_1^2/2
        - \textbf{u}_1 \cdot \bm{\sigma}_1^\text{eq}
    )
    &= 
    - \bm{\sigma}_1^\text{eq}  : \grad \textbf{u}_1
    + \phi_k \textbf{u}_1 \cdot (\textbf{g} \rho_k - \grad p_1 )
    -  n_p \textbf{u}_1\cdot \textbf{f}_\text{pm}\\
    \pddt (\phi_1\rho_1k_1)  
    + \div (
        \phi_1\rho_1k_1\textbf{u}_1
        + \textbf{q}_1^\text{k} 
        )
    &= 
    - \phi_1 \textbf{d}_1 
    +\phi_1 (\bm{\tau}_1 - \rho_1 \oneavg{\textbf{u}'_1\textbf{u}'_1} ) : \grad \textbf{u}_1
    - \phi_1 p_1  \div\textbf{u}_1
    -n_p  \textbf{c}_\text{pm}\\
    \pddt (\phi_1\rho_1e_1)  
    + \div (
        \phi_1 \rho_1e_1\textbf{u}_1
        +
        \textbf{q}_1^\text{e} 
        )
    &= 
    \phi_1 \textbf{d}_1 
    + n_p \textbf{e}_\text{pm}
\end{align}
It is interesting to notice that even through we considered particle fluid particle interactions the PFP stress do not impact the turbulent kinetic energy equations. 

\subsection{The dispersed phase equations}

Regarding the dispersed phase, we found the mass, momentum and total energy balance equations, 
\begin{align*}
    \pddt \left(n_p m_p\right)
    + \div \left(n_pm_p\textbf{u}_p
    \right)
    &= 
    0\\
    \pddt \left(n_p m_p \textbf{u}_p\right)
    + \div \left(n_p
    m_p \textbf{u}_p \textbf{u}_p 
    - \bm{\sigma}_p^\text{eq}
    \right)
    &= 
    n_p v_p  (  
    \rho_2 \textbf{g}
    - \grad p_1)
    + n_p \textbf{f}_\text{pm},\\
    \pddt(m_p n_pE_p^\text{tot})
    + \div(m_pn_p E_p^\text{tot} \textbf{u}_p 
    + \textbf{q}_p^\text{eq} - \textbf{u}_p \cdot \bm{\sigma}_p^\text{eq})
    &=  n_p v_p [\rho_2 \textbf{u}_p\cdot  \textbf{g} 
    - \textbf{u}_1 \cdot \grad p_1  -  p_1 \div \textbf{u}_1]\\
    + n_p (\mathcal{F}_p - \mathcal{F}_\text{pfp}):\grad \textbf{u}_1
    + n_p (\textbf{c}_\text{pm}
    - \textbf{e}_\text{pm})
    + n_p \textbf{u}_1 \cdot \textbf{f}_\text{pm}
\end{align*}
\begin{align*}
    &\bm{\sigma}_p^\text{eq}
    = -  m_p\pnavg{\textbf{u}_\alpha'\textbf{u}_\alpha'}
        + n_p \mathcal{F}_\text{pfp} 
    &\textbf{q}_p^\text{eq}
    =\textbf{q}_p^\text{e} 
    +\textbf{q}_p^\text{k}  
    +\textbf{q}_p^\text{w}  
    \\
    &\textbf{q}_1^\text{e}
    = m_p \pnavg{\textbf{u}_\alpha' e_\alpha'} 
    + n_p \mathcal{E}_\text{pfp} 
    &\textbf{q}_p^\text{k}
    = m_p \pnavg{\textbf{u}_\alpha' k_\alpha} 
    - n_p \mathcal{C}_\text{pfp} 
    \\
    &\textbf{q}_p^\text{w}
    = 
    + \pnavg{\textbf{u}_\alpha'(\rho_2 (w^0_2)^2/2 )'_\Omega}
    + \gamma \pnavg{\textbf{u}_\alpha' s_\alpha'}
    + n_p (\textbf{u}_p - \textbf{u}_1)\cdot \mathcal{F}_\text{pfp} 
\end{align*}
where we made use of the expression, 
\begin{align*}
    n_p (\bm{\sigma}_1^0 \cdot  \textbf{n}_2)^\Sigma_p
    &= 
    n_p ( \bm{\sigma}_1' \cdot  \textbf{n}_2)^\Sigma_p
    - n_p v_p \grad p_1\\
    n_p (\textbf{u}_1^0 \cdot \bm{\sigma}_1^0 \cdot  \textbf{n}_2)^\Sigma_p
    &= 
    n_p \textbf{u}_1 \cdot \textbf{f}_{pm}
    + n_p (\mathcal{F}_p - \mathcal{F}_\text{pfp}): \grad \textbf{u}_1
    + n_p \textbf{c}_\text{pm}
    + \div [n_p(\mathcal{C}_\text{pfp} + \textbf{u}_1 \cdot \mathcal{F}_\text{pfp})]
    - n_p v_p \div (\textbf{u}_1 p_1) \\
    n_p (\textbf{w}_1^0 \cdot \bm{\sigma}_1^0 \cdot  \textbf{n}_2)^\Sigma_p
    &= 
    n_p (\textbf{u}_1^0 \cdot \bm{\sigma}_1^0 \cdot  \textbf{n}_2)^\Sigma_p
    - n_p (\textbf{u}_\alpha \cdot \bm{\sigma}_1^0 \cdot  \textbf{n}_2)^\Sigma_p
    % n_p (\textbf{u}_1 \cdot \bm{\sigma}_1' \cdot  \textbf{n}_2)^\Sigma_p
    % + n_p (\textbf{u}_1' \cdot \bm{\sigma}_1^0 \cdot  \textbf{n}_2)^\Sigma_p
    % - n_p v_p \div (\textbf{u}_1 p_1) \\
\end{align*}
to express correctly the exchange term. 
Under this form it is clear that the diffusive source term $+ n_p (\mathcal{F}_p - \mathcal{F}_\text{pfp}):\grad \textbf{u}_1$ drive the exchange between both phase. 
The averaged total energy of the particle phase can be written,
The averaged particle energy $n_p E_p$ can be split into five components,
\begin{equation*}
    n_p m_p E_p(t) 
    = m_p n_p e_p 
    + n_p (\rho_2  (w_2^0)^2/2 )_p^\Omega
    + n_p s_p \gamma
    + m_p n_p k_p
    + m_p n_p (u_p)^2/2
    % + \textbf{u}_\alpha \cdot \int_{\Omega_\alpha(t)} \rho_2  \textbf{w}_2^0 d\Omega
\end{equation*}
The second term refer to other internal motion, \tb{take about the solely spining spheres}
Consequently, to derive an equation for $k_p$ one must derive an equation for all the components and subtract to the total energy equation. 
The macroscopic energy equation reads, 
\begin{align*}
    &\pddt \left(n_p m_p u_p^2/ 2\right)
    + \div \left(n_p
    m_p u_p^2 \textbf{u}_p /2
    - \textbf{u}_p \cdot \bm{\sigma}_p^\text{eq}
    \right)
    = 
    - \bm{\sigma}_p^\text{eq}  :\grad \textbf{u}_p
    +  n_p v_p \textbf{u}_p \cdot (
    \rho_2 \textbf{g}
    - \grad p_1 )
    + n_p \textbf{u}_p \cdot \textbf{f}_{pm},\\
    &\pddt \left(n_p m_p (u_\alpha^2)_p/ 2\right)
    + \div \left(n_p
    m_p (u_\alpha^2)_p/ 2 \textbf{u}_p 
    + \textbf{q}^k_p
    - \textbf{u}_p \cdot \bm{\sigma}_p^\text{eq}
    \right)
    = 
    n_p m_p \textbf{u}_p \cdot
    \textbf{g}
    + 
    (\textbf{u}_\alpha\cdot
    (\bm{\sigma}_1^0 \cdot \textbf{n}_2)^\Sigma)_p\\
    &\pddt \left(n_p (\rho_2 w^2 )_p^\Omega+\gamma s_p n_p\right)
    + \div 
    (n_p (\rho_2 w^2 )_p^\Omega+\gamma s_p n_p)
    \textbf{u}_p 
    +  \textbf{q}_p^\text{w}
    )
    = 
    - n_p \textbf{d}_p
    +  n_p (\textbf{u}_1 -\textbf{u}_p) \cdot  (\textbf{f}_{pm} - v_p \grad p_1)
    + n_p\textbf{c}_p\\
    &\pddt \left(n_p m_p e_p\right)
    + \div \left(n_p
    m_p e_p \textbf{u}_p 
    +  \textbf{q}_p^\text{e}
    \right)
    = 
    + n_p \textbf{d}_p
    + n_p \textbf{e}_{pm},\\
\end{align*}
Subtracting to the total energy the sum of these equations give an additional equation for the granular temperature. 
\begin{equation}
    \pddt(m_p n_pk_p)
    + \div(m_pn_p k_p \textbf{u}_p 
    + \textbf{q}_p^\text{k})
    = 
     -m_p \pnavg{\textbf{u}_\alpha'\textbf{u}_\alpha'} :\grad \textbf{u}_p
     + n_p (\textbf{u}_\alpha' \cdot (\bm{\sigma}_1' \cdot  \textbf{n}_2)^\Sigma)_p
    \\
\end{equation}
We see that we recover the classic equation of kinetic theory. 
Nevertheless, we obtain additional equation to describe internal energy within the particles. 


\subsection*{Second order moment equation}

If one need to solve for additional degree of freedom of the particle one can use the first order moment of momentum and first order moment of mass averaged equaiton which yields, 

\begin{align}
    \pddt \left(n_p \mathcal{M}_p\right)
    + \div \left(
        n_p \textbf{u}_p \mathcal{M}_p
    + \Sigma_p^\text{eq}
    \right)
    &=
    n_p \mathcal{S}_p\\
    \pddt \left(n_p \mathcal{P}_p\right)
    + \div \left(
        n_p \textbf{u}_p \mathcal{P}_p
    + \Sigma_p^\text{eq}
    \right)
    &=
    n_p v_p p_1 \textbf{I}
    + n_p \mathcal{F}_p
    + \left(
        \rho_2 \textbf{w}_2^0  \textbf{w}_2^0 
        - \bm{\sigma}_2'
    \right)^\Omega_p
    - n_p (\gamma \textbf{I}_{||})^\Sigma_p,
\end{align}

\subsection{A note on the bulk stress equaiton}
The first closures appearing in this equation is the equivalent stress. 
\section*{Brouillons }

\subsection{Continuous phase}
Here we try to make appear only the gradient of pressure. 
We consider Newtonian fluid such that $\bm{\sigma}_k^0 = -p^0_k \textbf{I} + \bm{\tau}_k^0$. 

\begin{align}
    \label{eq:dt_&vg_rho}
    \pddt (\phi_1 \rho_1)  
    + \div (
        \phi_1 \rho_1\textbf{u}_1
    )
    &= 
    0\\
    \label{eq:dt_&vg_rhou_1}
    \pddt (\phi_1 \rho_1\textbf{u}_1)  
    + \div (
        \phi_1 \rho_1\textbf{u}_1\textbf{u}_1
        - \bm{\sigma}_1^\text{eq}
    )
    &= 
    \phi_1  (\rho_1\textbf{g} - \grad p_1 )
    +  \avg{\delta_I \bm{\sigma}_1' \cdot \textbf{n}_1}\\
    \label{eq:dt_&vg_rhoE_1}
    \pddt (\phi_1\rho_1E_1)  
    + \div (
        \phi_1\rho_1E_1\textbf{u}_1
        + \bm{q}_1^\text{eq}
        - \textbf{u}_1 \cdot \bm{\sigma}_1^\text{eq}
        % - \textbf{u}_1^0 \cdot \bm{\sigma}_1^0 
        % + \textbf{q}_1^0
        )
    &= 
    \phi_1 \textbf{u}_1 \cdot (\rho_1\textbf{g}
    - \grad p_1 
    )
    - \phi_1 p_1 \div \textbf{u}_1\\
    &+ \avg{\delta_I (\textbf{u}_1' \cdot \bm{\sigma}_1^0 
    - \textbf{q}_1^0)\cdot \textbf{n}_1}
    + \textbf{u}_1 \cdot \avg{\delta_I ( \bm{\sigma}_1')\cdot \textbf{n}_1}
\end{align} 
\begin{align*}
    &\bm{\sigma}_1^\text{eq}
    = \phi_1 (
        \bm{\tau}_1%- n_p \textbf{M}_p
        - \rho_1\kavg{\textbf{u}_1'\textbf{u}_1'})  
    &\textbf{q}_1^\text{eq}
    =\textbf{q}_1^\text{e} 
    +\textbf{q}_1^\text{k}  \\
    &\textbf{q}_1^\text{e}
    = \phi_1\rho_1 \kavg{\textbf{u}_1' e_1'} 
    + \phi_1\textbf{q}_1 
    &\textbf{q}_1^\text{k}
    = \phi_1\rho_1 \kavg{\textbf{u}_1' k_1} 
    - \phi_1\kavg{\textbf{u}_1' \cdot \bm{\sigma}_1^0}
\end{align*}
In the above definition we wrote $\bm{\sigma}_1' =\bm{\sigma}_1^0 - p_1 \textbf{I}$. 
Both transfer term can be reformulated using an expansion within the particle volume and inter particles distances. 
First of all, 
\begin{align*}
    \avg{\delta_I \bm{\sigma}_k' \cdot \textbf{n}_k}
    = n_p \textbf{f}_p - \div (n_p \mathcal{F}_p)
\end{align*}
Additionally $n_p \textbf{f}_p$ can be further decomposed using the nearest particle statistic decomposition, 
\begin{align*}
    \avg{\delta_I \bm{\sigma}_k' \cdot \textbf{n}_k}
    = n_p \textbf{f}_\text{pm} + \div (n_p \mathcal{F}_\text{pfp} - n_p \mathcal{F}_p)
\end{align*}
with the definition,
\begin{align*}
    n_p \mathcal{F}_p 
    = \pavg{\int_{\Sigma_\alpha}  \textbf{r} \bm{\sigma}_k' \cdot \textbf{n}_k d\Sigma }
    &&
    n_p \mathcal{F}_\text{pfp} 
    = \int \textbf{r} \pavg{ \delta_\beta  h_{\alpha\beta}\int_{\Sigma_\alpha} \bm{\sigma}_k' \cdot \textbf{n}_k d\Sigma } d\textbf{r}
\end{align*} 
and, 
\begin{equation*}
    n_p \textbf{f}_\text{pm} 
    = \frac{1}{2}\int \textbf{f}^\text{nst}P_\text{nst}(\textbf{x} - \textbf{r}/2, \textbf{r}) + \textbf{f}^\text{nst}P_\text{nst}(\textbf{x} + \textbf{r}/2,\textbf{r}) d\textbf{r}
\end{equation*}
Is the pure drag force components. 

using the same formalism we can show equally that, 
\begin{align*}
    \avg{\delta_I (\textbf{u}_k' \cdot \bm{\sigma}_k^0 + \textbf{u}_k \cdot \bm{\sigma}_k' - \textbf{q}_k^0)\cdot \textbf{n}_k}
    &= n_p \textbf{c}_\text{pm}^\text{tot} 
    + \div (n_p \mathcal{C}_\text{pfp}^\text{tot} - n_p \mathcal{C}_p^\text{tot})\\
    &=
    n_p \textbf{c}_\text{pm} 
    + n_p \textbf{e}_\text{pm} 
    + n_p \textbf{u}_k \cdot \textbf{f}_\text{pm}\\
    &+ \div [n_p \mathcal{C}_\text{pfp} - n_p \mathcal{C}_p + n_p \mathcal{E}_\text{pfp} - n_p \mathcal{E}_p + n_p \textbf{u}_k \cdot (\mathcal{F}_\text{pfp} - \mathcal{F}_p) ]
\end{align*}
with the following definitions
\begin{align*}
    n_p \textbf{c}_p^\text{tot}
    &= \int \pavg{\delta_\beta h_{\alpha\beta}\int_{\Sigma_\alpha}  (\textbf{u}_k' \cdot \bm{\sigma}_k^0 + \textbf{u}_k \cdot \bm{\sigma}_k' - \textbf{q}_k^0) d\cdot \textbf{n}_k d\Sigma }(\textbf{x}\pm \textbf{r}/2,\mp\textbf{r}) d\textbf{r}\\
    &= n_p \textbf{c}_{pm} 
    + n_p \textbf{e}_{pm} 
    + n_p \textbf{u}_k \cdot \textbf{f}_{pm} 
    \\
    n_p \mathcal{C}_p^\text{tot}
    &= \pavg{\int_{\Sigma_\alpha}  \textbf{r} (\textbf{u}_k' \cdot \bm{\sigma}_k^0 + \textbf{u}_k \cdot \bm{\sigma}_k' - \textbf{q}_k^0)\cdot \textbf{n}_k d\Sigma }\\
    &= n_p \mathcal{C}_p 
    + n_p \mathcal{E}_p 
    + n_p \textbf{u}_k \cdot \mathcal{F}_p 
    \\
    n_p \mathcal{C}_\text{pfp}^\text{tot} 
    &= \int \textbf{r} \pavg{ \delta_\beta  h_{\alpha\beta}\int_{\Sigma_\alpha} (\textbf{u}_k' \cdot \bm{\sigma}_k^0 + \textbf{u}_k \cdot \bm{\sigma}_k' - \textbf{q}_k^0)\cdot \textbf{n}_k d\Sigma } d\textbf{r} \\
    &= n_p \mathcal{C}_\text{pfp} 
    +n_p \mathcal{E}_\text{pfp} 
    + n_p \textbf{u}_k \cdot \mathcal{F}_\text{pfp} \\
\end{align*} 
But, we could have shown that,
\begin{align*}
    \avg{\delta_I (\textbf{u}_k' \cdot \bm{\sigma}_k^0 + \textbf{u}_k \cdot \bm{\sigma}_k' - \textbf{q}_k')\cdot \textbf{n}_k}
    &= 
    \avg{\delta_I (\textbf{u}_k' \cdot \bm{\sigma}_k^0 - \textbf{q}_k')\cdot \textbf{n}_k}
    + \avg{\delta_I \textbf{u}_k \cdot \bm{\sigma}_k'\cdot \textbf{n}_k}\\
    &=
    n_p \textbf{c}_\text{pm} 
    + \div [n_p \mathcal{C}_\text{pfp} - n_p \mathcal{C}_p]
    + \textbf{u}_k \cdot 
    [n_p \textbf{f}_\text{pm} + \div (n_p \mathcal{F}_\text{pfp} - n_p \mathcal{F}_p)]\\
\end{align*}
Consequently, we must deduce that, 
\begin{equation*}
    + \textbf{u}_k \cdot \div (n_p \mathcal{F}_\text{pfp} - n_p \mathcal{F}_p)
    = 
    + \div [ n_p \textbf{u}_k \cdot (\mathcal{F}_\text{pfp} - \mathcal{F}_p) ]
\end{equation*}
Meaning, $(n_p \mathcal{F}_\text{pfp} - n_p \mathcal{F}_p) :  \grad \textbf{u}_k = 0$
Using the generic formulation \ref{eq:hybrid_avg_dt_chif} and the local expression of the mass, momentum and total energy expression, i.e. : \ref{eq:dt_rho},\ref{eq:dt_rhou_1} and \ref{eq:dt_rhoE_1} we easily find the averaged form of these equations as, 

\begin{align}
    \pddt (\phi_1 \rho_1)  
    + \div (
        \phi_1 \rho_1\textbf{u}_1
    )
    &= 
    0\\
    \pddt (\phi_1 \rho_1\textbf{u}_1)  
    + \div (
        \phi_1 \rho_1\textbf{u}_1\textbf{u}_1
        - \bm{\sigma}_1^\text{eq}
    )
    &= 
    \phi_1  (\rho_1\textbf{g} - \grad p_1 )
    - n_p \textbf{f}_\text{pm}\\
    \pddt (\phi_1\rho_1E_1)  
    + \div (
        \phi_1\rho_1E_1\textbf{u}_1
        + \bm{q}_1^\text{eq}
        - \textbf{u}_1 \cdot \bm{\sigma}_1^\text{eq}
        )
    &= 
    \phi_1 \textbf{u}_1 \cdot [\rho_1 \textbf{g} 
    - \grad p_1 ] - \phi_1 p_1 \div \textbf{u}_1\\
    & - n_p (\textbf{c}_\text{pm}
    + \textbf{e}_\text{pm})
    - n_p \textbf{u}_1 \cdot \textbf{f}_\text{pm}
\end{align} 
\todo{Is this formulation usefull for the NRJ equaiton ? }
where we have defined, 
\begin{align*}
    &\bm{\sigma}_1^\text{eq}
    = \phi_1(
        \bm{\tau}_1%- n_p \textbf{M}_p
        - \rho_1 
        \kavg{\textbf{u}_1'\textbf{u}_1'})
        - n_p \mathcal{F}_\text{pfp} + n_p \mathcal{F}_p
    &\textbf{q}_1^\text{eq}
    =\textbf{q}_1^\text{e} 
    +\textbf{q}_1^\text{k}  
    \\
    &\textbf{q}_1^\text{e}
    = \phi_1\rho_1 \kavg{\textbf{u}_1' e_1'} 
    + \phi_1 \textbf{q}_1
    + n_p \mathcal{E}_\text{pfp} 
    - n_p \mathcal{E}_p 
    &\textbf{q}_1^\text{k}
    = \phi_1\rho_1 \kavg{\textbf{u}_1' k_1} 
    - \phi_1\kavg{\textbf{u}_1' \cdot \bm{\sigma}_1^0} 
    + n_p \mathcal{C}_\text{pfp} 
    - n_p \mathcal{C}_p 
\end{align*}
Note that the phase averaged energy equation can be further decompose following, 
\begin{align*}
    E_1 = e_1 + k_1 + u_1^2/2
\end{align*}
where $K_1$ is the pseudo-turbulent kinetic energy defined such as, $\phi_1 k_1 = \avg{\chi_1 (u_1')^2/2}$. 
The Macroscopic kinetic energy equation can be obtain by taking the dot product with $\textbf{u}_1$. 
\begin{align}
    \pddt (\phi_1 \rho_1u_1^2/2)  
    + \div (
        \phi_1 \rho_1\textbf{u}_1u_1^2/2
        - \textbf{u}_1 \cdot \bm{\sigma}_1^\text{eq}
    )
    &= 
    - \bm{\sigma}_1^\text{eq}  : \grad \textbf{u}_1
    + \phi_k \textbf{u}_1 \cdot (\textbf{g} \rho_k - \grad p_1 )
    -  n_p \textbf{u}_1\cdot \textbf{f}_\text{pm}\\
    \pddt (\phi_1\rho_1k_1)  
    + \div (
        \phi_1\rho_1k_1\textbf{u}_1
        + \textbf{q}_1^\text{k} 
        )
    &= 
    - \phi_1 \textbf{d}_1 
    + \bm{\sigma}_1^\text{eq} : \grad \textbf{u}_1
    - \phi_1 p_1  \div\textbf{u}_1
    -n_p  \textbf{c}_\text{pm}\\
    \pddt (\phi_1\rho_1e_1)  
    + \div (
        \phi_1 \rho_1e_1\textbf{u}_1
        +
        \textbf{q}_1^\text{e} 
        )
    &= 
    \phi_1 \textbf{d}_1 
    - n_p \textbf{e}_\text{pm}
\end{align}
where $\textbf{d}_1 = \oneavg{\bm{\sigma}_1^0 : \grad \textbf{u}_1^0}$ is the averaged dissipation tensor. 
\todo[inline]{check the cst }
\subsection{The dispersed phase equations simple try}

Regarding the dispersed phase, we found the mass, momentum and total energy balance equations, 
\begin{align*}
    \pddt \left(n_p m_p\right)
    + \div \left(n_pm_p\textbf{u}_p
    \right)
    = 
    0\\
    \pddt \left(n_p m_p \textbf{u}_p\right)
    + \div \left(n_p
    m_p \textbf{u}_p \textbf{u}_p 
    - \bm{\sigma}_p^\text{eq}
    \right)
    = 
    n_p v_p \rho_2 \textbf{g}
    + n_p (\bm{\sigma}_1^0 \cdot \textbf{n}_2)_p^\Sigma,\\
    \pddt(m_p n_pE_p^\text{tot})
    + \div(m_pn_p E_p^\text{tot} \textbf{u}_p 
    + \textbf{q}_p^\text{eq} - \textbf{u}_p \cdot \bm{\sigma}_p^\text{eq})
    =  n_p v_p \rho_2 \textbf{u}_p\cdot  \textbf{g}\\
    % +  n_p ( \textbf{u}'_1 \cdot \bm{\sigma}_1^0 \cdot \textbf{n}_2)_p^\Sigma
    -  n_p (\textbf{q}_1^0 \cdot \textbf{n}_2)_p^\Sigma
    +  n_p (\textbf{u}_1^0 \cdot \bm{\sigma}_1^0\cdot \textbf{n}_2)_p^\Sigma
\end{align*}
where we have defined, 
\begin{align*}
    &\bm{\sigma}_p^\text{eq}
    = -  m_p\pnavg{\textbf{u}_\alpha'\textbf{u}_\alpha'}
    &\textbf{q}_p^\text{eq}
    =\textbf{q}_p^\text{e} 
    +\textbf{q}_p^\text{k}  
    +\textbf{q}_p^\text{w}  
    \\
    &\textbf{q}_1^\text{e}
    = m_p \pnavg{\textbf{u}_\alpha' e_\alpha'} 
    &\textbf{q}_p^\text{k}
    = m_p \pnavg{\textbf{u}_\alpha' k_\alpha} 
    \\
    &\textbf{q}_p^\text{w}
    = 
    + \pnavg{\textbf{u}_\alpha'(\rho_2 (w^0_2)^2/2 )'_\Omega}
    + \gamma \pnavg{\textbf{u}_\alpha' s_\alpha'}
\end{align*}
Also, 
\begin{align*}
    &\pddt \left(n_p m_p u_p^2/ 2\right)
    + \div \left(n_p
    m_p u_p^2/ 2 \textbf{u}_p 
    - \textbf{u}_p \cdot \bm{\sigma}_p^\text{eq}
    \right)
    = 
    - \bm{\sigma}_p^\text{eq}  :\grad \textbf{u}_p
    +  n_p v_p \textbf{u}_p \cdot 
    \rho_2 \textbf{g}
    + n_p \textbf{u}_p \cdot (\bm{\sigma}_1^0 \cdot \textbf{n}_2)^\Sigma_p,\\
    &\pddt \left(n_p m_p (u_\alpha^2)_p/ 2\right)
    + \div \left(n_p
    m_p (u_\alpha^2)_p/ 2 \textbf{u}_p 
    + \textbf{q}^k_p
    - \textbf{u}_p \cdot \bm{\sigma}_p^\text{eq}
    \right)
    = 
    n_p m_p \textbf{u}_p \cdot
    \textbf{g}
    + 
    (\textbf{u}_\alpha\cdot
    \textbf{f}_\alpha)_p\\
    &\pddt \left(n_p (\rho_2 w^2 )_p^\Omega+\gamma s_p n_p\right)
    + \div 
    (n_p (\rho_2 w^2 )_p^\Omega+\gamma s_p n_p
    \textbf{u}_p 
    +  \textbf{q}_p^\text{w}
    )
    = \\
    &- n_p (\bm{\sigma}_2^0 : \grad\textbf{u}_2^0)^\Omega_p
    + n_p (\textbf{u}_1^0 \cdot \bm{\sigma}_1^0 \cdot  \textbf{n}_2)^\Sigma_p
    - n_p (\textbf{u}_\alpha \cdot \bm{\sigma}_1^0 \cdot  \textbf{n}_2)^\Sigma_p
    \\
    &\pddt \left(n_p m_p e_p\right)
    + \div \left(n_p
    m_p e_p \textbf{u}_p 
    +  \textbf{q}_p^\text{e}
    \right)
    = 
    + n_p (\bm{\sigma}_2^0 : \grad\textbf{u}_2^0)^\Omega_p
    - n_p (\textbf{q}_1^0\cdot \textbf{n}_2)^\Sigma_p\\
\end{align*}
Thus, it is in fact trivial to show, 
\begin{multline*}
    \pddt(m_p n_pk_p)
    + \div(m_pn_p k_p \textbf{u}_p 
    + \textbf{q}_p^\text{k})
    = 
     \bm{\sigma}_p^\text{eq}  :\grad \textbf{u}_p
     + n_p (\textbf{u}_\alpha' \cdot \bm{\sigma}_1^0 \cdot  \textbf{n}_2)^\Sigma_p
    \\
\end{multline*}
anyhow the pfp doesn't appear in this equation. 
This source term might be expressed as follow, 

\tb{The colision source term arise from the pair probability not the drag force so it cannot appear there unless we consider the expansion of the above term}

\subsection{The continuous phase easy}

Using the generic formulation \ref{eq:hybrid_avg_dt_chif} and the local expression of the mass, momentum and total energy expression, i.e. : \ref{eq:dt_rho},\ref{eq:dt_rhou_k} and \ref{eq:dt_rhoE_k} we easily find the averaged form of these equations as, 

\begin{align}
    \label{eq:dt_&vg_rho}
    \pddt (\phi_k \rho_k)  
    + \div (
        \phi_k \rho_k\textbf{u}_k
    )
    &= 
    0\\
    \label{eq:dt_&vg_rhou_k}
    \pddt (\phi_k \rho_k\textbf{u}_k)  
    + \div (
        \phi_k \rho_k\textbf{u}_k\textbf{u}_k
        - \bm{\sigma}_k^\text{eq}
    )
    &= 
    \phi_k \rho_k \textbf{g} 
    +  \avg{\delta_I \bm{\sigma}_k^0 \cdot \textbf{n}_k}\\
    \label{eq:dt_&vg_rhoE_k}
    \pddt (\phi_k\rho_kE_k)  
    + \div (
        \phi_k\rho_kE_k\textbf{u}_k
        + \bm{q}_k^\text{eq}
        - \textbf{u}_k \cdot \bm{\sigma}_k^\text{eq}
        % - \textbf{u}_k^0 \cdot \bm{\sigma}_k^0 
        % + \textbf{q}_k^0
        )
    &= 
    \phi_k \rho_k\textbf{u}_k \cdot \textbf{g} 
    + \avg{\delta_I (\textbf{u}_k^0 \cdot \bm{\sigma}_k^0 - \textbf{q}_k^0)\cdot \textbf{n}_k}
\end{align} 
\todo{Is this formulation usefull for the NRJ equaiton ? }
where we have defined, 
\begin{align*}
    &\bm{\sigma}_k^\text{eq}
    = \phi_k\rho_k (
        \bm{\sigma}_k%- n_p \textbf{M}_p
        - \kavg{\textbf{u}_k'\textbf{u}_k'})  
    &\textbf{q}_k^\text{eq}
    =\textbf{q}_k^\text{e} +\textbf{q}_k^\text{k}  \\
    &\textbf{q}_k^\text{e}
    = \phi_k\rho_k \kavg{\textbf{u}_k' e_k'} 
    + \phi_k\textbf{q}_k 
    &\textbf{q}_k^\text{k}
    = \phi_k\rho_k \kavg{\textbf{u}_k' k_k} 
    - \phi_k\kavg{\textbf{u}_k' \cdot \bm{\sigma}_k^0}
\end{align*}
Note that the phase averaged energy equation can be further decompose following, 
\begin{align*}
    E_1 = e_1 + k_1 + u_k^2/2
\end{align*}
where $K_1$ is the pseudo-turbulent kinetic energy defined such as, $\phi_1 k_1 = \avg{\chi_1 (u_1')^2/2}$. 
The Macroscopic kinetic energy equation can be obtain by taking the dot product with $\textbf{u}_k$. 
\begin{align}
    \pddt (\phi_k \rho_ku_k^2/2)  
    + \div (
        \phi_k \rho_k\textbf{u}_ku_k^2/2
        - \textbf{u}_k \cdot \bm{\sigma}_k^\text{eq}
    )
    &= 
    - \bm{\sigma}_k^\text{eq} : \grad \textbf{u}_k
    + \phi_k \rho_k \textbf{u}_k\cdot \textbf{g} 
    +  \textbf{u}_k\cdot \avg{\delta_I \bm{\sigma}_k^0 \cdot \textbf{n}_k}\\
    \pddt (\phi_k\rho_kk_k)  
    + \div (
        \phi_k\rho_kk_k\textbf{u}_k
        + \textbf{q}_k^\text{k} 
        )
    &= 
    - \avg{\chi_k\bm{\sigma}_k^0 : \grad \textbf{u}_k^0}
    + \bm{\sigma}_k^\text{eq} : \grad \textbf{u}_k
    + \avg{\delta_I \textbf{u}_k' \cdot \bm{\sigma}_k^0 \cdot \textbf{n}_k}\\
    \pddt (\phi_k\rho_ke_k)  
    + \div (
        \phi_k \rho_ke_k\textbf{u}_k
        +
        \textbf{q}_k^\text{e} 
        )
    &= 
    \avg{\chi_k\bm{\sigma}_k^0 : \grad \textbf{u}_k^0}
    - \avg{\delta_I \textbf{q}_k^0 \cdot \textbf{n}_k} 
\end{align}

\subsection{The continuous phase full flux for both}

Using the generic formulation \ref{eq:hybrid_avg_dt_chif} and the local expression of the mass, momentum and total energy expression, i.e. : \ref{eq:dt_rho},\ref{eq:dt_rhou_k} and \ref{eq:dt_rhoE_k} we easily find the averaged form of these equations as, 

\begin{align}
    \label{eq:dt_&vg_rho}
    \pddt (\phi_1 \rho_1)  
    + \div (
        \phi_1 \rho_1\textbf{u}_1
    )
    &= 
    0\\
    \label{eq:dt_&vg_rhou_1}
    \pddt (\phi_1 \rho_1\textbf{u}_1)  
    + \div (
        \phi_1 \rho_1\textbf{u}_1\textbf{u}_1
        + \bm{\sigma}_1^\text{eq}
    )
    &= 
    \phi_1 (\rho_1 \textbf{g} 
    + \div \bm{\sigma}_1 ) 
    +  \avg{\delta_I \bm{\sigma}_1' \cdot \textbf{n}_1}\\
    \label{eq:dt_&vg_rhoE_1}
    \pddt (\phi_1\rho_1E_1)  
    + \div (
        \phi_1\rho_1E_1\textbf{u}_1
        + \bm{q}_1^\text{eq}
        + \textbf{u}_1 \cdot \bm{\sigma}_1^\text{eq}
        % - \textbf{u}_1^0 \cdot \bm{\sigma}_1^0 
        % + \textbf{q}_1^0
        )
    &= 
    \phi_1 [\rho_1\textbf{u}_1 \cdot \textbf{g} 
    + \div(\textbf{u}_1 \cdot \bm{\sigma}_1 - \textbf{q}_1)]
    + \textbf{u}_1 \cdot\avg{\delta_I \bm{\sigma}_1'\cdot \textbf{n}_1}\nonumber\\
    &+ \avg{\delta_I (\textbf{u}_1' \cdot \bm{\sigma}_1^0 )\cdot \textbf{n}_1}
    - \avg{\delta_I\textbf{q}_1'\cdot \textbf{n}_1}
\end{align} 
\begin{align*}
    &\bm{\sigma}_1^\text{eq}
    = \phi_1\rho_1\kavg{\textbf{u}_1'\textbf{u}_1'}
    &\textbf{q}_1^\text{eq}
    =\textbf{q}_1^\text{e} +\textbf{q}_1^\text{k}  \\
    &\textbf{q}_1^\text{e}
    = \phi_1\rho_1 \kavg{\textbf{u}_1' e_1'} 
    &\textbf{q}_1^\text{k}
    = \phi_1\rho_1 \kavg{\textbf{u}_1' k_1} 
    - \phi_1\kavg{\textbf{u}_1' \cdot \bm{\sigma}_1^0}
\end{align*}
\tb{This formulation will separate the stresslet into two components in the momentum equation therefore it is not good}
\begin{align}
    \pddt (\phi_1 \rho_1u_1^2/2)  
    + \div (
        \phi_1 \rho_1\textbf{u}_1u_1^2/2
        + \textbf{u}_1 \cdot \bm{\sigma}_1^\text{eq}
    )
    &= 
    + \bm{\sigma}_1^\text{eq} : \grad \textbf{u}_1
    + \phi_1  \textbf{u}_1\cdot(\rho_1 \textbf{g} + \div \bm{\sigma}_1) 
    +  \textbf{u}_1\cdot \avg{\delta_I \bm{\sigma}_1' \cdot \textbf{n}_1}\\
    \pddt (\phi_1\rho_1k_1)  
    + \div (
        \phi_1\rho_1k_1\textbf{u}_1
        + \textbf{q}_1^\text{k} 
        )
    &= 
    - \textbf{d}_1
    + (\phi_1 \bm{\sigma}_1 - \bm{\sigma}_1^\text{eq} ): \grad \textbf{u}_1
    + \avg{\delta_I \textbf{u}_1' \cdot \bm{\sigma}_1^0 \cdot \textbf{n}_1}\\
    \pddt (\phi_1\rho_1e_1)  
    + \div (
        \phi_1 \rho_1e_1\textbf{u}_1
        +
        \textbf{q}_1^\text{e} 
        )
    &= 
    \textbf{d}_1
    - \phi_1 \div \textbf{q}_1
    - \avg{\delta_I \textbf{q}_1' \cdot \textbf{n}_1} 
\end{align}
\tb{The higher moment of surface and inter particular distance can then be extracted }


Regarding the dispersed phase we have  if we assume a certain error we obtain :
\begin{align*}
    \pddt \left(n_p m_p\right)
    + \div \left(n_pm_p\textbf{u}_p
    \right)
    = 
    0\\
    \pddt \left(n_p m_p \textbf{u}_p\right)
    + \div \left(n_p
    m_p \textbf{u}_p \textbf{u}_p 
    + \bm{\sigma}_p^\text{eq}
    \right)
    = 
    n_p v_p (\rho_2 \textbf{g}
    + \div \bm{\sigma}_1)
    + n_p (\bm{\sigma}_1' \cdot \textbf{n}_2)_p^\Sigma,\\
    \pddt(m_p n_pE_p^\text{tot})
    + \div(m_pn_p E_p^\text{tot} \textbf{u}_p 
    + \textbf{q}_p^\text{eq} 
    + \textbf{u}_p \cdot \bm{\sigma}_p^\text{eq})
    =  n_p v_p [\rho_2 \textbf{u}_p\cdot  \textbf{g}
    + \div (\textbf{u}_1 \cdot \bm{\sigma}_1 - \textbf{q}_1)]\\
    % +  n_p ( \textbf{u}'_1 \cdot \bm{\sigma}_1^0 \cdot \textbf{n}_2)_p^\Sigma
    -  n_p (\textbf{q}_1' \cdot \textbf{n}_2)_p^\Sigma
    +  n_p (\textbf{u}_1 \cdot \bm{\sigma}_1' \cdot \textbf{n}_2)_p^\Sigma
    +  n_p (\textbf{u}_1' \cdot \bm{\sigma}_1^0\cdot \textbf{n}_2)_p^\Sigma
\end{align*}
\begin{align*}
    &\bm{\sigma}_p^\text{eq}
    = m_p\pnavg{\textbf{u}_\alpha'\textbf{u}_\alpha'}
    &\textbf{q}_p^\text{eq}
    =\textbf{q}_p^\text{e} 
    +\textbf{q}_p^\text{k}  
    +\textbf{q}_p^\text{w}  
    \\
    &\textbf{q}_1^\text{e}
    = m_p \pnavg{\textbf{u}_\alpha' e_\alpha'} 
    &\textbf{q}_p^\text{k}
    = m_p \pnavg{\textbf{u}_\alpha' k_\alpha} 
    \\
    &\textbf{q}_p^\text{w}
    = 
    + \pnavg{\textbf{u}_\alpha'(\rho_2 (w^0_2)^2/2 )'_\Omega}
    + \gamma \pnavg{\textbf{u}_\alpha' s_\alpha'}
\end{align*}
Following the energy decomposition : 
\begin{align*}
    \avg{\delta_I (\bm{\sigma}_1^0 ) \textbf{n}_2} - p_1 \grad \phi_1
    = 
    % \avg{\delta_I (\bm{\sigma}_1^0 + p_1)\cdot \textbf{n}_2}
    % = 
    \avg{\delta_I \bm{\sigma}_1'\cdot \textbf{n}_2}
    \\
    \avg{\delta_I (\textbf{u}_1^0 \cdot\bm{\sigma}_1^0 )} - \textbf{u}_1p_1\cdot \grad \phi_1
    % = \avg{\delta_I (\textbf{u}_1 \cdot \bm{\sigma}_1' + \textbf{u}_1' \cdot \bm{\sigma}_1^0 )\cdot \textbf{n}_2}
    = \textbf{u}_1 \cdot \avg{\delta_I \bm{\sigma}_1'\cdot \textbf{n}_2}
    + \avg{\delta_I (\textbf{u}_1' \cdot \bm{\sigma}_1^0 )\cdot \textbf{n}_2}
\end{align*}
we obtain the follwoing system of equaiton 
\begin{align*}
    &\pddt \left(n_p m_p u_p^2/ 2\right)
    + \div \left(n_p
    m_p u_p^2/ 2 \textbf{u}_p 
    + \textbf{u}_p \cdot \bm{\sigma}_p^\text{eq}
    \right)
    = 
     \bm{\sigma}_p^\text{eq}  :\grad \textbf{u}_p
    +  n_p v_p \textbf{u}_p \cdot 
    (\rho_2 \textbf{g} + \div \bm{\sigma}_1)
    + n_p \textbf{u}_p \cdot (\bm{\sigma}_1' \cdot \textbf{n}_2)^\Sigma_p,\\
    &\pddt \left(n_p m_p (u_\alpha^2)_p/ 2\right)
    + \div \left(n_p
    m_p (u_\alpha^2)_p/ 2 \textbf{u}_p 
    + \textbf{q}^k_p
    + \textbf{u}_p \cdot \bm{\sigma}_p^\text{eq}
    \right)
    = 
    n_p v_p \textbf{u}_p \cdot
    (\rho_2 \textbf{g} + \div \bm{\sigma}_1)
    + n_p\textbf{u}_p\cdot (\bm{\sigma}_1' \cdot \textbf{n}_2)^\Sigma_p\\
    &+n_p(\textbf{u}_\alpha'\cdot(\bm{\sigma}_1^0 \cdot \textbf{n}_2)^\Sigma)_p
    \\
    &\pddt \left(n_p (\rho_2 w^2 )_p^\Omega+\gamma s_p n_p\right)
    + \div 
    (n_p (\rho_2 w^2 )_p^\Omega+\gamma s_p n_p
    \textbf{u}_p 
    +  \textbf{q}_p^\text{w}
    )
    = 
    - n_p (\bm{\sigma}_2^0 : \grad\textbf{u}_2^0)^\Omega_p\\
    &+ n_p (\textbf{u}_1^0 \cdot \bm{\sigma}_1^0 \cdot  \textbf{n}_2)^\Sigma_p
    - n_p (\textbf{u}_\alpha \cdot \bm{\sigma}_1^0 \cdot  \textbf{n}_2)^\Sigma_p\\
    &\pddt \left(n_p m_p e_p\right)
    + \div \left(n_p
    m_p e_p \textbf{u}_p 
    +  \textbf{q}_p^\text{e}
    \right)
    = 
    + n_p (\bm{\sigma}_2^0 : \grad\textbf{u}_2^0)^\Omega_p
    - n_p v_p \div \textbf{q}_1
    - n_p (\textbf{q}_1'\cdot \textbf{n}_2)^\Sigma_p\\
\end{align*}
Thus, the granular temperature equation yields 
\begin{multline*}
    \pddt(m_p n_pk_p)
    + \div(m_pn_p k_p \textbf{u}_p 
    + \textbf{q}_p^\text{k})
    = 
     - \bm{\sigma}_p^\text{eq}  :\grad \textbf{u}_p
     + n_p (\textbf{u}_\alpha' \cdot (\bm{\sigma}_1^0 \cdot  \textbf{n}_2)^\Sigma)_p
    \\
\end{multline*}
Te last term of this equaiton can be written : 
\begin{multline*}
    n_p (\textbf{u}_\alpha' \cdot (\bm{\sigma}_1^0 \cdot  \textbf{n}_2)^\Sigma)_p
    = n_p (\textbf{u}_\alpha' \cdot (\bm{\sigma}_1' \cdot  \textbf{n}_2)^\Sigma)_p
    = n_p (\textbf{u}_\alpha' \cdot \textbf{f}_\alpha)_p
    = n_p (\textbf{u}_\alpha' \cdot \textbf{f}_\alpha')_p
    \\
\end{multline*}
which correspond to the covariance between the force acting on the particle $\alpha$ and its center of mass velocity. 
Thus, if we assume that $\textbf{f}_\alpha \sim - C_p m_p  (\textbf{u}_\alpha - \textbf{u}_1)$ with $C_p $ the drag coefficient, both quantities are indeed correlated.
Indeed, we can write, $n_p (\textbf{u}_\alpha' \cdot \textbf{f}_\alpha' )_p = n_p A (\textbf{u}_\alpha' \cdot \textbf{u}_\alpha' )_p = - n_p m_p C_p k_p $

In an homogeneous unsteady regime such as homogeneous buoyant rising droplets, the transport equation for $k_p$ reads, 
\begin{multline*}
    \pddt(m_p n_pk_p)
    = 
    - n_p m_p C_p k_p
    \\
\end{multline*}
Thus, $k_p(t) = A_0 e^{- n_p m_p C_p t}$ with $A_0$ a constant depending on the initial condition. 
Consequently, unless $C_p = 0$, the kinetic energy of the particle phase will vanish at $t \rightarrow \infty$.  

However, $k_p$ can still decrease due to contact forces. 

However, if $\textbf{f}_\alpha$ contained the contact force between particles, $n_p (\textbf{u}_\alpha' \cdot \textbf{f}_\alpha')_p$ would still be $0$ since the contact force between particle isn't correlated with the velocity of the particles. 

Therefore, this equation lack of a  \textit{collision kernel}, or a source term representing the correlation between the force applied on the particle $\alpha$ and the position of its neighbors.
It could look like $(\textbf{r}' \textbf{f}_\alpha')$ where $\textbf{r}$ is teh position of the neighbor. 

In fact, this problem can be fix if we consider (Comme dans les papier de balachandar)
that the force $\textbf{f}_\alpha$ is function of the particle velocity, and the position and velocity of its nearest neighbor. 
Such that $\textbf{f}_\alpha(\textbf{u}_\alpha,\textbf{r})  = - C_p m_p  (\textbf{u}_\alpha - \textbf{u}_1) +  C_c(\textbf{u}_\alpha - \textbf{u}_\beta)\cdot(\textbf{x}_\alpha - \textbf{x}_\beta) \textbf{n}$ for smooth particles, where \textbf{n} is the normal vector in the direction of $\textbf{x}_\alpha - \textbf{x}_\beta$ and $C_p$ a coefficient relating the relative velocity to the collision force. 
Then, the source term of the equation for $k_p$ yields,
\begin{multline*}
    n_p (\textbf{u}_\alpha' \cdot \textbf{f}_\alpha)_p
    = - C_p m_p k_p
    +  C_c [\textbf{u}_\alpha' \cdot (\textbf{u}_\alpha - \textbf{u}_\beta)\cdot(\textbf{x}_\alpha - \textbf{x}_\beta) \textbf{n}]_p
    = - C_p m_p k_p
    +  C_c [\textbf{u}_\alpha'  \textbf{u}_\alpha' : (\textbf{x}_\alpha - \textbf{x}_\beta) \textbf{n}]_p
    \\
\end{multline*}
the last term correspond to the collision source term of kinetic theory. 
If we reformulate $(\textbf{x}_\alpha - \textbf{x}_\beta) \textbf{n} = r \textbf{n}\textbf{n}$ we get the desired results. 


is the force contribution due to a particle in $\textbf{r}$ (either solid collision or fluid mediated interaction). 
In this case the source term can be written ,

\section{The dumping model}

Let decompose the drag force such that $\textbf{f}_\alpha = \textbf{f}_\alpha^h + \textbf{f}_\alpha^c$ with $^h$ being the hydrodynamical forces and $\textbf{f}^c_\alpha$ being the contatc forces. 
Let assume that the particle $\alpha$ interact with the particle $\beta$, where both particles has a radius $a$.
Let note $\textbf{r} = (\textbf{x}_\alpha - \textbf{x}_\beta) = \textbf{n} a$. 
Now let consider that the contact force can be modeled as a spring/dumping model with $k$ being the \textit{Raideur} and $c$ the dumping coefficient.
Then, the contact force between both particles can be written as, 
\begin{equation}
    \textbf{f}_\beta^c
    = \textbf{f}_{\alpha\beta}
    = (|\textbf{r}| - a) \textbf{n} k 
    + (\textbf{u}_\alpha - \textbf{u}_\beta) c
    \;\;\;\;\text{for} \;\;\; (|\textbf{r}| - a) < 0
\end{equation}
For smooth particle the second term can be reduced to the components along the normal vector \textbf{n}.  
Due to Newton's  law and since each interaction force cancel each other we have $\avg{\textbf{f}_\beta^c} \approx 0$. 
However, it is useful to notice that we have in the case of inhomogeneous scenario $\avg{\textbf{f}_\beta^c} \approx - \div \Sigma^c$.

Now let's focus on the source term of the granular temperature $\avg{\textbf{f}_\alpha\cdot \textbf{u}_\alpha'} = \avg{\textbf{f}_\alpha^h \cdot \textbf{u}_\alpha'} + \avg{\textbf{f}_\alpha^c\cdot \textbf{u}_\alpha'}$. 
The first source term $\avg{\textbf{f}_\alpha^h \cdot \textbf{u}_\alpha'} \sim k_p$ since $\textbf{f}_\alpha \sim \textbf{u}_\alpha$.

The part due to collision can be re written using the nearest averaged statistics formalism, 
\begin{align*}
    \avg{\textbf{f}_\alpha^c\cdot \textbf{u}_\alpha'}
    &= \int_{\textsc{R}^3}
    \int \sum_{\alpha \neq \beta} \delta_\alpha \delta_\beta h_{\alpha\beta}
    \textbf{u}'_\alpha \textbf{f}_\alpha^c d\PP d\textbf{r}\\
    &= \int_{\textsc{R}^3}
    \int \sum_{\alpha \neq \beta} \delta_\alpha \delta_\beta h_{\alpha\beta}
    \textbf{u}'_\alpha \cdot \textbf{n}
        (|\textbf{r}| - a) 
     d\PP d\textbf{r}
    + \int_{\textsc{R}^3}
    \int \sum_{\alpha \neq \beta} \delta_\alpha \delta_\beta h_{\alpha\beta}
    \textbf{u}'_\alpha 
         (\textbf{u}_\alpha - \textbf{u}_\beta) c
     d\PP d\textbf{r}\\
    &=k \int_{|\textbf{r}|<a}
    \nstavg{\textbf{u}'_\alpha \cdot \textbf{n}
        (|\textbf{r}| - a)   }
        P_{nst}(\textbf{x},\textbf{r})
     d\textbf{r}
    +c \int_{|\textbf{r}|<a}
        \nstavg{\textbf{u}'_\alpha 
         \cdot (\textbf{u}_\alpha - \textbf{u}_\beta)} P_\text{nst}(\textbf{x},\textbf{r})
     d\textbf{r}\\
\end{align*}
Elastic collisions are not dissipative therefore the first term cancel by definition. 
Thus, we are left with, 
\begin{align*}
    \avg{\textbf{f}_\alpha^c\cdot \textbf{u}_\alpha'}
    = c \int_{|\textbf{r}|<a}
        \nstavg{\textbf{u}'_\alpha 
         \cdot (\textbf{u}_\alpha - \textbf{u}_\beta)} P_\text{nst}(\textbf{x},\textbf{r})
     d\textbf{r} 
\end{align*}
\subsection{The dispersed phase equations}

Regarding the dispersed phase, we found the mass, momentum and total energy balance equations, 
\begin{align*}
    \pddt \left(n_p m_p\right)
    + \div \left(n_pm_p\textbf{u}_p
    \right)
    = 
    0\\
    \pddt \left(n_p m_p \textbf{u}_p\right)
    + \div \left(n_p
    m_p \textbf{u}_p \textbf{u}_p 
    - \bm{\sigma}_p^\text{eq}
    \right)
    = 
    n_p v_p  (  
    \rho_2 \textbf{g}
    - \grad p_1)
    + n_p (\bm{\sigma}_1'\cdot \textbf{n}_2)_p^\Sigma,\\
    \pddt(m_p n_pE_p^\text{tot})
    + \div(m_pn_p E_p^\text{tot} \textbf{u}_p 
    + \textbf{q}_p^\text{eq} - \textbf{u}_p \cdot \bm{\sigma}_p^\text{eq})
    =  n_p v_p [\rho_2 \textbf{u}_p\cdot  \textbf{g} 
    - \div (\textbf{u}_1 p_1)]\\
    +  n_p ( \textbf{u}'_1 \cdot \bm{\sigma}_1^0 \cdot \textbf{n}_2)_p^\Sigma
    -  n_p (\textbf{q}_1^0 \cdot \textbf{n}_2)_p^\Sigma
    +  n_p (\textbf{u}_1 \cdot \bm{\sigma}_1'\cdot \textbf{n}_2)_p^\Sigma
\end{align*}
where we have defined, 
\begin{align*}
    &\bm{\sigma}_p^\text{eq}
    = -  m_p\pnavg{\textbf{u}_\alpha'\textbf{u}_\alpha'}
    &\textbf{q}_p^\text{eq}
    =\textbf{q}_p^\text{e} 
    +\textbf{q}_p^\text{k}  
    +\textbf{q}_p^\text{w}  
    \\
    &\textbf{q}_1^\text{e}
    = m_p \pnavg{\textbf{u}_\alpha' e_\alpha'} 
    &\textbf{q}_p^\text{k}
    = m_p \pnavg{\textbf{u}_\alpha' k_\alpha} 
    \\
    &\textbf{q}_p^\text{w}
    = 
    + \pnavg{\textbf{u}_\alpha'(\rho_2 (w^0_2)^2/2 )'_\Omega}
    + \gamma \pnavg{\textbf{u}_\alpha' s_\alpha'}
\end{align*}

Now, subtracting each of these equations to the total NRJ equations yields, 


\tb{Think about doing a surface equation ? ? }
At the Lagrangian scale, 
\begin{equation*}
    \pavg{\ddt (m_\alpha E_\alpha + s_\alpha \gamma)}
    = 
     n_p (\rho_2 \textbf{u}_2^0  \cdot \textbf{g})^\Omega_p
    +n_p (\textbf{u}_1^0 \cdot \bm{\sigma}_1^0 \cdot  \textbf{n}_2)^\Sigma_p
    - n_p (\textbf{q}_1^0 \cdot \textbf{n}_2)^\Sigma_p
\end{equation*}
\begin{align}
    \pavg{\frac{1}{2}\ddt (m_\alpha u_\alpha^2)}
    &= 
    n_p (\rho_2 \textbf{u}_2^0 \cdot
    \textbf{g})_p^\Omega
    + 
    (\textbf{u}_\alpha\cdot
    \textbf{f}_\alpha)_p\\
    \pavg{\frac{1}{2}\ddt \left[\int_{\Omega_\alpha} \rho_2 (w_2^0)^2 d\Omega +s_\alpha \gamma\right] }
    &= (\textbf{w}_1^0 \cdot (\bm{\sigma}_1^0 \cdot \textbf{n}_2) )_p^\Sigma  
     - (\bm{\sigma}_2^0 : \grad\textbf{u}_2^0)_p^\Omega  
    \\
    \pavg{\ddt (m_\alpha e_\alpha )}
    &= 
     + n_p (\bm{\sigma}_2^0 : \grad\textbf{u}_2^0)^\Omega_p
    -  n_p (\textbf{q}_1^0 \cdot \textbf{n}_2 )_p^\Sigma
\end{align}
We first notice that, 
\begin{equation*}
    n_p\frac{1}{2}(m_\alpha u_\alpha^2)_p
    = n_p\frac{1}{2}m_p u_p^2
    +  k_p
\end{equation*}
Deriving the momentum kinetic NRJ equation for the particle phase and the above else we can have the following system of equations. 
\begin{align*}
    &\pddt \left(n_p m_p u_p^2/ 2\right)
    + \div \left(n_p
    m_p u_p^2/ 2 \textbf{u}_p 
    - \textbf{u}_p \cdot \bm{\sigma}_p^\text{eq}
    \right)
    = 
    - \bm{\sigma}_p^\text{eq}  :\grad \textbf{u}_p
    +  n_p v_p \textbf{u}_p \cdot (
    \rho_2 \textbf{g}
    - \grad p_1 )
    + n_p \textbf{u}_p \cdot (\bm{\sigma}'_1 \cdot \textbf{n}_2)^\Sigma_p,\\
    &\pddt \left(n_p (\rho_2 w^2 )_p^\Omega+\gamma s_p n_p\right)
    + \div 
    (n_p (\rho_2 w^2 )_p^\Omega+\gamma s_p n_p)
    \textbf{u}_p 
    +  \textbf{q}_p^\text{w}
    )
    = \\
    &- n_p (\bm{\sigma}_2^0 : \grad\textbf{u}_2^0)^\Omega_p
    + n_p (\textbf{u}_1 \cdot \bm{\sigma}_1' \cdot  \textbf{n}_2)^\Sigma_p
    + n_p (\textbf{u}_1' \cdot \bm{\sigma}_1^0 \cdot  \textbf{n}_2)^\Sigma_p
    -n_p v_p \grad (\textbf{u}_1p_1)
    - n_p (\textbf{u}_\alpha \cdot \bm{\sigma}_1^0 \cdot  \textbf{n}_2)^\Sigma_p
    \\
    &\pddt \left(n_p m_p e_p\right)
    + \div \left(n_p
    m_p e_p \textbf{u}_p 
    +  \textbf{q}_p^\text{e}
    \right)
    = 
    + n_p (\bm{\sigma}_2^0 : \grad\textbf{u}_2^0)^\Omega_p
    - n_p (\textbf{q}_1^0\cdot \textbf{n}_2)^\Sigma_p\\
\end{align*}

Now if we subtract these from the total NRJ equation one can show, 
\begin{multline*}
    \pddt(m_p n_pk_p)
    + \div(m_pn_p k_p \textbf{u}_p 
    + \textbf{q}_p^\text{k})
    = 
     \bm{\sigma}_p^\text{eq}  :\grad \textbf{u}_p
     + n_p v_p \textbf{u}_p \grad p_1
     + n_p (\textbf{u}_\alpha \cdot \bm{\sigma}_1^0 \cdot  \textbf{n}_2)^\Sigma_p
     - n_p \textbf{u}_p \cdot (\bm{\sigma}_1' \cdot  \textbf{n}_2)^\Sigma_p
    \\
\end{multline*}
Notice that, 















In order to be consistent with the fluid phase equations these terms must be written with $\textbf{f}_\text{pm} = n_p\textbf{u}_1 \cdot ((\bm{\sigma}_1^0 \cdot  \textbf{n}_2)^\Sigma_{nst}(\textbf{x} \pm \textbf{r}/2,\mp\textbf{r}) )_p$, and espetially  $\textbf{f}_\text{pm} = n_p ((\textbf{u}_1' \cdot\bm{\sigma}_1^0 \cdot  \textbf{n}_2)^\Sigma_{nst}(\textbf{x} \pm \textbf{r}/2,\pm\textbf{r}))_p$. Thus we need to reformulate. 
We use, 
\begin{align*}
    n_p (\bm{\sigma}_1^0 \cdot  \textbf{n}_2)^\Sigma_p
    = 
    n_p ( \bm{\sigma}_1' \cdot  \textbf{n}_2)^\Sigma_p
    - n_p (p_1   \textbf{n}_2)^\Sigma_p\\
    n_p (\textbf{u}_1^0 \cdot \bm{\sigma}_1^0 \cdot  \textbf{n}_2)^\Sigma_p
    = 
    n_p (\textbf{u}_1 \cdot \bm{\sigma}_1' \cdot  \textbf{n}_2)^\Sigma_p
    + n_p (\textbf{u}_1' \cdot \bm{\sigma}_1^0 \cdot  \textbf{n}_2)^\Sigma_p
    - n_p (\textbf{u}_1 p_1 \cdot  \textbf{n}_2)^\Sigma_p
\end{align*}
Notice that $\textbf{u}_1$ and $p_1$ varies slowly inside the volume of the particle.
Consequently we must use the relation, $\textbf{u}_1(\textbf{r}) = \textbf{u}_1(\textbf{x}_\alpha) + \textbf{r} \cdot\grad \textbf{u}_1(\textbf{x}_\alpha) \ldots$
and $p_1(\textbf{r}) = p_1(\textbf{x}_\alpha) + \textbf{r} \cdot\grad p_1(\textbf{x}_\alpha) \ldots$
to finnaly obtain, 
\begin{align*}
    n_p (\bm{\sigma}_1^0 \cdot  \textbf{n}_2)^\Sigma_p
    &= 
    n_p ( \bm{\sigma}_1' \cdot  \textbf{n}_2)^\Sigma_p
    - n_p v_p \grad p_1\\
    n_p (\textbf{u}_1^0 \cdot \bm{\sigma}_1^0 \cdot  \textbf{n}_2)^\Sigma_p
    &= 
    n_p (\textbf{u}_1 \cdot \bm{\sigma}_1' \cdot  \textbf{n}_2)^\Sigma_p
    + n_p (\textbf{u}_1' \cdot \bm{\sigma}_1^0 \cdot  \textbf{n}_2)^\Sigma_p
    - n_p v_p \div (\textbf{u}_1 p_1) \\
    &= 
    n_p \textbf{u}_1 \cdot( \bm{\sigma}_1' \cdot  \textbf{n}_2)^\Sigma_p
    + n_p (\textbf{r} \bm{\sigma}_1' \cdot  \textbf{n}_2)^\Sigma_p : \grad \textbf{u}_1
    + n_p (\textbf{u}_1' \cdot \bm{\sigma}_1^0 \cdot  \textbf{n}_2)^\Sigma_p
    - n_p v_p \div (\textbf{u}_1 p_1) \\
    &= 
    n_p \textbf{u}_1 \cdot \textbf{f}_{pm}
    + n_p (\mathcal{F}_p - \mathcal{F}_\text{pfp}): \grad \textbf{u}_1
    + n_p \textbf{c}_\text{pm}
    + \div [n_p(\mathcal{C}_\text{pfp} + \textbf{u}_1 \cdot \mathcal{F}_\text{pfp})]
    - n_p v_p \div (\textbf{u}_1 p_1) 
\end{align*}

\begin{align*}
    n_p (\textbf{w}_1^0 \cdot \bm{\sigma}_1^0 \cdot  \textbf{n}_2)^\Sigma_p
    &= 
    n_p (\textbf{u}_1^0 \cdot \bm{\sigma}_1^0 \cdot  \textbf{n}_2)^\Sigma_p
    - n_p (\textbf{u}_\alpha \cdot \bm{\sigma}_1^0 \cdot  \textbf{n}_2)^\Sigma_p\\
    n_p (\textbf{u}_\alpha \cdot \bm{\sigma}_1^0 \cdot  \textbf{n}_2)^\Sigma_p
    &=
    n_p (\textbf{u}_\alpha \cdot \bm{\sigma}_1' \cdot  \textbf{n}_2)^\Sigma_p
    - n_p (\textbf{u}_\alpha \cdot p_1 \cdot  \textbf{n}_2)^\Sigma_p
    % n_p (\textbf{u}_1 \cdot \bm{\sigma}_1' \cdot  \textbf{n}_2)^\Sigma_p
    % + n_p (\textbf{u}_1' \cdot \bm{\sigma}_1^0 \cdot  \textbf{n}_2)^\Sigma_p
    % - n_p v_p \div (\textbf{u}_1 p_1) \\
\end{align*}

In fact if we start back from the exact relation defined in the continuous pahse we can say that ,
\begin{align*}
    \avg{\delta_I (\bm{\sigma}_1^0 ) \textbf{n}_2} - p_1 \grad \phi_1
    = 
    % \avg{\delta_I (\bm{\sigma}_1^0 + p_1)\cdot \textbf{n}_2}
    % = 
    \avg{\delta_I \bm{\sigma}_1'\cdot \textbf{n}_2}
    \\
    \avg{\delta_I (\textbf{u}_1^0 \cdot\bm{\sigma}_1^0 )} - \textbf{u}_1p_1\cdot \grad \phi_1
    % = \avg{\delta_I (\textbf{u}_1 \cdot \bm{\sigma}_1' + \textbf{u}_1' \cdot \bm{\sigma}_1^0 )\cdot \textbf{n}_2}
    = \textbf{u}_1 \cdot \avg{\delta_I \bm{\sigma}_1'\cdot \textbf{n}_2}
    + \avg{\delta_I (\textbf{u}_1' \cdot \bm{\sigma}_1^0 )\cdot \textbf{n}_2}
\end{align*}

incoherence in teh exchange terms, 
\begin{align*}
    n_p (\textbf{u}_\alpha' \cdot \bm{\sigma}_1^0\cdot \textbf{n}_2)^\Sigma_p
    &= \int
    \sum_\alpha \delta_\alpha(\textbf{x} - \textbf{x}_\alpha)
    \textbf{u}_\alpha'(\CC,t)\cdot
    \left[\int_{\Sigma_\alpha} 
     (\bm{\sigma}_1^0 +p_1 \textbf{I})\cdot \textbf{n}_2
     d\textbf{r}
    - \int_{\Sigma_\alpha} 
     p_1  \textbf{n}_2
     d\textbf{r}\right]
     d\PP \\
    &= \int
    \sum_\alpha \delta_\alpha(\textbf{x} - \textbf{x}_\alpha)
    \textbf{u}_\alpha'(\CC,t)\cdot
    \left[\int_{\Sigma_\alpha} 
     \bm{\sigma}_1' +\cdot \textbf{n}_2
     d\textbf{r}
    - v_\alpha \grad p_1(\textbf{x}_\alpha)
    \right]
     d\PP \\
    &= n_p (\textbf{u}_\alpha' \cdot (\bm{\sigma}'_1\cdot \textbf{n}_2)^\Sigma )_p
    -  n_p (\textbf{u}_\alpha' v_p \cdot \grad p_1 )_p
     \\
    &= n_p (\textbf{u}_\alpha' \cdot (\bm{\sigma}'_1\cdot \textbf{n}_2)^\Sigma )_p
     \\
\end{align*}
Since $\textbf{u}_p$ is an Eulerian fields it must be evaluated at $\textbf{r}$ Therefore it must get out the 











The averaged particle energy $n_p E_p$ can be split into five components,
\begin{equation*}
    n_p m_p E_p(t) 
    = m_p n_p e_p 
    + \pnavg{\int_{\Omega_\alpha(t)} \rho_2  (w_2^0)^2/2 d\Omega}
    + m_p n_p k_p
    + m_p n_p (u_p)^2/2
    + n_p s_p \gamma
    % + \textbf{u}_\alpha \cdot \int_{\Omega_\alpha(t)} \rho_2  \textbf{w}_2^0 d\Omega
\end{equation*}
where $k_p = \pavg{(u_\alpha')^2/2}$.
one equation for each is riquiered 
Using the mass balance and the momentum balance dotted with $\textbf{u}_p$ we obtain the particle kinetic energy balance, 
\begin{align*}
    \pddt \left(n_p m_p u_p^2/ 2\right)
    + \div \left(n_p
    m_p u_p^2/ 2 \textbf{u}_p 
    - \textbf{u}_p \cdot \bm{\sigma}_p^\text{eq}
    \right)
    = 
    - \bm{\sigma}_p^\text{eq}  :\grad \textbf{u}_p
    +  n_p v_p \textbf{u}_p \cdot (
    \rho_2 \textbf{g}
    - \grad p_1 )
    + n_p \textbf{u}_p \cdot \textbf{f}_{pm},\\
    \pddt \left(n_p (\rho_2 w^2 )_p^\Omega+\gamma s_p n_p\right)
    + \div 
    (n_p (\rho_2 w^2 )_p^\Omega+\gamma s_p n_p)
    \textbf{u}_p 
    +  \textbf{q}_p^\text{w}
    )
    = 
    - n_p \textbf{d}_p
    +  n_p (\textbf{u}_1 -\textbf{u}_p) \cdot  (\textbf{f}_{pm} - v_p \grad p_1)
    + n_p\textbf{c}_p\\
    \pddt \left(n_p m_p e_p\right)
    + \div \left(n_p
    m_p e_p \textbf{u}_p 
    +  \textbf{q}_p^\text{e}
    \right)
    = 
    + n_p \textbf{d}_p
    + n_p \textbf{e}_{pm},\\
\end{align*}
\tb{we remark that the collision tensor appear exactly at the same place as the pfp}
Subtracting all 3 equation to the total energy finally gives, 
\begin{align*}
    \pddt(m_p n_pk_p)
    + \div(m_pn_p k_p \textbf{u}_p 
    + \textbf{q}_p^\text{k})
    = 
    \bm{\sigma}_p^\text{eq} : \grad \textbf{u}_p
    % - n_p \textbf{d}_p
    - n_p v_p p_1 \div \textbf{u}_1
    + n_p (( \textbf{u}_\alpha' \cdot \bm{\sigma}_1^0 \cdot \textbf{n}_2)^\Sigma_\text{nst}(\textbf{x}\pm\textbf{r}/2,\mp\textbf{r}) )_p^r \\
\end{align*}
The transfer term of the internal droplets' fluctuation reads, 
\begin{align*}
    \int_{\Sigma_\alpha} \textbf{w}_1^0 \cdot (\bm{\sigma}_1^0 \cdot \textbf{n}_2) d\Sigma  
    = 
    (\textbf{u}_1 -\textbf{u}_\alpha) \cdot \int_{\Sigma_\alpha}  (\bm{\sigma}_1^0 \cdot \textbf{n}_2) d\Sigma  
    + \int_{\Sigma_\alpha} \textbf{u}_1' \cdot (\bm{\sigma}_1^0 \cdot \textbf{n}_2) d\Sigma  
    = (\textbf{u}_1 -\textbf{u}_\alpha) \cdot  (\textbf{f}_\alpha - \grad p_1)
    + \textbf{c}_\alpha
\end{align*}
\begin{align*}
    n_p (\textbf{w}_1^0 \cdot \bm{\sigma}_1^0 \cdot \textbf{n}_2)^\Sigma_p
    &= 
    n_p ((\textbf{w}_1^0 \cdot \bm{\sigma}_1^0 \cdot \textbf{n}_2)^\Sigma_\text{nst}(\textbf{x}\pm\textbf{r}/2,\mp\textbf{r}) )_p^r 
    + \div (n_p ( \textbf{r}(\textbf{w}_1^0 \cdot \bm{\sigma}_1^0 \cdot \textbf{n}_2)^\Sigma_\text{nst}(\textbf{x},\textbf{r}) )_p^r )\\
    &= 
    n_p (((\textbf{u}_1-\textbf{u}_\alpha) \cdot \bm{\sigma}_1^0 \cdot \textbf{n}_2)^\Sigma_\text{nst}(\textbf{x}\pm\textbf{r}/2,\mp\textbf{r}) )_p^r 
    +n_p ((\textbf{u}_1' \cdot \bm{\sigma}_1^0 \cdot \textbf{n}_2)^\Sigma_\text{nst}(\textbf{x}\pm\textbf{r}/2,\mp\textbf{r}) )_p^r \\
    &+ \div [n_p ( \textbf{r}((\textbf{u}_1 - \textbf{u}_\alpha) \cdot \bm{\sigma}_1^0 \cdot \textbf{n}_2)^\Sigma_\text{nst}(\textbf{x},\textbf{r}) )_p^r 
    + n_p ( \textbf{r}(\textbf{u}_1' \cdot \bm{\sigma}_1^0 \cdot \textbf{n}_2)^\Sigma_\text{nst}(\textbf{x},\textbf{r}) )_p^r ]\\
    &= 
    n_p (((\textbf{u}_1-\textbf{u}_\alpha) \cdot \bm{\sigma}_1^0 \cdot \textbf{n}_2)^\Sigma_\text{nst}(\textbf{x}\pm\textbf{r}/2,\mp\textbf{r}) )_p^r 
    +n_p \textbf{c}_{pm} \\
    &+ \div [n_p ( \textbf{r}((\textbf{u}_1 - \textbf{u}_\alpha) \cdot \bm{\sigma}_1^0 \cdot \textbf{n}_2)^\Sigma_\text{nst}(\textbf{x},\textbf{r}) )_p^r 
    + n_p \mathcal{C}_\text{pfp}]\\
\end{align*}
The remaining terms can be expressed as, 
\begin{align*}
    n_p (((\textbf{u}_1-\textbf{u}_\alpha) \cdot \bm{\sigma}_1^0 \cdot \textbf{n}_2)^\Sigma_\text{nst}(\textbf{x}\pm\textbf{r}/2,\mp\textbf{r}) )_p^r 
    &= 
    n_p (\textbf{u}_1 - \textbf{u}_p) \cdot (( \bm{\sigma}_1^0 \cdot \textbf{n}_2)^\Sigma_\text{nst}(\textbf{x}\pm\textbf{r}/2,\mp\textbf{r}) )_p^r \\
    &- n_p (( \textbf{u}_\alpha' \cdot \bm{\sigma}_1^0 \cdot \textbf{n}_2)^\Sigma_\text{nst}(\textbf{x}\pm\textbf{r}/2,\mp\textbf{r}) )_p^r \\
    &= 
    n_p (\textbf{u}_1 - \textbf{u}_p) \cdot(\textbf{f}_p - \grad p_1) \\
    &- n_p (( \textbf{u}_\alpha' \cdot \bm{\sigma}_1^0 \cdot \textbf{n}_2)^\Sigma_\text{nst}(\textbf{x}\pm\textbf{r}/2,\mp\textbf{r}) )_p^r \\
\end{align*}
The higher moments terms can be expressed as, 
\begin{align*}
    n_p (\textbf{r}((\textbf{u}_1-\textbf{u}_\alpha) \cdot \bm{\sigma}_1^0 \cdot \textbf{n}_2)^\Sigma_\text{nst})_p^r 
    &= 
    n_p (\textbf{u}_1 - \textbf{u}_p) \cdot (\textbf{r} ( \bm{\sigma}_1^0 \cdot \textbf{n}_2)^\Sigma_\text{nst} )_p^r 
    - n_p ( \textbf{r} ( \textbf{u}_\alpha' \cdot \bm{\sigma}_1^0 \cdot \textbf{n}_2)^\Sigma_\text{nst} )_p^r \\
    &= 
    n_p (\textbf{u}_1 - \textbf{u}_p)\cdot \mathcal{F}_\text{pfp} 
    - n_p (\textbf{r} ( \textbf{u}_\alpha' \cdot \bm{\sigma}_1^0 \cdot \textbf{n}_2)^\Sigma_\text{nst} )_p^r \\
\end{align*}
Alternatively, without the nearest particle formalism we obtain, 
\begin{align*}
    n_p (\textbf{w}_1^0 \cdot \bm{\sigma}_1^0 \cdot \textbf{n}_2)^\Sigma_p
    &= 
    n_p ((\textbf{w}_1^0 \cdot \bm{\sigma}_1^0 \cdot \textbf{n}_2)^\Sigma)_p 
\end{align*}

Averaging the microscopic 
\begin{align}
    \frac{1}{2}\ddt (m_\alpha u_\alpha^2)
    &= 
    \textbf{u}_\alpha\cdot
    \textbf{g}m_\alpha
    + 
    \textbf{u}_\alpha\cdot
    \textbf{f}_\alpha\\
    \frac{1}{2}\ddt \int_{\Omega_\alpha} \rho_2 (w_2^0)^2 d\Omega 
    + \ddt (s_\alpha \gamma) 
    &= 
    \int_{\Sigma_\alpha} \textbf{w}_1^0 \cdot (\bm{\sigma}_1^0 \cdot \textbf{n}_2) d\Sigma  
     - \int_{\Omega_\alpha} \bm{\sigma}_2^0 : \grad\textbf{u}_2^0 d\Omega  
    \\
    \ddt (m_\alpha e_\alpha )
    &= 
     \int_{\Omega_\alpha} \bm{\sigma}_2^0 : \grad\textbf{u}_2^0 d\Omega  
    -  s_\alpha \textbf{q}_\alpha  
\end{align}
Additionally, we can add an equation for the first moment of momentum, 
\begin{multline}
    \pddt \left(n_p \mathcal{P}_p\right)
    + \div \left(
        n_p \textbf{u}_p \mathcal{P}_p
    + \Sigma_p^\text{eq}
    \right)
    =
    n_p v_p \bm{\sigma}_1 
    + n_p \mathcal{F}_p\\
    +\pnavg{\int_{\Omega_\alpha} \left(
        \rho_2 \textbf{w}_2^0  \textbf{w}_2^0 
        - \bm{\sigma}_2^0
        \right) d\Omega}
        - \gamma  \pnavg{\int_{\Sigma_\alpha} \textbf{I}_{||} d\Sigma},
\end{multline}



\subsubsection{Modeling of collisions}

Even through we do not consider pure contact between interfaces it is still indispensable to define some kind of collision with the framework of the hybrid model. 
A contact mediated by the fluid is still different from near close contact, since in the latter case it is capillary pressure that drives the interaction forces. 

\subsection*{The drag force term}

The drag force term is easily closed by numerical method and some theoretical developments in the limiting case. 
Let now study the stokes 

\subsection*{Stress tensor for the continuous phase }
Regarding the fluid stress it can be reformulated considering Newtonian fluid,
\begin{equation}
    \phi_1 \bm{\sigma}_1 
    = - \phi_1 p_1 \textbf{I}
    + \mu_1 \phi_1 \textbf{e}_1
\end{equation}
with $\textbf{e}_1$ being the averaged shear rate. 
The first model is then, 
\begin{align*}
    \phi_1 \textbf{e}_1
    = \phi_1 (\nabla \textbf{u}_1+ (\grad \textbf{u}_1)^T)
    + \avg{[(\textbf{u}_1^0 - \textbf{u}_1)  \textbf{n}_1 +  \textbf{n}_1(\textbf{u}_1^0 - \textbf{u}_1 )]\delta_I}
\end{align*}
In \citet[chap 9]{ishii1975thermo} they assume,
\begin{equation}
    \avg{[(\textbf{u}_1^0 - \textbf{u}_1)  \textbf{n}_1 +  \textbf{n}_1(\textbf{u}_1^0 - \textbf{u}_1 )]\delta_I}\\
    = 
    (\textbf{u}_2 - \textbf{u}_1)  \grad \phi_1 +  \grad \phi_1(\textbf{u}_2 - \textbf{u}_1 )\\
\end{equation}
But I didn't find out where the derivation came from. 
Alternatively we can say that, 
\begin{align*}
    \phi_1 \textbf{e}_1
    = \nabla \textbf{u}+ (\grad \textbf{u})^T
    - \avg{\chi_2 (\grad\textbf{u}_2^0 + \grad(\textbf{u}_2^0 )^T)}
    = \textbf{e}
    - \phi_2 \textbf{e}_2
\end{align*}

More generally the stress within a suspension can be written,
\begin{align*}
    \bm{\sigma}_1 \phi_1
    &=- \phi_1 p_1 \textbf{I}
    + \mu_1 \textbf{e}
    - \lambda \phi_2 \bm{\tau}_2\\
    \bm{\sigma}_1 
    &= - \left(p_1 + \frac{\lambda \phi_2}{\phi_1} p_2\right) \textbf{I}
    + \frac{\mu_1}{\phi_1} \textbf{e}
    - \frac{\lambda \phi_2}{\phi_1} \bm{\sigma}_2\\
    \bm{\sigma}
    &= - \phi_1 p_1  \textbf{I}
    + \mu_1 \textbf{e}
    + \bm{\sigma}_2 \phi_2 
    +\phi_I \bm{\sigma}_I 
    - \lambda \phi_2 \bm{\tau}_2
\end{align*}
We can reformulate the last expression in the usual way using the first moment of momentum eq, 
\begin{equation}
    -  \dot{\mathcal{P}_p}
    +  \mathscr{S}_p^*
    +  \mathscr{L}_p
    + \frac{1}{3}(\bm{\sigma}_1^0 \cdot \textbf{n}_2 \cdot \textbf{r})_p^\Sigma \textbf{I}
    + n_p (\rho_2 \textbf{w}_2^0  \textbf{w}_2^0 )^\Omega
    =   (\bm{\sigma}_2^0)^\Omega
    + (\bm{\sigma}_I)^\Sigma,
\end{equation}
Or in stokes condition, 
\begin{equation}
    n_p \mathscr{S}_p^*
+ n_p \mathscr{L}_p
+ n_p\frac{1}{3}(\bm{\sigma}_1^0 \cdot \textbf{n}_2 \cdot \textbf{r})_p^\Sigma \textbf{I}
    = n_p \left(
        \bm{\sigma}_2^0
    \right)_p^\Omega
    +n_p (\bm{\sigma}_I)^\Sigma_p
\end{equation}
where we defined, 
\begin{align*}
    \mathscr{S}_p^* =\frac{1}{2} \pnavg{\int_{\Sigma_\alpha} \left(
        \textbf{r} \bm{\sigma}_1^0 \cdot \textbf{n}_2
        +  \bm{\sigma}_1^0 \cdot \textbf{n}_2\textbf{r}
        -
          \frac{2}{3}(\bm{\sigma}_1^0 \cdot \textbf{n}_2 \cdot \textbf{r})\textbf{I}
        \right)  d\Sigma}\\
    \mathscr{L}_p =\frac{1}{2} \pnavg{\int_{\Sigma_\alpha} \left(
        \textbf{r} \bm{\sigma}_1^0 \cdot \textbf{n}_2
        - \bm{\sigma}_1^0 \cdot \textbf{n}_2\textbf{r}
        \right) d\Sigma}
\end{align*}
Thus in homogeneous suspension without inertia we have, 
\begin{align*}
    \bm{\sigma}
    &= [- \phi_1 p_1 
    + n_p (\bm{\sigma}_1^0 \cdot \textbf{n}_2 \cdot \textbf{r})^\Sigma_p] \textbf{I}
    + \mu_1 \textbf{e}
    + n_p \mathscr{S}
    + n_p \mathscr{L}
\end{align*}
where the stress let is defined as $\mathscr{S} = \mathscr{S}_p^* - \lambda \phi_2 \bm{\tau}_2$. 
The equivalent stress in the fluid phase averaged equation can be reformulated as, 
\begin{align*}
    \bm{\sigma}_1^\text{eq}
    = \phi_1(
    \bm{\tau}_1%- n_p \textbf{M}_p
    - \rho_1 
    \kavg{\textbf{u}_1'\textbf{u}_1'})
    - n_p \mathcal{F}_\text{pfp} + n_p \mathcal{F}_p
    &= - (\phi_1 \rho_1  \kavg{\textbf{u}_1'\textbf{u}_1'}
        + n_p \mathcal{F}_\text{pfp})
        + \mu_1 \textbf{e} 
        - \lambda \phi_2 \bm{\tau}_2
         + n_p \mathcal{F}_p\\
    &= - (\phi_1 \rho_1  \kavg{\textbf{u}_1'\textbf{u}_1'}
        + n_p \mathcal{F}_\text{pfp})
        + \mu_1 \textbf{e} 
        - \lambda \phi_2 \bm{\tau}_2
         + n_p \mathscr{S}_p^*
         + n_p \mathscr{L}_p\\
    &= - (\phi_1 \rho_1  \kavg{\textbf{u}_1'\textbf{u}_1'}
        + n_p \mathcal{F}_\text{pfp})
        + \mu_1 \textbf{e} 
         + n_p \mathscr{S}_p
         + n_p \mathscr{L}_p
         + n_p (\bm{\sigma}_1^0 \cdot \textbf{n}_2 \cdot\textbf{r})_p^\Sigma \textbf{I}
\end{align*}
Thus, in the most general way the fluid phase stress can be written as that. 
But the last term must go into the equivalent pressure and that is a major founding. 
For netrally buoyant spherical particles : 
\begin{equation*}
    + n_p (\bm{\sigma}_1^0 \cdot \textbf{n}_2 \cdot\textbf{r})_p^\Sigma \textbf{I}
    = 
    n_p/a (p_1^0 )_p^\Sigma \textbf{I}
\end{equation*}
This, is definitely not trivial but if one wish to compute the first moment dynamical contribution to the suspension the formulas is given by 
\begin{align*}
    n_p \mathscr{S}_p
+ n_p \mathscr{L}_p
+ n_p\frac{1}{3}(\bm{\sigma}_1^0 \cdot \textbf{n}_2 \cdot \textbf{r})_p^\Sigma \textbf{I}
    &= 
    n_p \left(
        \bm{\sigma}_2^0
    \right)_p^\Omega
    +n_p (\bm{\sigma}_I)^\Sigma_p
    - \lambda \phi_2 \bm{\tau}_2\\
    &= 
    - n_p \left(
        p_2^0
    \right)_p^\Omega \textbf{I}
    +n_p (\bm{\sigma}_I)^\Sigma_p
    + n_p (1 - \lambda)\left(
        \mu_2 \textbf{e}_2^0
    \right)_p^\Omega 
\end{align*}
Let take the trace times $\frac{1}{3}$ of this equation, 
\begin{align*}
    \frac{1}{3} n_p(\bm{\sigma}_1^0 \cdot \textbf{n}_2 \cdot \textbf{r})_p^\Sigma 
    = 
    - n_p \left(
        p_2^0
    \right)_p^\Omega 
    +n_p \frac{1}{3}(\bm{\sigma}_I)^\Sigma_p : \textbf{I}
\end{align*}
Now let's substitute this equation into the former one, 
\begin{align*}
    n_p \mathscr{S}_p
+ n_p \mathscr{L}_p
=
    +n_p (\bm{\sigma}_I - \frac{1}{3}(\bm{\sigma}_I : \textbf{I})\text{I})^\Sigma_p
    + n_p (1 - \lambda)\left(
        \mu_2 \textbf{e}_2^0
    \right)_p^\Omega 
\end{align*}

If the particle is spherical, and that we remove the isotropic part on both sides of the equation we obtain 

In the spherical particle case the symmetric part reads, 
\begin{align*}
    n_p \mathscr{S}_p
    &= 
    + n_p (1 - \lambda)\left(
        \mu_2 \textbf{e}_2^0
    \right)_p^\Omega 
\end{align*}
which is false. 

\subsubsection{A translating sphere}
In the dilute stokes regime the disturbance velocity around a droplet can be written, 
\begin{align*}
    u_i^\text{Ext}(\textbf{r})
    = \left(\frac{\delta_{ik}}{r} + \frac{r_ir_k}{r^3}\right)  g_k
    + \left(-\frac{\delta_{ik}}{r^3} + \frac{3r_ir_k}{r^5}\right)  d_k\\
    u_i^\text{In}(\textbf{r})
    = c_i
    + \left(2 r^2 \delta_{ik} - r_ir_k\right) e_k\\
    e_{ik}
    = \mu(
        3 \delta_{ij} r_k 
        + 3 \delta_{kj} r_i
        -2 r_j \delta_{ki}
    )e_j 
\end{align*}
Applying the non deformation at the interface and other shear free condition we find the constant to be, 
\begin{align*}
    &\textbf{g} = \frac{1}{4}\left(\frac{3\lambda + 2}{\lambda +1}\right) a \textbf{U}
    &\textbf{d} = -\frac{1}{4}\left(\frac{\lambda}{\lambda +1}\right) a^3 \textbf{U}\\
    &\textbf{c} = \frac{1}{2}\left(\frac{3\lambda + 3}{\lambda +1}\right) \textbf{U}
    &\textbf{e} = -\frac{1}{2}\left(\frac{\lambda}{\lambda +1}\right) \frac{1}{a^2} \textbf{U}\\
\end{align*}
Let consider isolated particles immersed in a viscous flow in the case $\textbf{u}_1' = \textbf{u}^{Ext}$.

The averaged internal shear rate $\phi_2 \textbf{e}_2$ can be thus estimated through the integral, 
\begin{align*}
    \avg{\chi_2 (\textbf{e}_2^0)_{ik}}
    &= \pavg{\int_{\Omega} \mu(
        3 \delta_{ij} r_k 
        + 3 \delta_{kj} r_i
        -2 r_j \delta_{ki}
    )e_j d\Omega}
    = 0\\
    &+ \div \pavg{\int_{\Omega} \textbf{r}\mu(
        3 \delta_{ij} r_k 
        + 3 \delta_{kj} r_i
        -2 r_j \delta_{ki}
    )e_j d\Omega}
\end{align*}
Also, the stress fields for such a flow is given by, 
\begin{equation*}
    T^G_{ijl} 
    = -6\frac{r_ir_jr_k}{r^5}
\end{equation*}

What about the first moment of momentum of the droplets, 
\begin{align*}
    (\mathcal{P}_p)_{ij}
    = \int_{\Omega} 
    u_i^\text{In} r_j 
    d\Omega
    = \int_{\Omega} 
    (c_i r_j 
    + 2 r^2  r_j e_i - r_i r_k r_j e_k)
    d\Omega
    = 0 
\end{align*}
Where we considered spherical particle with no deformation so it is obviously zero.
\begin{align*}
    (\textbf{w}_2^0 \textbf{w}_2^0)_{ij}^\Omega
    = \int_{\Omega} 
    u_i^\text{In}u_j^\text{In}
    d\Omega
    = \int_{\Omega} 
    ( c_i + \left(2 r^2 \delta_{ik} - r_ir_k\right) e_k)
    ( c_j + \left(2 r^2 \delta_{jl} - r_jr_l\right) e_l)
    d\Omega\\
    = \int_{\Omega} 
    (c_i c_j + c_i e_l (2 r^2 \delta_{jl} - r_jr_l )
    + (2 r^2 \delta_{ik} - r_ir_k)c_j e_k
    +  (2 r^2 \delta_{ik} - r_ir_k)(2 r^2 \delta_{jl} - r_jr_l)e_ke_l
    )
    d\Omega\\
    = c_i c_j v_\alpha
    + e_l c_i (2 \mathcal{M}_{kk} \delta_{jl} - \mathcal{M}_{jl})
    + e_k c_j (2 \mathcal{M}_{kk} \delta_{ik} - \mathcal{M}_{ik})\\
    + e_ke_l (\mathcal{M}_{kkkk}\delta_{ik}\delta_{jl}
    -2\mathcal{M}_{kkjl}\delta_{ik}
    -2\mathcal{M}_{kkik}\delta_{jl}
    + \mathcal{M}_{ikjl}) 
\end{align*}
\tb{problem on the units }

\subsubsection{A drop in shear flow}

The functional form of the internal velocity fields for a droplet immersed in a shear flow is,
\begin{align*}
    u_i^\text{Ext}(\textbf{r})
    = \left(\frac{\delta_{ij} r_l - \delta_{il} r_j - \delta_{jl} r_i}{r^3} 
    + \frac{r_ir_jr_l}{r^5}\right)  d_{jl}\\
    + \left(-3 \frac{\delta_{ij} r_l + \delta_{il} r_j + \delta_{jl} r_i}{r^5} 
    + 15\frac{r_ir_jr_l}{r^7}\right)  p_{jl}\\
    u_i^\text{In}(\textbf{r})
    = \left(- 4 \delta_{ij} r_l  + \delta_{il} r_j + \delta_{jl} r_i\right) f_{jl}\\
    e_{ik}
    = \mu [
        \partial_i \left(- 4 \delta_{kj} r_l  + \delta_{kl} r_j + \delta_{jl} r_k\right) f_{jl} + \partial_k \left(- 4 \delta_{ij} r_l  + \delta_{il} r_j + \delta_{jl} r_i\right) f_{jl} 
    ]\\
    = \mu [
        \left(- 4 \delta_{kj} \delta_{li}  + \delta_{kl} \delta_{ji} + \delta_{jl} \delta_{ki}\right) f_{jl} + \left(- 4 \delta_{ij} \delta_{lk}  + \delta_{il} \delta_{kj} + \delta_{jl} \delta_{ki}\right) f_{jl} 
    ]\\
    = \mu [
        \left(- 4 f_{ki}  + f_{ik} + f_{jj}\delta_{ki}\right)  + \left(- 4 f_{ik}  + f_{ki} + f_{jj} \delta_{ki}\right)  
    ]\\
    = \mu [
        \left(- 3 f_{ki}  - 3 f_{ik} +2 f_{jj}\delta_{ki}\right) 
    ]
\end{align*}
\todo[inline]{include the 3rd order terms to describe the internal flow}
Here i know that, 
\begin{align*}
    \textbf{d}
    = - \frac{1}{6} \left(
        \frac{2+5\lambda}{1+\lambda}a^3 \textbf{E}
    \right)
    &&
    \textbf{p}
    = - \frac{1}{6} \left(
        \frac{\lambda(2+5\lambda)}{(1+\lambda)(5\lambda+2)}a^3 \textbf{E}
    \right)
\end{align*}
Integrating this functional over the volume of a droplet yields, 
\begin{equation}
    \avg{\chi_2 (\textbf{e}_2^0)_{ik}}
    = \pavg{\int_{\Omega} 
        \mu [
        \left(- 3 f_{ki}  - 3 f_{ik} +2 f_{jj}\delta_{ki}\right) 
    ] d\Omega}
    = n_pv_p(-3 (f_{ki}+ f_{ik}) + 2 f_{jj} \delta_{ki})
\end{equation}
The only remaining thing is to do determine the form of $f_{ik}$. 


\subsection*{Continuous phase fluctuation term}

The Reynolds stress $\oneavg{\textbf{u}_1'\textbf{u}_1'}$ can be described in the limit of dilute non interaction particles by the wake. 
Therefore, by direct integration of $\textbf{u}^\text{Ext}$ we should be able to find a first correction of the velocity correlation. 
In a homogeneous flow of isolated particle ensemble average is equivalent to volume average thus, 
\begin{align*}
    \oneavg{\textbf{u}_1' \textbf{u}_1' }
    = \int u_i^\text{Ext} u_j^\text{Ext}(\textbf{x},\textbf{r}) P_1(\textbf{r}) d\textbf{r}\\
    = \int [\left(\frac{\delta_{ik}}{r} + \frac{r_ir_k}{r^3}\right)  g_k
    + \left(-\frac{\delta_{ik}}{r^3} + \frac{3r_ir_k}{r^5}\right)  d_k]
    [ \left(\frac{\delta_{jl}}{r} + \frac{r_jr_l}{r^3}\right)  g_l
    + \left(-\frac{\delta_{jl}}{r^3} + \frac{3r_jr_l}{r^5}\right)  d_l] d\textbf{r}
\end{align*}
The expansion of the fluctuation velocity is, i
\begin{align*}
    (\textbf{u}'_1)_i 
    (\textbf{u}'_1)_j
    &=
    \frac{g_i g_j}{r^{2}} 
    - \frac{d_i g_j}{r^{4}} 
    - \frac{d_j g_i}{r^{4}} 
    + \frac{g_i g_l x_j x_l}{r^{4}} 
    + \frac{g_j g_k x_i x_k}{r^{4}} \\
    &+ \frac{d_i d_j}{r^{6}} 
    - \frac{d_i g_l x_j x_l}{r^{6}} 
    - \frac{d_j g_k x_i x_k}{r^{6}} 
    + \frac{3 d_k g_j x_i x_k}{r^{6}} 
    + \frac{3 d_l g_i x_j x_l}{r^{6}} 
    + \frac{g_k g_l x_i x_j x_k x_l}{r^{6}} \\
    &- \frac{3 d_i d_l x_j x_l}{r^{8}} 
    - \frac{3 d_j d_k x_i x_k}{r^{8}} 
    + \frac{3 d_k g_l x_i x_j x_k x_l}{r^{8}} 
    + \frac{3 d_l g_k x_i x_j x_k x_l}{r^{8}} \\
    &+ \frac{9 d_k d_l x_i x_j x_k x_l}{r^{10}} 
\end{align*}
Since we kwon that this flow is axissymetri ctit can acctually be computed such thta
\begin{align*}
    (\textbf{u}'_1)_k 
    (\textbf{u}'_1)_l
    &= 
    ((\textbf{u}'_1)_i 
    (\textbf{u}'_1)_j p_j p_i) p_k p_l 
    + 
    ((\textbf{u}'_1)_i 
    (\textbf{u}'_1)_j (\delta_{ij} - p_j p_i))(\delta_{kl} -  p_k p_l )\\
    &= 
    ((\textbf{u}'_1)_i 
    (\textbf{u}'_1)_j )_{||} p_k p_l 
    + 
    ((\textbf{u}'_1)_i 
    (\textbf{u}'_1)_j )_{\bot}(\delta_{kl} -  p_k p_l )
\end{align*} 
where \textbf{p} is the normalized vector in the direction of \textbf{U}. 

Each of these terms must be integrated from $a$ to $\infty$ in a spherical coordinate frame. 
In spherical coordinate $d\textbf{r} = r^2 \sin\theta dr d\theta d\phi$. 
In this frame we have $x_0 = r \sin \theta \cos\phi$, $x_1 = r \sin \theta \sin\phi$ and $x_2 = r \cos\theta$.
Therefore, the first integration reads, 
\begin{equation*}
    \int_0^{2\pi} 
    \int_0^{\pi} 
    \int_1^{\infty} 
    \frac{1}{r^2} 
    r^2 \sin\theta dr d\theta d\phi
    = 
    4\pi 
    \int_1^\infty dr
\end{equation*}
This integral diverges thus it is not possibly feasible to compute such thing,
however we can relate the ensemble average to nearest conditional average by the relation : 
\begin{multline*}
    \avg{\chi_k \textbf{u}'_k\textbf{u}'_k}(\textbf{x},t)
    + \phi_k \textbf{u}_k\textbf{u}_k
    = \\
    \underbrace{\int (\nstavg{\chi_k \textbf{u}^0_k}  \nstavg{\chi_k \textbf{u}^0_k} / (\nstavg{\chi_k})  P_{nst}(\textbf{x},t,\textbf{r}) d\textbf{r} }_\text{PWFs}
    +\underbrace{\int \nstavg{\chi_k \textbf{v}_k^0\textbf{v}_k^0}  P_{nst}(\textbf{x},t,\textbf{r}) d\textbf{r}}_\text{WIA}
    \label{eq:def_uu}
\end{multline*}
where, $\textbf{v}_k^0  = \textbf{u}_k^0 - \nstavg{\chi_k \textbf{u}^0_k} / \nstavg{\chi_k}$ 
for a dillute random distribution, $P_\text{nst}^\text{th}(\textbf{y}|\textbf{x}) = n_p e^{-4 \pi n_p (r^3 - a^3)/3}$,

If we consider a flow in the $x$ direction we obtain for the three term sthe following integrands,
\begin{align*}
    \iint (\textbf{u}_1'\textbf{u}_1')_{00} r^2 \sin\theta d\phi d\theta
    = e^{\frac{4 \pi a^{3}}{3}}n_{p} \pi \frac{16}{5}\left(\frac{7   g^{2}_0  }{3} 
    + \frac{2   d_0 g_0 }{3 r^{2}} 
    + \frac{   d^{2}_0 }{ r^{4}} \right)e^{- \frac{4 \pi r^{3}}{3}}\\
    \iint (\textbf{u}_1'\textbf{u}_1')_{11}  r^2 \sin\theta d\phi d\theta
    =e^{\frac{4 \pi a^{3}}{3}}n_{p}\pi\frac{4}{5}
    \left(
        \frac{  g^{2}_0}{3} 
        + \frac{2   d_0 g_0 }{ r^{2}} 
        + \frac{3   d^{2}_0 }{ r^{4}}
    \right)e^{- \frac{4 \pi r^{3}}{3}}
\end{align*}
The remaining things to compute are the integral with respect to $\textbf{r}$ of the exponential function. 
Acknowledging that : 
\begin{align*}
    \int_a^\infty \frac{e^{- \frac{4 \pi r^{3}}{3}}}{r^4} dr
    = \frac{e^{- \frac{4 \pi a^{3}}{3}}}{3a^3}
    - \frac{4}{9} \pi \Gamma\left(0,\frac{4a^3\pi}{3}\right)
    = \frac{e^{- \frac{4 \pi a^{3}}{3}}}{3a^3}
    - \frac{4}{9} \pi E_1\left(\frac{4a^3\pi}{3}\right)
    \\
    \int_a^\infty \frac{e^{- \frac{4 \pi r^{3}}{3}}}{r^2} dr
    = \frac{E_{4/3}(\frac{4\pi a^3}{3})}{3a}\\
    \int_a^\infty e^{- \frac{4 \pi r^{3}}{3}} dr
    = \frac{a \Gamma\left(\frac{1}{3},\frac{4 a^2 \pi}{3}\right)}
    {6^{2/3} a \pi^{1/3}}
    = \frac{a E_{2/3}\left(-\frac{4 a^3 \pi}{3}\right)}
    {6^{2/3} a \pi^{1/3}}
\end{align*}
The final results gives, 
\begin{align*}
    \iint (\textbf{u}_1'\textbf{u}_1')_{00} r^2 \sin\theta d\phi d\theta
    = e^{\frac{4 \pi a^{3}}{3}}n_{p} \pi \frac{16}{5}\left(\frac{7   g^{2}_0  }{3} 
    + \frac{2   d_0 g_0 }{3 r^{2}} 
    + \frac{   d^{2}_0 }{ r^{4}} \right)e^{- \frac{4 \pi r^{3}}{3}}\\
    \iint (\textbf{u}_1'\textbf{u}_1')_{11}  r^2 \sin\theta d\phi d\theta
    =e^{\frac{4 \pi a^{3}}{3}}n_{p}\pi\frac{4}{5}
    \left(
        \frac{  g^{2}_0}{3} 
        + \frac{2   d_0 g_0 }{ r^{2}} 
        + \frac{3   d^{2}_0 }{ r^{4}}
    \right)e^{- \frac{4 \pi r^{3}}{3}}
\end{align*}
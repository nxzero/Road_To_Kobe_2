
%\subsubsection*{Equivalence between particle and continuous models}
\subsection{Equivalence between particle-averaged and phase-averaged equations}
\label{sec:equivalence}
To model the dispersed phase we can either use \ref{eq:avg_dt_chi_f} with $k=d$, or the particle-averaged equations: \ref{eq:avg_dt_dq_alpha_tot}, \ref{eq:avg_dt_dQ_alpha_tot} and possibly the higher moments equations in \ref{ap:Moments_equations}. 
Consequently, it is fair to address the question of the compatibility and differences between both formalisms. 

To begin with, it has been demonstrated in various studies \citep{nott2011suspension,jackson1997locally,zhang1994averaged}, that phase-averaged quantities can be expressed as a Taylor series expansion of particle-averaged quantities. 
Indeed, the dispersed phase indicator function $\chi_d(\textbf{x},t)$ can be expressed as a sum of phase indicator function, $\chi_d(\textbf{x},t) = \sum_\alpha\chi_\alpha(\textbf{x},t)$ where $\chi_\alpha =1$ in the particle domain $\Omega_\alpha(t)$ and $0$ otherwise. 
Then, notice that any dispersed phase quantity can be written as, 
\begin{equation*}
    f^0_d \chi_d(\textbf{x},t)
    = \sum_\alpha f^0_d \chi_\alpha(\textbf{x},t) 
    = \sum_\alpha \int_{\mathbb{R}^3} f^0_d \chi_\alpha(\textbf{x}_\alpha+\textbf{r},t)\delta(\textbf{x}- \textbf{x}_\alpha - \textbf{r}) d\textbf{r} 
\end{equation*}
Upon using the Taylor expansion of the Dirac delta function in the neighborhood of $\textbf{r}=0$ one obtain :  $\delta(\textbf{x}- \textbf{x}_\alpha - \textbf{r}) = \delta(\textbf{x}- \textbf{x}_\alpha) - \textbf{r}\cdot\grad \delta(\textbf{x} - \textbf{x}_\alpha) + \frac{\textbf{rr}}{2}:\grad\grad\delta(\textbf{x}- \textbf{x}_\alpha) - \ldots $.
Thus, the dispersed-phase field $f_d^0\chi_d$ can be re-written as, 
\begin{equation*}
    f^0_d \chi_d(\textbf{x},t)
    = \sum_\alpha \left[
          \left(\delta_\alpha\intO{f^0_d}\right)
        - \div\left(\delta_\alpha\intO{\textbf{r} f^0_d}\right)
        + \frac{1}{2}\grad\grad :\left(\delta_\alpha\intO{\textbf{rr} f^0_d}\right)
        \ldots
    \right]
\end{equation*}
where we recognize the zeroth, first and second order moments of $f_d^0$. 
Applying similar considerations to the interface indicator function $\delta_I$, and averaging over all configurations, we obtain the general relations that link continuous-averaged and particle-averaged fields, namely, 
\begin{align}
    \avg{\chi_df_d^0} 
    &=  \pavg{q_\alpha}
        - \div  
        \pavg{\textbf{Q}_\alpha}        
        + \frac{1}{2} \grad\grad : \pavg{\textbf{Q}_{2\alpha}}
        + \ldots  
        \nonumber\\
    \avg{\delta_I f_I^0} 
    &=  \pavg{q_{I\alpha}}        
        - \div \pavg{\textbf{Q}_{I\alpha}}
        + \frac{1}{2} \grad\grad : \pavg{\textbf{Q}_{I\alpha}^{2}}
        + \ldots  
    \label{eq:f_exp}
\end{align}
% It must be noted that \ref{eq:f_exp} is the bridge between the continuous phase average formalism and the particle formulation. 
Particularly we note that if $f_d^0 = \rho_d$ and $f_d^0 = \rho_d \textbf{u}_d^0$ we obtain, 
\begin{align}
    \label{eq:f_exp_exe1}
    \phi_d \rho_d
    = m_p n_p 
    + \frac{1}{2}\grad^2 : (n_p\textbf{M}_p)+\ldots,\\
    \phi_d \rho_d \textbf{u}_d
    = m_p n_p \textbf{u}_p 
    - \div (n_p\textbf{P}_p)+\ldots,
    \label{eq:f_exp_exe}
\end{align}
respectively. 
Meaning that $\phi_d\rho_d$ is related to the shape of the particles, represented by $\textbf{M}_p$ through \ref{eq:f_exp_exe1}.
Additionally, considering \ref{eq:f_exp_exe}, it becomes apparent that the phase-averaged velocity $\textbf{u}_d$ encompasses the first moment of momentum $\textbf{P}_p$, which as discussed (in \ref{sec:Lagrangian}) accounts for the rotational, dilatational, and stretching motions of the particles. 
The second terms on the right-hand side of \ref{eq:f_exp_exe1} and \ref{eq:f_exp_exe} become negligible for homogeneous mixture, i.e. if $n_p$, $\textbf{M}_p$ and $\textbf{P}_p$ are not function of \textbf{x}. 
Conversely, these terms might become significant if $n_p$, $\textbf{M}_p$ or $\textbf{P}_p$ are space-dependent.
For example, close to solid boundaries of a macroscopic flow strong gradients of $n_p$ are present at the particle length scale, since at the exact location of the boundaries we must respect $n_p = 0$. 


In order to establish the equivalence between both formalism, we follow the strategy of \citep{lhuillier2000bilan,lhuillier2009rheology} by taking the Taylor expansion of each terms in \ref{eq:avg_dt_chi_f} with $k=d$ using the relation \ref{eq:f_exp}. 
Since we made use of the surface transport equations in the particles phase equations : \ref{eq:avg_dt_dq_alpha_tot} and \ref{eq:avg_dt_dQ_alpha_tot}, we also consider \ref{eq:avg_dt_delta_f} to prove equivalence. 
As the resulting expression can become quite cumbersome, we will adopt the following definition. 
Let $\mathbb{C}_d$ represent the phase-averaged equation of conservation (\ref{eq:avg_dt_chi_f} with $k=d$) and $\mathbb{C}_I$ the averaged surface transport equation, namely, 
\begin{align*}
    \mathbb{C}_d
    &=
    - \pddt \avg{\chi_df_d^0}
    - \div \avg{\chi_d \mathbf{\Phi}_d^0 - \chi_df_d^0 \textbf{u}_d^0}
    + \avg{\chi_d s_d^0}
    + \avg{\delta_I\left[
        \mathbf{\Phi}_d^0
        + f_d^0
        \left(
            \textbf{u}_I^0
            - \textbf{u}_d^0
        \right)
    \right]
    \cdot \textbf{n}_d}.\\
    \mathbb{C}_I
    &= 
    -\pddt \avg{\delta_If_I^0}
    -\div \avg{\delta_I f_I^0 \textbf{u}_I^0-\delta_I \mathbf{\Phi}_{I||}^0 }
    + \avg{\delta_Is_I^0} 
    - \avg{\delta_I \Jump{
    f_k^0 (\textbf{u}_I^0 - \textbf{u}_k^0)
    + \mathbf{\Phi}_k^0
    } }. 
\end{align*}
It must be understood from \ref{eq:avg_dt_chi_f} and \ref{eq:avg_dt_delta_f} that $\mathbb{C}_d=0$ and $\mathbb{C}_I=0$.
By taking the Taylor expansion of each terms of $\mathbb{C}_d+\mathbb{C}_I$ we can show that,
\begin{equation}
    \mathbb{C}_d 
    + \mathbb{C}_I 
    = \mathbb{M}_0 - \div \mathbb{M}_1 + \frac{1}{2} \grad\grad : \mathbb{M}_2 \ldots = 0,
    \label{eq:scheme_equivalence}
\end{equation} 
where the expression $\mathbb{M}_0$ and $\mathbb{M}_1$ turn out to be, 
\begin{align*}
    &\mathbb{M}_0
    = 
    - \avg{\delta_\alpha \ddt {q_\alpha^\text{tot}}}
    % -\avg{\delta_\alpha\textbf{u}_\alpha q_\alpha^\text{tot}}
    + \pOavg{ s_d^0 }
    + \pSavg{ s_I^0 }
    + \pSavg{ 
    \left[\mathbf{\Phi}_f^0 
    + f_f^0 (\textbf{u}_I^0-\textbf{u}_f^0) \right] \cdot \textbf{n}_d },\\
    &\mathbb{M}_f =
    -  \avg{\delta_\alpha \ddt {\textbf{Q}_\alpha^\text{tot}}}
    % - \avg{\delta_\alpha\textbf{u}_\alpha \textbf{Q}_\alpha^\text{tot}}
     + \pOavg{ \left(
        \textbf{r} s_d^0         
        + f_d^0  \textbf{w}_d^0 
        - \mathbf{\Phi}_d^0
    \right) }
    + \pSavg{ \left(
        \textbf{r}s_I^0
        + f_I^0 \textbf{w}_I^0
        - \mathbf{\Phi}_{I||}^0
    \right) }\\
    &+ \pSavg{ \textbf{r} \left[
        \mathbf{\Phi}_f^0
        + f_f^0 (\textbf{u}_I^0-\textbf{u}_f^0)
    \right]\cdot \textbf{n}_d  },
\end{align*}
respectively. 
In the presence of \ref{eq:scheme_equivalence}, we reach the major conclusion of this work. 
Indeed, we can observe that $\mathbb{M}_0$ and $\mathbb{M}_1$ represent the zeroth and first order moments equations, respectively. 
Additionally, it is shown in \ref{ap:Moments_equations} that the coefficient $\mathbb{M}_n$ in \ref{eq:scheme_equivalence} correspond to the $n^{th}$ order moment particle-averaged conservation equation. 
From \ref{eq:scheme_equivalence} we conclude that \ref{eq:dt_chi_k_f_k} for $k=d$ encompasses the particles moments equations through a Taylor expansion around the particle center of mass. 
Thus, it is evident that one can use an arbitrary order of particles moments equations to achieve an arbitrarily accurate description of the dispersed phase, regardless of the properties of the multiphase flow.
In \cite{lhuillier2000bilan} they reached a similar conclusion when comparing the phase-averaged area density and particle-averaged area density equations for spherical particles. 
In \ref{ap:Moments_equations} we provide the expression for each $\mathbb{M}_n$ as well as the complete derivation of \ref{eq:scheme_equivalence}. 

Another approach is to notice that $\mathbb{M}_n=0$ for all $n$. Thus, we can rewrite \ref{eq:scheme_equivalence} such that all moments equations vanish, except $\mathbb{M}_0$, which gives, 
\begin{equation}
    \mathbb{C}_d 
    + \mathbb{C}_I
    = \mathbb{M}_0 = 0.
    \label{eq:proof2}
\end{equation}
This implies that equation \ref{eq:avg_dt_chi_f} with the surface transport equation \ref{eq:avg_dt_delta_f} is rigorously equivalent to \ref{eq:avg_dt_dq_alpha_tot}.
\citet[Appendix A]{zhang1997momentum} provided evidences that the particle-averaged momentum equation is as legitimate as the phase-averaged momentum equation, which is consistent with \ref{eq:proof2}. 
Additionally, \citet[Appendix A]{nott2011suspension} derived a similar expression than \ref{eq:proof2}, also in the case of the averaged momentum equation for suspension of solid spherical particles.
Thus, in light of \ref{eq:proof2}, we generalize the conclusion of these authors and demonstrated that this is also true for all conservation laws regardless of the dispersed phase nature.  
Considering the Lagrangian equations derived in \ref{sec:Lagrangian} this conclusion is not surprising at all since the phase-averaged and particle-averaged equations are all built on \ref{eq:dt_f_k} and \ref{eq:dt_f_I}.
However, if one does not consider a proper derivation as it is done in \ref{sec:Lagrangian} it might not be as obvious, even if \ref{eq:proof2} should remain true as demonstrated by \citet{zhang1997momentum,nott2011suspension}.
Nevertheless, it is important to note that the conclusion given by \ref{eq:proof2} is not entirely objective since following the same procedure we could show equally that $\mathbb{C}_d  = -\div\mathbb{M}_f=0$ and $\mathbb{C}_d  = \frac{1}{2}\grad\grad:\mathbb{M}_2=0$ and so on. 
Thus, it is more appropriate to examine the problem from the perspective of \ref{eq:scheme_equivalence}. 
Namely, the particle-averaged equations constitute a system of equations with one equation for each moment, while the phase-averaged equation contains all the particle-averaged equations within a single equation.
Therefore, the particle-averaged formalism encompasses more information since it provides one equation for each moment. 
Note that this gain in information has been possible through the consideration of the topology of the dispersed phase. 

The major consequence of this finding is that it enables us to better understand the role of the dispersed phase stresses $\phi_d\bm{\sigma}_d$ and $\phi_I\bm{\sigma}_I$, in the particle-averaged momentum equation such as it is written in a kinetic-like model (\ref{eq:avg_dt_dq_alpha_tot}). 
Indeed, as shown by \ref{eq:avg_dt_dq_alpha_tot}, $\phi_d\bm{\sigma}_d$ and $\phi_I\bm{\sigma}_I$ will not play a role in the particle-phase averaged momentum equations, i.e. the equation of $n_p m_p \textbf{u}_p$. 
However, as shown in \ref{eq:dt_avg_rhou_k} (for $k =d$), the phase averaged momentum $\rho_d \phi_d \textbf{u}_d$,  will be subject to the particle surface and internal stresses, since $\phi_d\bm{\sigma}_d$ and $\phi_I\bm{\sigma}_I$ are present in this equation.
This is made consistent if one considers that these stresses act as source terms in the higher moments of momentum equations, i.e.  $\mathbb{M}_1$,$\mathbb{M}_2$, \ldots, which are related to $\rho_d \phi_d \textbf{u}_d$ through \ref{eq:f_exp_exe}. 
In brief, the non-convective fluxes $\phi_d\bm{\sigma}_d$ and $\phi_I\bm{\sigma}_I$  have no direct impact on the particle-phase averaged momentum $n_p m_p\textbf{u}_p$, regardless of the particles nature and volume fraction. 
The influence of $\phi_d\bm{\sigma}_d$ and $\phi_I\bm{\sigma}_I$ on the particles' averaged momentum, $n_p m_p \textbf{u}_p$ is made through the dependence of the source terms or closure term present in \ref{eq:avg_dt_dq_alpha_tot} with the higher moments of the particles such as $\textbf{M}_p$, $\textbf{P}_p$. 
Indeed, $\textbf{P}_p$ and $\textbf{M}_p$ are directly related to the non-convective fluxes as suggested by the \ref{eq:dt_P_alpha}. 
Consequently, it must be understood that the kinetic-like equations (\ref{eq:avg_dt_dq_alpha_tot}) are formally exact and apply for any type of particle and particle volume fraction, as long as the closure terms are well modeled. 
Similar remarks can be made regarding the non-convective fluxes of the energy equation $\textbf{q}_2$ and for all other conservation equations. 



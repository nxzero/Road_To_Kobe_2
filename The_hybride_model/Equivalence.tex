
%\subsubsection*{Equivalence between particle and continuous models}
\subsection{Equivalence between particle-averaged and phase-averaged equations}

To model the dispersed phase we can either use \ref{eq:avg_dt_chi_f} with $k=2$, or the particle-average \ref{eq:avg_dt_dq_alpha_tot}, \ref{eq:avg_dt_dQ_alpha_tot} and possibly the higher moments equations. 
As mentioned in \ref{sec:Lagrangian} we notified that \ref{eq:dt_dq_alpha_tot} was already subject to an average over the particles' volume and the surfaces. 
Meaning that \ref{eq:avg_dt_dq_alpha_tot} is the results of two average processes. 
Consequently, it is fair to address the question of the compatibility and differences between both formalism, i.e. between \ref{eq:avg_dt_chi_f} and \ref{eq:avg_dt_dq_alpha_tot}. 

To begin with, it has been demonstrated in various studies \citep{nott2011suspension,jackson1997locally,zhang1994averaged}, that phase-averaged quantities can be expressed as a Taylor series expansion of particle-averaged quantities. 
The expressions read as follows, 
\begin{align}
    \avg{\chi_kf_k^0} 
    &=  \pavg{q_\alpha}
        - \div  
        \pavg{\mathcal{Q}_\alpha}        
        + \frac{1}{2} \grad\grad : \pavg{\mathcal{Q}_\alpha^2}
        + \ldots  \\
    \avg{\delta_I f_k^0} 
    &=  \pavg{q_{I\alpha}}        
        - \div \pavg{\mathcal{Q}_{I\alpha}}
        + \frac{1}{2} \grad\grad : \pavg{\mathcal{Q}_{I\alpha}^{2}}
        + \ldots  
    \label{eq:f_exp}
\end{align}
Then, in order to establish the equivalence between both formalism, we follow the strategy of \citep{lhuillier2000bilan} by taking the Taylor expansion of each terms in \ref{eq:avg_dt_chi_f} with $k=2$. 
As the resulting expression can become quite cumbersome, we will adopt the following definition. 
Let $\mathbb{C}$ represent the phase-averaged equation of conservation (\ref{eq:avg_dt_chi_f}), namely, 
\begin{equation*}
    \mathbb{C}
    =
    - \pddt \avg{\chi_2f_2^0}
    - \div \avg{\chi_2 \mathbf{\Phi}_2^0 - \chi_2f_2^0 \textbf{u}_2^0}
    + \avg{\chi_2 s_2^0}\\
    + \avg{\delta_I\left[
        \mathbf{\Phi}_2^0
        + f_2^0
        \left(
            \textbf{u}_I^0
            - \textbf{u}_2^0
        \right)
    \right]
    \cdot \textbf{n}_2}. 
\end{equation*}
Therefore, it must be understood from \ref{eq:avg_dt_chi_f} that $\mathbb{C}=0$.
Then, by taking the Taylor expansion of each terms of $\mathbb{C}$ according to \ref{eq:f_exp}, we can equally show that,
\begin{equation}
    \mathbb{C} = \mathbb{M}_0 - \div \mathbb{M}_1 + \frac{1}{2} \grad\grad : \mathbb{M}_2 \ldots = 0,
    \label{eq:scheme_equivalence}
\end{equation} 
where the expression $\mathbb{M}_0$ and $\mathbb{M}_1$ turn out to be, 
\begin{multline*}
    \mathbb{M}_0
    = 
    - \pddt \avg{\delta_\alpha (q_\alpha+q_{I_\alpha})}
    - \div \avg{\delta_\alpha\textbf{u}_\alpha(q_\alpha+q_{I_\alpha})}
    + \avg{\delta_\alpha\int_{\Omega_\alpha} s_2^0 d\Omega}\\
    + \avg{\delta_\alpha\int_{\Sigma_\alpha} s_I^0 d\Sigma}
    + \avg{\delta_\alpha\int_{\Sigma_\alpha} 
    \left[\mathbf{\Phi}_1^0 
    + f_1^0 (\textbf{u}_I^0-\textbf{u}_1^0) \right] \cdot \textbf{n}_2 d\Sigma},
\end{multline*}
and,
\begin{multline*}
    \mathbb{M}_1 =
    - \pddt \avg{\delta_\alpha (\mathcal{Q}_\alpha+\mathcal{Q}_{I_\alpha})}
    - \div \avg{\delta_\alpha\textbf{u}_\alpha(\mathcal{Q}_\alpha+\mathcal{Q}_{I_\alpha})}
    \\ + \avg{\delta_\alpha\int_{\Omega_\alpha} \left(
        \textbf{r} s_2^0         
        + f_2^0  \textbf{w}_2^0 
        - \mathbf{\Phi}_2^0
    \right) d\Omega}
    + \avg{\delta_\alpha\int_{\Sigma_\alpha} \left(
        \textbf{r}\textbf{S}_I^0
        + f_I^0 \textbf{w}_I^0
        - \mathbf{\Phi}_{I||}^0
    \right) d\Sigma}\\
    + \avg{\delta_\alpha\int_{\Sigma_\alpha} \textbf{r} \left[
        \mathbf{\Phi}_1^0
        + f_1^0 (\textbf{u}_I^0-\textbf{u}_1^0)
    \right]\cdot \textbf{n}_2  d\Sigma},
\end{multline*}
respectively. 
In the presence of \ref{eq:scheme_equivalence} we reach the major conclusion of this work. 
Indeed, we can observe that $\mathbb{M}_0$, $\mathbb{M}_1$ and $\mathbb{M}_2$ represent the zeroth, first and second order moments equations, respectively. 
In fact, it is shown in \ref{ap:Moments_equations} that the coefficient $\mathbb{M}_n$ in \ref{eq:scheme_equivalence} correspond to the $n^{th}$ order particle-average conservation equation. 
As a matter of fact, the phase average applied to the dispersed phase contains all the particle-averaged moments equations.
In \ref{ap:Moments_equations} we provide the expression for each moment equation $\mathbb{M}_n$ as well as the complete derivation of \ref{eq:scheme_equivalence}. 
In \cite{lhuillier2000bilan} they reached similar conclusion when comparing the area density phase-averaged and particle-averaged equations of conservation for spherical particles. 
Thus, from \ref{eq:scheme_equivalence} it is evident that one can use an arbitrary order of moments equations to reach an arbitrary accurate description of the dispersed phase.
This fact was previously suggested by \citet{zhang1997momentum}, here we provide the detailed derivation.  

Another approach is to notice that $\mathbb{M}_n=0$ for all $n$. Thus, we can rewrite \ref{eq:scheme_equivalence} such that all moments equations vanish, except $\mathbb{M}_0$, which gives, 
\begin{equation}
    \mathbb{C} = \mathbb{M}_0 = 0.
\end{equation}
This implies that equation \ref{eq:avg_dt_chi_f} is rigorously equivalent to \ref{eq:avg_dt_dq_alpha_tot} as shown by \cite[Appendix A]{nott2011suspension} for the specific case of the momentum conservation equation of solid spherical particle.
Nevertheless, it is important to note that this conclusion is not entirely objective since following the same procedure we could show equally that $\mathbb{C} = -\div\mathbb{M}_1=0$ and $\mathbb{C} = \frac{1}{2}\grad\grad:\mathbb{M}_2=0$ and so on for the other higher terms. 
Thus, it is more appropriate to examine the problem from the perspective of \ref{eq:scheme_equivalence}. 
Namely, the particle-averaged equations constitute a system of equations with one equation for each moment, while the phase-averaged equation contains all the terms of the particle-averaged equations within a single equation.
Therefore, the particle-averaged formalism encompasses more information since it provide one equation for each moment. 

% \subsection{The hybrid model}

% Now that we reached a clear understanding of the averaged models, we introduce the Hybrid model for dispersed two phase flows. 

% We first introduce the equation used to model the continuous phase. 
% It is derived by setting $k = 1$ in \ref{eq:avg_dt_chi_f}, and by using \ref{eq:f_exp} to expand the exchange term into an expansion serie.
% This yields the following expression~:
% \begin{multline}
%     \pddt \avg{\chi_1 f_1}
%     = \div \avg{\chi_1 \mathbf{\Phi}_1 - \chi_1 f_1 \textbf{u}_1}
%     - \pavg{\int_{\Sigma_\alpha}\left[
%         \mathbf{\Phi}_1  
%         + f_1
%         \left(
%             \textbf{u}_I
%             - \textbf{u}_1
%         \right)
%     \right]
%     \cdot \textbf{n}_2d\Sigma} \\
%     + \avg{\chi_1 \textbf{S}_1}
%     +  \div \pavg{\int_{\Sigma_\alpha} \textbf{r}\left[
%         \mathbf{\Phi}_1
%         + f_1
%         \left(
%             \textbf{u}_I
%             - \textbf{u}_1
%         \right)
%     \right]
%     \cdot \textbf{n}_2d\Sigma} 
%     \label{eq:hybrid_avg_dt_chif}
% \end{multline}
% \tb{reformuler ca avec le conditional average et le triks de la div }
% Notice that we switched the normal vector from $\textbf{n}_1$ to $-\textbf{n}_2$ in the second and the last integral on the RHS, which is the conventional way to express these couplings terms. 
% Moreover, it should be noted that we neglected the second and higher order moments of the coupling term between the two phases.

% Regarding the dispersed phase, we consider the zeroth and first moment equation (\ref{eq:avg_dt_dq_alpha_tot} and \ref{eq:avg_dt_dQ_alpha_tot}) as  a part of the hybrid model.
% Under this form the couplings terms in \ref{eq:hybrid_avg_dt_chif} correspond to the ones appearing in both, \ref{eq:avg_dt_dq_alpha_tot} and \ref{eq:avg_dt_dQ_alpha_tot} which is essential for ensuring a consistent model. 
% Then, \ref{eq:hybrid_avg_dt_chif}, \ref{eq:avg_dt_dq_alpha_tot} and \ref{eq:avg_dt_dQ_alpha_tot}, are the most general form of a first-order accurate hybrid model for arbitrary particle. 
% Lastly, it is worth noting that by incorporating more terms in the expansion of \ref{eq:avg_dt_chi_f} and higher order moments equations, one can reach an arbitrary order of accuracy, as stated by \citet{zhang1997momentum}. 

% As remarked by \citet{jackson1997locally} for the angular momentum equations of solid spherical particles, and here in a more general case :\ref{eq:hybrid_avg_dt_chif} and \ref{eq:avg_dt_dq_alpha_tot} may seem to be not coupled with the higher order moments equations, i.e. \ref{eq:avg_dt_dQ_alpha_tot}. 
% Indeed, the first order moment $\mathcal{Q}_\alpha$ do not appear explicitly in either \ref{eq:hybrid_avg_dt_chif} or \ref{eq:avg_dt_dq_alpha_tot}.
% However, the exchange terms 
% $\int_{\Sigma_\alpha}[
%     \mathbf{\Phi}_1  
%     + f_1
%     (
%         \textbf{u}_I
%         - \textbf{u}_1
%     )
% ]
% \cdot \textbf{n}_2d\Sigma$ 
% and 
% $\int_{\Sigma_\alpha} \textbf{r}[
%     \mathbf{\Phi}_1
%     + f_1
%     (
%         \textbf{u}_I
%         - \textbf{u}_1
%     )
% ]
% \cdot \textbf{n}_2d\Sigma$ 
% appearing in both \ref{eq:hybrid_avg_dt_chif} and \ref{eq:avg_dt_dq_alpha_tot} might depend on the higher order moments of the particles.
% As an example, in the momentum equation of the continuous phase, i.e. the ensemble average of \ref{eq:dt_rhou_k}, the exchange term corresponds to the averaged drag force, namely $\pnnavg{\textbf{f}_\alpha}$. 
% It is clear that $\pnnavg{\textbf{f}_\alpha}$ has a strong dependency with $\textbf{u}_\alpha$ and $\mathcal{M}_\alpha$ since the drag force is a function of both the particle's velocity and its shape. 
% Therefore, \ref{eq:avg_dt_dQ_alpha_tot} is linked to \ref{eq:hybrid_avg_dt_chif} and \ref{eq:avg_dt_dq_alpha_tot} solely through the dependence between the exchange terms and the properties of the particles, e.g. $q_\alpha$, $\mathcal{Q}^n_\alpha$ and possibly the higher order moments. 
% Consequently, the significance of the higher moments equations can be evaluated based on the dependency of the exchange terms in \ref{eq:hybrid_avg_dt_chif} with the moments,  $\mathcal{Q}_\alpha$ and potentially the higher-order moments of the particles. 

\subsection{The first order hybrid model}

Now that we reached a clear understanding of the averaged models, we introduce the Hybrid model for dispersed two phase flows. 

As it has been shown in earlier studies \citep{jackson2000dynamics} it is more physical to express the non-convective flux appearing in \ref{eq:avg_dt_chi_f} with the volume fraction outside the divergence operator. 
Therefor we use the following relation, 
\begin{equation*}
    \div \avg{\chi_1 \mathbf{\Phi}_1^0}
    = \phi_1 \div  \oneavg{ \mathbf{\Phi}_1^0}
    +  \mathbf{\Phi}_1 \cdot \avg{\delta_I \textbf{n}_2}\\
\end{equation*}
upon using \ref{eq:f_exp} and on the last term we find, 
\begin{equation}
    \div \avg{\chi_1 \mathbf{\Phi}_1}
    =  \phi_1 \div  \mathbf{\Phi}_1
    + \pnavg{\int_{\Sigma} \mathbf{\Phi}_1 \cdot  \textbf{n}_2 d\Sigma}
    +\div \left[
        \pnavg{\int_{\Sigma} \textbf{r}\mathbf{\Phi}_1 \cdot  \textbf{n}_2 d\Sigma}
    \right] 
    +\ldots
\end{equation}
This decomposition motivates us to re-write the exchange terms in \ref{eq:avg_dt_chi_f} as, 
\begin{equation*}
    \textbf{F}_\alpha = 
    \int_{\Sigma_\alpha}
    \left(
        \mathbf{\Phi}_1^0 
        - \mathbf{\Phi}_1
    \right)  
    \cdot \textbf{n}_2d\Sigma
    + 
    \int_{\Sigma_\alpha}f_1^0
    \left(
        \textbf{u}_I^0
        - \textbf{u}_1^0
    \right)
    \cdot \textbf{n}_2d\Sigma,
\end{equation*}
and 
\begin{equation*}
    \textbf{M}_\alpha = \int_{\Sigma_\alpha} \textbf{r}
        \left(\mathbf{\Phi}_1^0- \mathbf{\Phi}_1\right)\cdot \textbf{n}_2d\Sigma
        + \int_{\Sigma_\alpha} \textbf{r}f_1^0
        \left(
            \textbf{u}_I^0
            - \textbf{u}_1^0
        \right)
    \cdot \textbf{n}_2d\Sigma
\end{equation*}
These considerations finally leads us to the general form of the continuous phase equation : 
\begin{equation}
    \pddt (\phi_1 f_1)
    + \div (\phi_1 f_1\textbf{u}_1
    + \phi_1 \oneavg{\textbf{u}_1'f_1'}
    - n_p \textbf{M}_p )
    = 
    \phi_1(s_1  +\div \mathbf{\Phi}_1)
    - n_p \textbf{F}_p
    \label{eq:hybrid_avg_dt_chif}
\end{equation}

Regarding the dispersed phase, we consider the zeroth and first moment equation (\ref{eq:avg_dt_dq_alpha_tot} and \ref{eq:avg_dt_dQ_alpha_tot}) as  a part of the hybrid model.
Under this form the couplings terms in \ref{eq:hybrid_avg_dt_chif} do not correspond to the ones appearing in both, \ref{eq:avg_dt_dq_alpha_tot} and \ref{eq:avg_dt_dQ_alpha_tot} which is essential for ensuring a consistent model. 
To fix this problem we reformulated both exchange terms following \citet{zhang1997momentum} such that it correspond to $\mathbf{F}_\alpha$ and $\textbf{M}_\alpha$. 
Then we may rewrite the momentum and moment of momentum equations (\ref{eq:avg_dt_dq_alpha_tot} and \ref{eq:avg_dt_dQ_alpha_tot}) as : 
\begin{multline}
    \pddt \left[n_p (q_p+q^I_p)\right]
    + \div \left[n_p\textbf{u}_p
    (q_p+q^I_p)
    + \pnavg{\textbf{u}_\alpha'(q_\alpha'+q_{I_\alpha}')} \right]
    = 
    n_p v_p  \div \mathbf{\Phi}_1 \\
    + \pnavg{\int_{\Omega_\alpha} s_2 d\Omega}
    + \pnavg{\int_{\Sigma_\alpha} s_I d\Sigma}
    + n_p \textbf{F}_p,
    \label{eq:hybrid_avg_dt_q_alpha}
\end{multline}
and, 
\begin{multline}
    \pddt \left[n_p (\mathcal{Q}_p+\mathcal{Q}_{I_p})\right]
    + \div \left[
        n_p \textbf{u}_p (\mathcal{Q}_p+\mathcal{Q}_{I_p})
    + \pnavg{\textbf{u}_\alpha'(\mathcal{Q}_\alpha+\mathcal{Q}_{I_\alpha})'}
    \right]
    =
    n_p v_p \mathbf{\Phi}_1 
    + n_p \textbf{M}_p\\
    \pnavg{\int_{\Omega_\alpha} \left(
        \textbf{r} s_2^0         
        + f_2^0  \textbf{w}_2^0 
        - \mathbf{\Phi}_2^0
        \right) d\Omega}
        + \pnavg{\int_{\Sigma_\alpha} \left(
        \textbf{r}s_I^0
        + f_I^0 \textbf{w}_I^0
        - \mathbf{\Phi}_{I||}^0
    \right) d\Sigma},
    \label{eq:hybrid_avg_dt_Q_alpha}
\end{multline}
respectively. 

Then, \ref{eq:hybrid_avg_dt_chif}, \ref{eq:hybrid_avg_dt_q_alpha} and \ref{eq:hybrid_avg_dt_Q_alpha}, are the most general form of a first-order accurate hybrid model for arbitrary particle. 
Lastly, it is worth noting that by incorporating more terms in the expansion of \ref{eq:avg_dt_chi_f} and higher order moments equations, one can reach an arbitrary order of accuracy, as stated by \citet{zhang1997momentum}. 

As remarked by \citet{jackson1997locally} for the angular momentum equations of solid spherical particles, and here in a more general case :\ref{eq:hybrid_avg_dt_chif} and \ref{eq:avg_dt_dq_alpha_tot} may seem to be not coupled with the higher order moments equations, i.e. \ref{eq:avg_dt_dQ_alpha_tot}. 
Indeed, the first order moment $\mathcal{Q}_\alpha$ do not appear explicitly in either \ref{eq:hybrid_avg_dt_chif} or \ref{eq:avg_dt_dq_alpha_tot}.
However, the exchange terms 
$\textbf{F}_p$ 
and 
$\textbf{M}_p$ 
appearing in both \ref{eq:hybrid_avg_dt_chif} and \ref{eq:avg_dt_dq_alpha_tot} might depend on the higher order moments of the particles.
As an example, in the momentum equation of the continuous phase, i.e. the ensemble average of \ref{eq:dt_rhou_k}, the exchange term corresponds to the averaged drag force, namely $\textbf{f}_p$. 
It is clear that $\textbf{f}_p$ has a strong dependency with $\textbf{u}_p$ and $\mathcal{M}_p$ since the drag force is a function of both the particle's velocity and its shape. 
Therefore, \ref{eq:avg_dt_dQ_alpha_tot} is linked to \ref{eq:hybrid_avg_dt_chif} and \ref{eq:avg_dt_dq_alpha_tot} solely through the dependence between the exchange terms and the properties of the particles, e.g. $q_p$, $\mathcal{Q}^n_p$ and possibly the higher order moments. 
Consequently, the significance of the higher moments equations can be evaluated based on the dependency of the exchange terms in \ref{eq:hybrid_avg_dt_chif} with the moments,  $\mathcal{Q}_p$ and potentially the higher-order moments of the particles. 

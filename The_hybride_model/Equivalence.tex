
%\subsubsection*{Equivalence between particle and continuous models}
\subsection{Equivalence between particle average and phase average}

To model the dispersed phase we can either use \ref{eq:avg_dt_chi_f} with $k=2$, or the particle-average \ref{eq:avg_dt_dq_alpha_tot}, \ref{eq:avg_dt_dQ_alpha_tot} and possibly the higher moments. 
As mentioned in \ref{sec:dispersed-two-fluid} we notified that \ref{eq:dt_dq_alpha_tot} was already subject to an average over the particles' volume and the surfaces. 
Consequently, it is fair to address the question of the compatibility and differences  between both formalism, i.e. \ref{eq:avg_dt_chi_f} and \ref{eq:avg_dt_dq_alpha_tot}. 

To begin with, it has been demonstrated in various studies \citep{nott2011suspension,jackson1997locally,zhang1994averaged}, that volume and interface averaged quantities can be expressed as a Taylor series expansion of particle-field averaged quantities, yielding, 
\begin{align}
    \avg{\chi_kf_k} 
    &=  \pavg{q_\alpha}
        - \nablab \cdot  
        \pavg{\mathcal{Q}_\alpha}        
        + \frac{1}{2} \nablab\nablab : \pavg{\mathcal{Q}_\alpha^2}
        + \ldots  \\
    \avg{\delta_I f_k} 
    &=  \pavg{q_{I\alpha}}        
        - \nablab \cdot \pavg{\mathcal{Q}_{I\alpha}}
        + \frac{1}{2} \nablab\nablab : \pavg{\mathcal{Q}_{I\alpha}^{2}}
        + \ldots  
    \label{eq:f_exp}
\end{align}

Then, in order to establish the equivalence between both formalism, we follow the strategy of \citep{lhuillier2000bilan} by taking the Taylor expansion of each terms in \ref{eq:avg_dt_chi_f} for the case $k=2$. 
As the resulting expression can become quite cumbersome, we will adopt the following definition. 
Let $\mathbb{C}$ represent the phase-averaged equation of conservation (\ref{eq:avg_dt_chi_f}), namely, 
\begin{equation*}
    \mathbb{C}
    =
    - \pddt \avg{\chi_2f_2}
    - \nablabh \cdot \avg{\chi_2 \mathbf{\Phi}_2 - \chi_2f_2 \textbf{u}_2}
    + \avg{\chi_2 \textbf{S}_2}\\
    + \avg{\delta_I\left[
        \mathbf{\Phi}_2
        + f_2
        \left(
            \textbf{u}_I
            - \textbf{u}_2
        \right)
    \right]
    \cdot \textbf{n}_2}. 
\end{equation*}
Therefore, it must be understood from \ref{eq:avg_dt_chi_f} that $\mathbb{C}=0$.
Then, by taking the Taylor expansion of each terms of $\mathbb{C}$ according to \ref{eq:f_exp}, we can equally show that,
\begin{equation}
    \mathbb{C} = \mathbb{M}_0 - \nablab \cdot \mathbb{M}_1 + \frac{1}{2} \nablab\nablab : \mathbb{M}_2 \ldots = 0,
    \label{eq:scheme_equivalence}
\end{equation} 
where the expression $\mathbb{M}_0$ and $\mathbb{M}_1$ turn out to be, 
\begin{multline*}
    \mathbb{M}_0
    = 
    - \pddt \avg{\delta_\alpha (q_\alpha+q_{I_\alpha})}
    - \nablabh \cdot \avg{\delta_\alpha\textbf{u}_\alpha(q_\alpha+q_{I_\alpha})}
    + \avg{\delta_\alpha\int_{\Omega_\alpha} \textbf{S}_2 d\Omega}\\
    + \avg{\delta_\alpha\int_{\Sigma_\alpha} \textbf{S}_I d\Sigma}
    + \avg{\delta_\alpha\int_{\Sigma_\alpha} \left[\mathbf{\Phi}_1 + f_1 (\textbf{u}_I-\textbf{u}_1) \right] \cdot \textbf{n}_2 d\Sigma},
\end{multline*}
and,
\begin{multline*}
    \mathbb{M}_1 =
    - \pddt \avg{\delta_\alpha (\mathcal{Q}_\alpha+\mathcal{Q}_{I_\alpha})}
    - \nablabh \cdot \avg{\delta_\alpha\textbf{u}_\alpha(\mathcal{Q}_\alpha+\mathcal{Q}_{I_\alpha})}
    \\ + \avg{\delta_\alpha\int_{\Omega_\alpha} \left(
        \textbf{r} \textbf{S}_2         
        + f_2  \textbf{w}_2 
        - \mathbf{\Phi}_2
    \right) d\Omega}
    + \avg{\delta_\alpha\int_{\Sigma_\alpha} \left(
        \textbf{r}\textbf{S}_I
        + f_I \textbf{w}_I
        - \mathbf{\Phi}_{||}^I
    \right) d\Sigma}\\
    + \avg{\delta_\alpha\int_{\Sigma_\alpha} \textbf{r} \left[
        \mathbf{\Phi}_1
        + f_1 (\textbf{u}_I-\textbf{u}_1)
    \right]\cdot \textbf{n}_2  d\Sigma},
\end{multline*}
respectively. 
In the presence of \ref{eq:scheme_equivalence} we reach the major conclusion of this work. 
Indeed, we can observe that $\mathbb{M}_0$, $\mathbb{M}_1$ and $\mathbb{M}_2$ represent the zeroth, first and second order moments equations, respectively. 
In fact, it is shown in \ref{ap:Moments_equations} that the coefficient $\mathbb{M}_n$ in \ref{eq:scheme_equivalence} correspond to the $n^{th}$ order particle-average conservation equation. 
As a matter of fact, the phase average applied to the dispersed phase contains all the particle-averaged moments equations.
In \ref{ap:Moments_equations} we provide the expression for each moment equation $\mathbb{M}_n$ as well as the complete derivation of \ref{eq:scheme_equivalence}. 
In \cite{lhuillier2000bilan} they reached similar conclusion when comparing the area density phase-averaged and particle-averaged equations of conservation for spherical particles. 
Thus, from \ref{eq:scheme_equivalence} it is evident that one can use an arbitrary order of moments equations to reach an arbitrary accurate description of the dispersed phase.
This fact was previously suggested by \citet{zhang1997momentum}, here we provided the detailed derivation.  

% yielding directly, 
% \begin{align}    
%     0 &= 
%     - \pddt \pavg{q_\alpha} +  \nablab \cdot  \partial_t\pavg{\mathcal{Q}_\alpha} \nonumber\\
%     &+ \nablab \cdot \pavg{\int_{\Omega_\alpha}\left(\mathbf{\Phi}_k - f_k \textbf{u}_k \right)d\Omega}
%     -\nablab\nablab : \pavg{\int_{\Omega_\alpha}\textbf{r}\left(\mathbf{\Phi}_k - f_k \textbf{u}_k \right)d\Omega}
%     \nonumber\\
%     &+ \pavg{ \int_{\Omega_\alpha} \textbf{S}_k d\Omega}
%     - \nablab \cdot \pavg{ \int_{\Omega_\alpha} \textbf{r}\textbf{S}_k d\Omega}
%     \nonumber\\
%     &+ \pavg{\int_{\Sigma_\alpha}\left[
%         \mathbf{\Phi}_k
%         + f_k
%         \left(
%             \textbf{u}_I
%             - \textbf{u}_k
%         \right)
%     \right]
%     \cdot \textbf{n}_kd\Sigma} \nonumber\\
%     &-  \nablab \cdot \pavg{\int_{\Sigma_\alpha} \textbf{r}\left[
%         \mathbf{\Phi}_k
%         + f_k
%         \left(
%             \textbf{u}_I
%             - \textbf{u}_k
%         \right)
%     \right]
%     \cdot \textbf{n}_kd\Sigma} 
% \end{align}
% where we ignored the terms of second order or higher. 
% By using the relation : $\int_{\Omega_\alpha} f_k \textbf{u}_k d\Omega = q_\alpha\textbf{u}_\alpha  + \int_{\Omega_\alpha} f_k \textbf{w}_k d\Omega$
% together with $\int_{\Omega_\alpha} \textbf{r} \textbf{u}_k f_k d\Omega = Q_\alpha\textbf{u}_\alpha  + \int_{\Omega_\alpha}\textbf{r} f_k \textbf{w}_k d\Omega$ and  rearranging each terms of the equations gives,
% \begin{align}    
%     0 = 
%     &- \pavg{\ddt q_\alpha}
%     + \pavg{ \int_{\Omega_\alpha} \textbf{S}_k d\Omega}
%     + \pavg{\int_{\Sigma_\alpha}\left[
%         \mathbf{\Phi}_k
%         + f_k
%         \left(
%             \textbf{u}_I
%             - \textbf{u}_k
%         \right)
%     \right]
%     \cdot \textbf{n}_kd\Sigma} \nonumber\\
%     &-  \nablab \cdot  \left[
%         - \pavg{\ddt \mathcal{Q}_\alpha} 
%          + \pavg{ \int_{\Omega_\alpha} \left(
%             \textbf{r}\textbf{S}_k - \mathbf{\Phi}_k + \textbf{w}_kf_k 
%          \right)d\Omega}
%     \right.
%     \nonumber\\
%     &\left. 
%          + \pavg{\int_{\Sigma_\alpha} \textbf{r}\left[
%         \mathbf{\Phi}_k
%         + f_k
%         \left(
%             \textbf{u}_I
%             - \textbf{u}_k
%         \right)
%         \right]
%         \cdot \textbf{n}_kd\Sigma} 
%     \right]\nonumber\\
%     &+  \frac{1}{2}\nablab\nablab :
%          \pavg{ \int_{\Omega_\alpha} 2\textbf{r}\left(\mathbf{\Phi}_k + \textbf{w}_kf_k 
%          \right)d\Omega}
%          + \ldots
%     \label{eq:expansion_final}
% \end{align}
% In this last expansion we easily recognize that the first line correspond to \ref{eq:avg_dt_dq_alpha}, and each terms of the second and third lines constitute the average of \ref{eq:dt_Q_alpha}. 
% Besides, the higher order terms of this expansion constitute the higher moments equations derived in \ref{ap:moment_derivative} as it is shown in \ref{ap:Moments_equations}. 

Another approach to the problem is to notice that $\mathbb{M}_n=0$ for all $n$. Thus, we can rewrite \ref{eq:scheme_equivalence} such that all moments equations vanish, except $\mathbb{M}_0$, which gives, 
\begin{equation}
    \mathbb{C} = \mathbb{M}_0 = 0.
\end{equation}
This implies that equation \ref{eq:avg_dt_chi_f} is rigorously equivalent to \ref{eq:avg_dt_dq_alpha_tot} as shown by \cite[Appendix A]{nott2011suspension} for the specific case of the momentum conservation equation of solid spherical particle.
Nevertheless, it is important to note that this conclusion is not entirely objective since following the same procedure we could show equally that $\mathbb{C} = -\nablab\cdot\mathbb{M}_1=0$ and $\mathbb{C} = \frac{1}{2}\nablab\nablab:\mathbb{M}_2=0$ and so on for the other higher terms. 
Thus, it is preferable to look at the problem with the view of \ref{eq:scheme_equivalence}. 
Namely, the particle-averaged equations constitute a system of equations with one equation for each moment, while the phase-averaged equation contains all the terms of the particle-averaged equations within a single equation.
Therefore, the particle-averaged formalism is more meaningful since it provide one equation for each moment. 

%\tb{
%It is noteworthy to mention that in Population Balance Equations (PBE) we encounter a similar issue. 
%Indeed, for a given dispersed flows we wish to recover the size distribution of particles.
%Therefore, we derive with the help of Kinetic theory equation of the Probability density function of position and diameters size \citet{randolph2012theory}. 
%As solving for the PDF of size distribution is too expansive,  we then average PBE and derive moments equations \citep{fox2022hyperbolic}. 
%Applying similar consideration than above, we can carry out a Taylor expansion on each term of \ref{eq:dt_delta_alpha_q_alpha} in the phase space of the particles' diameter and recover all the size moment distribution equations. 
%Then if we want to recover the initial PDF completely we would need an infinite number of moments. 
%It is possible to generalize this thinking on any kind of internal coordinate of a particle.
%It is even possible to generalize it to relative internal coordinate \citet{zhang2021ensemble}.   
%As a conclusion, from \ref{eq:avg_dt_chi_f} we can recover any sort of moment equations expending on the variable used to carry out the expansion. 
%

\subsection{The hybrid model}

Now that we reached a clear understanding of the averaged models, it is time to introduce the Hybrid model for dispersed two phase flows. 
From \ref{fig:Scheme} we recall that the fluid or continuous phase is indexed $k=1$ while the particle phase by $k=2$.  

First, we introduce the equation used to model the continuous phase. 
It is derived by setting $k = 1$ in \ref{eq:avg_dt_chi_f}, and by using \ref{eq:f_exp} to expand the exchange term into a series.
This yields the following expression~:
\begin{multline}
    \pddt \avg{\chi_1 f_1}
    = \nablabh \cdot \avg{\chi_1 \mathbf{\Phi}_1 - \chi_1 f_1 \textbf{u}_1}
    - \pavg{\int_{\Sigma_\alpha}\left[
        \mathbf{\Phi}_1  
        + f_1
        \left(
            \textbf{u}_I
            - \textbf{u}_1
        \right)
    \right]
    \cdot \textbf{n}_2d\Sigma} \\
    + \avg{\chi_1 \textbf{S}_1}
    +  \nablab \cdot \pavg{\int_{\Sigma_\alpha} \textbf{r}\left[
        \mathbf{\Phi}_1
        + f_1
        \left(
            \textbf{u}_I
            - \textbf{u}_1
        \right)
    \right]
    \cdot \textbf{n}_2d\Sigma} 
    \label{eq:hybrid_avg_dt_chif}
\end{multline}
Notice that we switched the normal vector from $\textbf{n}_1$ to $-\textbf{n}_2$ in the second and the last integral on the RHS, which is the conventional way to express these couplings terms. 
Moreover, it should be noted that we neglected the second and higher order moments of the coupling term between the two phases \citep{jackson1997locally}.

Regarding the dispersed phase, we consider the zeroth and first moment equation (\ref{eq:avg_dt_dq_alpha_tot} and \ref{eq:avg_dt_dQ_alpha_tot}) as  a part of the hybrid model.
Under this form the couplings terms in \ref{eq:hybrid_avg_dt_chif} correspond to the ones appearing in both, \ref{eq:avg_dt_dq_alpha_tot} and \ref{eq:avg_dt_dQ_alpha_tot} which is of a great interest since we reduce the number of closure terms.
Then, \ref{eq:hybrid_avg_dt_chif}, \ref{eq:avg_dt_dq_alpha_tot} and \ref{eq:avg_dt_dQ_alpha_tot}, are the most general form of a first-order accurate hybrid model for arbitrary particle. 
Lastly, it is worth noting that by incorporating more terms in the expansion of \ref{eq:avg_dt_chi_f} and higher order moments equations, one can reach an arbitrary order of accuracy, as stated by \citet{zhang1997momentum}. 

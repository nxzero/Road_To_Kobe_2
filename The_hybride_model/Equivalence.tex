
% \subsection{The equivalence between continuous and particular averaged equations.}

In the averaged problem we can either use \ref{eq:avg_dt_chi_f} with $k=2$ or \ref{eq:avg_dt_dq_alpha} to describe the dispersed pahse. 
As mentioned in \ref{sec:dispersed-two-fluid} we notified that \ref{eq:dt_dq_alpha} and \ref{eq:dt_dq_I_alpha} were already subject to an average of length scale $a$. 
Consequently, it is fair to address the question of the compatibility and differences of the average of these equations. 
In this work we limit the derivation to the equivalence between \ref{eq:dt_chi_k_f_k} and \ref{eq:dt_dq_alpha} as for the interfacial equations the derivation reads essentially the same. 
To start with, it is well known that the weighting function $g(\textbf{x},\textbf{y})$ can be expressed as a Taylor series expansion around any particles center of mass $\textbf{y}_\alpha$ \citep{jackson1997locally},
\begin{equation}
    g(\textbf{x},\textbf{y})
    = g(\textbf{x},\textbf{y}_\alpha)
    - \textbf{r} \cdot \nablab g(\textbf{x},\textbf{y}_\alpha)
    + \frac{1}{2} \textbf{r}\textbf{r} : \nablab\nablab g(\textbf{x},\textbf{y}_\alpha)
    + \ldots
    \label{eq:g_exp}
\end{equation} 
where, we used the relation $\nablabh g_\alpha = - \nablab g_\alpha$, to make appear the macroscopic gradient operators. 
This relation is the starting point to carry out the comparison between both averaging methods. 
Indeed, from this Taylor expansion it can be shown that any phase and interface averaged quantities can be expressed as a Taylor expansion series of particular averaged quantities, yielding \citep{jackson1997locally},
\begin{align}
    \avg{\chi_kf_k} 
    &=  \pavg{q_\alpha}
        - \nablab \cdot  
        \pavg{\textbf{Q}_\alpha}        
        + \frac{1}{2} \nablab\nablab : \pavg{\textbf{Q}_\alpha^2}
        + \ldots  \\
    \avg{\delta_I f_k} 
    &=  \pavg{q_{I\alpha}}        
        - \nablab \cdot \pavg{\textbf{Q}_{I\alpha}}
        + \frac{1}{2} \nablab\nablab : \pavg{\textbf{Q}_{I\alpha}^{2}}
        + \ldots  
    \label{eq:f_exp}
\end{align}      
where, we have used the definition of the particular, volume and interface average, together with the \ref{eq:g_exp}, and we recall that the sum on the $\alpha$ particle is implicit. 
In this work we treat
To prove in which way those equations are equivalent we follow the strategy of \citep{lhuillier2000bilan} and take the Taylor expansion of each terms of \ref{eq:dt_dq_alpha} yielding, 
\begin{align}    
    0 &= 
    - \pddt \pavg{q_\alpha} +  \nablab \cdot  \partial_t\pavg{\textbf{Q}_\alpha} \ldots\nonumber\\
    &+ \nablab \cdot \pavg{\int_{\Omega_\alpha}\left(\mathbf{\Phi}_k - f_k \textbf{u}_k \right)d\Omega}
    -\nablab\nablab : \pavg{\int_{\Omega_\alpha}\textbf{r}\left(\mathbf{\Phi}_k - f_k \textbf{u}_k \right)d\Omega}
    \ldots\nonumber\\
    &+ \pavg{ \int_{\Omega_\alpha} \textbf{S}_k d\Omega}
    - \nablab \cdot \pavg{ \int_{\Omega_\alpha} \textbf{r}\textbf{S}_k d\Omega}
    \ldots\nonumber\\
    &+ \pavg{\int_{\Sigma_\alpha}\left[
        \mathbf{\Phi}_k
        + f_k
        \left(
            \textbf{u}_I
            - \textbf{u}_k
        \right)
    \right]
    \cdot \textbf{n}_kd\Sigma} \ldots\nonumber\\
    &-  \nablab \cdot \pavg{\int_{\Sigma_\alpha} \textbf{r}\left[
        \mathbf{\Phi}_k
        + f_k
        \left(
            \textbf{u}_I
            - \textbf{u}_k
        \right)
    \right]
    \cdot \textbf{n}_kd\Sigma} \ldots
\end{align}
where we have ignored the term of second order or higher. 
By using the relation : $\int_{\Omega_\alpha} f_k \textbf{u}_k d\Omega = q_\alpha\textbf{u}_\alpha  + \int_{\Omega_\alpha} f_k \textbf{w}_k d\Omega$
together with $\int_{\Omega_\alpha} \textbf{r} \textbf{u}_k f_k d\Omega = Q_\alpha\textbf{u}_\alpha  + \int_{\Omega_\alpha}\textbf{r} f_k \textbf{w}_k d\Omega$ and by  rearranging each terms of the equations we get, 
\begin{align}    
    0 = 
    &- \pavg{\ddt q_\alpha}
    + \pavg{ \int_{\Omega_\alpha} \textbf{S}_k d\Omega}
    + \pavg{\int_{\Sigma_\alpha}\left[
        \mathbf{\Phi}_k
        + f_k
        \left(
            \textbf{u}_I
            - \textbf{u}_k
        \right)
    \right]
    \cdot \textbf{n}_kd\Sigma} \nonumber\\
    &-  \nablab \cdot  \left[
        - \pavg{\ddt \textbf{Q}_\alpha} 
         + \pavg{ \int_{\Omega_\alpha} \left(
            \textbf{r}\textbf{S}_k - \mathbf{\Phi}_k + \textbf{w}_kf_k 
         \right)d\Omega}
    \right.
    \nonumber\\
    &\left. 
         + \pavg{\int_{\Sigma_\alpha} \textbf{r}\left[
        \mathbf{\Phi}_k
        + f_k
        \left(
            \textbf{u}_I
            - \textbf{u}_k
        \right)
        \right]
        \cdot \textbf{n}_kd\Sigma} 
    \right]\nonumber\\
    &+  \frac{1}{2}\nablab\nablab :
         \pavg{ \int_{\Omega_\alpha} 2\textbf{r}\left(\mathbf{\Phi}_k + \textbf{w}_kf_k 
         \right)d\Omega}
         + \ldots
    \label{eq:expansion_final}
\end{align}
In this last expansion we easily recognize each terms of the first line correspond to the \ref{eq:avg_dt_dq_alpha}, and each terms of the second and third line  make the average of \ref{eq:dt_Q_alpha}. 
Similar consideration can be made for the surfaces equations.
Therefore, the continuous phase averaged equations are a combination of the particular averaged equations, from the zeroth to an infinite number of moments equations. 
To expose this fact more clearly let's write \ref{eq:avg_dt_chi_f} by $C(\chi_2f_2) = 0$ where we introduced the operator $C$.
Similarly let's write \ref{eq:avg_dt_dq_alpha} $M(f_k) = 0$ and the higher order moments equation as  $M(\textbf{Q}_\alpha) =0$, $M(\textbf{Q}_\alpha) =0$ and so on. 
Then, what we demonstrated from \ref{eq:expansion_final} is that, 
\begin{equation}
    C(\chi_2 f_2) = M(q_\alpha) - \nablab \cdot M(\textbf{Q}_\alpha) + \frac{1}{2} \nablab\nablab : M(\textbf{Q}^2_\alpha) \ldots = 0,
    \label{eq:scheme_equivalence}
\end{equation} 
in agreement with \cite{lhuillier2000bilan}, which reach similar conclusion when comparing the area density continuous and particular averaged equation of spherical particle. 
Considering that $M(\textbf{Q}_\alpha) =0$ and $M(\textbf{Q}_\alpha) =0$ in \ref{eq:dt_Q_alpha} and \ref{ap:eq:dt_Q_alpha_n} we can rewrite \ref{eq:scheme_equivalence} such as, 
\begin{equation}
    C(\chi_2 f_2) = M(q_\alpha).
\end{equation}
Meaning that \ref{eq:avg_dt_chi_f} is rigorously equivalent to \ref{eq:avg_dt_dq_alpha} in agreement with what \cite{nott2011suspension} concluded.
Nevertheless, it is true if and only if \ref{eq:dt_Q_alpha} and the other higher moments equations are considered.
Therefore, regardless the way we look at the problem the continuous averaged equation, $C(\chi_kf_k)=0$, is similar to the whole set of particular equations $M(q_\alpha) =0$ $M(\textbf{Q}_\alpha) =0$, $M(\textbf{Q}^2_\alpha) =0$, and higher moments. 

It is noteworthy to mention that in Population Balance Equations (PBE) we encounter a similar issue. 
Indeed, for a given dispersed flows we wish to recover the size distribution of particles.
Therefore, we derive with the help of Kinetic theory equation of the Probability density function of position and diameters size \citet{randolph2012theory}. 
As solving for the PDF of size distribution is too expansive,  we then average PBE and derive moments equations \citep{fox2022hyperbolic}. 
Applying similar consideration than above, we can carry out a Taylor expansion on each term of \ref{eq:dt_delta_alpha_q_alpha} in the phase space of the particles' diameter and recover all the size moment distribution equations. 
Then if we want to recover the initial PDF completely we would need an infinite number of moments. 
It is possible to generalize this thinking on any kind of internal coordinate of a particle.
It is even possible to generalize it to relative coordinate \citet{zhang2021ensemble}.   
As a conclusion, from \ref{eq:avg_dt_chi_f} we can recover any sort of moment equations expending on the variable used to carry out the expansion. 


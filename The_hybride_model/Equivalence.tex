
%\subsubsection*{Equivalence between particle and continuous models}
\subsection{Link between particle-averaged and phase-averaged equations}
\label{sec:equivalence}
%To model the dispersed phase we can either use \ref{eq:avg_dt_chi_f} with $k=d$, or the particle-averaged equations: \ref{eq:avg_dt_dq_alpha_tot}, \ref{eq:avg_dt_dQ_alpha_tot} and possibly the higher moments equations in \ref{ap:Moments_equations}. 
%Consequently, it is fair to address the question of the compatibility and differences between both formalisms. 
To model the dispersed phase, there are two distinct approaches. 
We can either use \ref{eq:avg_dt_chi_f} with $k=d$, or we can employ the particle-averaged equations \ref{eq:avg_dt_dq_alpha_tot}, \ref{eq:avg_dt_dQ_alpha_tot} and potentially the higher moments equations found in \ref{ap:Moments_equations}.
% Consequently, it is important to address the compatibility between these two formalisms.
To better understand the physical significance of this choice and determine which formalism is more appropriate, it is essential to discuss the relationship that connects these two formalisms. 

It has been demonstrated in various studies \citep{buyevich1979flow,lhuillier1992ensemble,zhang1994averaged}, that phase-averaged quantities can be expressed as a Taylor series expansion of particle-averaged quantities. 
The aforementioned studies used the single-particle conditionally averaged approach to demonstrate this equivalence.  
%In this work, we adopt the "distributional" approach introduced by \citet{pahtz2023general}, as it elucidates the connection between phase quantities and moment expansions before applying any averaging formalism.
In this work, we follow the "distributional" approach proposed by \citet{pahtz2023general}, as it clarifies the relationship between phase quantities and moment expansions prior to the application of any averaging formalism.
%highly general, 
%as demonstrated below. 
%Furthermore, this approach elucidates the connection between phase and surface quantities and moment expansions.
%In this work we use instead the ``distributional'' approach introduced by \citet{pahtz2023general} since, as shown below it is very general. 
%Moreover this approach makes clear the link between the phasis quntites and the moments expansion.% yields more general and simpler formulation. 
As demonstrated by \citet{pahtz2023general} (see also \ref{app:expansion} for a demonstration in the sense of distribution) we may show
% \begin{equation}
%     (f^0_d \chi_\alpha)[\textbf{x}]
%     = 
%     \delta_\alpha
%     \int_{\mathbb{R}^3}
%         f^0_d\chi_\alpha
%     d\textbf{r}
%     + \div\left(    
%     \delta_\alpha
%     \int_{\mathbb{R}^3}
%     \textbf{r}
%     f^0_d\chi_\alpha
%     d\textbf{r}
%     \right)
%     + \ldots
% \end{equation}
\begin{equation}
    \chi_\alpha f^0_d 
    =\delta_\alpha\intO{f^0_d}
    - \div\left(\delta_\alpha\intO{\textbf{r} f^0_d}\right)
    + \frac{1}{2}\grad\grad :\left(\delta_\alpha\intO{\textbf{rr} f^0_d}\right)-
    \ldots 
    \label{eq:fd_asympt0}
\end{equation}
We recall that $\chi_\alpha = 1$ within the particle domain $\Omega_\alpha(\FF, t)$ and $0$ otherwise. 
Additionally, we have extended this approach to surface quantities in \ref{app:expansion} , resulting in
\begin{equation}
    \delta_{\Gamma\alpha}f_\Gamma^0  
=\delta_\alpha\intS{f^0_\Gamma}
- \div\left(\delta_\alpha\intS{\textbf{r} f^0_\Gamma}\right)
+ \frac{1}{2}\grad\grad :\left(\delta_\alpha\intS{\textbf{rr} f^0_\Gamma}\right)-
\ldots 
\label{eq:fG_asympt0}
\end{equation}
Likewise, we recall that $\delta_{\Gamma\alpha} = 1$ on the particle surface $\Gamma_\alpha(\FF, t)$ and $0$ otherwise.
We can identify the zeroth, first and second order moments of $f_d^0$ and $f_\Gamma^0$, into \ref{eq:fd_asympt0} and \ref{eq:fG_asympt0}, respectively. 
Then summing \ref{eq:fd_asympt0} and \ref{eq:fG_asympt0} we obtain,
\begin{equation}
    \chi_\alpha f^0_d  + \delta_{\Gamma\alpha} f_\Gamma^0  = \delta_\alpha\text Q_\alpha
    - \div  
    (\delta_\alpha\textbf{Q}_\alpha^{(1)})        
    + \frac{1}{2} \grad\grad : (\delta_\alpha\textbf{Q}_{\alpha}^{(2)})
    - \ldots  
    \label{eq:f_asympt0}
\end{equation}
It is important to observe that, even before any averaging procedure is applied, equation \ref{eq:f_asympt0} already demonstrate the link between the dispersed phase fields, even for a single particle. 
Notably, this relationship is valid in a distributional sense at the local level.%It is interesting to note that these relations hold in a distributional sense at a local level.

The dispersed phase indicator function $\chi_d$ can be expressed as a sum of phase indicator function, $\chi_d(\textbf{x},t,\FF) = \sum_\alpha\chi_\alpha(\textbf{x},t,\FF)$. 
Likewise, the interface indicator function $\delta_\Gamma$ can be written as $\delta_\Gamma =  \sum_\alpha  \delta_{\Gamma\alpha}$.
% Thus, any dispersed phase quantity pertaining to a single particle can be written as, 
% \begin{equation}
%    f^0_d \chi_\alpha(\textbf{x},t,\FF)
%    = 
%    \int_{\mathbb{R}^3} 
%     f^0_d \chi_\alpha(\textbf{x}_\alpha + \textbf{r},t,\FF)\delta(\textbf{x} - \textbf{x}_\alpha - \textbf{r}) 
%     d\textbf{r} 
%    \label{eq:taylor_f_d}
% \end{equation}
%In that case any surface-averaged quantities may be written, 
% \begin{equation}
%     f_\Gamma^0 \delta_\Gamma(\textbf{x},t,\FF) = 
%     \sum_\alpha 
%     \int_{\mathbb{R}^3} 
%      f_\Gamma^0 \delta_{\Gamma\alpha}(\textbf{x}_\alpha + \textbf{r},t,\FF)\delta(\textbf{x} - \textbf{x}_\alpha - \textbf{r}) 
%      d\textbf{r}. 
%     \label{eq:taylor_f_I}
% \end{equation}
% Note that \ref{eq:taylor_f_d} and \ref{eq:taylor_f_I} are well-defined in the distributional sense since the integral on the right-hand side of both equations correspond to a convolution product.
%Note that the integral on the right-hand side of \ref{eq:taylor_f_d} and \ref{eq:taylor_f_I} corresponds to a convolution product.
%Additionally, since the Dirac distribution $\delta(\textbf{x} - \textbf{x}_\alpha - \textbf{r})$, is the unit of convolution \ref{eq:taylor_f_d} is verified (see \citet[Chapter 9]{appel2007}).
%The convolution product of the Dirac delta and the derivative of the Heaviside distribution is also well-defined, see \citet[Chapter 9]{appel2007}.
%It follows that \ref{eq:taylor_f_d} and \ref{eq:taylor_f_I} are well-defined in the distributional sense. 
%Upon using the Taylor expansion of the Dirac delta function $\delta(\textbf{x} - \textbf{x}_\alpha - \textbf{r})$ in the neighborhood of $\textbf{r}=0$ one obtain,
% \begin{equation}
% \delta(\textbf{x} - \textbf{x}_\alpha - \textbf{r})
% = \delta(\textbf{x} - \textbf{x}_\alpha)
% - \textbf{r}\cdot\grad \delta(\textbf{x} - \textbf{x}_\alpha)
% + \frac{\textbf{rr}}{2}:\grad\grad\delta(\textbf{x} - \textbf{x}_\alpha) 
% - \ldots.
% % + \ldots
% \label{eq:exp_delta}
% \end{equation}
%Injecting \ref{eq:exp_delta} into \ref{eq:taylor_f_d} and \ref{eq:taylor_f_I}, and noticing that the indicator functions, $\chi_\alpha$ and $\delta_\Gamma$, reduce the domain of integration from $\mathbb{R}^3$ to, $\Omega_\alpha$ and $\Gamma_\alpha$, respectively,  yields: 
% \begin{align}
%     f^0_d \chi_d
%     =\delta_p\intO{f^0_d}
%     - \div\left(\delta_p\intO{\textbf{r} f^0_d}\right)
%     + \frac{1}{2}\grad\grad :\left(\delta_p\intO{\textbf{rr} f^0_d}\right)
%     \ldots 
%     \label{eq:fd_asympt}
%    \\
%    f_\Gamma^0 \delta_\Gamma 
%    =\delta_p\intS{f^0_\Gamma}
%    - \div\left(\delta_p\intS{\textbf{r} f^0_\Gamma}\right)
%    + \frac{1}{2}\grad\grad :\left(\delta_p\intS{\textbf{rr} f^0_\Gamma}\right)
%    \ldots 
%    \label{eq:fG_asympt}
% %    \\
% \end{align} 
%Applying similar considerations to the interface indicator function $\delta_\Gamma$, and 
Thus, summing \ref{eq:fd_asympt0} and \ref{eq:fG_asympt0} over all particles in the domain and averaging over all configurations, gives the general relations that link continuous-averaged and particle-averaged fields, namely \citep{lhuillier1992ensemble,lhuillier1998,lhuillier2000bilan}, 
\begin{align}
    \avg{\chi_df_d^0} 
    &=  \pavg{\text q_\alpha}
        - \div  
        \pavg{\textbf{q}_\alpha^{(1)}}        
        + \frac{1}{2} \grad\grad : \pavg{\textbf{q}_{\alpha}^{(2)}}
        + \ldots  \label{eq:f_exp_chi} \\
    \avg{\delta_\Gamma  f_\Gamma ^0} 
    &=  \pavg{\text q_{\Gamma \alpha}}        
        - \div \pavg{\textbf{q}_{\Gamma\alpha}^{(1)}}
        + \frac{1}{2} \grad\grad : \pavg{\textbf{q}_{\Gamma\alpha}^{(2)}}
        - \ldots  
    \label{eq:f_exp_delta}
\end{align}
%\color{black}
%\JL{j'ai ajoute la sommes des contributions dans les particules et de surfaces}
Summing \ref{eq:f_exp_chi} and \ref{eq:f_exp_delta} we obtain
\begin{equation}
    \avg{\chi_df_d^0+\delta_\Gamma  f_\Gamma ^0} = \pavg{\text Q_\alpha}
    - \div  
    \pavg{\textbf{Q}_\alpha^{(1)}}        
    + \frac{1}{2} \grad\grad : \pavg{\textbf{Q}_{\alpha}^{(2)}}
    - \ldots  \label{eq:f_exp}
\end{equation}
When considering an infinite number of terms in \ref{eq:f_exp} one might eventually obtain a converged approximation of $\avg{\chi_d f_d^0+\delta_\Gamma  f_\Gamma ^0}$. 
However, it is important to note that Taylor series have what is known as a \textit{radius of convergence} beyond which adding more terms does not necessarily improve the approximation \citep[Chapter 1]{appel2007}. 
In particular, for distances beyond a certain limit \textbf{r} the series might diverges depending on the behavior of the function $\avg{\chi_d f_d^0+\delta_\Gamma  f_\Gamma ^0}$ near the point $\textbf{x}$. 
For the purposes of this article, we will assume that the Taylor series has an infinite radius of convergence, although this assumption warrants further investigation.%function $f_d^0$ evaluated at a point $\textbf{x}_\alpha$ is greater than the particle size, then \ref{eq:f_exp} might converges and provides a good approximation.
%\JL{j'ai vraiment raccourci cette partie, car meme si je la trouve pertinente il y a des elements que je trouvais peu claire:
%\begin{itemize}
%\item tu dis que "to assume that $f_d$ is slowly variying at the scale of the particle", justement je ne pense pas que l'on veuille cela car sinon a quoi serve les moments d'ordres superieurs.
%Par ailleurs pr moi (mais peut etre que je me trompe), le rayon de convergence dsun' serie de Taylor n'a rien a voir avec le fait que la fonction dont on cherche la serie varie peu à l'endroit du developpement
%Enfin sur cette idée de precision de la série, pr moi le developpement en série se fait dans l'hypothèse ou la grandeur d'interet (moyennée) varie peut sur l'échelle des grandeurs macroscopiques
%en gros un developpement limite en $a/L$ ou $L$ est la taille des echelles macro.
%\end{itemize}
%}

%In brief, high care must be taken when using these kind of taylor expansion especially in that context since we do not know the exact form of $f_d$.  
%Nevertheless it is reasonable to assume that $f_d$ is slowly variying at the scale of the particle with a radius of convergence sufficiently large, in this case \ref{eq:f_exp} might provide a good approximation, 
%and we might expect an error of $\mathcal{O}[(a/L)^{n}]$ when the highest moment of the series is of order $n-1$ with $L$ being a macroscopic length scale. 
%It is within the context of this assumption that the following disscussion takes place. 

% \JL{pour l'instant j'ai eneleve la partie applicative (meme si elle me semble tres interessante). 
% D'ailleurs pq le second terme du dvt pr les conservation de la masse est nul ? 
% Ce serait bien de donner de petites lois d'echelles pr evaluer les ordres de grandeurs de chacun des termes.}
%Particularly we note that if $f_d^0 = \rho_d$ and $f_d^0 = \rho_d \textbf{u}_d^0$ we obtain, 
%\begin{align}
%    \label{eq:f_exp_exe1}
%    \phi_d \rho_d
%    = m_p n_p 
%    + \frac{1}{2}\grad^2 : (n_p\textbf{M}_p)+\ldots,\\
%    \phi_d \rho_d \textbf{u}_d
%    = m_p n_p \textbf{u}_p 
%    - \div (n_p\textbf{P}_p)+\ldots,
%    \label{eq:f_exp_exe}
%\end{align}
%respectively. 
%Meaning that $\phi_d\rho_d$ is related to the shape of the particles, represented by $\textbf{M}_p$ through \ref{eq:f_exp_exe1}.
%Additionally, considering \ref{eq:f_exp_exe}, it becomes apparent that the phase-averaged velocity $\textbf{u}_d$ encompasses the first moment of momentum $\textbf{P}_p$, which as discussed (in \ref{sec:Lagrangian}) accounts for the rotational, dilatational, and stretching motions of the particles. 
%The second terms on the right-hand side of \ref{eq:f_exp_exe1} and \ref{eq:f_exp_exe} become negligible for homogeneous mixture, i.e. if $n_p$, $\textbf{M}_p$ and $\textbf{P}_p$ are not function of \textbf{x}. 
%Conversely, these terms might become significant if $n_p$, $\textbf{M}_p$ or $\textbf{P}_p$ are space-dependent.
%For example, close to solid boundaries of a macroscopic flow strong gradients of $n_p$ are present at the particle length scale, since at the exact location of the boundaries we must respect $n_p = 0$. 
%In \cite{prosperetti1995finite} they study the importance of these terms, especially their remark that the approximation $\phi \approx n_p v_p$ may have significant consequence on the hyperbolicity of a two-phase flow system. 

To demonstrate the equivalence between the two formalisms, we follow a strategy similar to \citep{lhuillier2000bilan,lhuillier2009rheology}. 
%\JL{j'ai simplifie la description}
%We take the Taylor expansion of each terms in \ref{eq:avg_dt_chi_f} with $k=d$ using the relation \ref{eq:f_exp_chi}. 
%A similar procedure is followed for  the surface transport equations.
%Since we made use of the surface transport equations in the particles phase equations : \ref{eq:avg_dt_dq_alpha_tot} and \ref{eq:avg_dt_dQ_alpha_tot}, we also consider \ref{eq:avg_dt_delta_f} to prove equivalence. 
%As the resulting expression can become quite cumbersome, we will adopt the following definition. 
Let $\mathcal{C}_d$ denote the phase-averaged equation of conservation (\ref{eq:avg_dt_chi_f} with $k=d$) and $\mathcal{C}_\Gamma $ the averaged surface transport equation (\ref{eq:avg_dt_delta_f}).
Specifically, they are defined as follows
\begin{align}
    \mathcal{C}_d
    &=
    - \pddt \avg{\chi_df_d^0}
    - \div \avg{\chi_d \mathbf{\Phi}_d^0 - \chi_df_d^0 \textbf{u}_d^0}
    + \avg{\chi_d s_d^0}
    + \avg{\delta_\Gamma \left[
        \mathbf{\Phi}_d^0
        + f_d^0
        \left(
            \textbf{u}_\Gamma ^0
            - \textbf{u}_d^0
        \right)
    \right]
    \cdot \textbf{n}_d},\\
    \mathcal{C}_\Gamma 
    &= 
    -\pddt \avg{\delta_\Gamma f_\Gamma ^0}
    -\div \avg{\delta_\Gamma  f_\Gamma ^0 \textbf{u}_\Gamma ^0-\delta_\Gamma  \mathbf{\Phi}_{I||}^0 }
    + \avg{\delta_\Gamma s_\Gamma ^0} 
    - \avg{\delta_\Gamma  \Jump{
     \mathbf{\Phi}_k^0+
    f_k^0 (\textbf{u}_\Gamma ^0 - \textbf{u}_k^0)
    } }. 
\end{align}
It should be noted from \ref{eq:avg_dt_chi_f} and \ref{eq:avg_dt_delta_f} that $\mathcal{C}_d\equiv 0$ and $\mathcal{C}_\Gamma  \equiv 0$.
By applying the Taylor expansion to each term of $\mathcal{C}_d+\mathcal{C}_\Gamma $ as described in \ref{eq:f_exp} yields
\begin{equation}
    \mathcal{C}_d 
    + \mathcal{C}_\Gamma  
    = \mathcal{M}^{(0)} - \div \mathcal{M}^{(1)} + \frac{1}{2} \grad\grad : \mathcal{M}^{(2)} \ldots = 0,
    \label{eq:scheme_equivalence}
\end{equation} 
where the expressions for $\mathcal{M}^{(0)}$ and $\mathcal{M}^{(1)}$ are given by 
\begin{align}
    &\mathcal{M}^{(0)}
    = 
    - \avg{\delta_p \ddt {\text Q_\alpha}}
    % -\avg{\delta_p\textbf{u}_\alpha q_\alpha^\text{tot}}
    + \pOavg{ s_d^0 }
    + \pSavg{ s_\Gamma ^0 }
    + \pSavg{ 
    \left[\mathbf{\Phi}_f^0 
    + f_f^0 (\textbf{u}_\Gamma ^0-\textbf{u}_f^0) \right] \cdot \textbf{n}_d },\\
    &\mathcal{M}^{(1)} =
    -  \avg{\delta_p \ddt {\textbf{Q}_\alpha^{(1)}}}
    % - \avg{\delta_p\textbf{u}_\alpha \textbf{Q}_\alpha^\text{tot}}
     + \pOavg{ \left(
        \textbf{r} s_d^0         
        + f_d^0  \textbf{w}_d^0 
        - \mathbf{\Phi}_d^0
    \right) }
    + \pSavg{ \left(
        \textbf{r}s_\Gamma ^0
        + f_\Gamma ^0 \textbf{w}_\Gamma ^0
        - \mathbf{\Phi}_{\Gamma||}^0
    \right) } \nonumber\\
    &+ \pSavg{ \textbf{r} \left[
        \mathbf{\Phi}_f^0
        + f_f^0 (\textbf{u}_\Gamma ^0-\textbf{u}_f^0)
    \right]\cdot \textbf{n}_d  }.
\end{align}
Using \ref{eq:scheme_equivalence}, we reach one of the main conclusion of this study. 
We observe that $\mathcal{M}^{(0)}$ and $\mathcal{M}^{(1)}$ correspond to the zeroth and first-order moment equations, respectively. 
Additionally, as demonstrated in \ref{ap:Moments_equations} the coefficient $\mathcal{M}^{(n)}$ in \ref{eq:scheme_equivalence} represents the $n^{th}$ order moment in the particle-averaged conservation equation. 
From \ref{eq:scheme_equivalence} we conclude that combining \ref{eq:avg_dt_chi_f} for $k=d$ and \ref{eq:avg_dt_delta_f} effectively captures the particle moment equations through a Taylor expansion around the particle center of mass. 
%Thus, it is evident that one can use an arbitrary order of particles moments equations to achieve an arbitrarily accurate description of the dispersed phase, regardless of the properties of the multiphase flow.
Therefore, it is clear that by considering an arbitrary order of particle moments equations, one can achieve a highly accurate description of the dispersed phase.


The particle-averaged equations ($\mathcal{M}^{(0)}$\ldots $\mathcal{M}^{(n)}$) form a system with $n$ equations, one for each moment. 
In contrast, the dispersed phase-averaged equations ($\mathcal{C}_d$ and $\mathcal{C}_\Gamma$) consist of only two equations, which aggregate all the particle-averaged equations. 
This indicates that the particle-averaged formalism provides more information since it yields a separate equation for each moment, as opposed to the phase-averaged equations, which are limited to just two. 
This enhanced level of detail is achieved by considering the topology of the dispersed phase as demonstrated in the previous sections.
%Therefore, the particle-averaged formalism encompasses more information since it provides one equation for each moment, in opposition to the phase averaged equations which are only two. 
%Note that this gain in information has been possible through the consideration of the topology of the dispersed phase. 




%In \ref{ap:Moments_equations} we provide the expression for each $\mathcal{M}_n$ as well as the complete derivation of \ref{eq:scheme_equivalence}. 

%Another approach is to note that $\mathcal{M}_n=0$ for all $n$ since \ref{eq:dt_Q_n} holds for all $n$. 
%Thus, we can rewrite \ref{eq:scheme_equivalence} such that all moments equations vanish, except $\mathcal{M}_0$ (which is arbitrary), this gives, 
%\begin{equation}
%    \mathcal{C}_d 
%    + \mathcal{C}_\Gamma 
%    = \mathcal{M}^{(0)} = 0.
%    \label{eq:proof2}
%\end{equation}
%This implies that equation \ref{eq:avg_dt_chi_f} with the surface transport equation \ref{eq:avg_dt_delta_f} is rigorously equivalent to \ref{eq:avg_dt_dq_alpha_tot}.
%\citet[Appendix A]{zhang1997momentum} provided evidences that the particle-averaged momentum equation is as legitimate as the phase-averaged momentum equation, which is consistent with \ref{eq:proof2}. 
%Additionally, \citet[Appendix A]{nott2011suspension} derived a similar expression than \ref{eq:proof2}, also in the case of the averaged momentum equation for suspension of solid spherical particles.
%Thus, in light of \ref{eq:proof2}, we generalize the conclusion of these authors and demonstrated that this is also true for all conservation laws regardless of the dispersed phase nature.  
%Considering the Lagrangian equations derived in \ref{sec:Lagrangian} this conclusion is not surprising at all since the phase-averaged and particle-averaged equations are all built on \ref{eq:dt_f_k} and \ref{eq:dt_f_I}.
%However, if one does not consider a proper derivation of the lagrangian balance equations as it is done in \ref{sec:Lagrangian} it might not be as obvious, even if \ref{eq:proof2} should remain true as demonstrated by \citet{zhang1997momentum,nott2011suspension}.
%Nevertheless, it is important to note that the conclusion given by \ref{eq:proof2} is not entirely objective since following the same procedure we could show equally that $\mathcal{C}_d+\mathcal{C}_\Gamma  = -\div\mathcal{M}^{(1)}=0$ and $\mathcal{C}_d+\mathcal{C}_\Gamma  = \frac{1}{2}\grad\grad:\mathcal{M}^{(2)}=0$ and so on. 
%Thus, it is more appropriate to examine the problem from the perspective of \ref{eq:scheme_equivalence}. 
%Namely, the particle-averaged equations ($\mathcal{M}^{(1)}$\ldots $\mathcal{M}^{(n)}$) constitute a system of equations with $n$ equations, one equation for each moment, while the phase-averaged equations ($\mathcal{C}_d$ and $\mathcal{C}_\Gamma$) is a system of two equations made of all the particle-averaged equations.
%Therefore, the particle-averaged formalism encompasses more information since it provides one equation for each moment, in opposition to the phase averaged equations which are only two. 
%Note that this gain in information has been possible through the consideration of the topology of the dispersed phase. 




\subsection{Conservation equations}

Given that the aim of this work is not only to demonstrate the relationship between particle and phase-averaged formalisms but also to establish a comprehensive framework for analyzing dispersed two-phase flows, we now present \textit{the hybrid} set of conservation equations.
%Let us assume that we are interested by a macroscopic quantity $f$ that follows \ref{dt_f} at the local scale, (here $f^0$ could be the mass, momentum, concentration of chemical species etc \ldots) . 
The system of equations governing a macroscopic quantity $f$ consists of one equation for the fluid phase, which ensures the conservation of $f_f$ and $n$ equations for the dispersed phase, representing the conservation of the quantities $\textbf{Q}_p^{(n)}$.  
In its most general form, the hybrid description of $f$ can be expressed as
%The system of equations for a macroscopic quantity $f$ is constituted from one equation describing the fluid phase, meaning the conservation of $f_f$ and $n$ equations describing the dispersed phase, i.e. the conservation of the $\textbf{Q}_p^{(n)}$.  
%In all its generality the hybrid description of $f$ may be written,
\begin{align}
    \pddt (\phi_f f_f)
    +\div (\phi_f f_f \textbf{u}_f + \mathbf{\Phi}_f^\text{eff})
    &= 
    \phi_f s_f
    - \pSavg{\left[
        \mathbf{\Phi}_f^0
        + f_f^0
        \left(
            \textbf{u}_\Gamma^0
            - \textbf{u}_f^0
        \right)
    \right]
    \cdot \textbf{n}_d} ,
    \label{eq:avg_hybrid_dt_chi_f}\\
        % \pddt \pavg{[\textbf{Q}_\alpha^{(n)}]_{i_1\ldots i_n}^\alpha}
        % + \div  \pavg{\textbf{u}_\alpha [\textbf{Q}_\alpha^{(n)}]_{i_1\ldots i_n}^\alpha}
        % = \sum_{e=1}^{n} 
        % \pOavg{
        %     \prod^{n}_{\substack{ m=1 \\m \neq e}} r_{i_m} [f_d^0\textbf{w}_d^0  - \bm\Phi_d^0]_{i_e}
        % }\nonumber\\
        % + \pOavg{ \pri{1}{n} (\textbf{s}_d^0)_k }
        % +     
        % \sum_{e=1}^{n} 
        % \pSavg{
        %     \prod^{n}_{\substack{ m=1 \\m \neq e}} r_{i_m} [f_\Gamma^0\textbf{w}_\Gamma^0 - \bm\Phi_{||\Gamma}^0]_{i_e}
        % }
        % + \pSavg{ \pri{1}{n} (\textbf{s}_\Gamma^0)_k }\nonumber\\
        % +\pSavg{ \pri{1}{n} ([\bm\Phi_f^0 + \textbf{f}_f^0 \left(\textbf{u}_\Gamma^0 - \textbf{u}_f^0\right)]\cdot \textbf{n}_d)_k }.
        \pddt (n_p\text Q_p)
        + \div (n_p \text Q_p \textbf{u}_p + \pavg{\textbf{u}_\alpha' \text Q_\alpha'})
        &= \pOavg{ s_d^0 }
        + \pSavg{ s_\Gamma^0 }\nonumber\\
        &+ \pSavg{ \left[\mathbf{\Phi}_f^0 + f_f^0 (\textbf{u}_\Gamma^0-\textbf{u}_f^0) \right] \cdot \textbf{n}_d },
        \label{eq:avg_hybrid_q}
        \\
        \pddt (n_p\textbf{Q}_p^{(1)})
        + \div (n_p \textbf{Q}_p^{(1)} \textbf{u}_p + \pavg{\textbf{u}_\alpha' (\textbf{Q}_\alpha^{(1)})'})
        &=\pOavg{ \left(
            \textbf{r} s_d^0         
            + f_d^0  \textbf{w}_d^0 
            - \mathbf{\Phi}_d^0
        \right) }\nonumber\\
        + \pSavg{ \left(
            \textbf{r}s_\Gamma^0
            + f_\Gamma^0 \textbf{w}_\Gamma^0
            - \mathbf{\Phi}_{I||}^0
        \right) }
        &+ \pSavg{ \textbf{r} \left[
            \mathbf{\Phi}_f^0
            + f_f^0 (\textbf{u}_\Gamma^0-\textbf{u}_f^0)
        \right]\cdot \textbf{n}_d  },
        \label{eq:avg_hybrid_q_1}
        \\\nonumber
        \vdots
\end{align}
where the effective continuous phase non-convective flux term reads 
\begin{align}
    \mathbf{\Phi}_f^\text{eff}
    = \avg{\chi_f f_f' \textbf{u}_f'}
    - \avg{\chi_f \bm\Phi_f^0}
    - \pSavg{\textbf{r}\left[
        \mathbf{\Phi}_f^0
        + f_f^0
        \left(
            \textbf{u}_\Gamma^0
            - \textbf{u}_f^0
        \right)
    \right]
    \cdot \textbf{n}_d}
    + \div[\ldots].
\end{align}
%and \eqref{eq:avg_hybrid_dt_chi_f} is that in
The only difference in the conservation equation for the continuous phase between \eqref{eq:avg_dt_chi_f} and \eqref{eq:avg_hybrid_dt_chi_f} is the expansion of the exchange term $\avg{\delta_\Gamma \left[
    \mathbf{\Phi}_f^0
    + f_f^0
    \left(
        \textbf{u}_\Gamma ^0
        - \textbf{u}_f^0
    \right)
\right]
\cdot \textbf{n}_d}$
 into a Taylor series in a similar way to \ref{eq:f_exp_delta}. 
The presence of the terms $\div[\ldots]$ in the expression for  $\mathbf{\Phi}_f^\text{eff}$ suggests that higher-order moments of the interphase exchange term are involved.
Similarly, the ellipsis below \ref{eq:avg_hybrid_q_1} implies that an arbitrary number of dispersed phase moment equations can be introduced. In this format, it becomes evident that the exchange term on the right-hand side of \ref{eq:avg_hybrid_dt_chi_f} is identical to that on the right-hand side of \ref{eq:avg_hybrid_q}. 
Additionally, in the effective flux $\mathbf{\Phi}_f^\text{eff}$ the exchange term from  \ref{eq:avg_hybrid_q_1} appears, and this property continues for higher-order moments.  
Consequently, the zeroth order exchange term in the equation for $\text Q_\alpha^{(0)}$ plays the role of a source term for $f_f$, while the first and higher order exchange terms act as a source into the higher moment equations, and contribute to the effective non-convective fluxes for $f_f$. 



The system of equations presented here offers a clear understanding of the roles of the dispersed phase non-convective flux terms, $\bm\Phi_d$ and $\bm\Phi_\Gamma$, in the particle phase conservation equation. 
As evidenced in \ref{eq:avg_hybrid_q}, $\bm{\Phi}_d$ and $\bm{\Phi}_\Gamma$  do not influence the lowest order particle-phase averaged conservation equation, i.e. the equation of $n_p \text Q_p$. 
However, \ref{eq:avg_dt_chi_f} (for $k = d$) and \ref{eq:avg_dt_delta_f}, show that the phase-averaged quantities $f_d$ and $f_\Gamma$, are affected by the non-convective fluxes at the particle surface and internally, since $\bm{\Phi}_d$ and $\bm{\Phi}_\Gamma$ appear in these equations.
This may seem contradictory at first, but it is important to note that $\bm{\Phi}_d^0$ and $\bm{\Phi}_\Gamma^0$ serve as source terms in the conservation equations for higher moments such as $\textbf{Q}^{(1)}_p$ \ldots $\textbf{Q}^{(n)}_p$, which are related to $f_d$ and $f_\Gamma$ through \ref{eq:f_exp}.
In summary, the non-convective fluxes $\bm{\Phi}_d$ and $\bm{\Phi}_\Gamma$  are not explicitly related to $\text Q_p$, regardless of particle nature or volume fraction. 
Instead, their influence on $\text Q_p$ is mediated through the closure terms in equation \ref{eq:avg_hybrid_q}, which may depend on the higher moments $\textbf{Q}^{(1)}_p$ \ldots $\textbf{Q}^{(n)}_p$ or other higher-order particle-related moments.%\footnote{This is typically the configuration observed in dilute flows of axisymmetric fibers within the Stokes regime. In this regime, the force acting on the fiber (which represents the exchange term in the momentum equation) is dependent on the orientation tensor, which in turn is directly linked to the second-order moment of the mass distribution.}. 
These moments, in turn, explicitly depend on $\bm{\Phi}_d$ and $\bm{\Phi}_\Gamma$ as indicated by \ref{eq:avg_hybrid_q_1}. 
%Consequently, it must be understood that the kinetic-like equations (\ref{eq:avg_dt_dq_alpha_tot}) is formally exact and apply for any type of particle and particle volume fraction, as long as the closure terms are well modeled and that the Taylor expansion used in these expressions reaches a convergence on the scale of the particles as discussed below.
%The reader is invited to interpret this expression for the specific case of the momentum conservation law. 






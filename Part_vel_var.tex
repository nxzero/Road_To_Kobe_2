\documentclass[12pt]{My_preprint}
\title{
    Theoretical calculation of the droplet velocity varience in mono disperse buoyant emulsions for low inertia and dilute regime.
    }

\author[1,2]{Nicolas Fintzi}
% \author[1]{Jean-Lou Pierson}
% \author[2]{Stephane Popinet}
\affil[1]{IFP Energies Nouvelles, Rond-point de l’echangeur de Solaize, 69360 Solaize}
\affil[2]{Sorbonne Universit\'e, Institut Jean le Rond d'Alembert, 4 place Jussieu, 75252 PARIS CEDEX 05, France}
\normalmarginpar


\begin{document}

\maketitle

\begin{abstract}
    In this study we derive an analytical expression for the particle phase velocity variance tensor. 
    This derivation is restricted to the dilute regime where the dispersed phase volume fraction labeled $\phi$ is vanishingly small and for particle Reynolds number $Re \ll 1$. 
    In this context we derive an analytical formula for the particle velocity variance generated by translating bubbles or droplets inside a Newtonian fluid under the effect of gravity. 
    To by-pass the integral convergence difficulties generated due to the $\mathcal{O}(r^{-1})$ decay of the disturbance velocity field we make use of the Nearest-Particle-Statistical (NPS) as it is introduced in \citet{zhang2023evolution}. 
\end{abstract}

\section{Faxen theorem for non-isolated particles}


Since computing the forces and velocities of a particle often require Faxen laws it is important to re-write these laws. 


Let us first re-write the Stokes single partcile conditionally averaged eq

\begin{align}
    P_1 \div \textbf{u}^{1d} = 0 \\
    % \pddt (P_1 \textbf{u}^{1d})
    % + P_1 \div (
    %  \textbf{u}^{1d} \textbf{u}^{1d}  
    % + \textbf{u} \textbf{u}^{1d} 
    % + \textbf{u}^{1d} \textbf{u} 
    - P_1\{
        \grad p_f^{1d}
        + \div [\mu_f (\grad \textbf{v}^1 + (\grad \textbf{v}^1)^\dagger)]
    \}
    = 
    + \div  \bm\sigma^1_\text{eff}
    + \kappa \pavg { (\delta_1 - P_1) \intO{\bm\sigma_f^0\cdot \textbf{n}}}
\end{align}
with $\bm\Sigma^{1d} \rho_f  = -p_f^{1d} \bm\delta + \mu_f [\grad \textbf{u}^{1d}+(\grad \textbf{u}^{1d})^\dagger]$ the mean Newtonian stress contribution and, 
\begin{multline*}
    P_1\bm\sigma^1_\text{eq}
    =
    + P_1 [[\phi_d^1 p_f^1 - \phi_d p_f]]\bm\delta 
    + \pavg{(\delta_1 - P_1) \intS{\left[
        \textbf{r}\bm{\sigma}_f^0 \cdot \textbf{n}
        -  2 \mu_f (\textbf{u}_f^0 \textbf{n} + \textbf{n} \textbf{u}_f^0)
        \right] 
    }},
\end{multline*}



In a simpler form assuming homogeneous medimum,
\begin{align}
     \div \textbf{v}^{1} = 0 \\
    % \pddt (P_1 \textbf{u}^{1d})
    % + P_1 \div (
    %  \textbf{u}^{1d} \textbf{u}^{1d}  
    % + \textbf{u} \textbf{u}^{1d} 
    % + \textbf{u}^{1d} \textbf{u} 
    \div( \Sigma + \bm\Sigma^1_\text{eff})
    = 
    - \kappa (n_p\textbf{f})^{1d}
\end{align}

Applying faxen than
\begin{equation*}
    \intS{\hat{\textbf{U}}\cdot  \bm\sigma_f^{(1)} \cdot \textbf{n}}
    % + \intS{(\hat{\textbf{u}}_f - \hat{\textbf{U}})\cdot  \bm\sigma_f^{(1)} \cdot \textbf{n}}
    = 
    \intS{\textbf{u}_f^{(1)}\cdot  \hat{\bm\sigma}_f \cdot \textbf{n}}
    - \intOf{\hat{\textbf{u}}_f\cdot + (\bm\Sigma^1_\text{eff} - \kappa (n_p\textbf{f})^{1d})}
    % \intOf{\textbf{u}_f^{(0)}\cdot ( \div \hat{\bm\sigma}_f)}
\end{equation*}

\section{The original problem with classic pair stats}

we should have a factor of $1/N$ or something in front of the int 


\section{Strategy}

As in the previous chapter we reformulate teh ensemble average of the partcile phase velocity variance, 
\begin{equation}
    \pavg{\textbf{u}_\alpha'\textbf{u}_\alpha'}
\end{equation}
in terms of the nearest particle statistics conditioned quantities. 
We use (2.15) of \citet{zhang2021ensemble}, namely, 
\begin{equation}
    \int_{\mathbb{R}^3}
    \sum_{j\neq i}
    \delta(\textbf{x}_j[\FF,t] - \textbf{y}) h_{ij}[t,\FF] d\textbf{y}
    = 1
\end{equation}
And introduce the relation, 
\begin{equation}
    \pavg{\textbf{u}_\alpha'\textbf{u}_\alpha'}
    = 
    n_p[\textbf{x},t]
    \int_{\mathbb{R}^3}
    (\textbf{v}^\text{nst}_p
    \textbf{v}^\text{nst}_p)[\textbf{x},\textbf{y},t]
    P_\text{nst}[\textbf{y}|\textbf{x},t]
    d\textbf{y}
    + 
    \int_{\mathbb{R}^3}
    \pavg{
        \sum_{j\neq i }
        h_{ij} 
        (\textbf{u}_\alpha'' - \textbf{u}^\text{nst}_p)
        (\textbf{u}_\alpha'' - \textbf{u}^\text{nst}_p)
    }d \textbf{y}
\end{equation}
where we recall that $P_{nst}$ is the probability of finding the nearest neighbor at the position of \textbf{r} knowing that the particle is present at \textbf{x} at time $t$. 
$\textbf{v}^\text{nst}_p = \textbf{u}_p^\text{nst} - \textbf{u}_p$ where $\textbf{u}_p^\text{nst}$ si the mean particle velocities with nearest neighbor at $\textbf{r}$. 
And $\textbf{u}_\alpha'' = \textbf{u}_\alpha - \textbf{u}^\text{nst}_p$ is the particle velocity fluctuation around the conditional mean $\textbf{u}^\text{nst}_p$. 


\section{How to derive of the conditional particle velocity}

Bring the problem to the two particle conditional field. 
This is done by making a balence between the velocity and the buoyancy. 


\section{derivation of the conditional average with the method of reflexion}


According to \citet{kim2013microhydrodynamics,zhang2021ensemble} teh particle phase velcoity fluctuation can be obtained directly with the faxen laws.
Following the notation of \citet{kim2013microhydrodynamics} we note $\textbf{v}_1$ and $\textbf{v}_2$ the velocity fields generated by the particle at \textbf{x} and at \textbf{y} respectively. 
The external body forces inducing the motion of the particles are noted $\textbf{b}$ and $\textbf{b}_2$. 
For instance, we keep them arbitrary however note that in the DNS $\textbf{b}_1 = \textbf{b}_2 = m\textbf{g} = \textbf{b}$. 

The substantial difference between this analysis and \citet{kim2013microhydrodynamics} analysis is that $\textbf{v}_2$ is the nearest particle averaged.
Meaning that, $\textbf{v}_2$ is given by, 
\begin{equation}
    \textbf{v}_2 = 
    \textbf{b} \cdot \left[
        1
        + \frac{\lambda}{2(3\lambda +2)}\grad^2
    \right]\frac{\mathcal{G}(\textbf{z},\textbf{y})}{8a\pi\mu_f}
    - \phi \frac{\textbf{b}}{4 \pi \mu_f a} |\textbf{z} - \textbf{y}|^2 
\end{equation}
where all the distance have been made dimenisonless by $a$. 
and, 
\begin{align}
    \textbf{b} = 2 \pi \mu_f a \left(\frac{2+3\lambda}{1+\lambda}\right) \textbf{U}_1^{(0)}\\
    \textbf{b} = 2 \pi \mu_f a \left(\frac{2+3\lambda}{1+\lambda}\right) \textbf{U}_2^{(0)}
\end{align}
with $\textbf{k}_2 = \textbf{b}_2/U$. 
We recall that, 
\begin{equation}
    \mathcal{G}(\textbf{z},\textbf{y})
    = \frac{\bm\delta}{r}+\frac{\textbf{rr}}{r^3}
\end{equation}
with $\textbf{r} = |\textbf{z} - \textbf{x}|$. 
Notice and it will be useful for the following analysis that the gradient of $\mathcal{G}$ and $r$ over its first variable reads as, 
\begin{align}
    \partial_k \mathcal{G}_{ij}
    = 
    \frac{( - \delta_{ij}r_k+ \delta_{ki}r_j +\delta_{kj}r_i)}{r^3}
    - 3\frac{r_ir_jr_k}{r^5}\\
    \partial_{kl} \mathcal{G}_{ij}
    = 
    \frac{( - \delta_{ij}\delta_{kl}+ \delta_{ki}\delta_{jl} +\delta_{kj}\delta_{il})}{r^3}
    - 3\frac{( - \delta_{ij}r_kr_l+ \delta_{ki}r_jr_l +\delta_{kj}r_ir_l)}{r^5}\\
    - 3\frac{(\delta_{il}r_jr_k + r_ir_k \delta_{jl} + r_ir_j\delta_{kl})}{r^5}
    + 15\frac{r_ir_jr_kr_l}{r^7}
    \\
    \grad^2 \mathcal{G}_{ij}
    = 
    \frac{2 \bm\delta}{r^3}
    - 6\frac{\textbf{rr}}{r^5}\\
    \partial_k \grad^2 \mathcal{G}_{ij}
    = 
    - 6\frac{(\delta_{ij} r_k + \delta_{ki} r_j+ r_i \delta_{jk})}{r^5}
    + 30\frac{r_ir_jr_k}{r^7}
    \\
    \grad^4 \mathcal{G}_{ij}
    = 
   0
    \\
    % - 3\frac{(r_jr_i + r_ir_j + r_ir_j3)}{r^5}
    % + 15\frac{r_ir_j}{r^5}
    % \\
    \partial_k r^2 = 2 r_k\\
    \partial_{kl} r^2 = 2 \delta_{kl}
\end{align}
Regarding, $\textbf{v}_1$ it can be considered at the lowest order as the one of an isolated particle. 
\begin{equation}
    \textbf{v}_1 = 
    \textbf{b} \cdot \left[
        1
        + \frac{\lambda}{2(3\lambda +2)}\grad^2
    \right]\frac{\mathcal{G}(\textbf{z},\textbf{x})}{8\pi\mu_f}. 
    % + \phi A |\textbf{x} - \textbf{y}|^2 \textbf{k}_2
\end{equation}

At the first order in reflextion the velocity of the dorplet at \textbf{x} knowing its nearest neighbor is at \textbf{y}, can be computed according to faxen law as, 
\begin{align}
    2 \pi \mu_f a \textbf{U}_1^{(1)}
    &= \left(1 + \frac{\lambda}{2(3\lambda +2)}\grad^2\right)\textbf{v}_2|_{\textbf{z} = \textbf{x}}
\end{align}
where $\textbf{U}^{(1)}$ is the first reflexion. 
From the expression of $\mathbf{G}$ and its derivative evaluated at $\textbf{x}$ we may write, 
\begin{align}
    2 \pi \mu_f a \textbf{U}_1^{(1)}
    &= 
    \frac{\textbf{b}}{4}\cdot\left(1+ \frac{\lambda}{2(3\lambda +2)}\grad^2\right) \left\{\left(1 + \frac{\lambda}{2(3\lambda +2)}\grad^2\right)\mathcal{G}|_{\textbf{z} = \textbf{x}}  - 2 \phi r^2 \right\} \\
    &= 
    \frac{\textbf{b}}{4}\cdot\left(1+ \frac{\lambda}{3\lambda +2}\grad^2+ \frac{\lambda^2}{4(3\lambda +2)^2}\grad^4\right)\mathcal{G}|_{\textbf{z} = \textbf{x}}
    - \phi \textbf{b} (\frac{r^2}{2} +1 )\\
    &=\frac{\textbf{b}}{4}\cdot\left\{ \frac{\bm\delta}{r}+ \frac{\textbf{rr}}{r^3}+ \frac{2\lambda}{3\lambda +2} \left(
        \frac{\bm\delta}{r^3} - 3 \frac{\textbf{rr}}{r^5}
    \right)
    - \phi \bm\delta (\frac{r^2}{2} +1 )
    \right\}
\end{align}

In other word, 
\begin{align}
    \textbf{U}_1^{(1)}
    &=\left(\frac{2+3\lambda}{\lambda +1}\right)\frac{    \textbf{U}_1^{(0)}    }{4}\cdot\left\{ \frac{\bm\delta}{r}+ \frac{\textbf{rr}}{r^3}+ \frac{2\lambda}{3\lambda +2} \left(
        \frac{\bm\delta}{r^3} - 3 \frac{\textbf{rr}}{r^5}
    \right)
    - \phi \bm\delta (\frac{r^2}{2} +1 )
    \right\}
\end{align}




The particle phase pdf is given in dimensionless form by, 
\begin{equation*}
    P_\text{nst}[\textbf{y}|\textbf{x}]
    =
    n_p e^{- \phi (r^3 - 8)}
\end{equation*}

Integrating the first int we have, 
\begin{equation*}
    \pavg{\textbf{u}_\alpha'\textbf{u}_\alpha'}
    = 
    n_p[\textbf{x},t]
    \int_{\mathbb{R}^3}
    (\textbf{v}^\text{nst}_p
    \textbf{v}^\text{nst}_p)[\textbf{x},\textbf{y},t]
    P_\text{nst}[\textbf{y}|\textbf{x},t]
    d\textbf{y}
\end{equation*}
which gives, 
\begin{equation*}
    \pavg{\textbf{u}_\alpha'\textbf{u}_\alpha'} / n_p 
    = 
\end{equation*}

\bibliography{Bib/bib_bulles.bib}
\appendix

\end{document}


\section{3D averaged equations}
\label{sec:averaged equations}
The distributional form of the local scale mass and momentum equations reads as, 
\tb{this is over complicated because only a single dispersed phase is nessesary then condition}
Thus, the system bulk-bubbles-particles might be described by 6 equaitons, 
\begin{align}
    \pddt (\chi_k \rho_k)
    + \div (\chi_k \rho_k \textbf{u}_k^0)
    &= 
    0\\
    \div \textbf{u}^0
    &= 
    0
\end{align}
auto complie
\begin{equation}
    \pddt (\chi_k \rho_k \textbf{u}^0_k)
    + \div (
        \chi_k \rho_k \textbf{u}^0_k \textbf{u}_k^0
        )
    = 
    \chi_k \rho_k \textbf{g}
    + \div (\chi_k \bm\sigma_k^0  + \delta_{\Gamma k} \bm\sigma_{\Gamma k}^0 )
    + \delta_{\Gamma k}  \bm\sigma_f^0 \cdot \textbf{n}_k
\end{equation}
\begin{equation}
    \rho_f (\pddt + \textbf{u}^0 \cdot \grad)\textbf{u}^0
    = 
    \rho_f \textbf{g}
    + \div \bm\sigma^0_*
    +\kappa_k  \delta_{\Gamma k}  \bm\sigma_f^0 \cdot \textbf{n}_k 
    % +\kappa_p  \delta_{\Gamma p}  \bm\sigma_f^0 \cdot \textbf{n}_p 
\end{equation}
with the mixture stress defined as, 
\begin{equation}
    \bm\sigma^0_*
    =
    \chi_f \bm\sigma_f^0  
    +\zeta_k^{-1} (\chi_k \bm\sigma_k^0 + \delta_{\Gamma k} \bm\sigma_{\Gamma k}^0)  
\end{equation}
where $\zeta_k = \rho_k/\rho_f$ and  $ \kappa_k  = \zeta^{-1}_k - 1 $, and we assume Einstein summation on the index $k$.


Now we apply an ensemble average procedure to this equaiton which gives for the bulk phase: 
\begin{align*}
    \div \textbf{u} &= 0\\
     (\pddt + \textbf{u} \cdot \grad) n_k &= \div (n_k \textbf{u}_r)\\
    n_k  (\pddt + \textbf{u}_k \cdot \grad) \textbf{u}_k &= 
    % n_k\textbf{u}_r \cdot \grad \textbf{u}_k
    - \div \pavg{\textbf{u}_k'\textbf{u}_k'}+n_k \textbf{g}
    +\textbf{M}_k^{(0)}/m_k\\
    \rho_f (\pddt + \textbf{u} \cdot \grad)\textbf{u}
    % + \div \avg{\rho_f \textbf{u}' \textbf{u}'}
    &= 
    (1 + \kappa_k \phi_k)\div\bm\Sigma
    + \rho_f \textbf{g}
    + \div \bm\sigma_*
    +\kappa_k  \avg{\delta_{\Gamma k} \bm\sigma_f' \cdot \textbf{n}_k} 
\end{align*}
The \textit{Mean newtonian stress} and the mean effective stress is now defined as, 
\begin{align}
    \bm\sigma_* &= 
    - \rho_f \avg{\textbf{u}'\textbf{u}'}
    +\zeta_k^{-1} \avg{\chi_k \bm\sigma_k' + \delta_{\Gamma k} \bm\sigma_\Gamma^0} 
    - \avg{2\mu_f \chi_k \textbf{e}_k^*}
\end{align}
note that the last relation might also be written as, 
\begin{align}
    \rho_f (\pddt + \textbf{u} \cdot \grad)\textbf{u}
    % + \div \avg{\rho_f \textbf{u}' \textbf{u}'}
    &= 
    \div\bm\Sigma
    + \rho_f \textbf{g}
    + \div \bm\sigma_*
    +\kappa_k  \textbf{M}_k^{(0)}\\
    \bm\sigma_* &= 
    - \rho_f \avg{\textbf{u}'\textbf{u}'}
    + \textbf{M}_k^{(1)}
    -\div \textbf{M}_k^{(2)}
\end{align}
where we might introduce,
\begin{align}
    \textbf{M}_\alpha^{(0)} &=
    \intS{\bm{\sigma}_f' \cdot \textbf{n}}
   +\intO{\div \bm\Sigma}
   \\
   \textbf{M}_\alpha^{(1)} &=
   \intS{\textbf{r}\bm{\sigma}_f' \cdot \textbf{n}}
   -2\mu_f \intO{\textbf{e}_d'}
   +\intO{\textbf{r}\div \bm\Sigma}
   \\
   \textbf{M}_\alpha^{(2)} &=
   \frac{1}{2}\intS{\textbf{rr}\bm{\sigma}_f' \cdot \textbf{n}}
   -2\mu_f \intO{\textbf{re}_d''}
   +\intO{\textbf{rr}(\div \bm\Sigma+ \rho_f\textbf{g})}
    \\
\end{align}

\section{Averaged equaiton 1D}
\subsection{definitions}
Now we define sections averaged equatitetes such that,
\begin{align*}
    S X^S &= \int_{S(u(\textbf{x}),v(\textbf{x}))} X(\textbf{x},t)dudv\\
    S X^{S'} &=S  X - \int_{S(u(\textbf{x}),v(\textbf{x}))} X(\textbf{x},t)dudv
\end{align*}
we note particularily the property,
\begin{equation}
    \int_S \grad (\ldots) dS 
    = 
    \int_S \textbf{n} (\textbf{n}\cdot \grad) (\ldots) dS 
    + \int_S (\bm\delta -\textbf{n} \textbf{n})\cdot \grad (\ldots) dS 
\end{equation}

Because $\grad = \textbf{e}_x \partial_x+ \textbf{e}_y \partial_y+\textbf{e}_z \partial_z$ we have $\textbf{n}\cdot \grad = \partial_x$ Assuming $\textbf{n}=\textbf{e}_x$.
Hence,
\begin{equation}
    \int_S \grad (\ldots) dS 
    = 
    \textbf{n} (\textbf{n}\cdot \grad)  (\ldots)^S
    + \int_C \textbf{N} (\ldots) dC
\end{equation}
\begin{equation}
    \int_S \grad \grad (\ldots) dS 
    = 
    \textbf{nn} (\textbf{n}\cdot \grad)^2  (\ldots)^S
    +\textbf{n} (\textbf{n}\cdot \grad) \int_C \textbf{N} (\ldots) dC
    + \int_C \textbf{N} \grad (\ldots) dC
\end{equation}
from which  we deduce that 
\begin{equation}
    \int_S \div \textbf{u} dS 
    = 
    (\textbf{n}\cdot \grad)  (\textbf{n} \cdot \textbf{u})^S
    + \int_C \textbf{N} \cdot \textbf{u}dC
    =
    (\textbf{n}\cdot \grad)  (\textbf{n} \cdot \textbf{u})^S
\end{equation}
\begin{equation}
    \int_S \grad^2 \textbf{u} dS 
    = 
    \partial_x^2  \textbf{u}^S
    +(\textbf{n}\cdot \grad) \int_C \textbf{n} \cdot \textbf{N} \textbf{u} dC
    + \int_C \textbf{N}\cdot  \grad \textbf{u} dC
    =
    \partial_x^2  \textbf{u}^S
    + \int_C \textbf{N}\cdot  \grad \textbf{u} dC
\end{equation}
\subsection{equations}
Based on these definitions and assuming that $\textbf{u}^S =\textbf{n}( \textbf{u}^S \cdot \textbf{n})$ and $\textbf{u}\cdot \textbf{N} = 0$ we have,0
\begin{align*}
    \pddx u^S &= 0\\
    (\pddt +  u^S \pddx) n_k^S 
    + \pddx (u_k n_k')^S
    &= 
    \pddx (u_r^S n_k^S)
    \\
    n_k^S (\pddt 
    + u_k^S \pddx)  \textbf{u}_k
    +\pddt (n_k' \textbf{u}_k)^S 
    + \pddx (n_k' u_k \textbf{u}_k )^S
    &= 
    n_k^S  \textbf{g}
    - \pddx (\pavg{ u_k'\textbf{u}_k'})^S
    +(\textbf{M}_k^{(0)})^S/m_k\\
    \rho_f (\pddt + u^S \cdot \grad)\textbf{u}^S
    + \pddx ( u \textbf{u}')^S
    &= 
    S \rho_f \textbf{g}
    - \textbf{n} \pddx p_f^S
    + (\textbf{N}\cdot  \bm\Sigma)^C 
    + \pddx (\textbf{n} \cdot \bm\sigma_*)^S
    +\kappa_k (\textbf{M}_k^{(0)})^S 
\end{align*}
where we noticed that $(\textbf{N}\cdot \bm\sigma^*)^C =0$ .


\section{0D model}
Definition,
\begin{equation}
    X^V = \int_V X(\textbf{x},t) dV
\end{equation}
such that it gives,
\begin{align}
    \int_V \grad (\ldots) dV
    = \int_S \textbf{N} (\ldots) dS\\
    \int_V \grad (\grad \ldots) dV
    = \int_S \textbf{N} (\grad \ldots) dS\\
\end{align}

which directly gives,
\begin{align*}
    (\textbf{N}\cdot \textbf{u})^S &= 0\\
    \pddt n_k^V &=  (\textbf{N}\cdot \textbf{u}_kn_k )^S\\
    \pddt  (n_k  \textbf{u}_k)^V + (\textbf{N}\cdot \textbf{u}_k \textbf{u}_k n_k)^S &= 
    - (\textbf{N}\cdot \pavg{\textbf{u}_k'\textbf{u}_k'})^S+ n_k^V \textbf{g}
    +(\textbf{M}_k^{(0)})^V/m_k\\
    \rho_f \pddt \textbf{u}^V + \rho_f (\textbf{N}\cdot \textbf{uu})^S
    % + \div \avg{\rho_f \textbf{u}' \textbf{u}'}
    &= 
    (\textbf{N}\cdot\bm\Sigma)^S
    + \rho_f V \textbf{g}
    + (\textbf{N}\cdot \bm\sigma_*)^S
    +\kappa_k  (\textbf{M}^{(0)})^V
\end{align*}


\section{Closure for non inertial particles }

Let us consider a develloped flow in an axisymmetric pipe of radius $R$.
In polar coord we note,
\begin{equation}
    \textbf{u}(r)=\textbf{e}_x u_x(r) + \textbf{e}_r u_r(r)+ \textbf{e}_\theta u_\theta(r)
\end{equation}
The mass conservation in polar coord then gives,
\begin{equation}
    \grad \cdot \textbf{u} 
    = \textbf{e}_x \pddx u_x  
    + \frac{1}{r}\partial_r (r u_r) 
    + \frac{1}{r}\partial_\theta u_\theta
    = 
    \frac{1}{r}\partial_r (r u_r) 
    = 0
\end{equation}
By direct integration one obtains, 
\begin{equation}
    u_r = 0,
\end{equation}
so 
\begin{equation}
    \textbf{u}=\textbf{e}_x u_x(r)
\end{equation}
Hence a rotating flow with $u_\theta$ could be possible but we will neglect this for instance. 
Also one can note that,
\begin{align}
    \grad \textbf{u}
    =
    \textbf{e}_r \partial_r \textbf{u}
    =\textbf{e}_r\textbf{e}_x \partial_r u_x\\
    \grad^2 \textbf{u}
    =
    \frac{1}{r^2}\partial_r(r \partial_r \textbf{u}) 
    = 
    \textbf{e}_x \frac{1}{r^2}\partial_r(r \partial_r u_x) 
\end{align}

In general dilute stokes flows we have, 
\begin{align}
    \textbf{M}^{(0)} 
    &=
    \phi
    \frac{\mu_f}{a^2}
    \frac{3(2+3\lambda)}{2(1+\lambda)}\textbf{u}_r
    + \phi\mu_f  \frac{3\lambda}{4(\lambda +1)} \grad^2 \textbf{u}
    + \phi \div\bm\Sigma
    \\
    \textbf{M}^{(1)} 
    &= \mu_f \phi 
    \frac{(5\lambda +2)}{4(\lambda +1)}(\grad \textbf{u}+ ^\dagger\grad \textbf{u}) 
    \\
    \textbf{M}^{(2)} 
    &=
    - \mu_f \phi \frac{3\lambda}{4(\lambda +1)}[\bm\delta \textbf{u}_r + \frac{1}{2\lambda}\textbf{u}_r \bm\delta ]
\end{align}
For inertial less particles one may write,
\begin{align*}
    -
    \frac{\mu_f}{a^2}
    \frac{3(2+3\lambda)}{2(1+\lambda)}\phi\textbf{u}_r
    =
    \phi(\rho_p \textbf{g} + \div\bm\Sigma)
    + \phi\mu_f  \frac{3\lambda}{4(\lambda +1)} \grad^2 \textbf{u}
    =
    \phi\rho\textbf{g} 
    + \phi\mu_f  \frac{3\lambda}{4(\lambda +1)} \grad^2 \textbf{u}
\end{align*}
where $\rho = \rho_p -\rho_f$. note that $\rho = \rho_f (\rho_d /\rho_f -1) = -\rho_d (\rho_f /\rho_d -1) = -\rho_d\kappa$. 
\begin{align}
    \rho_f (\pddt + \textbf{u} \cdot \grad)\textbf{u}
    % + \div \avg{\rho_f \textbf{u}' \textbf{u}'}
    &= 
    \div\bm\Sigma
    + (\rho_f - \kappa \phi \rho_p) \textbf{g}
    + \div \bm\sigma_*
    \\
    \bm\sigma_* &= 
    - \rho_f \avg{\textbf{u}'\textbf{u}'}
    +  \mu_f \phi 
    \frac{(5\lambda +2)}{4(\lambda +1)}(\grad \textbf{u}+ ^\dagger\grad \textbf{u}) 
    + \mu_f \frac{3\lambda}{4(\lambda +1)}[\grad(\phi \textbf{u}_r) + \frac{1}{2\lambda}\div(\phi\textbf{u}_r) \bm\delta ]
\end{align}


\subsection{Homogeneous case}
if $\phi = cste$, then we have, 
\begin{equation}
    -
    \frac{\mu_f}{a^2}
    \frac{3(2+3\lambda)}{2(1+\lambda)}\div (\phi\textbf{u}_r)
    =
    - \phi \grad^2 p_f 
    = O(\phi^2)
\end{equation}
Hence, the final form of the equation becomes,
\begin{align}
    \rho_f (\pddt + \textbf{u} \cdot \grad)\textbf{u}
    % + \div \avg{\rho_f \textbf{u}' \textbf{u}'}
    &= 
    - \grad p_f 
    +\mu_{ein}\grad^2 \textbf{u}
    + (\rho_f - \phi \rho) \textbf{g}
\end{align}
only the pressure gradient will remain, so we end up will Navier Stokes equations with a Reynols stress closure. 
If noting that $\textbf{u}\cdot \grad \textbf{u}= \textbf{e}_x \cdot \textbf{e}_r \textbf{e}_x ... = 0 $ we end up with the stokes equations, namely
\begin{equation}
    0
    = 
    - \textbf{e}_x \partial_x p_f 
    - \textbf{e}_r \partial_r p_r  
    + \mu_{ein} \textbf{e}_x \frac{1}{r}\partial_r (r\partial_r u_x )
    + g (\rho_f - \phi \rho) \textbf{e}_x
\end{equation}
So that, $p_f = p_f(x) $ with $\grad^2 p_f =\partial_x^2 p_f = 0$ hence $p_f = ax+b$.
Now we can solve for the velocity witch gives, 
\begin{align*}
    u(r) = r^2 \partial_x p_f^* /4 + C ln(r) + C\\
    u(r=R) = 0  = R^2 \partial_x p_f^* /4 + C ln(R) + C\\
    \partial_r u(r=0) = r \partial_x p_f^* /2 + C /r = 0\\
\end{align*}
So that 
\begin{align*}
    \textbf{u} = \textbf{e}_x \frac{(\partial_x p_f + g(\rho_f - \phi \rho))}{4\mu_{ein}} (r^2 - R^2)\\
    \grad \textbf{u} = \textbf{e}_r \textbf{e}_x \frac{(\partial_x p_f + g(\rho_f - \phi \rho))}{2\mu_{ein}} r
\end{align*}
also,
\begin{equation}
    \textbf{u}^S = - \textbf{e}_x \pi\frac{(\partial_x p_f^S + g(\rho_f - \phi^S \rho))}{8\mu_{ein}} R^4\\
\end{equation}
Because we assumed $p_f^S = p_f$ one may write, 
\begin{align*}
    \textbf{u} = \frac{2 \textbf{u}^S}{\pi R^4} (R^2 - r^2)\\
    \grad \textbf{u} = - \textbf{e}_r \frac{4 \textbf{u}^S}{\pi R^4} r
\end{align*}

The stress of such a flow on the side is written,
\begin{equation}
    \textbf{e}_r \cdot \bm\Sigma = - p_f \textbf{e}_r - \mu_f  \frac{4 \textbf{u}^S}{\pi R^4} r
\end{equation}
If we integrate all that over a section surface we obtain
\begin{align}
    (\textbf{e}_r \cdot \bm\Sigma)^C
    =
    -2 \pi R p_f \textbf{e}_r - \mu_f  \frac{8 \textbf{u}^S}{\pi R^2}
\end{align}
Because,


A usefull quantity is the dyadic,
\begin{equation}
    (\textbf{uu})^S 
    = 
    \frac{4 \textbf{u}^S\textbf{u}^S}{3 \pi R^2}
    =  
    \frac{\textbf{u}^S\textbf{u}^S}{\pi R^2}(1+ 1/3)
    \\
\end{equation}

mass momentum and of the bulk is solved now what about the kinetic of flotation ? 

One can also deduce the bubbles velocity, namely,
\begin{align*}
    \textbf{u}_p
    =
    \textbf{u}
    +
    \phi\rho\textbf{g} 
    + \phi\mu_f  \frac{3\lambda}{4(\lambda +1)} \grad^2 \textbf{u}
\end{align*}
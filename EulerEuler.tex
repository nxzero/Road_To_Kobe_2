\documentclass[12pt]{My_preprint}
\title{A single-phase-like pressure based solver for the modeling of averaged dispersed two-phase flows.  }

\author[1,2]{Nicolas Fintzi}
\normalmarginpar


\begin{document}

\maketitle

\begin{abstract}
    Instead of using the classic \textit{two-fluid} formulation introduced in the first chapters we present in this work a better and simpler formulation for two-phase flow modeling. 
\end{abstract}


\section{Mass et momentum equation at the local scale. }

At the local scale the mass and momentum of continuous phase, denoted by $_f$ is described by the set of equations, 
\begin{align*}
    \pddt \chi_f + \div \textbf{u}_f^0 \chi_f  = 0\\
    \pddt  \textbf{u}_f^0 \chi_f   
    + \div \textbf{u}_f^0 \textbf{u}_f^0 \chi_f  = 
    \frac{1}{\rho_f}[\div \chi_f \bm\sigma_f^0 
    - \delta_\Gamma \bm\sigma_f \cdot \textbf{n}]
    + \textbf{g}\chi_f 
\end{align*}
Similarly the dispersed phase might be written, 
\begin{align*}
    \pddt \chi_d + \div \textbf{u}_d^0 \chi_d  = 0\\
    \pddt  \textbf{u}_d^0 \chi_d   
    + \div \textbf{u}_d^0 \textbf{u}_d^0 \chi_d   = 
    \frac{1}{\zeta \rho_f} [\div \chi_d \bm\sigma_d^0 
    + \delta_\Gamma \bm\sigma_d \cdot \textbf{n}]
    + \textbf{g}\chi_d 
\end{align*}
These equations are completed by the boundary, 
\begin{equation}
    \frac{1}{\zeta \rho_f}
    \Jump{\delta_\Gamma \bm\sigma_k} 
    = 
    \frac{1}{\zeta \rho_f}
    \div\delta_\Gamma \bm\sigma_\Gamma^0
\end{equation}
Summing these equation one obtain the \textit{single-fluid} formulation of the acceleration conservation equaiton, namely, 
\begin{align*}
    \pddt \chi_f + \div \textbf{u}_f^0 \chi_f  = 0\\
    \pddt  \textbf{u}^0   
    + \div \textbf{u}^0 \textbf{u}^0  = 
    \div[\frac{1}{\rho_f}\chi_f \bm\sigma_f^0 
    +\frac{1}{\rho_d} (\chi_d \bm\sigma_d^0  + \delta_\Gamma \bm\sigma_\Gamma^0)]
    + \frac{1}{\rho_f}\left(\frac{1}{\zeta} -1\right)
    \delta_\Gamma \bm\sigma_f \cdot \textbf{n}
    + \textbf{g}
\end{align*}




Averaged this 
\begin{equation}
    \div\textbf{u}=0.
\end{equation}
\begin{equation}
    \pddt \textbf{u}  
    + \div (
    \textbf{u}\textbf{u}
    )
    = 
    \div \bm{\sigma}^\text{eq} + 
    \textbf{g} 
    + \left(\frac{1-\zeta}{\rho_f\zeta}\right) \avg{\delta_I \bm{\sigma}_f^0 \cdot \textbf{n}} 
\end{equation}
\begin{equation}
    \bm\sigma^\text{eq} = 
    - \avg{ \textbf{u}'\textbf{u}'}
    + \frac{\phi_f}{\rho_f}\bm\sigma_f
    + \frac{\phi_d}{\rho_d}\bm\sigma_d
    + \frac{\phi_\Gamma}{\rho_d} \bm\sigma_\Gamma. 
\end{equation}

The stress, 
\begin{align}
    \frac{\phi_f}{\rho_f}\bm\sigma_f
    = 
    - \frac{1}{\rho_f} p_f \bm\delta 
    + \frac{\mu_f}{\rho_f}(\grad \textbf{u} + \grad \textbf{u})
    - \frac{1}{\rho_f} (2\mu_f \textbf{e}_d - p_f )\phi_d\\
    \frac{\phi_d}{\rho_d}\bm\sigma_d
    + \frac{\phi_\Gamma}{\rho_d} \bm\sigma_\Gamma \approx
    \pOavg{ 
    \textbf{w}_d^0\textbf{w}_d^0  
    }
    -\frac{1}{2}\frac{d^2 \textbf{V}_p}{dt^2}
    +\frac{1}{\rho_d}\pSavg{ \
        \textbf{r}\bm{\sigma}_1^0 \cdot \textbf{n}
    }
\end{align}
And, 
\begin{equation*}
    \left(\frac{1-\zeta}{\rho_f\zeta}\right) \avg{\delta_I \bm{\sigma}_f^0 \cdot \textbf{n}} 
    = 
    \left(\frac{1-\zeta}{\rho_f\zeta}\right) \pSavg{\bm{\sigma}_f^0 \cdot \textbf{n}} 
    -\div \left(\frac{1-\zeta}{\rho_f\zeta}\right) \pSavg{\textbf{r}\bm{\sigma}_f^0 \cdot \textbf{n}} 
\end{equation*}

This results in, 
Averaged this 
\begin{equation}
    \div\textbf{u}=0.
\end{equation}
\begin{equation}
    \pddt \textbf{u}  
    + \div (
    \textbf{u}\textbf{u}
    )
    = 
    \frac{1}{\rho_f}\div \bm{\Sigma} 
    + \div \bm{\sigma}^\text{eq} 
    + \left(\frac{1-\zeta}{\rho_f\zeta}\right) \pSavg{ \bm{\sigma}_f^0 \cdot \textbf{n}} 
    + \textbf{g} 
\end{equation}
\begin{equation}
    \bm\sigma^\text{eq} = 
    - \avg{ \textbf{u}'\textbf{u}'}
    +  \pOavg{ 
        \textbf{w}_d^0\textbf{w}_d^0  
        }
        -\frac{1}{2}\frac{d^2 \textbf{V}_p}{dt^2}
        +\frac{1}{\rho_f}\pSavg{ 
            \textbf{r}\bm{\sigma}_1^0 \cdot \textbf{n}
            - \mu_f(\textbf{u}_f^0\textbf{n}
            + \textbf{n}\textbf{u}_f^0)
            + p_f
        }
\end{equation}
The last term represnt the stresslet due to the disturbance field

In terms
\begin{align*}
    \avg{ \textbf{u}'\textbf{u}'}
    = 
    \avg{\chi_f \textbf{u}^0_f\textbf{u}^0_f}
    + \avg{\chi_d \textbf{u}^0_d\textbf{u}^0_d}
    - \textbf{uu}
\end{align*}
with 
\begin{align*}
    \avg{\chi_d \textbf{u}^0_d\textbf{u}^0_d}
    &= 
    \pOavg{\textbf{u}^0_d\textbf{u}^0_d}
    -\div \pOavg{\textbf{r}\textbf{u}^0_d\textbf{u}^0_d}\\
    &= 
    \pavg{\textbf{u}_\alpha\textbf{u}_\alpha v_\alpha}
    % \pOavg{\textbf{u}_\alpha\textbf{w}^0_d}
    % + \pOavg{\textbf{w}_d^0\textbf{u}_\alpha}
    + \pOavg{\textbf{w}_d^0\textbf{w}^0_d} \ldots
    % -\div \pOavg{\textbf{r}\textbf{u}^0_d\textbf{ u}^0_d}\\
\end{align*}
And, 
\begin{align*}
    \textbf{uu} 
    &= 
    [\textbf{u}_f\phi_f + \phi_d \textbf{u}_d ] 
    [\textbf{u}_f + \phi_d( \textbf{u}_d - \textbf{u}_f)] \\
    &= 
    \phi_f \textbf{u}_f \textbf{u}_f
    + \phi_d [\textbf{u}_d \textbf{u}_f
    + \phi_f  \textbf{u}_f ( \textbf{u}_d - \textbf{u}_f)
    +  \textbf{u}_d \phi_d( \textbf{u}_d - \textbf{u}_f) ]\\
    &= 
    \phi_f \textbf{u}_f \textbf{u}_f
    + \phi_d [\textbf{u}_d \textbf{u}_f
    + \textbf{u} ( \textbf{u}_d - \textbf{u}_f)]\\
    % \textbf{u}_f \textbf{u}_f\phi_f 
    % + \phi_d [\textbf{u}_f \textbf{u}_d  +
    %  ( \textbf{u}_d - \textbf{u}_f)\textbf{u}_f ]
    %  + \phi_d^2 ( \textbf{u}_d - \textbf{u}_f)( \textbf{u}_d - \textbf{u}_f)
\end{align*}
\section{Stokes dilute regime regime }

The particle phase eq say, 
\begin{equation*}
    0 
    = 
    \rho_d \phi \textbf{g}
    + \pSavg{{\bm{\sigma}_f^0 \cdot \textbf{n}_d}},
\end{equation*}
And from the closure in stokes regime we have, 
\begin{equation*}
    \pSavg{\bm{\sigma}_f^0\cdot \textbf{n}_d} = 
    \phi_d \div\bm\Sigma
    + \frac{3\phi_d\mu_f}{2 a^2} 
    \left(\frac{3\lambda+2}{\lambda+1}\right) \textbf{u}_{f p} 
    + \frac{3\phi_d\mu_f}{4} \left(\frac{\lambda}{\lambda+1}\right)\grad^2\textbf{u}
    = - 
    \rho_d \phi \textbf{g}
\end{equation*}
We deuce that at all times, 
\begin{align*}
    \frac{6\phi_d\mu_f}{d^2} 
    \left(\frac{3\lambda+2}{\lambda+1}\right) \textbf{u}_{f p} 
    = 
    \phi_d \grad p_f
    - \rho_d \phi \textbf{g}
    - \phi_d \mu_f \left[
        1 + \frac{3}{4} \left(\frac{\lambda}{\lambda+1}\right)
    \right]\grad^2 \textbf{u}
\end{align*}
which simplify to (with $\left(\frac{3\lambda+2}{\lambda+1}\right) = L$), 
\begin{align*}
   \textbf{u}_{f p} 
    = 
    \frac{d^2}{6\mu_f L }
    \left\{
        \grad p_f
        - \rho_f\zeta   \textbf{g}
        -  \mu_f \left[
            1 + \frac{3}{4} \left(\frac{\lambda}{\lambda+1}\right)
            \right]\grad^2 \textbf{u}
    \right\}
\end{align*}


We deduce that the source term in the momentum eq, 
\begin{equation*}
    \textbf{g} 
    + \left(\frac{1-\zeta}{\rho_f\zeta}\right)
    \pSavg{{\bm{\sigma}_f^0 \cdot \textbf{n}_d}}
    =
    \textbf{g}
    +  
    \phi (- 1 + \zeta)
    \textbf{g}
    = (1 + \phi(\zeta-1) )\textbf{g}
\end{equation*}


The first moment term is, 
\begin{equation*}
    \pSavg{\textbf{r}\bm{\sigma}_f^0 \cdot \textbf{n}} -
    2\pSavg{\mu\textbf{e}_d^0} 
    = 
    - \phi_d p_f\bm\delta
    + \frac{5\lambda +2}{\lambda +1}
    \textbf{E} \phi \mu_f
\end{equation*}

\bibliography{Bib/bib_bulles.bib}
\appendix

\end{document}




\subsection{Problem statement}
Objective of this section :
\begin{itemize}
    \item Introduce the dimensionless parameters.
    \item Present the physical parameters of some industrial processes to locate our problematic. 
    \item Introduce the dimensionless parameters range investigated in this study.
\end{itemize}
We investigate the dynamic of homogeneous mono-disperse emulsion subject to buoyancy forces. 
Both the dispersed (resp. continuous) phase is considered as Newtonian fluid defined by viscosity $\mu_d$ (resp. $\mu_c$), and density $\rho_d$ ($\mu_c$).
Throughout this work, the indices $d$ and $c$ indicate properties belonging to the dispersed and continuous phase, respectively. 
The interface between both fluid is considered as infinitely thin and deprived of any impurities so that it can only be described with the surface tension coefficient $\sigma$. 
In this work the density, viscosity of each phase, and surface tension coefficient, will be considered constant during  the time of a numerical experiment.
In dimensionless form the physics of the flow is described by $4$ dimensionless parameters: 
The viscosity and density ratio, $\mu_r = \mu_d / \mu_c$ and $\rho_r = \rho_d / \rho_c$, respectively. 
The \textit{Galileo} number, 
\begin{equation*}
    Ga =\sqrt{\rho_c(\rho_c - \rho_d) g d^3} / \mu_c
\end{equation*}
where $a$ is the equivalent radius of the droplets.
And the \textit{Bond} number, 
\begin{equation*}
    Bo =\frac{(\rho_c - \rho_d) g d^2}{\sigma}
\end{equation*}
with $g$ the gravity constant. 
The \textit{Galileo} number measure the influence of the buoyancy forces against the viscous forces.
Whereas the \textit{Bond} number evaluate the ratio between buoyancy and capillary forces. 
In addition to these $4$ parameters we introduce the number of particles, $N_b$, and the dispersed phase volume fraction $\phi_d$ which fully describe the topology of a finite domain of the flow. 


\begin{table}[h!]
    \centering
    \caption{Dimensionless parameter range investigated in this work.}
    \begin{tabular}{ccccccc}\hline
        $Ga$&$Bo$&$\phi$&$\mu_r$&$\rho_r$&$N_b$&$t^*_{end}$\\ \hline\hline
        $5\rightarrow 100$&$1$&$1\% \rightarrow 20\%$&$0.1 \& 1$&$1.111$&$125$&$500$\\ \hline
    \end{tabular}
    \label{tab:parameters}
\end{table}
We wish to investigate the moderate inertial emulsion regime with quasi spherical droplets. \todo{gives real parameters values compared to experiment}
Thus, the \textit{Bond} number must be low enough to obtain nearly spherical drops, and the viscosity and density ratio must approach the oil/water situation. 
It will be shown in the next few sections that a $Bo =1$ gives reasonable results. 
Additionally, for a statistical convergence reason explained in \ref{sec:preliminary} we choose $N_b = 125$. 
Therefore, in the following we will keep the dimensionless parameters within the ranges depicted in \ref{tab:parameters}.
In summary, we investigated $6$ \textit{Galileo} number $Ga = 5,10,25,50,75,100$, four volume fractions $\phi = 0.01,0.05,0.1,0.15,0.2$, and two viscosity ratios $\mu_r =0.1,1$. 
This makes a total of $60$ representative simulations of $125$ droplets which is expensive. 
Therefore, in the next sections we develop on our numerical strategy to make efficient multi VOF simulations. 





\subsection{Theoretical framework}
Objectives
\begin{itemize}
    \item Introduce the closure problem and the form of the closure terms. 
    \item Focus only on the drag force and velocity fluctuations. 
\end{itemize}
To clarify the role of the different forces and stresses in a multiphase flow model we start by listing the averaged momentum equations for dispersed two-phase flows obtained by an ensemble averaging procedure.
For mono disperse fluid particles they read as \citep{zhang1997momentum,jackson1997locally}\todo{citer le papier with Daniel},
\begin{align}
    \pddt (\phi_c\rho_c \textbf{u}_c)
    + \nablabh \cdot \left(\phi_c\rho_c \textbf{u}_c\textbf{u}_c + \phi_c  \bm{\sigma}_c^{\text{Re}} - n_p\textbf{M}_p \right)
    &= \phi_c 
    \left(\nablabh \cdot \bm{\sigma}_c
    + \rho_c \textbf{g}\right)
    - n_p \textbf{f}_p 
    \label{eq:dt_uc}
    \\
    \pddt (\phi_d\rho_d \textbf{u}_p)
    + \nablabh \cdot \left(\phi_d\rho_d \textbf{u}_p\textbf{u}_p+ \phi_d \bm{\sigma}_p^{\text{Re}}\right)
    &= 
    \phi_d \left(\nablabh \cdot \bm{\sigma}_c
    + \rho_d \textbf{g}\right)
    + n_p \textbf{f}_p 
    \label{eq:dy_up}
\end{align}
where the subscript $c$ and $p$ denote continuous and particle phase averaged quantities, respectively.
$\phi$ is the volume fraction, $n_p$ the particle number density, \textbf{u} the averaged velocity, $\bm{\sigma}$ the averaged stress tensor, $\bm{\sigma}^{\text{Re}}$ the stress due to velocity fluctuation or pseudo turbulent stress.
Additionally, due to the presence of the particles we have the interphase drag force term : $\textbf{f}_p$, and the interphase stress or first moment : $\textbf{M}_p$. 


In this study we focus on three of these closure terms, namely the interphase drag force : $\textbf{f}_p$, and the Reynolds stresses : $\bm{\sigma}^{\text{Re}}_c$ and $\bm{\sigma}^{\text{Re}}_p$. 
Using the statistical average formalism on all configuration of the flow termed by $\CC$, we can define these quantities as,  
\begin{align}
    \textbf{f}_p (\textbf{x},t) &= \
    \frac{1}{n_p}\int \sum_i \delta(\textbf{x} - \textbf{x}_\alpha(t,\CC)) \textbf{f}_\alpha d\PP\\
    \textbf{f}_\alpha(t,\CC) &= \int_{S_\alpha} \bm{\sigma}_c'(\textbf{x},t,\CC) \cdot \textbf{n} dS \\
    \bm{\sigma}^{\text{Re}}_p(\textbf{x},t) &=\frac{1}{n_p} \int \sum_i \delta(\textbf{x} - \textbf{x}_\alpha(t,\CC)) \rho_d \textbf{u}_\alpha'\textbf{u}_\alpha' d\PP\\
    \label{eq:def_uuc}
    \bm{\sigma}^{\text{Re}}_c(\textbf{x},t) &= \frac{1}{\phi_c}\int \rho_c\chi_c \textbf{u}_c' \textbf{u}_c'd\PP
\end{align}
where $d\PP$ is a probability measure. 
$\textbf{x}_\alpha$ is the particles center of mass potion and $\delta$ the Dirac delta function. 
The superscript $'$ indicate the relative values of a quantity with respect to its phasic mean values, specifically $\bm{\sigma}_c' = \bm{\sigma}_c^0(\textbf{x},t, \CC)  - \bm{\sigma}_c(\textbf{x},t)$, $\textbf{u}_\alpha' = \textbf{u}_\alpha(t,\CC) - \textbf{u}_p(\textbf{x},t)$ and $\textbf{u}_c' = \textbf{u}_c^0(\textbf{x},t, \CC)  -\textbf{u}_c(\textbf{x},t)$, with $\bm{\sigma}_c^0 $, $\textbf{u}_c^0$ and $\textbf{u}_\alpha$ being the local stress of the fluid phase, the local velocity of the fluid phase and the velocity of the center of mass of the particle $\alpha$, respectively. 








\subsection{Simulations setup}
Objective of this section :
\begin{itemize}
    \item Introduce the dimensionless parameters.
    \item Present the physical parameters of some industrial processes to locate our problematic. 
    \item Introduce the dimensionless parameters range investigated in this study.
    \item Present the tri-periodic box within which we add droplets in vof 
\end{itemize}
We investigate the dynamic of homogeneous mono-disperse emulsion subject to buoyancy forces. 
Both the dispersed (resp. continuous) phase is considered as Newtonian fluid defined by viscosity $\mu_d$ (resp. $\mu_c$), and density $\rho_d$ ($\mu_c$).
Throughout this work, the indices $d$ and $c$ indicate properties belonging to the dispersed and continuous phase, respectively. 
The interface between both fluid is considered as infinitely thin and deprived of any impurities so that it can only be described with the surface tension coefficient $\sigma$. 
In this work the density, viscosity of each phase, and surface tension coefficient, will be considered constant during  the time of a numerical experiment.
In dimensionless form the physics of the flow is described by $4$ dimensionless parameters: 
The viscosity and density ratio, $\mu_r = \mu_d / \mu_c$ and $\rho_r = \rho_d / \rho_c$, respectively. 
The \textit{Galileo} number, 
\begin{equation*}
    Ga =\sqrt{\rho_c(\rho_c - \rho_d) g d^3} / \mu_c
\end{equation*}
where $a$ is the equivalent radius of the droplets.
And the \textit{Bond} number, 
\begin{equation*}
    Bo =\frac{(\rho_c - \rho_d) g d^2}{\sigma}
\end{equation*}
with $g$ the gravity constant. 
The \textit{Galileo} number measure the influence of the buoyancy forces against the viscous forces.
Whereas the \textit{Bond} number evaluate the ratio between buoyancy and capillary forces. 
In addition to these $4$ parameters we introduce the number of particles, $N_b$, and the dispersed phase volume fraction $\phi_d$ which fully describe the topology of a finite domain of the flow. 


\begin{table}[h!]
    \centering
    \caption{Dimensionless parameter range investigated in this work.}
    \begin{tabular}{ccccccc}\hline
        $Ga$&$Bo$&$\phi$&$\mu_r$&$\rho_r$&$N_b$&$t^*_{end}$\\ \hline\hline
        $5\rightarrow 100$&$1$&$1\% \rightarrow 20\%$&$0.1 \& 1$&$1.111$&$125$&$500$\\ \hline
    \end{tabular}
    \label{tab:parameters}
\end{table}
\JL{il faut choisir entre $\mu_r$ et $\lambda$... Pr ailleurs je pense qu'l y a une coquille dans ta definition de $\mu_r$.}

We wish to investigate the moderate inertial emulsion regime with quasi spherical droplets. \todo{gives real parameters values compared to experiment ? Yes. tu peux le faire facilement en prenant par exmeple des gouttes allant de 500 microns a 2 mm. Cela ajoutera un peu de poids au choix des parametres. Par ailleurs se pose la question de lancer quelques cas à plus bas nombre de Bond (car experimentalement ils le seront) voir si cela change les resultats. De meme, pq ne pas lancer quelques cas ou les gouttes sont moins visqueuses que le fluide environnant ?}
Thus, the \textit{Bond} number must be low enough to obtain nearly spherical drops, and the viscosity and density ratio must approach the oil/water situation. 
It will be shown in the next few sections that a $Bo =1$ gives reasonable results. 
Additionally, for a statistical convergence reason explained in \ref{sec:preliminary} we choose $N_b = 125$. 
Therefore, in the following we will keep the dimensionless parameters within the ranges depicted in \ref{tab:parameters}.
In summary, we investigated $6$ \textit{Galileo} number $Ga = 5,10,25,50,75,100$, four volume fractions $\phi = 0.01,0.05,0.1,0.15,0.2$, and two viscosity ratios $\mu_r =0.1,1$. 
This makes a total of $60$ representative simulations of $125$ droplets which is expensive. 
Therefore, in the next sections we develop on our numerical strategy to make efficient multi VOF simulations. 

\tb{It is clear that at $\phi>0.1$ the mono disperse simulation aren't physicaly realistic since coalesence should arise at those volume fraction.}

To mimic infinitely large homogeneous emulsions we consider a tri periodic cubic domain of length $L$, within which, both phases are subject to the incompressible Navier Stokes equations with the corresponding boundaries conditions. To prevent an uniform acceleration of the fluids phase one need to impose a pressure gradient (or equivalently a body force) to the momentum equation. This body force may be straightforwadly obtained from the dispersed phase momentum equations. In particular in the present configurations the momentum balance simplifies to
\begin{align}
0 &= -\epsilon\frac{\partial p}{\partial z} - \rho_c \epsilon g - n_p \textbf{f}_p \label{eq:uf_triperio}
    \\
0 &= -\phi\frac{\partial p}{\partial z}     - \rho_d \phi g + n_p \textbf{f}_p \label{eq:up_triperio}
\end{align}

Adding those two equations together we obtain
\begin{equation}
-\frac{\partial p}{\partial z} = \rho_c (1-\phi) g + \rho_d \phi g 
\end{equation}










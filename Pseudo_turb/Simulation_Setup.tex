


%\subsection{Simulation set-up}
\subsection{The Basilisk flow solver}
Objective : 
\begin{itemize}
    %\item Present the tri-periodic box within which we add droplets in vof 
    \item Present : Governing equations under single fluid formulation 
    \item Present in details the numerical scheme. 
    \item Finally, present why the numerical coalescence is a problem
\end{itemize}


The one fluid formulation of the mass and momentum conservation equation reads as,
\begin{align}
    \label{eq:dt_rho}
    \pddt \rho
    + \nablabh\cdot(\rho\textbf{u})
    &= 0, \\
    \pddt (\rho \textbf{u})
    +\nablabh \cdot (\rho  \textbf{u} \textbf{u} - \bm{\sigma})
    &= 
    (\avg{\rho} - \rho)\textbf{g}
    + \textbf{f}_\sigma\delta(\textbf{x} -\textbf{x}_I)
    \label{eq:dt_urho}
\end{align}
respectively.  
\todo{mettre sous forme adim \citep{hidman2023assessing} ? pas besoin}
In \ref{eq:dt_urho} we introduced : the velocity of the fluid $\textbf{u}$,  the Newtonian stress  tensor $\bm{\sigma} = -p \textbf{I} + 2\mu \textbf{S}$ with $p$ the pressure fields and $\textbf{S}$ the symmetrical part of the velocity gradient.
The mean averaged density is defined as $\avg{\rho} = \rho_d\phi + (1-\phi) \rho_f$. 
Then, $\avg{\rho} \textbf{g}$ act as an additional body forces in \ref{eq:dt_urho}  to enforce a net-zero momentum source over the computational domain \citep{bunner2002dynamics} \JL{a presenter apres dans problem statement}. 
The last term of \ref{eq:dt_urho} account for surface tension force  times the Dirac delta function $\delta(\textbf{x}-\textbf{x}_I)$ since it acts at the interface position $\textbf{x}_I$. 
Additionally, we solve a transport equation for the approximation of the phase indicator function, i.e. the color function $\alpha_d$, it reads,
\begin{equation}
    \pddt \alpha_d + \textbf{u} \cdot \nablab \alpha_d =0.
    \label{eq:dt_alpha}
\end{equation}
The equations are discretized with a centered scheme on a Staggered multigrid solver from \url{http://basilisk.fr}. 
The two-phase flow solver use the volume of fluid method. 
The reconstruction of the interfaces is computed using Piecewise Linear Interface Calculation. 
For a more detailed description of the solver we refer the reader to \citet{popinet2018numerical}. \todo{Make a more precise description such as in \cite{hidman2023assessing} ? Oui et je ne suis pas sur que la reference de Stephane soit la bonne. Tu peux aussi t'inspirer des papiers de Luc Deike and Co}

In the next section we will see that this equation will be sightly be modified to avoid coalesce.

It is known that with the VOF method, we experience premature coalesce events between droplets \citet[Appendix B]{innocenti2020direct}.
Which can be problematic to gather datas,\citet{loisy2017buoyancy}
However, if we wish to reach a quasi steady state regime we must be able to conserve a specific population of droplets within time, in our case a mono-disperse configuration. 
To tackle this issue we used a specific algorithm which prevent coalescence. 



\textbf{Objetcives : }

\begin{enumerate}
    \item Compute closure terms for buoyant emulsion
    \item Needs Statistically steady numerical experience to be representative 
    \item Therefore we introduce the no-calesence algorithm to obtain be statistically steady
    \item Allowing us to carry out massive DNS to compute closures terms. 
\end{enumerate}
\vspace*{1cm}
\textbf{notes}
\begin{itemize}
    \item biblio Lhuillier
    \item In the perspective of macroscopic modeling we have to close some terms 
    \begin{itemize}
        \item oil water processes
        \item vapour water process mu r = 6 
    \end{itemize}
    \item In the literature there is absolutely nothing for emulsion 
    \item Thus we study $Bo = 1$ $\mu_r = 0.1$ and $\rho_r = 1.11$. 
    \item Within the framework of hybrid model averaged equations in a mono-disperse case 
    \item Present theritical : 
     \begin{itemize}
        \item Momentum averaged equation for fluid particle two-fluid formulation. 
        \item Closure terms. (Merabahdi good intro)
    \end{itemize}
\end{itemize}
\todo[inline]{Introduce the dimensionless parameter in introduction to point out the gap for unity density and viscosity ratio. 
Do the same as \citet{hidman2023assessing}}

Even in the Stokes flow regime and in the limit of very dilute flows the calculation of the the sedimentation velocity of spherical inclusion is still a matter of research.  Batchelor in this seminal work showed by assuming that the particle where randomly distributed that the velocity of the particle was $U/U_0=1-\phi$, by making use oa re normalization technique to avoid the effect of non-convergin integral. This results has been extended to drops by wachollder and ... finding ... Over the past decades many experimental works have tried to find he relative motion between the particle and
the fluid phases result in microstructures affecting the sedimentation
velocity. En fiat cichocky en utilisant n-body formulation montre visiblement qu'il ya une vrai micro-strucutre. This question was raise first by Saffman whco by using the formalism of distribution showed the dramatic difference between the thrrre different configurations..."
Fait par wachholder etc As a result most of the study make use of the empirical correlation given by Richardson Zaki

The configuration is also influenced by particle inertia. Citer Yin et Koch ... plus les papiers sur les ecoulements à bulles (Bunner et Tryggvason, Loisy)

Eventuellement citer quelques papiers en regime potentiel ? Discuter plus en details de la microstrucutre a la Jacques ?

Pour une sphere solide regime de Stokes on a n (R-Z) a ecrire. Pour des bulles ? experience a bas Re pour des bulles spheriques et non contaminees ?



%has a long history even in the Stokes regime. 




\section{Introduction}

In the context of Euler-Euler models, this study focuses on the momentum exchange term in the averaged Navier-Stokes equations, or the drag force density term. 
The objective of this work is to quantify the drag force in a ``homogeneous'' and steady-state configuration.
This means that we consider only a constant and uniform relative motion between the dispersed and continuous phase and a constant mean volume fraction of droplets. 

In this configuration, a drag force correlation should predict the mean drag force density as a function of the local \textit{Reynolds} number based on the particle size, denoted as $Re$, and the volume fraction of the dispersed phase, $\phi$. Additionally, since we are considering droplets with a constant density ratio ($\zeta$) but arbitrary viscosity, the model must account for the viscosity ratio, $\lambda$. 
To the authors' knowledge, no existing model in the literature comprehensively describes the dependence of drag force density in terms of $Re$, $\phi$, and $\lambda$. 
Therefore, we aim to address this gap by proposing a new model incorporating these dependencies.
For isolated spherical droplets, \citet{magnaudet1997forces} proposed a semi-empirical formula based on the well-known models of \citet{schiller1933} for solid particles and \citet{mei1994} for spherical bubbles, which accurately predict the drag force as a function of both the \textit{Reynolds} number and the viscosity ratio.
Meanwhile, \citet{richardson1954} introduced robust empirical relations to predict the sedimentation velocity of solid particles in terms of volume fraction $\phi$ at low \textit{Reynolds} numbers. 
The Richardson-Zaki laws can be written as $u_p/u_0 = (1-\phi)^n$, where $u_p$ is the dispersed phase velocity and $u_0$ is the terminal velocity of an isolated particle. 
The exponent $n$ is an empirical constant known as the Richardson-Zaki exponent. 
\citet{ishii1979drag} extended the values of the Richardson-Zaki exponent for spherical bubbles in the Stokes regime. 
More recently, \citet{kramer2019improvement} proposed an accurate relation to better predict the exponent $n$ in the Richardson-Zaki relation for solid particles, for arbitrary \textit{Reynolds} numbers. 

In this work, we used the findings of these authors and combined their models to construct a robust model applicable at intermediate values of $\lambda$.
Since \citet{magnaudet1997forces} introduced a model valid for intermediate $\lambda$ at $\phi = 0$, and \citet{ishii1979drag} proposed corrections to the Richardson-Zaki exponent for droplets at $Re \approx 0$, the real challenge here is to accurately model the $\phi$ dependency of the drag force for finite Reynolds numbers, at intermediate $\lambda$.

The organization of this study is as follows:
In \ref{sec:methodology_drag} we present the theoretical and numerical methodology adopted to express the drag force in terms of the buoyancy force or the relative velocity of the dispersed and continuous phase within the averaged Navier-Stokes equations framework.
In \ref{sec:model_drag} we outline the theoretical development that leads to the creation of our new drag force model. 
Finally in \ref{sec:validation_drag} we demonstrate the validity of our model by comparing its predictions, for the drag force and sedimentation velocity, with the DNS results at intermediates values of $\lambda$.
\subsection{The closure problem}
Objectives
\begin{itemize}
    \item Introduce the closure problem and the form of the closure terms. 
    \item Focus only on the drag force and velocity fluctuations. 
\end{itemize}
To clarify the role of the different forces and stresses in a multiphase flow model we start by listing the averaged momentum equations for dispersed two-phase flows obtained by an ensemble averaging procedure.
For mono disperse fluid particles they read as \citep{zhang1997momentum,jackson1997locally}
\begin{align}
    \pddt (\phi_f\rho_f \textbf{u}_f)
    + \div \left(\phi_f\rho_f \textbf{u}_f\textbf{u}_f + \phi_f  \bm{\sigma}_f^{\text{Re}} - n_p\textbf{M}_p \right)
    &= \phi_f 
    \left(\div \bm{\sigma}_f
    + \rho_f \textbf{g}\right)
    - n_p \textbf{f}_p 
    \label{eq:dt_uc}
    \\
    \pddt (\phi_d\rho_d \textbf{u}_p)
    + \div \left(\phi_d\rho_d \textbf{u}_p\textbf{u}_p+ \phi_d \bm{\sigma}_p^{\text{Re}}\right)
    &= 
    \phi_d \left(\div \bm{\sigma}_f
    + \rho_d \textbf{g}\right)
    + n_p \textbf{f}_p 
    \label{eq:dy_up}
\end{align}
where the subscript $c$ and $p$ denote continuous and particle phase averaged quantities, respectively.
$\phi$ is the volume fraction, $n_p$ the particle number density, \textbf{u} the averaged velocity, $\bm{\sigma}$ the averaged stress tensor, $\bm{\sigma}^{\text{Re}}$ the stress due to velocity fluctuation or pseudo turbulent stress.
Additionally, due to the presence of the particles we have the interphase drag force term : $\textbf{f}_p$, and the interphase stress or first moment : $\textbf{M}_p$. 


In this study we focus on three of these closure terms, namely the interphase drag force : $\textbf{f}_p$, and the Reynolds stresses : $\bm{\sigma}^{\text{Re}}_f$ and $\bm{\sigma}^{\text{Re}}_p$. 
Using the statistical average formalism on all configuration of the flow termed by $\CC$, we can define these quantities as,  
\begin{align}
    \textbf{f}_p (\textbf{x},t) &= \
    \frac{1}{n_p}\int \sum_i \delta(\textbf{x} - \textbf{x}_\alpha(t,\CC)) \textbf{f}_\alpha d\PP\\
    \label{eq:f_alpha}
    \text{with} \;\;\; &\textbf{f}_\alpha(t,\CC) = \int_{S_\alpha} \bm{\sigma}_f'(\textbf{x},t,\CC) \cdot \textbf{n} dS \\
    \bm{\sigma}^{\text{Re}}_p(\textbf{x},t) &=\frac{1}{n_p} \int \sum_i \delta(\textbf{x} - \textbf{x}_\alpha(t,\CC)) \rho_d \textbf{u}_\alpha'\textbf{u}_\alpha' d\PP\\
    \label{eq:def_uuc}
    \bm{\sigma}^{\text{Re}}_f(\textbf{x},t) &= \frac{1}{\phi_f}\int \rho_f\chi_f \textbf{u}_f' \textbf{u}_f'd\PP
\end{align}
where $d\PP$ is a probability measure. 
$\textbf{x}_\alpha$ is the center of mass of a particle labeled $\alpha$, $\chi_f(\textbf{x},t\CC)$ is teh phase indicator function of the continuous phase, and $\delta$ the Dirac delta function. 
The superscript $'$ indicate the relative values of a quantity with respect to its phasic mean values, specifically $\bm{\sigma}_f' = \bm{\sigma}_f^0(\textbf{x},t, \CC)  - \bm{\sigma}_f(\textbf{x},t)$, $\textbf{u}_\alpha' = \textbf{u}_\alpha(t,\CC) - \textbf{u}_p(\textbf{x},t)$ and $\textbf{u}_f' = \textbf{u}_f^0(\textbf{x},t, \CC)  -\textbf{u}_f(\textbf{x},t)$, with $\bm{\sigma}_f^0 $, $\textbf{u}_f^0$ and $\textbf{u}_\alpha$ being the local stress of the fluid phase, the local velocity of the fluid phase and the velocity of the center of mass of the particle $\alpha$, respectively. 






\section{Computational methodology}

We have considered the same set of data and configuration than in our previous study. 
except that we extended our data to $3$ viscosity ratios, namely $\lambda =10,1,0.1$.
Thus, our rage of study becomes, 
\begin{table}[h!]
    \centering
    \caption{Dimensionless parameter range investigated in this work.}
    \begin{tabular}{|ccccccc|ccc|}
        \hline
        \multicolumn{7}{|c}{Primary parameters} & \multicolumn{3}{||c|}{Secondary parameters}\\ \hline
        \multicolumn{1}{|c|}{$Ga$}                               & \multicolumn{1}{c|}{$Bo$}                   & \multicolumn{1}{c|}{$\phi$} & \multicolumn{1}{c|}{$\lambda$}                    & \multicolumn{1}{c|}{$\zeta$}                & \multicolumn{1}{c|}{$N_b$} & $t^*_\text{end}$ & \multicolumn{1}{||c|}{$\mathcal{L}/d$} & \multicolumn{1}{c|}{$Re$}  & $We$   \\ \hline
        \multicolumn{1}{|c|}{\multirow{4}{*}{$5\rightarrow 80$}} & \multicolumn{1}{c|}{\multirow{4}{*}{$0.5$}} & \multicolumn{1}{c|}{$1\%$}  & \multicolumn{1}{c|}{\multirow{4}{*}{$10$ \& $1$\&$0.1$}} & \multicolumn{1}{c|}{\multirow{4}{*}{$0.9$}} & \multicolumn{1}{c|}{$160$} & $400$           & \multicolumn{1}{||c|}{$20$}            & \multicolumn{1}{c|}{$1.3\to 110$} & {$0.03\to 0.95$} \\ 
        \multicolumn{1}{|c|}{}                                   & \multicolumn{1}{c|}{}                       & \multicolumn{1}{c|}{$5\%$}  & \multicolumn{1}{c|}{}                             & \multicolumn{1}{c|}{}                       & \multicolumn{1}{c|}{$800$} & $400$           & \multicolumn{1}{||c|}{$20$}            & \multicolumn{1}{c|}{$1.0\to 92$} &  {$0.02\to 0.67$}\\ 
        \multicolumn{1}{|c|}{}                                   & \multicolumn{1}{c|}{}                       & \multicolumn{1}{c|}{$10\%$} & \multicolumn{1}{c|}{}                             & \multicolumn{1}{c|}{}                       & \multicolumn{1}{c|}{$200$} & $1000$           & \multicolumn{1}{||c|}{$10$}            & \multicolumn{1}{c|}{$1.9\to 77$}&  {$0.01\to 0.47$}\\ 
        \multicolumn{1}{|c|}{}                                   & \multicolumn{1}{c|}{}                       & \multicolumn{1}{c|}{$20\%$} & \multicolumn{1}{c|}{}                             & \multicolumn{1}{c|}{}                       & \multicolumn{1}{c|}{$400$} & $1000$           & \multicolumn{1}{||c|}{$10$}            & \multicolumn{1}{c|}{$1.7\to 62$}&  {$9\cdot 10^{-3}\to 0.31$}\\ \hline
        \end{tabular}
\end{table}

\subsection{Statistics computations}

Following \citet{du2022analysis} we consider ergodicity at all time of the numerical experiment.
Thus, the ensemble average of a quantity $X$ can be approximated by a spatial average $\Xavg{X}$ and a time average $\Tavg{X}$ such that $\int X d\PP = \avg{X} \approx \Xavg{\Tavg{X}} = \Tavg{\Xavg{X}}$.
Consequently, the ensemble average of a numerical field, $X$, is taken through space and time such that,
\begin{equation}
    \avg{X}
    = \Tavg{\Xavg{X}}
    = \frac{1}{ t_{end} - t_0}\int_{t_0}^{t_{end}} 
    \Xavg{X}(t) dt
\end{equation}
where, 
\begin{equation}
    \Xavg{X}(t)
    = \frac{1}{L^3}\int 
    X(\textbf{x},t) d\textbf{x}
\end{equation}
where $L$ is the length of the cubic domain.
$t_0$ and $t_{end}$ is the starting time of sampling and the time duration of the simulation, respectively.
In practice, we take $t_0$ such that the simulation reach a statistically steady regime for $t>t_0$.  
Both $t_{end} $ and $t_0$ are given in \ref{ap:A} after several validations studies. 

Therefore, to compute the phase average of the local phase velocity $\textbf{u}_k^0$, we simply perform an integration over space and time, 
\begin{equation*}
    \textbf{u}_k = \frac{1}{\phi_k} \Tavg{\Xavg{\chi_k \textbf{u}_k^0}}
\end{equation*}
To compute continuous phase averaged quantities such as \ref{eq:def_uuc} it is a little more complicated,
\begin{equation}
    \phi_f \bm{\sigma}^{\text{Re}}_f /\rho_f
    % = \Tavg{\Xavg{\chi_f \textbf{u}_f' \textbf{u}_f'}}
    = \Tavg{\Xavg{\chi_f (\textbf{u}_f^0 -\textbf{u}_f ) (\textbf{u}_f^0 -\textbf{u}_f)}}
    = \Tavg{\Xavg{\chi_f \textbf{u}_f^0 \textbf{u}_f^0}}
    -  \phi_f  \textbf{u}_f \textbf{u}_f.
    \label{eq:def_uuc_num} 
\end{equation}
where the indicator funciton $\chi_f$ must be understood as its approximation in the DNS, i.e the color function $1 - \alpha_d$. 
Consequently, \ref{eq:def_uuc_num} indicate that we must take the average of the product of the velocities, and then we retrieve the mean velocities' product. 
% Additionally,  note that the Reynolds stress can be decomposed by such as : 
% \begin{align*}
%     \phi_f \bm{\sigma}^{\text{Re}}_f /\rho_f
%     &= 
%     \Tavg{\Xavg{\chi_f (\textbf{u}_f^0 -\Xavg{\chi_f\textbf{u}_f^0} / \Xavg{\chi_f} ) (\textbf{u}_f^0 -\Xavg{\chi_f\textbf{u}_f^0} / \Xavg{\chi_f} )}}\\
%     &+ \Tavg{\Xavg{\chi_f} (\Xavg{\chi_f\textbf{u}_f^0} / \Xavg{\chi_f} - \textbf{u}_f ) (\Xavg{\chi_f\textbf{u}_f^0} / \Xavg{\chi_f} - \textbf{u}_f)}\\
% \end{align*}
% where the first term is the space fluctuation relative to the instantaneous mean velocity of the fluid $\Xavg{\chi_f\textbf{u}_f^0} / \Xavg{\chi_f}$, and the second is the time fluctuation of the instantaneous mean velocity of the fluid. 

Similar expression can be derived for the particular phase by integrating the property over the volume of the particle which is done throught the use of the \texttt{tag.h} function as explaine in the preceding chapter, then we average on each particle at all time.

Regarding the averaged surface quantities such as the mean drag force, we demonstrate now that it is possible to compute it directly through the difference of relative velocities. 
\section{Methodology}

\subsection{The closure problem}
To clarify the role of the different forces and stresses in a multiphase flow model we start by listing the averaged momentum equations for dispersed two-phase flows obtained by an ensemble averaging procedure.
For mono-disperse emulsion the momentum equations of the continuous phase and dispersed phase can be written \citep{zhang1997momentum,jackson1997locally}
\begin{align}
    \pddt (\phi_f\rho_f \textbf{u}_f)
    + \div \left(\phi_f\rho_f \textbf{u}_f\textbf{u}_f + \bm{\sigma}_f^{\text{eff}}\right)
    &= \phi_f 
    \left(\div \bm{\sigma}_f
    + \rho_f \textbf{g}\right)
    - n_p \textbf{f}_p, 
    \label{eq:dt_uf}
    \\
    \pddt (n_m  m_p  \textbf{u}_p)
    + \div \left(n_p m_p  \textbf{u}_p\textbf{u}_p
    +  \bm{\sigma}_p^{\text{eff}}\right)
    &= 
    n_p v_p \left(\div \bm{\sigma}_f
    + \rho_d \textbf{g}\right)
    + n_p \textbf{f}_p, 
    \label{eq:dt_up}
\end{align}
respectively. 
The subscript $f$ and $p$ refer to continuous phase and particle phase averaged quantities, respectively.
$\phi_f$ is the volume fraction of the continuous phase, $n_p$ the particle number density, $\textbf{u}_f$ (resp. $\textbf{u}_p$) the averaged velocity of the fluid (resp. dispersed) phase, $\bm{\sigma}_f$ the averaged continuous phase stress tensor.
$\bm{\sigma}^{\text{eff}}_p$ and $\bm{\sigma}^{\text{eff}}_f$ are the effective stresses of the dispersed and continuous phase, respectively.  
Finally, $\textbf{f}_p$ represents the interphase momentum exchange, or drag force density term. 

Both momentum equations are completed by a transport equation for $\phi_f$ and a volume conservation laws, namely, 
\begin{align}
    % \phi_d \approx  \\
    \phi_f + n_pv_p  - \frac{v_\alpha d^2 }{20}\grad n_p \approx 1\\
    \pddt \phi_f + \div (\textbf{u}_f \phi_f)= 1
\end{align}
where we have introduced, $d$, as the diameter of the droplets. 
We recall that the first equation is only an approximation, that has been derived using the relation between volume fraction and number density\citep{zhang1997momentum}. 

The number density, continuous phase volume fraction, drag force density, effective stresses of the dispersed, and continuous phase can be expressed as ensemble average of local-non averaged quantities and read as, 
\begin{align}
    n_p &= \pavg{}
    \label{eq:n_p}\\
    \phi_f &= \avg{\chi_f}
    \label{eq:chi_f}\\
    \bm\sigma_f &= p_f \bm\delta + \mu_f (\grad \textbf{u}_f + \dagger \grad \textbf{u}_f) - \frac{\mu_f}{\phi_f} \avg{\delta_\Gamma (\textbf{u}_f' \textbf{n}+ \textbf{n}\textbf{u}_f')}
    \label{eq:sigma_f}\\
    n_p \textbf{f}_p  &= \pSavg{\bm\sigma'_f\cdot \textbf{n}}
    \label{eq:f_alpha}
    \\
    \bm{\sigma}_f^{\text{eff}} &= \pavg{\textbf{u}_\alpha'\textbf{u}_\alpha'}
    \label{eq:def_uup}
    \\
    \bm{\sigma}^{\text{eff}}_f &= \avg{\chi_f \textbf{u}_f'\textbf{u}_f'} - \pSavg{\textbf{r}\bm\sigma'_f\cdot \textbf{n}}
    + \frac{1}{2}\div\pSavg{\textbf{rr}\bm\sigma'_f\cdot \textbf{n}}
    \label{eq:def_sigma_eff_f}
\end{align}
respectively. 
Where we introduced: $\avg{\ldots}$ as an ensemble average procedure, 
$\textbf{u}_\alpha$ is the center of mass velocity of a particle labeled $\alpha$, $\chi_f$ is the phase indicator function of the continuous phase, and $\delta_p$ the Dirac delta function pointing on the particle center of masses, $\delta_\Gamma$ the interface indicator function, and \textbf{n} the normal of the surface pointing toward the continuous phase. 
The superscript $'$ indicates the relative values of a quantity with respect to its phasic mean values. 
Specifically $\bm{\sigma}_f' = \bm{\sigma}_f^0  - \bm{\sigma}_f$, 
$\textbf{u}_\alpha' = \textbf{u}_\alpha - \textbf{u}_p$ and $\textbf{u}_f' = \textbf{u}_f^0  -\textbf{u}_f$, with $\bm{\sigma}_f^0 $ and $\textbf{u}_f^0$ being the local stress of the continuous phase, the local velocity of the continuous phase, respectively. 

In the present situation, i.e. where the mixture is only governed by conservation of mass and momentum,  the ``closure problem'' consist in finding explicit expressions for the terms of the form $\avg{\ldots}$ in \ref{eq:sigma_f}, \ref{eq:f_alpha}, \ref{eq:def_uup} and \ref{{eq:def_sigma_eff_f}} in terms of the unknown of the problem, i.e. $n_p$, $\phi_f$, $\textbf{u}_p$ and $\textbf{u}_f$. 
Notice that since 


\subsection{ The momentum balance in homogeneous sedimentation}

In this work we restrict our attention to the homogeneous emulsion of buoyant droplets. 
Additionally, we will focus on the closure given by \ref{eq:f_alpha}. 

By statdy-state and ``homogeneous'', we imply that the ensemble averaged quantities, $\textbf{u}_p$, $\textbf{u}_f$, $n_p$ and $\phi_f$ are not function of space and time variables, i.e. $\textbf{x}$ and $t$. 
However, notice  that the mean continuous phase pressure, $p_f$, may be a function of $\textbf{x}$, even in the homogeneous case. 
Since every closure terms present in \ref{eq:dt_uf} and \ref{eq:dt_up} are ensemble average of fluctuating quantities they cannot be a function of the mean hydrostatic pressure, $p_f$.
Consequently, from \ref{eq:n_p} to \ref{eq:def_sigma_eff_f} the only space dependent quantity is $p_f$. 

In this situation, the averaged momentum conservation equations \ref{eq:dt_uf} and \ref{eq:dt_uf} can be re-written as, 
\begin{align}
    % \pddt (\phi_f\rho_f \textbf{u}_f)
    % + \div \left(\phi_f\rho_f \textbf{u}_f\textbf{u}_f + \phi_f  \bm{\sigma}_f^{\text{Re}} - n_p\textbf{M}_p \right)
    0 
    &= \phi_f 
    \left(\grad p_f
    + \rho_f \textbf{g}\right)
    - n_p \textbf{f}_p, 
    \label{eq:dt_uf_steady}
    \\
    % \pddt (\phi_d\rho_d \textbf{u}_p)
    % + \div \left(\phi_d\rho_d \textbf{u}_p\textbf{u}_p+ \phi_d \bm{\sigma}_p^{\text{Re}}\right)
    0
    &= 
    \phi_d \left(\grad p_f
    + \rho_d \textbf{g}\right)
    + n_p \textbf{f}_p. 
    % \label{eq:dy_up}
    \label{eq:dt_up_steady}
\end{align}
where we have used the relation $n_p v_p = \phi_d$, which is not an approximation anymore since $\grad n_p = 0$ in this specific case.
Our goal is to build a model for the force density closure $\textbf{f}_p$, then \ref{eq:dt_uf_steady} and \ref{eq:dt_up_steady} represent out starting point to accomplish this task. 



Multiplying \ref{eq:dt_uf_steady} by $\phi_d$ and \ref{eq:dt_up_steady} by $\phi_f$, and subtracting the resulting equations, gives directly the equilibrium between buoyancy and drag force density, namely, 
\begin{align}
     \textbf{f}_p
    &= 
    \frac{4}{3}\frac{d^3 \pi}{8}\ \phi_f (\rho_f -\rho_d ) \textbf{g}. 
    \label{eq:f_p_buoyant}
\end{align}
% In  mono-disperse suspension of droplets $n_p = \phi_d / v_p$ with $v_p =4/3\pi d^3/8$ the volume of a particle which yields the final results, 
% \begin{equation*}
%     \textbf{f}_p
%     = 
%     \frac{4}{3}\pi\frac{d^3}{8}\phi_f (\rho_f -\rho_d ) \textbf{g}
%     \label{eq:drag}
% \end{equation*}
% It is convinient to make dimensionless this force with Hadamard-Ribczynski formula, which is, 
% \begin{equation*}
%     \textbf{f}^0_p = \pi \mu_f d A \textbf{u}_{pf}
% \end{equation*}
% Dividing one by the other gives the dimensionless force
% \begin{equation*}
%     \textbf{f}^*_p 
%     = 
%     \frac{4}{3A}\frac{d^2 \phi_f (\rho_f -\rho_d ) \textbf{g}}{8 u_{pf}\mu_f}
% \end{equation*}
% This can be made dimensionless with $\phi_f$
Let us assume that the force density can be written in the form, 
\begin{equation*}
    \textbf{f}_p = C_d  \pi \rho_f \frac{d^2}{8} u_{pf}^2
    \label{eq:f_p_def}
\end{equation*}
where $C_d$ is a dimensionless coefficient, and $u_{pf}^2 = (\textbf{u}_p - \textbf{u}_f)\cdot (\textbf{u}_p - \textbf{u}_f)$, is the relative mean phase velocity squared. 
Using \ref{eq:f_p_buoyant}  and \ref{eq:f_p_def} we can show that $C_d$ is related to the mean phase relative velocity with, 
\begin{equation}
    C_d  
    = 
    \frac{4}{3}
    \frac{d \phi_f (\rho_f -\rho_d ) \textbf{g}}{\rho_f u_{pf}^2}
    \label{eq:C_d}
\end{equation}
The right-hand side of \ref{eq:C_d} can be reformulated according to three dimensionless groups, it yields, 
\begin{equation}
    C_d = 
    \frac{4\phi_f}{3} \left(\frac{Ga}{Re}\right)^2
    \label{eq:C_d_adim}
\end{equation}
where $Ga$ is the \textit{Galileo} number and $Re$ the \textit{Reynolds} number, namely, 
\begin{align}
    Ga^2 = \frac{
    d^3
    \rho_f
    (\rho_f -\rho_d ) g
}{\mu_f^2},
&& 
Re =   \frac{\rho_f d u_{pf}}{\mu_f}.
\end{align}

In summary, the force density term, $\textbf{f}_p$, can be written in terms of the dimensionless constant $C_d$, which can itself be written in terms of $Ga$, $\phi_f$ and $Re$ in the present context. 
The \textit{Galileo} number and the volume fraction $\phi_f$ are known parameters in our problem, indeed they are only function of the physical and geometrical properties of the mixture.
Consequently, the closure problem for $\textbf{f}_p$ consist in measuring the \textit{Reynolds} number or the mean relative motion between phases, knowing $\phi_f$ and $Ga$ a priori. 
The mean relative motion between phases, can be obtained either experimentally, theoretically or numerically. 
In this work, we combine experimental measures and theoretical results from the literature to provide the most complete model for $C_d$.
Then, based on the DNS presented in the following section we will be able to validate and see the limitation of our model. 

% \tb{
% This can be directly computed into our DNS. 
% \begin{equation*}
%     C_d  \phi_f^2 \frac{\rho_f^2 d^2 u_{pf}^2}{\mu_f^2}
%     = 
%     \frac{4}{3}
%     \phi_f^3 
%     \frac{
%         d^3
%         \rho_f
%         (\rho_f -\rho_d ) g
%     }{\mu_f^2}
% \end{equation*}
% Let us define the Galileo number as $Ga^2 = \frac{
%     d^3
%     \rho_f
%     (\rho_f -\rho_d ) g
% }{\mu_f^2}$ and the Reynolds number based on the drift velocity as, $Re =  \phi_f \frac{\rho_f d u_{pf}}{\mu_f}$,
% Then, the relation between the Reynolds and Galileo is given by, 
% \begin{equation*}
%     Re
%     = 
%     Ga
%     \sqrt{\frac{4\phi_f^3}{3 C_d}}
% \end{equation*}
% This the relative velocity is given by, 
% In stokes and dilute regime the $C_d$ noted $D_c^0$ is given by Hadamard-Ribczynski solution and reads, 
% \begin{equation*}
%     C_d^0 = \frac{8}{Re} \left(\frac{3\lambda +2 }{\lambda +1}\right)
%     = \frac{8}{Re}A
% \end{equation*}
% where we introduced the constant $A = \left(\frac{3\lambda +2 }{\lambda +1}\right)$. 
% Thus, the Reynolds number obtained for a given \textit{Galileo} number in stokes regime is, 
% \begin{equation*}
%     Re^0
%     = 
%     \frac{Ga^2}{6 A}
%     % \phi_f^3  
% \end{equation*}
% Since this is valid for an isolated particle we fixed $\phi_f=1$, this will be our renormalization constant. 
% \begin{equation*}
%     Re^*
%     = 
%     \frac{6A}{Ga}
%     \sqrt{\frac{4\phi_f^3}{3 C_d}}
% \end{equation*}
 
% \paragraph{Relation between Galileo and Reynolds numbers :}
% From the two previous expressions we can write the equality, 
% \begin{equation*}
%     \textbf{f}_p = C_d  \pi \rho_f \frac{d^2}{8} u_{pf}^2
% \end{equation*} 
% The hadamar ribinsky formula reads, 
% \begin{equation*}
%     \textbf{f}_p^0 =\pi \mu_f d A \textbf{u}_{pf}
% \end{equation*}
% dividing one by the other and by $\phi_f^2$ gives directly, 
% \begin{equation*}
%     \textbf{f}_p^* =   \frac{C_d  Re}{8 A} = \frac{C_d}{C_d^*}
% \end{equation*}
% }
\subsection{Computational methodology}


To represent a statistically steady-state and homogeneous buoyant emulsion we carry out DNS of tri-periodic rising droplets. 
Notice that for the statistics to converge, the domain have to be large enough, and the simulation time long enough. 

% We display in \ref{tab:dimensionless_numbers} the dimensionless numbers explored in this work. 
% \begin{table}[h!]
%     \centering
%     \caption{Dimensionless parameter range investigated in this work.}
%     \label{tab:dimensionless_numbers}
%     \begin{tabular}{|ccccccc|ccc|}
%         \hline
%         \multicolumn{7}{|c}{Primary parameters} & \multicolumn{3}{||c|}{Secondary parameters}\\ \hline
%         \multicolumn{1}{|c|}{$Ga$}                               & \multicolumn{1}{c|}{$Bo$}                   & \multicolumn{1}{c|}{$\phi$} & \multicolumn{1}{c|}{$\lambda$}                    & \multicolumn{1}{c|}{$\zeta$}                & \multicolumn{1}{c|}{$N_b$} & $t^*_\text{end}$ & \multicolumn{1}{||c|}{$\mathcal{L}/d$} & \multicolumn{1}{c|}{$Re$}  & $We$   \\ \hline
%         \multicolumn{1}{|c|}{\multirow{4}{*}{$5\rightarrow 80$}} & \multicolumn{1}{c|}{\multirow{4}{*}{$0.5$}} & \multicolumn{1}{c|}{$1\%$}  & \multicolumn{1}{c|}{\multirow{4}{*}{$10$ \& $1$\&$0.1$}} & \multicolumn{1}{c|}{\multirow{4}{*}{$0.9$}} & \multicolumn{1}{c|}{$160$} & $400$           & \multicolumn{1}{||c|}{$20$}            & \multicolumn{1}{c|}{$1.3\to 110$} & {$0.03\to 0.95$} \\ 
%         \multicolumn{1}{|c|}{}                                   & \multicolumn{1}{c|}{}                       & \multicolumn{1}{c|}{$5\%$}  & \multicolumn{1}{c|}{}                             & \multicolumn{1}{c|}{}                       & \multicolumn{1}{c|}{$800$} & $400$           & \multicolumn{1}{||c|}{$20$}            & \multicolumn{1}{c|}{$1.0\to 92$} &  {$0.02\to 0.67$}\\ 
%         \multicolumn{1}{|c|}{}                                   & \multicolumn{1}{c|}{}                       & \multicolumn{1}{c|}{$10\%$} & \multicolumn{1}{c|}{}                             & \multicolumn{1}{c|}{}                       & \multicolumn{1}{c|}{$200$} & $1000$           & \multicolumn{1}{||c|}{$10$}            & \multicolumn{1}{c|}{$1.9\to 77$}&  {$0.01\to 0.47$}\\ 
%         \multicolumn{1}{|c|}{}                                   & \multicolumn{1}{c|}{}                       & \multicolumn{1}{c|}{$20\%$} & \multicolumn{1}{c|}{}                             & \multicolumn{1}{c|}{}                       & \multicolumn{1}{c|}{$400$} & $1000$           & \multicolumn{1}{||c|}{$10$}            & \multicolumn{1}{c|}{$1.7\to 62$}&  {$9\cdot 10^{-3}\to 0.31$}\\ \hline
%         \end{tabular}
% \end{table}


The study's primary objective is to measure the mean relative velocity between the droplets and the continuous phase.
Thus, obtaining a sufficient number of DNS samples is crucial to ensure a good statistical convergence of these mean quantities. 
Also, the physical quantities measured in the simulations must remain independent of the domain size. 
Regarding the grid spacing $\Delta$, we show in \ref{ap:convergence}  that using a definition of $d/\Delta = 25$ is enough to obtain representative results for $Re$. 
We set $\mathcal{L}/d = 10$, which is roughly what \citet{hidman2023assessing} used for their DNS of fully-periodic buoyant rising bubbles.
Likewise, we use a number of particles per domain of at least $N_b = 160$ for all our cases, which introduces the need for a larger domain ($\mathcal{L}/d = 20$) for the dilute cases, so that the  $d/\Delta = 25$ is respected. 
Each DNS lasts for a time: $t^*_\text{end} = 400 \sqrt{d/g}$ for the larger domains ($\mathcal{L}/d=20$) and $t^*_\text{end} = 1000 \sqrt{d/g}$ for the smaller domain.
% It is shown in \ref{ap:validation} that these parameters are sufficient to obtain well converged statistics.  
It is shown in \citet{fintzi2024buoyancy} and \ref{ap:convergence} that these parameters are sufficient to obtain well converged statistics. 
\begin{table}[h!]
    \centering
    \caption{Dimensionless parameter range investigated in this work.}
    \begin{tabular}{|ccccccc|ccc|}
        \hline
        \multicolumn{7}{|c}{Primary parameters} & \multicolumn{3}{||c|}{Secondary parameters}\\ \hline
        \multicolumn{1}{|c|}{$Ga$}                               & \multicolumn{1}{c|}{$Bo$}                   & \multicolumn{1}{c|}{$\phi$} & \multicolumn{1}{c|}{$\lambda$}                    & \multicolumn{1}{c|}{$\zeta$}                & \multicolumn{1}{c|}{$N_b$} & $t^*_\text{end}$ & \multicolumn{1}{||c|}{$\mathcal{L}/d$} & \multicolumn{1}{c|}{$Re$}  & $We$   \\ \hline
        \multicolumn{1}{|c|}{\multirow{4}{*}{$5\rightarrow 80$}} & \multicolumn{1}{c|}{\multirow{4}{*}{$0.5$}} & \multicolumn{1}{c|}{$1\%$}  & \multicolumn{1}{c|}{\multirow{4}{*}{$10$ $\to$ $0.1$}} & \multicolumn{1}{c|}{\multirow{4}{*}{$0.9$}} & \multicolumn{1}{c|}{$160$} & $400$           & \multicolumn{1}{||c|}{$20$}            & \multicolumn{1}{c|}{$1.3\to 110$} & {$0.03\to 0.95$} \\ 
        \multicolumn{1}{|c|}{}                                   & \multicolumn{1}{c|}{}                       & \multicolumn{1}{c|}{$5\%$}  & \multicolumn{1}{c|}{}                             & \multicolumn{1}{c|}{}                       & \multicolumn{1}{c|}{$800$} & $400$           & \multicolumn{1}{||c|}{$20$}            & \multicolumn{1}{c|}{$1.0\to 92$} &  {$0.02\to 0.67$}\\ 
        \multicolumn{1}{|c|}{}                                   & \multicolumn{1}{c|}{}                       & \multicolumn{1}{c|}{$10\%$} & \multicolumn{1}{c|}{}                             & \multicolumn{1}{c|}{}                       & \multicolumn{1}{c|}{$200$} & $1000$           & \multicolumn{1}{||c|}{$10$}            & \multicolumn{1}{c|}{$1.9\to 77$}&  {$0.01\to 0.47$}\\ 
        \multicolumn{1}{|c|}{}                                   & \multicolumn{1}{c|}{}                       & \multicolumn{1}{c|}{$20\%$} & \multicolumn{1}{c|}{}                             & \multicolumn{1}{c|}{}                       & \multicolumn{1}{c|}{$400$} & $1000$           & \multicolumn{1}{||c|}{$10$}            & \multicolumn{1}{c|}{$1.7\to 62$}&  {$9\cdot 10^{-3}\to 0.31$}\\ \hline
        \end{tabular}
    \label{tab:simulations}
\end{table}
This study presents DNS results with dimensionless parameters in ranges outlined in \ref{tab:simulations}.
In summary, we investigated $5$ \textit{Galileo} number $Ga = 5,10,25,50,80$, $4$ different volume fractions $\phi = 0.01,0.05,0.1,0.2$, and three viscosity ratios $\lambda =0.1,1,10$ with $Bo = 0.5$ and $\zeta = 0.9$.
This makes a total of $60$ representative simulations.

Due to numerical constraint we used a slightly higher \textit{Bond} number ($Bo = 0.5$) in this study, compared to the previous set of DNS used in \ref{chap:microstructure}. 
\begin{figure}[h!]
    \centering
    \includegraphics[height = 0.3\textwidth]{image/HOMOGENEOUS_final/PA/chi.pdf}
    \caption{Mean aspect ratio of the droplets $\chi_p$, as a function of the \textit{Galileo} number, and the volume faction $\phi$,  for two different viscosity ratios.  
    The symbols correspond to different volume fraction ($\pmb\bigcirc$) $\phi = 0.01$; ($\pmb\triangle$) $ \phi = 0.05$; ($\pmb\square$) $\phi = 0.1$ ($\pmb\lozenge$) $\phi = 0.2$.
    The hollow symbols correspond to $\lambda = 1$, the filled symbols to $\lambda = 10$, and the small symbol to $\lambda = 0.1$.
    The nearly imperceptible vertical bars on each symbol, represent the standard deviation around the mean.  }
    \label{fig:chi2}
\end{figure}
To verify that the droplets remain in average approximately spherical we plotted in \ref{fig:chi2} the mean aspect ratio $\chi_p$ of the droplet for all of our simulation. 
A rigorous definition of this aspect ratio is given in \citet{fintzi2024buoyancy} or \citet{bunner2003effect}. 
Anyhow, at $Bo = à.5$ the maximum mean droplet deformation $\chi_p$ represents about $4\%$ of deformation, compared to the $2\%$ obtained for $Bo = 0.2$. 
This means that the droplet shape deviate approximately of $4\%$ from their original spherical shape. 


\subsection{Approximation of the ensemble average}

Following \citet{du2022analysis} we consider ergodicity at all time of the numerical experiment.
Thus, the ensemble average of a quantity $X$ can be approximated by a spatial average $\Xavg{X}$ and a time average $\Tavg{X}$ such that $\avg{X} \approx \Xavg{\Tavg{X}} = \Tavg{\Xavg{X}}$.
Consequently, the ensemble average of a numerical field, $X$, is taken through space and time such that,
\begin{equation}
    \avg{X}
    = \Tavg{\Xavg{X}}
    = \frac{1}{ t_{end} - t_0}\int_{t_0}^{t_{end}} 
    \Xavg{X}(t) dt
\end{equation}
where, 
\begin{equation}
    \Xavg{X}(t)
    = \frac{1}{L^3}\int 
    X(\textbf{x},t) d\textbf{x}
\end{equation}
where $L$ is the length of one side of the cubic numerical domain.
$t_0$ and $t_{end}$ are the starting time of sampling, and the ending time of sampling which is also the ending time of the simulation, respectively.
In practice, we take $t_0$ such that the simulation reach a statistically steady regime. 
It has been found that $t_0 < 50\sqrt{g/d}$.  
% $t_{end}$ is given in \ref{tab:simulations} and is shown to be sufficient as demonstrated in \ref{ap:convergence}. 

To compute the phase average of the local phase velocity $\textbf{u}_k^0$, we simply perform an integration over space and time, 
\begin{equation}
    \textbf{u}_k = \frac{1}{\phi_k} \Tavg{\Xavg{\chi_k \textbf{u}_k^0}}
\end{equation}
where the indicator function $\chi_f$ must be understood as its approximation in the DNS, which is the color function used by the code \url{http://basilisk.fr}. 

Since the homogeneous and statistically steady-state hypothesis are supposed to be true, the mean of the droplet center of mass velocity is equivalent to the dispersed phase phasic mean velocity, $\textbf{u}_d$.
Thus, we may compute the ensemble average relative velocity used in \ref{eq:C_d} with the operation, 
\begin{equation}
    \textbf{u}_{pf} = 
    \frac{1}{\phi_d} \Tavg{\Xavg{\chi_d \textbf{u}_d^0}}
    - \frac{1}{\phi_f} \Tavg{\Xavg{\chi_f \textbf{u}_f^0}}. 
\end{equation} 

Now that we have all the tools in hand, we present in the next section our generalized model for the drag force adapted to droplet of arbitrary viscosity. 
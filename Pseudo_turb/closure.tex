\subsection{The closure problem}
Objectives
\begin{itemize}
    \item Introduce the closure problem and the form of the closure terms. 
    \item Focus only on the drag force and velocity fluctuations. 
\end{itemize}
To clarify the role of the different forces and stresses in a multiphase flow model we start by listing the averaged momentum equations for dispersed two-phase flows obtained by an ensemble averaging procedure.
For mono disperse fluid particles they read as \citep{zhang1997momentum,jackson1997locally}\todo{citer le papier with Daniel},
\begin{align}
    \pddt (\phi_c\rho_c \textbf{u}_c)
    + \div \left(\phi_c\rho_c \textbf{u}_c\textbf{u}_c + \phi_c  \bm{\sigma}_c^{\text{Re}} - n_p\textbf{M}_p \right)
    &= \phi_c 
    \left(\div \bm{\sigma}_c
    + \rho_c \textbf{g}\right)
    - n_p \textbf{f}_p 
    \label{eq:dt_uc}
    \\
    \pddt (\phi_d\rho_d \textbf{u}_p)
    + \div \left(\phi_d\rho_d \textbf{u}_p\textbf{u}_p+ \phi_d \bm{\sigma}_p^{\text{Re}}\right)
    &= 
    \phi_d \left(\div \bm{\sigma}_c
    + \rho_d \textbf{g}\right)
    + n_p \textbf{f}_p 
    \label{eq:dy_up}
\end{align}
where the subscript $c$ and $p$ denote continuous and particle phase averaged quantities, respectively.
$\phi$ is the volume fraction, $n_p$ the particle number density, \textbf{u} the averaged velocity, $\bm{\sigma}$ the averaged stress tensor, $\bm{\sigma}^{\text{Re}}$ the stress due to velocity fluctuation or pseudo turbulent stress.
Additionally, due to the presence of the particles we have the interphase drag force term : $\textbf{f}_p$, and the interphase stress or first moment : $\textbf{M}_p$. 


In this study we focus on three of these closure terms, namely the interphase drag force : $\textbf{f}_p$, and the Reynolds stresses : $\bm{\sigma}^{\text{Re}}_c$ and $\bm{\sigma}^{\text{Re}}_p$. 
Using the statistical average formalism on all configuration of the flow termed by $\CC$, we can define these quantities as,  
\begin{align}
    \textbf{f}_p (\textbf{x},t) &= \
    \frac{1}{n_p}\int \sum_i \delta(\textbf{x} - \textbf{x}_\alpha(t,\CC)) \textbf{f}_\alpha d\PP\\
    \label{eq:f_alpha}
    \text{with} \;\;\; &\textbf{f}_\alpha(t,\CC) = \int_{S_\alpha} \bm{\sigma}_c'(\textbf{x},t,\CC) \cdot \textbf{n} dS \\
    \bm{\sigma}^{\text{Re}}_p(\textbf{x},t) &=\frac{1}{n_p} \int \sum_i \delta(\textbf{x} - \textbf{x}_\alpha(t,\CC)) \rho_d \textbf{u}_\alpha'\textbf{u}_\alpha' d\PP\\
    \label{eq:def_uuc}
    \bm{\sigma}^{\text{Re}}_c(\textbf{x},t) &= \frac{1}{\phi_c}\int \rho_c\chi_c \textbf{u}_c' \textbf{u}_c'd\PP
\end{align}
where $d\PP$ is a probability measure. 
$\textbf{x}_\alpha$ is the center of mass of a particle labeled $\alpha$, $\chi_c(\textbf{x},t\CC)$ is teh phase indicator function of the continuous phase, and $\delta$ the Dirac delta function. 
The superscript $'$ indicate the relative values of a quantity with respect to its phasic mean values, specifically $\bm{\sigma}_c' = \bm{\sigma}_c^0(\textbf{x},t, \CC)  - \bm{\sigma}_c(\textbf{x},t)$, $\textbf{u}_\alpha' = \textbf{u}_\alpha(t,\CC) - \textbf{u}_p(\textbf{x},t)$ and $\textbf{u}_c' = \textbf{u}_c^0(\textbf{x},t, \CC)  -\textbf{u}_c(\textbf{x},t)$, with $\bm{\sigma}_c^0 $, $\textbf{u}_c^0$ and $\textbf{u}_\alpha$ being the local stress of the fluid phase, the local velocity of the fluid phase and the velocity of the center of mass of the particle $\alpha$, respectively. 






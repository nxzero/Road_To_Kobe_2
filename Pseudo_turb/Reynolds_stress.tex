
Objectives : 
\begin{itemize}
    \item Present the decomposition of the fluid reynolds stress according to isotropic and deviatoric part.
    \item Show the relation between the flowlines graphs and the actual values of the velocity fluctuation.
    \item Compare our case with \citet{almeras2019fluctuations}
\end{itemize}



We decompose both Reynolds stress into an isotropic part and deviatoric part such that, 
\begin{align}
    \bm{\sigma}^{\text{Re}}_p &=  \rho_d \phi_d K^*_p
    \left[
        \textbf{I}(\textbf{u}_p - \textbf{u}_c)\cdot (\textbf{u}_p - \textbf{u}_c) 
        +\textbf{B}_c \cdot (\textbf{u}_p - \textbf{u}_c)(\textbf{u}_p - \textbf{u}_c)
    \right]\\
    \bm{\sigma}^{\text{Re}}_c &=  \rho_c \phi_c K_c^*
    \left[
        \textbf{I}(\textbf{u}_p - \textbf{u}_c)\cdot (\textbf{u}_p - \textbf{u}_c) 
        +\textbf{B}_p \cdot (\textbf{u}_p - \textbf{u}_c)(\textbf{u}_p - \textbf{u}_c)
    \right]
\end{align}
where the $K^*$ is the dimensionless pseudo-turbulent  kinetic energy, $\textbf{I}$ a unit tensor and $\textbf{B}$ a tensor accounting for the deviation of the Reynolds stress left to determine. 

\subsection{The continuous phase Reynolds stress}
\todo{try \citet{almeras2021statistics} fits}
\todo{also check \citet{almeras2019fluctuations} results}
\tb{might be good to plot the velocity field to examine where does the fluctuation arise (pseudo-turbe or turbe)}
Look at \citep{wang2021numerical} and Mahra.. 2015 



The fluid averaged kinetic energy can be easily scaled on the numerical results shown \ref{fig:Tf_Bf}(left).
\begin{figure}[h!]
    \centering
    \includegraphics[height=0.3\textwidth]{image/HOMOGENEOUS/fCA/Tf_l_1.pdf}
    \includegraphics[height=0.3\textwidth]{image/HOMOGENEOUS/fCA/Bf_l_1.pdf}

    \includegraphics[height=0.3\textwidth]{image/HOMOGENEOUS/fCA/Tf_l_10.pdf}
    \includegraphics[height=0.3\textwidth]{image/HOMOGENEOUS/fCA/Bf_l_10.pdf}
    \caption{(left) Dimensionless turbulent kinetic energy in terms of the \textit{Galileo} number for different $\phi$. (dots) Numerical simulations, (dashed line) empirical formula \ref{eq:Tf_scaling}.
    The symbols correspond to different volume fraction ($\bullet$) $\phi = 1\%$, ($\blacktriangle$) $\phi = 5\%$, ($\blacksquare$) $\phi = 10\%$, ($\blacklozenge$) $\phi = 15\%$ and ($\blacktriangleright$) $\phi = 20\%$.
    (right) deviatoric part of the Reynolds stress, ($- \cdot -$)  vertical components, $B_{yy}$, ($- -$)  horizontal components, $B_{xx} = B_{zz}$.}
    \label{fig:Tf_Bf}
\end{figure}
\subsection{The particles phase Reynolds stress}

\tb{Maybe include velocity fluctuation and compare to : Lingxin2021 }

\tb{Include Gaussian distribution of bubbles !!! ! ! ! }
Now let's focus on the particular averaged Reynolds stress tensor.
\ref{fig:Talpha_Balpha} shows that the granular temperature behavior is quite similar from the continuous averaged turbulent kinetic energy.
\begin{figure}[h!]
    \centering
    \includegraphics[height=0.3\textwidth]{image/HOMOGENEOUS/fPA/Talpha_l_1.pdf}
    \includegraphics[height=0.3\textwidth]{image/HOMOGENEOUS/fPA/Balpha_l_1.pdf}

    \includegraphics[height=0.3\textwidth]{image/HOMOGENEOUS/fPA/Talpha_l_10.pdf}
    \includegraphics[height=0.3\textwidth]{image/HOMOGENEOUS/fPA/Balpha_l_10.pdf}
    \caption{(left) Dimensionless turbulent kinetic energy in terms of the \textit{Galileo} number for different $\phi$. (dots) Numerical simulations, (dashed line) empirical formula \ref{eq:Talpha_scaling}.
    (right) deviatoric part of the Reynolds stress, ($\bullet$) are the vertical components, $B_{yy}$, ($\blacktriangle$) are the horizontal components, $B_{xx} = B_{zz}$.}
    \label{fig:Talpha_Balpha}
\end{figure}
We can also provide a scaling for the granular temperature, it reads as,  
\begin{equation}
    \frac{\pnavg{T_\alpha}}{U^2}  \approx \frac{\phi}{Ga^2} 2.86\cdot10^{4} 
    \label{eq:Talpha_scaling}
\end{equation}
From \ref{fig:Talpha_Balpha} we observe that this scaling is valid for the lowest \textit{Galileo}. 
The deviatoric part of $\pnavg{T_\alpha}$ is displayed on \ref{fig:Talpha_Balpha}.
It tells us that the Reynolds stress for the particular phase tends to be isotropic in the same way as $\cavg{T}$. 
Indeed, the components of $\pnavg{\textbf{B}}$ go to zero with increasing $Ga$ and $\phi$. 
This behavior is explained by the higher rate of collision present for higher volume fraction and inertia \citep[chapter 1]{jackson2000dynamics}
\section{Conclusion}

In this study, we have demonstrated, using the ensemble-averaged momentum equations for dispersed multiphase flows, the relationship between buoyancy forces and the mean drag force density applied to the dispersed phase, which consists of identically sized spherical droplets. 

In the homogeneous regime, we established a relationship between the sedimentation velocity of the dispersed phase, the volume fraction $\phi$, and the Reynolds number $Re$, and $\lambda$ by applying a Richardson-Zaki-like relation, already validated in Stokes flow for both solid particles and spherical bubbles .
Building on the approach of \citet{jackson2000}, we demonstrated how to relate the sedimentation velocity to the mean drag force in both the Stokes and Newtonian regimes. 
This led to the development of a new drag force model applicable across all physical values of  $Re$, $\phi$ and $\lambda$ as long as the droplets remain spherical and mono-disperse. 

In the dilute limit, our correlation is a priori validated by the literature, as it employs the well-known Schiller-Neuman, and Mei drag force coefficient, for solid particle and spherical bubbles, respectively. 
For solid particles ($\lambda \to \infty$), the $\phi$-dependency, converges to the recent model of \citet{kramer2019improvement}, which is an improvement of the Richardson-Zaki relation valid for arbitrary $\phi$ and $Re$.
To validate our model in the intermediate $\lambda$ regime, we performed DNS of buoyant emulsions in a tri-periodic domain, since it well approximates a homogeneous uniform multiphase flow (once the properties are averaged).
The DNS results show good agreement with our model accurately capturing the $\phi$-dependence for a given $\lambda$ and $Re$.
Thus, our model, which was built on already existing correlations valid at $\lambda\to\infty$ and $\lambda = 0$, is shown to be valid at intermediate values of $\lambda$, making it valid for all $Re$, $\lambda$ and $\phi$.  

The main advantage of the formulation provided by \ref{eq:C_d_finalRe} is its robustness since Richardson-Zaki relation used here is valid at very high volume fraction  ($\phi \approx 0.5$) and arbitrary Reynolds numbers, while in the dilute limit Schiller-Neuman, and Mei drag force coefficient are proven to be accurate up to $Re = 800$. 
Thus, we provided a robust drag force coefficient that can directly be used in Euler-Euler framework for simulations of emulsions of arbitrary viscosity ratio. 
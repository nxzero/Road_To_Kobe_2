\section{Conclusion}

In this study, we have demonstrated the relationship between buoyancy forces and the mean drag force density on droplets, derived from the ensemble-averaged momentum equations. 
In the homogeneous regime, we linked the sedimentation velocity of the dispersed phase with volume fraction and Reynolds number through the Richardson-Zaki relation in Stokes flow, both for solid particle and bubbles.

We then extended the Richardson-Zaki framework to viscous droplets of arbitrary viscosity ratio, valid for arbitrary $Re$ and $\phi$. 
In the dilute limit, our correlation is validated by the literature, as it employs the well-known Schiller-Neuman, and Mei drag force coefficient, for solid particle and spherical bubbles, respectively. 
For solid particles ($\lambda \to \infty$), our model converges to the recent model of \citet{kramer2019improvement}, which is an improvement of the Richardson-Zaki relation valid for arbitrary $\lambda$ and $Re$.

To validate the intermediate $\lambda$ regime, we performed DNS of buoyant emulsions in a tri-periodic box. 
The results show good agreement, with our model accurately capturing the $\phi$-dependence for a given $\lambda$ and $Re$.

The main advantage of the formulation provided by \ref{eq:C_d_finalRe} is its robustness, since Richardson-Zaki relation have been shown to be valid at very high volume fraction  ($\phi \approx 0.5$) and Reynolds number, while in the dilute limit Schiller-Neuman, and Mei drag force coefficient are proven to be accurate up to $Re = 800$. 
Thus, we provided, a robust drag force coefficient that can directly be used in Euler-Euler framework for simulation of emulsion of arbitrary viscosity ratio. 
\documentclass[12pt]{My_preprint}
\title{
    Statistical modeling of the \textit{Flotation} process with ``Collision Efficiency'' set-up 
    }

\author[1,2]{Nicolas Fintzi}
\normalmarginpar


\begin{document}

\maketitle

\begin{abstract}
\end{abstract}

\section{Three-phase system}

We consider a three-phase system made of small particles (pollutant) and rising bubbles dispersed into a continuous phase. 

\begin{table}[h!]
    \centering
    \begin{tabular}{cccccc}
        & Continuous phase  & Bubbles   & Particles & ratio $b$-$f$                       & ratio $p$-$f$\\
        Radius         & $-$          & $a_b$ or $a$  & $a_p$  & $\beta = a_p /a$&$-$ \\
        Density         & $\rho_f$          & $\rho_b$  & $\rho_p$  & $-$&$-$ \\
        Viscosity       & $\mu_f$           & $\mu_b$   & $\infty$ &  $-$&$-$ \\
        % Volume fraction & $\phi_f$          & $\phi_b$  & $\phi_p$  &                              & \\
        % number density  &              & $n_b$     & $n_p$     &                           & \\
        Denisty ratio   &  $-$  &  $-$   &  $-$    & $\zeta_{b} = \rho_b/\rho_f$     & $\zeta_{p} = \rho_p/\rho_f$\\
        Viscosity ratio &  $-$  &  $-$   &  $-$    & $\lambda_{b}= \lambda = \mu_b/\mu_f$      & $\lambda_{p} = \rho_p/\rho_f$\\
        Velocity ($\textbf{u}_k^0$) &    $\textbf{u}_f^0$        &  $\textbf{u}_b^0$   &  $\textbf{u}_p^0$    &    $-$   & $-$\\
        Pressure ($p_k^0$) &    $p_f$ &  $p_b$   &   $-$&  $-$& $-$\\
        Stress ($\bm\tau_k^0$) &    $\mu_f (\grad \textbf{u}_f^0 + \grad \textbf{u}_f^0) $        & $\mu_b (\grad \textbf{u}_b^0 + \grad \textbf{u}_b^0) $    & $\mathbb{C}:\textbf{E}$&  $-$     &$-$ \\
        Outward normal &     $-$       &   $-$  &  $-$    & $\textbf{n}_b$      & $\textbf{n}_p$\\
        Surface stress ($\bm\sigma_\Gamma$) &   $-$         &  $-$   &   $-$   & $\gamma  (\bm\delta -\textbf{n}_b\textbf{n}_b)$    &$-$ \\
        PIF ($\chi_k$ and $\delta_{\Gamma}$) &$\chi_f$& $\chi_b$    &  $\chi_p$    &  $\delta_{\Gamma b}$  & $\delta_{\Gamma p}$  \\
        PIF ($\delta_k$) &$-$& $\delta_b$    &  $\delta_p$    & $-$  & $-$ 
    \end{tabular}
    \caption{Phases properties and constitutive laws.}
\end{table}


\subsection{Local scale equations}

\begin{table}[h!]  
    \centering
    \begin{tabular}{ccl}
     & Mass & Momentum \\
    conserved quantity & $\rho_k$ & $\rho_k \textbf{u}_k^0$ \\
    source term  & $0$ & $\rho_k \textbf{g}$ \\
    diffusive flux & 0 & $\bm\sigma_k^0 = -p_k^0 \bm\delta + \mu_k (\grad \textbf{u}_k^0 + \grad \textbf{u}_k^0)$ \\
    interface quantity & $0$ & $0$ \\
    surface diffusive flux & 0 & $\bm\sigma_\Gamma^0 = \gamma (\bm\delta - \textbf{n}_b\textbf{n}_b)$ \\
    \end{tabular}
    \caption{Definition of the physical quantities, and local constitutive laws}
    \label{tab:qte_Newtonian}
\end{table}
The general form of the volume/interface conservation equation reads as, 
\begin{align}
    \pddt (\chi_k f_k^0)
    + \div (
        \chi_k f_k^0 \textbf{u}_k^0
        - \chi_k \mathbf{\Phi}_k^0 
        )
    &= 
    \chi_k s_k^0
    + \delta_\Gamma \left[
        f_k^0
        \left(
            \textbf{u}_\Gamma^0
            - \textbf{u}_k^0
        \right)
        + \mathbf{\Phi}_k^0
    \right]
    \cdot \textbf{n}_k.
    \label{eq:dt_chi_k_f_k}\\
    \pddt (\delta_\Gamma f_\Gamma^0)  
    + \div (
        \delta_\Gamma  f_\Gamma^0 \textbf{u}_\Gamma^0
        - \delta_\Gamma  \mathbf{\Phi}_{\Gamma||}^0 
        )
    &= 
    \delta_\Gamma s_\Gamma^0
    - \delta_\Gamma \Jump{
    f_k^0 (\textbf{u}_\Gamma^0 - \textbf{u}_k^0)
    + \mathbf{\Phi}_k^0},
    \label{eq:dt_delta_I_f_I}
\end{align}


Thus, the system bulk-bubbles-particles might be described by 6 equaitons, 
\begin{align}
    \pddt (\chi_b \rho_b)
    + \div 
        \chi_b \rho_b \textbf{u}_b^0
    &= 
    0\\
    \pddt (\chi_p \rho_p)
    + \div 
        \chi_p \rho_p \textbf{u}_p^0
    &= 
    0\\
    \div \textbf{u}^0
    &= 
    0
\end{align}
\begin{align}
    \pddt (\chi_b \rho_b \textbf{u}^0_k)
    + \div (
        \chi_b \rho_b \textbf{u}^0_b \textbf{u}_b^0
        )
    &= 
    \chi_b \rho_b \textbf{g}
    + \div (\chi_b \bm\sigma_b^0  + \delta_{\Gamma b} \bm\sigma_\Gamma^0 )
    + \delta_{\Gamma b}  \bm\sigma_f^0 \cdot \textbf{n}_b\\
    \pddt (\chi_p \rho_p \textbf{u}^0_k)
    + \div (
        \chi_p \rho_p \textbf{u}^0_p \textbf{u}_p^0
        )
    &= 
    \chi_p \rho_p \textbf{g}
    + \div \chi_p \bm\sigma_p^0  
    + \delta_{\Gamma b}  \bm\sigma_f^0 \cdot \textbf{n}_p
\end{align}
\begin{equation}
    \rho_f (\pddt + \textbf{u}^0 \cdot \grad)\textbf{u}^0
    = 
    \rho_f \textbf{g}
    + \div \bm\sigma^0_*
    +\kappa_b  \delta_{\Gamma b}  \bm\sigma_f^0 \cdot \textbf{n}_b 
    +\kappa_p  \delta_{\Gamma p}  \bm\sigma_f^0 \cdot \textbf{n}_p 
\end{equation}
with the mixture stress defined as, 
\begin{equation}
    \bm\sigma^0_*
    =
    \chi_f \bm\sigma_f^0  
    +\zeta_b^{-1} (\chi_b \bm\sigma_b^0 + \delta_{\Gamma b} \bm\sigma_\Gamma^0)  
    +\zeta_p^{-1} \chi_p \bm\sigma_p^0 
\end{equation}
with, 
\begin{equation}
    \kappa_k 
    = \frac{1  - \zeta_k}{\zeta_k}
    = \zeta^{-1}_k - 1
\end{equation}
The average of this equation gives, 
\begin{multline}
    \rho_f (\pddt + \textbf{u} \cdot \grad)\textbf{u}
    + \div \avg{\rho_f \textbf{u}' \textbf{u}'}
    = 
    \rho_f \textbf{g}
    + \div \bm\sigma_*
    +\kappa_b  \avg{\delta_{\Gamma p} \bm\sigma_f' \cdot \textbf{n}_b} \\
    +\kappa_p  \avg{\delta_{\Gamma b}  \bm\sigma_f' \cdot \textbf{n}_p} 
    + (\kappa_p \phi_p + \kappa_b \phi_b)\div\bm\Sigma
\end{multline}
The \textit{Mean newtonian stress} and the mean effective stress is now defined as, 
\begin{align}
    \bm\sigma_* &= 
    \bm\Sigma 
    % - \phi_d \bm\Sigma 
    - \avg{2\mu_f \chi_{p} \textbf{e}_p'}- \avg{2\mu_f \chi_{b} \textbf{e}_d'}
    % +\kappa_b \avg{\chi_b \bm\Sigma}  
    % +\kappa_p \avg{\chi_p \bm\Sigma} 
    +\zeta_b^{-1} \avg{\chi_b \bm\sigma_b' + \delta_{\Gamma b} \bm\sigma_\Gamma^0}  
    +\zeta_p^{-1} \avg{\chi_p \bm\sigma_p'} \\
    \bm\Sigma
    &=
    -p_f\bm\delta
    + \mu_f (\grad \textbf{u} + ^\dagger\grad \textbf{u})
    =
    -p_f\bm\delta
    +2 \mu_f \textbf{E}
\end{align}
where $\textbf{e}_{p/d} = \textbf{e}_{p/d}^0 - \textbf{E}$. 


\section{Averaged equations bi-disperse suspension hybrid formulation}

Averaging the above equations gives directly, 
\paragraph*{Mass conservation : hybrid form}
\begin{align}
    m_b (\pddt + \textbf{u}_b \cdot \grad )n_b &= - n_b \div \textbf{u}_b\\
    m_p (\pddt + \textbf{u}_p \cdot \grad )n_p &= - n_p \div \textbf{u}_p\\
    \div \textbf{u}
    &= 
    0
\end{align}
\paragraph*{Momentum conservation: hybrid form}
\begin{align}
    n_b m_b (\pddt + \textbf{u}_b\cdot  \grad) \textbf{u}_b
    + \div \pavg[b]{ m_b \textbf{u}_b' \textbf{u}_b'}
    &= 
    m_b n_b \textbf{g}
    % + \pSavg[b]{\bm\sigma_f' \cdot \textbf{n}_b}
    % + \pOavg[b]{\div \bm\Sigma}
    + \pavg[b]{\textbf{M}^{(0)}_\alpha}
    \\
    n_p m_p (\pddt + \textbf{u}_p\cdot  \grad) \textbf{u}_p
    + \div \pavg{ m_p \textbf{u}_p' \textbf{u}_p'}
    &= 
    m_p n_p \textbf{g}
    + \pavg[b]{\textbf{M}^{(0)}_\alpha}
    \\
    \rho_f (\pddt + \textbf{u} \cdot \grad)\textbf{u}
    + \div \avg{\rho_f \textbf{u}' \textbf{u}'}
    &= 
    \div \bm\sigma_*
    + \rho_f \textbf{g}
    +\kappa_b   n_b \textbf{M}_b^{(1)}
    +\kappa_p   n_p \textbf{M}_p^{(1)}
\end{align}
In hybrid form the last equation reads, 
\begin{equation}
    \bm\sigma_* 
    =
    \bm\Sigma
    +  n_b \textbf{M}_b^{(1)}
    +  n_p \textbf{M}_p^{(1)}
    -\div (  n_b \textbf{M}_b^{(2)}
    +  n_p \textbf{M}_p^{(2)})
\end{equation}
with $n_{p/b} \textbf{M}_{p/b}^{(n)} = \pavg[p/b]{\textbf{M}_\alpha^{(n)}}$ and , 
\begin{align}
    \textbf{M}_\alpha^{(0)} &=
    \intS{\bm{\sigma}_f' \cdot \textbf{n}}
   +\intO{\div \bm\Sigma}
   \\
   \textbf{M}_\alpha^{(1)} &=
   \intS{\textbf{r}\bm{\sigma}_f' \cdot \textbf{n}}
   -2\mu_f \intO{\textbf{e}_d'}
   +\intO{\textbf{r}\div \bm\Sigma}
   \\
   \textbf{M}_\alpha^{(2)} &=
   \frac{1}{2}\intS{\textbf{rr}\bm{\sigma}_f' \cdot \textbf{n}}
   -2\mu_f \intO{\textbf{re}_d''}
   +\intO{\textbf{rr}(\div \bm\Sigma+ \rho_f\textbf{g})}
    \\
\end{align}
where $\alpha$ is either a bubble or a particles. 
In this formulaiton we have neglected the inertia in the higher moments equaitons this include angular momentum 

\paragraph*{Probability of attachments }


Now we need to determine $n_p$ such that we consider that the particles goes from particle to bubbles and basically despair, changing the density of the bubble but not the drag for now. 


We assume that the net rate of attachments of particles onto bubbles corresponds to the input flux of particles on the surface of teh (spherical) buubles, extended by the radius of the particles. 

The number density of particle on the surface of the bubbles might be expressed as, 
\begin{equation}
    n_b P_b 
    = 
    \avg{\delta(\textbf{x}-\textbf{x}_\alpha) \int_{\Omega_{b+p}} \delta(\textbf{x} + \textbf{r} - \textbf{x}_\beta) d\Omega(\textbf{r})}
    = 
    \int_{\Omega_{b+p}}\avg{\delta(\textbf{x}-\textbf{x}_\alpha)  \delta(\textbf{x} + \textbf{r} - \textbf{x}_\beta) }d\Omega(\textbf{r})
\end{equation}
where $\Omega_{p+b}$ correspond to the volume of the bubble + dadius of particles. 
We have assumed in the second equality that the shape of bubbles where exactly determined by center of mass. 
Let $\delta(\textbf{x} +\textbf{r}- \textbf{x}_\beta)$ and $\delta(\textbf{x} - \textbf{x}_\alpha)$ the dirac delta function tracking the center of mass of a given particle and bubbles respectively and then, 
\begin{align}
    \pddt (\delta_\alpha\delta_\beta)
    + \textbf{u}_\alpha \cdot \grad (\delta_\beta \delta_\alpha) 
    + \textbf{u}_{\beta\alpha} \cdot \pddr (\delta_\beta \delta_\alpha) = 0
\end{align}
averaging this 
\begin{align}
    \pddt P
    +  \div (\textbf{u}_p^{(2)} P) 
    +  \pddr\cdot (\textbf{u}_{pb}^{(2)} P) = 0
\end{align}
We state that $P(\textbf{x},\textbf{r}) = \avg{\delta_\alpha \delta_\beta}$, so that 
\begin{equation}
    \pddt (n_bP_b)
    +  \div (n_b P_b \textbf{u}_b^{(2)} )
    + \int_{\Gamma_{b+p}} P \textbf{u}_{pb}^{(2)}\cdot \textbf{n} d\Gamma(\textbf{r})
    = 0 
\end{equation}
Also, because 
\begin{equation}
    \pddt n_b + \div n_b \textbf{u}_b = 0 
\end{equation}
and because $\textbf{u}_b^{(2)} = \textbf{u}_b$ if we consider that bubbles velocity are uncorrelated with particles presence, we obtain ,
\begin{equation}
    n_p (\pddt 
    +  \textbf{u}_b \cdot \grad ) P_b
    + \int_{\Gamma_{b+p}} P \textbf{u}_{pb}^{(2)}\cdot \textbf{n} d\Gamma(\textbf{r})
    = 0 
\end{equation}

where $\textbf{u}_{pb} = \textbf{u}_p(\textbf{r},\textbf{x}) - \textbf{u}_b(\textbf{r},\textbf{x})$ is the velocity of the bubble center of mass minus the one of the particle averaged conditionally on the presence of both. 
With the Reynolds transport theorem we also arrive at the same thing uncomment below, however note that by doing one may eventually be able to consider bubbles deformations. 


We consider in the following that $\textbf{u}_b^{(2)} = \textbf{u}_b$ such that the mean bubble velocity isn't influenced so much by nearby particles, the inverse is not true. 

Assuming input bubbles are clean, and that $S_{out}$ is the only outflow surface of the tank of volume $V_f$ we may write, 
\begin{equation}
    \pddt \intOf{ n_bP_b}
    + \intS[out]{ n_b P_b \textbf{u}_b^{(2)} \cdot \textbf{n}_{out}}
    =  
    - \intOf{\int_{\Gamma_{b+p}} P \textbf{u}_{pb}^{(2)}\cdot \textbf{n} d\Gamma(\textbf{r})}
\end{equation}
In  the steady state regime, 
\begin{equation}
    \intS[out]{ n_b P_b \textbf{u}_b^{(2)} \cdot \textbf{n}_{out}}
    =  
    - \intOf{\int_{\Gamma_{b+p}} P \textbf{u}_{pb}^{(2)}\cdot \textbf{n} d\Gamma(\textbf{r})}
\end{equation}
where the left-hand side correspond exactly to the number of particles outflow in the domain. 


\section{Stokestain particles models }
We assume, force-free and torque free particles and bubbles. 
Additionally, we neglect in the first place the terms $\mathcal{O}(Re \phi_{p/b})$ and $\mathcal{O}(a^2/L^2)$ Assuming that $L$ is teh length sale of variation. 

\subsubsection*{Clean particles and droplets}

\begin{align}
    n_p \textbf{M}_b^{(0)}
    &=
    \phi_b
    \frac{\mu_f}{a^2}
    \frac{3(2+3\lambda_b)}{2(1+\lambda_b)}(\textbf{u} - \textbf{u}_b)
    + \phi_b\mu_f  \frac{3\lambda_b}{4(\lambda_b +1)} \grad^2 \textbf{u}
    + \phi_b \div\bm\Sigma
    \\
    n_b \textbf{M}_b^{(1)}
    &= \mu_f \phi_b 
    \frac{(5\lambda_b +2)}{(\lambda_b +1)}\textbf{E} 
    % + \phi_b a^2 \mu_f \frac{\lambda_b}{2(\lambda_b +1)}\grad^2 \textbf{E}
    % + \pOavg{\textbf{r}\div\bm\Sigma}
    \\
    n_b \textbf{M}_b^{(2)} 
    &=
    - \mu_f \phi_b \frac{3\lambda_b}{4(\lambda_b +1)}[\bm\delta \textbf{u}_r + \frac{1}{2\lambda}\textbf{u}_r \bm\delta ]
    % + \propto \grad\grad \textbf{u}
    % +\pOavg{\textbf{rr}(\div\bm\Sigma+\rho_f \textbf{g})}
\end{align}
For particles it is the same closures with $a \to \beta a$ if we neglect interactions. 
Although one can hardly neglect interaction for the drag on a single particle. 

In the most simple senario one may write,
\begin{align}
    % n_b m_b (\pddt + \textbf{u}_b\cdot  \grad) \textbf{u}_b
    % + \div \pavg[b]{ m_b \textbf{u}_b' \textbf{u}_b'}
    0
    &= 
    m_b n_b \textbf{g}
    % + \pSavg[b]{\bm\sigma_f' \cdot \textbf{n}_b}
    % + \pOavg[b]{\div \bm\Sigma}
    + \pavg[b]{\textbf{M}^{(0)}_\alpha}
    \\
    % n_p m_p (\pddt + \textbf{u}_p\cdot  \grad) \textbf{u}_p
    % + \div \pavg{ m_p \textbf{u}_p' \textbf{u}_p'}
    0
    &= 
    m_p n_p \textbf{g}
    + \pavg[b]{\textbf{M}^{(0)}_\alpha}
    \\
    \rho_f (\pddt + \textbf{u} \cdot \grad)\textbf{u}
    + \div \avg{\rho_f \textbf{u}' \textbf{u}'}
    &= 
    \div \bm\sigma_*
    + \rho_f \textbf{g}
    +\kappa_b   n_b \textbf{M}_b^{(1)}
    +\kappa_p   n_p \textbf{M}_p^{(1)}
\end{align}
\begin{align*}
    \bm\sigma_*  = 
    + \bm\Sigma
    + \mu_f \phi  \frac{(5\lambda +2)}{(\lambda +1)}\textbf{E} 
    + \mu_f \frac{3\lambda}{4(\lambda +1)}[\grad (\phi \textbf{u}_r)+\grad (\phi\textbf{u}_r)] 
    - \frac{3\lambda-2}{4(\lambda+1)}\div (\phi\textbf{u}_r) \bm\delta 
\end{align*}
where we assumed $\phi = \phi_p + \phi_b$ and $\textbf{u}_r\phi \to \phi_b(\textbf{u}-\textbf{u}_b ) + \phi_p (\textbf{u}- \textbf{u}_p)$

At this order of approximation $\mathcal{O}(\phi)$ note that it is possible by multiplying the above equation of the bulk  by $\phi$ to obtain, 
\begin{equation}
    0
    = 
    \phi \div \bm\Sigma
    + \phi \rho_f \textbf{g}
\end{equation}
Hence, in the $\textbf{M}^{(0)}$ one can simply replace $\phi \div \bm\Sigma$ by $-\rho_f \textbf{g}$. 
So that the mean velocity of bubbles is given by, 
\begin{equation}
    0
    =
    (\textbf{u} - \textbf{u}_b)
    +  a^2 \frac{\lambda_b}{2(2+3\lambda_b)} \grad^2 \textbf{u}
    + \frac{a^2}{\mu_f}\frac{2(\lambda_b +1 )}{3(2+3\lambda_b)} (\rho_b - \rho_f )\textbf{g}
\end{equation}


\subsubsection{The most simple formulaiton for the rate of attachment}
Using the closure terms and the momentum balance law we obtain, 
\begin{align}
    \textbf{u}_p^{(2)} 
    &=
    \frac{(a\beta)^2}{\mu_f}
    \frac{2(1+\lambda_p)}{3(2+3\lambda_p)}\Delta\rho_p\textbf{g}
    + \textbf{u}^{(1)}
    + (a\beta)^2 \frac{\lambda_p}{2(3\lambda_p +2)} \grad^2 \textbf{u}^{(1)}\\
    \textbf{u}_b^{(2)} 
    &=
    \frac{a^2}{\mu_f}
    \frac{2(1+\lambda)}{3(2+3\lambda)}\Delta\rho_b\textbf{g}
    + \textbf{u}^{(1)}
    + a^2 \frac{\lambda}{2(3\lambda +2)} \grad^2 \textbf{u}^{(1)}
\end{align}
where $\Delta\rho_k = \rho_k - \rho_f$ $\textbf{u}^{(1)}$ is the bulk fluid velocity at the position of the particle (resp. bubble) averaged conditionally, on the presence of the bubble (resp. particle). 
Because the presnce of the particle does not affect the bubble motion one may replace $\textbf{u}^{(1)} \to \textbf{u}$ in the second formula.
If one does not use this approximation it is possible go further using the method of reflection, with more than one reflection.  

Substracting both formulation gives, 
\begin{align}
    \textbf{u}_p^{(2)} - \textbf{u}_b^{(2)}
    &=
    \textbf{u}^{(1)} - \textbf{u}
    + (a\beta)^2 \frac{\lambda_p}{2(3\lambda_p +2)} \grad^2 (\textbf{u}^{(1)} - \textbf{u})\\
    &+
    \left[\beta^2\frac{(1+\lambda_p)}{(2+3\lambda_p)}\Delta\rho_p  -\frac{(1+\lambda)}{(2+3\lambda)}\Delta\rho_b  \right] \frac{2a^2}{3\mu_f}\textbf{g}\\
    &+ \left[\beta^2\frac{\lambda_p}{3\lambda_p +2} - \frac{\lambda}{3\lambda +2} \right] \frac{a^2}{2}\grad^2 \textbf{u}
\end{align}
The second term correspond to the velocity generated by the difference in buoyancy force. 
The third term is related to the differences of velocity generated by the mean background quadratic field. 

The first term represents the action of the bubble on the particle, i.e. it is the drag force on the particle due to the disturbance field generated by the bubble.
the disturbance field generated by the bubble is, 
\begin{align*}
    \textbf{u}^{(1)} - \textbf{u}
    &=
    \frac{3}{4}\left[
        \frac{(2+3\lambda)}{3(1+\lambda)}
        + 
        a^2\frac{\lambda\grad^2}{6(\lambda+1)}
        \right]\mathcal{G}(\textbf{r})\cdot
    (\textbf{u}- \textbf{u}_b)\\
    % p_f^{(1)} - p_f
    % &=
    % -\frac{3\lambda_b+2}{2(\lambda_b+1)}r^{-2} \textbf{n}\cdot (\textbf{u}- \textbf{u}_b)\\
    \mathcal{G}(\textbf{r})
    &=
    a r^{-1}(\bm\delta + \textbf{nn})
\end{align*}
Note that since it is the disturbance force that generate the flow we may replace 
,
\begin{equation}
    \textbf{u}-\textbf{u}_b \to - \frac{a^2 \Delta \rho_b}{\mu_f}
    \frac{2(1+\lambda)}{3(2+3\lambda)}\textbf{g}
\end{equation}
Then one may show that, 
\begin{align}
    (\textbf{u}^{(1)} - \textbf{u})\cdot\textbf{n} &= 
    - \frac{a^2 \Delta \rho_b}{\mu_f}
    \frac{2(1+\lambda)}{3(2+3\lambda)}
    f_u(\lambda,r) \textbf{n}\cdot \textbf{g}\\
    (a\beta)^2 \frac{\lambda_p}{2(3\lambda_p +2)} \grad^2 (\textbf{u}^{(1)} - \textbf{u})\cdot \textbf{n}
    &=
    - \frac{a^2 \Delta \rho_b}{\mu_f}
    \frac{2(1+\lambda)}{3(2+3\lambda)}
    f_{\grad^2 u}(\lambda,\lambda_p,\beta,r) \textbf{n}\cdot \textbf{g}
\end{align}
with,
\begin{align}
    f_u = - \frac{\lambda \left(3 r^{2} - 1\right) + 2 r^{2}}{2 r^{3} \left(\lambda + 1\right)}
    &&
    f_{\grad^2 u } 
    =
    \frac{\beta^{2} \lambda_{p} \left(3 \lambda + 2\right)}{2 r^{3} \left(\lambda + 1\right) \left(3 \lambda_{p} + 2\right)}
\end{align}
\tb{at this point note that the flow generated by the mean quadratic contribution might not be negligible at all since $r^{-1}$.}
with $f$ defined below. 
The final result for the relative velocity reads, 
% \begin{align}
%     (\textbf{u}_p^{(2)} - \textbf{u}_b^{(2)})\cdot \textbf{n}
%     &=
%     - \frac{a^2 \Delta \rho_b}{\mu_f}
%     \frac{2(1+\lambda)}{3(2+3\lambda)}
%     f(\lambda_b,\lambda_p,\beta,r) \textbf{n}\cdot \textbf{g}\\
%     &+
%     \left[\beta^2\frac{(1+\lambda_p)}{(2+3\lambda_p)}\Delta\rho_p  -\frac{(1+\lambda)}{(2+3\lambda)}\Delta\rho_b  \right] \frac{2a^2}{3\mu_f}\textbf{g}\cdot \textbf{n}\\
%     &+ \left[\beta^2\frac{\lambda_p}{3\lambda_p +2} - \frac{\lambda}{3\lambda +2} \right] \frac{a^2}{2}\grad^2 \textbf{u}\cdot \textbf{n}
% \end{align}
% which may be written, 
\begin{align}
    (\textbf{u}_p^{(2)} - \textbf{u}_b^{(2)})\cdot \textbf{n}
    &=
    \left[\beta^2\frac{(1+\lambda_p)(2+3\lambda)}{(1+\lambda)(2+3\lambda_p)}\frac{\Delta\rho_p}{\Delta\rho_b}  - (1 + f)  \right]  \frac{2(1+\lambda)}{3(2+3\lambda)} \frac{\Delta\rho_b a^2}{\mu_f}\textbf{g}\cdot \textbf{n}\\
    &+ \left[\beta^2\frac{\lambda_p(3\lambda+2)}{\lambda(3\lambda_p +2)} - 1 \right] a^2\frac{\lambda}{2(3\lambda +2)}\grad^2 \textbf{u}\cdot \textbf{n}
\end{align}

This relation is to be integrated at the surface of the droplet, since we want only positive $\textbf{u}_{pb}\cdot\textbf{n}$ and that $\textbf{u}_{bp}$ si only dependent on the azimuthal and polar angle via $\textbf{n}$, 
\tb{really have to clear sign}
\begin{equation}
    (\textbf{n} \cdot \textbf{u}_{pb})^-
    = 
     (C_1 \textbf{n} \cdot \textbf{g})^-
    \text{ or }
    (C_2 \textbf{n} \cdot \grad^2\textbf{u})^-
    =
    |C_1 \textbf{g} \text{ or } C_2 \grad^2\textbf{u} |
    (C_1 \textbf{n} \cdot \textbf{p})^-
\end{equation}
where \textbf{p} is a constant unit vector defined as, $\textbf{p} = \textbf{g}C_1 / |\textbf{g} C_1|$. In the reference frame of \textbf{p} one can compute the integral,  
\begin{equation}
    \int_{r=a(1+\beta)} (\textbf{n} \cdot \textbf{p})^-dS
    = a^2(1+\beta)^2
    \iint (\cos \theta)^- \sin\theta d\theta d\varphi
    = - \pi a^2(1+\beta)^2
\end{equation}
In fact whether the particle approach or not the bubble because there is particles all around it always get into the surface $a^2 \pi$. 

The final results reads, 
\begin{multline*} 
    \int_{r=a_b+a_p } 
    P(\textbf{x},\textbf{r}) (\textbf{u}_{pb}^{(2)}\cdot \textbf{n})^- dS(\textbf{r})
    =
    - n_pn_b \pi a^2(1+\beta)^2 \\\left|
        \left[\beta^2\frac{(1+\lambda_p)(2+3\lambda)}{(1+\lambda)(2+3\lambda_p)}\frac{\Delta\rho_p}{\Delta\rho_b}  - (1 + f_{r=\beta+1})  \right]  \frac{2(1+\lambda)}{3(2+3\lambda)} \frac{\Delta\rho_b a^2}{\mu_f}\textbf{g} \right.\\\left.
        + \left[\beta^2\frac{\lambda_p(3\lambda+2)}{\lambda(3\lambda_p +2)} - 1 \right] \frac{\lambda}{2(3\lambda +2)}a^2 \grad^2 \textbf{u}
    \right|
\end{multline*} 
If we consider that $\textbf{u}_b^*$ is the buoyancy of a droplet in and otherwise steady flow and $S = a^2 \pi (1+\beta)^2 n_p $ the number of droplet per unit of  length, then the product $S \textbf{u}_b^*$ is the flow rate of incident particles. 
The factor $f$ represent the correction due to the hydrodynamic. 
Likewise, $a^2 |\grad^2 \textbf{u}| S$ represent the flow rate of particle on the droplet effective surface. 



Integrating this formula over the whole tank volume gives the same formulation except that $n_bn_p \to N_p N_b$ by definition and $\grad^2 \textbf{u} \to \avg{\grad^2 \textbf{u}}_m$ 



\subsubsection{Formulaiton In terms of an unknown $\textbf{u}_b$}
Let consider the particle transport equation conditioned on a nearby bubble to determine $\textbf{u}_{pb}^{(2)}$. 
Assuming no Brownian motions, and Stokestain  particles, we can deduce that the velocity of an isolated particle is given by, 
\begin{equation}
    \phi_p
    \frac{\mu_f}{a_p^2}
    \frac{3(2+3\lambda_p)}{2(1+\lambda_p)}\textbf{u}_p^{(2)}
    =
    \phi_p( \rho_p\textbf{g}+ \div\bm\Sigma^{(1)} )
    +
    \phi_p\frac{\mu_f}{a_p^2}
    \frac{3(2+3\lambda_p)}{2(1+\lambda_p)}\textbf{u}^{(1)} 
    + \phi_p\mu_f  \frac{3\lambda_p}{4(\lambda_p +1)} \grad^2 \textbf{u}^{(1)}
\end{equation}
Hence the relative velocity is given by, 
\begin{equation}
    \textbf{u}_p^{(2)} - \textbf{u}_b
    =
    \frac{a_p^2}{\mu_f}
    \frac{2(1+\lambda_p)}{3(2+3\lambda_p)}( \rho_p\textbf{g}+ \div\bm\Sigma^{(1)} )
    + (\textbf{u}^{(1)} - \textbf{u}_b)
    + a_p^2 \frac{\lambda_p}{2(3\lambda_p +2)} \grad^2 \textbf{u}^{(1)}
\end{equation}

where $\textbf{u}^{(1)}$ is the bluk flow velocity at \textbf{r} (particle location) conditioned on that there is a bubble at \textbf{x} going with velocity $\textbf{u}_b$. 
This is equivalent to consider only one reflection. 
Using classic result from the literature we find that the disturbance velocity field generated by a translating bubble is, 
\begin{align*}
    \textbf{u}^{(1)} - \textbf{u}
    &=
    \frac{3}{4}\left[
        \frac{(2+3\lambda_b)}{3(1+\lambda_b)}
        + 
        \frac{\lambda_b\grad^2}{6(\lambda_b+1)}
        \right]\mathcal{G}(\textbf{r})\cdot
    (\textbf{u}- \textbf{u}_b)\\
    p_f^{(1)} - p_f
    &=
    -\frac{3\lambda_b+2}{2(\lambda_b+1)}r^{-2} \textbf{n}\cdot (\textbf{u}- \textbf{u}_b)\\
    \mathcal{G}(\textbf{r})
    &=
    r^{-1}(\bm\delta + \textbf{nn})
\end{align*}
Note that the stress is made dimensionless as $p_f  a / (U \mu_f)$ and the distance with $a$

Computing the above terms, and projecting on the normal gives, 
\begin{align}
    (\textbf{u}_p^{(2)} -\textbf{u}_b)\cdot \textbf{n}
    &=
    \frac{a_p^2}{\mu_f}
    \frac{2(1+\lambda_p)}{3(2+3\lambda_p)}( \rho_p\textbf{g}+ \div\bm\Sigma)\cdot\textbf{n}
    + a_p^2 \frac{\lambda_p}{2(3\lambda_p +2)} \grad^2 \textbf{u} \cdot\textbf{n}
    + f(\lambda_b,\lambda_p,\beta,r) \textbf{n}\cdot (\textbf{u} - \textbf{u}_b)
\end{align}
with, 
\begin{multline}
    f(\lambda_b,\lambda_p=\infty,r,\beta) =\\
    \frac{16 \beta^{2} r^{2} + \lambda \lambda_{p} \left(27 \beta^{2} r^{2} - 40 \beta^{2} + 6 r^{5} - 9 r^{4} + 3 r^{2}\right) + \lambda \left(24 \beta^{2} r^{2} - 40 \beta^{2} + 4 r^{5} - 6 r^{4} + 2 r^{2}\right)}{2 r^{5} \left(\lambda + 1\right) \left(3 \lambda_{p} + 2\right)}\\
    +
    \frac{ \lambda_{p} \left(18 \beta^{2} r^{2} + 6 r^{5} - 6 r^{4}\right) + 4 r^{5} - 4 r^{4}}{2 r^{5} \left(\lambda + 1\right) \left(3 \lambda_{p} + 2\right)}
\end{multline}
The function $f$ can be computed analytically and turns out being rather complicated, 
the mean quantity can be considered not varying too much at the length scale of the buuble over which we performed the integrals. 
Therefore the only terms, which depends on the angles are the $\textbf{n}$'s must be integrated however since we want to keep all positives component with respect to an other unit vector, let say \textbf{p} (can be $\grad^2 \textbf{u}$ or $\textbf{g}$ or $\textbf{u}-\textbf{u}_b$) we obtain 
\begin{equation}
    \int_{r=a(1+\beta)} (\textbf{n} \cdot \textbf{p})dS
    = a^2(1+\beta)^2
    \iint (\textbf{n} \cdot \textbf{p})^- \sin\theta d\theta d\varphi
\end{equation}
Because the surface of the droplet is isotropic and that we integrate overall the surface of the sphere the direction of \textbf{p} which depends on the situation is quite arbitrary, hence let $\textbf{p} = \textbf{e}_z$, 
\begin{equation}
    \int_{r=a(1+\beta)} (\textbf{n} \cdot \textbf{p})dS
    = a^2(1+\beta)^2
    \iint (\cos \theta)^- \sin\theta d\theta d\varphi
    = \pi a^2(1+\beta)^2
\end{equation}
So basically this represents half the surface area because $\textbf{p}\cdot \textbf{n}$ conserve the same sign up to an angle of $\pi/2$ in all case, which represent half of that. 


The rate of attachment is then given as, 
\begin{multline*} 
    \int_{r=a_b+a_p } 
    P(\textbf{x},\textbf{r}) (\textbf{u}_{pb}^{(2)}\cdot \textbf{n})^- dS(\textbf{r})
    =
    n_pn_b \pi a^2(1+\beta)^2 [
        \frac{a_p^2}{\mu_f}
        \frac{2(1+\lambda_p)}{3(2+3\lambda_p)}( \rho_p|\textbf{g}|+ |\div\bm\Sigma|)\\
        + a_p^2 \frac{\lambda_p}{2(3\lambda_p +2)} |\grad^2 \textbf{u}|
        + f(\lambda_b,\lambda_p,\beta,1+\beta)  |\textbf{u} - \textbf{u}_b|
    ]
\end{multline*}
with, 
\begin{multline}
    f(\beta, \lambda_b, \lambda_p)
    =
    \frac{- 40 \beta^{2} \lambda \left(\lambda_{p} + 1\right) + 2 \left(\beta + 1\right)^{5} \left(3 \lambda \lambda_{p} + 2 \lambda + 3 \lambda_{p} + 2\right) - \left(\beta + 1\right)^{4} \left(9 \lambda \lambda_{p} + 6 \lambda + 6 \lambda_{p} + 4\right) }{2 \left(\beta + 1\right)^{5} \left(3 \lambda \lambda_{p} + 2 \lambda + 3 \lambda_{p} + 2\right)}\\
    + \frac{ \left(\beta + 1\right)^{2} \left(27 \beta^{2} \lambda \lambda_{p} + 24 \beta^{2} \lambda + 18 \beta^{2} \lambda_{p} + 16 \beta^{2} + 3 \lambda \lambda_{p} + 2 \lambda\right)}{2 \left(\beta + 1\right)^{5} \left(3 \lambda \lambda_{p} + 2 \lambda + 3 \lambda_{p} + 2\right)}
\end{multline}
where we have assumed that $P(\textbf{x},\textbf{r}) = n_p(\textbf{r})n_b(\textbf{x})$ which might be far from the real cases, this may be determined using the transport equation of the pdf. 

\bibliography{Bib/bib_bulles.bib}
\appendix

\end{document}


\documentclass[12pt]{My_preprint}
\title{
    Statistical modeling of the \textit{Flotation} process
    }

\author[1,2]{Nicolas Fintzi}
\normalmarginpar


\begin{document}

\maketitle

\begin{abstract}
\end{abstract}

\section{Three-phase system}

We consider a three-phase system made of small particles (pollutant) and rising bubbles dispersed into a continuous phase. 

\begin{table}[h!]
    \centering
    \begin{tabular}{cccccc}
        & Continuous phase  & Bubbles   & Particles & ratio $b$-$f$                       & ratio $p$-$f$\\
        Density         & $\rho_f$          & $\rho_b$  & $\rho_p$  & $-$&$-$ \\
        Viscosity       & $\mu_f$           & $\mu_b$   & $\infty$ &  $-$&$-$ \\
        % Volume fraction & $\phi_f$          & $\phi_b$  & $\phi_p$  &                              & \\
        % number density  &              & $n_b$     & $n_p$     &                           & \\
        Denisty ratio   &  $-$  &  $-$   &  $-$    & $\kappa_{bf} = \rho_b/\rho_f$     & $\zeta_{pf} = \rho_p/\rho_f$\\
        Viscosity ratio &  $-$  &  $-$   &  $-$    & $\lambda_{bf} = \mu_b/\mu_f$      & $\lambda_{pf} = \rho_p/\rho_f$\\
        Velocity ($\textbf{u}_k^0$) &    $\textbf{u}_f^0$        &  $\textbf{u}_b^0$   &  $\textbf{u}_p^0$    &    $-$   & $-$\\
        Pressure ($p_k^0$) &    $p_f$ &  $p_b$   &   $-$&  $-$& $-$\\
        Stress ($\bm\tau_k^0$) &    $\mu_f (\grad \textbf{u}_f^0 + \grad \textbf{u}_f^0) $        & $\mu_b (\grad \textbf{u}_b^0 + \grad \textbf{u}_b^0) $    & $\mathbb{C}:\textbf{E}$&  $-$     &$-$ \\
        Outward normal &     $-$       &   $-$  &  $-$    & $\textbf{n}_b$      & $\textbf{n}_p$\\
        Surface stress ($\bm\sigma_\Gamma$) &   $-$         &  $-$   &   $-$   & $\gamma  (\bm\delta -\textbf{n}_b\textbf{n}_b)$    &$-$ \\
        PIF ($\chi_k$ and $\delta_{\Gamma}$) &$\chi_f$& $\chi_b$    &  $\chi_p$    &  $\delta_{\Gamma b}$  & $\delta_{\Gamma p}$  \\
        PIF ($\delta_k$) &$-$& $\delta_b$    &  $\delta_p$    & $-$  & $-$ 
    \end{tabular}
    \caption{Phases properties and constitutive laws.}
\end{table}


\subsection{Local scale equations}

\begin{table}[h!]  
    \centering
    \begin{tabular}{ccl}
     & Mass & Momentum \\
    conserved quantity & $\rho_k$ & $\rho_k \textbf{u}_k^0$ \\
    source term  & $0$ & $\rho_k \textbf{g}$ \\
    diffusive flux & 0 & $\bm\sigma_k^0 = -p_k^0 \bm\delta + \mu_k (\grad \textbf{u}_k^0 + \grad \textbf{u}_k^0)$ \\
    interface quantity & $0$ & $0$ \\
    surface diffusive flux & 0 & $\bm\sigma_\Gamma^0 = \gamma (\bm\delta - \textbf{n}_b\textbf{n}_b)$ \\
    \end{tabular}
    \caption{Definition of the physical quantities, and local constitutive laws}
    \label{tab:qte_Newtonian}
\end{table}
The general form of the volume/interface conservation equation reads as, 
\begin{align}
    \pddt (\chi_k f_k^0)
    + \div (
        \chi_k f_k^0 \textbf{u}_k^0
        - \chi_k \mathbf{\Phi}_k^0 
        )
    &= 
    \chi_k s_k^0
    + \delta_\Gamma \left[
        f_k^0
        \left(
            \textbf{u}_\Gamma^0
            - \textbf{u}_k^0
        \right)
        + \mathbf{\Phi}_k^0
    \right]
    \cdot \textbf{n}_k.
    \label{eq:dt_chi_k_f_k}\\
    \pddt (\delta_\Gamma f_\Gamma^0)  
    + \div (
        \delta_\Gamma  f_\Gamma^0 \textbf{u}_\Gamma^0
        - \delta_\Gamma  \mathbf{\Phi}_{\Gamma||}^0 
        )
    &= 
    \delta_\Gamma s_\Gamma^0
    - \delta_\Gamma \Jump{
    f_k^0 (\textbf{u}_\Gamma^0 - \textbf{u}_k^0)
    + \mathbf{\Phi}_k^0},
    \label{eq:dt_delta_I_f_I}
\end{align}


Thus, the system bulk-bubbles-particles might be described by 6 equaitons, 
\begin{align}
    \pddt (\chi_b \rho_b)
    + \div 
        \chi_b \rho_b \textbf{u}_b^0
    &= 
    0\\
    \pddt (\chi_p \rho_p)
    + \div 
        \chi_p \rho_p \textbf{u}_p^0
    &= 
    0\\
    \div \textbf{u}^0
    &= 
    0
\end{align}
\begin{align}
    \pddt (\chi_b \rho_b \textbf{u}^0_k)
    + \div (
        \chi_b \rho_b \textbf{u}^0_b \textbf{u}_b^0
        )
    &= 
    \chi_b \rho_b \textbf{g}
    + \div (\chi_b \bm\sigma_b^0  + \delta_{\Gamma b} \bm\sigma_\Gamma^0 )
    + \delta_{\Gamma b}  \bm\sigma_f^0 \cdot \textbf{n}_b\\
    \pddt (\chi_p \rho_p \textbf{u}^0_k)
    + \div (
        \chi_p \rho_p \textbf{u}^0_p \textbf{u}_p^0
        )
    &= 
    \chi_p \rho_p \textbf{g}
    + \div \chi_p \bm\sigma_p^0  
    + \delta_{\Gamma b}  \bm\sigma_f^0 \cdot \textbf{n}_p
\end{align}
\begin{equation}
    \rho_f (\pddt + \textbf{u}^0 \cdot \grad)\textbf{u}^0
    = 
    \rho_f \textbf{g}
    + \div \bm\sigma^0_*
    +\kappa_b  \delta_{\Gamma b}  \bm\sigma_f^0 \cdot \textbf{n}_b 
    +\kappa_p  \delta_{\Gamma p}  \bm\sigma_f^0 \cdot \textbf{n}_p 
\end{equation}
with the mixture stress defined as, 
\begin{equation}
    \bm\sigma^0_*
    =
    \chi_f \bm\sigma_f^0  
    +\zeta_b^{-1} (\chi_b \bm\sigma_b^0 + \delta_{\Gamma b} \bm\sigma_\Gamma^0)  
    +\zeta_p^{-1} \chi_p \bm\sigma_p^0 
\end{equation}
with, 
\begin{equation}
    \kappa_k = \frac{1  - \zeta_k}{\zeta_k}
\end{equation}


\section{Averaged equations for moo-dispersed suspension}

Averaging the above equations gives directly, 
\paragraph*{Mass conservation : hybrid form}
\begin{align}
    m_b (\pddt + \textbf{u}_b \cdot \grad )n_b &= - n_b \div \textbf{u}_b\\
    m_p (\pddt + \textbf{u}_p \cdot \grad )n_p &= - n_p \div \textbf{u}_p\\
    \div \textbf{u}
    &= 
    0
\end{align}
\paragraph*{Momentum conservation: hybrid form}
\begin{align}
    n_b m_b (\pddt + \textbf{u}_b\cdot  \grad) \textbf{u}_b
    + \div \pavg[b]{ m_b \textbf{u}_b' \textbf{u}_b'}
    &= 
    m_b n_b \textbf{g}
    + \pSavg[b]{\bm\sigma_f^0 \cdot \textbf{n}_b}\\
    n_p m_p (\pddt + \textbf{u}_p\cdot  \grad) \textbf{u}_p
    + \div \pavg{ m_p \textbf{u}_p' \textbf{u}_p'}
    &= 
    m_p n_p \textbf{g}
    + \pSavg{\bm\sigma_f^0 \cdot \textbf{n}_p}\\
    \rho_f (\pddt + \textbf{u} \cdot \grad)\textbf{u}
    + \div \avg{\rho_f \textbf{u}' \textbf{u}'}
    &= 
    \rho_f \textbf{g}
    + \div (\bm\Sigma + \bm\sigma_*)
    +\kappa_b  \avg{\delta_{\Gamma p} \bm\sigma_f^0 \cdot \textbf{n}_b} 
    +\kappa_p  \avg{\delta_{\Gamma b}  \bm\sigma_f^0 \cdot \textbf{n}_p} 
\end{align}
The \textit{Mean newtonian stress} and the mean effective stress is now defined as, 
\begin{align}
    \bm\Sigma
    &=
    -p_f\bm\delta
    + \mu_f (\grad \textbf{u} + ^\dagger\grad \textbf{u})\\
    \bm\sigma_*
    &=
    % -p_f\bm\delta
    % + \mu_f (\grad \textbf{u} + ^\dagger\grad \textbf{u})
     (\phi_b + \phi_p) (p_f \bm\delta - 2 \mu_f \textbf{e}_b)
    +\zeta_b^{-1} \avg{\chi_b \bm\sigma_b^0 + \delta_{\Gamma b} \bm\sigma_\Gamma^0}  
    +\zeta_p^{-1} \avg{\chi_p \bm\sigma_p^0} 
\end{align}
Note that we consider no shear inside the solid particles $\textbf{e}_p = 0$. 

\section{Methodology to identify/quantify attached particles}

We do not use the classification method. 
Instead, we try to find closure for the particles' velocity dispersion as well as the drag force. 

\subsection{Drag force modeling}
\begin{equation}
    \pSavg{\bm\sigma_f^0 \cdot \textbf{n}_p}
    =
    \int_{S_p}
    \pavg{\bm\sigma_f' }
    \cdot \textbf{n}_p
    d\textbf{r}
    +
    \int_{S_p}
    \pavg{\bm\Sigma}
    \cdot \textbf{n}_p
    d\textbf{r}
\end{equation}
where $\pavg{\bm\sigma_f'} = \pavg{[-p_f'\bm\delta + \mu_f (\grad \textbf{u}'+^\dagger \grad \textbf{u}')]}$, with $\textbf{u}' = \textbf{u}^0 - \textbf{u}$ is one particle conditionally averaged stress. 
We introduce $\delta_1' = \delta_p - n_p$ such that the disturbance velocity $\textbf{v}_p = \avg{\delta_1' \textbf{u}^0}$. 
Precisely, 
\begin{equation}
    \delta_1'[\textbf{z}_1,\FF,t] 
    = 
    \sum_i^{N_p}\delta(\textbf{z}_1 - \textbf{x}_i[\FF,t]) - 
    \avg{\sum_i^{N_p}\delta(\textbf{z}_1 - \textbf{x}_i[\FF,t])}
\end{equation}

The conditionally averaged equation in Stokes regime might be written, 
\begin{align}
    n_p \div \textbf{v} &= 0 \\
    \div \bm\Sigma^1
    &= 
    + \div \bm\sigma^1_*
    +\kappa_b  \avg{\delta_1' \delta_{\Gamma b}  \bm\sigma_f^0 \cdot \textbf{n}_b }
    +\kappa_p  \avg{\delta_1' \delta_{\Gamma p}  \bm\sigma_f^0 \cdot \textbf{n}_p }\\
    \bm\Sigma^1
    &=
    -n_p p_f^1\bm\delta
    + \mu_f n_p (\grad \textbf{v}+^\dagger \grad \textbf{v})\\
    \bm\sigma^1_*
    &=
    + n_p (\phi_d^1 p_f^1 - \phi_d p_f )
    +\zeta_b^{-1} \avg{\delta_1' (\chi_b \bm\sigma_b^0 + \delta_{\Gamma b} \bm\sigma_\Gamma^0 - 2\mu_f \textbf{e}_b^0 )  }
    +\zeta_p^{-1} \avg{\delta_1' \chi_p \bm\sigma_p^0 }
\end{align}
If we neglecte the $\mathcal{O}(\phi_p^2,\phi_b^2)$ and $\mathcal{O}(\phi_p\phi_b)$ we may stipulate that,
\begin{align*}
    \avg{\delta'_1 \delta_{b}}
    &= \mathcal{O}(n_p n_b)\\
    \avg{\delta'_1 \delta_{p}}
    &=
    \delta(\textbf{z}_1 - \textbf{x}) n_p
    + \mathcal{O}(n_p^2)
\end{align*} 

Because of this simple observation one may re-write the expression out of the domain $|\textbf{z}_1 - \textbf{x}| >a_p$ of the particles as, 
\begin{align}
    \div \textbf{v} &= 0 \\
    - \grad p_f'
    + \mu_f \grad^2 \textbf{v}
    &= 0 
\end{align}
which becomes the classical Stokes equation. 
Hence the force traction on the particles can be computed the usual way as, 
\begin{align}
    \pSavg{\bm\sigma_f'\cdot \textbf{n}} &
    =
    \phi_p
    \frac{\mu_f}{a_p^2}
    \frac{3(2+3\lambda)}{2(1+\lambda)}\textbf{u}_r
    + \phi_p\mu_f  \frac{3\lambda}{4(\lambda +1)} \grad^2 \textbf{u}
    \\
    \pSavg{\textbf{r}\bm\sigma_f'\cdot \textbf{n}} &
    = \mu_f \phi_p 
    \frac{3(5\lambda +2)}{10(\lambda +1)}[\grad \textbf{u}+ (\grad \textbf{u})^\dagger]
    \\
    \pSavg{\textbf{rr}\bm\sigma_f'\cdot \textbf{n}} &
    =
    \mu_f \phi_p \frac{3}{5(\lambda +1)} (\textbf{u}_r \bm\delta + \bm\delta \textbf{u}_r)
    + \mu_f \phi_p \frac{3(5\lambda +2)}{10(\lambda+1)}\bm\delta \textbf{u}_r
\end{align}
The same Methodology can be adopted for the droplets except that in this case the internal shear will be needed as well. 

\subsection{Velocity variance modeling}

We assert that the interaction particle-bubbles is predominant.
Hence, the good way to measure the particles' velocity variance is given by 
\begin{align}
    \pavg{\textbf{u}_i'\textbf{u}_i'}
    &=
    \int_{\mathbb{R}^3}
    \pavg{\textbf{u}_i'\textbf{u}_i' \sum_j^{N_b} \delta(\textbf{x} + \textbf{r} - \textbf{x}_j[\FF,t])h_j[\textbf{x},\FF,t]}
    d \textbf{r}\\
    &=
    \int_{\mathbb{R}^3}
    \textbf{v}_p^\text{b-nst}
    \textbf{v}_p^\text{b-nst}
    P_\text{b-nst}[\textbf{x},\textbf{r}]
    d \textbf{r}\\
    &+
    \int_{\mathbb{R}^3}
    \pavg{\textbf{u}_i''\textbf{u}_i'' \sum_j^{N_b} \delta(\textbf{x} + \textbf{r} - \textbf{x}_j[\FF,t]h_j[\textbf{x},\FF,t])}
    d \textbf{r}
\end{align}
because,
\begin{equation}
    \int \sum_j^{N_b} \delta(\textbf{r} - \textbf{x}_j[\FF,t])h_j[\FF,t] d\textbf{r} = 1
\end{equation}
The second term is clearly too trick to compute however the first one is okay. 
Indeed, because, 
\begin{equation}
    \textbf{v}_p^\text{b-nst}
    P_\text{b-nst}[\textbf{x},\textbf{r}]
    =
    \pavg{(\textbf{u}_i - \textbf{u}_p) \delta_p\sum_j^{N_b} \delta(\textbf{x} + \textbf{r} - \textbf{x}_j)h_j}
\end{equation}
Or more simply
\begin{equation}
    \textbf{v}_p^\text{b-nst}
    P_\text{b-nst}[\textbf{x},\textbf{r}]
    =
    \avg{(\textbf{u}_i - \textbf{u}_p)\delta_\text{b-nst}}
\end{equation}


To compute this term one must use the particle balance equations, but conditioned by that distribution, it  reads, in steady state, and by neglecting the left-hand side, 
\begin{equation}
    - \pSavg{\delta_\text{b-nst}\bm\sigma' \cdot \textbf{n}_p}
    = 
    + \pSavg{\delta_\text{b-nst}\bm\Sigma \cdot \textbf{n}_p}
    + \avg{m_p \textbf{g} \delta_\text{b-nst}}
\end{equation}
Hence the drag force is either given by the constant Buoyancy force applied on the particle plus the `mean' hydrodynamic drag (pressure gradient etc), or by the hydrodynamic one, the sum of the two gives 0. 
Thus, if one obtain an expression for the left hand side in term of the bubbles and particles velocity, he may obtain $\textbf{u}_{p}^\text{b-nst}$ using this eq. 

To find teh drag closure one must compute the field,
\begin{equation}
    \textbf{v}^\text{pb-nst}
    =
    \avg{\textbf{u}^0 \delta_\text{pb-nst'}}
    =
    \avg{\textbf{u}^0 (\delta_p \delta_\text{b-nst} - \avg{\delta_p\delta_\text{b-nst}})}
\end{equation}

\begin{align}
    n_p \div \textbf{v} &= 0 \\
    \div \bm\Sigma^\text{nst}
    &= 
    + \div \bm\sigma^\text{nst}_*
    +\kappa_b  \avg{\delta_\text{pb-nst'} \delta_{\Gamma b}  \bm\sigma_f^0 \cdot \textbf{n}_b }
    +\kappa_p  \avg{\delta_\text{pb-nst'} \delta_{\Gamma p}  \bm\sigma_f^0 \cdot \textbf{n}_p }\\
    \bm\Sigma^1
    &=
    -n_p p_f^1\bm\delta
    + \mu_f n_p (\grad \textbf{v}+^\dagger \grad \textbf{v})\\
    \bm\sigma^1_*
    &=
    + n_p (\phi_d^1 p_f^1 - \phi_d p_f )
    +\zeta_b^{-1} \avg{\delta_1' (\chi_b \bm\sigma_b^0 + \delta_{\Gamma b} \bm\sigma_\Gamma^0 - 2\mu_f \textbf{e}_b^0 )  }
    +\zeta_p^{-1} \avg{\delta_1' \chi_p \bm\sigma_p^0 }
\end{align}

\bibliography{Bib/bib_bulles.bib}
\appendix

\end{document}



\section{Slightly inertial suspension Rheology}\label{sec:averaged_equations}
Before presenting the averaged system of equations~\eqref{eq:dt_phif,eq:div_u,eq:dt_phip,eq:dt_up,eq:dt_uf2} under a (nearly) closed form, we start by presenting the forces, first moment, and second moment of forces  under an  averaged and dimensional form. 
Indeed, the results shown in the above sections still need to be integrated over all particles' center of mass velocities \textbf{w}~\eqref{eq:conditional_average}, and to be put in dimensional form, hence some manipulations are still necessary.



\subsection{Averaged zeroth, first, and second moments of forces}

First, all the stresses on the left-hand-side of~\ref{eq:drag_force_application_last,eq:first_mom_trans_res,eq:second_mom_final} have been made dimensionless using the scale $\mu_f |\textbf{w}_r|/a$, and the lengths using $a$.
Additionally, recall that the Reynolds number is given by $Re= |\textbf{w}_r|\rho_f a/\mu_f$. 

Hence, the viscous terms in~\ref{eq:drag_force_application_last,eq:first_mom_trans_res,eq:second_mom_final} scale linearly with $\textbf{w}_r = \textbf{u} - \textbf{w}$.
Thus, upon averaging overall \textbf{w} one simply get: 
\begin{equation}
    \int_{\mathbb{R}^3} P[\textbf{w}|\textbf{x},t]
    \textbf{w}_r d\textbf{w} = \textbf{u} - \textbf{u}_p = \textbf{u}_r,
\end{equation}
where we have introduced $\textbf{u}_r$ as the mean relative velocity between phases. 

In dimensional form the first moment of forces, given by~\ref{eq:first_mom_trans_res}, scale as $\textbf{w}_r\textbf{w}_r$.
Hence, upon averaging overall $\textbf{w}$ one obtain,   
\begin{equation}
    n_p[\textbf{x},t]\int_{\mathbb{R}^3} P[\textbf{w}|\textbf{x},t]
    \textbf{u}_r\textbf{u}_r d\textbf{w}
    % =
    % n_p\int_{\mathbb{R}^3} P[\textbf{w}]
    % (\textbf{u}-\textbf{w})(\textbf{u}-\textbf{w}) 
    % d\textbf{w}
    = 
    n_p \textbf{u}_r \textbf{u}_r
    + \pavg{\textbf{u}_\alpha'\textbf{u}_\alpha'},
    \label{eq:standard_dev1}
\end{equation}
where we recall that $\pavg{\textbf{u}_\alpha'\textbf{u}_\alpha'}$ is the droplet center of mass velocity variances responsible for the momentum flux in~\ref{eq:dt_up}. 

Turning our attention to the Oseen force and the inertial contribution of the second moment of force i.e.,~\ref{eq:drag_force_application_last,eq:second_mom_final}. 
In dimensional form, we remark that we must integrate terms proportional to $|\textbf{w}_r|\textbf{w}_r$.
Because  $|\textbf{w}_r|\textbf{w}_r$ is a power function of $\textbf{w}_r$, its average cannot be written for an arbitrary distribution $P[\textbf{w}|\textbf{x},t]$ in terms of $\textbf{u}_r$ and $\pavg{\textbf{u}_\alpha'\textbf{u}_\alpha'}$.  
Nevertheless, if we assume that $\textbf{w}$ always stays in the neighborhood of $\textbf{u}_p$, then one can Taylor expand the function $|\textbf{w}_r|$ around $\textbf{u}_r$ and obtain the approximation (see~\ref{ap:varience}): 
\begin{align}
    n_p[\textbf{x},t]  \int_{\mathbb{R}^3} 
    |\textbf{w}_r| \textbf{w}_r
    P[\textbf{w}|\textbf{x},t]
    d\textbf{w}
    &=
    n_p  
    |\textbf{u}_r|\textbf{u}_r + 
    \pavg{\textbf{u}_\alpha'\textbf{u}_\alpha'}:
    (\bm\delta \textbf{p} +\textbf{p} \bm\delta  - \textbf{ppp})
    +O(\pavg{(\textbf{u}_\alpha')^{(n)}})\nonumber \\
    &=
    n_p  
    |\textbf{u}_r|\textbf{u}_r + 
    n_p \textbf{R}_p
    +O(\pavg{(\textbf{u}_\alpha')^{(n)}})
    \label{eq:standard_dev2}
\end{align}
with $\textbf{p} = \textbf{u}_r|\textbf{u}_r|^{-1}$ is the unit vector in the direction of $\textbf{u}_r$, and $n_p\textbf{R}_p$ a shorthand for the term involving the velocity fluctuation. 
The error generated by this approximation scale as $O(\pavg{(\textbf{u}_\alpha')^{(n)}})$ where $n\ge 3$, hence the error represent the third, fourth, \ldots, and $n^{th}$, order moments of the distribution of $P[\textbf{w},\textbf{x},t]$ that we have neglected.
Consequently, if  $P[\textbf{w}|\textbf{x},t]$ is a normal distribution (relatively true in viscous flows), then this formula is exact because the skewness and all other higher moments of the distribution  are zero. 


% \begin{align}
%     \label{eq:forces_reformulated1}
%     \intS[p]{\bm\sigma_{o}\cdot \textbf{n}}
%     &=
%     v_p \frac{\mu_f}{a^2}  \frac{3(2+3\lambda)}{2(\lambda+1)}\textbf{u}_r+ 
%     + v_p \frac{\rho_f}{a} \frac{3 (2+3\lambda)^2}{16 (\lambda+1)^2}
%     |\textbf{u}_r| \textbf{u}_r,\\
%         \intS[p]{\textbf{r}\bm\sigma_{o}\cdot \textbf{n}}
%         - \intO[i]{2\mu_f \textbf{e}_i}
%         &= 
%         - \rho_f v_p  \frac{63 \lambda^{3} + 150 \lambda^{2} + 112 \lambda + 28}{80 (\lambda+1)^{3}}  \textbf{u}_r \textbf{u}_r
%        \\ \nonumber
%         % \label{eq:first_mom_trans_res2}
%         % \intS[p]{\textbf{r} \cdot  \bm\sigma_{o}\cdot \textbf{n}}
%         &+ v_p \rho_f   \frac{48\lambda^3 + 105 \lambda^2 + 62\lambda + 8}{240 (\lambda+1)^3} (\textbf{u}_r\cdot \textbf{u}_r) \bm\delta\\
%     \label{eq:forces_reformulated2}
%         \frac{1}{2}\intS[p]{r_kr_j ( \bm\sigma_{o}\cdot \textbf{n})_i}
%         - \intO[i]  { 2\mu_f r_k (\textbf{e}_i)_{ji}}
%         &= v_p \mu_f \frac{ 3}{4(\lambda+1)} [
%             \frac{2}{3}\delta_{ij} (\textbf{u}_r)_k 
%             + \lambda \delta_{jk} (\textbf{u}_r)_i
%         ]\\ 
%         &+ v_p a \rho_f  
%         \frac{3}{256(\lambda+1)^2} [
%             (3\lambda^2+20\lambda +12) \delta_{ij} |\textbf{u}_r|(\textbf{u}_r)_k \\
%             &-( 3\lambda+2)^2 \delta_{ik} |\textbf{u}_r|(\textbf{u}_r)_j 
%             + 8 \lambda (3\lambda+2)\delta_{jk} |\textbf{u}_r|(\textbf{u}_r)_i
%         ], 
%         \label{eq:forces_reformulated3}
% \end{align} 

In summaries averaging~\ref{eq:drag_force_application_last,eq:first_mom_trans_res,eq:second_mom_final} overall \textbf{w} and using~\ref{eq:standard_dev1,eq:standard_dev2} we obtain the ensemble averaged zeroth, first, and second moment of forces, which read, in dimensional form: 
\begin{align}
    \label{eq:forces_reformulated1_avg}
    \pSavg{\bm\sigma_f^*\cdot \textbf{n}}
    =
    \frac{\mu_f}{a^2}  \frac{3(2+3\lambda)}{2(\lambda+1)} \phi \textbf{u}_r 
    + \frac{\rho_f}{a} \frac{3 (2+3\lambda)^2}{16 (\lambda+1)^2}
    \phi (|\textbf{u}_r| \textbf{u}_r+ \textbf{R}_p),\\
    \pSavg{\textbf{r}\bm\sigma_f^*\cdot \textbf{n}}
    - \pOavg[i]{2\mu_f \textbf{e}^*_d}
    = 
    - \rho_f  \frac{63 \lambda^{3} + 150 \lambda^{2} + 112 \lambda + 28}{80 (\lambda+1)^{3}} [\phi \textbf{u}_r \textbf{u}_r + v_p \pavg{\textbf{u}_\alpha'\textbf{u}_\alpha'} ] \nonumber \\ 
    % \label{eq:first_mom_trans_res2}
    % \intS[p]{\textbf{r} \cdot  \bm\sigma_{o}\cdot \textbf{n}}
    + \rho_f   \frac{48\lambda^3 + 105 \lambda^2 + 62\lambda + 8}{240 (\lambda+1)^3} (\phi \textbf{u}_r\cdot \textbf{u}_r+ v_p \pavg{\textbf{u}_\alpha'\cdot \textbf{u}_\alpha'}) \bm\delta
    \label{eq:forces_reformulated2_avg}\\
    \frac{1}{2}\pSavg{r_kr_j ( \bm\sigma_{o}\cdot \textbf{n})_i}
    - \pOavg[i]{2 \mu_f r_k (\textbf{e}_i)_{ji}}
    =
    \phi \mu_f \frac{ 3}{4(\lambda+1)} [
        \frac{2}{3}\delta_{ij} (\textbf{u}_r)_k 
        + \lambda \delta_{jk} (\textbf{u}_r)_i
    ]\nonumber\\ 
    + \phi \rho_f 
    \frac{3}{256(\lambda+1)^2} [
    (3\lambda^2+20\lambda +12) \delta_{ij} ( |\textbf{u}_r| \textbf{u}_r+\textbf{R}_p)_k\nonumber \\
    -( 3\lambda+2)^2 \delta_{ik} (|\textbf{u}_r| \textbf{u}_r+\textbf{R}_p)_j
    + 8 \lambda (3\lambda+2)\delta_{jk}  (|\textbf{u}_r| \textbf{u}_r +\textbf{R}_p)_i
    ], 
    \label{eq:forces_reformulated3_avg}
\end{align}
As discussed in~\cite[Eq. (5.35)]{fintzi2025averaged}, in the averaged momentum equation~\ref{eq:dt_uf2} only the permutation, 
\begin{multline}
    K_{i(jk)}
    + K_{j(ik)}
    - K_{k(ij)}
    =
    \phi \mu_f \frac{3\lambda}{4(\lambda+1)} [\delta_{ik} (\textbf{u}_r)_j + \delta_{jk} (\textbf{u}_r)_i]
    + \mu_f \phi  \frac{ 2 - 3\lambda}{4(\lambda+1)}  \delta_{ij} (\textbf{u}_r)_k\\
    + \phi a \rho_f \frac{ 3\lambda}{32(\lambda+1)^2} ( 3\lambda+2) [\delta_{ik} (\textbf{u}_r|\textbf{u}|)_j + \delta_{jk} (\textbf{u}_r|\textbf{u}|)_i]\\
    - 
    \phi a \rho_f \frac{3}{128(\lambda+1)^2} ( 15\lambda^2+4\lambda -4) \delta_{ij} (\textbf{u}_r|\textbf{u}_r|+\textbf{R}_p)_k.
    \label{eq:symmetry}
\end{multline}
will be of physical relevance, because this tensor appears under the double divergence sign, i.e. $\partial_k \partial_j$.
Here $K_{ijk}$ represents the relation given by~\ref{eq:forces_reformulated3_avg}. 
In~\ref{eq:symmetry}, the $(\ldots)$ represents symmetry over two indices. 
This transformation is useful to highlight that the equivalent stress is symmetric along $ij$, which is a necessary requirement to conserve the objectivity of the effective stress~\citep{renee2023thermomecanique}\footnote{
    For example the stress tensor: $\bm\sigma = \grad \textbf{u}+ ^\dagger \grad \textbf{u}$ is objective, in opposition to $\bm\sigma = \grad \textbf{u}$, which is not. 
    The same apply for the gradient of $\textbf{u}_r$. 
}. 

\subsection{Averaged system of equations}

The results from the Stokes regime given in \citet{fintzi2025averaged}, together with the one given in the previous sections~\eqref{eq:forces_reformulated1_avg,eq:forces_reformulated2_avg,eq:forces_reformulated3_avg,eq:symmetry} can be incorporated into~\ref{eq:dt_phif,eq:div_u,eq:dt_phip,eq:dt_up,eq:dt_uf2}, it yields, 
\begin{align}
    \label{eq:firstSys}
    \phi_f + \phi &\approx 1,\\
    \div \textbf{u} &= 0,\\
    (\pddt + \textbf{u} \cdot \grad)\phi
    &=
    \div (\phi \textbf{u}_r),\\
    \rho_d \phi (\pddt + \textbf{u}_p \cdot \grad)\textbf{u}_p
    % + \div \pavg{m_p \textbf{u}_\alpha'\textbf{u}_\alpha'}
    &=
    \phi(\div \bm\Sigma
    + \rho_d  \textbf{g})
    + \div \bm\sigma_p^\text{eff}
    + \textbf{F},
    \label{eq:before_the_lastSys}
    \\
    \phi_f \rho_f(\pddt + \textbf{u}_f  \cdot \grad) \textbf{u}_f
    % - \div \avg{\chi_f\rho_f \textbf{u}_f'\textbf{u}_f'}
    &= \phi_f 
    \left(\div \bm\Sigma
    + \rho_f \textbf{g}\right)
    + \div \bm\sigma_f^\text{eff}
    -\textbf{F},
    \label{eq:lastSys}
\end{align}
where,
\begin{align}
    \textbf{F}=&
    \phi
    \frac{\mu_f}{a^2}
    \frac{3(2+3\lambda)}{2(1+\lambda)}\textbf{u}_r
    + \phi\mu_f  \frac{3\lambda}{4(\lambda +1)} \grad^2 \textbf{u}
    + \frac{\rho_f}{a} \frac{3 (2+3\lambda)^2}{16 (\lambda+1)^2}
    \phi (|\textbf{u}_r| \textbf{u}_r+ \textbf{R}_p),
    \label{eq:drag_final}
    \\
    \bm\sigma_f^\text{eff}
    =&
     \mu_f \phi \frac{5\lambda +2}{2(\lambda+1)} \textbf{E}
    - \mu_f \frac{3\lambda}{4(\lambda+1)} [
    \grad(\phi \textbf{u}_r)
    + \grad(\phi \textbf{u}_r)^\dagger]
    + \mu_f \frac{3\lambda - 2}{4(\lambda+1)} \div(\phi \textbf{u}_r)  \bm\delta\nonumber\nonumber \\
    &- \rho_f  \frac{63 \lambda^{3} + 150 \lambda^{2} + 112 \lambda + 28}{80 (\lambda+1)^{3}} [\phi \textbf{u}_r \textbf{u}_r + v_p \pavg{\textbf{u}_\alpha'\textbf{u}_\alpha'} ] \nonumber \\ 
    % \label{eq:first_mom_trans_res2}
    % \intS[p]{\textbf{r} \cdot  \bm\sigma_{o}\cdot \textbf{n}}
    &+ \rho_f   \frac{48\lambda^3 + 105 \lambda^2 + 62\lambda + 8}{240 (\lambda+1)^3} (\phi \textbf{u}_r\cdot \textbf{u}_r+ v_p \pavg{\textbf{u}_\alpha'\cdot \textbf{u}_\alpha'}) \bm\delta\nonumber \\
    &- a\rho_f\frac{ 3\lambda}{32(\lambda+1)^2} ( 3\lambda+2) [\grad (\phi|\textbf{u}_r|\textbf{u}_r+\phi\textbf{R}_p) + \grad (\phi|\textbf{u}_r|\textbf{u}_r+\phi\textbf{R}_p)]\nonumber \\
    &+ a\rho_f\frac{3}{128(\lambda+1)^2} ( 15\lambda^2+4\lambda -4) \bm\delta \div(\phi|\textbf{u}_r|\textbf{u}_r+ \phi\textbf{R}_p)\nonumber \\
    &-\avg{\chi_f\rho_f \textbf{u}_f'\textbf{u}_f'}
    \label{eq:sigma_feffff}\\
    \bm\sigma_p^\text{eff},
    =&-\rho_d v_p\pavg{\textbf{u}_\alpha'\textbf{u}_\alpha'}.
\end{align}
This system of equation is closed entirely, at the exception to the velocity variance terms: $\pavg{\textbf{u}_\alpha'\textbf{u}_\alpha'}$, and $\avg{\chi_f \textbf{u}_f'\textbf{u}_f'}$.
One may either solve transport equations for these variances, as is done in classical turbulence modelling \citep{pope2001turbulent} and in kinetic theory \citep{rao2008introduction}, respectively, or adopt explicit closures that relate them to $\textbf{u}_r$, $\lambda$, $\phi$, and their gradients. 
It is reasonable to expect that the functional form of such closures does not introduce new behaviors beyond those already present in $\bm{\sigma}_f^\text{eff}$, so that the following discussion remains qualitatively valid after their inclusion\footnote{
    Hints on their modelling are provided in \citet{fintzi2025averaged}. 
    Additionally, a closure for the velocity variance of the continuous and dispersed phases in the Oseen regime can be found in \citet{koch1993hydrodynamic}. 
}.


The first important point that we would like to address is the presence of the term $\sim \rho_f \phi  \textbf{u}_r\textbf{u}_r$ in $\bm\sigma_f^\text{eff}$. 
This term is clearly non-negligible in buoyant suspension of droplets because of the strong relative motions between phases, induced by buoyancy forces.  
The second moment of force is proportional to the gradient of $\sim a \rho_f \phi  \textbf{u}_r|\textbf{u}_r|$. 
Depending on the magnitude of $\grad \phi$ and $\grad \textbf{u}_r$ (or $\grad |\textbf{u}_r|$) this term may be more or less important than the first moment contribution. 
However, the separation of scale hypothesis used implicitly in this procedure assumes that these gradients posses a length scale $L$ a lot larger than the radius of a droplet $a$. 
Consequently, the fourth and fifth lines of~\ref{eq:sigma_feffff} (i.e., the second moment contribution) are of $O(a/L)$ compared to the second and third lines (first moment contribution), hence the inertial part of the second moment is probably irrelevant in the modeling of the suspension Rheology. 
Overall, the mixture momentum equation (the sum of~\ref{eq:before_the_lastSys,eq:lastSys}) is not a Newtonian fluid due to the stokes contribution of the second moment, and of the inertia contribution of the first moment, and, to a lesser extent, the inertial contribution of the second moment. 

In the averaged momentum equation of the continuous phase~\eqref{eq:lastSys}, it is the divergence of the stress that matters; thus the first moment contributes through the terms $\rho_f \textbf{u}_r\textbf{u}_r \cdot \grad \phi$, and $\rho_f\phi  \div (\textbf{u}_r \textbf{u}_r)$. 
Consequently, if strong gradients of volume fraction or strong inhomogeneities in $\textbf{u}_r$ are present (as is the case near a wall, see \citet{cox1971suspended}), these stresses may be important.
 

Owing to the rigorous averaging procedure used here, we show that the velocity variance of the dispersed phase appears explicitly in $\textbf{F}$. 
This point does not appear to have been raised previously. Although the Oseen drag law is valid only in a restricted $Re$ range, these results highlight that when using an isolated-particle model in an averaged framework, one must also average over the center of mass velocity of the test particle, thereby introducing moments of the dispersed phase velocity distribution.
For example, the model of \citet{schiller1933}, when averaged over the particle center of mass velocity, acquires additional terms velocity variance terms. 
The same comments apply to the first and second moments of forces. 


Lastly, note that this system of equation can be completed with the average of~\ref{eq:beta_coef} which provides an estimation of the averaged deformation of the droplet in the suspension. 
Then following,~\citet{taylor1964deformation} one could compute the Stokes drag force on a deformed droplet, but also the whole first moment on a deformed droplet~\citep{schowalter1968rheological}, to obtain a system of equation that accurate at $O(We)$\footnote{A formula for the second moment of force in terms of the deformation could also be required}. 
But this is reported to a future work, hence for instance~\ref{eq:lastSys} is accurate at $O(Re \phi)$ and at $O(1)$ in $Ca$.  




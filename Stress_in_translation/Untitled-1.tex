
\subsection{First moment of force on a droplet in pure translation}

\begin{align}
    \hat{\textbf{u}}_r = 0
    && 
    \hat{\textbf{b}} = \textbf{b}|_{\textbf{x}=\textbf{y}} 
    && 
    \hat{\textbf{u}}_{o/i} = \mathcal{U}_{o/i}^\text{(1-b)}\cdot  \textbf{b}|_{\textbf{x}=\textbf{y}} 
    && 
    \hat{\bm\sigma}_{o/i} = \mathcal{S}_{o/i}^{(1-b)}\cdot \textbf{b}|_{\textbf{x}=\textbf{y}} \\
    \hat{\textbf{u}}_r = 0
    && 
    \hat{\textbf{b}} = \textbf{r}\cdot \grad\textbf{b}|_{\textbf{x}=\textbf{y}} 
    && 
    \hat{\textbf{u}}_{o/i} = \mathcal{U}_{o/i}^\text{(2-b)}:  \grad\textbf{b}|_{\textbf{x}=\textbf{y}} 
    && 
    \hat{\bm\sigma}_{o/i} = \mathcal{S}_{o/i}^{(2-b)}:\grad \textbf{b}|_{\textbf{x}=\textbf{y}} \\
    \hat{\textbf{u}}_r = 0
    && 
    \hat{\textbf{b}} = \frac{1}{2}\textbf{rr} : \grad\grad\textbf{b}|_{\textbf{x}=\textbf{y}} 
    && 
    \hat{\textbf{u}}_{o/i} = \mathcal{U}_{o/i}^\text{(3-b)}\vdots  \grad\grad\textbf{b}|_{\textbf{x}=\textbf{y}} 
    && 
    \hat{\bm\sigma}_{o/i} = \mathcal{S}_{o/i}^{(3-b)}\vdots\grad\grad \textbf{b}|_{\textbf{x}=\textbf{y}} \\
    \label{eq:test_prob1}
\end{align}

which gives, 
\begin{multline}
    \intS[p]{2(\textbf{u}_{i} + \textbf{u}_r)}
    =
    \intS[p]{\textbf{u}_{r}\cdot \mathcal{S}^\text{(1-b)}_{o/i} \cdot \textbf{n}}
    - \intO[i]{\mathcal{S}^\text{(1-b)}_{o/i} :\grad\textbf{u}_f}
    - (1-\lambda) \intO[p]{\mathcal{U}^\text{(1-b)}_{i} \cdot \grad^2 \textbf{u}_f }\\
    + \intS[p]{2 \mathcal{U}^\text{(1-b)}_{i}  \cdot  \textbf{b}}
    + \zeta Re \intO{\mathcal{U}^\text{(1-b)}_{i} \cdot \textbf{f}_{i}} 
    + Re\intO[o]{\mathcal{U}^\text{(1-b)}_{o}\cdot \textbf{f}_{o}}.
\end{multline}
\begin{multline}
    \intS[p]{2(\textbf{u}_{i} + \textbf{u}_r)\textbf{r}}
    =
    \intS[p]{\textbf{u}_{r}\cdot \mathcal{S}^\text{(2-b)}_{o/i} \cdot \textbf{n}}
    - \intO[i]{\mathcal{S}^\text{(2-b)}_{o/i} :\grad\textbf{u}_f}
    - (1-\lambda) \intO[p]{\mathcal{U}^\text{(2-b)}_{i} \cdot \grad^2 \textbf{u}_f }\\
    + \intS[p]{2\mathcal{U}^\text{(2-b)}_{i} \cdot  \textbf{b}}
    % \\ 
    + \zeta Re \intO{\mathcal{U}^\text{(2-b)}_{i} \cdot \textbf{f}_{i}} 
    + Re\intO[o]{\mathcal{U}^\text{(2-b)}_{o}\cdot \textbf{f}_{o}}.
    \label{eq:un}
\end{multline}
% \begin{multline}
%     \intS[p]{\hat{\textbf{u}}_{r} \cdot  \bm\sigma_{o}\cdot \textbf{n}}
%     - \intO[i]{2\textbf{e}_i : \grad\hat{\textbf{u}}_f}
%     - (1-\lambda) \intO[p]{(\textbf{u}_{i} + \textbf{u}_r)\cdot \grad^2 \hat{\textbf{u}}_f}
%     + \intS[p]{2(\textbf{u}_{i} + \textbf{u}_r)\cdot \hat{\textbf{b}}}
%     \\
%     =
%     \intS[p]{\textbf{u}_{r}\cdot \hat{\bm\sigma}_{o}\cdot \textbf{n}}
%     - \intO[i]{2\hat{\textbf{e}}_i :\grad\textbf{u}_f}
%     - (1-\lambda) \intO[p]{(\hat{\textbf{u}}_{i} + \hat{\textbf{u}}_r) \cdot \grad^2 \textbf{u}_f }
%     + \intS[p]{2(\hat{\textbf{u}}_{i} + \hat{\textbf{u}}_r) \cdot  \textbf{b}}
%     \\ 
%     + \zeta Re \intO{(\hat{\textbf{u}}_i+\hat{\textbf{u}}_r)\cdot \textbf{f}_{i}} 
%     + Re\intO[o]{\hat{\textbf{u}}_{o}\cdot \textbf{f}_{o}}.
% \end{multline}
where the forcing term of this problem is determined by the imposed flow

\subsection{Uniform translation without Marangoni forces}
In the `true' problem we consider in the first place a uniform translation situation.
Such that, 
\begin{align*}
    \textbf{u}_r = \textbf{u}_r|_{\textbf{x}=\textbf{y}} . 
\end{align*}
In this particlar cases we obtain the following formulas,
\begin{multline}
    \intS[p]{\bm\sigma_{o}\cdot \textbf{n}}
    =
    \textbf{u}_{r}\cdot\intS[p]{ \mathcal{S}_o^{(1)} \cdot \textbf{n}}
    % - \intO[i]{ \mathcal{S}_{i}^{(1)} :\grad\textbf{u}_f}
    % + (\lambda-1) \intO[p]{(\mathcal{U}_{i}^{(1)} + \bm\delta) \cdot \grad^2   \textbf{u}_f }\\ 
    + \zeta Re \intO{(\mathcal{U}_{i}^{(1)} + \bm\delta)\cdot \textbf{f}_{i}} 
    + Re\intO[o]{\mathcal{U}_{o}^{(1)}\cdot \textbf{f}_{o}},\\
    \intS[p]{\textbf{r}  \bm\sigma_{o}\cdot \textbf{n}}
    - \intO[i]{2\textbf{e}_i}
    =
    \textbf{u}_{r}\cdot\intS[p]{  \mathcal{S}_o^{(2)}\cdot \textbf{n}}
    % - \intO[i]{ \mathcal{S}_i^{(2)} :\grad\textbf{u}_f}
    % + (\lambda-1) \intO[p]{(\mathcal{U}_{i}^{(2)}  + \textbf{r}\bm\delta) \cdot \grad^2   \textbf{u}_f }
    + \zeta Re \intO{(\mathcal{U}_{i}^{(2)}  + \textbf{r}\bm\delta)\cdot \textbf{f}_{i}} 
    + Re\intO[o]{\mathcal{U}_{o}^{(2)}\cdot \textbf{f}_{o}},\\
    \intS[p]{\textbf{rr}  \bm\sigma_{o}\cdot \textbf{n}}
    - \intO[i]{2\textbf{re}_i}
    % + (\lambda-1) \intO[p]{2(\textbf{u}_{i} + \textbf{u}_r):\bm\delta}
    % \\
    =
    \textbf{u}_{r}\cdot\intS[p]{ \mathcal{S}_o^{(3)}\cdot \textbf{n}}
    % - \intO[i]{\mathcal{S}_i^{(3)} :\grad\textbf{u}_f}\\
    % + (\lambda-1) \intO[p]{(\mathcal{U}_{i}^{(3)} + \textbf{rr}\bm\delta) \cdot \grad^2 \textbf{u}_f }
    + \zeta Re \intO{(\mathcal{U}_{i}^{(3)} + \textbf{rr}\bm\delta)\cdot \textbf{f}_{i}} 
    + Re\intO[o]{\mathcal{U}_{o}^{(3)} \cdot \textbf{f}_{o}},
\end{multline}
Additionally, one may need to compute the internal shear independently in which case he uses \ref{eq:un} 
\begin{equation}
    \intS[p]{2\textbf{u}_{i} \textbf{r}}
    =
    \textbf{u}_{r}\cdot \intS[p]{ \mathcal{S}^\text{(2-b)}_{o/i} \cdot \textbf{n}}
    % \\ 
    + \zeta Re \intO{\mathcal{U}^\text{(2-b)}_{i} \cdot \textbf{f}_{i}} 
    + Re\intO[o]{\mathcal{U}^\text{(2-b)}_{o}\cdot \textbf{f}_{o}}.
\end{equation}
Because, $\textbf{u}_r$ is constant, its contribution on
where the inertial terms become, 
\begin{equation}
    \textbf{f}_{o/i} = \pddt \textbf{u}_{o/i} + (\textbf{u}_{o/i} + \textbf{u}_r) \cdot \grad \textbf{u}_{o/i}
\end{equation}
The first term of each equation corresponds by definition to the stokes contribution drag force contribution to the droplets, the two remaining ones correspond to the inertial contributions. 

Basically the additional contribution of the first moment will be there, $\textbf{u}_r\textbf{u}_r$ regarding the other contribution it will be about that $\pddt \textbf{u}_r$ on the second moment 


\tb{ALLL THESE TERMS MUST BE COMPUTED NOW}
\subsection{Pure linear flow translation}
Now we consider that, 
\begin{equation}
    \textbf{u}_r = \textbf{r}\cdot \grad \textbf{u}_f |_{\textbf{x}=\textbf{y}}
\end{equation}

\begin{multline}
    \intS[p]{\bm\sigma_{o}\cdot \textbf{n}}
    =
    \grad \textbf{u}_f\cdot \intS[p]{\textbf{r} \mathcal{S}_o^{(1)} \cdot \textbf{n}}
    - \grad\textbf{u}_f:\intO[i]{ \mathcal{S}_{i}^{(1)}}
    \\ 
    + \zeta Re \intO{(\mathcal{U}_{i}^{(1)} + \bm\delta)\cdot \textbf{f}_{i}} 
    + Re\intO[o]{\mathcal{U}_{o}^{(1)}\cdot \textbf{f}_{o}},
\end{multline}
\begin{multline}
    \intS[p]{\textbf{r}  \bm\sigma_{o}\cdot \textbf{n}}
    - \intO[i]{2\textbf{e}_i}
    =
    \grad \textbf{u}_f\cdot \intS[p]{\textbf{r}  \mathcal{S}_o^{(2)}\cdot \textbf{n}}
    - \grad\textbf{u}_f:\intO[i]{ \mathcal{S}_i^{(2)}}
    \\ 
    + \zeta Re \intO{(\mathcal{U}_{i}^{(2)}  + \textbf{r}\bm\delta)\cdot \textbf{f}_{i}} 
    + Re\intO[o]{\mathcal{U}_{o}^{(2)}\cdot \textbf{f}_{o}},
\end{multline}
\begin{multline}
    \intS[p]{\textbf{rr}  \bm\sigma_{o}\cdot \textbf{n}}
    - \intO[i]{2\textbf{re}_i }
    % + (\lambda-1) \intO[p]{2(\textbf{u}_{i} + \textbf{u}_r):\bm\delta}
    % \\
    =
    \grad \textbf{u}_f\cdot \intS[p]{\textbf{r} \mathcal{S}_o^{(3)}\cdot \textbf{n}}
    - \grad\textbf{u}_f:\intO[i]{\mathcal{S}_i^{(3)}}\\
    + \zeta Re \intO{(\mathcal{U}_{i}^{(3)} + \textbf{rr}\bm\delta)\cdot \textbf{f}_{i}} 
    + Re\intO[o]{\mathcal{U}_{o}^{(3)} \cdot \textbf{f}_{o}}.
\end{multline}
In that case the inertial contribution reads, 
\begin{equation}
    \textbf{f}_{i/o} 
    =
    \pddt \textbf{u}_{i/o}
    + (\textbf{u}_{i/o} + \textbf{r}\cdot \grad \textbf{u}_f)\cdot \grad \textbf{u}_{i/o}
    + \textbf{u}_{i/o}\cdot \grad \textbf{u}_f
\end{equation}
\subsection{Pure quadratic flows}
Now we consider that, 
\begin{equation}
    \textbf{u}_r = \frac{1}{2}\textbf{rr} : \grad\grad \textbf{u}_f |_{\textbf{x}=\textbf{y}}
\end{equation}

Hence it gives, 
\begin{multline}
    \intS[p]{\bm\sigma_{o}\cdot \textbf{n}}
    =
    \frac{1}{2}\grad\grad \textbf{u}_f : \intS[p]{\textbf{rr} \mathcal{S}_o^{(1)} \cdot \textbf{n}}
    -\grad\grad \textbf{u}_f:\intO[i]{ \textbf{rr}\mathcal{S}_{i}^{(1)}} 
    % + (\lambda-1) \intO[p]{(\mathcal{U}_{i}^{(1)} + \bm\delta) \cdot \grad^2   \textbf{u}_f }
    \\ 
    + \zeta Re \intO{(\mathcal{U}_{i}^{(1)} + \bm\delta)\cdot \textbf{f}_{i}} 
    + Re\intO[o]{\mathcal{U}_{o}^{(1)}\cdot \textbf{f}_{o}},
\end{multline}
\begin{multline}
    \intS[p]{\textbf{r}  \bm\sigma_{o}\cdot \textbf{n}}
    - \intO[i]{2\textbf{e}_i}
    =
    \frac{1}{2}\grad\grad \textbf{u}_f : \intS[p]{\textbf{rr}  \mathcal{S}_o^{(2)}\cdot \textbf{n}}
    -\grad\grad \textbf{u}_f:\intO[i]{ \textbf{r}\mathcal{S}_i^{(2)}} 
    % + (\lambda-1) \intO[p]{(\mathcal{U}_{i}^{(2)}  + \textbf{r}\bm\delta) \cdot \grad^2   \textbf{u}_f }
    \\ 
    + \zeta Re \intO{(\mathcal{U}_{i}^{(2)}  + \textbf{r}\bm\delta)\cdot \textbf{f}_{i}} 
    + Re\intO[o]{\mathcal{U}_{o}^{(2)}\cdot \textbf{f}_{o}},
\end{multline}
\begin{multline}
    \intS[p]{\textbf{rr}  \bm\sigma_{o}\cdot \textbf{n}}
    - \intO[i]{2\textbf{re}_i }
    % + (\lambda-1) \intO[p]{2(\textbf{u}_{i} + \textbf{u}_r):\bm\delta}
    % \\
    =
    \frac{1}{2}\grad\grad \textbf{u}_f : \intS[p]{\textbf{rr} \mathcal{S}_o^{(3)}\cdot \textbf{n}}
    -\grad\grad \textbf{u}_f : \intO[i]{\textbf{r}\mathcal{S}_i^{(3)}}\\
    % + (\lambda-1) \intO[p]{(\mathcal{U}_{i}^{(3)} + \textbf{rr}\bm\delta) \cdot \grad^2 \textbf{u}_f }
    + \zeta Re \intO{(\mathcal{U}_{i}^{(3)} + \textbf{rr}\bm\delta)\cdot \textbf{f}_{i}} 
    + Re\intO[o]{\mathcal{U}_{o}^{(3)} \cdot \textbf{f}_{o}}.
\end{multline}
with, 
\begin{equation}
    \textbf{f}_{i/o}
    =
    \pddt \textbf{u}_{i/o}
    + 
    (\textbf{u}_{i/o} + \frac{1}{2}\textbf{rr} : \grad\grad \textbf{u}_f)\cdot \grad \textbf{u}_{i/o}
    + \textbf{r}\textbf{u}_{i/o} : \grad\grad \textbf{u}_f
\end{equation}

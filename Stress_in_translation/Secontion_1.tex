\section{Context}

To better illustrate the role of various forces and stresses in a multiphase flow model, we begin by presenting the averaged momentum equations for the dispersed and continuous phases, derived through an ensemble averaging procedure.
For mono-disperse emulsions, the mass and momentum equations of the continuous phase and dispersed phase can be written \citep{fintzi2024averaged} (\tb{+my phd}),
\begin{align}
    % \phi_f + \phi &= 1\\
    (\pddt + \textbf{u}_f  \cdot \grad) \phi_f
    &= - \phi_f \div \textbf{u}_f\\
    (\pddt + \textbf{u}_p \cdot \grad)n_p
    &=
    - n_p \div \textbf{u}_p\\
    \phi_f \rho_f(\pddt + \textbf{u}_f  \cdot \grad) \textbf{u}_f
    % - \div \avg{\chi_f\rho_f \textbf{u}_f'\textbf{u}_f'}
    &= \phi_f 
    \left(\div \bm{\Sigma}_f
    + \rho_f \textbf{g}\right)
    + \div \bm\sigma_f^\text{eff}
    -n_p \textbf{f}\\
    m_pn_p (\pddt + \textbf{u}_p \cdot \grad)\textbf{u}_p
    % + \div \pavg{m_p \textbf{u}_\alpha'\textbf{u}_\alpha'}
    &=
    n_pm_p(\div \bm\Sigma_f
    + \rho_d  \textbf{g})
    + \div \bm\sigma_p^\text{eff}
    + n_p\textbf{f}_p
    \\
\end{align}
respectively. 
The subscript $f$ and $p$ refer to continuous phase and droplets phase averaged quantities, respectively.
The vector $\textbf{g}$ is the acceleration of gravity and $\rho_k$ is the density of the phase $k$. 
$\phi_f$ is the volume fraction of the continuous phase, $n_p$ the particle number density, $\textbf{u}_f$ (resp. $\textbf{u}_p$) the averaged velocity of the fluid (resp. dispersed) phase, $\bm\Sigma_f = - p_f \bm\delta + \mu_f [\grad \textbf{u}_f +  (\grad \textbf{u}_f)^\dagger ]$ the \textit{mean Newtonian stress} of the continuous phase stress tensor, with $p_f$ being the mean hydrodynamic pressure.
$\bm{\sigma}^{\text{eff}}_p$ and $\bm{\sigma}^{\text{eff}}_f$ are the effective stresses of the dispersed and continuous phase, respectively.  
Finally, $\textbf{f}_p$ represents the interphase momentum exchange, or drag force density term. 

The number density, continuous phase volume fraction, drag force density, and effective stresses of the dispersed and continuous phase, can be expressed formally as,
\begin{align}
    n_p &= \pavg{}
    \label{eq:n_p}\\
    \phi_f &= \avg{\chi_f}
    \label{eq:chi_f}\\
    n_p \textbf{f}_p  &= \pSavg{\bm\sigma'_f\cdot \textbf{n}}
    \label{eq:f_alpha}
    \\
    \bm{\sigma}_p^{\text{eff}} &= \pavg{\textbf{u}_\alpha'\textbf{u}_\alpha'}
    \label{eq:def_uup}
    \\
    \bm{\sigma}^{\text{eff}}_f 
    &= 
    - \avg{\chi_f\rho_f \textbf{u}_f'\textbf{u}_f'} 
    + \pSavg{[\textbf{r}\bm\sigma'_f\cdot \textbf{n} - \mu_f (\textbf{u}_f' \textbf{n} + \textbf{n} \textbf{u}_f')]}\nonumber\\
    &- \div
        \pSavg{[\frac{1}{2}\textbf{rr}\bm\sigma'_f\cdot \textbf{n}- \mu_f\textbf{r} (\textbf{u}_f' \textbf{n} + \textbf{n} \textbf{u}_f')]}
        + \grad\grad (\ldots)
    \label{eq:def_sigma_eff_f}
\end{align}
respectively. 
Where the operator $\avg{\ldots}$ corresponds to an ensemble average procedure, 
$\textbf{u}_\alpha$ is the center of mass velocity of a particle labeled $\alpha$, $\chi_f$ is the phase indicator function of the continuous phase, and $\delta_p$ the Dirac delta function pointing on the particle center of mass, $\delta_\Gamma$ the interfaces' indicator function, and \textbf{n} the normal of the surface pointing outward the droplets. 
The superscript $'$ indicates the relative values of a quantity with respect to its phase-averaged value. 
Specifically $\bm{\sigma}_f' = \bm{\sigma}_f^0  - \bm{\Sigma}_f$, $p_f' = p_f^0 - p_f$, $\textbf{u}_\alpha' = \textbf{u}_\alpha - \textbf{u}_p$ and $\textbf{u}_f' = \textbf{u}_f^0  -\textbf{u}_f$, with $\bm{\sigma}_f^0 = -p_f^0 + \mu_f [\grad \textbf{u}_f^0 + (\grad \textbf{u}_f^0)^\dagger] $, and $\textbf{u}_f^0$, the local stress of the continuous phase and the local velocity of the continuous phase, respectively. 

In the present situation, i.e. where the mixture is only governed by conservation of mass and momentum,  the ``closure problem'' consists in finding explicit expressions for the terms in \ref{eq:f_alpha}, \ref{eq:def_uup} and \ref{{eq:def_sigma_eff_f}}, in terms of the unknown of the problem, i.e. $n_p$, $\phi_f$, $p_f$, $\textbf{u}_p$ and $\textbf{u}_f$. 

In this contribution we focus on the last two terms of $\bm\sigma^\text{eff}_f$, i.e. the first and second moment of hydrodynamic forces acted upon spherical droplets.
Specifically we aim to study the effect of non-vanishing Reynolds number (based on the relative motion $|\textbf{u}_f - \textbf{u}_p|$) on these terms. 
Additionally, we will consider a situation where the emulsion is dilute $(\phi_f -1) \ll 1$. 
Hence, one must find the values of the disturbance pressure ($p_f'$) and velocity ($\textbf{u}_f'$) relative to a test particle, accurate at $\mathcal{O}(\phi Re)$. 

\tb{SUMMARY OF THE RESULTS }


\tb{THE EQUAITON HAVE TO BE TERMED RELATIVE TO $\textbf{u}$ so that $\div \textbf{u} = 0 $ in the reciprocal }



In the following we revisit the methodology of \citet{stone2001inertial,raja2010inertial,dabade2015}  to compute the first moment of force on a spherical droplet embedded in a uniform flow at low but finite inertia effects. 
In opposition to \citet{stone2001inertial} and \citet{raja2010inertial} who considered neutrally-buoyant sherical inclusion embedded in a shear flow, we focus on the relative motions of droplets with the continuous phase. 
\citet{dabade2015} computed the torque (skew-symmetric part of the first moment) on a spheroidal particle embedded in a uniform flow. 
Our study is similar to the work of \citet{dabade2015}, however we aim to compute the whole first moment tensor (not only the torque), and we consider droplets (instead of solid spheroidal particles). 
Therefore, we propose an original method, mainly inspired from \citet{stone2001inertial}, to compute the first moment of force on droplets at first order $Re$.  
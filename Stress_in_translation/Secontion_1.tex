\section{Introduction}

To better illustrate the role of various forces and stresses in a multiphase flow model, we begin by presenting the averaged momentum equations for the dispersed and continuous phases, derived through an ensemble averaging procedure.
For mono-disperse emulsions, the mass and momentum equations of the continuous phase and dispersed phase can be written as \citep{fintzi2024averaged},
\begin{align}
    % \phi_f + \phi &= 1\\
    (\pddt + \textbf{u}_f  \cdot \grad) \phi_f
    &= - \phi_f \div \textbf{u}_f\\
    (\pddt + \textbf{u}_p \cdot \grad)n_p
    &=
    - n_p \div \textbf{u}_p\\
    m_pn_p (\pddt + \textbf{u}_p \cdot \grad)\textbf{u}_p
    % + \div \pavg{m_p \textbf{u}_\alpha'\textbf{u}_\alpha'}
    &=
    n_pm_p(\div \bm\Sigma
    + \rho_d  \textbf{g})
    + \div \bm\sigma_p^\text{eff}
    + n_p\textbf{f}_p
    \\
    \phi_f \rho_f(\pddt + \textbf{u}_f  \cdot \grad) \textbf{u}_f
    &= \phi_f 
    \left(\div \bm{\Sigma}
    + \rho_f \textbf{g}\right)
    + \div \bm\sigma_f^{\text{eff}}
    - n_p\textbf{f}_p
    \label{eq:dt_uf2}
\end{align}
respectively. 
The subscript $f$ and $p$ refer to continuous phase and droplets phase averaged quantities, respectively.
The vector $\textbf{g}$ is the acceleration of gravity and $\rho_k$ is the density of the phase $k$. 
$\phi_f$ is the volume fraction of the continuous phase, $n_p$ the particle number density.
In the following we use the notation $\phi \to n_p v_p$, with $v_p$ the mean volume of droplet.
$\textbf{u}_f$ (resp. $\textbf{u}_p$) is the averaged velocity of the fluid (resp. dispersed) phase, $\bm\Sigma = - p_f \bm\delta + \mu_f [\grad \textbf{u} +  (\grad \textbf{u})^\dagger ]$ the \textit{mean Newtonian stress} of the mixture, with $p_f$ being the mean hydrodynamic pressure and $\textbf{u}=\phi_f \textbf{u}_f + \phi \textbf{u}_p$ the volume averaged velocity of the mixture.
$\bm{\sigma}^{\text{eff}}_p$ and $\bm{\sigma}^{\text{eff}}_f$ are the effective stresses of the dispersed and continuous phase, respectively.  
Finally, $\textbf{f}_p$ represents the interphase momentum exchange, or drag force density term. 

The number density, continuous phase volume fraction, drag force density, and effective stresses of the dispersed and continuous phase, can be expressed formally as,
\begin{align}
    n_p &= \pavg{}
    \label{eq:n_p}\\
    \phi_f &= \avg{\chi_f}
    \label{eq:chi_f}\\
    n_p \textbf{f}_p  &= \pSavg{\bm\sigma^*_f\cdot \textbf{n}}
    \label{eq:f_alpha}
    \\
    \bm{\sigma}_p^{\text{eff}} &= \pavg{\textbf{u}_\alpha'\textbf{u}_\alpha'}
    \label{eq:def_uup}
    \\
    \bm{\sigma}^{\text{eff}}_f 
    &= 
    - \avg{\chi_f\rho_f \textbf{u}_f'\textbf{u}_f'} 
    + \pavg{\intS{\textbf{r}\bm\sigma^{*}_f\cdot \textbf{n}} - \delta_p\intO{2\mu_f\textbf{e}_d^*}}\nonumber\\
    &- \div
        \pavg{ \frac{1}{2}\intS{\textbf{rr}\bm\sigma^{*}_f\cdot \textbf{n}}
        - \delta_p\intO{2\mu_f \textbf{r} \textbf{e}_d^*}}
        + \grad\grad (\ldots). 
    \label{eq:def_sigma_eff_f}
\end{align}
respectively. 
Where the operator $\avg{\ldots}$ corresponds to an ensemble average procedure, 
$\textbf{u}_\alpha$ is the center of mass velocity of a particle labeled $\alpha$, $\chi_f$ is the phase indicator function of the continuous phase, and $\delta_p$ the Dirac delta function pointing on the particle center of mass, $\delta_\Gamma$ the interfaces' indicator function, and \textbf{n} the normal of the surface pointing outward the droplets. 
The superscript $'$ indicates the relative values of a quantity with respect to its phase-averaged value. 
Specifically $p_f' = p_f^0 - p_f$, $\textbf{u}_\alpha' = \textbf{u}_\alpha - \textbf{u}_p$ and $\textbf{u}_f' = \textbf{u}_f^0  -\textbf{u}_f$, with $\textbf{u}_f^0$,  the local velocity of the continuous phase, respectively. 
The superscript $^*$ represents the relative values of a quantity with respect to the mixture volume averaged value, such that $\bm{\sigma}_f^* = \bm{\sigma}_f^0  - \bm{\Sigma}$ and $\textbf{e}_d^* = \textbf{e}_d^0 - \textbf{E}$ with $\bm{\sigma}_f^0 = -p_f^0 + \mu_f [\grad \textbf{u}_f^0 + (\grad \textbf{u}_f^0)^\dagger]$ the local stress of the continuous phase. 




In the present situation the ``closure problem'' consists in finding explicit expressions for the terms of the form $\avg{\ldots}$, in \ref{eq:f_alpha}, \ref{eq:def_uup} and \ref{{eq:def_sigma_eff_f}}, in terms of the unknown of the problem, i.e. $n_p$, $\phi_f$, $p_f$, $\textbf{u}_p$ and $\textbf{u}_f$. 
In this work we focus on the last two terms of $\bm\sigma^\text{eff}_f$, i.e. the first and second moment of hydrodynamic forces acted upon spherical droplets.
Hence, one must find an expression for the disturbance fields $\bm\sigma_f^*$ evaluated at the surface of a test droplet and of the field $\textbf{e}_d^*$ inside the test droplet. 
Because they are evaluated at the surface and in the interior of the test droplets, these terms may be written as, 
\begin{align}
    \bm\sigma_f^* =p_f' \bm\delta 
    + \mu_f [\grad \textbf{u}^* + (\grad \textbf{u}^*)^\dagger]
    && 
    \textbf{e}_d^* = [\grad \textbf{u}^* + (\grad \textbf{u}^*)^\dagger] /2
\end{align}
with $\textbf{u}^* = \textbf{u}^0 - \textbf{u}$ and $\textbf{u}^0$ the local velocity of the mixture, which becomes $\textbf{u}_d^0$ inside the test droplet and $\textbf{u}_f^0$ at the surface of the fluid droplet. 

We limit this theoretical investigation to dilute suspension. 
Likewise, we consider only vanishingly small, but non-zero, values of the droplet Reynolds number, defined as,
\begin{equation}
    Re  = \frac{|\textbf{u} - \textbf{u}_p|\rho_f a}{\mu_f}, 
\end{equation}
where $a$ is the radius of the droplets. 
Consequently, the values of $\bm\sigma_f^*$ and $\textbf{e}_d^*$ will only be accurate at order $O(1)$ in the volume fraction (neglecting the droplets interactions) and $O(Re)$ in the Reynolds number. 
Hence, after integration over the surface of the droplet and ensemble averaging these terms will bring a contribution of $O(Re \phi)$ in the averaged momentum equation. 
Note that in the following we consider arbitrary density and viscosity ratio,
\begin{align}
    \lambda = \mu_d/\mu_f,  && \zeta =\rho_d /\rho_f,
\end{align}
where $\mu_d$ and $\rho_d$ are the viscosity and density of the droplets.

In the following we revisit the methodology of \citet{stone2001inertial,raja2010inertial,dabade2015}  to compute the first moment of force on a spherical droplet embedded in a uniform flow at low but finite inertia effects. 
In opposition to \citet{stone2001inertial} and \citet{raja2010inertial} who considered neutrally-buoyant sherical inclusion embedded in a shear flow, we focus on the relative motions of droplets with the continuous phase. 
\citet{dabade2015} computed the torque (skew-symmetric part of the first moment) on a spheroidal particle embedded in a uniform flow. 
Our study is similar to the work of \citet{dabade2015}, however we aim to compute the whole first moment tensor (not only the torque), and we consider droplets (instead of solid spheroidal particles). 
Therefore, we propose an original method, mainly inspired from \citet{stone2001inertial}, to compute the first moment of force on droplets at first order $Re$.  

\tb{The second objectif of this study is to propose a general formula of the reciprocal theorem to allows extension of these results in future study}
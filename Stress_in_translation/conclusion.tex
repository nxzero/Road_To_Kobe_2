\section{Conclusion}

% \citet{einstein1905neue,taylor1932viscosity} demonstrated how the first moment of the hydrodynamic forces (Stresslet) applied on a particle immersed in pure linear flow induced an additional viscosity to the mixture. 
% Later~\citet{zhang1994ensemble,lhuillier1996contribution,jackson1997locally,zhang1997momentum} demonstrated that the second moment of forces were also contributing to the stresses inducing a non-newtonian behaviors, even in the Stokes and dilute limit.  

In this work we computed the moments of force on the surface of a test droplet in the situation of uniform relative motions between the droplet and the continuous phase. 
We considered low but finite Reynolds number $Re$. 
The averaged first moment of force is given by~\ref{eq:forces_reformulated2_avg}, scales as $O(\rho_f \phi u_r^2)$, hence contributing to the averaged Stress of the suspension on the same ground as  \citet{einstein1905neue} or \citep{taylor1932viscosity} correction to the viscosity of the mixture. 
In a lesser extend the inertial part of the second moment also contribute to the Rheology. 
This first point constitutes the main result of the paper. 

Others important conclusion reached through this work includes: a general reciprocal formula to derive the forces and moments on droplets, and the explicit appearing of the velocity variance term in the drag force term. 







\subsection{Deformation of the droplets due to inertial translation}
\label{sec:deformation}

The aim of the present subsection is to re-derive the result of~\citet{taylor1964deformation} for the deformation of a translating droplet in inertial flow.
Hence, we proceed by making some assumptions which we know are true from~\citet{taylor1964deformation} to simplify the present discussions.  
First, we consider that the droplet will deform into an ellipsoidal shape, hence we will use a second order spherical harmonic to model the droplet surface. 
Second, we do not consider the deformation induced by linear and quadratic components of the flow, even in the Stokes limit (see \citet{fintzi2025averaged}) only the uniform translational motion will be taken into account. 

In these conditions, the points $\textbf{x}_\Gamma$ lying on the surface of the droplet can be defined by the equation\citep{nadim1991motion,nadim1996concise}: 
\begin{equation}
    \textbf{x}_\Gamma(\textbf{n}) = [1+ \textbf{H}:(\textbf{e} \textbf{e}- \frac{1}{3}\bm\delta)] \textbf{e}, 
    \label{eq:r_gamma}
\end{equation}
with, \textbf{e} the radial unit vector whose origin is at the center of the droplet ($\textbf{r}=0$), and \textbf{H} a yet unknown second-order tensor quantifying the droplet deformation.

According to \citet[Eq. 5.32 ]{fintzi2025averaged} the droplet deformation can be obtained from the first moment of momentum equation. 
In steady state homogeneous regime and in dimensionless, form it reads, 
% \begin{multline}
%     % n_p (\pddt + \textbf{u}_p \cdot \grad) \textbf{S}_p
%     % +\div  \pavg{\textbf{u}_\alpha'\textbf{S}_\alpha}
%     \intS[p]{\gamma (\frac{1}{3}\bm\delta - \textbf{nn})}
%     =
%     \rho_d \intO[p]{
%         \textbf{w}_d^0  \textbf{w}_d^0 
%         -\frac{1}{3} (\textbf{w}_d^0 \cdot  \textbf{w}_d^0) \bm\delta
%     }
%     +\intS[p]{\frac{1}{2}(\textbf{r}\bm\sigma_f^*+^\dagger\textbf{r}\bm\sigma_f^*-\frac{2}{3}(\bm\sigma_f^* \cdot \textbf{r})\bm\delta)\cdot \textbf{n}}\nonumber\\
%     - \intO[p]{2 \mu_d\textbf{e}_d^*} 
%     + (1-\lambda)\intO[p]{2\mu_f\textbf{E}} 
%     + \frac{1}{2}\intO[p]{\textbf{r}(\div\bm\Sigma)+ (\div\bm\Sigma) \textbf{r}}
% \end{multline}
% Because the capillary stress is the force that resists deformation, this equation is able to provide us with an equation for \textbf{H}.
\begin{multline}
    % n_p (\pddt + \textbf{u}_p \cdot \grad) \textbf{S}_p
    % +\div  \pavg{\textbf{u}_\alpha'\textbf{S}_\alpha}
    \intS[p]{ (\frac{1}{3}\bm\delta - \textbf{nn})}
    =
    Ca Re \zeta\intO[p]{
        [\textbf{w}_d^0  \textbf{w}_d^0 
        -\frac{1}{3} (\textbf{w}_d^0 \cdot  \textbf{w}_d^0) \bm\delta]
    }
    +Ca \intS[p]{\frac{1}{2}(\textbf{r}\bm\sigma_f^*+^\dagger\textbf{r}\bm\sigma_f^*-\frac{2}{3}(\bm\sigma_f^* \cdot \textbf{r})\bm\delta)\cdot \textbf{n}}\\
    - Ca\lambda\intO[p]{2 \textbf{e}_d^*} 
    + (1-\lambda)Ca\intO[p]{2\textbf{E}} 
    + \frac{1}{2}Ca\intO[p]{\textbf{r}(\div\bm\Sigma)+ (\div\bm\Sigma) \textbf{r}}.
    \label{eq:second_mom_steady_state}
\end{multline}
Here, $\textbf{w}_d^0 = \textbf{u}_d^0 - \textbf{u}_\alpha$ is the droplet internal  velocity relative to the center of mass velocity of the same droplet, and we recall that $Ca = \mu_f U/\gamma$ is the capillary number. 
\ref{eq:second_mom_steady_state} represents the balance between the surface tension stresses (on left-hand side), against all the other stresses of different nature on the right-hand side. 

The second-order tensor on the left-hand side of~\ref{eq:second_mom_steady_state} represents the stresses induced by droplet deformation; it corresponds to the elastic stress.
If the droplet shape can be represented by~\ref{eq:r_gamma}, then this term can be computed and reads \citep{lhuillier1987phenomenology},
\begin{equation}
    \intS[p]{ (\frac{1}{3}\bm\delta - \textbf{nn})}
    =
    \frac{32}{15} \pi \textbf{H}.
    \label{eq:def_H}
\end{equation}
Hence, the constant $\tfrac{32}{15}\pi$ may be interpreted as the dimensionless ``stiffness'' of the droplet, since it relates the deformation to the stresses. 

The first term on the right-hand side of~\ref{eq:second_mom_steady_state} can be computed at $O(Re)$ using only the Stokes flow solution (i.e. $\textbf{w}_d^0 \to (\textbf{U}^{(1)}_i+\bm\delta)\cdot \textbf{w}_r$). It reads,
\begin{equation}
    \intO[p]{
        [\textbf{w}_d^0  \textbf{w}_d^0 
        -\frac{1}{3} (\textbf{w}_d^0 \cdot  \textbf{w}_d^0) \bm\delta]
    }
    \approx \frac{\pi}{15(1+\lambda)^2}   [
        \textbf{w}_r\textbf{w}_r-\frac{1}{3}(\textbf{w}_r\cdot\textbf{w}_r)\bm\delta
    ].
    \label{eq:eq_ww}
\end{equation}
The second and third terms of~\ref{eq:second_mom_steady_state} can be obtained adding up~\ref{eq:first_mom_trans_res} to $2(1-\lambda)$ times~\ref{eq:first_mom_trans_res3}, and then taking the traceless part of the resulting expression. 
The fourth term represents the contribution of the ensemble averaged shear which is zero in our configuration. 
Finally, the last term on the right-hand side of~\ref{eq:second_mom_steady_state} is  in the case of pure uniform flow, a constant pressure gradient $\grad p_f$, which vanish upon integration, because  $\intO[p]{\textbf{r}} =0$. 


Making use of~\ref{eq:eq_ww,eq:def_H,eq:first_mom_trans_res,eq:first_mom_trans_res3}, and the above remarks, enables us to solve~\ref{eq:second_mom_steady_state} for \textbf{H}, namely 
\begin{equation}
    \textbf{H}
    = 
    We  \left\{
        \frac{\zeta}{32(1+\lambda)^2}
        - \frac{243\lambda^3+684\lambda^2+638\lambda+200}{640(\lambda+1)^3} 
    \right\}[
        \textbf{w}_r\textbf{w}_r - \frac{1}{3}(\textbf{w}_r\cdot \textbf{w}_r)\bm\delta
    ].
    \label{eq:beta_coef}
\end{equation}
Considering this expression and~\ref{eq:r_gamma}, one can note that the term in braces in \ref{eq:beta_coef} is exactly the coefficient obtained by \citet[Eq. (21)]{taylor1964deformation}. 
Hence, we re-demonstrated the results of \citet{taylor1964deformation} but in tensor form using the first moment of momentum equation of the test droplet. 
Finding back the result of \citet{taylor1964deformation} gives us great confidence in the formulas for the first and second moment of forces derived in the preceding section.

The main advantage of using the moment of momentum equation to derive droplet deformation lies in its flexibility: different and more complex configurations can be incorporated by supplying alternative closures for the terms on the right-hand side of~\ref{eq:second_mom_steady_state}. 
For example, beyond inertial translation, one may account for mean linear and quadratic flows as well as Marangoni stresses. 
A second advantage is that~\ref{eq:second_mom_steady_state} is a Lagrangian equation that can be embedded in the averaged system of equations~\eqref{eq:dt_phif,eq:div_u,eq:dt_phip,eq:dt_up,eq:dt_uf2}, thereby extending the modelling to non-spherical and deformable droplets while preserving the Lagrangian modeling of the fluid particles \citep{fintzi2025averaged}.

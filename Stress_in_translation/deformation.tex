
\section{Deformation of the droplets and impact on the force moments}
\label{sec:deformation}

According to \citet{fintzi2024averaged} the droplet deformation can now be obtained from the first moment of momentum equation of the droplets. 
In steady state homogeneous regime it reads, \tb{how do we obtain that }
\begin{multline}
    % n_p (\pddt + \textbf{u}_p \cdot \grad) \textbf{S}_p
    % +\div  \pavg{\textbf{u}_\alpha'\textbf{S}_\alpha}
    \intS[p]{\gamma (\frac{1}{3}\bm\delta - \textbf{nn})}
    =
    \rho_d \intO[p]{
        \textbf{w}_d^0  \textbf{w}_d^0 
        -\frac{1}{3} (\textbf{w}_d^0 \cdot  \textbf{w}_d^0) \bm\delta
    }
    +\intS[p]{\frac{1}{2}(\textbf{r}\bm\sigma_f^*+^\dagger\textbf{r}\bm\sigma_f^*-\frac{2}{3}(\bm\sigma_f^* \cdot \textbf{r})\bm\delta)\cdot \textbf{n}}\nonumber\\
    - \intO[p]{2 \mu_d\textbf{e}_d^*} 
    + (1-\lambda)\intO[p]{2\mu_f\textbf{E}} 
    + \frac{1}{2}\intO[p]{\textbf{r}(\div\bm\Sigma)+ (\div\bm\Sigma) \textbf{r}}
\end{multline}
\tb{put the correct notations }
% \begin{multline}
%     % n_p (\pddt + \textbf{u}_p \cdot \grad) \textbf{S}_p
%     % +\div  \pavg{\textbf{u}_\alpha'\textbf{S}_\alpha}
%     \intS[p]{ (\frac{1}{3}\bm\delta - \textbf{nn})}
%     =
%     a\frac{\rho_f U^2}{\gamma}\frac{\rho_d}{\rho_f} \intO[p]{
%         \textbf{w}_d^0  \textbf{w}_d^0 
%         -\frac{1}{3} (\textbf{w}_d^0 \cdot  \textbf{w}_d^0) \bm\delta
%     }
%     +\frac{\mu_f U }{\gamma}\intS[p]{\frac{1}{2}(\textbf{r}\bm\sigma_f^*+^\dagger\textbf{r}\bm\sigma_f^*-\frac{2}{3}(\bm\sigma_f^* \cdot \textbf{r})\bm\delta)\cdot \textbf{n}}\nonumber\\
%     - \frac{\mu_f U}{\gamma}\frac{\mu_d}{\mu_f}\intO[p]{2 \textbf{e}_d^*} 
%     + (1-\lambda)\frac{\mu_fU}{\gamma}\intO[p]{2\textbf{E}} 
%     + \frac{1}{2}\frac{\mu_f U }{\gamma}\intO[p]{\textbf{r}(\div\bm\Sigma)+ (\div\bm\Sigma) \textbf{r}}
% \end{multline}
In dimensionless form we obtain, 
\begin{multline}
    % n_p (\pddt + \textbf{u}_p \cdot \grad) \textbf{S}_p
    % +\div  \pavg{\textbf{u}_\alpha'\textbf{S}_\alpha}
    \intS[p]{ (\frac{1}{3}\bm\delta - \textbf{nn})}
    =
    We \zeta\intO[p]{
        \textbf{w}_d^0  \textbf{w}_d^0 
        -\frac{1}{3} (\textbf{w}_d^0 \cdot  \textbf{w}_d^0) \bm\delta
    }
    +Ca \intS[p]{\frac{1}{2}(\textbf{r}\bm\sigma_f^*+^\dagger\textbf{r}\bm\sigma_f^*-\frac{2}{3}(\bm\sigma_f^* \cdot \textbf{r})\bm\delta)\cdot \textbf{n}}\\
    - Ca\lambda\intO[p]{2 \textbf{e}_d^*} 
    + (1-\lambda)Ca\intO[p]{2\textbf{E}} 
    + \frac{1}{2}Ca\intO[p]{\textbf{r}(\div\bm\Sigma)+ (\div\bm\Sigma) \textbf{r}},
    \label{eq:second_mom_steady_state}
\end{multline}
where we have introduced the \textit{Weber} number $We = Re \cdot Ca$. 
If one then average this equation  he finds the deformation of the test droplet being, 
where the stresses have been made dimensionless according to $\bm\sigma^* = \mu_f \grad \textbf{u}^* = \mu_f U/a$. 

Indeed, the second order tensor on the left-hand side represents the shape. 
For instance if we assume that the droplet shape can be represented by a series expansion of surface harmonics \citep{fintzi2024averaged,nadim1996concise,nadim1991motion} then this term can be written,
\begin{equation}
    \intS[p]{ (\frac{1}{3}\bm\delta - \textbf{nn})}
    =
    \frac{32}{15} \pi \textbf{H},
\end{equation}
see \citet{fintzi2024averaged} to see how \textbf{H} is defined. 

The first term on the right-hand side of~\ref{eq:second_mom_steady_state} is the only one missing. 
It can be computed at $O(Re)$ using only the Stokes flow solution of $\textbf{w}_d^0$ (i.e. $\textbf{w}_d^0 \to \textbf{U}^{(1)}_i\cdot \textbf{u}_r + \textbf{u}_r - \textbf{u}_p$) because of the $We$ present in front of this term. It reads
\begin{equation}
    \intO[p]{
        \textbf{w}_d^0  \textbf{w}_d^0 
        -\frac{1}{3} (\textbf{w}_d^0 \cdot  \textbf{w}_d^0) \bm\delta
    }
    \approx \frac{\pi}{15(1+\lambda)^2}   [
        \textbf{u}_r\textbf{u}_r-\frac{1}{3}(\textbf{u}_r\cdot\textbf{u}_r)\bm\delta
    ]
\end{equation}
The second and third terms can be obtained adding up~\ref{eq:first_mom_trans_res} to~\ref{eq:first_mom_trans_res3} times $(1-\lambda)$. 
The fourth term represents the contribution of the ensemble averaged shear which is zero in our configuration \tb{Or I include those terms but only the Stokes part, but then what happen to Ossen ? }. 
Finally, the last term on the right-hand side of~\ref{eq:second_mom_steady_state} is related to the first mean stress.
It is in the case of pure uniform flow, a constant pressure gradient $\grad p_f$, which vanish upon integration, because  $\intO[p]{\textbf{r}} =0$. 

Compiling all these remarks together gives a tensor formula for the deformation, 
\begin{equation}
    \textbf{H}
    = 
    We  \left\{
        \frac{\zeta}{32(1+\lambda)^2}
        - \frac{279\lambda^3+717\lambda^2+599\lambda+170}{640(\lambda+1)^3} 
    \right\}[
        \textbf{u}_r\textbf{u}_r - \frac{1}{3}(\textbf{u}_r\cdot \textbf{u}_r)\bm\delta
    ]
    \label{eq:beta_coef}
\end{equation}
Note that the first term of the coefficient $\beta$ obtained here agrees with the deformation derived by \citet{taylor1964deformation}, (noted $\zeta$ in their work), if one account for the two different definitions of the ``deformation'' used in our work and in \citet{taylor1964deformation}, inducing a factor equal to $2$ in front of his coefficient $\zeta$.
However, the second term of \ref{eq:beta_coef} found by \citet{taylor1964deformation}, does not agree exactly.
At this stage, the authors cannot understand the source of this inconsistency between the results.

The shape is then given by,
\begin{equation}
    \textbf{r} = r[1+ \textbf{H}:(\textbf{nn}- \frac{1}{3}\bm\delta)] \textbf{n}
\end{equation}

In \citet{taylor1964deformation} they obtain the coefficient, 
\begin{equation}
    \frac{\zeta}{32(1+\lambda)^2}
    - \frac{243\lambda^3+684\lambda^2+638\lambda+200}{640(\lambda+1)^3} 
\end{equation}
Instead of what we obtained in~\ref{eq:beta_coef}. 
At this stage it is complicated to identify the source of the disagreement. 
Both expression only agree at $\lambda = 1$ hence when the term $1-\lambda$ vanish. 
It makes us think that the trouble comes from the addition of that shearing term. 

Although we computed the second mode of deformation, note that at least the third mode will also be solicited by the second moment of forces. 
Nevertheless, the computational cost becomes increasingly complicated. 
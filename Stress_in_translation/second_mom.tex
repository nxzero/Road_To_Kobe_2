




\section{Points source, point force, and derivatives}
\label{ap:singularity_sol}
In this appendix we quickly demonstrate how to obtain the solution of a point source, point forces, and higher order derivatives of these flows. 
First, we recall the useful relation, 
\begin{equation}
    \delta(\textbf{x})
    =-\frac{1}{4\pi}\grad^2(1/r)
    =-\frac{1}{8\pi}\grad^4r.
    \label{eq:usefuk_rel}
\end{equation}
The first equality only holds in the sense of generalized functions, the second equality can be obtained directly by differentiation of $r$. 
These relations are used several times in the demonstration below. 

\paragraph{Point sources solutions :}
We start by the points source solutions, which obey the non-homogeneous Stokes equaitons: 
\begin{align}
    \div\textbf{u}= 0 
    &&
    \grad^2\textbf{u}=\grad p+ (\textbf{Q}^{(n)}\otimes \grad^{(n)})\grad\delta(\textbf{r})
\end{align}
Taking the divergence of the momentum equation gives,
\begin{align}
    \grad^2 p = \frac{1}{4\pi}\grad^{(n)}\grad^2 \grad^2(1/r) \otimes\textbf{Q}^{(n)}
    \Longleftrightarrow
    \textbf{P} = \frac{1}{4\pi}\grad^{(n)}\grad^2(1/r)\otimes\textbf{Q}^{(n)} = 0 
\end{align}
Hence, the momentum equation reads, 
\begin{equation}
    \grad^2\textbf{u}= -\frac{1}{4\pi } \grad^{(n+1)} \grad^2 (1/r) \otimes\textbf{Q}^{(n)}
    \Longleftrightarrow
    \textbf{u}= - \frac{1}{4\pi }\grad^{(n+1)}(1/r)\otimes\textbf{Q}^{(n)}
    \label{eq:pts_source}
\end{equation}
Because $\grad\textbf{u}$ is already symmetric and $p=0$ we have, 
\begin{equation}
    \bm\sigma = 2\grad \textbf{u}=- \frac{1}{4\pi }2 \grad^{(n+2)}(1/r)\otimes \textbf{Q}^{(n)}. 
\end{equation}
\paragraph{Point force solutions :} We seek a solution for,
\begin{align}
    \div\textbf{u}= 0 
    &&
    \grad^2\textbf{u}=\grad p+ (\textbf{B}^{(n+1)}\otimes \grad^{(n)})\delta(\textbf{r})
\end{align}
where $\textbf{B}^{(n+1)}$ is an arbitrary $n+1$ order tensor. 
Taking the divergence of the momentum equation yields, 
\begin{align}
    p = - \grad^{(n+1)} \delta(\textbf{r})\otimes \textbf{B}^{(n+1)}
    = \frac{1}{4\pi} \grad^{(n+1)} (1/r)\otimes \textbf{B}^{(n+1)}
\end{align}
Hence, the momentum equation can be re-written, 
\begin{align}
    \grad^2\textbf{u}
    % =   \frac{1}{4\pi} \grad^{(n+1)} (1/r) - \frac{1}{4\pi}\bm\delta \grad^{(n)} \grad^2(1/r)
    =  \frac{1}{4\pi} \grad^{(n)}[\grad^{(2)} - \bm\delta \grad^2](1/r)
    \otimes \textbf{B}^{(n+1)}
    \label{eq:lap_pts_forces}
\end{align}
Using the second equality of~\ref{eq:usefuk_rel} we directly deduce the solution of~\ref{eq:lap_pts_forces} as,
\begin{equation}
    \textbf{u}= 
    \frac{1}{8\pi} \grad^{(n)}[\grad^{(2)} - \bm\delta \grad^2] r\otimes \textbf{B}^{(n+1)}. 
    \label{eq:point_force}
\end{equation}
The stress tensor then reads,
\begin{align}
    \bm\sigma
    =
    - p \bm\delta
    + \grad \textbf{u}
    + ^\dagger \grad \textbf{u}
    % &=
    % -\frac{1}{4\pi}\grad^{(n+1)}(1/r)\bm\delta 
    % +\frac{1}{8\pi}\grad^{(n)}[2\grad^{(3)} - (\grad\bm\delta + ^\dagger\grad\bm\delta) \grad^2] r\\
    % &=
    % -\frac{1}{8\pi}\grad^{(n)}\bm\delta \grad\grad^2 r 
    % +\frac{1}{8\pi}\grad^{(n)}[2\grad^{(3)} - (\grad\bm\delta + ^\dagger\grad\bm\delta) \grad^2] r\\
    &=
    \frac{1}{8\pi}\grad^{(n)}[2\grad^{(3)} - (\grad\bm\delta + ^\dagger\grad\bm\delta+\bm\delta\grad) \grad^2] r\otimes \textbf{B}^{(n+1)}.
\end{align}
Taking $n=0$ one recover the free space Green function of Stokes flows \citep{pozrikidis2011introduction}. 

\section{Reciprocal relation for the second moment of force}
\label{ap:second_mom}

To derive an equation for the second moment of forces we set,
\begin{align}
    \hat{\textbf{u}}_r =\frac{1}{2}\textbf{rr}\cdot  \grad\grad \hat{\textbf{u}}|_{\textbf{x}=\textbf{y}},
    &&
    \hat{\textbf{b}} = 0, 
\end{align}
and use the solutions
\begin{align}    
    \hat{\textbf{u}}_{o/i} = \textbf{U}_{o/i}^{(3)} \vdots \grad\grad\hat{\textbf{u}}|_{\textbf{x}=\textbf{y}}, 
    &&
    \hat{\bm\sigma}_{o/i} = \textbf{S}_{o/i}^{(3)} \vdots \grad\grad\hat{\textbf{u}}|_{\textbf{x}=\textbf{y}},
\end{align}
in~\ref{eq:int_final_step}. 
In index notation we obtain the general relation, 
\begin{multline}
    \frac{1}{2}\intS[p]{  (\bm\sigma_{o}\cdot \textbf{n})_i r_jr_k}
    - \intO[i]{2(\textbf{e}_i )_{ij} r_k}
    % - (1-\lambda) \intO[p]{(\textbf{u}_{i} + \textbf{w}_r)\cdot \grad^2 \hat{\textbf{u}}}
    % + \intS[p]{(\textbf{u}_{i} + \textbf{w}_r)\cdot \hat{\textbf{b}}}
    \overset{i\neq j,k}{=}
    \intS[p]{(\textbf{w}_{r})_l (\textbf{S}^{(3)}_{o})_{ijklm} n_m}
    - \intO[i]{(\textbf{S}^{(3)}_{o})_{ijklm} (\grad\textbf{u})_{lm}}\\
    - (1-\lambda) \intO[i]{((\textbf{U}^{(3)}_{i})_{ijkl} + \frac{1}{2}r_kr_l\delta_{il})(\grad^2 \textbf{u})_l }
    + \intS[p]{((\textbf{U}^{(3)}_{i})_{ijkl} + \frac{1}{2}r_kr_l\delta_{il}) b_l}\\
    + \zeta Re \intO{((\textbf{U}^{(3)}_{i})_{ijkl} + \frac{1}{2}r_kr_l\delta_{il}) (\textbf{f}_{i})_l} 
    + Re\intO[o]{(\textbf{U}^{(3)}_{i})_{ijkl}\cdot \textbf{f}_{o}}.
    \label{eq:first_formula}
\end{multline}
This formula is valid for a droplet immersed in a yet arbitrary flow of background velocity \textbf{u} and jump at the interface \textbf{b}.

In the situation in which we are interested in~\ref{sec:compute_moments} \textbf{u} is uniform and $\textbf{b}=0$ hence this formula reduce to: 
\begin{multline}
    \frac{1}{2}\intS[p]{  (\bm\sigma_{o}\cdot \textbf{n})_i r_jr_k}
    - \intO[i]{2(\textbf{e}_i )_{ij} r_k}
    % - (1-\lambda) \intO[p]{(\textbf{u}_{i} + \textbf{w}_r)\cdot \grad^2 \hat{\textbf{u}}}
    % + \intS[p]{(\textbf{u}_{i} + \textbf{w}_r)\cdot \hat{\textbf{b}}}
    \overset{i\neq j,k}{=}
    (\textbf{w}_{r})_l \intS[p]{ (\textbf{S}^{(3)}_{o})_{ijklm} n_m}
    % - \intO[i]{(\textbf{S}^{(3)}_{o})_{ijklm} (\grad\textbf{u})_{lm}}\\
    % - (1-\lambda) \intO[i]{((\textbf{U}^{(3)}_{i})_{ijkl} + \frac{1}{2}r_kr_l\delta_{il})(\grad^2 \textbf{u})_l }
    % + \intS[p]{((\textbf{U}^{(3)}_{i})_{ijkl} + \frac{1}{2}r_kr_l\delta_{il}) b_l}
    \\
    + \zeta Re \intO{((\textbf{U}^{(3)}_{i})_{ijkl} + \frac{1}{2}r_kr_l\delta_{il}) (\textbf{f}_{i})_l} 
    + Re\intO[o]{(\textbf{U}^{(3)}_{i})_{ijkl}\cdot \textbf{f}_{o}}.
    \label{eq:second_formula}
\end{multline}
Whether it is for~\ref{eq:first_formula,eq:second_formula}, note that it is only valid for $i\neq j,k$. 
Indeed, we have factor out by $(\grad\grad\hat{\textbf{u}})_{kji}$, and $(\grad\grad\hat{\textbf{u}})_{kii} = (\grad\grad\hat{\textbf{u}})_{iji} = 0$. 

Hence, if we note $K_{ijk}$ the complete second moment, then~\ref{eq:second_mom_text} only provides the deviatoric part of $K_{ijk}$, on any contraction over $i,j$ or $i,k$. 
Therefore, one still need a formula for the trace of~\ref{eq:first_formula,eq:second_formula} and find a way to add up the trace to the yet undefined traceless part (on $i,j$ and $i,k$) of $K_{ijk}$.  


\subsection{Decomposition of a third order tensor into traceless tensor and isotopic contribution}

However, the decomposition of $K_{ijk}$ into a deviatoric part and isotopic part, on only two indices is not a trivial task. 
Following the strategy of \citet{nadim1991motion} we consider an arbitrary third order tensor,
\begin{equation}
    K_{ijk},
\end{equation}
that represents the second moment of hydrodynamic forces. 

We assume that $\textbf{K}$ can be decomposed into the sum
\begin{equation}
    K_{ijk} = G_{ijk}  + (\textbf{v}^1)_k \delta_{ij} + (\textbf{v}^2)_j \delta_{ik}, 
   \label{eq:defK}
\end{equation}
where the $\textbf{v}^n$ are first order tensor left to determine, and \textbf{G} is defined as the traceless part of \textbf{K} over contractions of the pairs of indices: $ik$ and $ik$. 
That way $\textbf{v}^1$ and $\textbf{v}^2$ represent the isotopic part of \textbf{K} along the pairs of indices $ij$ and $ik$, respectively. 

To find out the expression of the $\textbf{v}^n$ in terms of components of $\textbf{K}$ we take successively the trace of~\ref{eq:defK} on $ij$ and $ik$. 
It gives the system of equations, 
\begin{align}
    K_{llk} - 3(\textbf{v}^1)_i - (\textbf{v}^2)_i &= 0,\\
    K_{lil} - (\textbf{v}^1)_i - 3(\textbf{v}^2)_i &= 0,\\
\end{align}
From which we deduce that, 
\begin{align}
    (\textbf{v}^1) &=  -K_{ljl}/8 + 3K_{llj}/8, \\
    (\textbf{v}^2) &=  3K_{lkl}/8 - K_{llk}/8.
\end{align}
Finally, we can define the traceless part of \textbf{K} (on only two pairs of indices), as
\begin{equation}
    G_{ijk} = 
    K_{ijk}
    -\left(\frac{3}{8}\right)K_{ljl}\delta_{ik} - \left(\frac{3}{8}\right)K_{llk}\delta_{ij}  + \left(\frac{1}{8}\right)K_{lkl}\delta_{ij} + \left(\frac{1}{8}\right)K_{llj}\delta_{ik}
    \label{eq:defG}
\end{equation}
Hence the formula derived for the second moment~\ref{eq:first_formula,eq:second_formula} only provide the tensor $G_{ijk}$ as it is defined in~\ref{eq:defG} not the entire second moment $K_{ijk}$. 

To find out the complete second moment one thus need formulas for $\textbf{v}^1$ and $\textbf{v}^2$ which are entirely determined by $K_{lkl}$ and $K_{llk}$, i.e.~the trace of the second moment ($K_{ijk}$) on the index $ik$ and $ij$, respectively. 


\subsection{Reciprocal theorem for the trace of the first and second moments}
Formulas for $K_{ijk}\delta_{ik}$ and $K_{ijk}\delta_{ij}$, can be obtained using the point source dipole solution (\ref{eq:pts_source} with $n=1$) or the point force solution (\ref{eq:point_force} with $n=0$). 
We start with~\ref{eq:first_step_out} integrated on the exterior of the droplet: 
\begin{equation}
    -\intS[p]{\hat{\textbf{u}}_{o} \cdot  \bm\sigma_{o}\cdot \textbf{n}}
    =
    -\intS[p]{\textbf{u}_{o} \cdot \hat{\bm\sigma}_{o}\cdot \textbf{n}}
    + 
    Re\intO[o]{\hat{\textbf{u}}_{o}\cdot \textbf{f}_{o}}.
    \label{eq:first_step_trace}
\end{equation}
Then we consider the test solutions given by~\ref{eq:pts_source} with $n=1$, and~\ref{eq:point_force} with $n=0$, which read
\begin{align}
    \textbf{u} = \frac{-1}{8\pi r}(\bm\delta + \textbf{nn}),
    && \bm\sigma\cdot \textbf{n} = \frac{6}{8\pi r^{-2}}\textbf{nn},
    \label{eq:first_expr}
    \\
    \textbf{u} = \frac{-1}{4\pi r^3}(3\textbf{nn}- \bm\delta),
    && \bm\sigma\cdot \textbf{n} = \frac{- 6}{4\pi r^4} (\bm\delta - 3\textbf{nn}),
    \label{eq:second_expr}
\end{align}
respectively. 
Inserting~\ref{eq:first_expr} and~\ref{eq:second_expr} in~\ref{eq:first_step_trace}, we get
\begin{align}
    \label{eq:trace1}
    \intS[p]{(\bm\delta + \textbf{nn}) \cdot  \bm\sigma_{o}\cdot \textbf{n}}
    =
    - 6\intS[p]{\textbf{u}_{o} \cdot \textbf{nn}}
    + 
    Re\intO[o]{(\bm\delta + \textbf{nn})r^{-1}\cdot \textbf{f}_{o}},\\
    \intS[p]{(3\textbf{nn}-\bm\delta) \cdot  \bm\sigma_{o}\cdot \textbf{n}}
    =
    6\intS[p]{\textbf{u}_{o} \cdot (\bm\delta - 3\textbf{nn})}
    + 
    Re\intO[o]{(\bm\delta - 3\textbf{nn})r^{-3}\cdot \textbf{f}_{o}},
    \label{eq:trace2}
\end{align}
respectively. 
Noting the two important identities: 
\begin{equation}
    \intS[p]{\textbf{u}_{o} \cdot \textbf{nn}}
    =
    \intO[p]{\div (\textbf{u}_{o}\textbf{n})}
    = \intO[p]{ \textbf{u}_i}
    = - \textbf{w}_r \intO[p]{},
\end{equation}
and, 
\begin{align*}
    \intS[p]{\textbf{u}_{o} \cdot (\bm\delta - 3\textbf{nn})}
    &=
    - 3 \intS[p]{\textbf{u}_{o}\cdot \textbf{nn}}
    + \intS[p]{\textbf{u}_{o} \textbf{n}\cdot \textbf{n}}\\
    &=
    \intS[p]{\textbf{u}_{o}\cdot \textbf{nn}}
    + \intS[p]{\textbf{u}_{o} \textbf{n}\cdot \textbf{n}}
    - 4 \intS[p]{\textbf{u}_{o}\cdot \textbf{nn}}\\
    &=
    \intO[p]{\grad (\textbf{u}_{i}\cdot \textbf{r})}
    + \intO[p]{\div(\textbf{n}\textbf{u}_{i})}
    - 4 \intO[p]{\div (\textbf{u}_{i}\textbf{n})}\\
    &=
    \intO[p]{\grad \textbf{u}_{i}\cdot \textbf{r}}
    + \intO[p]{\textbf{u}_{i}}
    + 3\intO[p]{\textbf{u}_{i}}
    + \intO[p]{\textbf{r}\cdot \grad\textbf{u}_{i}}
    - 4 \intO[p]{\textbf{u}_{i} }\\
    &=
    \intO[p]{\grad \textbf{u}_{i}\cdot \textbf{r}}
    + \intO[p]{\textbf{r}\cdot \grad\textbf{u}_{i}}
    \\
    &=
    \intO[p]{2 \textbf{r}\cdot \textbf{e}_i},
    \\
\end{align*}
we may reformulate~\ref{eq:trace1,eq:trace2} as,
\begin{align}
    \frac{1}{2}\intS[p]{\textbf{nn} \cdot  \bm\sigma_{o}\cdot \textbf{n}}
    =
    - \frac{1}{2}\intS[p]{ \bm\sigma_{o}\cdot \textbf{n}}
    + 3\textbf{w}_r \intO[p]{}
    + \frac{Re}{2}\intO[o]{(\bm\delta + \textbf{nn})r^{-1}\cdot \textbf{f}_{o}}
    \label{eq:trace3}
    \\
    \frac{1}{2}\intS[p]{\textbf{nn} \cdot  \bm\sigma_{o}\cdot \textbf{n}}
    - \intS[p]{2 \textbf{r}\textbf{e}_i}
    =
    \frac{1}{6}\intS[p]{\bm\sigma_{o}\cdot \textbf{n}}
    + 
    \frac{Re}{6}\intO[o]{(\bm\delta - 3\textbf{nn})r^{-3}\cdot \textbf{f}_{o}}.
    \label{eq:trace4}
\end{align}
The first term on the right-hand side of~\ref{eq:trace3,eq:trace4} is the drag forces acting upon the test droplet. 
Consequently, it can be substituted by its own reciprocal relation~\eqref{eq:drag_force}. Doing so gives, 
\begin{multline}
    \frac{1}{2}\intS[p]{\textbf{nn} \cdot  \bm\sigma_{o}\cdot \textbf{n}}
    =
    - \frac{1}{2}
    \textbf{u}_{r}\cdot  \intS[p]{\textbf{S}_{o}^{(1)}\cdot \textbf{n}}
    + 3\textbf{w}_r \intO[p]{}+ \frac{Re}{2}\intO[o]{(\bm\delta + \textbf{nn})r^{-1}\cdot \textbf{f}_{o}}\\
    + \frac{(1-\lambda) }{2} \intO{(\textbf{U}_{i}^{(1)} + \bm\delta)\cdot \grad^2\textbf{u}} 
    - \frac{1}{2} \intO{(\textbf{U}_{i}^{(1)} + \bm\delta)\cdot \textbf{b}} \\
    - \zeta \frac{Re}{2} \intO{(\textbf{U}_{i}^{(1)} + \bm\delta)\cdot \textbf{f}_{i}} 
    - \frac{Re}{2}\intO[o]{\textbf{U}_{o}^{(1)}\cdot \textbf{f}_{o}}\\
    \label{eq:traceIJ}
\end{multline}
\begin{multline}
    \frac{1}{2}\intS[p]{\textbf{nn} \cdot  \bm\sigma_{o}\cdot \textbf{n}}
    - \intS[p]{2 \textbf{r}\cdot \textbf{e}_i}
    =
    +\frac{1}{6}
    \textbf{u}_{r}\cdot  \intS[p]{\textbf{S}_{o}^{(1)}\cdot \textbf{n}}
    + \frac{Re}{6}\intO[o]{(\bm\delta - 3\textbf{nn})r^{-3}\cdot \textbf{f}_{o}}\\
    - \frac{(1-\lambda)}{6} \intO{(\textbf{U}_{i}^{(1)} + \bm\delta)\cdot \grad^2 \textbf{u}} 
    + \frac{1}{6} \intO{(\textbf{U}_{i}^{(1)} + \bm\delta)\cdot \textbf{b}} 
    + \zeta \frac{Re}{6} \intO{(\textbf{U}_{i}^{(1)} + \bm\delta)\cdot \textbf{f}_{i}} 
    + \frac{Re}{6}\intO[o]{\textbf{U}_{o}^{(1)}\cdot \textbf{f}_{o}}.
    \label{eq:traceIK}
\end{multline}
\subsection{Summary}

We first derived~\ref{eq:first_formula}, it was explained that if, $K_{ijk}$ where the whole second moment of forces, then~\ref{eq:first_formula} would represent the tensor $G_{ijk}$ in the decomposition:
\begin{equation}
    K_{ijk}
    = G_{ijk} 
    + \frac{1}{8}(3K_{ljl}
    - K_{llj})\delta_{ik}
    + \frac{1}{8}(3K_{llk} 
    - K_{lkl})\delta_{ij} 
    \label{eq:defK2}
\end{equation}
Hence, to obtain the total second moment of forces, one must then add the traces $K_{llj}$ and $K_{ljl}$ to $G_{ijk}$. 
In the previous section, we derived a formula for $K_{ijk}\delta_{ij}$ which is given by~\ref{eq:traceIJ}, and a second one for $K_{ijk}\delta_{ij}$~\eqref{eq:traceIK}.
Therefore, combining~\ref{eq:first_formula,eq:traceIJ,eq:traceIK} following the decomposition given by~\ref{eq:defK2}, one directly obtains a formula for the complete second moment of forces. 


\section{Proof of the average of the Oseen force. }
\label{ap:varience}

In this appendix, we provide a justification for~\ref{eq:standard_dev2}.
Firstly, we write: 
\begin{equation}
    n_p[\textbf{x},t] \int_{\mathbb{R}^3} 
    |\textbf{w}_r| \textbf{w}_r
    P
    d\textbf{w}_r
    =
    n_p  
    |\textbf{u}_r|\textbf{u}_r
    + n_p  \int_{\mathbb{R}^3} 
     (|\textbf{w}_r | - |\textbf{u}_r|)\textbf{w}_r
    P
    d\textbf{w}_r,
    \label{eq:first_step_appendix}
\end{equation}
and focus on the second term on the right-hand side of this equation. 
Provided that the deviation of \textbf{w} around the mean velocity $\textbf{u}_p$ is small, one can use a Taylor expansion of the function $|\textbf{w}_r|$ around $|\textbf{u}_r|$, and write:
\begin{equation}
    |\textbf{w}_r| =   
    |\textbf{w}_r|
    + (\textbf{w}_r - \textbf{u}_r)\cdot \left. \frac{\partial  |\textbf{w}_r|}{\partial \textbf{w}_r} \right|_{\textbf{w}_r = \textbf{u}_r}
    + (\textbf{w}_r - \textbf{u}_r)(\textbf{w}_r - \textbf{u}_r) : \left. \frac{\partial  |\textbf{w}_r|}{\partial \textbf{w}_r\partial \textbf{w}_r}  \right|_{\textbf{w}_r = \textbf{u}_r} \ldots
\end{equation}
The partial derivatives read, 
\begin{align}
    \grad|\textbf{x}| &=\textbf{x} |\textbf{x}|^{-1},\\
    \grad\grad|\textbf{x}| &= \bm\delta |\textbf{x}|^{-1} - \textbf{xx} |\textbf{x}|^{-3},
\end{align}
so that the final expression is:  
\begin{equation}
    |\textbf{w}_r| 
    - |\textbf{u}_r|
    =
    (\textbf{w}_r - \textbf{u}_r)\cdot \textbf{u}_r |\textbf{u}_r|^{-1}
    +(\textbf{w}_r - \textbf{u}_r)(\textbf{w}_r - \textbf{u}_r):
    (\bm\delta |\textbf{u}_r|^{-1} - \textbf{u}_r\textbf{u}_r|\textbf{u}_r|^{-3}). 
\end{equation}
Inserting this expression into~\ref{eq:first_step_appendix}, and integrating overall $\textbf{w}$, while noting that $\textbf{w}_r - \textbf{u}_r = -\textbf{w} + \textbf{u}_p = -\textbf{w}'$ is the fluctuation of the center of mass velocity around $\textbf{u}_p$, gives: 
\begin{multline*}
    n_p \int_{\mathbb{R}^3}(|\textbf{w}_r|  - |\textbf{u}_r|) \textbf{w}_r P d\textbf{w} 
    % = 
    % n_p \int_{\mathbb{R}^3}
    % [
    % -\textbf{w}' \cdot \textbf{u}_r |\textbf{u}_r|^{-1}
    % +\textbf{w}' \textbf{w}' :
    % (\bm\delta |\textbf{u}_r|^{-1} - \textbf{u}_r\textbf{u}_r|\textbf{u}_r|^{-3})
    % ] \textbf{w}_r P d\textbf{w} \\
    =
    n_p \int_{\mathbb{R}^3}
    [
    -\textbf{w}' \cdot \textbf{u}_r |\textbf{u}_r|^{-1}
    +\textbf{w}' \textbf{w}' :
    (\bm\delta |\textbf{u}_r|^{-1} - \textbf{u}_r\textbf{u}_r|\textbf{u}_r|^{-3})
    ] \textbf{u}_r P d\textbf{w}\\
    +
    n_p \int_{\mathbb{R}^3}
    [
    +\textbf{w}' \textbf{w}' \cdot \textbf{u}_r |\textbf{u}_r|^{-1}
    -\textbf{w}' \textbf{w}' \textbf{w}' :
    (\bm\delta |\textbf{u}_r|^{-1} - \textbf{u}_r\textbf{u}_r|\textbf{u}_r|^{-3})
    ]  P d\textbf{w}.
\end{multline*}
Finally, by using the relation,
\begin{align*}
    n_p \int_{\mathbb{R}^3}\textbf{w}'\textbf{w}' P d\textbf{w} 
    &=\pavg{\textbf{u}_\alpha'\textbf{u}_\alpha'}, 
\end{align*}
and neglecting the triple covariance terms, we get, 
\begin{equation}
    n_p  \int_{\mathbb{R}^3} 
    |\textbf{w}_r| \textbf{w}_r
    P(\textbf{w})
    d\textbf{w}
    =
    n_p  
    |\textbf{u}_r|\textbf{u}_r + 
    \pavg{\textbf{u}_\alpha'\textbf{u}_\alpha'}:
    (\bm\delta \textbf{p} +\textbf{p} \bm\delta  - \textbf{ppp})
    +O(\pavg{(\textbf{u}_\alpha')^{(n)}}),
\end{equation}
with $\textbf{p} = \textbf{u}_r|\textbf{u}_r|^{-1}$, and $n$ an integer $\ge 3$, which proves~\ref{eq:standard_dev2}. 

\section{Ossen solution proper for quadratic background flow}

The equations to be solved are, 
\begin{align}
    \div \textbf{u} &= 0
    \\
    -\grad p + \grad^2 \textbf{u}
    &= 
    Re [
        % \pddt \textbf{u}
        + \textbf{u}\cdot (\textbf{r}\cdot \grad)\grad \textbf{u}_r
        + (\textbf{u} + \frac{1}{2}(\textbf{rr}:\grad\grad)\textbf{u}_r) \cdot \grad \textbf{u}
        ]
\end{align}
where we considered $\textbf{u}_r = \textbf{e}$ a constant unit vector. 
The Stokes flow solution predict a $\textbf{u}\sim r^{-1}\grad\grad \textbf{u}_r$, hence the left-hand side terms scale as, $O(r^{-3})$ and the right-hand side terms as, $O(Re)$, $O(Re r^{-3})$, and $O(Re)$. 
Consequently, the left-hand side term is equal to the right-hand side terms for, $Re = 1$, and $r = Re^3$. 
But since $Re$ is already so small $Re^3$ is small and hence this term becomes important quickly. 

Hence, far from the droplet, let say at $r = O(Re^{-1})$ we have the following estimate, 
\begin{align}
    -\grad p + \grad^2 \textbf{u} = O(Re^3) &&
    Re \pddt \textbf{u} = St O(Re^2) && 
    Re \textbf{u}\cdot \grad \textbf{u} = O(Re^4) \\
    Re \textbf{u}\cdot (\textbf{r}\cdot \grad)\grad \textbf{u}_r = O(Re^3) 
    &&
    Re \textbf{e}\cdot \grad \textbf{u} = O(Re^3) 
    &&
\end{align}



\section{Ossen solution proper}

The equations to be solved are, 
\begin{align}
    \div \textbf{u} &= 0
    \\
    -\grad p + \grad^2 \textbf{u}
    &= 
    Re [
    \pddt \textbf{u}
    + (\textbf{u}+ \textbf{e})\cdot \grad \textbf{u}]
\end{align}
where we considered $\textbf{u}_r = \textbf{e}$ a constant unit vector. 
The Stokes flow solution predict a $\textbf{u}\sim r^{-1}$, hence the left-hand side terms scale as, $O(r^{-3})$ and the right-hand side terms as, $O(Re r^{-1})$, $O(Re r^{-3})$, and $O(Re r^{-2})$ respectively. 
Consequently, the left-hand side term is equal to the right-hand side terms for, $r =Re^{-1/2}$, $Re = 1$, and $r = Re^{-1}$. 

Hence, far from the droplet, let say at $r = O(Re^{-1})$ we have the following estimate, 
\begin{align}
    -\grad p + \grad^2 \textbf{u} = O(Re^3) &&
    Re \pddt \textbf{u} = St O(Re^2) && 
    Re \textbf{u}\cdot \grad \textbf{u} = O(Re^4) && 
    Re \textbf{e}\cdot \grad \textbf{u} = O(Re^3) 
\end{align}
meaning that the advective term can be neglected in that region. 
We decide to neglect the unsteadiness of the flow at this stage and so it leads to Ossen equation: 
\begin{align}
    \div \textbf{u} &= 0
    \\
    -\grad p + \grad^2 \textbf{u}
    &= 
    Re  \textbf{e} \cdot \grad \textbf{u} 
    \text{  valid for } Re < 1\;  r = O(Re^{-1})
    \label{eq:ossen_eq}
\end{align}
We consider the change of variables, $\wt{r} = Re r$, when $r$ is the Ossen length, $r=Re^{-1}$, we have $\wt{r} = 1$,
In stretched coordinates the above equation becomes, 
\begin{align}
     \wt{\grad}\cdot  \textbf{u} &= 0
    \\
    - 
    \wt{\grad} \wt{p} +  \wt{\grad}^2 \textbf{u}
    &= 
    \textbf{e} \cdot \wt{\grad} \textbf{u}, 
    \label{eq:ossen_eq_streatching}
\end{align}
which holds for $\wt{r} = 1$, with $p Re^{-1}= \wt{p}$. 
Because, $d^3\wt{\textbf{r}}= Re^3 d^3 \textbf{r}$ we have $ \delta(\wt{\textbf{r}}) = \delta(\textbf{r})/Re^3$. Hence a point force may be applied on \ref{eq:ossen_eq} or \label{eq:ossen_eq_streatching} with a ratio of $Re^3$. 

For simplicity, we consider the forced Ossen equation in regular coordinates. And set the matching condition by including a point force at the origin. 
\begin{align}
    -\grad p + \grad^2 \textbf{u}
    + \textbf{f}\delta(\textbf{r})
    &= 
    Re  \textbf{e} \cdot \grad \textbf{u} 
\end{align}
where \textbf{f} is the intensity of the point force generated by the droplet, made dimensionless by $a^2 / |\textbf{u}_r| \mu_f$, hence 
\begin{equation}
    \textbf{f}= \intS[p]{\bm\sigma^{(0)}\cdot \textbf{n}}=f \textbf{e} = \frac{2\pi(3\lambda+2)}{\lambda+1}\textbf{e}
\end{equation}
where $\bm\sigma^{(0)}$ corresponds to the stokes flow condition of the stress, in the same configuration. 

Using the fact that $\delta(\textbf{r}) = -\grad^2 (1/4\pi r)$ and $\div \textbf{u}=0$ we can eliminate the velocity from the momentum equation to find, 
\begin{align}
    p 
    &= 
    -\frac{1}{4\pi} (\textbf{f} \cdot \grad)  (1/r)
\end{align}
The momentum equation is then, 
\begin{equation}
    (\grad^2 
    - Re  \textbf{e} \cdot \grad) \textbf{u} 
    + \frac{1}{4\pi}\textbf{f}\cdot [\grad\grad - \bm\delta \grad^2](1/r)
    = 0
\end{equation}
This encourages us to write \textbf{u} as, 
\begin{equation}
    \textbf{u} = \textbf{f}\cdot [\grad\grad - \bm\delta \grad^2]H
\end{equation}
where $H$ is a yet unknown scalar to be determined. 
It is governed by the equation: 
\begin{equation}
    (\grad^2 
    - Re  \textbf{e} \cdot \grad) H 
    + \frac{1}{4\pi r}
    = 0
\end{equation}
Setting $Q=\grad^2 H$ and taking the divergence twice on this equation gives, 
\begin{equation}
    (\grad^2 
    - Re  \textbf{e} \cdot \grad) Q
    = 
    \grad^2\frac{1}{4\pi r}= \delta(\textbf{r})
\end{equation}
We then assume that $Q =e^{Re/2  \textbf{e}\cdot \textbf{r}} Q^* $, where $Q^*$ is a scalar function satisfying,
\begin{equation} 
    [\grad^2-(\frac{Re}{2})^2] Q^* 
    =
    \delta(\textbf{r})
    \Longleftrightarrow
    Q = - \frac{e^{\frac{Re}{2}(\textbf{e}\cdot \textbf{r} - r)}}{4\pi r}
    = - \frac{e^{X}}{4\pi r}
\end{equation}
where we have set $X = Re/2 (\textbf{e}\cdot \textbf{r}-r)$. 
To find $H$ and then $\textbf{u}$ we need to solve at least for  $Q = \grad^2  H$ 
This is done by assuming that $H$ is a function of the variable $X$ alone, in which case $\grad H = Re/2(\textbf{e} - \textbf{n}) \partial_X H$ because $\grad X =Re/2 (\textbf{e}-\textbf{n})$. We also note that $\grad \grad X = Re/2 \grad (\textbf{e}-\textbf{n}) = Re/2 (\textbf{nn}- \bm\delta)r^{-1}$. 
Consequently, 
\begin{multline}
    Q = \grad^2 H 
    = 
    \grad^2 X \partial_X H
    + \grad X \cdot \grad X \partial_X^2 H\\
    = - Re r^{-1} \partial_X H
    - X Re  \frac{1}{r} \partial_X^2 H
    = \frac{- Re }{r}\partial_X(X\partial_X H)
    = \frac{- Re }{r}\partial_X(X\partial_X H)
\end{multline}
So that,
\begin{equation}
    X \partial_X H =\int_0^X  \frac{e^{X'}}{Re 4\pi} dX' = \frac{e^X- 1}{4\pi Re}
    \Longleftrightarrow 
    \grad H = \frac{e^X- 1}{8\pi X} (\textbf{e}-\textbf{n})
\end{equation}
One can verify that, 
\begin{align}
    \grad\cdot \grad H 
    &=
    \grad \frac{e^X- 1}{8\pi X} (\textbf{e}-\textbf{n})\\
    &=
    -2\frac{e^X- 1}{8\pi X r} 
    +
    \frac{e^X(X-1)+ 1}{16 \pi X^2 }Re(\textbf{e}-\textbf{n})\cdot (\textbf{e}- \textbf{n})\\
    &=
    -\frac{e^X- 1}{4\pi X r} 
    +
    Re\frac{e^X(X-1)+ 1}{8 \pi X^2 }
    -
    Re\frac{e^X(X-1)+ 1}{8 \pi X^2 } \textbf{e}\cdot \textbf{n}\\
    &=
    -\frac{e^X- 1}{4\pi X r} 
    +
    Re\frac{e^X(X-1)+ 1}{8 \pi X^2 }
    -
    Re\frac{e^X(X-1)+ 1}{8 \pi X^2 } (2X/r + 1)\\
    &=
    -
    \frac{e^X}{4 \pi  r} \\
\end{align}
The ossenlet is then given by, 
\begin{equation}
    \textbf{O}
    = [\grad\grad - \bm\delta\grad^2] H 
    = 
    \frac{e^{X}}{4\pi r}\bm\delta
    +\grad \left(\frac{e^X- 1}{8\pi X} (\textbf{e}-\textbf{n})\right)
\end{equation}
Notting that,
\begin{align}    
    \grad (\textbf{e}- \textbf{n}) = (\textbf{nn} - \bm\delta ) r^{-1}\\
    \grad (\textbf{nn} - \bm\delta ) r^{-1} = (\bm\delta \textbf{n}+ \bm\delta \textbf{n}^\dagger + \textbf{n}\bm\delta - 3 \textbf{nnn}) r^{-2}\\
    \grad \frac{e^X- 1}{X} = \frac{e^X(X-1)+1}{X^2}\grad X
    = \frac{e^X(X-1)+1}{X^2} Re/2 (\textbf{e}- \textbf{n})\\
    \grad \frac{e^X(X-1)+1}{X^2} 
    % = \frac{(2-2X+X^2) e^X-2}{X^3} \grad X
    = \frac{(2-2X+X^2) e^X-2}{X^3} Re/2 (\textbf{e}- \textbf{n})\\
    % \grad (\textbf{nn} - \bm\delta ) r^{-1} = - 3 \textbf{nnn} r^{-2} + (\bm\delta \textbf{n}+ \bm\delta \textbf{n}^\dagger + \textbf{n}\bm\delta) r^{-2}
\end{align}
we get
\begin{equation}
    \textbf{O} =
    \frac{e^X}{4\pi r}\bm\delta
    + \frac{e^X-1}{X 8 \pi r}  ( \textbf{nn} -\bm\delta)
    + 
    Re (\textbf{e} - \textbf{n})
    (\textbf{e} - \textbf{n})
    \frac{(X-1) e^X + 1}{X^2 16 \pi}
\end{equation}
The gradient of 
\begin{multline}
    \grad \textbf{O} =
    - \frac{e^X}{4\pi r^2}\textbf{n} \bm\delta
    + Re\frac{e^X}{8\pi r}(\textbf{e}-\textbf{n})\bm\delta
    + \frac{e^X-1}{X 8 \pi r^2} (\bm\delta \textbf{n}+ \bm\delta \textbf{n}^\dagger + \textbf{n}\bm\delta - 3 \textbf{nnn})  \\
    + Re\frac{e^X(X-1)+1}{16 \pi X^2 r}  (\textbf{e}- \textbf{n})(\textbf{nn} -\bm\delta)\\
    + 
    Re [
        (\textbf{nn} -\bm\delta)
        (\textbf{e} - \textbf{n})
        + 
        (\textbf{e} - \textbf{n})
        (\textbf{nn} -\bm\delta)
        ]
    \frac{(X-1) e^X + 1}{X^2 16 \pi r} \\
    + 
    Re^2 
    (\textbf{e} - \textbf{n})
    (\textbf{e} - \textbf{n})
    (\textbf{e} - \textbf{n})
    \frac{(2-2X +X^2) e^X - 2}{X^3 32 \pi}
\end{multline}
\begin{multline}
    \div\textbf{O} =
    - \frac{e^X}{4\pi r^2}\textbf{n} 
    + Re\frac{e^X}{8\pi r}(\textbf{e}-\textbf{n})
    + \frac{e^X-1}{X 4 \pi r^2} \textbf{n}  \\
    +3 Re\frac{e^X(X-1)+1}{16 \pi X^2 r} (\textbf{e}\cdot \textbf{nn} -\textbf{e})\\
    % + 
    % Re [
    %     (\textbf{nn} -\bm\delta)
    %     \cdot \textbf{e} 
    %     + 
    %     \textbf{e}\cdot 
    %     (\textbf{nn} -\bm\delta)
    %     ]
    % \frac{(X-1) e^X + 1}{X^2 16 \pi r} \\
    + 
    Re^2 
    (2 -2 \textbf{e}\cdot \textbf{n})
    (\textbf{e} - \textbf{n})
    \frac{(2-2X +X^2) e^X - 2}{X^3 32 \pi}
\end{multline}
\begin{multline}
    \div\textbf{O} =
    - \frac{e^X}{4\pi r^2}\textbf{n} \cdot \textbf{e}
    + Re\frac{e^X}{8\pi r}(1-\textbf{n}\cdot \textbf{e})
    + \frac{e^X-1}{X 4 \pi r^2} \textbf{n}\cdot \textbf{e}  \\
    +3 Re\frac{e^X(X-1)+1}{16 \pi X^2 r} ((\textbf{e}\cdot \textbf{n})^2 -1)\\
    % + 
    % Re [
    %     (\textbf{nn} -\bm\delta)
    %     \cdot \textbf{e} 
    %     + 
    %     \textbf{e}\cdot 
    %     (\textbf{nn} -\bm\delta)
    %     ]
    % \frac{(X-1) e^X + 1}{X^2 16 \pi r} \\
    + 
    Re^2 
    (2 -2 \textbf{e}\cdot \textbf{n})
    (1 - \textbf{n}\cdot \textbf{e})
    \frac{(2-2X +X^2) e^X - 2}{X^3 32 \pi}
\end{multline}
\begin{multline}
    \div\textbf{O} =
    - \frac{e^X}{4\pi r^2}\textbf{n} \cdot \textbf{e}
    + Re\frac{e^X}{8\pi r}(1-\textbf{n}\cdot \textbf{e})
    + \frac{e^X-1}{X 4 \pi r^2} \textbf{n}\cdot \textbf{e}  \\
    +3 Re\frac{e^X(X-1)+1}{16 \pi X^2 r} ((\textbf{e}\cdot \textbf{n})^2 -1)\\
    % + 
    % Re [
    %     (\textbf{nn} -\bm\delta)
    %     \cdot \textbf{e} 
    %     + 
    %     \textbf{e}\cdot 
    %     (\textbf{nn} -\bm\delta)
    %     ]
    % \frac{(X-1) e^X + 1}{X^2 16 \pi r} \\
    + 
    Re^2 
    (2 -2 \textbf{e}\cdot \textbf{n})
    (1 - \textbf{n}\cdot \textbf{e})
    \frac{(2-2X +X^2) e^X - 2}{X^3 32 \pi}
\end{multline}
Let $X = r/2 Re (\textbf{e}\cdot \textbf{n}-1)$ so $\textbf{e}\cdot \textbf{n}= 2 X Re/r+1$ and $(\textbf{e}\cdot \textbf{n})^2-1= (2 X Re/r)^2+4 X Re/r$. 
multiplying by \textbf{e} gives, 
Or in terms of stretched coordinates $\wt{r}\to r Re$,  $X = \wt{r}/2 (\textbf{e}\cdot \textbf{n}- 1)$,

\subsubsection{taylor exp}
For small $Re$, i.e. we can carry a taylor exp, $(e^X - 1)/X \approx 1+ X/2$, 
\begin{align*}
    \textbf{O}
    &= 
    \frac{1+X}{4\pi r}\bm\delta
    +\grad \left(\frac{1+X/2}{8\pi} (\textbf{e}-\textbf{n})\right)\\
    &= 
    \frac{1+X}{4\pi r}\bm\delta
    +\frac{1+X/2}{8\pi} (\textbf{nn}-\bm\delta) r^{-1}
    +\frac{Re}{32\pi} (\textbf{e}-\textbf{n})(\textbf{e}-\textbf{n})\\
    &= 
    \frac{1}{8\pi} (\textbf{nn}+\bm\delta) r^{-1}
    % +\frac{4X}{16\pi r}\bm\delta
    +\frac{Re}{32\pi} [(\textbf{e}\cdot \textbf{n} - 1) (\textbf{nn}+3\bm\delta)
    +(\textbf{e}-\textbf{n})(\textbf{e}-\textbf{n})]
\end{align*}
or for small $\wt{r}$ with $\wt{r} C\to X$, 
\begin{align}
    \textbf{O}Re = 
    &= 
    \frac{1+C \wt{r}}{4\pi \wt{r}}\bm\delta
    + \frac{1 + C\wt{r}/2}{8 \pi \wt{r}}  (\textbf{nn} -\bm\delta)
    + 
    (\textbf{e} - \textbf{n})
    (\textbf{e} - \textbf{n})
    \frac{1/2+\wt{r} C/3}{16 \pi}\\
    &= 
    \frac{1}{8\wt{r}}(\bm\delta+\textbf{nn})
    + \frac{1}{16\pi} (C(3\bm\delta + \textbf{nn})
    + (\textbf{e} - \textbf{n})
    (\textbf{e} - \textbf{n})/2
    )
    + \wt{r}\ldots
\end{align}
\tb{here we clearly see that the divergence doesn't compute with the ``taking the serie'' operator}



\subsection{Solving the inner problem}

Carrying the inner expansion such that $(p,\textbf{u}) = (p_0,\textbf{u}_0)+ Re (p_1,\textbf{u}_1)$ gives,
In steady state,
\begin{align}
    \div \textbf{u}_1 &= 0
    \\
    -\grad p + \grad^2 \textbf{u}_1
    = 
    (\textbf{u}_0+ \textbf{e})\cdot \grad \textbf{u}_0
\end{align}
with the BC, 
\begin{align}
    \textbf{u}_1 \cdot \textbf{n} = 0 
    &&
    (\textbf{u}_i)_1  =(\textbf{u}_o)_1
    &&
    \textbf{n}\cdot [(\textbf{e}_o)_1 - \lambda(\textbf{e}_i)_1]\cdot (\bm\delta - \textbf{nn})
    =0 \\
    \lim_{r\to \infty }(p_1,\textbf{u}_1)= (0,\textbf{O}_1 \cdot \textbf{e}f)
\end{align}
Then we split $(p_1,\textbf{u}_1)$ into an homogeneous solution satisfying the boundary at the surface of the droplet and $0$ at infinity and another forced solution. 

Setting $\textbf{u}_1 \to \textbf{u}_1 - \textbf{O}_1\cdot \textbf{e}$ for the forced solution yields the eq 
\begin{align}
    -\grad p 
    + \grad^2 \textbf{u}_1
    = 
    - \grad^2 \textbf{u}_{out}
    + (\textbf{u}_0 + \bm\delta) \cdot \grad \textbf{u}_0= \textbf{A}(\textbf{r}): \textbf{ee}\\
    -\grad^2  p 
    = 
    + \grad\textbf{u}_0 : \grad \textbf{u}_0
    = \textbf{B}(\textbf{r}): \textbf{ee}
    \\
    \lim_{r\to\infty} (p_1 \textbf{u}_1) = (0,0)
\end{align}
with the condition that it vanish at infinity. 
\begin{equation}
    \frac{1}{2}\div\grad (\textbf{x} p )= \frac{1}{2}\div(\bm\delta  p + \textbf{x}\grad p)
    =\frac{1}{2}(\grad  p + \grad p + \textbf{x}\grad^2 p)
    =\grad  p +\frac{1}{2}\textbf{x}\grad^2 p
\end{equation}
because the pressure is not harmonics that strategy of setting $x\grad p$ into the vel doesn't work. 

Because of the present symmetry we can guess that $p = \textbf{P}^{(2)}:\textbf{ee}$ and $\textbf{U}^{(2)}:\textbf{ee}$. 
Both field satify,
\begin{align}
    -\grad^2 \textbf{P}^{(2)} = \textbf{B}(\textbf{r})\\
    \grad^2 \textbf{U}^{(2)} = \textbf{A}(\textbf{r})+\grad \textbf{P}^{(2)}+
\end{align}
Because the problem is spherically symmetric we may assume that the derivatives according to the polar and azimuthal angles are zero hence, 
\begin{align}
    \frac{1}{r^2}\partial_r (r^2 \partial_r \textbf{P}^{(2)}) = -\textbf{B}(\textbf{r})
    \Longrightarrow 
    \partial_r (r^2 \partial_r \textbf{P}^{(2)}) = -\int_0^r \frac{1}{r^2}\int_0^r (r')^2 \textbf{B}(\textbf{n} r') dr' dr
    \\
    \grad^2 \textbf{U}^{(2)} = \textbf{A}(\textbf{r})+\grad P^{(2)}+
\end{align}








\paragraph{green function approach}
Where $\textbf{A}(\textbf{r})$ is a function of space. 
The divergence of $(\textbf{u}_0 \cdot \grad) \textbf{u}_0$ gives, $\grad \textbf{u}_0 : \grad \textbf{u}_0$ hence the pressure equaitons reads, 
Because,
\begin{equation}
    \textbf{A}(\textbf{r}) = \int \delta(\textbf{r}' - \textbf{r}) \textbf{A}(\textbf{r}') d^3 \textbf{r}',
\end{equation}
we deduce that the velocity fields is given by 
\paragraph{first idea as solution}
\begin{align}
    \textbf{u}_1(\textbf{r})
    &=\textbf{ee}:\int \mathcal{G}(\textbf{r}' - \textbf{r}) \textbf{A}(\textbf{r}') d^3 \textbf{r}'\\
    &=
    \textbf{ee}:  \mathcal{G}( \textbf{r})
    \int  \textbf{A}(\textbf{r}') d^3 \textbf{r}'
    + \textbf{ee}:  \grad \mathcal{G}( \textbf{r})\cdot 
    \int \textbf{r}' \textbf{A}(\textbf{r}') d^3 \textbf{r}'
\end{align}
\tb{these integral divergeges .. }
where $\mathcal{G}$ is the GF centered at $\textbf{r}'$ and evaluated at \textbf{r} (which is the point where the velocity is evaluated). 
Given that, 
\begin{equation}
    \mathcal{G}(\textbf{r}'-\textbf{r}=\textbf{r}_o) = \frac{1}{8\pi}(\bm\delta - \textbf{nn})r_o^{-1}
\end{equation}
\paragraph{find solution as}

Because the problem is spherically symmetric is may only depend on \textbf{r} and $r$. 
\begin{equation}
    p = \textbf{r}^{(n)}_{i_1i_2i_3\ldots i_n}r^{-m}
\end{equation}
\begin{equation}
    \partial_k 
    \textbf{r}^{(n)}_{i_1i_2i_3\ldots i_n}r^{-m}
    =
    \sum_{e=1}^{n} \textbf{r}^{(n-1)}_{i_1i_2i_3\ldots i_n}
\end{equation}

\begin{equation}
    P_{i_1\ldots i_n} = r^m \prod_{e=1}^{n} x_{i_e} 
\end{equation}
\begin{align*}
    \partial_k P_{i_1\ldots i_n} = 
    r^m
    \sum_{l=1}^n \delta_{ki_l}\prod_{e\neq k}^{n} x_{i_e} 
    + m r^{m-2} x_k  \prod_{e=1}^{n} x_{i_e} \\
\end{align*}



\section{Ossen force for a translating droplet}
\begin{equation}
    \intS[p]{\bm\sigma_{o}\cdot \textbf{n}}
    % - \intO[i]{2\textbf{e}_i : \grad\hat{\textbf{u}}}
    % - (1-\lambda) \intO[p]{(\textbf{u}_{i} + \textbf{u}_r)\cdot \grad^2 \hat{\textbf{u}}}
    % + \intS[p]{2(\textbf{u}_{i} + \textbf{u}_r)\cdot \hat{\textbf{b}}}
    =
    \intS[p]{\textbf{u}_{r}\cdot \textbf{S}_{o}^{(1)}\cdot \textbf{n}}
    % - \intO[i]{ \textbf{S}_{i}^{(1)} :\grad\textbf{u}}
    % - (1-\lambda) \intO[p]{(\textbf{U}_{i}^{(1)} + \bm\delta) \cdot \grad^2 \textbf{u} }\\
    % + \intS[p]{2 (\textbf{U}_{i/o}^{(1)} + \bm\delta) \cdot  \textbf{b}}
    + \zeta Re \intO{(\textbf{U}_{i}^{(1)} + \bm\delta)\cdot \textbf{f}_{i}}
    + Re\intO[o]{\textbf{U}_{o}^{(1)}\cdot \textbf{f}_{o}},
\end{equation}

In case of pure translation the yet unknown forcing term,
\begin{equation}
    \textbf{f}_{i/o} =
    \pddt \textbf{u}_{i/o}
    + (\textbf{u}_{i/o}+\textbf{u}_r)\cdot \grad \textbf{u}_{i/o}.
    % + \textbf{u}_{i}\cdot \grad \textbf{u}_r
    % + \textbf{u}_r \cdot \grad \textbf{u}_{i}
\end{equation}
To determine the Stresslet at $O(Re)$ accurate one needs to determine the integral on the right-hand-side of \ref{eq:first_mom_trans,eq:first_mom_trans2,eq:first_mom_trans3} accurate at $O(1)$ in $Re$.
At this order $\textbf{f}_{i/o}$ can be simply estimated using the Stokes flow solution for $\textbf{u}_{i/o}$ around a translating sphere.
More precicely, at distances $r$ sufficiently low  $\textbf{u}_o$ may be approximated by the solution of Stokes equation, and at $r$ sufficiently large $\textbf{u}_o$ may be approximated by the solution of Ossen equation.

Indeed, for $r < Re^{-1}$ and for $r > Re^{-1}$, $\textbf{u}_o$ is governed at $O(1)$ in $Re$,  as
\begin{align*}
    - \grad p_f
    + \grad^2 \textbf{u}_{o/i}
    &= 0 \\
    - \wt{\grad} \wt{p}_f
    + \wt{\grad}^2 \wt{\textbf{u}}_{o}
    &= \textbf{u}_r\cdot \wt{\grad} \wt{\textbf{u}}_{o}
\end{align*}
where the $\wt{\ldots}$ indicate that we are working in terms of stretched coordinates.
wThe advective term can then be written,
\begin{align}
    \intO[o]{\textbf{U}_{o}^{(1)}\cdot \textbf{f}_{o}}
    &=
    \intO[in]{\textbf{U}_{o}^{(1)}\cdot (\pddt \textbf{u}_{o}^{(0)}
    + (\textbf{u}_{o}^{(0)}+\textbf{u}_r)\cdot \grad \textbf{u}_{o}^{(0)})}\\
    &+\intO[out]{\textbf{U}_{o}^{(1)}\cdot (\pddt \textbf{u}_{o}^{out}
    + (\textbf{u}_{o}^{out}+\textbf{u}_r)\cdot \grad \textbf{u}_{o}^{out})}
    % \textbf{f}_{i/o} =
    % \pddt (\textbf{u}_{i/o}^{(0)} + Re\wt{\textbf{u}}_{i/o})
    % + (\textbf{u}_{i/o}^{(0)}+Re\wt{\textbf{u}}_{o}+\textbf{u}_r)\cdot \grad \textbf{u}_{i/o}^{(0)}.
    % + Re (\textbf{u}_{i/o}^{(0)}+Re\wt{\textbf{u}}_{o}+\textbf{u}_r)\cdot \grad \wt{\textbf{u}}_{o}.
    % + \textbf{u}_{i}\cdot \grad \textbf{u}_r
    % + \textbf{u}_r \cdot \grad \textbf{u}_{i}
\end{align}
The tensor reads,
\begin{align*}
    (\textbf{P}_o^{(1)})_{k_1}
    &=
    \frac{3\lambda +2}{2(\lambda+1)}\partial_{k_1}r^{-1}\\
    (\textbf{U}_o^{(1)})_{i k_1}
    &=
    x_i \frac{3\lambda +2}{4(\lambda+1)}\partial_{k_1}r^{-1}
    - \frac{3\lambda +2}{4(\lambda+1)}\delta_{ik_1}r^{-1}
    +\frac{\lambda}{4(\lambda+1)} \partial_i \partial_{k_1} r^{-1}
\end{align*}
In stretched coordinates the force reads,
\begin{align*}
    % (\textbf{P}_o^{(1)})_{k_1}
    % &=
    % \frac{3\lambda +2}{2(\lambda+1)}\partial_{k_1}r^{-1}\\
    \frac{1}{Re}(\textbf{U}_o^{(1)})_{i k_1}
    &=
     \frac{3\lambda +2}{4(\lambda+1)} (\hat{x}_i\hat{\partial}_{k_1}\hat{r}^{-1}
    - \delta_{ik_1}\hat{r}^{-1})
    + Re^2\frac{\lambda}{4(\lambda+1)} \hat{\partial}_i \hat{\partial}_{k_1} \hat{r}^{-1}
\end{align*}
The first integral turns out to be zero because it is symmetrically symmetric and odd numbers of $\textbf{n}$.
Outter variables integral
\begin{align}
    \wt{\textbf{r}} = Re \textbf{r}
    && \wt{\grad} = Re^{1} \grad
    && \wt{\textbf{u}}(\wt{\textbf{r}}) = \textbf{u}(\textbf{r})
    && \wt{p}(\wt{\textbf{r}}) = Re^{-1} p(\textbf{r}),
\end{align}
Because $\textbf{U}\sim r^{-1} \to \wt{r}^{-1} Re $ and $\textbf{u}_{out} \sim 1/r \to Re/(\wt{r} )$ in terms of stretched coordinate it goes as $\wt{\textbf{U}} Re = \textbf{U}$
\begin{align}
    \intO[o]{\textbf{U}_{o}^{(1)}\cdot \textbf{f}_{o}}
    &=
    % \intO[in]{\textbf{U}_{o}^{(1)}\cdot (\pddt \textbf{u}_{o}^{(0)}
    % + (\textbf{u}_{o}^{(0)}+\textbf{u}_r)\cdot \grad \textbf{u}_{o}^{(0)})}\\
    \ldots +  \int_{\mathbb{R}^3}Re \wt{\textbf{U}}_{o}^{(1)}\cdot [Re\pddt \textbf{u}_o^{out}+  (Re \textbf{u}_{o}^{out}+\textbf{u}_r)\cdot Re \wt{\grad}  Re\textbf{u}_{o}^{out} ]\frac{1}{Re^3 ? } d\wt{\textbf{r}}\\
    &=\int_{\mathbb{R}^3} \wt{\textbf{U}}_{o}^{(1)}\cdot  [\pddt \wt{\textbf{u}}^{out}+ \textbf{u}_r \cdot  \wt{\grad}  \wt{\textbf{u}}_{o}^{out}] d\wt{\textbf{r}}
    % \textbf{f}_{i/o} =
    % \pddt (\textbf{u}_{i/o}^{(0)} + Re\wt{\textbf{u}}_{i/o})
    % + (\textbf{u}_{i/o}^{(0)}+Re\wt{\textbf{u}}_{o}+\textbf{u}_r)\cdot \grad \textbf{u}_{i/o}^{(0)}.
    % + Re (\textbf{u}_{i/o}^{(0)}+Re\wt{\textbf{u}}_{o}+\textbf{u}_r)\cdot \grad \wt{\textbf{u}}_{o}.
    % + \textbf{u}_{i}\cdot \grad \textbf{u}_r
    % + \textbf{u}_r \cdot \grad \textbf{u}_{i}
\end{align}
The domain of Integrating should be $1$ to infty because with the streatching $O(Re)$ becomes one right.

The Fourier transform of Ossen and stokes functions are,
\begin{align*}
    F(\textbf{U}_o) &= \frac{1}{k^2}\left(\bm\delta - \hat{\textbf{k}}\hat{\textbf{k}}\right)\\
    F(\textbf{u}_{out})
    % &= \frac{\textbf{b}}{\textbf{k}\cdot(\textbf{k} - i\textbf{e})}\left(\bm\delta - \frac{\textbf{kk}}{k^2}\right)
    &= \frac{\textbf{b}}{k^2 - i\textbf{k}\cdot \textbf{e}}\cdot \left(\bm\delta - \hat{\textbf{k}}\hat{\textbf{k}}\right)
\end{align*}
\begin{align*}
    \textbf{u}_r\cdot \int \textbf{U}\cdot \grad \textbf{u} d^3 \wt{r}
    &=
    \frac{\textbf{u}_r\cdot }{(2\pi)^6}\int \int  F(\grad \textbf{u}) e^{i\textbf{k}\cdot \textbf{r}} d\textbf{k}
    \cdot \int F(\textbf{U}) e^{i\textbf{k}'\cdot \textbf{r}} d\textbf{k}'  d^3 \textbf{r}\\
    &=
    \frac{\textbf{u}_r\cdot }{(2\pi)^6}\int \int \int   F(\grad \textbf{u})
    \cdot F(\textbf{U}) e^{i(\textbf{k}+ \textbf{k}')\cdot \textbf{r}} d\textbf{k}' d\textbf{k}  d^3 \textbf{r}\\
    &=
    \frac{\textbf{u}_r\cdot }{(2\pi)^3} \int \int   F(\grad \textbf{u})
    \cdot F(\textbf{U})  \delta(\textbf{k}+ \textbf{k}') d\textbf{k}   d\textbf{k}' \\
    &=
    \frac{\textbf{u}_r\cdot }{(2\pi)^3} \int  i \textbf{k}F( \textbf{u})(\textbf{k})
    \cdot F(\textbf{U})(-\textbf{k}) d\textbf{k}  d^3 \\
\end{align*}
notting that $\textbf{u}_r= \textbf{e}$ the product gives,
% \begin{align*}\textbf{be}:
%     \frac{i}{\textbf{k}\cdot(\textbf{k} - i\textbf{e})} \textbf{k}\left(\bm\delta - \frac{\textbf{kk}}{k^2}\right)
%     \cdot
%     \frac{1}{k^2}\left(\bm\delta - \frac{\textbf{kk}}{k^2}\right)
%     &= \textbf{be}:
%     \frac{1}{ - k^2 (i k^2 - \textbf{k}\cdot \textbf{e})}
%     \textbf{k} \left(\bm\delta - \frac{\textbf{kk}}{k^2}\right)\\
%     &= \textbf{be}:
%     \frac{1}{ - k^2 (i k^2 - \textbf{k}\cdot \textbf{e})}
%     \textbf{k} \left(\bm\delta - \frac{\textbf{kk}}{k^2}\right)\\
%     &= \textbf{be}:
%     % \frac{1}{ k^2 ( -i k^2 + \textbf{k}\cdot \textbf{e})}
%     \frac{\textbf{k} \bm\delta k^2 - \textbf{kkk}}{k^4( -i k^2 + \textbf{k}\cdot \textbf{e})}
%     % ( -i k^2 + \textbf{k}\cdot \textbf{e}) \\
% \end{align*}
\begin{equation}
    i \textbf{e}\cdot \textbf{k}F( \textbf{u})(\textbf{k})
    \cdot F(\textbf{U})(-\textbf{k})
    =
    i \textbf{be} :
    \frac{\hat{\textbf{k} }(\bm\delta - \hat{\textbf{k}}\hat{\textbf{k}})}{k^2(k - i\hat{\textbf{k}}\cdot \textbf{e})}
    =
    \textbf{be} :
    \frac{\hat{\textbf{k} }(\bm\delta - \hat{\textbf{k}}\hat{\textbf{k}})}{k^2(k^2 + (\hat{\textbf{k}}\cdot \textbf{e})^2)}(ik - \hat{\textbf{k}}\cdot \textbf{e})
\end{equation}
\begin{equation}
    i \textbf{e}\cdot \textbf{k}F( \textbf{u})(\textbf{k})
    \cdot F(\textbf{U})(-\textbf{k})
    =
     \textbf{be} \frac{1}{k^2}
    \cdot  \frac{\textbf{k} (ik^2 - \textbf{k}\cdot \textbf{e})}{k^4 + (\textbf{k}\cdot \textbf{e})^2} \left(\bm\delta - \hat{\textbf{k}}\hat{\textbf{k}}\right)
    =
    \textbf{be}
   \cdot  \frac{\hat{\textbf{k}} (ik - \hat{\textbf{k}}\cdot \textbf{e})}{k^4 + (k\hat{\textbf{k}}\cdot \textbf{e})^2} \left(\bm\delta - \hat{\textbf{k}}\hat{\textbf{k}}\right)
\end{equation}
Then the force is given by,
\begin{align}
    \textbf{e}\cdot \int \textbf{U}\cdot \grad \textbf{u} d^3 \wt{r}
    &=
    \textbf{be}
    :
    \int \frac{\hat{\textbf{k}} (ik - \hat{\textbf{k}}\cdot \textbf{e})}{k^2 + (\hat{\textbf{k}}\cdot \textbf{e})^2} \left(\bm\delta - \hat{\textbf{k}}\hat{\textbf{k}}\right) \sin\theta dkd\theta d\varphi\\
    &=
    \textbf{be}
    :
    \int \frac{\hat{\textbf{k}}  (ik - \cos\theta)}{k^2 + (\cos\theta)^2} \left(\bm\delta - \hat{\textbf{k}} \hat{\textbf{k}}\right) \sin\theta dkd\theta d\varphi\\
    % &=
    % \textbf{be}
    % :
    % \int \frac{\hat{\textbf{k}} (ik - \cos\theta)}{k^2 + (\cos\theta)^2} \left(\bm\delta - \hat{\textbf{k}}\hat{\textbf{k}}\right) \sin\theta dkd\theta d\varphi\\
\end{align}
with
\begin{equation}
    \textbf{k} = k(
        \cos\theta \textbf{e}
        + \sin\theta\cos\varphi \textbf{e}_1
        + \sin\theta\sin\varphi \textbf{e}_2
    )
\end{equation}
setting $\textbf{b}=\textbf{e}$ which is required by matching condition we obtain,
\begin{equation}
    \int \frac{\cos\theta (ik - \cos\theta)}{k^2 + (\cos\theta)^2} \left(\textbf{e} - \hat{\textbf{k}} \cos\theta \right) \sin\theta dkd\theta d\varphi
    =- \textbf{e} \pi^2 /2\\
\end{equation}
For $\hat{\textbf{k}} = \cos\theta$ and $\textbf{e}=1$ we have
\begin{equation}
    \int \frac{\cos\theta (ik - \cos\theta)}{k^2 + (\cos\theta)^2}  \sin^3\theta dkd\theta d\varphi
     = - \pi^2 /2 \\
\end{equation}
for $\textbf{e}=0$ and $\hat{\textbf{k}}=\sin\theta \cos\varphi$ we have,
\begin{equation}
    -\int \frac{\cos^2\theta (ik - \cos\theta)}{k^2 + (\cos\theta)^2} \sin^2\theta   dkd\theta  \cos\varphi d\varphi
    = 0 \\
\end{equation}

\tb{the direct integration of the ossen flow require its grad}

\subsection{The unsteady term}
\begin{align}
    \intO[o]{\textbf{U}_{o}^{(1)}\cdot \textbf{f}_{o}}
    &=
    % \intO[in]{\textbf{U}_{o}^{(1)}\cdot (\pddt \textbf{u}_{o}^{(0)}
    % + (\textbf{u}_{o}^{(0)}+\textbf{u}_r)\cdot \grad \textbf{u}_{o}^{(0)})}\\
    \ldots +  \int_{\mathbb{R}^3}Re \wt{\textbf{U}}_{o}^{(1)}\cdot [Re\pddt \textbf{u}_o^{out}+  (Re \textbf{u}_{o}^{out}+\textbf{u}_r)\cdot Re \wt{\grad}  Re\textbf{u}_{o}^{out} ]\frac{1}{Re^3 ? } d\wt{\textbf{r}}\\
    &=\int_{\mathbb{R}^3} \wt{\textbf{U}}_{o}^{(1)}\cdot  [\pddt \wt{\textbf{u}}^{out}+ \textbf{u}_r \cdot  \wt{\grad}  \wt{\textbf{u}}_{o}^{out}] d\wt{\textbf{r}}
\end{align}

\section{Second moment of forces expression}
\label{ap:second_mom}
To derive an equation for the second moment of forces we set,
\begin{align}
    \label{eq:grad_grad_ur_cst}
    \hat{\textbf{u}}_r =\frac{1}{2}\textbf{rr}\cdot  \grad\grad \textbf{u}|_{\textbf{x}=\textbf{y}}
    &&
    \hat{\textbf{b}} = 0 &&\text{Solutions: }
    &&
    \hat{\textbf{u}}_{o/i} = \textbf{U}_{o/i}^{(3)} \vdots \grad\grad\hat{\textbf{u}}|_{\textbf{x}=\textbf{y}}
    &&
    \hat{\bm\sigma}_{o/i} = \textbf{S}_{o/i}^{(3)} \vdots \grad\grad\hat{\textbf{u}}|_{\textbf{x}=\textbf{y}}.
\end{align}
Then, the second moment of forces may be obtained injecting~\ref{eq:grad_grad_ur_cst} into~\ref{eq:int_final_step}, it reads,
\begin{multline}
    \frac{1}{2}\intS[p]{\textbf{rr}  \bm\sigma_{o}\cdot \textbf{n}}
    - \intO[i]{2\textbf{e}_i \textbf{r}}
    % - (1-\lambda) \intO[p]{(\textbf{u}_{i} + \textbf{u}_r)}
    % + \intS[p]{2(\textbf{u}_{i} + \textbf{u}_r)\cdot \hat{\textbf{b}}}
    =
    \intS[p]{\textbf{u}_{r}\cdot \textbf{S}^{(3)}_{o}\cdot \textbf{n}}
    - \intO[i]{\textbf{S}_i^{(3)} :\grad\textbf{u}}
    - (1-\lambda) \intO[p]{( \textbf{U}_{i}^{(3)} + \frac{1}{2} \textbf{rr}\bm\delta ) \cdot \grad^2 \textbf{u} }\\
    + \intS[p]{2( \textbf{U}_{i}^{(3)} + \frac{1}{2} \textbf{rr}\bm\delta ) \cdot  \textbf{b}}
    + \zeta Re \intO{( \textbf{U}_{i}^{(3)} + \frac{1}{2} \textbf{rr}\bm\delta )\cdot \textbf{f}_{i}}
    + Re\intO[o]{\textbf{U}_{o}^{(3)}\cdot \textbf{f}_{o}}.
    \label{eq:second_mom}
\end{multline}
Where we have noticed that the integral of $\textbf{u}_i$ over the droplet volume is $-\textbf{u}_r$ in the absence of mass transfer, whence the third term of~\ref{eq:int_final_step} also vanished in~\ref{eq:second_mom}.
Likewise, to derive \ref{eq:second_mom} we have factorized by the third order tensor $(\grad\grad \hat{\textbf{u}})_{ijk}$ which is zero upon the contraction of the index $i$ with either $j$ or $k$.
Hence, only the traceless part on the index $ik$ and $jk$ of \ref{eq:second_mom} is meaningful.

When only translational motion are taken into account one arrive at,
\begin{multline}
    \frac{1}{2}\intS[p]{\textbf{rr}  \bm\sigma_{o}\cdot \textbf{n}}
    - \intO[i]{2\textbf{e}_i \textbf{r}}
    % - (1-\lambda) \intO[p]{(\textbf{u}_{i} + \textbf{u}_r)}
    % + \intS[p]{2(\textbf{u}_{i} + \textbf{u}_r)\cdot \hat{\textbf{b}}}
    =
    \frac{1}{2}\intS[p]{\textbf{rr} \textbf{S}^{(3)}_{o}\cdot \textbf{n}}
    - \intO[i]{\textbf{S}_i^{(3)} \textbf{r}}\\
    % - (1-\lambda) \intO[p]{( \textbf{U}_{i}^{(3)} + \frac{1}{2} \textbf{rr}\bm\delta ) \cdot \grad^2 \textbf{u} }\\
    % + \intS[p]{2( \textbf{U}_{i}^{(3)} + \frac{1}{2} \textbf{rr}\bm\delta ) \cdot  \textbf{b}}
    + \zeta Re \intO{( \textbf{U}_{i}^{(3)} + \frac{1}{2} \textbf{rr}\bm\delta )\cdot \textbf{f}_{i}}
    + Re\intO[o]{\textbf{U}_{o}^{(3)}\cdot \textbf{f}_{o}}.
\end{multline}
We know due to symmetry arguments that the inner fields may not contribute.
Hence, one only need to compute the external contributions.
\begin{align*}
    (\textbf{P}_o^{(3)})_{k_1 k_2 k_3}&=
        % \delta_{k_1 k_2 } \partial_{k_3} r^{-1}
        % + \delta_{k_1 k_3 } \partial_{k_2} r^{-1}
        \frac{\lambda}{4(\lambda+1)}\delta_{k_3 k_2 } \partial_{k_1} r^{-1}
        +\frac{7\lambda+2}{24(\lambda +1)} \partial_{k_1} \partial_{k_2} \partial_{k_3} r^{-1} \\
    (\textbf{U}^{(3)}_o)_{ik_1k_2k_3}
    &=
    \frac{\lambda}{8(\lambda+1)}\delta_{k_3 k_2 } x_i\partial_{k_1} r^{-1}
    + \frac{7\lambda+2}{48(\lambda +1)} x_i \partial_{k_1} \partial_{k_2} \partial_{k_3} r^{-1} \\
    % + Cf[4]*\delta_{k_3 i}*\delta_{k_1 k_2}*r^{-1}
    &+ \frac{-\lambda}{8(\lambda+1)}\delta_{k_1 i} \delta_{k_3 k_2} r^{-1}
    % + Cf[6]*\delta_{k_2 i}*\delta_{k_1 k_3}*r^{-1}
    + \frac{\lambda-1}{6(\lambda+4)(\lambda+1)}\partial_{i  }\partial_{k_1}\delta_{k_2 k_3} r^{-1}\\
    % + Cf[8]\partial_{k_3}\partial_{ i}\delta_{k_1 k_2}
    % + Cf[9]\partial_{k_2}\partial_{ i}\delta_{k_3 k_1}
    &+ \frac{13 L^2 + 10 L - 8}{48(L + 4)(L + 1)} \partial_{k_1}\partial_{k_2} \delta_{k_3 i} r^{-1}
    + \frac{(4 - L) (3 L + 2)}{48(L + 4)(L + 1)} \partial_{k_1}\partial_{k_3} \delta_{i k_2}r^{-1}\\
    &+ \frac{(4 - L) (3 L + 2)}{48(L + 4)(L + 1)}\partial_{k_2}\partial_{k_3} \delta_{i k_1}r^{-1}
    +\frac{\lambda}{48(\lambda+1)}\partial_{i}\partial_{k_1}\partial_{k_2}\partial_{k_3}r^{-1}
\end{align*}

Because,
\begin{align}
    \wt{\textbf{r}} = Re \textbf{r}
    && \wt{\grad} = Re^{1} \grad
    && \wt{\textbf{u}}(\wt{\textbf{r}}) = \textbf{u}(\textbf{r})
    && \wt{p}(\wt{\textbf{r}}) = Re^{-1} p(\textbf{r}),
\end{align}
at the leading order,
\begin{align*}
    \frac{1}{Re}(\textbf{U}^{(3)}_o)_{ik_1k_2k_3}
    &=
    \frac{\lambda}{8(\lambda+1)}\delta_{k_3 k_2 } \hat{x}_i\hat{\partial}_{k_1} \hat{r}^{-1}
    + Re^2 \frac{7\lambda+2}{48(\lambda +1)} \hat{x}_i \hat{\partial}_{k_1} \hat{\partial}_{k_2} \hat{\partial}_{k_3} \hat{r}^{-1} \\
    % + Cf[4]*\delta_{k_3 i}*\delta_{k_1 k_2}*\hat{r}^{-1}
    &+ \frac{-\lambda}{8(\lambda+1)}\delta_{k_1 i} \delta_{k_3 k_2} \hat{r}^{-1}
    % + Cf[6]*\delta_{k_2 i}*\delta_{k_1 k_3}*\hat{r}^{-1}
    + Re^2 \frac{\lambda-1}{6(\lambda+4)(\lambda+1)}\hat{\partial}_{i  }\hat{\partial}_{k_1}\delta_{k_2 k_3} \hat{r}^{-1}\\
    % + Cf[8]\hat{\partial}_{k_3}\hat{\partial}_{ i}\delta_{k_1 k_2}
    % + Cf[9]\hat{\partial}_{k_2}\hat{\partial}_{ i}\delta_{k_3 k_1}
    &+ Re^2 \frac{13 L^2 + 10 L - 8}{48(L + 4)(L + 1)} \hat{\partial}_{k_1}\hat{\partial}_{k_2} \delta_{k_3 i} \hat{r}^{-1}
    + Re^2 \frac{(4 - L) (3 L + 2)}{48(L + 4)(L + 1)} \hat{\partial}_{k_1}\hat{\partial}_{k_3} \delta_{i k_2}\hat{r}^{-1}\\
    &+ Re^2 \frac{(4 - L) (3 L + 2)}{48(L + 4)(L + 1)}\hat{\partial}_{k_2}\hat{\partial}_{k_3} \delta_{i k_1}\hat{r}^{-1}
    +Re^4 \frac{\lambda}{48(\lambda+1)}\hat{\partial}_{i}\hat{\partial}_{k_1}\hat{\partial}_{k_2}\hat{\partial}_{k_3}\hat{r}^{-1}
\end{align*}

So at the leading order in $Re$
\begin{align*}
    \frac{1}{Re}(\textbf{U}^{(3)}_o)_{ik_1k_2k_3}
    =
    \frac{\lambda}{8(\lambda+1)} \delta_{k_3 k_2 } ( \hat{x}_i\hat{\partial}_{k_1} \hat{r}^{-1}
    - \delta_{k_1 i} \hat{r}^{-1})
\end{align*}
The Fourier transform of that field reads,
\begin{align}
    F(\frac{1}{Re}(\textbf{U}^{(3)}_o)_{ik_1k_2k_3}) =
    8\pi \frac{\lambda}{8(\lambda+1)} \delta_{k_3 k_2 } (
        k_1 k_{k_1} k^{-4}
        - \delta_{i k_1} k^{-2}
    )
    =
    F\frac{\delta_{k_3k_2}}{k^2}\left(
        \frac{k_ik_{k_1}}{k^2}
        -\delta_{ik_1}
    \right)
\end{align}
the outter sol reads, 
\begin{align}
    F(\textbf{u}^{out})
    = 
    -\frac{b\textbf{b}}{k^2  + i \textbf{e}\cdot \textbf{k}}\cdot (\frac{\textbf{kk}}{k^2} - \bm\delta)
\end{align}
where $\textbf{b} = \textbf{e}b$ is the strength of the point force generated by the droplet at $O(1)$ in $Re$.
So that the term is,
\begin{multline}
    \intO[o]{\textbf{U}_{o}^{(3)}\cdot \textbf{f}_{o}}
    =
    \textbf{e}\cdot 
    \int_{\mathbb{R}^3}  \wt{\grad}  \wt{\textbf{u}}_{o}^{out}\cdot \wt{\textbf{U}}_{o}^{(3)} d\wt{\textbf{r}}
    =
    \frac{\textbf{e}}{8\pi^3}\cdot \int
    i \textbf{k} F(\textbf{u}^{out}) \cdot F(\textbf{U}^{(3)}) d^3 \textbf{k}\\
    =
    b F \delta_{k_2k_3}\frac{\textbf{be}}{8\pi^3} : \int
    i \textbf{k} F(\textbf{u}^{out}_*) \cdot F(\textbf{U}^{(3)}_*) d^3 \textbf{k}
\end{multline}
\tb{en principe on pourrait juste se contenter de la taylor expde la solution d'ossen}
The integrand reads, 
\begin{equation*}
    i \frac{- \textbf{k}}{k^2(k^2  + i (\textbf{e}\cdot \textbf{k}))} (\frac{k_{k_1}k_{j_1}}{k^2} - \delta_{k_1j_1})   
    =
    \frac{ \textbf{k}(-ik^2  + (\textbf{e}\cdot \textbf{k}))}{k^2(k^4  + (\textbf{e}\cdot \textbf{k})^2)} (\frac{k_{k_1}k_{j_1}}{k^2} - \delta_{k_1j_1})   
\end{equation*}
neglecting the odd number of \textbf{k} terms we arrive at, 
\begin{align}
    b F \delta_{k_2k_3}\frac{\textbf{be}}{8\pi^3} : \int
    i \textbf{k} F(\textbf{u}^{out}_*) \cdot F(\textbf{U}^{(3)}_*) d^3 \textbf{k}
    &=
    b F \delta_{k_2k_3}\frac{\textbf{be}}{8\pi^3} : 
    \int
    \frac{ \textbf{k} (\textbf{e}\cdot \textbf{k})}{k^2(k^4  + (\textbf{e}\cdot \textbf{k})^2)} (\frac{k_{k_1}k_{j_1}}{k^2} - \delta_{k_1j_1})  d^3 \textbf{k}\\
    &=
    b F \delta_{k_2k_3}\frac{1}{8\pi^3} : 
    \int
    \frac{ \cos^2\theta }{k^2  +  \cos^2\theta} (\cos\theta\frac{k_{k_1}}{k} - b_{k_1})  \sin\theta d\theta dk d\varphi\\
    &=
    b F \delta_{k_2k_3}\frac{1}{16\pi} e_{k_1} 
    \\
\end{align}
The only difference with the force is that we have got $Fb$ instead of $b^2$ as a factor, but if we factorize the whole expression it turns out to be the same. 

So that the final expression for the Stresslet IS, 
\begin{align}
    \frac{1}{2}\intS{\textbf{xx}\bm\sigma_{out}\cdot \textbf{n}}
    - \intO{2\textbf{r}\textbf{e}_{in}} &=
    +\frac{2\pi}{15(\lambda +1)} \textbf{u}_r \bm\delta
    + \frac{\pi \lambda }{(\lambda+1)}
    [
        1
        +
        \frac{3\lambda+2}{8(\lambda+1)} Re 
    ]\bm\delta \textbf{u}_r 
    % + Re \frac{\pi \lambda}{\lambda+1} b\frac{1}{16\pi}\bm\delta \textbf{e}
    % + \frac{4-4\pi}{10(\lambda +1)} ^\dagger\textbf{u}_r \bm\delta
\end{align}
where $b$ is the intensity of the point force given by $b = 2 \pi \frac{2+3\lambda}{1+\lambda}$


To obtain the dimensional form we multiply by $a^4$ and for the stress $\mu_f U/a$
It gives 
\begin{align}
    \frac{1}{2}\intS{\textbf{xx}\bm\sigma_{out}\cdot \textbf{n}}
    - \intO{2\textbf{r}\textbf{e}_{in}} &=
    \mu_f \phi  \frac{\pi}{10(\lambda +1)} \textbf{u}_r \bm\delta
    +\phi  \frac{3\pi \lambda }{4(\lambda+1)}
    [
        \mu_f  
        +
        \frac{3\lambda+2}{8(\lambda+1)} 
        a \rho_f |\textbf{u}_r|
    ]\bm\delta \textbf{u}_r 
    % + Re \frac{\pi \lambda}{\lambda+1} b\frac{1}{16\pi}\bm\delta \textbf{e}
    % + \frac{4-4\pi}{10(\lambda +1)} ^\dagger\textbf{u}_r \bm\delta
\end{align}

\subsection{Inertial correction to the Drag force and to the second moment of forces}
\begin{equation}
    \intS[p]{\bm\sigma_{o}\cdot \textbf{n}}
    % - \intO[i]{2\textbf{e}_i : \grad\hat{\textbf{u}}}
    % - (1-\lambda) \intO[p]{(\textbf{u}_{i} + \textbf{u}_r)\cdot \grad^2 \hat{\textbf{u}}}
    % + \intS[p]{2(\textbf{u}_{i} + \textbf{u}_r)\cdot \hat{\textbf{b}}}
    =
    \textbf{u}_{r}\cdot\intS[p]{ \textbf{S}_{o}^{(1)}\cdot \textbf{n}}
    % - \intO[i]{ \textbf{S}_{i}^{(1)} :\grad\textbf{u}}
    % - (1-\lambda) \intO[p]{(\textbf{U}_{i}^{(1)} + \bm\delta) \cdot \grad^2 \textbf{u} }\\
    % + \intS[p]{2 (\textbf{U}_{i/o}^{(1)} + \bm\delta) \cdot  \textbf{b}}
    + \zeta Re \intO{(\textbf{U}_{i}^{(1)} + \bm\delta)\cdot \textbf{f}_{i}}
    + Re\intO[o]{\textbf{U}_{o}^{(1)}\cdot \textbf{f}_{o}},
    % \label{eq:drag_force}
\end{equation}
the inertial term is driven by,
\begin{equation}
    \textbf{f}_{i/o} =
    \pddt \textbf{u}_{i}
    + (\textbf{u}_{i}+\textbf{u}_r)\cdot \grad \textbf{u}_{i}.
    % + \textbf{u}_{i}\cdot \grad \textbf{u}_r
    % + \textbf{u}_r \cdot \grad \textbf{u}_{i}
\end{equation}
Which is solved by,
\begin{equation}
    - \grad p_f
    + \grad^2 \textbf{u}_o
    = Re [\pddt \textbf{u}_{i}
    + (\textbf{u}_{i}+\textbf{u}_r)\cdot \grad \textbf{u}_{i}]
\end{equation}
The velocity may be determined from the zeroth order Stokes expansion or first order ossen term far away
we set,
\begin{align}
    \textbf{u} &= \textbf{u}^{(0)} + Re \textbf{u}^{(1)}\\
    \wt{\textbf{u}} &=  Re \wt{\textbf{u}}^{(1)}\\
\end{align}
which are governed by the equaitons,
\begin{equation}
    - \grad p^{(0)} + \grad^2 \textbf{u}^{(0)} = 0
\end{equation}

We want to solve for the green function
\begin{align}
    \div \textbf{u}&= 0\\
    - \grad p
    + \mu \grad^2 \textbf{u}
    + \textbf{f}\delta(\textbf{r})
    &=
    \rho \textbf{U}\cdot \grad \textbf{u}
\end{align}
Or in dimensionless form
\begin{align}
    \div \textbf{u}&= 0\\
    - \grad p
    +  \grad^2 \textbf{u}
    + \textbf{f}\delta(\textbf{r})
    &=
    \textbf{e}\cdot \grad \textbf{u}
\end{align}
Where $\textbf{e} = \textbf{U}/|\textbf{U}| Re$ and $Re = \frac{|\textbf{U}|a \rho}{\mu}$.

One may be tempted solving this equation with FFT he get,
\begin{equation}
    \hat{p} = - i \textbf{f}\cdot \textbf{k} / k^2
\end{equation}
and so,
\begin{align}
    \textbf{u}
    &= -\frac{\textbf{f}}{k^2  + i \textbf{e}\cdot \textbf{k}}\cdot (\frac{\textbf{kk}}{k^2} - \bm\delta)
\end{align}
Using the identity,
\begin{equation}
    \frac{1}{k^2 \textbf{k}\cdot (i\textbf{e} + \textbf{k})}
    =\frac{1}{i\textbf{e}\cdot \textbf{k}}\left(\frac{1}{k^2} - \frac{1}{\textbf{k}\cdot (i\textbf{e}+\textbf{k})}\right)
\end{equation}
we arrrive at,
\begin{equation}
    \textbf{u}
    = \frac{\textbf{f}}{k^2  + i \textbf{e}\cdot \textbf{k}}
    -
    \frac{\textbf{f}\cdot \textbf{kk}}{i\textbf{e}\cdot \textbf{k}}\left(\frac{1}{k^2} - \frac{1}{\textbf{k}\cdot (i\textbf{e}+\textbf{k})}\right)
\end{equation}

If we assume a linearity with \textbf{b} which is verified \citet{pozrikidis2011introduction} we have
% \begin{equation}
%     \textbf{O}
%     = \frac{\bm\delta}{k^2  + i \textbf{e}\cdot \textbf{k}}
%     -
%     \frac{\textbf{kk}}{i\textbf{e}\cdot \textbf{k}}\left(\frac{1}{k^2} - \frac{1}{\textbf{k}\cdot (i\textbf{e}+\textbf{k})}\right)
% \end{equation}
% we are able to compute $\textbf{e}\cdot \textbf{O}$ however the full expression for \textbf{O} remain unreachable.

\section{Podzrikidis solution}
Using the fact that $\delta(\textbf{r}) = -  \grad^2(1/r) / 4\pi$ we find,
\begin{equation}
     p = - \frac{1}{4\pi}\textbf{f} \cdot \grad (1/r)
\end{equation}
Hence the equation for \textbf{u} becomes,
\begin{equation}
    \grad^2 \textbf{u}
    - \textbf{e}\cdot \grad \textbf{u}
    - \frac{\textbf{f}}{4\pi}\cdot [\bm\delta\grad^2  -  \grad\grad] (1/r)
    = 0
\end{equation}
Then we assume that \textbf{u} is given by,
\begin{equation}
    \textbf{u} = \textbf{f}\cdot [\bm\delta\grad^2  -  \grad\grad] H
\end{equation}
which satisfy,
\begin{equation}
    \textbf{f}\cdot [\bm\delta\grad^2  -  \grad\grad]\{
        \grad^2 H
        - \textbf{e}\cdot \grad H
        - \frac{1}{4\pi r }
        \}
    = 0
    \Longleftrightarrow
        (\grad^2
        - \textbf{e}\cdot \grad) H
        =
         \frac{1}{4\pi r }
\end{equation}
Then setting $Q = \grad^2 H$ we arrive at,
\begin{equation}
        (\grad^2 - \textbf{e}\cdot \grad) Q
        + \delta(\textbf{x})
    = 0
\end{equation}
To get rid of the advective term we define $Q = e^{\textbf{e}\cdot \textbf{r}/2} G$ and solve for $G$, it gives,
 \begin{align}
    \grad  Q
    = \grad e^{\textbf{e}\cdot \textbf{r}/2} G+ e^{\textbf{e}\cdot \textbf{r}/2} \grad G\\
    \grad\grad  Q
    % = \grad (\grad e^{\textbf{e}\cdot \textbf{r}/2} G+ e^{\textbf{e}\cdot \textbf{r}/2} \grad G)
    =
    \grad \grad e^{\textbf{e}\cdot \textbf{r}/2} G
    +2  \grad e^{\textbf{e}\cdot \textbf{r}/2} \grad G
    % + \grad e^{\textbf{e}\cdot \textbf{r}/2} \grad G
    +e^{\textbf{e}\cdot \textbf{r}/2} \grad\grad  G\\
    \grad e^{\textbf{e}\cdot \textbf{r}/2}
    =
    \textbf{e}/2 e^{\textbf{e}\cdot \textbf{r}/2} \\
    \grad\grad e^{\textbf{e}\cdot \textbf{r}/2}
    =
    \textbf{ee}/4  e^{\textbf{e}\cdot \textbf{r}/2}
 \end{align}
 So that
 \begin{equation}
    (\grad^2  -e^2/4    )G
    =
    - e^{\textbf{e}\cdot \textbf{r}/2} \delta(\textbf{x})
    =
    -  \delta(\textbf{x})
\end{equation}
which is the Helmholtz equation.
Defining the Fourier transform,
\begin{align}
    \hat{G} = \iiint G(\textbf{r}) e^{-i\textbf{k}\cdot \textbf{r}} d^3 \textbf{r}
    &&
    G = \frac{1}{(2\pi)^3}\iiint \hat{G}(\textbf{k}) e^{i\textbf{k}\cdot \textbf{r}} d^3 \textbf{k}
\end{align}
we directly find,
\begin{equation}
    \hat{G} = \frac{1}{k^2 + (e/2)^2}
    \Longleftrightarrow
    G = \frac{e^{-(e/2) r}}{4\pi r}
\end{equation}
Thus,
\begin{equation}
    Q \equiv \grad^2 H = \frac{e^{(\textbf{e}\cdot \textbf{r} -e r)/2}}{4\pi r}
    = \frac{e^{X}}{4\pi r}
\end{equation}
We deduce that the Fourier transform of $Q$ and its inverse,
\begin{align}
    \hat{Q} =\frac{1}{\textbf{k}\cdot (\textbf{k} +i\textbf{e})}
    && Q =\frac{e^{(\textbf{e}\cdot \textbf{r} -e r)/2}}{4\pi r}.
\end{align}
Using this TF and the equation for $H$ one may be able to determine,
\begin{equation}
     \hat{H} = \frac{-1}{k^2 (k^2 + i \textbf{e}\cdot \textbf{k})}\Longleftrightarrow
     - i \textbf{e}\cdot \textbf{k} \hat{H}=
     \frac{1}{k^2}  - \frac{1}{\textbf{k}\cdot (\textbf{k}+i\textbf{e})}
\end{equation}
whose solution gives,
\begin{equation}
    - \textbf{e}\cdot \grad H =  \frac{1- e^{(\textbf{e}\cdot \textbf{r} -e r)/2}}{4\pi r}.
\end{equation}

\subsection{Podzrikidis method}
Assuming $H(X(\textbf{r},\textbf{e}))$ with $X= (\textbf{e}\cdot \textbf{r} -e r)/2$ then,
\begin{align*}
    \grad\cdot (\grad H) &=
    \grad\cdot \frac{\partial X}{\partial \textbf{r}} \frac{\partial H}{\partial X}
    +
    \frac{\partial X}{\partial \textbf{r}}\cdot \frac{\partial X}{\partial \textbf{r}} \frac{\partial^2 H}{(\partial X)^2}\\
    &=
    - e r^{-1} (\frac{\partial H}{\partial X}
    +
    X
    \frac{\partial^2 H}{(\partial X)^2})\\
    &=
    - e r^{-1} \frac{\partial}{\partial X}(X\frac{\partial H}{\partial X})\\
\end{align*}
where we used,
\begin{align}
    \grad\frac{\partial X}{\partial \textbf{r}}
    &=
    - e r^{-1} \\
    \frac{\partial X}{\partial \textbf{r}}\cdot \frac{\partial X}{\partial \textbf{r}}
    &=
    e(e r - \textbf{e}\cdot \textbf{r} )/(r2)
    =- \frac{eX}{r}
\end{align}
Hence,
\begin{equation}
    \grad^2 H = Q \equiv
    \frac{\partial}{\partial X}(X\frac{\partial H}{\partial X})
    =- \frac{e^{X}}{4\pi e}
    \Longrightarrow
    \frac{\partial H}{\partial X}
    = \frac{1 - e^X}{X 4 \pi e}
\end{equation}
Hence, the gradient is,
\begin{equation}
    \grad H = \frac{\partial X}{\partial \textbf{r}}\frac{\partial H}{\partial X}
    = \frac{1 - e^X}{X 8 \pi e} (\textbf{e} - e \textbf{r}r^{-1})
    = \frac{1 - e^{(\textbf{e}\cdot \textbf{r} -e r)/2}}{(\textbf{e}\cdot \textbf{r} -e r) 4 \pi e} (\textbf{e} - e \textbf{r}r^{-1})
\end{equation}
\tb{Re do from there}
Or
\begin{align}
    \grad\grad H
    &=
    \frac{\partial X}{\partial \textbf{r}} \frac{\partial X}{\partial \textbf{r}} \frac{\partial^2 H}{(\partial X)^2}
    +\frac{\partial H}{\partial X}  \grad\frac{\partial X}{\partial \textbf{r}} \\
    &=
    \frac{1}{4}(\textbf{e} - e \textbf{r}r^{-1})
    (\textbf{e} - e \textbf{r}r^{-1})
    (\frac{-e^X}{X 4 \pi e}
    - \frac{1 - e^X}{X^2 4 \pi e})
    + \frac{1 - e^X}{X 8  \pi e}  (- e \bm\delta r^{-1} + e \textbf{rr} r^{-3})\\
    &=
    \frac{e}{4}(\textbf{e}/e - \textbf{n})
    (\textbf{e}/e - \textbf{n})
    \frac{(1-X) e^X - 1}{X^2 4 \pi}
    + \frac{e^X- 1}{X 8 \pi }  (\bm\delta  - \textbf{nn} )r^{-1}
\end{align}

If we dot this eq by $\textbf{e}$ and add a minus sign we get
\begin{equation}
     - \textbf{e}\cdot \grad H
    = -\frac{1 - e^{(\textbf{e}\cdot \textbf{r} -e r)/2}}{(\textbf{e}\cdot \textbf{r} -e r) 4 \pi e} (e^2 - e \textbf{e}\cdot \textbf{r}r^{-1})
    = \frac{1 - e^{(\textbf{e}\cdot \textbf{r} -e r)/2}}{4 \pi r}
\end{equation}


Anyhow, the ossenlet is given by
\begin{equation}
    \textbf{O} = \bm\delta Q - \grad\grad H
    =
    \frac{e^X}{4\pi r}\bm\delta
    -
    \frac{e}{4}(\textbf{e}/e - \textbf{n})
    (\textbf{e}/e - \textbf{n})
    \frac{(1-X) e^X - 1}{X^2 4 \pi}
    - \frac{e^X- 1}{X 8 \pi }  (\bm\delta  - \textbf{nn} )r^{-1}
\end{equation}
hence the velocity $\textbf{u}=\textbf{O}\cdot \textbf{f}$.


If we substitute $e \to Re$ and $\textbf{e}/e \to \textbf{e}$ we get $X = Re(\textbf{e}\cdot \textbf{r} - r)/2 = (\textbf{e}\cdot \wt{\textbf{r} }- \wt{r})/2  $ and $\grad X = Re (\textbf{e} - \textbf{n})$ the following
\begin{align}
    \textbf{O}
    &=
    \frac{e^X}{4\pi r}\bm\delta
    -\grad \left(\frac{1-e^X}{X8\pi}(\textbf{e}- \textbf{n})\right)\\
    % &=
    % \frac{e^X}{4\pi r}\bm\delta
    % -
    % \frac{Re}{4}(\textbf{e} - \textbf{n})
    % (\textbf{e} - \textbf{n})
    % \frac{(1-X) e^X - 1}{X^2 4 \pi}
    % - \frac{e^X- 1}{X 8 \pi r}  (\bm\delta  - \textbf{nn} )
\end{align}
Notting that,
\begin{align}
    \grad (\textbf{e}- \textbf{n}) = (\textbf{nn} - \bm\delta ) r^{-1}\\
    \grad (\textbf{nn} - \bm\delta ) r^{-1} = - 3 \textbf{nnn} r^{-2} + (\bm\delta \textbf{n}+ \bm\delta \textbf{n}^\dagger + \textbf{n}\bm\delta) r^{-2}
\end{align}
we get
\begin{equation}
    \textbf{O} =
    \frac{e^X}{4\pi r}\bm\delta
    + \frac{1 - e^X}{X 8 \pi r}  (\bm\delta  - \textbf{nn} )
    -
    \frac{Re}{4}(\textbf{e} - \textbf{n})
    (\textbf{e} - \textbf{n})
    \frac{(1-X) e^X - 1}{X^2 4 \pi}
\end{equation}

Assuming $r$ is independent of time, which is the case in conditional statistics, we may have,
\begin{multline}
    \pddt \textbf{O} =
    \pddt X \left\{\frac{e^X}{4\pi r}\bm\delta
    +
    [
        \frac{ e^X (1-X) -1}{X^2 8 \pi r}  
        ] 
    (\bm\delta  - \textbf{nn} )
    -
    \frac{Re}{4}(\textbf{e} - \textbf{n})
    (\textbf{e} - \textbf{n})
    [
        -2 \frac{(1-X+ X^2 /2) e^X - 1}{X^3 4 \pi}
        ]\right\}\\
    \pddt \textbf{O} =
    \textbf{e}\cdot \grad X \cdot \left\{\frac{e^X}{4\pi r}\bm\delta
    +
    [
        \frac{ e^X (1-X) -1}{X^2 8 \pi r}  
        ] 
    (\bm\delta  - \textbf{nn} )
    -
    \frac{Re}{4}(\textbf{e} - \textbf{n})
    (\textbf{e} - \textbf{n})
    [
        -2 \frac{(1-X+ X^2 /2) e^X - 1}{X^3 4 \pi}
        ]
    \right. \\ \left.     
    \right\}
\end{multline}



\section{Re-doing feuillebois derivation}

We consider Maxey-Riley-Gatignol equations, in a pure translating senario and steady states.
\begin{align}
    \div \textbf{u}_{o} &= 0
    \\
    -\grad p_o + \grad^2 \textbf{u}_o
    &=
    Re  (\textbf{u}_o + \textbf{e})\cdot \grad \textbf{u}_{o}
\end{align}
and,
\begin{align}
    \div \textbf{u}_{i} &= 0,
    \\
    \div\bm\sigma_{i}
    &=
    \frac{\zeta}{\lambda}Re  (\textbf{u}_{i} + \textbf{e}) \cdot \grad \textbf{u}_{i}
\end{align}
with the BCs,
\begin{align}
    \textbf{u}_i - \textbf{u}_o &= 0\\
    [\textbf{u}_i + \textbf{e}]\cdot \textbf{n} = 0 \\
    \textbf{n}\cdot [\textbf{e}_o - \lambda \textbf{e}_i ]\cdot (\bm\delta - \textbf{nn}) = 0
\end{align}
where $\textbf{e} = \textbf{u}_r / |\textbf{u}_r|$ and it is also the velocity scale of the Reynolds number.
At $r = \infty$,
\begin{equation}
    \lim_{r\to\infty} (p_o,\textbf{u}_o) = (0,\textbf{0}).
\end{equation}
\subsection{Inner expansion}
we consider that $\textbf{u}_o = \textbf{u}_o^{(0)} + Re \textbf{u}_o^{(1)}$ and so on.
\begin{align}
    \div \textbf{u}_o^{(0)} &= 0
    \\
    -\grad p_o^{(0)} + \grad^2 \textbf{u}_o^{(0)}
    &= 0\\
    \div \textbf{u}_o^{(1)} &= 0
    \\
    -\grad p_o^{(1)} + \grad^2 \textbf{u}_o^{(1)}
    &=
    Re  (\textbf{u}_o^{(0)} + \textbf{e})\cdot \grad \textbf{u}_o^{(0)}\\
\end{align}
where we notice that only the BCs at $r= 1$ are available.
Indeed small $Re$ implies small $r$ so we cannot apply the boundary far from the droplet in this situation.
\subsection{Outter expansion}
Before doing the expansion we apply the change of variables,
\begin{align}
    \wt{\textbf{r}} = Re \textbf{r}
    && \wt{\grad} = Re^{1} \grad
    && \wt{\textbf{u}}(\wt{\textbf{r}}) = \textbf{u}(\textbf{r})
    && \wt{p}(\wt{\textbf{r}}) = Re^{-1} p(\textbf{r}),
\end{align}
which implies big $\textbf{r}$ when $\wt{\textbf{r}} = O(\textbf{1})$.
In this situation the equation for $\textbf{u}_o$ reads,
\begin{align}
    \div \textbf{u}_{o} &= 0
    \\
    - \wt{\grad} \wt{p} +  \wt{\grad}^2 \wt{\textbf{u}}
    &=
      (\wt{\textbf{u}} + \textbf{e})\cdot \wt{\grad} \wt{\textbf{u}}
\end{align}
Where we must only apply
Hence at a distance large enough from the particle both terms are equivalent.
Then assuming $\wt{\textbf{u}} = \wt{\textbf{u}}_0 + Re\wt{\textbf{u}}_1$ and neglecting the $Re^2$ terms we get,
\begin{align}
    \div \wt{\textbf{u}}_0 &= 0
    \\
    - \wt{\grad} \wt{p}_0 +  \wt{\grad}^2 \wt{\textbf{u}}_0
    &=
     (\textbf{e}+ \wt{\textbf{u}}_0) \cdot \wt{\grad} \wt{\textbf{u}}_0\\
    \div \wt{\textbf{u}}_1 &= 0
    \\
    - \wt{\grad} \wt{p}_1 +  \wt{\grad}^2 \wt{\textbf{u}}_1
    &=
     \textbf{e} \cdot \wt{\grad} \wt{\textbf{u}}_1
\end{align}
where the boundary at infinity are to be applied and not those close to the droplet.
The solutions for $\wt{\textbf{u}}_0, p_0 = (0,0)$ due to the BCs at infinity and the absence of BCs.
\subsection{Matching principle}
\begin{equation}
    \lim_{\wt{\textbf{r}}\to 0} \wt{\textbf{u}}_0
    = \lim_{\textbf{r}\to \infty} \textbf{u}^{(0)}
\end{equation}
because the problem is regular at this order.

Hence,
\begin{equation}
    (p^{(0)}_{i/o}, \textbf{u}^{(0)}_{i/o}) = (\textbf{P}_{i/o}, \textbf{U}_{i/o}) \cdot \textbf{e}
\end{equation}

\subsection{Solution $\wt{\textbf{u}}_1$ at first order}

Back into normal coordinate the equation for $\wt{\textbf{u}}_1$ reads as,
\begin{equation}
    - {\grad} {p}_1 +  {\grad}^2 {\textbf{u}}_1
    =
     Re \textbf{e} \cdot {\grad} {\textbf{u}}_1
\end{equation}
which green function reads $X = Re (\textbf{e}\cdot \textbf{r} - r)/2$,
\begin{equation}
    \textbf{u}_1 =
    \frac{e^X}{4\pi r} \textbf{b}
    -
    \frac{Re}{4}(\textbf{e} - \textbf{n})
    (\textbf{e} - \textbf{n})\cdot \textbf{b}
    \frac{(1-X) e^X - 1}{X^2 4 \pi}
    - \frac{e^X- 1}{X 8 \pi r}  (\bm\delta  - \textbf{nn} )\cdot \textbf{b}
\end{equation}
% Or more compactly by,
% \begin{align}
%     \textbf{u}_1 &=
%     \frac{e^X}{4\pi r} \textbf{b}
%     - \grad \left(
%         \frac{1-e^X}{X8\pi}(\textbf{e} - \textbf{n})
%     \right)\cdot \textbf{b}\\
% \end{align}
Anyhow for small X we get,
\begin{align}
    \textbf{u}_1 &\approx
    \frac{1+X}{4\pi r} \textbf{b}
    -
    \frac{Re}{4}(\textbf{e} - \textbf{n})
    (\textbf{e} - \textbf{n})\cdot \textbf{b}
    \frac{-1/2-X/3}{ 4 \pi}
    - \frac{1+X/2}{8 \pi r}  (\bm\delta  - \textbf{nn} )\cdot \textbf{b}\\
    % &= \frac{1+Re (\textbf{e}\cdot \textbf{r} - r)/2}{4\pi r} \textbf{b}
    % +
    % Re(\textbf{e} - \textbf{n})
    % (\textbf{e} - \textbf{n})\cdot \textbf{b}
    % \frac{1}{ 32 \pi}
    % - \frac{1+Re (\textbf{e}\cdot \textbf{r} - r)/4}{8 \pi r}  (\bm\delta  - \textbf{nn} )\cdot \textbf{b}\\
    % &\approx \frac{1}{4\pi r} \textbf{b}
    % - \frac{1}{8 \pi r}  (\bm\delta  - \textbf{nn} )\cdot \textbf{b}
    % =
    % \frac{1}{8 \pi r}  (\bm\delta  + \textbf{nn} )\cdot \textbf{b}\\
    % &+
    % \frac{Re (\textbf{e}\cdot \textbf{r} - r)}{8\pi r} \textbf{b}
    % +
    % Re(\textbf{e} - \textbf{n})
    % (\textbf{e} - \textbf{n})\cdot \textbf{b}
    % \frac{1}{ 32 \pi}
    % - \frac{Re (\textbf{e}\cdot \textbf{r} - r)}{32 \pi r}  (\bm\delta  - \textbf{nn} )\cdot \textbf{b}\\
    &=\frac{1}{8 \pi r}  (\bm\delta  + \textbf{nn} )\cdot \textbf{b} +
     \frac{Re}{32\pi }\textbf{b}\cdot \left\{
        % (\textbf{e}\cdot \textbf{r} - r) \bm\delta
        % +
         (\textbf{e} - \textbf{n})
        (\textbf{e} - \textbf{n})
        +(\textbf{e}\cdot \textbf{n} - 1)  (3\bm\delta  + \textbf{nn} )
    \right\}
\end{align}
which
 is the Stokeslet for sa shere in pure stokes flow hence the matching is okay.
The outter expansin written in normal coordinate must remains consistant with the inner expansion for small $(Re,r)$.

\section{Fourier transform of spherical harmonics functions}

We define,
\begin{align}
    \hat{f}(\textbf{k}) = F(f) = \int f(\textbf{r}) e^{-i\textbf{k}\cdot \textbf{r}} d\textbf{r}
    &&
    f(\textbf{r}) = F^{-1}(g)= \frac{1}{(2\pi)^3}\int g(\textbf{k}) e^{i\textbf{k}\cdot \textbf{r}} d\textbf{k}
\end{align}
Then, some interesting properties can be stated,
\begin{align}
    F(\grad f)
    % = \int \grad f(\textbf{r}) e^{-i\textbf{k}\cdot \textbf{r}} d\textbf{r}
    % = [f \frac{1}{-i \textbf{k}}e^{-i \textbf{k}\cdot \textbf{r}}]_{\mathbb{R}^3}
    % - \int \grad f(\textbf{r}) \grad e^{-i\textbf{k}\cdot \textbf{r}} d\textbf{r}
    =
    i \textbf{k} \int f(\textbf{r}) e^{-i\textbf{k}\cdot \textbf{r}} d\textbf{r}
    =
    i \textbf{k} g
    &&
    F^{-1}(\grad_\textbf{k} g)
    =
    \frac{-i\textbf{r}}{8\pi^3} \int g(\textbf{k}) e^{i\textbf{k}\cdot \textbf{r}} d\textbf{r}
    =
    - i \textbf{r} f
\end{align}
We deduce that, inversly,
\begin{align}
    F^{-1}(\textbf{k} g)
    =
    - i \grad f
    &&
    F(\textbf{r} f)
    = i \grad_\textbf{k} g
\end{align}
These properties can by applied as much as we want hence,
\begin{align}
    F(\grad^{(n)} f) = (i\textbf{k})^{(n)}g
    &&
    F^{-1}(\grad_\textbf{k}^{(n)} g) = (-i\textbf{r})^{(n)} f\\
    F(\textbf{r}^{(n)} f) = (i\grad_\textbf{k})^{(n)} g
    &&
    F^{-1}(\textbf{k}^{(n)} g) = (-i\grad)^{(n)}f
\end{align}

Any stokes flow solution is written in terms of decaying harmonics functions hence like,
\begin{equation}
    \textbf{r}^{(m)}\grad^{(n)}(1/r),
\end{equation}
for purposes of generality we added the $\textbf{r}^{(m)}$ (because we can).
The two fundamentals TF are given by,
\begin{align}
    % F(\delta(\textbf{r})) = 1 && F(1) = \frac{1}{8\pi^3}\delta(\textbf{k}) &&
     F(1/r^2) = \frac{1}{2\pi^2k},
     && F(1/r) = \frac{4\pi}{k^2},
\end{align}
% The demo is
% \begin{align}
%     \int \frac{1}{r^2}e^{-i\textbf{k}\cdot \textbf{r}} d\textbf{r}
%     &= \int \frac{\sin\theta}{1}e^{-ikr\cos\theta} drd\theta d\phi\\
%     &= 2\pi \int \frac{1}{r k i}[e^{-ikr} - e^{ikr}]_0^\pi dr
%     &= 2\pi \int \frac{-2 \sin(kr)}{r^2 k i} dr
% \end{align}
Hence, the spherical harmonics can be represented as,
\begin{align}
    F(\textbf{r}^{(m)}\grad^{(n)}(1/r))
    % =(i\grad_\textbf{k})^{(m)} (i\textbf{k})^{(n)} F(1/r)
    =4\pi i^{m+n}
    \grad_\textbf{k}^{(m)} \frac{\textbf{k}^{(n)}  }{k^2}\\
    F(\grad^{(n)}(\textbf{r}^{(m)}/r))
    % =(i\textbf{k})^{(n)} (i\grad_\textbf{k})^{(m)}  F(1/r)
    =4\pi i^{m+n}
    \textbf{k}^{(n)}  \grad_\textbf{k}^{(m)} \frac{ 1}{k^2}
\end{align}
And the ``pairs'' harmonics as,
\begin{align}
    F(\textbf{r}^{(m)}\grad^{(n)}(1/r^2))
    % =(i\grad_\textbf{k})^{(m)} (i\textbf{k})^{(n)} F(1/r)
    =\frac{i^{m+n}}{2\pi^2}
    \grad_\textbf{k}^{(m)} \frac{\textbf{k}^{(n)}  }{k}
    \\
    F(\grad^{(n)}(\textbf{r}^{(m)}/r^2))
    % =(i\textbf{k})^{(n)} (i\grad_\textbf{k})^{(m)}  F(1/r)
    =\frac{i^{m+n}}{2\pi^2}
    \textbf{k}^{(n)}  \grad_\textbf{k}^{(m)} \frac{ 1}{k}
\end{align}
Taking the inverse of thoses transform one get,
\begin{align*}
    F^{-1}(\textbf{k}^{(n)}  \grad_\textbf{k}^{(m)} \frac{ 1}{k})
    =
    \frac{2\pi^2}{i^{m+n}}\grad^{(n)}(\textbf{r}^{(m)}/r^2)\\
    % =(i\textbf{k})^{(n)} (i\grad_\textbf{k})^{(m)}  F(1/r)
    F^{-1}(\textbf{k}^{(n)}  \grad_\textbf{k}^{(m)} \frac{ 1}{k^2})
    =
    \frac{1}{4\pi i^{m+n}}\grad^{(n)}(\textbf{r}^{(m)}/r)
\end{align*}
The transform of the $r^n$ with $n$ positive \textbf{and pair} is found to be,
\begin{align}
    F(r^{2n-2})
    % =(i\grad_\textbf{k})^{(m)} (i\textbf{k})^{(2n)} F(1/r)
    % =\frac{i^{2n}}{2\pi^2}
    % \grad_\textbf{k}^{2n} \frac{1}{k}
    =\frac{i^{2n}}{2\pi^2}
    \grad_\textbf{k}^{2n-2} \delta(\textbf{k})
    && \forall n\in \mathbb{N} \\
    F(r^{2n-1})
    % =(i\grad_\textbf{k})^{(m)} (i\textbf{k})^{(n)} F(1/r)
    =4\pi i^{2n}
    \grad_\textbf{k}^{2n} \frac{1}{k^2}
    &&  \forall n\in \mathbb{N}\\
\end{align}
The inverse transform of the $k^n$ with $n$ positive and pair is given by
\begin{align*}
    F^{-1}(k^{2n-1} )
    =
    \frac{2\pi^2}{i^{2n}}\grad^{2n}(1/r^2)\\
    % =(i\textbf{k})^{(n)} (i\grad_\textbf{k})^{(m)}  F(1/r)
    F^{-1}(k^{2n-2})
    =
    \frac{1}{4\pi i^{2n}}\grad^{2n-1}\delta(\textbf{r})
\end{align*}


which corresponds to the spherical harmonics in fourier space.
These have been obtain by contraction of the previous formulas on the $\textbf{r}^{(n)}$ and setting no derivatives.
For $n =2$ we get,
% \begin{align}
%     F(r^{2})
%     % =(i\grad_\textbf{k})^{(m)} (i\textbf{k})^{(2n)} F(1/r)
%     =\frac{-1}{2\pi^2}
%     \grad_\textbf{k}^{2} \delta(\textbf{k})
%      \\
%     F(r^1)
%     % =(i\grad_\textbf{k})^{(m)} (i\textbf{k})^{(n)} F(1/r)
%     =
%     - 8\pi
%     \frac{1}{k^4}
%     \\
%     F(r^{3})
%     % =(i\grad_\textbf{k})^{(m)} (i\textbf{k})^{(n)} F(1/r)
%     =
%     - 96\pi \frac{1}{k^6}
%     \\
% \end{align}
The problem still is if $n$ is odd and negative. In this case we don't have any formulas.
It seems that you are supposed to invert the tensor not the scalar function which reduce to that shit.


Because,
\begin{align}
    \grad (1/r) &= - \textbf{r} r^{-3}\\
    \grad^{(2)} (1/r) &= - \bm\delta r^{-3} + 3 \textbf{rr} r^{-5}\\
    \grad^{(3)} (1/r) &=  3 (\textbf{r}\bm\delta +^\dagger\textbf{r}\bm\delta + \bm\delta \textbf{r}) r^{-5} - 15  \textbf{rrr} r^{-7}\\
    \grad ^{(4)} (1/r) &=  3r^{-5}(\bm\delta \bm\delta) ^{Sym}-15 (\bm\delta \textbf{rr})^{Sym} r^{-7} - 105  \textbf{rrrr} r^{-9}\\
    \vdots\\
    \grad (1/r^2) &= - 2\textbf{r} r^{-4}\\
    \grad^{(2)} (1/r^2) &= -2 \bm\delta r^{-4} + 8 \textbf{rr} r^{-6}\\
    \grad^{(3)} (1/r^2) &=  6 (\textbf{r}\bm\delta +^\dagger\textbf{r}\bm\delta + \bm\delta \textbf{r}) r^{-8} - 48  \textbf{rrr} r^{-8}\\
    \grad^{(4)} (1/r) &=  8r^{-6}(\bm\delta \bm\delta) ^{Sym}-48 (\bm\delta \textbf{rr})^{Sym} r^{-8} - 384  \textbf{rrrr} r^{-10}\\
\end{align}

\begin{align}
    F(\frac{-\textbf{r}}{r^3}) =F(\grad(1/r))= 4\pi i \textbf{k}/k^2
    &&
    \text{also}
    - F(\frac{\textbf{r}}{r^3})
    = (i\grad_\textbf{k}) F(1/r^{3})
    = 4\pi i \textbf{k}/k^2
\end{align}
So finding this transform involve integrating this function on the whole space
\begin{equation}
    \grad_k F(1/r^{3}) = 4\pi \textbf{k} / k^2
\end{equation}
if $F = -4\pi \ln k+c_1$ then, $\grad F = 4\pi \frac{\textbf{k}}{k^2}$
the minus sign gives $ln(1/k)$ héhé


\begin{equation}
    F(\grad(1/r^3)) = i\textbf{k} (-4\pi \ln k+c_1 ) = F(-\textbf{r}r^{-4}) = -i\grad_k F(r^{-4})
\end{equation}
multiplying both side by $\textbf{k}\cdot $ we get
\begin{equation}
    k^2 (-4\pi \ln k+c_1
     )  = -\partial_k F(r^{-4})
     \longleftrightarrow
     F(r^{-4})=4\pi [\frac{1}{3}k^3\ln(k)- \frac{1}{9}] + k^3/3c_1 + c_2
\end{equation}

\begin{equation}
    \avg{\chi_f \bm\sigma_f^0 :\grad \textbf{u}_f }
    + \avg{\chi_f \bm\sigma_f^0 :\grad \textbf{u}_f' }
    =
    \int
    \avg{\chi_f \bm\sigma_f^0 :\grad \textbf{u}_f^0 \delta_{nst}} d\textbf{r}
    =
\end{equation}

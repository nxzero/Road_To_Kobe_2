\chapter{The hybrid model for dispersed multiphase flow made of fluid particles}
% \chapterquote{The motion of a suspension can be viewed in two ways.}{R. Jackson}{Poet}
\label{chap:avg}

As stated in \ref{chap:intro} it is computationally too expensive to carry out direct numerical simulation of multiphase flows at the industrial scales. 
Additionally, the microscale interactions themselves are not relevant to optimize these processes, only the macroscopic quantities are significant. 
Therefore, we require laws that can describe the macroscopic scale without having to solve the computationally expensive multiphase problems at the particle scale. 
This is the aim of averaging or Up-scaling technics.
Indeed, by averaging the quantities over a representative volume, timescale or realization, it is possible to derive macro-scale conservation equations for the mean fields, such as the mean velocity and pressure fields. 
Even though we solve these equations for averaged variables, the macro-scale equations still require information related to the microscale phenomena.
The connection between the micro and macro scales is established through different closure laws in these equations. 
These terms, are referred to as the closure terms, they are the mathematical representations of the microscale behavior incorporated into the macro-scale equations
For dispersed multiphase flows, several methods are then available to derive these averaged equations.  
Each of these methods are suited to a given problem depending on the flow's nature and topology. 
In the following, we make an overview of those technics. 

Multiphase flows can be categorized into dispersed flows, such as bubbly flows or emulsions, and arbitrary flows, where the mixture does not exhibit a distinct geometry.
Even though the industrial context of this work focus on dispersed two phases flows we will be interested in a more general formalism in this manuscript. 
Now, let's have a closer look to the different model exposed in the literature. 
Numerous authors introduced theoretical framework to derive averaged equations.
\citet{drew1983mathematical} and \citet{ishii2010thermo} performed ensemble average on the equations of motion, for a mixture of immiscible fluid. 
They make no assumption on the phases' topology therefore this formalism is somewhat the most general. 

Regarding the dispersed two-phase flows, many authors derived averaged equations considering in the first place solely solid spherical particles mono-disperse suspension.
In \citet{jackson1997locally,anderson1967fluid} the continuous-phase conservation equations are averaged with the volume average method.
Then, they consider the dispersed phase as a Lagrangian-phase and perform the average defining an average operator related to the center of mass of the particles instead of their whole volume. 
Consequently, the first set of equations results from the continuous average of the continuous phase equations, and the second from the particular average of the dispersed phase equations, it is therefore known as the \textit{hybrid model}.  
Finally, they considered a suspension of solid spherical particles in the limit of low Reynolds number, and derived theoretical expressions for the closure terms based on classic theoretical results of hydrodynamic.
A different methodology was proposed by \citet{zhang1994averaged} where they also considered a suspension of equal rigid spheres.
They carried out ensemble average method on both phases and derive rigorously the averaged equations similar to \citet{jackson1997locally}.
The closures terms were then derived considering inviscid flows in the dilute limit. 
The statistical averaging method, used in \citet{zhang1994averaged} can  be applied considering an arbitrary number of particles' internal coordinates.  
As an example, in \citet{zhang1994ensemble} they consider variable diameter of spheres with a constant mean diameter and derived with the statistical averaged conservation equations. 
These considerations involve inevitably additional closure terms related to the microscale variation of the diameters.
On another hand, the time averaging method is used by \citet{ishii2010thermo} to derive the macroscopic governing equations, which again, yields the same equations of conservation. 
Although the statistical average turns out to be more practical for the general cases, the other average methods are of course equivalent and lead to the same macroscopic equations. 

Most of the studies mentioned before focus on mono-disperse two phases flows. 
However, fluid-fluid multiphase flows are poly disperse in nature.  
The size distribution of the particles greatly influence the hydrodynamics, energy transfer and mass transfer phenomenons.
Besides, the closure terms of the averaged equations of motion depend highly on the size distribution of particle size. 
Consequently, the population balance equations (PBE) are needed to predict the size distribution in poly disperse flows.
PBE have been derived in the context of Crystallizer in chemical industry \citep{randolph2012theory}.
Numerous authors \citep{marchisio2013computational,fox2022hyperbolic,morel2015mathematical} have been using  and developing those models in the purpose of bubbly flows and emulsion modeling. 

As can be observed, there is numerous framework and tools to average conservation equations for dispersed multiphase flows. 
The suitability of a particular method for the derivation of the averaged equations depend on the physical characteristics of the phases present and the topology of the flow being considered.
In other words, one method may be better adapted than another depending on these factors.
The studies regarding the derivation of the dynamical conservation equations mostly focus on suspension of mono-disperse solid spherical particles.
In some cases they consider other type of particles but still with the mono-disperse assumption.  
As it will be shown the conservation equations using this framework and the PBE use different type of averaging technics and therefore involve different type of averaged velocity fields.
This distinction is often neglected while solving the PBE equations together with the conservation equations from the hybrid models \citet{KAMP20011363}.
While \citep{zaepffel2011modelisation} acknowledges that it is possible to consider the issue, it is noted that doing so can result in mathematical complexities.

In this chapter we present the classic equations of conservation for multiphase flow based on the formulation of \citet{drew1983mathematical,kataoka1986local,ishii2010thermo}. 
Which can be referred to as the \textit{continuous} averaged equations for multiphase flows. 
We then derive a new Lagrangian model for fluid particles, in the continuity of the study of \citet{morel2015mathematical,zaepffel2011modelisation}.
\tb{
By making use of this Lagrangian description, we present the classic way to model poly-dispersity with PBE based on \citet{sporleder2012population,marchisio2013computational,randolph2012theory} and introduce the dynamic Lagrangian equation of kinetic theory.
As it will be shown this formalism is not explicitly linked to the continuous phase averaged equations as both average methods are different.} 
Afterward we apply volume average to the Lagrangian local balance equations, so that we obtain the \textit{particular} averaged equations for the dispersed phase.  
Besides, we demonstrate that it is possible to include change in topology in that model. 
Up to this point we present a set of equation for the continuous phase obtained with the \textit{continuous} averaged method and another for the dispersed phase derived with the \textit{particular} average method. 
Together they form the hybrid model for fluid particles. 
The link and compatibility between the PBE and the classic  generalized hybrid model is discussed, and the latter is shown to fix the inconsistency that the PBE induced. 
On the other hand, as it will be shown in this chapter, the particular averaged equations consider by essence point of mass particles whether it is obtained with statistical approach and PBE or volume averaged Lagrangian equations.
While the continuous average method applied to the dispersed phase by definition doesn't make this point of mass assumption.  
Meaning that two option are available to average the dispersed phase, both leading to different averaged equations as one is continuous and the other discrete. 
Therefore, in the last section of this chapter we address the problem of the equivalence between both method of derivation. 
It is shown that both method are in fact consistent and lead to the same hybrid models but still yields some differences. 
At last, we present the set of equation  and closure terms of the hybrid model for fluid particles.  



\phantomsection\addcontentsline{toc}{chapter}{Front matter}
\chapter*{\centering Abstract}
\phantomsection\addcontentsline{toc}{section}{Abstract}

Buoyancy-driven droplet flows are encountered in many chemical engineering processes such as gravity separators and liquid-liquid extractors. 
These systems exhibit a wide range of scales, from the size of individual inclusions (as small as a few micrometers) to the size of the reactor (often exceeding one meter), making fully resolved simulations computationally impractical.
As a result, the current engineering practice relies on averaged equations of motion for both the dispersed and continuous phases.

Regarding the modeling of dispersed two-phase flows, previous studies have primarily focused on suspensions of solid spherical particles. 
However, significantly less work has been devoted to the modeling of emulsions. 
Hence, the primary focus of this PhD work is the derivation of a set of averaged equations capable of describing suspensions of arbitrarily shaped fluid inclusions and complex surface properties. 
The dispersed phase is represented through Lagrangian-averaged conservation laws, while the continuous phase is modeled using Eulerian-averaged conservation laws. 
Consequently, this formalism is referred to as the ``hybrid model''. 

The second aspect of this PhD is the development of closure models to feed the equations of the ``hybrid model''. 
Specifically, we derive closure laws for the momentum and energy-averaged equations in the dilute and Stokes flow regimes considering mono-disperse emulsions of spherical droplets. 
These theoretical investigations are supplemented by Direct Numerical Simulations (DNS) of buoyant emulsions of droplets, conducted using the open-source code \texttt{Basilisk C}. 
Based on theoretical analysis, DNS results, and findings from the literature, we propose models for the interphase drag force, stresslet tensor, and Reynolds (or pseudoturbulence) stress tensor applicable to rising homogeneous emulsions with finite inertial effects and in non-dilute regimes. 
Our results suggest that including the drift velocity contribution, not only in the inter-phase drag force but also in the effective stress of the continuous phase averaged momentum equation, is essential to model the rheology of emulsions. 

The final contribution of this work concerns the study of the nearest-particle Statistics pair distribution, used to characterize the microstructure of the emulsion and the relative kinematics   of interacting droplets.
It is shown that the microstructure geometry can be described using the second moment of the nearest-particle pair distribution, which quantifies features such as clusters and layers of droplets.
We then analyze the microstructure kinematics through the derivation of a transport equation for this quantity.
In particular, it is shown that the mean interaction time of droplets corresponds to the relaxation time of the second moment of the nearest-particle pair distribution. 
Hence, this timescale governs the formation of the microstructure.


% Overall this PhD work offers mathematical tools necessary to model emulsions within a multiscale ``hybrid'' approach.


\newpage
\chapter*{\centering R\'esum\'e}
\phantomsection\addcontentsline{toc}{section}{R\'esum\'e}

 
Les \'ecoulements de gouttes entra\^in\'es par la flottabilit\'e se rencontrent dans de nombreux proc\'ed\'es de g\'enie chimique, tels que les s\'eparateurs gravitaires et les extracteurs liquide-liquide.
Ces syst\`emes pr\'esentent une large gamme d'\'echelles, allant de la taille des inclusions individuelles (aussi petites que quelques microm\`etres) \`a celle des r\'eacteurs (souvent sup\'erieure \`a un m\`etre), ce qui rend les simulations enti\`erement r\'esolues impossibles \`a r\'ealiser avec les ressources informatiques actuelles.
Par cons\'equent, les pratiques actuelles de mod\'elisation reposent sur l'utilisation des \'equations moyenn\'ees qui d\'ecrivent l'\'evolution des phases dispers\'ees et continues.


Concernant la mod\'elisation des \'ecoulements diphasiques dispers\'es, les recherches se sont principalement concentr\'ees sur les suspensions de particules solides sph\'eriques, tandis que beaucoup moins de travaux ont \'et\'e consacr\'es \`a la mod\'elisation des \'emulsions.
L'objectif principal de cette th\`ese est donc la d\'erivation d'un ensemble d'\'equations moyen\-n\'ees capables de d\'ecrire des \'ecoulements dispers\'es avec des inclusions fluides. 
La phase dispers\'ee est repr\'esent\'ee par des lois de conservation lagrangiennes  moyenn\'ees, tandis que la phase continue est mod\'elis\'ee par des lois de conservation eul\'eriennes moyenn\'ees.
Ce formalisme est donc appel\'e  ``mod\`ele hybride''.

Le deuxi\`eme aspect de cette th\`ese est le d\'eveloppement de mod\`eles de fermeture pour alimenter les \'equations du  ``mod\`ele hybride'' .
Plus pr\'ecis\'ement, nous d\'erivons des lois de fermeture pour les \'equations moyenn\'ees de quantit\'e de mouvement et d'\'energie cin\'etique dans les r\'egimes dilu\'es et de Stokes, en consid\'erant des \'emulsions monodisperses de gouttes sph\'eriques.
Ces \'etudes th\'eoriques sont compl\'et\'ees par des simulations num\'eriques directes (DNS) d'\'emulsions de gouttes soumises \`a la gravit\'e, r\'ealis\'ees \`a l'aide du code open source \texttt{Basilisk C}.
Avec les analyses th\'eoriques, r\'esultats DNS, et des donn\'ees de la litt\'erature, nous proposons des mod\`eles pour la force de tra\^in\'ee interphasique et le tenseur des contraintes de Reynolds (ou pseudoturbulence), applicables aux \'emulsions homog\`enes ascendantes dans des r\'egimes non dilu\'es et \`a effets inertiels finis.
Nos r\'esultats sugg\`erent que l'inclusion de la contribution de la vitesse relative entre phase dispers\'ee et continue, non seulement dans la force de tra\^in\'ee interphasique, mais aussi dans la contrainte effective de la phase continue, est essentielle pour pr\'edire la rh\'eologie des \'emulsions.

La contribution finale de ce travail concerne l'\'etude de la fonction de corr\'elation de paires des voisines les plus proches, utilis\'ees pour caract\'eriser la microstructure de l'\'emulsion et la cin\'ematique relative des gouttes en interaction.
Nous montrons que la g\'eom\'etrie de la microstructure peut \^etre d\'ecrite en utilisant le second moment de cette distribution, qui quantifie des caract\'eristiques telles que les amas et les couches de gouttes form\'ees dans l'\'ecoulement.
Nous proposons ensuite d'analyser la cin\'ematique de la microstructure \`a travers la d\'erivation d'une \'equation de transport pour ce tenseur.
En particulier, il est montr\'e que le temps moyen d'interaction des gouttes correspond au temps de relaxation du second moment de la distribution des paires de particules les plus proches.
Ce temps caract\'eristique gouverne ainsi la formation de la microstructure.

\chapter{Getting the closure terms though Direct Numerical simulations}
\label{chap:DNS}

In the preceding chapter we arrived to a clear picture of the needed closures terms.
We conclude that even through a lot of studies have been conducted on the closures for bubbly flow, few of them provided closures for emulsions.
Consequently, in this chapter we will focus on how to obtain the closures terms through DNS. 
The physics of suspension or emulsion is rather complex and involves specific phenomenons.
Indeed, depending on the physical parameters, in a buoyant driven emulsion, the drops are more likely to arrange them self to line up horizontally "raft" or in column \citep{tryggvason2011direct} \citep{guazzelli2011}. 
Besides, \citet{davis1985sedimentation} Studied the physics of the Sedimentation processes of poly disperse suspension of particles. 
They found out that different regions in a vessel, was concentrated in a given size distribution.
In dispersed pipe flows \citet{morel2010comparison} show that the larger bubbles tend to go toward the axis of the pipe, while the smaller outward of the bubble columns.
Consequently, to model those phenomenons with Euler-Euler models, accurate closure are needed while computing the averaged Navier-Stokes equations and population balance equations.
In the first section we will review the current stats of the Euler-Euler simulation modeling to get a much clearer view on what is done nowadays.
The identification of the issues and capabilities of the models will suggest the actual needs concerning the closure terms modeling.
Then we make an overview of the author who studied the microscale with DNS in order to find how they computed the closure terms.
Once the methods are clearly identified we expose our strategy and the DNS setup to model representative volume of emulsion. 
In the last parts we present a glimpse of our first DNS results and conclusion on the future projects. 


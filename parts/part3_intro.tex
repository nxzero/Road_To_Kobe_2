\section{Introduction}
    

In the preceding chapters, we arrived at a clear picture of the needed closure terms.
We conclude that even though a lot of studies have been conducted on the closures for bubbly flow, few of them provided closures for emulsions.
Consequently, in this chapter, we will focus on how to obtain the closure terms through DNS. 
The physics of suspension or emulsion is rather complex and involves specific phenomena.
For example, depending on the physical parameters, in a buoyant-driven emulsion, the drops are more likely to arrange themself to line up horizontally "raft" or in column \citep{tryggvason2011direct} \citep{guazzelli2011}. 
Besides, \citet{davis1985sedimentation} Studied the physics of the Sedimentation processes of polydisperse suspension of particles. 
They found out that different regions in a vessel were concentrated in a given size distribution.
In dispersed pipe flows \citet{morel2010comparison} show that the larger bubbles tend to go toward the axis of the pipe, while the smaller outward of the bubble columns.
Consequently, to model those phenomenons with Euler-Euler models, numerous closures are needed to represent accurately this kind of phenomenon.
Ideally, these closures should account for every parameter, including polydispersity and variations in volume fraction. However, as this constitutes an enormous task, we focus here solely on the homogeneous and monodisperse scenario   

Thus, we will first review the current stats of the Euler-Euler simulation modeling to get a much clearer view of what is done nowadays.
The identification of the issues and capabilities of the models will suggest the actual needs concerning the closure terms modeling.
Then we make an overview of the studies focusing on the micro scale with DNS in order to understand what is the usual procedure to compute the closure terms through DNS.
Once the methods are identified we expose our strategy and the DNS setup to model a representative volume of buoyant emulsion. 
In the following chapters, we present the actual DNS results and development of closure terms. 


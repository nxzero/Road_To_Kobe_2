\chapter*{\centering Remerciement}
\phantomsection\addcontentsline{toc}{section}{Acknowledgements}
%\begin{stretchpars}
\noindent 
Je tiens avant tout à remercier Jean-Lou Pierson de m'avoir accueilli pour mon stage de master à l'IFPEN, puis pour ce sujet de thèse et enfin pour un post d'ingénieur ! 
Je te remercie de m'avoir fait découvrir le monde de la recherche, 
de m'avoir enseigné tout un tas de concept théorique quand je suis arrivé, 
de m'avoir poussé toujours plus loin dans mes projets, 
d'être un exemple en termes de rigueur scientifique (ce qui n'était vraiment pas mon fort quand je suis arrivé), et 
d'avoir toujours sur répondre à mes questions et me guider dans ma thèse vers les bons papiers et livres (merci de m'avoir donné à lire ``Jackson 1997'' je crois que c'est avec ça que ma thèse a commencé).
J'ai énormément  appris à travailler à tes côtés, un grand merci !
PS: désolé de t'avoir fait travailler les dimanches sur ma thèse,  maintenant il faut faire de la \textit{slow science} ! 

Merci également à Lionel Gamet pour avoir encadré mon stage de Master 2 et d'avoir continué à m'aider de temps à autre pour ma thèse, et surtout merci de m'avoir enseigné les subtilités de \texttt{Linux}, avec le fameux \texttt{.rpmmacros}, on en a bien besoin là où on est !  
Un plaisir de continuer à travailler avec toi.
 
\'Evidement un grand merci à Stéphane Popinet pour avoir dirigé cette thèse, merci pour tous tes conseils et les échanges qu'on a pu avoir, qui ont su me guider du début à la fin de ce travail. 
J'ai beaucoup appris quand je me torturais à comprendre le \texttt{ast/translate.c} de \texttt{Basilisk}, mais c'est une fierté pour moi d'avoir pu contribuer à ton code \texttt{Basilisk} et plus généralement d'avoir travaillé avec toi. 
Merci de m'avoir accueilli à la Sorbonne, et de m'avoir fait présenter mon travail à Daniel Lhuillier et Rodey Fox, ce qui a constitué le point pivot de cette thèse.  

Donc en plus de mes encadrants j'aimerais bien évidement remercier Daniel Lhuillier qui a pris part à la direction de cette thèse de manière  officieuse. 
Tu m'as montré qu'il était possible d'exposer la dérivation et la physique des équations moyennées avec simplicité et pédagogie, j'espère pouvoir en faire autant dans ma carrière, merci !  

Je tiens à remercier également les deux rapporteurs de cette thèse Duan Zong Zang et Olivier Simonin pour avoir lu ce manuscrit en détail et avoir mis le doigt sur certains problèmes et points importants, et pour toutes les discussions intéressantes qu'on à pu avoir à l'ICMF 2025. 
Je remercie également les autres membres du jury Fabien Candelier, Aurore Naso et Stéphane Zaleski pour leur intérêt et leurs retours sur ce travail. 
Merci à tous les autres chercheurs avec qui j'ai pu interagir régulièrement ou ponctuellement (Prabuh nott,  Thomas Pathz, Howard Stone\ldots), cela m'a permis de réaliser ce travail. 


Merci à tous les membres du département R174 pour leur accueil quand je suis arrivé en thèse et pour mon arrivée en CDI il y a quelques mois. 
Merci à mes camarades de thèse Kamel et Paul pour toutes ces discussions scientifiques inspirantes, et les autres discussions moins scientifiques, mais tout autant inspirantes ! Et bon courage à vous deux pour la suite de  votre thèse ça va le faire. 

Merci à mes amis proches qui se reconnaiteront, et à ma famille pour tout le soutien apporté durant ces trois ans de thèse et plus généralement dans mes études. 

Et enfin merci à Camille Chavassieux qui aura su m'éloigner du travail et m'apporter un soutien quotidien et constant durant ces trois ans, indispensable pour être heureux et équilibré dans la vie.  

\newpage
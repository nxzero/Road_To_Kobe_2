\chapter*{\centering Remerciement}
\phantomsection\addcontentsline{toc}{section}{Acknowledgements}
%\begin{stretchpars}
\noindent 
Je tiens avant tout a remercier Jean-Lou Pierson de m'avoir aceuillie pour mon stage de master à l'IFPEN, puis de m'avoir acceuil une nouvelle fois pour ce sujet de thèse et enfin de m'avoir acceuil une dèrnière fois à l'IFPEN pour un post d'ingénieur ! 
Je te remerci de m'avoir fait découvrir le monde de la recherche, 
de m'avoir enseigner la ``slender-body-theory'' quand je suis arrivé, 
de m'avoir poussé toujours plus loin dans mes projet, 
et de m'avoir fait confiance dans mes idées  (car on a un peu changer le programme de la thèses),
d'être un exemple en terme de rigeurs scientique (ce qui n'était vraiment pas mon fort qd je suis arrivé), 
d'avior toujour sur répondre à mes questions et me guider dans ma thèse vers les bons papiers et livres (merci de m'avoir donné à lire "Jackson 1997" je crois que c'est avec ca que ma thèse a commencé, et de me prêter ta bible: "Jackson 2000", désolé de l'avior un peu abimé d'ailleurs !).
J'ai énormément  appris a travailler à tes cotés, un grand merci !
PS: désolé de t'avoir fait travailler les dimanches sur ma thèse,  maintenant il faut faire de la \textit{slow science} ! 

Merci à Lionel Gamet bien-sur d'avior encadré mon stage de Master 2 également, de m'avoir enseigner les subtiliées de \texttt{Linux} et (particulièrement du fameux \texttt{.rpmmacros}) et de m'avoir appris a faire des maillage butterfly.  
Un plaisir de continuer a travailler avec toi ! 
PS: merci de nous avoir aidés a organiser la soutenance de thèse car on était un peu perdu. 
 
Merci a  Stéphane popinet mon directeur de thèse pour tout ses sage conseilles qui ont su me guider durant toute ma thèse. 
Sache que c'est une fièreté pour moi d'avoir travailler avec toi et d'avoir pu contributer à ton code \texttt{Basilisk}, j'ai bceaucoup appris en matière de programmation quand je me torturais a comprendre comment était codé le \texttt{qcc.h}. 
Merci de m'avior aceuillie a la Sorbonne, et de m'avoir fait présenter mon travail à Daniel Lhuillier et Rodey Fox ce qui constitué le point de départ de ce travail.  
PS: désoloé pour toutes ces fautes d'ortographes quand tu as relu le manuscript ! 

En plus de mes enquadrant j'aimerais bien evidement remercier Daniel Lhuillier qui a pris part à la direction de cette thèse de manière  officieuse. 
Tu m'a montré qu'il était possible d'exposer et de comprendre les equations du modèle hybrid avec simplicitées et pédagogie, ce qui a donner lieu a la rédaction des 3 premier chapitres de ce manuscript, et pour ca je te remercie. 

Je tiens a remercier également les deux rapporteur de cette thèse Duan Zong Zang et Olivier Simonin pour avoir lu ce manuscript en détail et avoir mis en anvant certain problèmes et point important. 
Je remerci égalemnt les autre membre du jury Fabien Candelier, Aurore Naso et Stephane Z. 
Merci a tous les autres chercheur avec qui j'ai pu interagir regulièrement ou ponctuellement (Prabuh nott,  Thomas Pathz, Howard Stone.)


Merci à tous les menbres du département R174 pour leurs acceuil quand je suis arrivé en thèse et pour leurs second accueil a mon arrivé en CDI. 
Merci à mes cammarades de thèse Kamel et Paul pour toutes ces discutions scientique inspirantes, et les autres discutions moin scienttifique mais tout autant inspirante; et bon courage a vous pour la suite de la thèse. 

Merci à mes autres amis proche qui se reconnaisseront, et a ma famille pour tout le soutiens porté durant ces trois ans de thèse et plus généralement dans mes étude. 

Enfin merci a Camille Chavassieux qui aura su m'éloigner du travail et m'apporter un soutiens quotidien pour être heureux et équilibré dans la vie.  

\newpage
\subsubsection{The particular-phase average} 
To derive the averaged equation for the dispersed phase two options are available.
We can derive the particular average from the dispersed-phase average \ref{eq:davgmomentum}. 
Or we derive it by applying particular average to Newton second law of motion and mass balance equation. 
The former method gives exactly the same results as the latter one for rigid particles in stokes regime \citep{nott2011suspension}. 
Indeed, the dispersed phase can be assimilated as a Lagrangian phase. 
Therefore, we define, the center of mass $\bm{y}_\alpha$ and the point velocity of the center of mass, $\textbf{u}_\alpha$, for a given particle $\alpha$.
Then the momentum balance for a Lagrangian particle reads as, 
\begin{equation}
    \label{eq:Newtion2law}
    \rho_d V_\alpha \frac{d\textbf{u}_\alpha}{dt} = \bm{f}_\alpha + V_\alpha \bm{b}_{ext}
\end{equation}
with $\bm{f}_\alpha$ the external forces on the particle,$\bm{b}_{ext}$ a constant body force fields as the gravity and $V_\alpha$ the volume of the particle $\alpha$. 
For now, we consider the volume $V_\alpha$ identical for every particle. 
We won't go into details since we derive equivalent averaged equations in the next section.
The particular-average of the mass balance equation (or the particular average of the equivalent Lagrangian phase) reads as,
\begin{equation}
    \label{eq:pavgMASS}
    \frac{\partial n}{\partial t} + \bm{\nabla}\left(n\left<\bm{u_\alpha}\right>^p\right) = 0,
\end{equation} 
where $n$ is the number density of particles.
The conservation of mass of $n$ is null there because of compressibility. 
When coalescence and break-up are taken into account (see \ref{chap:PBE}) the mass balance equation will be given a source term. 
Then we apply particular average to \ref{eq:Newtion2law}.
It yields, 
\begin{equation}
    \label{eq:pavgsp}
    \rho_d V_\alpha \left[\frac{\partial }{\partial t}(n\left<\bm{u_\alpha}\right>^p) 
    + \bm{\nabla}\cdot(n\left<\bm{u_\alpha}\right>^p\left<\bm{u_\alpha}\right>^p)\right] 
    = n V_\alpha \left<\bm{b}_{ext}\right>^p 
    + n\left<\bm{f_\alpha}\right>^p 
    - \bm{\nabla}\cdot(n\left<\bm{u_\alpha'u_\alpha'}\right>^p),
\end{equation} 
where $\left<\bm{f}\right>^p$ is the particular-average of the external forces (hydrodynamic and contact force).
We can notice that in the particular-average of the dispersed phase also include a term related to the velocity fluctuation $\bm{u'u'}$. 
Those terms are similar to the Reynolds stress term for turbulence modeling.
Nevertheless, they are not the same things, this will be discussed in a following section. 
The Lagrangian angular momentum balance can give rise of an additional particular-averaged equation. 
Even though, it is in practice rarely use, we give the expression for informative purpose.
The Lagrangian momentum balance for a rigid particle at its center of mass reads as,
\begin{equation}
    \mathcal{\bm{I}}\bm{\dot{\omega}_\alpha} = - \bm{\epsilon} : \bm{M^{h}_\alpha} + \bm{\tau_\alpha},
    \label{eq:newtion2law2}
\end{equation}
where $\bm{\omega}_\alpha$ is the angular velocity, $\mathcal{\bm{I}}$ is the inertia tensor about the center of mass and $\tau_i$ the external torque on the particle $\alpha$. 
$\bm{\epsilon}$ is the Levi-Civita $3^{th}$ order tensor, therefore $\bm{\tau^h_\alpha} = - \bm{\epsilon} : \bm{M_\alpha}$ is the hydrodynamical torque on the particle.  
It is important to notice that this equation involves the inertia tensor $\mathcal{\bm{I}}$.
Indeed, for fluid particles this quantity isn't properly defined.
This is partly why the above momentum balance does not remain valid for fluid particles.
From \citet{jackson1997locally} we found the particular-average of the angular momentum as,
\begin{equation}
    \bm{\mathcal{I}} \left[\pddt(n\left<\bm{\omega_\alpha}\right>^p)
    +\bm{\nabla}\cdot(n\left<\bm{\omega_\alpha }\right>^p\left<\bm{u_\alpha}\right>^p)\right] 
    = -\bm{\nabla}\cdot(n\left<\bm{\omega'u'}\right>^p) 
    - \bm{\epsilon} : n\left<\bm{M_\alpha}\right>^p 
    + n\left<\bm{\tau_\alpha}\right>^p.
    \label{eq:Iavg}
\end{equation}

Up to now we derived a set of 5 equations \ref{eq:pavgsp}, \ref{eq:favgsp}, \ref{eq:Iavg}, \ref{eq:Cmassad} and \ref{eq:Cmassad} for dispersed two phase flow of solid particles. 
Whether the dispersed phase is considered as a continuous or Lagrangian.  
Note, that there is ensemble-average equations for the mass and momentum balance too (see \cite{nott2011suspension}). 
In practice only 3 of these equations need to be solved to obtain $\left<\textbf{u}\right>^P,\left<\textbf{u}\right>^f$ and $\left<p\right>^f$. 
Usually we solve the fluid-phase average \ref{eq:favg}, the mass conservation \ref{eq:Cmassad} and the averaged-Lagrangian phase \ref{eq:pavgsp} (if we consider hybrid model).
We recall that the fluid phase averaged equation is valid for all particle's nature (deformable or solid). 
However, the averaged equation for the particular phase has been derived under the assumption of mono disperse solid particles (it is valid for any shape though).
By defining the shape of the particles and considering stokes regime, it is feasible to express theoretically the unknown terms and close the set of equations. 
Even though most of the authors considered spherical particles, some of them closed the equations for ellipsoidal particles \citep{batchelor1970stress} or poly disperse spherical particles \citep{zhang1994averaged}.
In all cases where a closure has been derived, the author considered dilute suspension and spherical particles mostly (\cite{jackson1997locally}, \cite{zhang1994ensemble} and \cite{batchelor1970stress}).  
Nevertheless, the dispersed phase can be a fluid phases, as in bubbly flows or in our case, emulsion. 
That is why we need to investigate the models taking in account the physical properties of droplets as a dispersed-phase.
That implies, taking in account the poly disperse size distribution of the drops, the surface tension and mass transfer, the deformation of the particles and a variable size distribution. 
All those phenomenons results in a quite tough problem.
Consequently, in the next few sections, we derive a new set of averaged equations considering only, a poly disperse two-phase flow with deformable particles.
Meaning that we neglect the other aspects for now.
We refer the reader to the book of \citet{morel2015mathematical} to see the derivation of the hybrid-model considering mass transfer only.  


Besides, we make use of the Taylor expansion of $g(\textbf{x},\bm{y})$ at the center of the particle $\bm{y}_\alpha$ to express the integral term in \ref{eq:favg} as a sum of terms \citep{jackson1997locally}.  
The average product of the velocity can be express as, $\left<\bm{uu}\right>^f = \left<\textbf{u}\right>^f\left<\textbf{u}\right>^f + \left<\bm{u'u'}\right>^f$, where $\bm{u'}$ is the velocity fluctuation around the mean. 
Therefore, the fluid-phase averaged momentum equation can be restated as, 
\begin{multline}
    \rho_f\pddt ((1-\phi)\left<\textbf{u}\right>^f) 
    + \rho_f\bm{\nabla}\cdot\left((1-\phi) \left<\textbf{u}\right>^f\left<\textbf{u}\right>^f\right)
    = (1-\phi)\left<\bm{b}_{ext}\right>^f 
    +\bm{\nabla}\cdot\left[(1-\phi) \left(2\mu_f\left<\bm{e}\right>^f 
    -\left<\bm{p}\right>^f\right)\right]\\
    -n\left<\bm{f}_\alpha\right>^p
    +\bm{\nabla}\cdot
    \underbrace{
        \left[
            - \left<\bm{u'u'}\right>^f 
            +n\left<\bm{M}_\alpha\right>^p 
            -\frac{1}{2}\bm{\nabla}\cdot(n\left<\bm{M}^{2h}_\alpha\right>^p) 
            + \ldots
        \right]
    }_{\text{Hydrodynamic particle stress : } \bm{\Sigma^q}},
    \label{eq:favgsp}
\end{multline}
where, $n$ is the number density of particles, it appears when taking the particular average (\ref{eq:partia}). 
We have decomposed $\bm{b}$ into a constant body force fields, $\bm{b}_{ext}$ (such as the gravity), and an inter particular force, $\bm{f}^b_\alpha$. 
Only the former contribute to the fluid phase momentum therefore, $\bm{f}_\alpha =\bm{f}_\alpha^h$
Notice that the surface force $\bm{f}_\sigma$ do not contribute to the drag force. 
Indeed, as mentioned before, the integral $\int_{S_\alpha} \bm{f}_\sigma dS = 0$.  
% Then, we defined $\bm{f}_\alpha$ as the sum of the hydrodynamical forces, $\bm{f}_\alpha^h$, and the contact forces $\bm{f}_\alpha^c$, on the particle $\alpha$. 
Also, $\bm{M}_\alpha$ and $\bm{M}^{2}_\alpha$ are the moments of respectively first and second order acting on the particle $\alpha$.
As, it has been done for the drag force the moment $\bm{M}_\alpha$ is the sum of several contributions, the hydrodynamic moment $\bm{M}^h$ the body force moment $\bm{M}^b$ and the surface tension traction $\bm{M}^h$.
Similarly, the body force moment do not contribute to the momentum of the fluid phase, thus $\bm{M}_\alpha = \bm{M}^h_\alpha +\bm{M}_\alpha^b$. 
The deviatoric part of first moments $M_{ij}^\alpha - \frac{1}{3}M_{ii}^\alpha\delta_{ij}$, is in fact made of an antisymmetric tensor, the toque, and a symmetric tensor called the stresslet.
All those terms are particular average, indeed, they are all defined at the center of mass of the particles $\alpha$. 
This is made possible by injecting the Taylor expansion of $g$ derived at the center of mass of the particle, i.e. $\bm{y}_\alpha$, into the integral term in \ref{eq:avgFnet} \citep{jackson1997locally}.
The $q^{th}$ moments of the particle $\alpha$ is,
\begin{equation*}
    \bm{M}^q_\alpha = \int_{S_\alpha} (\bm{\bm{r}_\alpha)^q } \bm{n}\cdot\bm{\sigma}dS,
    \label{eq:qthM}
\end{equation*}
where $\bm{\bm{r}_\alpha }$ is the distance from the center of mass $\bm{y}_\alpha$ to a point on the surface $S_\alpha$. 
Thus, the particle contribution to the suspension stress reads as,
\begin{equation*}
    \bm{\Sigma}^p = -\left<\bm{u'u'}\right>^f + \sum_{q=1}^\infty \left[\frac{(-1)^q}{q!} \bm{\nabla}^q n\left< \bm{M}^q\right>\right]
\end{equation*}
where have we use, 
\begin{align*}
    g(\textbf{x},\bm{y}) 
    &= g(\bm{x,y}_\alpha) 
    + \bm{r}_\alpha \cdot \hat{\bm{\nabla}} g|_{\bm{y}_\alpha} 
    + \frac{1}{2!} \bm{\bm{r}_\alpha \bm{r}_\alpha }:\hat{\bm{\nabla}}\hat{\bm{\nabla}}g|_{\bm{y}_\alpha} 
    + \ldots 
    &= \sum_{q=1}^\infty \left[\frac{1}{q!} \bm{\bm{r}_\alpha}^q\hat{\bm{\nabla}}^q  g|_{\bm{y}_\alpha}\right]
\end{align*}
where $\hat{\bm{\nabla}}$ is the local divergent operator, it reads, $\hat{\bm{\nabla}} = \frac{\partial}{\partial y_i}$,
 the $|_{\bm{y}_\alpha}$ notation mean that we take the value of the gradients at $\bm{y}_\alpha$.
And by using the property $\hat{\bm{\nabla}} g |_\alpha= - \bm{\nabla} g_\alpha $ (see \cite{anderson1967fluid}) we get, 
\begin{equation}
    g(\textbf{x},\bm{y}) 
    = \sum_{q=1}^\infty \left[\frac{(-1)^q}{q!} \bm{\bm{r}_\alpha}^q\bm{\nabla}^q  g|_{\bm{y}_\alpha}\right].
    \label{eq:expansion}
\end{equation}
The power notation, for $\bm{\bm{r}_\alpha}^q$ and $\bm{\nabla}^q$, implies that we take the tensor product $q$ times on the first order tensor,$\bm{\bm{r}_\alpha}^q$ or $\bm{\nabla}^q$.
In Einstein summation notation and for $q = 4$ it would give, $\bm{a}^4 = a_i a_j a_k a_l$, where $\bm{a}$ is a first order tensor and $\bm{a}^4$ a fourth order tensor.  
In a more rigorous manner we can note that, 
\begin{equation}
    \bm{\bm{r}_\alpha}^q\bm{\nabla}^q  
    = r^\alpha_{i_1}r^\alpha_{i_2}\ldots r^\alpha_{i_q}
    \partial_{i_1}\partial_{i_1}\ldots\partial_{i_q} ,
    =\Pi^q_{k=1}r^\alpha_{i_k}\Pi^q_{k=1}\partial_{i_k}
\end{equation}
where we have use, Einstein summation convention, i.e. we sum on all the $i_k$, and we used the shortcut, $\partial_{k} = \partial /\partial y_{k}$. 
As we can see this product results in a scalar, so any terms in \ref{eq:expansion} is of the order of $g$ which means a scalar. 
At this point it is interesting to say a few words on \ref{eq:favgsp}. 
First, this equation has been derived considering a dispersed phase and a Newtonian fluid for the continuous phase.
Thus, at this point of the development we did not make any assumption on the nature of the dispersed phase (except the fact that it is a dispersed phase obviously).   
Also, one can recognize in \ref{eq:favgsp}, the classic Navier-Stokes equation at which we added two more terms, namely, $-n\left<\bm{f}_\alpha\right>^p$ and $\bm{\nabla}\cdot\bm{\Sigma}^p$. 
The former term represents the force generated by the particles on the fluid phase, and the latter represent the additional stress brought by the presence of the particles.
Those two terms, need to be known before solving \ref{eq:favgsp}, because we solve this equation for the averaged velocity and averaged pressure only.
Therefore, we call the remaining terms, closures terms.
In \ref{sec:closure} we will discuss the different ways to express the closure terms depending on the local parameters, namely, $Re$, $\phi$, $Bo$,$\rho_r$ and $\mu_r$.

\subsection{Balance equations for an arbitrary particle shape and nature.}
\tb{Carry out the energy or kinetic energy balance}
\tb{then show that some distributions are better than other for minimizing the energy}
Let's define basic quantities of the particles. 
The mass, $m_\alpha$, the surface mass $\mathcal{S}_\alpha$, the momentum $\textbf{p}_\alpha$ and \textit{moment of momentum} $\bm{P}_\alpha$,
are defined such that,
\begin{equation}
    m_\alpha = \int_{V_\alpha} \rho_d dV,\;\;\;
    \mathcal{S}_\alpha = \int_{S_\alpha} \rho_s dS,\;\;\;
    \bm{p}_\alpha = \int_{V_\alpha} \rho_d \bm{u} dV,\;\;\;
    \bm{P}_\alpha = \int_{V_\alpha} \rho_d \bm{u}\bm{r}_\alpha dV,\;\;\;
    \tb{
    \mathcal{E^\sigma}_\alpha = \int_{S_\alpha} \rho_s \sigma dS,\;\;\;
}
\end{equation}
where $V_\alpha$ is the volume of the particle $\alpha$.
$S_\alpha$ its surface and $\rho_s$ the surface density. 
$\bm{r}_\alpha$ is defined such that, $\bm{r}_\alpha = \bm{y} - \bm{y_\alpha}$ with $m_\alpha\bm{y_\alpha} = \int_{V_\alpha} \rho_d\bm{y}dV$ is still the center of mass of the particle $\alpha$. 
We note $\bm{u}_\alpha = d\bm{y}_\alpha/dt$ the velocity of the center of mass $\alpha$, and we define the fluctuation velocity around $\bm{u}_\alpha$ as $\bm{w}_\alpha = \bm{u} - \bm{u}_\alpha$.

Then we can rewrite the above set of equations as, 
\begin{equation}
    \bm{p}_\alpha = m_\alpha \bm{u}_\alpha 
    + \int_{V_\alpha} \rho_d \bm{w}_\alpha dV,\;\;\;
    \bm{P}_\alpha = \int_{V_\alpha} \rho_d \bm{w}_\alpha\bm{r}_\alpha dV.
    \label{eq:decomposition}
\end{equation}
\paragraph*{Velocity of reference :}
The fact that $\bm{u}_\alpha$ is the velocity at the center of mass is totally arbitrary.
In fact, we could define $\bm{u}_\alpha$ as $\bm{p}_\alpha / V_\alpha$ which is more representative 
as an average. 
From now on we will note, $\bm{\overline{u}}_\alpha$ and $\bm{\overline{w}}_\alpha$, the velocity and the fluctuations of 
the velocity around the mean velocity $\bm{p}_\alpha/V_\alpha$, similarly 
$\bm{u}^c_\alpha$ and $\bm{w}^c_\alpha$ will be the velocity and fluctuation of the velocities around 
$\bm{u}(\bm{y}_\alpha)$.
Using $\bm{\overline{u}}_\alpha$ we notice that by definition 
$\int_{V_\alpha} \bm{\overline{w}}_\alpha dV = 0$, which is very convenient while deriving the averaged equations. 
Although the choice of $\bm{u}_\alpha$ change the decomposition of the linear momentum,
it does not influence the calculation of the moment of momentum since,
$\bm{P}_\alpha = \int \bm{\overline{w}}_\alpha\bm{r}_\alpha dV = \int \bm{w}^c_\alpha\bm{r}_\alpha dV$.

The antisymmetric part of  $\bm{P}_\alpha$ correspond to the angular momentum. 
It is defined as $\bm{\mathcal{I}}\bm{\omega}_\alpha = \bm{\epsilon} : \bm{P}_\alpha =\int_{V_\alpha}\rho_d \bm{w}_\alpha \times \bm{r}_\alpha dV $, we recall that the two dots represent the double contraction product, and the $\times$ represent the cross product.
Likewise, the symmetric part represent the strain momentum of the particle.
More generally we can introduce the $q^{th}$ moment of momentum, and the $q^{th}$ shape tensor of the particle $\alpha$  by respectively,
\begin{equation}
    \bm{P}_\alpha^q 
    =  \rho_d\int_{V_\alpha} \bm{r}_\alpha^q \bm{w}_\alpha dV,\;\;\;
    \mathcal{G}_\alpha^q =  \rho_d\int_{V_\alpha} \bm{r}_\alpha^q dV,
    \label{eq:qthmoment}
\end{equation}
this expression will find its use in the next sections, for conciseness, we will keep the notation, $\bm{P}_\alpha^1 =\bm{P}_\alpha$ and $\mathcal{G}^2_\alpha = \mathcal{G}_\alpha$. 
Now that the mains quantities are properly defined, we need to take their time derivative to express the force balance and moment balance equations. 
Here is the general Reynolds transport equation for any quantity $f$, 
\begin{equation}
    \ddt \int_{V_\alpha} f dV 
    = \int_{V_\alpha} \frac{\partial f}{\partial t}dV 
    + \int_{S_\alpha} f \bm{u_I}\cdot \bm{n}_\alpha d S_f,
\end{equation}
where $\bm{u}_I$ is the velocity of the interface, $S_\alpha$ the surface of the particle $\alpha$, and $\bm{n}_\alpha$ the unit normal vector on $S_\alpha$. 
By adding and subtracting by $\int_{S_\alpha} f \bm{u}\cdot \bm{n}_\alpha dS$ This integral can be express as,
\begin{align}
    \label{eq:timetransport}
    \ddt \int_{V_\alpha} f dV &
    = \int_{V_\alpha}\left[ \frac{\partial f}{\partial t} + \hat{\bm{\nabla}}\cdot\left(f\bm{u}\right) \right]dV + \int_{S_\alpha} f (\bm{u_I}-\bm{u})\cdot \bm{n}_\alpha d S,\\
    &= \int_{V_\alpha} \frac{\hat{D} f}{\hat{D} t} dV + \int_{S_\alpha} f (\bm{u_I}-\bm{u})\cdot \bm{n}_\alpha d S,\\
\end{align}
where we clearly distinguish the local material derivative and the phase transfer terms.
If we inject respectively, $\rho_d$ and $\rho_d \bm{u}$ for $f$ in \ref{eq:timetransport} and make use of \ref{eq:CmomentumOnefluide} (noticing that the integral of the surface force cancel out), we obtain, 
\begin{equation}
    \label{eq:masscons}
    \frac{d m_\alpha}{dt} 
    = \underbrace{\int_{S_\alpha} T_\alpha dS}_{\text{Mass transfer}},\\
\end{equation}    
\begin{equation}
    \frac{d\bm{p}_\alpha}{dt} 
    = \int_{V_\alpha} \bm{u} \rho_d dV 
    = \underbrace{\int_{V_\alpha} \bm{b} dV}_{\text{Body forces}} 
    + \underbrace{\int_{S_\alpha} \bm{\sigma}_f \cdot \bm{n} dS}_{\text{External forces}}
    + \underbrace{\int_{S_\alpha} \bm{u} T_\alpha dS}_{\text{Momentum mass transfer}}
\end{equation}
where $T_\alpha = \rho_d\left(\bm{u_I}-\bm{u}\right)\cdot\bm{n}_\alpha$ is the mass transfer term and $\bm{\sigma}_f$ is the stress in the fluid phase.
This term appears naturally by making use of the surface tension force and the jump condition.
We note $\bm{b''} = \bm{b} - \bm{b}^{\text{ext}}$ the fluctuation of the body force around a uniform field of body forces, such as gravity. 
The uniform field is noted $\bm{b}^{\text{ext}}$.
$\bm{f}^h_\alpha$ is the external forces due to hydrodynamic interactions.
The fluctuation of the body forces can be due to inter-particular forces or contact forces for example, and we will write  $\bm{f}^b_\alpha = \int_{V_\alpha} \bm{b''} dV$.
Therefore, we note $\bm{f}_\alpha = \bm{f}^h_\alpha + \bm{f}^b_\alpha$ for the total force acting on the body.
Then the Lagrangian momentum balance for a fluid particle becomes, 
\tb{carry out this for an arbitrary governing equaitons}
\begin{equation}
    \frac{d m_\alpha \bm{u}_{\alpha}}{dt} 
    = V_\alpha\bm{b}^{\text{ext}} 
    + \bm{f}_\alpha
    + \int_{S_\alpha} \bm{u} T_\alpha dS
    - \ddt   \int_{V_\alpha} \rho_d \bm{w}_\alpha dV
\end{equation}
We recognize the force balance \ref{eq:Newtion2law} to which we add terms due to mass transfer and rate of fluctuations. 

\tb{For the energy of surface $\mathcal{E}^\sigma_\alpha$ we need a different treatment as it is defined on a surface and not a volume. 
Indeed, by making use of Leibniz integral rule and Gauss theorem for surface, we have, 
\begin{equation}
    \ddt  \mathcal{E}^\sigma_\alpha 
    = \int_{S_\alpha} \frac{\partial }{\partial t} \sigma
    + \bm{\nabla}_s \cdot (\sigma \bm{u}) dS
\end{equation}
Using integration by part on a closed surface and considering the boundary condition \ref{eq:stressjump} yeilds, 
\begin{equation}
    \ddt  \mathcal{E}^\sigma_\alpha 
    = \int_{S_\alpha} \sigma \kappa \bm{u} \cdot \bm{n}dS
\end{equation}}
\tb{It can be shown that the surface transport equation witht the particular approch is the same that continuous approach \citet{lhuill}}

If we neglect mass transfer, 
then, the Lagrangian mass conservation, force balance and moment of momentum balance equations turn into,
\begin{equation}
    \label{eq:massdef}
    \frac{d V_\alpha}{dt} 
    = 0,
\end{equation}
\begin{equation}
    \label{eq:momentumdef}
    \frac{d \bm{p}_\alpha}{dt} 
    = \rho_d \ddt  \left(V_\alpha \bm{u_\alpha} 
    + \int_{V_\alpha} \bm{w}_\alpha dV\right)
    = V_\alpha\bm{b}_{\text{ext}} 
    + \bm{f}_\alpha,
\end{equation}
\tb{taylor expansion of p makes appear the rotaion into this eq. (see my notes)}
\begin{equation}
    \label{eq:momentMumdef}
    \frac{d\bm{P}_\alpha}{dt} 
    = \bm{M}_\alpha^{h}
    + \bm{M}_\alpha^{b}
    + \bm{M}_\alpha^\sigma
    - \int_{V_\alpha} \bm{\sigma} dV
    + \rho_d\int_{V_\alpha} \bm{u}\bm{w}_\alpha dV
\end{equation}




One last equation useful for the understanding, is the time derivative of the second order shape tensor $\mathcal{G}^2_\alpha$ derived and generalized to an arbitrary order in \ref{ap:cinematic}. 
Neglecting the mass transfer the equation reads as,
\begin{equation}
    \frac{d \mathcal{G}_\alpha}{dt} 
    = 2 \bm{P}_\alpha^{\text{Sym}},
\end{equation}
Physically this expression means that the rate of change of the shape of the particle equal the symmetric part of the moment of momentum. 
\tb{The antisymmetric part, or the rate of rotation ins't involved as the derivation is relative to the orientation of the particle. }



where $\bm{M}^{b}_\alpha = \int_{V_\alpha}\bm{r}_\alpha \bm{b''}dV$ is the first moment due external forces,
$\bm{M}^{\sigma}_\alpha = \int_{S_\alpha}\bm{r}_\alpha \bm{f}^\sigma dS$ is the first moment due tension surface forces (see in \tb{appendix for derivation of surface average}),
 and $\bm{M}^{h}_\alpha = \int_{S_\alpha}\bm{r}_\alpha \bm{\sigma} \cdot \bm{n} dS$ is the first moment due to the hydrodynamical interaction with the fluid phase. 
Remark that those two terms can be gathered together using the jump condition at interface, yielding
$\bm{M}^{h+\sigma}_\alpha = \int_{S_\alpha} \bm{r}_\alpha (\bm{f}_\sigma + \bm{\sigma}) \cdot \bm{n} dS= \int_{S_\alpha} \bm{r}_\alpha \bm{\sigma}_f \cdot \bm{n} dS$ where we can notice the subscript $_f$ meaning the stress on the fluid phase.   


\tb{\citet[sec 1.1.1]{zaepffel2011modelisation} Also the steslet allows us to evaluate the int of the stress over the surf since sigma is solely symmetric.}
\tb{Make the general case of averaging  a quantity $\psi$ just like maxwell equaitons
Then go into details}
\subsection{Poly disperse averaged equation for deformable particles.}
The volume average can be applied to any Eulerian quantities that are defined 
(by definition) continuously at all point in the domain.
In the cases where the quantity $f$ is defined only at a discrete set of points 
$\textbf{y}_i$, we need to define another type of average. 
Such discrete quantities can arise while looking at Lagrangian objects.
Indeed, the position of the center of a particle or the velocity defined solely at its 
center is a discrete quantity. 
This average is called the particular average, and its operator yields, 
\begin{equation}
    \label{eq:partia}
    n(\textbf{x},t)\left<f\right>^p(\textbf{x},t) 
    = \sum_{\alpha} g(\textbf{x},\bm{y_\alpha}) f_k(\bm{y_\alpha},t),
\end{equation}
where, $n$ is the number density of the elements indexed by $\alpha$, it is defined such that, 
\begin{equation}
    n(\textbf{x},t) 
    = \sum_{\alpha} g(\textbf{x},\bm{y_\alpha}).
\end{equation}
\tb{The detailed derivation must be done in Appendix.
In this part draw the main conclusion of the expansion notation without any hypothesis}

\tb{present in the first place the transport of an arbitrary quantity $\Phi$ like in the thesis of \cite{paisant2014modelisation}}
In this section we carry out the particle-phase average of \ref{eq:momentumdef}, \ref{eq:massdef} and \ref{eq:momentMumdef}.
In the first place we consider the general situation.
Then, we dive into specific cases where the drops are linearly deformable or solid spheres.
At last, we consider a poly disperse suspension of non-deformable particles.


Let's first average the Lagrangian mass balance \ref{eq:massdef} with the particular average operator.
It yields,   
\begin{equation}
    \label{eq:LpolyMassCons}
    \frac{\partial }{\partial t}(n\left<V_\alpha\right>^p) 
    + \bm{\nabla}\cdot(n\left<\bm{u_\alpha}V_\alpha\right>^p)
    = 0,
\end{equation}  
where we have used the following property of \citep{anderson1967fluid},
\begin{equation*}
    n \left<\frac{d f}{dt}\right>^p 
    = \frac{\partial }{\partial t}(n\left<f\right>^p) 
    + \bm{\nabla}\cdot(n\left<\bm{u}_\alpha f\right>^p),
\end{equation*}
valid for any physical property $f$.
The density of the dispersed phase is constant,
Besides, the mean of the product $\left<\bm{u_\alpha}V_\alpha\right>^p = \left<\bm{u_\alpha}\right>^p\left<V_\alpha\right>^p+\left<\bm{u'_\alpha}V'_\alpha\right>^p$, where the second term on the right-hand side is the fluctuations of $V_\alpha$ and $\bm{u}_\alpha$ around the mean values, $\left<V_\alpha\right>^p$ and $\left<\bm{u}_\alpha\right>^p$. 
Therefore, the equation reduce to,
\begin{equation}
    \frac{\partial }{\partial t}(n\left<V_\alpha\right>^p) 
    + \bm{\nabla}\cdot(n\left<V_\alpha\right>^p\left<\bm{u_\alpha}\right>^p )
    = 
    - \bm{\nabla}\cdot(n\left<\bm{u'_\alpha}V'_\alpha\right>^p),
    \label{eq:pmassavg}
\end{equation}  
where, $V_\alpha'$ is the fluctuations around the mean value of $V_\alpha$,
namely, $V_\alpha' = V_\alpha - \left<V_\alpha\right>$.
One can notice that the product $nV_\alpha \approx \phi$, where $\phi$ is the fraction of the dispersed phase introduced in the previous section. 
This relation seems trivial, but it is in fact not quite true, that why we use the approximation sign. 
In fact, it is possible to determine the exact relation between $\phi$ and $nV_\alpha$ by making the link between particular and dispersed phase average.

So, in the following, we will demonstrate how to derive the particular phase averaged equation from \ref{eq:massdef}, instead of using the Lagrangian mass conservation. 
First, notice that in the previous section we used similar calculation to express the moment of superior order, $\bm{M}^q$, in the momentum fluid phase averaged equation. 
So, let's define the relation between the dispersed phase average operator and the particular phase average operator, for any physical quantity  $f$ defined in the dispersed phase.
The dispersed phase average of $f$ is defined as $\left<f\right>^d = \int g\chi f dV$, or as a sum of average namely, $\left<f\right>^d = \sum_\alpha \int_{V_\alpha}g f dV$ which is strictly equivalent.
Then, from the expansion of the product $gf$ at the center of mass $\bm{y}_\alpha$ (similarly to \ref{eq:expansion}), it is possible to write, 
\begin{align}
    \int g \chi f  dV 
    &= \sum_\alpha \int_{V_\alpha} g f  dV \\
    &= \sum_\alpha g_\alpha f_\alpha \int   dV 
    + \sum_\alpha  \hat{\bm{\nabla}} \cdot (g f)_\alpha \int \bm{r}_\alpha dV 
    + \sum_\alpha \frac{1}{2}\hat{\bm{\nabla}}\hat{\bm{\nabla}} : (g f)_\alpha \int \bm{r}_\alpha\bm{r}_\alpha dV 
    + \ldots. \\
    &= \sum_\alpha g_\alpha f_\alpha V_\alpha 
    + \sum_\alpha  \hat{\bm{\nabla}} \cdot (g f)_\alpha \mathcal{G}_\alpha^1 
    + \sum_\alpha \frac{1}{2}\hat{\bm{\nabla}}\hat{\bm{\nabla}} : (g f)_\alpha \mathcal{G}_\alpha^2 
    + \ldots. \\
    \label{eq:exp}
    &= \sum_{l=0}^\infty \sum_\alpha \left[\frac{1}{l!} \hat{\bm{\nabla}}^l gf|_\alpha \mathcal{G}_\alpha^l\right],
\end{align}
where we have used the $l^{th}$ order shape tensor, $\mathcal{G}_\alpha^l$, such that, $(\mathcal{G}_\alpha^l)_{i_1 i_2\ldots i_l} = \int r^\alpha_{i_1}r^\alpha_{i_2}\ldots r^\alpha_{i_l}dV$.
Additionally, we used the notation, $f_\alpha$, for the value of $f$ evaluated at $\bm{y}_\alpha$ and $\hat{\bm{\nabla}}^q gf|_\alpha$ is the $q^{th}$ derivative evaluated at $\bm{y}_\alpha$.
The operator between, $\hat{\bm{\nabla}}^l gf|_\alpha$ and $\mathcal{G}_\alpha^l$ is the $l^{th}$ order contracted product (e.g. the $0^{th}$ order contracted product is $\cdot$, the double contracted product is $:$, the third, $\vdots$ and so on\ldots). 
In indices notation, $\hat{\bm{\nabla}}^l gf|_\alpha \mathcal{G}_\alpha^l  = (\mathcal{G}_\alpha^l)_{i_1 i_2\ldots i_l} \hat{\partial}_{i_1} \hat{\partial}_{i_2}\ldots \hat{\partial}_{i_l} gf|_\alpha$.
An interesting fact is that for a spatially homogeneous system only the first term of the expansion is non-null. 
Thus, the particular average (represented by the first term) is rigorously equivalent to the dispersed-phase average in homogeneous flows. 
Now, by using the general Leibniz rule on the product, $\hat{\bm{\nabla}}^l gf|_\alpha$ (development in \ref{ap:cinematic}), we can deduce the following relation,
\begin{equation}
    \int g \chi f  dV 
    = \sum_{l=0}^\infty \sum_\alpha \left[\frac{1}{l!} \sum_{q = 0}^l \binom{l}{q} \hat{\bm{\nabla}}^{l-q}f|_\alpha \hat{\bm{\nabla}}^q g|_\alpha \mathcal{G}_\alpha^l\right].
\end{equation}
Next, using the property, $\hat{\bm{\nabla}}^{q} g|_\alpha = (-1)^q \bm{\nabla}^{q} g_\alpha$, interchanging the order of the sum and taking the particular average, yields the following relation,
\begin{equation}
    \int g \chi f  dV 
    = \sum_{l=0}^\infty \sum_{q = 0}^l \frac{(-1)^q}{q!(l-q)!}  \bm{\nabla}^q  \sum_\alpha \left(\hat{\bm{\nabla}}^{l-q} f|_\alpha \mathcal{G}_\alpha^l g_\alpha\right)
    = \sum_{l=0}^\infty \sum_{q = 0}^l \frac{(-1)^q}{q!(l-q)!}  \bm{\nabla}^q \left( n \left<\hat{\bm{\nabla}}^{l-q} f|_\alpha \mathcal{G}_\alpha^l\right>^p\right).
    \label{eq:bexp},
\end{equation}
When taking $f = 1$ we clearly recognize the dispersed phase volume fraction on the left hands side of \ref{eq:bexp}.
Thus, after simplification we have,
\begin{equation}
    \phi
    = \sum_{l=0}^\infty \frac{(-1)^l}{l!}  \bm{\nabla}^l \left( n \left<\mathcal{G}_\alpha^l\right>^p\right),
\end{equation}
where we have used the property : $\hat{\bm{\nabla}}^{l-q} (1)= 0 \;\; \text{if}\;\; q \neq l$. 
Notice that the second order term cancel since $\int_{V_\alpha} \bm{r}_\alpha dV= \mathcal{G}_\alpha^1 = 0$ (see \ref{ap:cinematic}), so the error while taking $\phi = n \left<V_\alpha\right>^p$ is only $\mathcal{O}\left(L\right)^2$.
Then, from \ref{eq:bexp} we can also express the dispersed phase velocity as an expansion around the center of particles. 
Taking $f = \bm{u}$,
\begin{equation}
    \phi\left<\bm{u}\right>^d 
    = \sum_{l=0}^\infty \sum_{q = 0}^l \frac{(-1)^q}{q!(l-q)!}  \bm{\nabla}^q  n \left<\hat{\bm{\nabla}}^{l-q} \bm{u}|_\alpha \mathcal{G}_\alpha^l\right>^p.
\end{equation}
By carrying only the expansion along $g$, from \ref{eq:expansion}, we can arrive to this compelling relation,
\begin{equation}
    \phi\left<\bm{u}\right>^d
    =\left(n \left<\bm{p}_\alpha\right>^p\right)
    - \bm{\nabla} \cdot\left(n \left<\bm{P}_\alpha\right>^p\right)
    + \frac{1}{2}\bm{\nabla\nabla} : \left(n \left<\bm{P}_\alpha^2 \right>^p\right)
    + \ldots
    = \sum_{q=0}^\infty \left[\frac{(-1)^q}{q!} \bm{\nabla}^q \left(n \left<\bm{P}_\alpha^q\right>^p\right)  \right].
\end{equation}
As a matter of fact, the expansion of the dispersed phase average of the velocity involve the first moments of momentum tensor and higher moment of momentum.
Therefore, \ref{eq:Cmassad} include the moments of momentum in its formulation.  
Following this thinking we could also show that the particle averaged angular momentum balance (\ref{eq:Iavg}) is in fact included inside the dispersed phase average \ref{eq:davgmomentum} along with the \ref{eq:pavgsp}. 
The $0^{th}$ order being the particular average \ref{eq:pavgsp}. 
Back to our derivation,
if we inject the preceding expansion in the mass balance of the dispersed phase (\ref{eq:Cmassad}) we get this expression,
\begin{equation}
    \frac{\partial }{\partial t}\left[
        \sum_{l=0}^\infty \frac{(-1)^l}{l!}  \bm{\nabla}^l  \left(n \left<\mathcal{G}_\alpha^l\right>^p\right)
    \right]
    + \bm{\nabla}\cdot\left[
        \sum_{l=0}^\infty \sum_{q = 0}^l \frac{(-1)^q}{q!(l-q)!}  \bm{\nabla}^q  \left(n \left<\hat{\bm{\nabla}}^{l-q} \bm{u}|_\alpha \mathcal{G}_\alpha^l\right>^p\right)
    \right]
    = 0,
\end{equation}  
Now, considering Einstein summation notation convention, and permuting the operators gives,
\begin{equation}
    \sum^\infty_{l} \left[
        \partial_t
        \frac{(-1)^l}{l!}  
        \prod_{m=1}^l 
        \partial_{i_m}
        \left(
            n \left<\mathcal{G}^\alpha_{i_1i_2\ldots i_l}\right>^p
            \right) 
            +
            \sum_{q=0}^l
            \frac{(-1)^q}{q!(l-q)!}  
            \prod_{m=l-q+1}^{l} 
            \partial_{i_m}
            \partial_k 
            n \left<\prod_{m=1}^{l-q}\hat{\partial}_{i_m} u_k|_\alpha \mathcal{G}^\alpha_{i_1i_2\ldots i_l}\right>^p
    \right]
    = 0,
    \label{eq:pmavgl}
\end{equation}
This equation is rigorously equivalent to the dispersed phase average derived earlier. 
Moreover, under this form, it is obvious that it is a scalar equation (just like the transport equation of the dispersed phase volume fraction). 
Therefore, to be solved this equation needs another set of $l$ equations to account for all the yet unknown shape tensors. 
Those, equations are in fact the transport equation of each shape tensors (derived in \ref{ap:cinematic}). 
Which makes in total $l+1$ equations to solve this problem. 
If we consider only the $0^{th}$ order terms, it yields the particular average of \ref{eq:massdef}.
Which means that \ref{eq:Cmassad} include all the higher order terms of \ref{eq:pmavgl}. 
Hence, by considering only the $0^{th}$ order equation (which is always the case in practice), we neglect the internal gradient of the velocity and the shape tensors of the particle. 
Notice that the first order terms are all null since $\mathcal{G}^1_\alpha = 0$.
Therefore, the overall error by taking the particular mass average instead of the dispersed one, is always of $\mathcal{O}\left(L\right)^2$.
In practice, we never consider the second or higher order terms, but it is interesting to derive it for the understanding of the subject.
If we keep only the second order terms in the mass balance, it yields,
\begin{equation}
    \ldots +
    \frac{1}{2}
    \frac{\partial }{\partial t}\left(
          \bm{\nabla}^2  n \left<\mathcal{G}_\alpha^2\right>^p
    \right) 
    + \bm{\nabla}\cdot\left[
        + \frac{1}{2}  n \left<\mathcal{G}_\alpha^2 : \hat{\bm{\nabla}}^{2} \bm{u}|_\alpha \right>^p
        - \bm{\nabla} \cdot \left(n \left<\mathcal{G}_\alpha^2 \cdot \hat{\bm{\nabla}} \bm{u}|_\alpha \right>^p\right)
        + \frac{1}{2} \bm{\nabla}^{2} : \left(n \left<\mathcal{G}_\alpha^2 \bm{u}_\alpha  \right>^p\right)
    \right]
    =0,
\end{equation}
The second order cinematic term (i.e. $\hat{\bm{\nabla}}^{2} \bm{u}|_\alpha$) is the physical representation of the late or advance internal motions inside a particle around its mean velocity $\bm{u}_\alpha$.  
This is due to the particle internal inertia, thus, we can suppose that this term vanishes when $\mu_d$ is high enough.
The second order term $\mathcal{G}_\alpha^2$ is called the shape tensor, a lengthy explanation is made in \ref{ap:cinematic}. 
It describes the spreading of the geometry in each direction. 
It is in fact the general form of the inertial tensor $\mathcal{I}_\alpha$.
Likewise, it can be shown that the norm of $\mathcal{G}_\alpha^2$ is the Gyration radius or turning radius. 
Which mean that this equation take in account the bulk of the individual particles in the mass balance.
While the first order consider only the center of mass as a part of the balance. 
Notice that for each order, closures are needed.
Indeed, $\left<V_\alpha'\bm{u}_\alpha'\right>^p$ is for the $0^th$ order equation, then the higher order one can be expressed by, 
$\left<(\hat{\bm{\nabla}}^{l-q} \bm{u}|_\alpha)' (\mathcal{G}_\alpha^l)'\right>^p$, where the $'$ represent the fluctuation around the mean value. 
Now, we deal with one last remark on this equation, the special case of solid particles. 
The velocity field of a solid particle is linear (see \ref{ap:cinematic}). 
Consequently, the velocity terms under the divergence operator are all null for all $l - q > 1$. 
Therefore, we can write the second order terms as, 
\begin{equation}
    \ldots +    
    \frac{1}{2}  \bm{\nabla}^2
    \frac{\partial }{\partial t}\left(
          n \left<\mathcal{G}_\alpha^2\right>^p
    \right) 
    + \bm{\nabla}\cdot\left[
        \frac{1}{2} \bm{\nabla}^{2} : (n \left< \bm{u}_\alpha \mathcal{G}_\alpha^2\right>^p)
        - \bm{\nabla} \cdot \left(  n \left<\hat{\bm{\nabla}} \bm{u}|_\alpha \cdot \mathcal{G}_\alpha^2\right>^p\right)
    \right]
    = 0,
    \label{eq:mass}
\end{equation}
Where the gradient of the velocity turn out to be the angular velocity since we consider a rigid particle.
In this case we can write the gradient of the velocity as its antisymmetric component only, $(\hat{\bm{\nabla}} \bm{u})_{ij} = (\bm{\Omega})_{ij} = \frac{1}{2} \left[\hat{\partial}_j u_i -\hat{\partial}_i u_j \right]$\citep{guazzelli2011},
where, $\bm{\Omega}$ is the rotation tensor defined by, $\bm{\epsilon} : \bm{\Omega} = \bm{\omega}$, with, $\bm{\omega}$ the angular velocity of the particle.
One can notice that $\Omega_{ij} = 0 \; \forall \; i = j$.  
If we consider the special case of spherical particles the rotation should not change the mass transfer.
Indeed, as spherical particles are completely symmetric under rotation the mass balance should stay unchanged. 
Therefore, the terms involving the gradient of the velocity in \ref{eq:mass} should cancel out. 
Indeed, for any tensor $\mathcal{G}_\alpha^2$ that follow, $(\mathcal{G}_\alpha)_{11} = (\mathcal{G}_\alpha)_{22} = (\mathcal{G}_\alpha)_{33} =(\mathcal{G}_\alpha)_{33} =G_\alpha \;\;\;\forall G_\alpha \in \mathbb{R}^+$ and $(\mathcal{G}_\alpha)_{ij} = \delta_{ij} G_\alpha$, (i.e. for spherical particles, see \ref{ap:cinematic}),
we have, 
\begin{equation*}
    (\hat{\bm{\nabla}} \bm{u}|_\alpha \cdot \mathcal{G}_\alpha^2)_{ij} 
    = \hat{\partial}_k u_i|_\alpha (\mathcal{G}_\alpha^2)_{kj}
    = (\hat{\partial}_k u_i -\hat{\partial}_i u_k)  \delta_{kj} G_\alpha
    = 0. 
\end{equation*}
Using the two previous remark on spherical particles, yields the following form for the second order terms in the mass balance,
\begin{equation}
    \ldots +
    \frac{1}{2}  \bm{\nabla}^2
    \frac{\partial }{\partial t}\left(
          n \left<\bm{I}G_\alpha^2\right>^p
    \right) 
    + \bm{\nabla}\cdot\left[
        \frac{1}{2} \bm{\nabla}^{2} : (n \left< \bm{u}_\alpha \bm{I}G_\alpha^2\right>^p)
    \right] = 0.
\end{equation}
\begin{equation}
    \ldots +
    \frac{1}{2}  \partial_i\partial_i
    \left[
        \partial_t\left(
            n \left<G_\alpha^2\right>^p
            \right)
        + \partial_k
        \left(
            n \left< (u_\alpha)_k G_\alpha^2\right>^p
        \right)
    \right]
    = 0.
\end{equation}
Moreover, we can notice that for mono-shaped spherical particles, the scalar $G_\alpha^2 = k V_\alpha$, where $k$ is a constant linked to the shape of a sphere (see \ref{ap:cinematic}).
Thus, the constant $k$ can be factorized. 
Now let's write the mass conservation keeping the $0^{th}$ order terms and the $2^{th}$ order terms, 
\begin{equation}
    \partial_t\left(
        n \left<V_\alpha\right>^p
        \right)
    + \partial_k
    \left(
        n \left< (u_\alpha)_k V_\alpha\right>^p
    \right)
    +
    \frac{k}{2}  \partial_i\partial_i
    \left[
        \partial_t\left(
            n \left<V_\alpha\right>^p
            \right)
        + \partial_k
        \left(
            n \left< (u_\alpha)_k V_\alpha\right>^p
        \right)
    \right]
    = 0.
\end{equation}
In tensor form it reads, 
\begin{equation}
    \pddt\left(
        n \left<V_\alpha\right>^p
        \right)
    + \bm{\nabla}\cdot
    \left(
        n \left< \bm{u}_\alpha V_\alpha\right>^p
    \right)
    +
    \frac{k}{2}  \bm{\Delta}
    \left[
        \pddt\left(
            n \left<V_\alpha\right>^p
            \right)
        + \bm{\nabla}\cdot
        \left(
            n \left< \bm{u}_\alpha V_\alpha\right>^p
        \right)
    \right]
    = 0.
\end{equation}
We recognize under the bracket the mass conservation equation at the zeroth order, but under a Laplacian operator. 
Higher order terms would lead to the same form, for spherical particles.
\tb{Show that in appendix .}
The main conclusion of this study is that higher order equations take in account the specific geometry of the particles. 
Indeed, the second order terms include the gyration radius in the mass balance (which is equivalent to the bulk for spheres since they have the same shape tensor under all referential).
While it is indeed negligible for spheres it might not be for very elongated particles like fibers or cylinders as there is a clear difference between their center of mass and their bulk.
Also, we see that terms related to the angular velocities were involved in the mass balance, and those terms canceled out for isotope particles.
Again, for elongated particles the rotation will have an impact on the mass balance.  
Consequently, for elongated particles we need to solve one additional equation for the shape tensor, and one other equation for the angular momentum, since the kinematic terms are involve in the mass conservation, and as we will see in the momentum conservation.  
Comments on the $3^{th}$ order equation can be made too. 
In \ref{ap:cinematic} we showed that $\mathcal{G}^3$ is related to the asymmetry of the particles.
Therefore, it is in general always null, except for droplets and other asymmetric particles. 
Moreover, For droplets particles the higher order terms in the velocity gradient are not zero, therefor further investigation need to be done to understand the relevant terms of this equation for droplets.
Although we showed that some terms of the mass balance were non-zero or zero for certain geometry  it must be said that they are all second order or above.
That is why in practice we always consider only the zeroth order terms in the mass balance. 
In fact, we carried out this expansion to introduce and clarify the expansion of the momentum balance equation that will be done in a future work. 
\tb{In the momentum dispersed average e should find after the expansion terms of 1er order which are coupled with the terms of the moment equation}

Now, we average the momentum conservation by applying \ref{eq:partia} to equation \ref{eq:momentumdef}. 
It yields,
\begin{equation}
    \frac{\partial }{\partial t}(n\left<\bm{p}_\alpha\right>^p) 
    + \bm{\nabla}\cdot(n\left<\bm{u_\alpha}\bm{p}_\alpha\right>^p)
    = n \left<V_\alpha\bm{b_{ext}}\right>^p 
    + n\left<\bm{f_\alpha}\right>^p,
    \label{eq:palphaavg}
\end{equation}  
Now we can develop each terms of equation \ref{eq:palphaavg} using \ref{eq:momentumdef}. 
After a bit of algebra we can show that,
\begin{equation*}
    \frac{1}{\rho_d}\left<\bm{p}_\alpha\right>^p 
    = \left<V_\alpha \bm{u}_\alpha\right>^p
    + \left<\int_{V_\alpha} \bm{w_\alpha}dV\right>^p 
    = \left<V_\alpha\right>^p \left<\bm{u}_\alpha\right>^p
    + \left<V_\alpha' \bm{u}_\alpha'\right>^p
    + \left<\int_{V_\alpha} \bm{w_\alpha}dV\right>^p,
\end{equation*}
\begin{multline*}
    \frac{1}{\rho_d}\left<\bm{u}_\alpha\bm{p}_\alpha\right>^p 
    = \left<V_\alpha \bm{u}_\alpha\bm{u}_\alpha\right>^p
    + \left<\bm{u_\alpha}\int_{V_\alpha} \bm{w_\alpha}dV\right>^p 
    = \left<V_\alpha\right>^p \left<\bm{u}_\alpha\right>^p \left<\bm{u}_\alpha\right>^p
    + \left<V_\alpha' \bm{u}_\alpha'\bm{u}_\alpha'\right>^p\\
    + \left<V_\alpha\right>^p \left<\bm{u}_\alpha'\bm{u}_\alpha'\right>^p
    + 2 \left<\bm{u}_\alpha\right>^p \left<V_\alpha'\bm{u}_\alpha'\right>^p
    + \left<\bm{u_\alpha}\right>^p\left<\int_{V_\alpha} \bm{w_\alpha}dV\right>^p
    + \left<\bm{u_\alpha}'\left(\int_{V_\alpha} \bm{w_\alpha}dV\right)'\right>^p,
\end{multline*}
\begin{equation*}
    n \left<V_\alpha\bm{b_{ext}}\right>^p 
    = n \left<V_\alpha\right>^p\bm{b_{ext}}
\end{equation*}
Beware that, $ \bm{u}_\alpha'' \neq \bm{u}_\alpha'$ since $\bm{u}_\alpha''$ is the fluctuation of a property inside a given particle $\alpha$ and $\bm{u}_\alpha'$ is the variation of the means around the entire set of particles $\alpha$. 
Then, if we simplify and inject the above terms in \ref{eq:palphaavg} it reads,
\tb{$\phi$ in the following expressoin isn't rigourously true.}
\begin{equation}
    \rho_d 
    \frac{\partial }{\partial t}
    \left[
        n\left<V_\alpha\right> \left<\bm{u}_\alpha\right>^p
    \right] 
    + \rho_d\bm{\nabla}\cdot
    \left[
        n\left<V_\alpha\right> \left<\bm{u}_\alpha\right>^p \left<\bm{u}_\alpha\right>^p
    + \left<\bm{u_\alpha}\right>^p \bm{F_1}
    \right]
    = n\left<V_\alpha\right>\bm{b_{ext}} 
    + n\left<\bm{f_\alpha}\right>^p
    - \frac{\partial }{\partial t}\bm{F_2},
    - \bm{\nabla}\cdot\bm{F_3},
    \label{eq:particlesAVG}
\end{equation} 
where,
\begin{equation*}
    \bm{F_1}
    = 2\left<V_\alpha'\bm{u}_\alpha'\right>^p
    +  \left<\int_{V_\alpha} \bm{w_\alpha}dV\right>^p,
\end{equation*} 
\begin{equation*}
    \bm{F_2}/\rho_d
    = \left<V_\alpha' u_\alpha'\right> 
    +\left<\int_{V_\alpha} \bm{w_\alpha}dV\right>^p,
\end{equation*}
and
\begin{equation*}
    \bm{F_3}/\rho_d
    = \left<V_\alpha' \bm{u}_\alpha'\bm{u}_\alpha'\right>^p
    + \left<V_\alpha\right>^p \left<\bm{u}_\alpha'\bm{u}_\alpha'\right>^p
    +\left<\bm{u_\alpha}'\left(\int_{V_\alpha} \bm{w_\alpha}dV\right)'\right>^p.
\end{equation*}
\tb{must carry out the same decomposition for moment of momentum}
In the above expression we have gathered all  the terms function of $\bm{u_\alpha}$ to the left side and the others to the right side. 
The $\bm{F_i}$ terms are all the closures terms that need to be provided in order to solve \ref{eq:pavgsp}. 
As for the mass balance one could recover  \ref{eq:particlesAVG} from the dispersed phase average \ref{eq:davgmomentum}.
Indeed, \ref{eq:davgmomentum} reduce considerably  by replacing the dispersed-phase average terms by their expansion using \ref{eq:exp} of the particular average. 
Then both sides cancel out, and we get-back to the particular phase average.
Indeed, \citep{nott2011suspension} proved this fact for solid spherical particles. 
The remaining question is, is it still valid for deformable poly-dispersed particles ? 
\tb{Show that it is or not true}
\tb{discus the physical meaning of the different terms and the isolated effect of the poly dispersity.}



Next, we average the \textit{moment of momentum} balance, \ref{eq:momentMumdef}. 
Applying the same process as for the momentum equation it yields, 
\begin{multline}
    \left[
        \frac{\partial }{\partial t}(n\left<\bm{P}_\alpha\right>^p) 
    + \bm{\nabla}\cdot(n\left<\bm{u_\alpha}\bm{P}_\alpha\right>^p)
    \right] 
    = \left<\bm{M}_\alpha^{h}\right>^p
    + \left< \bm{M}_\alpha^{b}\right>^p
    + \left< \bm{M}_\alpha^{\sigma}\right>^p
    - \left< \int_{V_\alpha} \bm{\sigma} dV\right>^p
    + \left< \int_{V_\alpha}\rho_d \bm{u}\bm{w}_\alpha dV\right>^p
    \label{eq:avgmoment}
\end{multline} 
Unlike the preceding equation here the cinematic terms like $\bm{u}_\alpha$ and $\bm{\omega}_\alpha$ aren't explicitly shown. 
We recall that the antisymmetric part of $\bm{P}_\alpha$, gives $\mathcal{I}_\alpha \bm{\omega}_\alpha$.
Therefore, there must be a way to decompose $\bm{P}_\alpha$ into a product of cinematic tensor times a shape property tensor. 
In \citet{willen2019resolved} and \citet{Pumir2013} they make use of such decomposition. 
The shape property tensor, is defined as $\mathcal{G}_\alpha = \int_{V_\alpha} \bm{\bm{r}_\alpha}\bm{\bm{r}_\alpha} dV$, it is the shape tensor defined in the previous sections.
It is widely used in polymer science to describe the shape of particles. 
It is the general form of the momentum tensor since $\mathcal{I}_\alpha = \text{Tr}(\mathcal{G}_\alpha)-\mathcal{G}_\alpha$. 
While, they use the gradient of the velocity at the center of the set to describe the cinematic part. 
Which makes a lot of sens considering that the gradient represent the rate of strain and rotation. 
Nevertheless, this decomposition is valid only if we consider the velocity field inside $V_\alpha$ as linear, namely if,
\begin{equation}
    \bm{u}(\bm{y}) 
    = 
    \bm{u}_\alpha 
    + \bm{\nabla u}|_\alpha \bm{r}_\alpha
    = 
    \bm{u}_\alpha 
    + \bm{\omega}_\alpha \times \bm{r}_\alpha
    + \bm{E}_\alpha \cdot \bm{r}_\alpha
    \label{eq:lindep}
\end{equation} 
where we recognize the rate of strain $\bm{E}_\alpha$ and angular velocity $\bm{\omega}_\alpha$.
We also introduce the rotation tensor defined as,  $\bm{\Omega}_\alpha \cdot \bm{y} = \bm{\omega}_\alpha \times \bm{y}$.
In this particular case, we can indeed show that $\bm{P}_\alpha = \bm{\nabla u}|_\alpha \cdot \mathcal{G}_\alpha$ (see \ref{ap:cinematic}).
In the next section we give the expression of the momentum and moment of momentum conservation considering linear flows.


\subsubsection{Hypothesis of linear deformation}
In this section we rewrite the equations for particles experiencing linear deformation.
Which is the case of all plastique particles in the limit of small motion, consequently it could be valid for highly viscous drops. 
If we consider only linear deformations, then the integral term $\int_{V_\alpha} \bm{w_\alpha}dV = 0$.
And the momentum balance equation become the classic momentum balance for solid particles (see \ref{ap:cinematic}). 
Consequently, the coefficient in \ref{eq:Cmomentum} reduce to, 
\begin{equation*}
    \bm{F_1}
    = 2\left<V_\alpha'\bm{u}_\alpha'\right>^p
\end{equation*} 
\begin{equation*}
    \bm{F_2}/\rho_d
    = 0
\end{equation*}
\begin{equation*}
    \bm{F_3}/\rho_d
    = \left<V_\alpha' \bm{u}_\alpha'\bm{u}_\alpha'\right>^p
    + \left<V_\alpha\right>^p \left<\bm{u}_\alpha'\bm{u}_\alpha'\right>^p
\end{equation*}
Then, this new equation of conservation correspond to the poly-disperse momentum conservation for solid particles. 

Now let's derive the linear approximation of the moment of momentum equation. 
Using, \ref{eq:lindep}, the moment of momentum conservation equation becomes, 
\begin{equation}
    \left[
        \frac{\partial }{\partial t}(n\left<\bm{\nabla u}|_{\alpha} \cdot \mathcal{G}_\alpha\right>^p) 
    + \bm{\nabla}\cdot(n\left<\bm{u_\alpha}\bm{\nabla u}|_{\alpha} \cdot \mathcal{G}_\alpha\right>^p)
    \right] 
    = \left<\bm{M}_\alpha^{h}\right>^p
    + \left< \bm{M}_\alpha^{b}\right>^p
    + \left< \bm{M}_\alpha^{\sigma}\right>^p
    - \left< \int_{V_\alpha} \bm{\sigma} dV\right>^p
    + \left<  (\bm{\nabla u}|_{\alpha})^2 \mathcal{G}_\alpha\right>^p.
    \label{eq:bilandumoment}
\end{equation} 
The antisymmetric part of the tensors, $\bm{w}_\alpha \bm{w}_\alpha = 0$ and $\bm{\sigma} = 0$ (see \ref{ap:cinematic}), besides.
Consequently, the antisymmetric part of the moment of momentum conservation equation yields,
\begin{equation}
        \frac{\partial }{\partial t}(n\left<\bm{\omega_\alpha} \cdot \mathcal{I}_\alpha\right>^p) 
    + \bm{\nabla}\cdot(n\left<\bm{u_\alpha}\bm{\omega_\alpha} \cdot \mathcal{I}_\alpha\right>^p)
    = \left< \bm{\tau}_\alpha^{h}\right>^p
    + \left< \bm{\tau}_\alpha^{b}\right>^p
    + \left< \bm{\tau}_\alpha^{\sigma}\right>^p
\end{equation} 
where we have used the decomposition in \ref{ap:cinematic} for $\bm{\nabla u}|_{\alpha} \cdot \mathcal{G}_\alpha$.
This equation is equivalent to \ref{eq:Iavg}, except that $\bm{\mathcal{I}_\alpha}$ is inside the average.
Next, taking the symmetric part of \ref{eq:bilandumoment} and using \ref{eq:stresslet} we get an equation for the stain rate, 
\begin{equation}
    \frac{\partial }{\partial t}(n\left<\bm{E_\alpha}\cdot \mathcal{G}_\alpha\right>^p) 
    + \bm{\nabla}\cdot(n\left<\bm{u_\alpha}\bm{E_\alpha}\cdot \mathcal{G}_\alpha\right>^p)
    = \left<\bm{S}_\alpha^{h}\right>^p
    + \left< \bm{S}_\alpha^{b}\right>^p
    + \left< \bm{S}_\alpha^{\sigma}\right>^p
    - \left< \int_{V_\alpha} \bm{\sigma} dV\right>^p
    + \left< (\bm{\nabla u}|_{\alpha})^2 \mathcal{G}_\alpha\right>^p.
\end{equation} 
Note that for solid particle suspensions the momentum, angular momentum, and the mass balance equations are exactly the same as the ones presented above for linear deformation. 
Indeed, in the solid particle case, the motion inside the particle is limited to a constant translation velocity and a rotation.  
Therefore, only the stresslet equation becomes obsolete. 

As we could see in this section, at the first order, the most important term is the averaged drag force. 
This averaged drag force is function at the first order of the volume fraction of the dispersed phase $\phi$ and the size and shape of all the droplets $L^\alpha$ and $\mathcal{G}^\alpha$.
Thus, to get accurate closure one has to determine the distribution of diameters inside the emulsion.
We consider the transport of the averaged volume and shape with these averaged equations. 
Nevertheless, it is not yet possible to describe the evolution of the distribution of the size of droplets within time and space.  
Indeed, this distribution can change as the particles coalesce and break, and due to segregation effects.  
The mathematical tools used to predict the probability density function of the size distribution, are the population balance equations.
That is why in the next section we are going to derive the population balance equations.


\section{Microscopic scale governing equations for two-phase flow}
\label{sec:conservation_laws}

In this section we define the local scale governing equations of multiphase flow using the formulation of \citet{kataoka1986local} and \citet{drew1983mathematical}. 
Before diving into the derivation we would like to emphasize that the local position vector at the local scale is noted $\textbf{y}$.
This clarification will have its usage in the next section where we will use another variable for the global location, namely the vector $\textbf{x}$. 
Also, in this manuscript we write in bold symbols the vectors and tensors, while the scalar variables are written with the usual font. 

Any conservation equation for an arbitrary quantity $f_k(\textbf{y})$, where $f_k$ represent the quantity $f$ but defined in the arbitrary phase $k$, will takes the form,
\begin{equation}
    \pddt f_k
    = \grad \cdot \left(
        \bm{\Phi}_k
        - f_k\textbf{u}_k
        \right)
    + \textbf{S}_k
    \label{eq:general_conservation}
\end{equation}
where $\grad\cdot()$ is the local divergence operator defined as, $\frac{\partial}{\partial \textbf{y}}\cdot()$, $\textbf{u}_k$ is the phase velocity defined in the phase $k$. 
$\bm{\Phi}_k$ is the non-conservative flux corresponding to the quantity $f_k$,
The non-conservative flux is often expressed through a constitutive equation depending on the nature of the flow such as the stress tensor for the momentum. 
And $\textbf{S}_k$ is the volumetric source of $f_k$, such as the body forces still for the momentum equation.
It is important to note that \ref{eq:general_conservation} is solely defined in the volume occupied by the phase $k$.
To generalize the conservation equation over the whole domain, i.e. in each phase and also at the interfaces, we will need supplementary equations. 
To carry out the derivation of these equations we first introduce the Phase Indicator Function (PIF),
\begin{equation}
    \chi_k(\textbf{y}) =  \left\{
      \begin{tabular}{cc}
        $1 \;\text{if} \;\textbf{y} \in V_k$\\
        $0 \;\text{if} \;\textbf{y} \notin V_k$
      \end{tabular}
      \right.,
      \label{eq:phase_indicator}
\end{equation}
with $V_k$ the volume occupied by the $k^{th}$ phase.
As an example, $V_c$ is the volume occupied by the continuous phase and $V_d$ by the dispersed phase. 
With the use of this function we will be able to generalize the phase quantities to general fields, and to derive two kinds of conservation equations, namely, the single-fluid formulation and the two-fluid formulation. 

\subsection{Topological equations}

Before introducing the physical balance equations such as the mass and momentum conservation equations, we study the transport of the volume occupied by a phase $k$ and the interface between the phases considered. 
Therefore, the topological balance equations correspond to the transport equations of $\chi_k$, and the transport of its \textit{roots} along the velocity of its interfaces. 
First, the transport equation of $\chi_k$ reads as \citep{drew1983mathematical,kataoka1986local,morel2015mathematical},
\begin{equation}
    \pddt \chi_k
    + \textbf{u}_I \cdot \grad \chi_k 
    = 0,
    \label{eq:phaseindicator_transport}
\end{equation}
where $\textbf{u}_I$ is the velocity of the interface.
It is important here to make the distinction between the velocity \textbf{of the interface} which is different from the velocity of the phase $k$ \textbf{at the interface} location which is noted $\textbf{u}_k$.
Besides, it can be shown \citep{tryggvason2011direct} that, 
\begin{equation}
    \grad \chi_k 
    = - \delta_I \textbf{n}_k   
\end{equation}
where we have introduced the interface indicator function $\delta_I$, defined as $\delta(\textbf{y}-\textbf{y}_I)$, where $\delta$ is the Dirac-delta function and $\textbf{y}_I$ the position vector of the interfaces. 
We also define $\textbf{n}_k$ as the outward normal vector of the phase $k$. 
We apply the Reynolds transport theorem on $\delta(\textbf{y}-\textbf{y}_I)$, which permit us to obtain the surface balance equation \citep{lhuillier2003dynamics,morel2015mathematical}, 
\begin{equation}
    \pddt \delta_I
    + \grad \cdot\left[
        (\textbf{u}_I\cdot\textbf{n})\textbf{n}\delta_I 
    \right]
    = \delta_I (\textbf{u}_I\cdot\textbf{n})
    (\grad_I\cdot\textbf{n}),
    \label{eq:interface_transport}
\end{equation}
where, $\grad_I$ is the interfacial gradient operator defined as $\grad_I\cdot() = (\textbf{I}-\textbf{nn}):\grad()$.
Note that the normal vector \textbf{n} appear twice in each term, the sign has therefore no importance and there isn't any ambiguity regarding the definition of \textbf{n}, so we removed the index $k$.  
The first term on the right-hand side (RHS) of this equation corresponds to the source term due to the deformation of the interface.
Indeed, it is expressed as the normal velocity $\textbf{u}\cdot\textbf{n}$ times the curvature of the interface defined as $\kappa = - \grad \cdot \textbf{n}$.
Consequently, we can stipulate that the interface evolution is explicitly function of the velocity times the curvature at the interface and its own advection along the interfacial velocity. 
We would like to highlight that this source term doesn't explicitly include topological change in the flow. 
Indeed, it is important to understand that coalescence and break-up phenomenons affect the curvature term, which, in turn, have an impact on the evolution of the interface. 
\footnote{Besides, we will see in the last part of this chapter, that this term is somehow equivalent to the Coalescence kernel of the kinetic theory, which acts directly acts on the area transport equation.} 
As pointed out by \citet{morel2007surface}, this equation can be rewritten under the more common form,
\begin{equation}
    \pddt \delta_I
    + \grad \cdot (\delta_I \textbf{u}_I)
    = \delta_I \grad_I \cdot \textbf{u}_I.
    \label{eq:interface_transport2}
\end{equation}
Note that now the velocity involved is simply $\textbf{u}_I$ and not the normal velocity $\textbf{u} \cdot \textbf{n}$ as in \ref{eq:interface_transport}.
Therefore, this equation has the form of a traditional transport equation, which is the reason why it is preferred. 

\subsection{Generalized balance equations}

As stipulated above, \ref{eq:general_conservation} is solely defined on phase $k$.
We thus need to generalize this equation over the whole domain constituted by all phases.  
There are two possible formulations to carry out this task.
In the following, we give a brief overview of both formulations in the case of a general conservation equation such as \ref{eq:general_conservation}.
The first formulation is the so-called \textit{Two-fluid formulation}.
In this formulation we transport the quantity $f_k\chi_k$, so that $f_k\chi_k$ is defined over the whole domain, since $\chi_k f_k = f_k$ where $\textbf{y} \in V_k$ and $\chi_k f_k = 0$ else where.
So we derive the so called \textit{two-fluid} formulation by multiplying \ref{eq:general_conservation} by the PIF, $\chi_k$ and rearranging each term. 
By making use of the \ref{eq:phaseindicator_transport}, it yields  
\begin{equation}
    \pddt (\chi_k f_k)
    = \grad \cdot (\chi_k \bm{\Phi}_k - \chi_k f_k \textbf{u}_k)
    + \chi_k \textbf{S}_k
    + \left[
        \bm{\Phi}_k 
        + f_k 
        \left(
            \textbf{u}_I
            - \textbf{u}_k
        \right) 
    \right]
    \cdot \textbf{n}_k \delta_I, 
    \label{eq:two-fluid_global}
\end{equation}
where the last term is the interfacial source term representing the interactions across the phases. 
As an example, if $f_k$ turns out to be the momentum, then the interfacial source term would be the drag forces between the phases (see next section). 
For a more detailed derivation of this equation see \ref{ap:average}.
As desired, in this equation all quantities $f_k$ are factor of $\chi_k$. 
So we transport the field $\chi_k f_k$, which is a field quantity defined over the whole domain. 
Note that the interfacial term is factor of $\delta_I$ instead of $\chi_k$ as it is defined on the interface.
Anyhow, \ref{eq:two-fluid_global} is defined over the entire domain and is therefore considered as a valid transport equation. 

The consistency across several phase's conservation equations, is made through the interfacial source term.  
And more precisely it is done with the use of the \textbf{jump condition} at the interface.
It is defined as the sum of the interfacial terms of each phase $k$ over the domain of the interfaces $\textbf{y}_I$. 
It will or will not be null depending on the nature of $f_k$. 
Therefore we define this jump condition as
\begin{equation}
    \sum_k 
    \left[
        \bm{\Phi}_k 
        + f_k 
        \left(
            \textbf{u}_I
            - \textbf{u}_k
        \right) 
    \right]
    \cdot \textbf{n}_k\delta_I
    = \textbf{J}_I\delta_I
    \label{eq:general_jump}
\end{equation}
where $\textbf{J}_I$ is the \textit{jump quantity} related to $f_k$, such as the surface tension force if $f_k$ were the momentum.

Now that we properly defined any conservation laws inside and across each phase, let's derive the \textit{single-fluid} formulation for \ref{eq:general_conservation}.
To do so we sum on all phases \ref{eq:two-fluid_global}, and make use of the jump condition \ref{eq:general_jump} to simplify the interfacial terms. 
Besides, we define the arbitrary quantity $f$ being the sum of the $f_k\chi_k$'s on all phases, i.e. $f = \sum_k \chi_k f_k$.
By taking into account these remarks it is trivial to show that, 
\begin{equation}
    \pddt f
    = \grad \cdot (\bm{\Phi} - f \textbf{u})
    + \textbf{S}
    + \textbf{J}_I \delta_I,
    \label{eq:single-fluid_global}
\end{equation}
which is the \textit{single-fluid} formulation. 
In the following we expose the most important conservation equations for our purpose by replacing $f_k$ or $f$ by the desired quantities.  

\subsection{Mass balances}

Let's introduce the incompressible mass balance for multiphase flows by taking $f_k = \rho_k$. 
According to \ref{eq:two-fluid_global} the \textit{two-fluid} formulation mass balance reads as,
\begin{equation}
    \rho_k\pddt\chi_k
    = 
    - \rho_k\grad\cdot(\textbf{u}_k \chi_k)  
    + \rho_k (\textbf{u}_k-\textbf{u}_I) \cdot \textbf{n}_k \delta_I,
    \label{eq:two-fluid_mass}
\end{equation}
where the first term on the right-hand side (RHS) is the advection of the mass and second term the mass transfer source term.
The points located on the interface must respect the mass jump condition,
\begin{equation}
    \sum_k \left[
        \rho_k (\textbf{u}_k-\textbf{u}_I) \cdot \textbf{n}_k
    \right]
    =\sum_k M_k 
    = 0,
    \label{eq:mass_jump}
\end{equation}
where we introduced the notation $M_k$ to refer to the mass transfer term, and we set $\sum M_k =0$ since any mass leaving a phase is conserved in the neighboring phase.
Note that this equation is valid under the condition that there is no accumulation of mass at the 
interface which is usually what is considered (see \citet{morel2015mathematical}). 

By summing \ref{eq:two-fluid_mass} on all phases and using the jump condition \ref{eq:mass_jump} and the definition of the field quantities we can derive the \textit{single-fluid} formulation of the mass balance equation, namely,  
\begin{equation}
    \pddt \rho
    + \grad\cdot(\rho\textbf{u}_k). 
    = 0 
    \label{eq:single-fluid_mass}
\end{equation}
Note that even though the densities $\rho_k$ are constant in each phase, the overall density $\rho$ isn't since it varies across the phases, therefore it must be considered inside the derivatives.


\subsection{Momentum balances}

We can apply the same routine to obtain the momentum balance equation than for the mass balance, i.e. we transport the momentum field, thus we set $f_k = \rho_k \textbf{u}_k$ in \ref{eq:two-fluid_global}.
Then, the momentum balance in the \textit{two-fluid} formulation is given by,
\begin{equation}
    \pddt (\rho_k \chi_k \textbf{u}_k)
    = \grad \cdot \left(
        \chi_k\textbf{T}_k 
        -\rho_k \chi_k \textbf{u}_k \textbf{u}_k 
    \right) 
    + \textbf{b}_k\chi_k
    +  \delta_I\left(
        M_k \textbf{u}_k
        + \textbf{T}_k \cdot \textbf{n}_k 
    \right),
    \label{eq:two-fuild_momentum}
\end{equation}
where $\textbf{T}_k$ is the stress tensor.
In this work we consider solely Newtonian fluid therefore,
\begin{equation*}
    \textbf{T}_k 
    = -\grad p_k 
    + \mu_k \left[\grad \textbf{u} - (\grad\textbf{u})^T\right],
\end{equation*}
with $p_k$ the pressure field and $\textbf{b}_k$ the body forces present in the phase $k$.
The last term on the RHS is due to the exchange momentum at the interface.
The force acting on the interface is made up of two contribution : the drag force represented by $\textbf{T}\cdot\textbf{n}$, and the momentum transfer resulting from mass exchange, which is represented by $M_k \textbf{u}_k$.
The jump condition at the interface for the momentum reads as, 
\begin{equation*}
    \sum_k \left(
        M_k \textbf{u}_k
        + \textbf{T}_k \cdot \textbf{n}_k
    \right)
    = \textbf{f}_I,
    \label{eq:stressjump}
\end{equation*}
where $\textbf{f}_I$ is the \textit{jump force}. 
If we consider no momentum jump caused by the mass transfer, this force is solely due to the surface tension forces. 
Therefore, it can be expressed as \citep{tryggvason2011direct},  
\begin{equation}
    \textbf{f}_I    
    = \sigma \kappa \textbf{n}
    +\grad_I \sigma,
    \label{eq:f_I}
\end{equation}
where we recall that $\kappa$ is the curvature of the interface defined by $\kappa = -\grad_I\cdot\textbf{n}$ or $\kappa = -\grad\cdot\textbf{n}$. 
We can note that the surface tension force include a term expressed as the divergence of $\sigma$, which is the contribution of non-constant superficial surface tension coefficient which it is commonly known as the  Marangoni forces.
In this work we discard the effect of these forces and consider solely a constant surface tension coefficient.  

Next, we introduce the \textit{single-fluid} formulation of the momentum balance by adding on all phases \ref{eq:two-fuild_momentum} and using the momentum jump condition \ref{eq:stressjump}, 
\begin{equation}
   \pddt (\rho \textbf{u})
    = \grad \cdot (\textbf{T} -\rho  \textbf{u} \textbf{u})
    + \textbf{b}
    + \textbf{f}_I\delta_I.
    \label{eq:one-fuild_momentum}
\end{equation}
where, $\textbf{f}_I\delta_I$ is the contribution of the surface force due to \ref{eq:stressjump} to the bulk momentum $\rho \textbf{u}$. 
It is interesting to note that in the static case, when $\textbf{u}=0$ on all the domain, we recover the famous Laplace equation, i.e. $ -\bm{b} = \hat{\bm{\nabla}}p +\sigma \kappa \textbf{n} $.

\subsection{Energy balance}

Finally, the balance equation for the total Energy transport can be obtained by transporting the total energy $\rho_k E_k$, defined such that $\rho_k E_k = \rho_k(e_k + \frac{1}{2}u_k^2)$, 
where $e_k$ is the internal energy and $u_k^2$ the dot product $\textbf{u}_k\cdot\textbf{u}_k$.
Therefore, $\rho_k E_k$ is the sum of the internal energy and the kinetic energy of phase $k$.
The total energy balance for phase $k$ is obtained by setting $f_k = \rho_k E_k$ in \ref{eq:two-fluid_global} and reads as
\begin{multline}
    \pddt \left(
        \chi_k \rho_k E_k
    \right)
    = \grad \cdot \left[
        \chi_k\left(
            \textbf{T}_k\cdot\textbf{u}_k
            - \textbf{q}_k
            - \textbf{u}_k \rho_k E_k 
        \right)
    \right]
    + \chi_k\textbf{b}_k\cdot\textbf{u}_k\\
    +\left[
        M_k E_k
        + \left(
        \textbf{T}_k\cdot\textbf{u}_k
        - \textbf{q}_k
        \right) 
        \cdot\textbf{n}_k   
    \right]\delta_I,
    \label{eq:two-fuild_totenergy}
\end{multline}
where $\textbf{q}_k$ is the internal energy flux in the phase $k$, usually defined with a Fourier 
law, namely, $\textbf{q}_k = D_k\grad T_k$ with $T_k$ the temperature and $D_k$ the thermal 
conductivity. 
The first term on the RHS correspond to the work of the stress $\textbf{T}_k\cdot \textbf{u}_k$ the internal flux, $\textbf{q}_k$,  and the advection of the total energy, $\textbf{u}_k \rho_k E_k$. 
The source term $\textbf{b}\cdot\textbf{u}_k$ correspond to the work of the body forces. 
Lastly, the interfacial term contains the total energy exchange caused by mass transfer, $M_k E_k$, the exchange of work $\textbf{T}_k\cdot\textbf{u}_k$ due to interfacial stress, and the internal flux from the neighboring phases $\textbf{q}_k$.   
The total energy jump condition reads as, 
\begin{equation}
    \sum_k \left[
        M_k E_k
        + \left(
            \textbf{T}_k \cdot \textbf{u}_k
            - \textbf{q}_k 
        \right)
        \cdot \textbf{n}_k
    \right]
    = \textbf{f}_I\cdot\textbf{u}_I.
    \label{eq:total_energy_jump}
\end{equation}
where we assumed that the \textit{energy jump} is solely due to the work of the interfacial forces. 
Indeed, the jump due to different thermal conductivity and energy mass exchange is not necessarily null. 
However, it will not be addressed in this manuscript as it falls outside the scope of its focus. 

Likewise, the \textit{single-fluid} formulation is given by adding up \ref{eq:two-fuild_totenergy}
on all phases and making use the above jump condition, it is then trivial to show that
\begin{equation}
    \pddt (\rho E)
    = \grad \cdot \left(
        \textbf{T}\cdot\textbf{u}
        - \textbf{q}
        -\rho E \textbf{u} 
    \right)
    + \textbf{b}\cdot\textbf{u}
    +(\textbf{f}_I\cdot\textbf{u})\delta_I.
    \label{eq:one-fuild_totenergy}
\end{equation}
This equation can be split into two distinct equation, one for the kinetic energy and the other for the internal energy. 
The kinetic energy equation is obtained by taking the dot product between 
\ref{eq:two-fuild_momentum} and the velocity vector $\textbf{u}$, yielding,
\begin{equation}
    \frac{1}{2}\pddt (\rho u^2)
    +\frac{1}{2} \grad \cdot (\rho u^2 \textbf{u})
    =\textbf{u}\cdot\left(
        \grad\cdot\textbf{T}
        +\textbf{b}
        +\textbf{f}_I\delta_I.
    \right)
    \label{eq:one-fuild_meca}
\end{equation}
The LHS of this equation is the advection of the kinetic energy $\frac{1}{2}\rho u^2$, while on the RHS we recognize the work of the stress, body forces and surface tension forces. 
Additionally, by taking the difference between the mechanical and the total energy equation, one 
obtain the internal energy equation \citep{tryggvason2011direct}
\begin{equation}
    \pddt (\rho e)
    = 
    \textbf{T}:\grad\textbf{u}
    - \grad\cdot(\textbf{q} + \rho  e\textbf{u}),
    \label{eq:one-fuild_internal_energy}
\end{equation}
where the first term on the RHS is the source term caused by friction in the fluid and the remaining terms have already been defined. 

As a conclusion, in the \textit{two-fluid} approach the interfacial force term is taken into account though the jump condition. 
It is mostly used in theoretical developments, since it describes explicitly the forces between the phases in presence. 
Where in the \textit{single-fluid} method, it is included as a body forces term in the momentum conservation balance (\ref{eq:two-fuild_momentum}) and in the total energy conservation, \ref{eq:two-fuild_totenergy}. 
This formulation allows a greater ease to implement multiphase flows into CFD code for reason that will be detailed in the next section.
More detailed and derivations of these equations can be found in the book of \citet{morel2015mathematical}. 

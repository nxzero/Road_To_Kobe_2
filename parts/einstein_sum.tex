\subsection{Numerical approximation of the ensemble average}

In this section, we briefly explain how to compute ensemble averaged quantities based on the DNS results. 

\subsubsection{Continuous averaged quantities}

Following \citet{du2022analysis} we consider \textit{ergodicity} at all times of the numerical experiment.
Thus, the ensemble average of an arbitrary quantity $X$ can be approximated by a spatial average $\Xavg{X}$ followed by a time average $\Tavg{X}$. 
Specifically, the ensemble average of a numerical field, $X$, is taken through space and time following,
\begin{equation}
    \avg{X}
    = \Tavg{\Xavg{X}}
    = \frac{1}{ t_{end} - t_0}\int_{t_0}^{t_{end}} 
    \Xavg{X}(t) dt
\end{equation}
where, 
\begin{equation}
    \Xavg{X}(t)
    = \frac{1}{L^3}\int 
    X(\textbf{x},t) d\textbf{x}
\end{equation}
where $L$ is the length of the cubic domain.
$t_0$ and $t_{end}$ are the starting and ending times of sampling, respectively.
In practice, we take $t_0$ such that the simulation reaches a statistically steady regime.  
It has been observed that $t_0$ is reached at about $50 \sqrt{g/d}$. 

To compute the phase average of the local phase velocity $\textbf{u}_k^0$, we simply perform an integration over space and time, 
\begin{equation}
    \textbf{u}_k = \frac{1}{\phi_k} \Tavg{\Xavg{\chi_k \textbf{u}_k^0}}
\end{equation}
where the indicator function $\chi_f$ must be understood as its approximation in the DNS, which is the color function used by the CFD code (\url{http://basilisk.fr}). 
Since the homogeneous and statistically steady-state hypotheses are supposed to be true, the mean of the droplet center of mass velocity is equivalent to the dispersed phase averaged velocity, $\textbf{u}_d$.
Thus, we may compute the ensemble average relative velocity with the operation, 
\begin{equation}
    \textbf{u}_{pf} = 
    \frac{1}{\phi_d} \Tavg{\Xavg{\chi_d \textbf{u}_d^0}}
    - \frac{1}{\phi_f} \Tavg{\Xavg{\chi_f \textbf{u}_f^0}}. 
\end{equation} 
To compute the ensemble average of the continuous phase velocity variance, we obtain
\begin{equation}
    \avg{\chi_f\textbf{u}_f' \textbf{u}_f'}
    % = \Tavg{\Xavg{\chi_f \textbf{u}_f' \textbf{u}_f'}}
    = \Tavg{\Xavg{\chi_f (\textbf{u}_f^0 -\textbf{u}_f ) (\textbf{u}_f^0 -\textbf{u}_f)}}
    = \Tavg{\Xavg{\chi_f \textbf{u}_f^0 \textbf{u}_f^0}}
    -  \phi_f  \textbf{u}_f \textbf{u}_f.
    \label{eq:def_uuf_num} 
\end{equation}
Consequently, \ref{eq:def_uuf_num} indicates that we must take the average of the product of the velocities of the continuous phase domain, and then we retrieve the mean velocities' product.  

\subsubsection{Computation of Lagrangian quantities}

When several droplets belong to the same VoF tracer it is challenging to distinguish them.
Therefore, some comments are in order regarding the computation of the Lagrangian quantities such as the mean the center of mass velocity of the particles, center of mass position\ldots   

As detailed above the \href{http://basilisk.fr/src/tag.h}{tag.h} algorithm permits us to distinguish the different droplet volumes within a given VoF field. 
Let define $\texttt{t}_i[\text{c},t]$ as the ``tag'' field corresponding to the droplet $i$, evaluated at the cell ``c'' of the mesh and at time $t$.
This field is equal to 1 for cells inside the volume of the particle $i$ and 0 outside.
Then, the \href{http://basilisk.fr/src/tag.h}{tag.h} algorithm provides us with $N$ ``tag'' field, i.e. one for each droplet. 
Based on $\texttt{t}_i$, and on the other fields solved by \ref{eq:dt_C} and \ref{eq:dt_urho}, we can easily define the particles Lagrangian properties mentioned above. 
The mass, position and momentum of a particle are then computed as follows 
\begin{align}
    m_i(t)
    = \sum_{\text{c} = 1}^\text{last cell}
    \rho [\text{c},t]
    \texttt{t}_i[\text{c},t]dv[c],
    \\ 
    \textbf{x}_i(t)
    = 
    \frac{1}{m_i(t)}
    \sum_{\text{c} = 1}^\text{last cell}
    \textbf{x} 
    \rho [\text{c},t]
    \texttt{t}_i[\text{c},t]dv[c],
    \\ 
    \textbf{u}_i(t)
    = 
    \frac{1}{m_i(t)}
    \sum_{\text{c} = 1}^\text{last cell}
    \textbf{u} [\text{c},t]
    \rho [\text{c},t]
    \texttt{t}_i[\text{c},t]dv[c],
    \label{eq:Lagrangian_num2}
\end{align}
where the sum from ``$\text{c} = 1$'' to ``$\text{last cell}$'' denotes a summation over the entire mesh grid, and $\rho[\text{c},t]$ and $\textbf{u}[\text{c},t]$ are the density field and velocity field evaluated at the cell ``c'' and at a simulation time step $t$. 
Likewise, $\textbf{x}[\text{c}]$ and $dv[c]$ correspond to the Cartesian coordinate and the volume of the cell, respectively. 

From \ref{eq:Lagrangian_num2} we can compute the second moment of mass of the particles and the first moment of momentum according to the relations, 
\begin{align*}
  \textbf{M}_i(t)
  = 
  \sum_{\text{c} = 1}^\text{last cell}
  (\textbf{x} - \textbf{x}_i[t])
  (\textbf{x} - \textbf{x}_i[t])
  \rho [\text{c},t]
  \texttt{t}_i[\text{c},t]dv[c],
  \\ 
  \textbf{u}_i(t)
  = 
  \sum_{\text{c} = 1}^\text{last cell}
  (\textbf{x} - \textbf{x}_i[t])
  \textbf{u} [\text{c},t]
  \rho [\text{c},t]
  \texttt{t}_i[\text{c},t]dv[c],
\end{align*}
and so on for the other Lagrangian properties. 

The ensemble average of the Lagrangian velocity is then defined as, 
\begin{equation}
  \textbf{u}_p = \frac{1}{N_p}\sum^{N_b}_i \textbf{u}_i, 
\end{equation}
and so on for the other properties. 

\section{Additional details on the algorithms}
In addition to the developments above, implementing the algorithm required further modifications to the \texttt{Basilisk C} source code. 
Specifically, since computing the closure terms involves extensive tensor algebra, we implemented a new macro in the \texttt{Basilisk C} compiler that enables users to write code directly using Einstein summation convention. 
Additionally, we introduced a new standard struct, \texttt{mat3}, alongside the existing \texttt{coord} struct. 
The \texttt{mat3} struct represents second-order tensors, while \texttt{coord} is used for vectors. 
By incorporating these two structs into the Basilisk C language, we developed an automated macro allowing \texttt{MPI} reduction directly on \texttt{coord} and \texttt{mat3} structures. 

Overall, we provided all necessary tools in the Basilisk pre-compiler to enable programming directly with tensor algebra.
Below is a brief description of the main functionality of the so-called, \texttt{Einstein\_sum}$(\ldots)$ macro.

\subsection{Einstein summation notation}

This \href{http://basilisk.fr/src/ast/einstein_sum.h}{header file} contains the functions related to the Einstein summation macro implemented by the $\texttt{Basilisk C}$ preprocessor. This macro allows the user to write tensor algebra in index notation using $\texttt{Basilisk C}$ code. The syntax follows the famous \href{https://en.wikipedia.org/wiki/Einstein\_notation}{Einstein summation convention}.

\subsubsection{ Mathematical definitions}

Let us give the basis of the Einstein summation convention through the example of a simple scalar product between two second-order tensors in $3D$ space. In mathematics and physics, it is customary to write second-order tensors as $\textbf{A} = \sum_{i,j = 1}^3 A_{ij} \bm e_i\bm e_j$ where $\bm e_i$ and $\bm e_j$ are basis vectors in $3D$ space and $A_{ij}$ is the component of the tensor $\textbf A$ projected on the vectors $\bm e_i\bm e_j$. In tensor notation the inner product between two arbitrary second-order tensors: $\textbf{A}$ and $\textbf{B}$, can be written as $\textbf{C} =  \textbf{A}\cdot\textbf{B}$. If the basis is orthonormal, the components of the tensor $\textbf C$ are given by
$$ C_{ij} = \sum_{k=1}^3 A_{ik} B_{kj}.$$
Since the summation operator can be quite cumbersome in the Einstein summation convention we rewrite this operation as
$$ C_{ij} = A_{ik} B_{kj} $$
where the summation over $k$ from $1$ to $3$ is implicit.
We took the example of a simple scalar product, nevertheless note that this notation extends to any tensor operations which can be much more complex.   
Therefore, to remove any ambiguity in the Einstein summation notation, one has to follow the following rules:
1. each index can appear at most twice in any term,
2. repeated indices are implicitly summed over,
3. each term must contain identical non-repeated indices.

For example the expression: $M_{ij}v_iv_j$ is valid and indicate a double summation over $i$ and $j$, 
while the rexpression $M_{ij}v_ju_j$ is ambigous since the index $j$ appears $3$ times. 
For a more complete and rigourous definition of the Einstein summation convention we recommand the Wikipedia page: \href{https://en.wikipedia.org/wiki/Einstein\_notation}{Einstein summation convention}.

\subsubsection{ User interface}

To write a scalar product in $\texttt{Basilisk C}$ one needs first to define vector and second-order tensor-like structures. The default vector structure is the `coord` struct. The second or higher-order structures must be defined by the user. 
\begin{lstlisting}[language=C]
typedef struct {
coord x,y,z;
} mytensor;
mytensor A,B,C; 
\end{lstlisting}
Note that the name `mytensor` is arbitrary. However, the name of the structure members must be $x$, $y$ and $z$. Following the Einstein summation convention the scalar product introduced in the previous paragraph can be written in $\texttt{Basilisk C}$ as
\begin{lstlisting}[language=C]
einstein_sum(i,j,k) {
  C.i.j = A.i.k*B.k.j; 
}
\end{lstlisting}
With this macro, the preprocessor of $\texttt{Basilisk C}$ interprets all lines of code within the braces as tensor operations. The letters given within the parenthesis indicate the indices on which the Einstein summation takes place. In this case a summation will be applied on the index $k$ and permutations will be performed on the indices $i$ and $j$. 
To verify that the $\texttt{Basilisk C}$ preprocessor gives the desired results, one can precompile the $\texttt{Basilisk C}$ file with the command line
\begin{lstlisting}
qcc -source my_file.c
\end{lstlisting}
The results of the precompilation stored in \texttt{\_my\_file.c} will be
\begin{lstlisting}[language=C]
{
  C.x.x = A.x.x*B.x.x+ A.x.y*B.y.x+ A.x.z*B.z.x;
  C.x.y = A.x.x*B.x.y+ A.x.y*B.y.y+ A.x.z*B.z.y;
  C.x.z = A.x.x*B.x.z+ A.x.y*B.y.z+ A.x.z*B.z.z;
  C.y.x = A.y.x*B.x.x+ A.y.y*B.y.x+ A.y.z*B.z.x;
  C.y.y = A.y.x*B.x.y+ A.y.y*B.y.y+ A.y.z*B.z.y;
  C.y.z = A.y.x*B.x.z+ A.y.y*B.y.z+ A.y.z*B.z.z;
  C.z.x = A.z.x*B.x.x+ A.z.y*B.y.x+ A.z.z*B.z.x;
  C.z.y = A.z.x*B.x.y+ A.z.y*B.y.y+ A.z.z*B.z.y;
  C.z.z = A.z.x*B.x.z+ A.z.y*B.y.z+ A.z.z*B.z.z;
}
\end{lstlisting}
Note that the preprocessor duplicated the lines for each permutation of $i$ and $j$ and applied a summation over the index $k$. This macro also applies to $\texttt{Basilisk C}$ vector fields (such as the velocity field `u.x[]`) and any other user-defined structures with members named $x$, $y$ and $z$.

For higher-order rank tensors one can note that these expressions become quite cumbersome.
It is for that reason that the \texttt{einstein\_sum} macro has been created. 
For more examples one can check the test file \href{einstein\_sum.c}{/src/test/einstein\_sum.c}.

\subsubsection{Specific cases}

As long as the user follows the rules mentioned above, the desired result will be obtained. 
However, the \texttt{einstein\_sum} macro is somewhat more permissive than the original rules of the convention. 
For example the code
\begin{lstlisting}[language=C]
U.i = M.i.j*u.j*v.j;
\end{lstlisting}
which is **not defined correctly** will still be compiled, and in $2D$ will give
\begin{lstlisting}[language=C]
U.x = M.x.x*u.x*v.x + M.x.y*u.y*v.y;
U.y = M.y.x*u.x*v.x + M.y.y*u.y*v.y;
\end{lstlisting}
The use of C functions within the macro is also allowed. 
Let `function(args)` be an arbitrary C function. 
Then the expression, 
\begin{lstlisting}[language=C]
U.i = function (M.i.j*u.j*v.j);
\end{lstlisting}
will be expanded to
\begin{lstlisting}[language=C]
U.x = function (M.x.x*u.x*v.x + M.x.y*u.y*v.y);
U.y = function (M.y.x*u.x*v.x + M.y.y*u.y*v.y);
\end{lstlisting}

The summation is applied at the lowest level of the expression where all the indices that must be summed over are present. 
Here the summation takes place inside the parenthesis since all indices \texttt{j} are contained within them. 

A useful example of the use of a function is the computation of the L2 norm of the second-order tensor $\textbf C$
\begin{lstlisting}[language=C]
L2 = sqrt (C.i.j*C.i.j);
\end{lstlisting}
Lastly, if no equality sign is identified, the preprocessor will not perform any summation operations. It will only carry the permutation on the current line of code. For example if one wants to print the content of a tensor one may write 
\begin{lstlisting}[language=C]
einstein_sum (i,j) {
  fprintf (stderr, "%g\n", C.i.j);
}
\end{lstlisting}
which gives in $2D$
\begin{lstlisting}[language=C]
  fprintf (stderr, "%g\n", C.x.x);
  fprintf (stderr, "%g\n", C.x.y);
  fprintf (stderr, "%g\n", C.y.x);
  fprintf (stderr, "%g\n", C.y.y);
\end{lstlisting}


In summary the \texttt{einstein\_sum} macro follows the following steps:
\begin{enumerate}
    \item  It Identifies if a given line of code is an assignment expression with a "=" sign or not. 
    \begin{enumerate}
    \item If it is, it identifies the indices on the right and left-hand side of the expression. 
       It then carries a summation over the indices that appear only on the right-hand side. 
    \item If it is not an equality, it does nothing at this step. 
    \end{enumerate}
    \item  It then copies the lines of codes and carries permutations on the indices present on the line (excluding the one that has been summed over). 
    \item Steps 1. and 2. are repeated for all lines within the macro. 
\end{enumerate}
Note that these steps do not follow the original Einstein summation convention rules. 
Therefore, some expressions may be ambiguous and the user is advised to always check for the pre-compiler results using \texttt{qcc -source my\_file.c} before using the code.

\subsection{Performing  MPI reduction on list of \texttt{coord} and \texttt{mat3} types}

The implementation of the reduction feature is available on \texttt{Basilisk} website by tracking the Darcs update authored by me.

No further comments are necessary regarding this development, as the functionality mirrors that of the existing \texttt{reduction()} macro.



% \section{Concluding remarks}

Now that we have all the tools required to perform DNS, we can proceed to study the closure terms in the next few chapters. Although we may be able to extend the theoretical closures presented in \ref{chap:daniel2}, the reader must be reminded that, with these DNS, we are still confined to a very specific scenario. 
The effects of polydispersity and non-homogeneous flow regimes cannot be studied within this setup and will therefore be disregarded in the closure modeling. 
This further highlights the complexity and the actual resources needed to numerically complete the closure of the Navier-Stokes equations.






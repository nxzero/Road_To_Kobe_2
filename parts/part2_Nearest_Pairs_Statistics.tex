\section{Mesoscale equations}

Let $\delta_i(\textbf{x},t) = \delta(\textbf{x} - \textbf{x}_i(t))$ with $\delta$ the Dirac function and $\textbf{x}_i$ the position of the particle $i$.  
Similarly, we define $\delta_j(\textbf{y},t) =\delta(\textbf{y} - \textbf{x}_j(t)) $ for the particle label $j$. 
Then, it is trivial to show that, 
\begin{align*}
    \pddt \delta_i + \partial_\textbf{x} \cdot (\textbf{u}_i \delta_i) &= 0\\
    \pddt \delta_j + \partial_\textbf{y} \cdot (\textbf{u}_j \delta_j) &= 0.
\end{align*}
where we introduced $\textbf{u} = \ddt \textbf{x}$
and both equation equal $0$ since we do not consider particle appearance or vanishing. 
Manipulating those equations we can then have, 
\begin{equation}
    \pddt (\delta_j\delta_i) + \partial_\textbf{x} \cdot (\textbf{u}_i \delta_i\delta_j) + \partial_\textbf{y} \cdot (\textbf{u}_j \delta_i\delta_j) = 0.
    \label{eq:delta_beta_delat_alpha}
\end{equation}
The pair distribution function is defined as,
\begin{equation}
    P_2(\textbf{x},\textbf{y}) = \int \sum_i\delta_i(\textbf{x},t) \sum_j\delta_j(\textbf{y},t) P(\mathscr{C})d\mathscr{C}
\end{equation}
and the conditional pair average of a quantity \textbf{q} is defined as, 
\begin{equation}
    \avg{\textbf{q}}_2(\textbf{x},\textbf{y})P_2(\textbf{x},\textbf{y}) = \int \sum_i\delta_i(\textbf{x},t) \sum_j\delta_j(\textbf{y},t) \textbf{q} P(\mathscr{C})d\mathscr{C}
\end{equation}
Therefore, summing \ref{eq:delta_beta_delat_alpha} over all $i$ and $j$ and integrating on every configuration gives, 
\begin{equation}
    \pddt P_2 + \partial_\textbf{x} \cdot (\condavg{\textbf{u}_i}{2}P_2) + \partial_\textbf{y} \cdot (\condavg{\textbf{u}_j}{2}P_2) = 0.
    \label{eq:P_2}
\end{equation}
where then obtain an evolution equation for $P_2(\textbf{x},\textbf{y},t)$
integrating \ref{eq:P_2} over $\textbf{y}$ would give us the classic number density equation. 

Now let's define 
\begin{equation*}
    h_{ij} 
    = \frac{1}{N_i}
    \prod_{k \neq i,j}
    H(|\textbf{x}_k - \textbf{x}_i| - |\textbf{x}_j - \textbf{x}_i|)
    = \frac{1}{N_i}
    \prod_{k \neq i,j}
    H_{kij}
\end{equation*}
\begin{equation*}
    N_i
    = 
    \sum_{j\neq i}
    \prod_{k \neq i,j}
    H(|\textbf{x}_k - \textbf{x}_i| - |\textbf{x}_j - \textbf{x}_i|)
    = 
    \sum_{j\neq i}
    \prod_{k \neq i,j}
    H_{kij}
\end{equation*}
The function $h_{ij} (\mathscr{C}) = 1/N_i$ , if particle $j$ is one
of the nearest neighbors to particle $i$ in configuration $\mathscr{C}$ ;
and $h_{ij} (\mathscr{C} ) = 0$ otherwise.

Multiplying \ref{eq:delta_beta_delat_alpha} by $h_{ij}$ gives, 
\begin{equation}
    \pddt (\delta_j\delta_i h_{ij}) + \partial_\textbf{x} \cdot (\textbf{u}_i \delta_i\delta_j h_{ij}) + \partial_\textbf{y} \cdot (\textbf{u}_j \delta_i\delta_j h_{ij}) = \delta_j\delta_i\pddt h_{ij}
    \label{eq:delta_beta_delat_alpha_h}
\end{equation}
Considering implicit summation on $i$ and $j$ and integrating equation \ref{eq:delta_beta_delat_alpha_h}  yields, 

\begin{multline}
    \pddt \left(\int \delta_j\delta_i h_{ij} d\mathscr{P} \right) 
    + \partial_\textbf{x} \cdot \left(\int\textbf{u}_i \delta_i\delta_j h_{ij} d\mathscr{P} \right) 
    + \partial_\textbf{y} \cdot \left( \int \textbf{u}_j \delta_i\delta_j h_{ij} d\mathscr{P} \right) 
    \\=\int \delta_j\delta_i\pddt h_{ij} \mathscr{P}
\end{multline}
defining $\nstavg{\textbf{u}_i} P_{nst}= \int\textbf{u}_i \delta_i\delta_j h_{ij} d\mathscr{P} $ and $P_{nst} = \int \delta_j\delta_i h_{ij} d\mathscr{P} $ we obtain :
\begin{equation}
    \pddt \left(P_{nst} \right) 
    + \partial_\textbf{x} \cdot \left(\nstavg{\textbf{u}_i} P_{nst}\right) 
    + \partial_\textbf{y} \cdot \left( \nstavg{\textbf{u}_j} P_{nst} \right) 
    =\int \delta_j\delta_i\pddt h_{ij} \mathscr{P}
\end{equation}

\subsection*{Computation of the kernel }
We now want to find the expression of $\pddt h$. 
\begin{align*}
    \pddt h_{ij}
    &= \pddt \left(
        \frac{1}{N_i}
    \right)
    \prod_{k \neq i,j}  H_{kij}
    + \frac{1}{N_i} \pddt \left[
        \prod_{k \neq i,j}
        H_{kij} 
    \right] \\
    &= - \frac{\pddt N_i}{N_i^2}
    \prod_{k \neq i,j}  H_{kij}
    + \frac{1}{N_i} \sum_{l \neq i,j} \left[
        \pddt H_{lij}
        \prod_{k \neq i,j,l}
        H_{kij} 
    \right] \\
\end{align*}
From now on we decompose the calculation. 
Notice that we need to derive $\pddt N_i$ which is, 
\begin{equation}
   \pddt N_i
    = 
    \sum_{j\neq i}
    \pddt
    \prod_{k \neq i,j}
    H_{kij}\\
    = \sum_{j\neq i}
    \sum_{l \neq i,j} \left[
        \pddt H_{lij}
        \prod_{k \neq i,j,l}
        H_{kij} 
    \right]
\end{equation}
Then we can imporove the notaition
\begin{multline}
    \pddt h_{ij}
    = \\
    - h_{ij} \frac{1}{N_i}
    \sum_{j\neq i}
    \sum_{l \neq i,j} \left[
        \pddt H_{lij}
        \prod_{k \neq i,j,l}
        H_{kij} 
    \right]
    + \frac{1}{N_i} \sum_{l \neq i,j} \left[
        \pddt H_{lij}
        \prod_{k \neq i,j,l}
        H_{kij} 
    \right] 
\end{multline}

Considering that 
\begin{equation*}
    \pddt H(|\textbf{x}_l - \textbf{x}_i| - |\textbf{x}_j - \textbf{x}_i|)
    = \left(
        \pddt|\textbf{x}_l - \textbf{x}_i| - \pddt|\textbf{x}_j - \textbf{x}_i|
    \right)
    \delta(|\textbf{x}_l - \textbf{x}_i| - |\textbf{x}_j - \textbf{x}_i|)
\end{equation*}
where we assumed that $\pddt H(x) = \delta(x)$, and if we take $\pddt |f(x)| = \frac{f(x) f'(x)}{|f(x)|}$
\begin{multline*}
    \pddt H(|\textbf{x}_l - \textbf{x}_i| - |\textbf{x}_j - \textbf{x}_i|)
    \\= 
    \left[
    \frac{(\textbf{x}_l - \textbf{x}_i) \cdot (\textbf{u}_l - \textbf{u}_i)}{|\textbf{x}_l - \textbf{x}_i|}
    - 
    \frac{(\textbf{x}_j - \textbf{x}_i) \cdot(\textbf{u}_j - \textbf{u}_i)}{|\textbf{x}_j - \textbf{x}_i|}
    \right]\delta(|\textbf{x}_l - \textbf{x}_i| - |\textbf{x}_j - \textbf{x}_i|)
    \\= 
    \left[
    \textbf{n}_{li}\cdot \textbf{w}_{li}
    - 
    \textbf{n}_{ji}\cdot\textbf{w}_{ji}
    \right]\delta(|\textbf{x}_l - \textbf{x}_i| - |\textbf{x}_j - \textbf{x}_i|)
    \\= U_{lij}\delta_{lij}
\end{multline*}
Therefore, $\pddt H$ is the difference between the relative velocities between the $j$ and $l$ with the particle $i$ particles along their relative position. 

Finally, 
\begin{multline}
    \pddt h_{ij}
    = 
   \frac{1}{N_i} \sum_{l \neq i,j} \left[
    U_{lij}
    \delta_{lij}
    \prod_{k \neq i,j,l}
    H_{kij} 
    \right] 
    -  \frac{h_{ij}}{N_i}
    \sum_{j\neq i}
    \sum_{l \neq i,j} \left[
        U_{lij}
        \delta_{lij}
        \prod_{k \neq i,j,l}
        H_{kij} 
    \right]
    \label{eq:dt_h_ij}
\end{multline}
To summary : 
\begin{itemize}
    \item $H_{kij} =  1$ if $|\textbf{x}_k - \textbf{x}_i| > |\textbf{x}_j - \textbf{x}_i|$ if $j$ is closer to $i$ than $k$ to $i$.
    \item $H_{kij} =  0$ if $|\textbf{x}_k - \textbf{x}_i| < |\textbf{x}_j - \textbf{x}_i|$ if $k$ is closer to $i$ than $j$ to $i$.
    \item $\delta_{lij} =  0$ if $|\textbf{x}_l - \textbf{x}_i| \neq |\textbf{x}_j - \textbf{x}_i|$ if $k$ and $i$ are at a different distance than $j$ and $i$. 
    \item $\delta_{lij} =  1$ if $|\textbf{x}_l - \textbf{x}_i| = |\textbf{x}_j - \textbf{x}_i|$ if $k$ and $i$ are at the same distance as $j$ and $i$. 
\end{itemize}
Meaning that the first term of \ref{eq:dt_h_ij} is :
\begin{itemize}
    \item $\prod_{k \neq i,j,l} H_{kij} = 1$ if $j$ is the nearest neighbor to $i$ excluding the particle $l$. 
    \item $U_{lij}\delta_{lij}\prod_{k \neq i,j,l} H_{kij} = U_{lij}$ if $j$ and $l$ are at the same distance to $i$. 
    \item $1/N_i\sum_{l\neq i,j}U_{lij}\delta_{lij}\prod_{k \neq i,j,l} H_{kij} = U_{lij}/N_i$ if there is at least one particle $l$ at the same distance than $j$ to $i$.  
\end{itemize}

On the second term we add up this same contribution over all the neighbor $j$. 
Since the problem is symmetric when having two nearest neighbor this contribution will ultimately vanish. 


As a result we obtain, 
\begin{equation*}
    \pddt h_{ij} 
    = \frac{1}{N_i} \sum_{l \neq i,j} \left[
        \left[
    \textbf{n}_{li}\cdot \textbf{w}_{li}
    - 
    \textbf{n}_{ji}\cdot\textbf{w}_{ji}
    \right]\delta(|\textbf{x}_l - \textbf{x}_i| - |\textbf{x}_j - \textbf{x}_i|)
        \prod_{k \neq i,j,l}
        H_{kij} 
        \right] 
\end{equation*}
\begin{equation*}
    \pddt h_{ij} 
    = \frac{1}{N_i} \sum_{l \neq i,j} \left[
        \left[
    \textbf{n}_{li}\cdot \textbf{w}_{li}
    - 
    \textbf{n}_{ji}\cdot\textbf{w}_{ji}
    \right]\delta(|\textbf{r}_{li}| - |\textbf{r}_{ji}|)
        \prod_{k \neq i,j,l}
        H_{kij} 
        \right] 
\end{equation*}

Then, 
\begin{equation*}
    \int \delta_j\delta_i\pddt h_{ij} \mathscr{P}
    = \int \delta_j \delta_i
    \frac{1}{N_i} \sum_{l \neq i,j} \left[
        \left[
    \textbf{n}_{li}\cdot \textbf{w}_{li}
    - 
    \textbf{n}_{ji}\cdot\textbf{w}_{ji}
    \right]\delta(|\textbf{r}_{li}| - |\textbf{r}_{ji}|)
        \prod_{k \neq i,j,l}
        H_{kij} 
        \right] d\mathscr{P}
\end{equation*}
Since $r_{li} = r_{ji}$ we drop the restriction on $k$ yielding, 
\begin{equation*}
    \int \delta_j\delta_i\pddt h_{ij} \mathscr{P}
    = \int \delta_j \delta_i
    \sum_{l \neq i,j} \left[
    \textbf{n}_{li}\cdot \textbf{w}_{li}
    - 
    \textbf{n}_{ji}\cdot\textbf{w}_{ji}
    \right]\delta(|\textbf{r}_{li}| - |\textbf{r}_{ji}|)
        h_{ij}d\mathscr{P}
\end{equation*}


\section{Transport of relative properties}

We start from 
\begin{equation}
    \pddt (\delta_j\delta_i h_{ij}) + \partial_\textbf{x} \cdot (\textbf{u}_i \delta_i\delta_j h_{ij}) + \partial_\textbf{y} \cdot (\textbf{u}_j \delta_i\delta_j h_{ij}) = \delta_j\delta_i\pddt h_{ij}
\end{equation}
Let $q_{ij}$ be a relative property to the particle $i$ and $j$ and time, then multiplying this equation by $q_{ij}$ gives,
\begin{equation}
    \pddt (\delta_j\delta_i h_{ij}q_{ij}) 
    + \partial_\textbf{x} \cdot (\textbf{u}_i \delta_i\delta_j h_{ij}q_{ij}) 
    + \partial_\textbf{y} \cdot (\textbf{u}_j \delta_i\delta_j h_{ij}q_{ij}) 
    = q_{ij}\delta_j\delta_i\pddt h_{ij}
    + \delta_j\delta_i h_{ij}\dot{q_{ij}}
\end{equation}
Introducing, 
\begin{equation*}
    \nstavg{q_{ij}} P_{nst}(\textbf{x},\textbf{y})= \int\delta_i\delta_j h_{ij}q_{ij} d\mathscr{P} 
\end{equation*}
and integrating the previous equation yields, 
\begin{equation}
    \pddt (\nstavg{q_{ij}} P_{nst}) 
    + \partial_\textbf{x} \cdot (\nstavg{\textbf{u}_i q_{ij}} P_{nst}) 
    + \partial_\textbf{y} \cdot (\nstavg{\textbf{u}_j q_{ij}}P_{nst}) 
    = q_{ij}\dot{P_{nst}}
    + \nstavg{\partial_t q_{ij}}P_{nst}
\end{equation}

let's mark $\textbf{r} = \textbf{y} - \textbf{x}$
\begin{equation*}
    \nstavg{q_{ij}} P_{nst}(\textbf{x},\textbf{x}+\textbf{r})= \int\delta_i(\textbf{x})\delta_j(\textbf{x}+\textbf{r}) h_{ij}q_{ij} d\mathscr{P} 
\end{equation*}
reformulating in conditional probability, 
\begin{equation*}
    \nstavg{q_{ij}} P_{nst}(\textbf{x}+\textbf{r}|\textbf{x}) P(\textbf{x})= \int\delta_i(\textbf{x})\delta_j(\textbf{x}+\textbf{r}) h_{ij}q_{ij} d\mathscr{P} 
\end{equation*}
the mean relative velocity $w_{ij}$
\begin{align*}
    \nstavg{w_{ij}} P_{nst}(\textbf{x},\textbf{y}) 
    &= \int\delta_i\delta_j w_{ij}h_{ij} d\mathscr{P} \\
    &= \int\delta_i\delta_j (u_j - u_i)h_{ij} d\mathscr{P} \\
    &= \int\delta_i\delta_j u_j h_{ij} d\mathscr{P} 
    - \int\delta_i\delta_j u_i h_{ij} d\mathscr{P} \\
    &= (\nstavg{u_j} - \nstavg{u_i}) P_{nst}(\textbf{x},\textbf{y})
\end{align*}
first term : velocity of the particle at \textbf{x} knowing \textbf{y}.
Second :  velocity of the particle at \textbf{y} knowing \textbf{x}.
Since $P_{nst}$ isn't necessarily symmetric it is not nulle. 
but,  
\begin{align*}
    \avg{\delta_i w_{ij}}
    &=\iint\delta_i\delta_j w_{ij}h_{ij} d\mathscr{P}d\textbf{y}\\
    &= \int \nstavg{u_j} P_{nst}(\textbf{x},\textbf{y}) d\textbf{y}
    - \int \nstavg{u_i} P_{nst}(\textbf{x},\textbf{y}) d\textbf{y}
    = 
    \avg{\delta_i (u_j-u_i)}
    = 0
\end{align*}
Product , 
\begin{align*}
    \nstavg{w_{ij}w_{ij}} P_{nst}(\textbf{x},\textbf{y}) 
    &= \int\delta_i\delta_j (u_j - u_i)(u_j - u_i) h_{ij} d\mathscr{P} \\
    &= \int\delta_i\delta_j (u_ju_j + u_iu_i - u_iu_j - u_ju_i) h_{ij} d\mathscr{P} \\
    &= (\nstavg{u_iu_i}+ \nstavg{u_ju_j} - \nstavg{u_iu_j}- \nstavg{u_ju_i}) P_{nst}(\textbf{x},\textbf{y})
\end{align*}
So the general avg 
\begin{align*}
    \avg{\delta_i w_{ij}w_{ij}}
    &=\int \nstavg{w_{ij}w_{ij}} P_{nst}(\textbf{x},\textbf{y}) d\textbf{y}\\
    &= \int(\nstavg{u_iu_i}+ \nstavg{u_ju_j} - \nstavg{u_iu_j}- \nstavg{u_ju_i}) P_{nst}(\textbf{x},\textbf{y})d\textbf{y}\\
    &= \avg{\delta_i u_iu_i} 
    + \avg{\delta_i u_ju_j}
    - \avg{\delta_i u_iu_j}
    - \avg{\delta_i u_ju_i}
\end{align*}
The last term is interesting indeed, 
\begin{align*}
    \avg{\delta_i u_ju_i}
    =\iint \delta_i \delta_j u_ju_i h_{ij} d\mathscr{P} d\textbf{y}
\end{align*}

so we can arrive to the conclsion 
\begin{align*}
    2\avg{\delta_i u_iu_i} 
    =\avg{\delta_i w_{ij}w_{ij}}
    + 2\avg{\delta_i u_iu_j}
\end{align*}
or 
\begin{align}
    \pavg{u_iu_i}
    = \frac{1}{2}\pavg{w_{ij}w_{ij}}
    + \pavg{u_iu_j}
\end{align}

\subsection*{Yet Another transport}
Let $\textbf{y} = \textbf{x} + \textbf{r}$, then $\delta_j(\textbf{x}_j - \textbf{x}-\textbf{r})$. 
Remark that 
$\pddt \delta_j(\textbf{x}_j - \textbf{x}-\textbf{r}) 
= \textbf{u}_j \cdot \partial_{\textbf{x}_j} \delta_j(\textbf{x}_j - \textbf{x}-\textbf{r})
= - \textbf{u}_j \cdot  \partial_{\textbf{r}}   \delta_j(\textbf{x}_j - \textbf{x}-\textbf{r})
= - \textbf{u}_j \cdot  \partial_{\textbf{x}}   \delta_j(\textbf{x}_j - \textbf{x}-\textbf{r})
$
Therefore by adding and substracting ,
\begin{align*}
    \pddt \delta_i
    + \textbf{u}_i \cdot \partial_{\textbf{x}}    \delta_i
    &= 0 \\
    \pddt \delta_j
    + (\textbf{u}_j - \textbf{u}_i) \cdot \partial_{\textbf{r}}    \delta_j 
    + \textbf{u}_i \cdot \partial_{\textbf{x}}    \delta_j 
    &= 0 
\end{align*}
Then multiplying the second eq by $\delta_i$ 
\begin{equation*}    
    \pddt (\delta_i\delta_j)
    + \partial_{\textbf{r}} \cdot  (\textbf{w}_{ij} \delta_j \delta_i )
    +\partial_{\textbf{x}} \cdot (\textbf{u}_i\delta_j \delta_i )
\end{equation*}
multipliying by $h_{ij}$ gives
\begin{equation*}    
    \pddt (\delta_i\delta_j h_{ij})
    +\partial_{\textbf{x}} \cdot (\textbf{u}_i\delta_i \delta_j h_{ij} )
    + \partial_{\textbf{r}} \cdot  (\textbf{w}_{ij} \delta_i \delta_j h_{ij} )
    = \delta_i\delta_j \pddt h_{ij}
\end{equation*}
multipliying by an arbitratry qte $q_{ij}$ gives
\begin{equation*}    
    \pddt (\delta_i\delta_j h_{ij}q_{ij})
    +\partial_{\textbf{x}} \cdot (\textbf{u}_i\delta_i \delta_j h_{ij}q_{ij} )
    + \partial_{\textbf{r}} \cdot  (\textbf{w}_{ij} \delta_i \delta_j h_{ij}q_{ij} )
    = 
    \delta_i\delta_j q_{ij}\pddt h_{ij}
    + \delta_i\delta_j h_{ij}\pddt q_{ij}
\end{equation*}
\subsubsection*{other interenal properties}
Let $\delta_p = (p - p_\alpha(t))$ be a function of state evolving according to, 
\begin{equation*}
    \pddt \delta_p
    + \textbf{u}_p \cdot \partial_{\textbf{x}}    \delta_p
    = 0 
\end{equation*}
where $\textbf{u}_p = \ddt p_\alpha(t)$ using this last eq gives, 
\subsubsection*{Average}
Considering,
\begin{equation*}
    \nstavg{q_{ij}} P_{nst}(\textbf{x},\textbf{r})= \int\delta_i\delta_{ij} q_{ij} h_{ij} d\mathscr{P} 
\end{equation*}
Thus averaging the above eq gives, 
\begin{multline*}
    \pddt (\nstavg{q_{ij}} P_{nst}(\textbf{x},\textbf{r})) 
    + \partial_\textbf{x} \cdot ( \nstavg{q_{ij}\textbf{u}_i} P_{nst}(\textbf{x},\textbf{r}))\\
    + \partial_\textbf{r} \cdot ( \nstavg{q_{ij}\textbf{w}_{ij}} P_{nst}(\textbf{x},\textbf{r})) 
    = 
    \avg{\delta_i\delta_{j}q_{ij}\pddt h_{ij} }
    + \nstavg{\partial_t q_{ij}}P_{nst}
\end{multline*}
Integrating this equation along \textbf{r} gives, 
\begin{equation}
    \pddt \avg{\delta_i q_{ij}} 
    + \partial_\textbf{x} \cdot  \avg{\delta_i q_{ij}\textbf{u}_i}
    = 
    \avg{\delta_i \ddt q_{ij}}
    + \int \condavg{q_{ij}\pddt h_{ij}}{2}P_2d\textbf{r}
\end{equation}
The third term canceled du to theorem of divergence, and the last one probably would too 
This equation implies that $ \int\avg{\delta_i\delta_{ij}q_{ij}\pddt h_{ij} }d\textbf{r} =0 $  ? or does it ??
Yes it must bu null\ldots



\subsubsection*{First moments}
Let multiply by \textbf{r}. 
\begin{multline*}
      \pddt (\delta_i\delta_{j}q_{ij}\textbf{r}h_{ij}) 
    + \partial_\textbf{x} \cdot (\textbf{u}_i \textbf{r}\delta_i\delta_{j}q_{ij}h_{ij})
    + \partial_\textbf{r} \cdot (\textbf{w}_{ij} \textbf{r}\delta_i\delta_{j}q_{ij}h_{ij}) \\
    =  \textbf{w}_{ij} \delta_i\delta_{j}q_{ij}h_{ij}
    + \delta_i\delta_{j}\textbf{r}\pddt q_{ij}h_{ij} 
    + \delta_i\delta_{j}q_{ij}\textbf{r}\pddt h_{ij} 
\end{multline*}
averaging on $\iint \ldots d\mathscr{P}d\textbf{y}$ gives, 
\begin{equation*}
      \pddt \avg{\delta_iq_{ij}\textbf{r}} 
    + \partial_\textbf{x} \cdot \avg{\delta_i\textbf{u}_i \textbf{r}q_{ij}}
    = \avg{\delta_i\textbf{w}_{ij}q_{ij}}
    + \avg{\delta_i\textbf{r}\pddt q_{ij}} 
    + \iint\delta_i\delta_{j}q_{ij}\textbf{r}\pddt h_{ij} d\mathscr{P}d\textbf{r}
\end{equation*}
\subsection*{Position :}
Let $\textbf{r} = \textbf{x}_j - \textbf{x}_i$, thus, 
$\ddt \textbf{r} = \ddt (\textbf{x}_j - \textbf{x}_i) = \textbf{u}_j - \textbf{u}_i$. 
Thus injecting $q_{ij} =\textbf{r}$, 
\begin{equation*}
    \pddt \avg{\delta_i \textbf{r}\textbf{r}} 
  + \partial_\textbf{x} \cdot \avg{\delta_i\textbf{u}_i \textbf{r} \textbf{r}}
  = \avg{\delta_i\textbf{w}_{ij} \textbf{r}}
  + \avg{\delta_i\textbf{r} \textbf{w}_{ij}} 
  + \int \avg{\delta_i\delta_{j} \textbf{r}\textbf{r}\pddt h_{ij} } d\textbf{r}
\end{equation*}
\paragraph*{first moment}
Let $\mathbb{M}_1 = \avg{\delta_i \textbf{r}}$ then, 
\paragraph*{second moment}
Let $\mathbb{M} = \avg{\delta_i \textbf{r}\textbf{r}}$ then, 
$\avg{\delta_i\textbf{u}_i \textbf{r} \textbf{r}} 
= P_1\condavg{\textbf{u}_i \textbf{r} \textbf{r}}{1} 
= P_1\condavg{\textbf{u}_i}{1} \condavg{\textbf{r} \textbf{r}}{1} 
+ P_1\condavg{\textbf{u}_i' (\textbf{r} \textbf{r})'}{1}
=\condavg{\textbf{u}_i}{1} \mathbb{M} 
+ \avg{\delta_i\textbf{u}_i' (\textbf{r} \textbf{r})'}$
Besides, 
\begin{equation*}
    \int \int\delta_i\delta_{j} \textbf{r}\textbf{r}\pddt h_{ij} d\mathscr{P} d\textbf{r}
    =P_1\condavg{\textbf{r}\textbf{r}\pddt h_{ij}}{1}
    = \mathbb{M}\condavg{\pddt h_{ij}}{1}
    +\avg{\delta_i(\textbf{r}\textbf{r})'(\pddt h_{ij})'}
\end{equation*}

Therefore the evolution of $\mathbb{M}$ yields, 
\begin{equation*}
    \pddt \mathbb{M} 
  + \partial_\textbf{x} \cdot (\mathbb{M} \condavg{\textbf{u}_i}{1}
  + \avg{\delta_i\textbf{u}_i' (\textbf{r} \textbf{r})'})
  = 
   \mathbb{S}
  + P_1\condavg{\textbf{r}\textbf{r}\pddt h_{ij}}{1}
\end{equation*}
The last term on the RHS represent the mean square distance at which particle switch nearest neighbors, $\mathbb{S} = \avg{\delta_i\textbf{w}_{ij} \textbf{r}}
+ \avg{\delta_i\textbf{r} \textbf{w}_{ij}} $



\subsubsection*{Momentum :}
Let's $\textbf{p}_{ij} = \textbf{m}_j \textbf{u}_j-\textbf{m}_i \textbf{u}_i =\int_{\Omega_j}\rho_2 \textbf{u}_2d\Omega
- \int_{\Omega_i}\rho_2 \textbf{u}_2d\Omega$, 
\begin{equation}
\ddt \textbf{p}_{ij} 
= \int_{\Omega_j}\rho_2\textbf{g}d\Omega 
- \int_{\Omega_i}\rho_2\textbf{g}d\Omega
+ \int_{\Sigma_j} \textbf{T}_1 \cdot \textbf{n}_2d\Sigma
-\int_{\Sigma_i} \textbf{T}_1 \cdot \textbf{n}_2d\Sigma
\end{equation}
\begin{equation}
\ddt \textbf{p}_{i} 
= \ddt (\textbf{m}_i \textbf{u}_i)
= \int_{\Omega_i}\rho_2\textbf{g}d\Omega
+ \int_{\Sigma_i} \textbf{T}_1 \cdot \textbf{n}_2d\Sigma
\end{equation}
Consequently, 
\begin{align*}
    \ddt \textbf{w}_{ij} 
    &= \frac{1}{m_j}\int_{\Omega_j}\rho_2\textbf{g}d\Omega 
    - \frac{1}{m_i}\int_{\Omega_i}\rho_2\textbf{g}d\Omega
    + \frac{1}{m_j}\int_{\Sigma_j} \textbf{T}_1 \cdot \textbf{n}_2d\Sigma
    - \frac{1}{m_i}\int_{\Sigma_i} \textbf{T}_1 \cdot \textbf{n}_2d\Sigma\\
    &=\textbf{B}_{ij} + \textbf{F}_{ij}
\end{align*}
and, 
\begin{equation}
    \ddt \textbf{u}_{i} 
    = \frac{1}{m_i}\int_{\Omega_i}\rho_2\textbf{g}d\Omega
    + \frac{1}{m_i}\int_{\Sigma_i} \textbf{T}_1 \cdot \textbf{n}_2d\Sigma
    = \textbf{B}_i + \textbf{F}_i
\end{equation}
using the avg procedure 
\begin{equation*}
    \pddt \avg{\delta_i\textbf{w}_{ij}\textbf{r}} 
  + \partial_\textbf{x} \cdot \avg{\delta_i\textbf{u}_i \textbf{r}\textbf{w}_{ij}}
  = \avg{\delta_i\textbf{w}_{ij}\textbf{w}_{ij}}
  + \avg{\delta_i\textbf{r} \textbf{B}_{ij}} 
  + \avg{\delta_i\textbf{r} \textbf{F}_{ij}} 
  + \avg{\delta_i\delta_{j}\textbf{w}_{ij}\textbf{r}\pddt h_{ij} }
\end{equation*}
Let $\mathbb{P} = \avg{\delta_i \textbf{rw}_{ij}}$, then, 
\begin{equation*}
    \pddt \mathbb{P} 
  + \partial_\textbf{x} \cdot (\condavg{\textbf{u}_i}{1} \mathbb{P}
  + P_1\condavg{\textbf{u}_i' (\textbf{r}\textbf{w}_{ij})'}{1})
  = \avg{\delta_i\textbf{w}_{ij}\textbf{w}_{ij}}
  + \avg{\delta_i\textbf{r} \textbf{B}_{ij}} 
  + \avg{\delta_i\textbf{r} \textbf{F}_{ij}} 
  + P_1\condavg{\textbf{w}_{ij}\textbf{r}\pddt h_{ij}}{1}
\end{equation*}

Now we consider the transport of the velocity, 
\begin{equation*}
    \pddt \avg{\delta_i\textbf{u}_{i}\textbf{r}} 
  + \partial_\textbf{x} \cdot \avg{\delta_i \textbf{r}\textbf{u}_i\textbf{u}_{i}}
  = \avg{\delta_i\textbf{u}_{i}\textbf{w}_{ij}}
  + \Sigma_{PFP}
  + \avg{\delta_i\delta_{j}\textbf{w}_{ij}\textbf{r}\pddt h_{ij} }
\end{equation*}
where we considered that $\avg{\delta_i\textbf{r} \textbf{B}_{i}} = 0 $ since $\textbf{B}_i$ is constant for a given particle and $\Sigma_{PFP} = \avg{\delta_i\textbf{r} \textbf{F}_{i}}$ 



Besides assuming $\textbf{w}_{ij} = \textbf{u}_j - \textbf{u}_i$, 
\begin{equation}
    \avg{\delta_i \textbf{ru}_{i}}
    = 
    \int{\textbf{r}\nstavg{\textbf{u}_{i}}} P_{nst}d\textbf{r}
    = \int{\textbf{r}\nstavg{\textbf{u}_{j}}} P_{nst}d\textbf{r}
    - \int{\textbf{r}\nstavg{\textbf{w}_{ij}}} P_{nst}d\textbf{r}
\end{equation}
If we decompose, $\textbf{r}\nstavg{\textbf{u}_{i}} = \nstavg{\textbf{u}_{i}}$
\subsection{Conditional particular shape}
\begin{multline*}
    \pddt (\nstavg{\mathcal{M}_{i}} P_{nst}) 
    + \partial_\textbf{x} \cdot ( \nstavg{\mathcal{M}_{i}\textbf{u}_i} P_{nst})\\
    + \partial_\textbf{r} \cdot ( \nstavg{\mathcal{M}_{i}\textbf{w}_{ij}} P_{nst}) 
    = 
    \condavg{\mathcal{M}_{i}\pddt h_{ij}}{2}P_2
    + \nstavg{\mathcal{S}_i}P_{nst}
\end{multline*}
\subsection{Conditional particular stress}

Evolution of the stresslet of an arbitratry dispersed phase without mass transfer. 
For a signgle partile, 
\begin{equation}    
     \ddt \mathcal{S}_i
    =  \int_{\Omega_i} \rho_2 \textbf{w}_2 \textbf{w}_2d\Omega
    - \int_{\Omega_i}\mathbf{T}_2d\Omega
    -  \textbf{M}_i^\sigma
    +   \textbf{S}_i
\end{equation}
\begin{multline*}
    \pddt (\nstavg{\mathcal{S}_{i}} P_{nst}) 
    + \partial_\textbf{x} \cdot ( \nstavg{\mathcal{S}_{i}\textbf{u}_i} P_{nst})
    + \partial_\textbf{r} \cdot ( \nstavg{\mathcal{S}_{i}\textbf{w}_{ij}} P_{nst}) \\
    = 
    \condavg{\mathcal{S}_{i}\pddt h_{ij}}{2}P_2
    + P_{nst}\left(
        \nstavg{\int_{\Omega_i} \rho_2 \textbf{w}_2 \textbf{w}_2d\Omega}
        - \nstavg{\int_{\Omega_i}\mathbf{T}_2d\Omega}
        - \nstavg{\textbf{M}_i^\sigma}
        + \nstavg{\textbf{S}_i}
    \right)
\end{multline*}

Integrating over $\textbf{r}$ : 
\begin{multline*}
    \avg{\delta_i \int_{\Omega_i}\mathbf{T}_2d\Omega}
    =
    \avg{\delta_i\mathcal{S}_{i}\pddt h_{ij}}
    - \partial_t \avg{\delta_i \mathcal{S}_i}
    - \partial_\textbf{x} \cdot \avg{\delta_i  \textbf{u}_i \mathcal{S}_i}\\
    + \avg{\delta_i\int_{\Omega_i} \rho_2 \textbf{w}_2 \textbf{w}_2d\Omega}
    - \avg{\delta_i\int_{\Omega_i}\mathbf{T}_2d\Omega}
    - \avg{\delta_i\textbf{M}_i^\sigma}
    + \avg{\delta_i\textbf{S}_i}
\end{multline*}

\begin{multline*}
    \pddt (\delta_i\delta_{j}q_{ij}\textbf{r}h_{ij}) 
  + \partial_\textbf{x} \cdot (\textbf{u}_i \textbf{r}\delta_i\delta_{j}q_{ij}h_{ij})
  + \partial_\textbf{r} \cdot (\textbf{w}_{ij} \textbf{r}\delta_i\delta_{j}q_{ij}h_{ij}) \\
  =  \textbf{w}_{ij} \delta_i\delta_{j}q_{ij}h_{ij}
  + \delta_i\delta_{j}\textbf{r}\pddt q_{ij}h_{ij} 
  + \delta_i\delta_{j}q_{ij}\textbf{r}\pddt h_{ij} 
\end{multline*}

\section*{Rate of coalesence }

the rate of coalescence assuming instantaneous film drainage and homogeneous diameter D is :
\begin{equation*}
    R^+ 
    = \int_{\textbf{r}<D}\delta_i(\textbf{x} - \textbf{x}_i)\delta_{j}(\textbf{x} + \textbf{r} - \textbf{x}_j) q_{ij} h_{ij} d\mathscr{P}  
\end{equation*} 



\section*{Particle stress evolution eq}
Let $\textbf{f}_i$ be the drag force applied on the particle $i$, then we can derive,
\begin{equation*}
    \pddt \avg{\delta_i \textbf{f}_i\textbf{r}} 
  + \partial_\textbf{x} \cdot \avg{\delta_i\textbf{u}_i \textbf{r} \textbf{f}_i}
  = \avg{\delta_i\textbf{w}_{ij} \textbf{f}_i}
  + \avg{\delta_i\textbf{r}\pddt  \textbf{f}_i} 
  + \int\avg{\delta_i\delta_{j} \textbf{f}_i\textbf{r}\pddt h_{ij} }d\textbf{r}
\end{equation*}
By making use of the particle stress definition, 
\begin{equation*}
    \pddt \Sigma_{PFP}
  + \partial_\textbf{x} \cdot \left(
    \condavg{\textbf{u}}{1} \Sigma_{PFP}
  \right)
  = \avg{\delta_i\textbf{w}_{ij} \textbf{f}_i}
  + \avg{\delta_i\textbf{r}\pddt  \textbf{f}_i} 
  + \int \avg{\delta_i\delta_{j} \textbf{f}_i\textbf{r}\pddt h_{ij}}d\textbf{r}
\end{equation*}

The issue addressed in this work is to find an appropriate method to model buoyancy-driven droplets suspensions and more generally disperse multiphase flow throughout statistical modeling.
Indeed, due to the multiscale physical phenomenon present in those flows it is rather difficult to model the classical governing equations of fluids mechanics.  

The development presented in this work can be summarized into 9 key points:
\begin{enumerate}
    \item In \ref{chap:daniel1} 
    we demonstrate the relation between the particle-averaged and phase-averaged equations. 
    We show that the dispersed phase-averaged equations can be interpreted as a series expansion of the particle-averaged moment equations. 
    The chapter concludes by presenting a ``hybrid'' set of equations, consisting of phase-averaged equations for the continuous fluid phase, complemented by an arbitrary number of moment conservation equations for the dispersed phase that includes interfacial properties of the droplets.
    \item In \ref{chap:daniel15} we expose the mass, momentum, and energy averaged equations, using the ``hybrid'' formalism, and discuss the energy exchanges present in an emulsion, we also provide an explicit and general formulation for the continuous and bulk phase effective stresses.
    \item In \ref{chap:deformable} we consider droplets with spheroidal geometry and show how to derive equations for the droplets mean deformation and rate of deformation tensor. 
    This leads to a set of equations that are similar to the classic model that describe the oscillation of the second mode of deformation of a droplet.
    \item     
    In \ref{chap:daniel2} we propose a generic routine to derive the \textit{single-particle conditional averaged} Navier-Stokes equations. 
    Then, we revisit the derivation of (2.10) of \citet{batchelor1972sedimentation} which relates continuous phase ensemble-averaged quantities to single-particle conditionally averaged quantities.
    Notably, we show that the assumptions made by \citet{batchelor1972sedimentation} to derive its formula are not sufficient to arrive at the actual expression given in his paper.
    This explains the non-converging issue sometimes encountered using this relation \citep{batchelor1972sedimentation}. 
    \item In \ref{chap:daniel2}, we re-derived most of the closures present in the hybrid model. 
    While many of these were already established \citep[Appendix A]{zhang1997momentum}, we introduced several new closures, specifically: 
    The pseudo-turbulent stress $\avg{\chi_f \textbf{u}_f'\textbf{u}_f'}$ generated by a mean shear flow $\textbf{E}_f$ on droplets; 
    The pseudo-turbulent kinetic energy transfer resulting from the local work on the surfaces of the droplets; 
    The continuous phase droplets induced dissipation $\avg{\chi_f \bm\sigma_f^0:\grad \textbf{u}_f^0}$; 
    The term representing the averaged viscous dissipation within the droplets volume. 
    The term representing the averaged kinetic energy within the droplets volume. 
    \item Then we propose an analysis of the droplets mean deformation and effective suspension stress, in the presence of mean relative motion between the phases, considering low but finite inertia effects. 
    With the reciprocal theorem we show that due to the effect of non-vanishing inertia, the \textit{Stresslet}  term present in the effective stress has a contribution proportional to the mean relative velocity. 
    Consequently, uniform relative translation between phases generates effective stresses through the \textit{Stresslet} term. 
    The \textit{Stresslet} term is then compared to the numerical results, good agreements are obtained. 
    This contribution is also compared to the \textit{Reynolds} stress term since it posses the same functional form.
    Both contribution are of the same order of magnitude in the moderately dense regime ($\phi =0.2$). 
    \item In \ref{chap:mono-disperse}, we propose a new drag force model taking into account the values of the viscosity ratio $\lambda$. 
    The model is built on already existing correlations valid at $\lambda\to\infty$ and $\lambda = 0$, it includes: the Richardson-Zaki relation, Schiller-Neuman, and Mei drag force coefficients.
    Then we show based on the DNS results that it is also valid at intermediate values of $\lambda$.
    The main advantage of this model is its robustness since the Richardson-Zaki relation is valid at very high volume fractions  ($\phi \approx 0.5$) and arbitrary Reynolds numbers, while in the dilute limit the Schiller-Neuman, and Mei drag force coefficients are proven to be accurate up to $Re = 800$. 
    Thus, we provided a robust drag force coefficient that can directly be used in Euler-Euler frameworks for simulations of emulsions of arbitrary $\lambda$.   
    \item In \ref{chap:pseudoturbulence} we derive an analytical formula for the \textit{Reynolds} stress tensor in the low inertia and dilute regime for an arbitrary viscosity ratio $\lambda$. 
    Then we used the numerical results (see \ref{chap:DNS}) to extend the validity of this model to arbitrary $Re$ and $\phi$. 
    Good agreements are obtained by comparing our model to the present model and experimental results of the literature for solid particles and bubbles.  
    \item Finally, in \ref{chap:microstructure,chap:microstructure_kin} we provide a methodology to characterize the microstructure geometry and microstructure kinematics of a buoyant emulsion. 
    The geometry is characterized using the nearest-neighbor distribution, which is evaluated as a function of the dimensionless parameters. 
    % We also developed tools characterizing the 
    Then, we determine the relaxation time that it takes for the microstructure (characterized by the nearest-neighbor distribution) to research its stationary equilibrium geometry, and describe the kinematics of interaction between pairs of droplets. 
\end{enumerate}

We would like to end this conclusion by presenting what we believe is the most minimalistic averaged model to describe the momentum of a buoyancy-driven droplets suspensions. 
Assuming that the kinetic energy of the continuous phase follows a quasi-steady equilibrium, we may write the averaged momentum conservation of the continuous phase as, 
\begin{align}
    % &\pddt (\phi_f \rho_f)  
    % + \div (
    %     \phi_f \rho_f\textbf{u}_f
    % )
    % = 
    % 0,\\
    % \pddt (\phi_f\rho_f \textbf{u}_f)
    % + \div (\phi_f\rho_f \textbf{u}_f\textbf{u}_f)
    \phi_f\rho_f (\pddt + \textbf{u}_f\cdot \grad)\textbf{u}_f
    = \phi_f 
    \left(\div \bm{\Sigma}_f
    + \rho_f \textbf{g}\right)
    + \div  \bm{\sigma}_f^{\text{eff}}
    - n_p \textbf{f}_p,
    \label{eq:momentum}
\end{align}
With, 
\begin{align}
    n_p \textbf{f}_p  
    &= 
    \underbrace{f(Re,\phi, \lambda) \times \textbf{u}_{fp}}_\text{Drag force}
    \label{eq:draggggg}
    \\
    \label{eq:newtonian}
  \bm\Sigma_f &= - \underbrace{p_f \bm\delta + \mu_f [\grad \textbf{u}_f +  (\grad \textbf{u}_f)^\dagger ]}_\text{Averaged newtonian stress} 
  \\
    \bm{\sigma}^{\text{eff}}_f 
    &= \underbrace{ C_E(Re,\phi,\lambda)  [\grad \textbf{u}_f +  (\grad \textbf{u}_f)^\dagger ] }_\text{``Einstein viscosity''-like contributions}
    \nonumber\\
    &+ 
    \underbrace{
      C_1(\phi,\lambda,Re)\textbf{u}_{fp}\textbf{u}_{fp}
      +  C_2(\phi,\lambda,Re)(\textbf{u}_{fp}\cdot \textbf{u}_{fp})     \bm\delta}_\text{Particle Induced Turbulence and Stresslet contributions}
    \label{eq:stress}
\end{align}
Where \ref{eq:momentum} is to be complemented by the dispersed phase momentum equation and the corresponding mass conservation equations. 

In this case, the dominant contribution of the drag force term is its component proportional to the relative velocity $\textbf{u}_{fp}$. 
A model for the scalar function $f$ is given in \ref{chap:mono-disperse}. 
The scalar $C_E$ represents the contribution of the \textit{Stresslet} term which can be interpreted as an additional coefficient to the continuous phase viscosity as it is proportional to the mean rate of strain of the continuous phase (see \ref{chap:closure-disperse}).
The coefficients $C_1$ and $C_2$ are more complex. 
They represent the contribution of \textit{Stresslet} term (see \ref{chap:closure-disperse}) and also of the Reynolds stress tensor (see \ref{chap:pseudoturbulence}). 



As seen in \ref{chap:closure-disperse} (see also \ref{ap:momentum_formulation}) the formulation given by \ref{eq:draggggg,eq:stress} does not include all the contributions of the closure terms. 
Indeed, $\textbf{f}_p$ and $\bm\sigma_f^\text{eff}$ should also take into account the effect of relative acceleration between phases (added mass effects), the effects of non-homogeneity, etc.
More generally these terms should include all the other physical contributions that were not taken into account in the steady-state, mono-disperse, and homogeneous flow situation. 
Nevertheless, we state that \ref{eq:draggggg,eq:stress} constitute the most minimalistic approach to model buoyancy-driven droplets suspensions. 
Because flotation drives a significant part of the physics in our processes, the dependency on the relative velocity $\textbf{u}_{fp}$ in the effective stress is crucial to model the rheology of the emulsions.
Consequently, these terms must be included in the Euler-Euler simulation codes going forward.


\chapter*{Future investigations}




\paragraph*{Pragmatic points: }
Firstly,  we would like to point out the points that we know to be necessary for the modeling strategy of the processes presented in \ref{part:intro}.
\begin{enumerate}
    \item One must include the mass transfer phenomenon.
    The averaged equations can directly be derived from \ref{chap:daniel1}.
    Once the relevant closure terms are identified  they can be determined and modeled using DNS \citep{hidman2023assessing}. 
    \item In the flotation process, we have a third phase in the problem, this phase has to be included in the hybrid formulation. 
    \item One last point that we know to be relevant is the consideration of Population-Balence-Equaitons, implying that we must find closure for these models as well and extend the current closure terms to poly-disperse situations. 
    This implies studying the fluid drainage problem presented in \ref{part:intro} and developing accurate coalescence kernels. 
    \item In \ref{chap:microstructure_kin} we have studied the kinematics of the microstructure. 
    As mentioned in this chapter to go further it is necessary to study the dynamic of interaction to quantify relative forces between droplets. 
    \begin{figure}[h!]
        \centering
        \includegraphics[width=0.3\textwidth]{image/HOMOGENEOUS_final/Dist/F_rel_l_1_Ga_80_PHI_5}
        \caption{Averaged force vector applied on the droplet present at the origin, conditioned on the presence of a nearest neighbor.
        The radial distance is made dimensionless by the droplets' diameter. 
        The color map represents the magnitude of the forces.}
        \label{fig:perspective_forces}
    \end{figure}
    On \ref{fig:perspective_forces} we display a result regarding the interaction forces statistic. 
    One can remark that \ref{fig:perspective_forces} is a visual representation of the particle-fluid-particle stress (as defined in \citep{zhang2021ensemble}). 
    \item The closure terms of the disperse phase, specifically the particles center of mass velocity fluctuations $\pavg{\textbf{u}_\alpha'\textbf{u}_\alpha'}$, remain to be closed. 
    Using a combination of nearest-particle statistics and reflection method it is possible to derive an analytical model in the Stokes and dilute regime. 
    The methodology is very similar to what is done in \citet{zhang2021ensemble} to compute the sedimentation velocity of non-dilute suspension of spheres (without the renormalization method). 
    Following the same workflow as in \ref{chap:pseudoturbulence} and using \ref{eq:particle_center_of_mass_velocities} we found the expression: 
    \begin{equation*}
        \pavg{\textbf{u}_\alpha'\textbf{u}_\alpha'}
        % = 
        % n_p[\textbf{x},t]
        % \int_{\mathbb{R}^3}
        % (\textbf{v}^\text{nst}_p
        % \textbf{v}^\text{nst}_p)[\textbf{x},\textbf{y},t]
        % P_\text{nst}[\textbf{y}|\textbf{x},t]
        % d\textbf{y}
        = 
        C_1[\textbf{u}_{fp}\textbf{u}_{fp} - \frac{1}{3}(\textbf{u}_{fp}\cdot \textbf{u}_{fp})\bm\delta] 
        + C_2(\textbf{u}_{fp}\cdot \textbf{u}_{fp})\bm\delta,
    \end{equation*}
    \begin{align}
      C_1 = \frac{1}{960}\left(\frac{2+3\lambda}{\lambda+1}\right)^2 \left[
        108\Gamma(1/3)
        - 80\Gamma (4/3)
        +15\Gamma(7/3)
      \right]\phi^{2/3}\\
      C_2 = \frac{1}{576}\left(
        \frac{2+3\lambda}{\lambda+1}\right)^2 \left[
        24\Gamma(1/3)
        - 16\Gamma (4/3)
        +3\Gamma(7/3)
      \right]\phi^{2/3}.
    \end{align}
    For solid particle ($\lambda=\infty$) we find $\sqrt{k_p} = 1.52\phi^{1/3}$, while the experimental results reported in \citet{guazzelli2011fluctuations} suggest $\sqrt{k_p} = 2\phi^{1/3}$ and $3\phi^{1/3}$. 
    The first results are promising, more work on this topic is needed.
    \item Another point of major importance not presented in this manuscript is the study of particle-fluid-particle stress, or long range interaction stress. 
    According to \cite{Lhuillier_2009,nott2011suspension,zhang2021ensemble} the drag force term can be expressed as a pure drag force term (that is the closure provided in \ref{chap:mono-disperse}) plus the divergence of a stress, the so-called particle-fluid-particle stress. 
    The latter stress contributes (with $\pavg{\textbf{u}_\alpha'\textbf{u}_\alpha'}$), to the dispersed phase momentum fluxes.  
    At this stage it is hard to estimate if this stress is more or less important than the center of mass velocity variance, anyhow studies remain to be done on this topic. 
\end{enumerate}

\paragraph*{``Might be necessary''-points: }
Then there are the projects that might be relevant to this work, 
however, at this stage, it is hard to estimate if they are all essential.
\begin{enumerate}
    \item In \ref{chap:closure-disperse} we studied the dependence of the first moment of forces on mean relative motions.
    It seems important to develop a model that is more accurate for higher Reynolds number and volume fraction since it could be predominant for the rheology of dense emulsion. 
    \item The study of the modeling of the second moment of the hydrodynamic forces seems important as well. 
    \item Finally, all of our closure terms consider a steady-state situation, meaning that we neglect the contribution from relative phase acceleration. 
    For the drag force we know that the added mass effect cannot be neglected. 
    For the other moments of forces, this kind of contribution must be determined as well. 
    \item Up to now we considered homogeneous rising emulsion, however, we may take into account the gradient of droplets concentration in all of our closure terms. 
    The recent study of \citet{wang2024effect} started to include the gradient of droplet volume fraction $\grad\phi$ in the drag force closure for example.  
    \item PR-DNS can be quite expansive to develop closure terms. 
    Thus, one might consider solving the single-particle conditioned averaged equations to produce closure.  
    Indeed, as demonstrated by \citet{hinch1977averaged} by considering closure within those equations one can derive more complete closure terms, such as the second-order correction of the equivalent stress and sedimentation velocity in his case. 
    Nevertheless, this approach is theoretically difficult, limiting the number of problems that can be treated. 
    Solving this equation using a numerical approach would enable us to consider more complicated scenarios. 
    For example, we could prescribe a given pair distribution function in the equations; it would represent the mean gradient of volume fraction, layers in the microstructure etc. 
    This would lead to closure models in terms of that pair-distribution. 
\end{enumerate}


    
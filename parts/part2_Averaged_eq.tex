
\section{Averaged equations for two-phase flows.}
\label{sec:introavg}

As stated in the introduction of this chapter we now present the continuous averaged equations for multiphase flow. 
It is important to keep in mind that up to now and all along this section we do not assume anything on the topology of the phase.


\subsection{The averaging concepts}

We start this section by presenting the most common technics of averages and their operators. 
Therefore, we note $\left<f\right>(\textbf{x})$ the average of an arbitrary quantity $f(\textbf{y})$, 
where $\textbf{y}$ is the spacial coordinate in the laboratory reference frame, 
and $\textbf{x}$ macro scale coordinate at which we evaluate the average.
Therefore, we define the \textbf{volume average} operator such as,
\begin{equation}
    \left<f\right>(\textbf{x},t) = \int g(\textbf{x},\textbf{y}) f(\textbf{y},t)dV,
    \label{eq:avg}
\end{equation}
where $g(\textbf{x},\textbf{y})$ is the smoothing (or weighting) function introduced by 
\citet{jackson1997locally,marle1982macroscopic}.
The first argument, $\textbf{x}$, is location at which we take the average, and \textbf{y} is the variable of integration or the local coordinate vector.
The smoothing function $g(\textbf{x},\textbf{y})$ must follow two properties, the first one is normalization to unity 
$\int g(\textbf{x},\textbf{y}) dV = 1 \;\forall g$.
The second one is that $g$ vanish for all $\textbf{y}$ far form $\textbf{x}$, thus $\lim\limits_{|\textbf{r}| \to \infty} g(\textbf{x},\textbf{y}) = 0$ with $\textbf{r} = \textbf{x} - \textbf{y}$.
Also, the radius $R$, of the weighting function $g$, it is defined as $1/2 = \int_{|\textbf{r}|<R} g(\textbf{x},\textbf{y})dV$.
Those characteristics ensure that the integral of \ref{eq:avg} is convergent and well normalized. 

All Eulerian quantities are function of time also. 
Thus, we can average a quantity $f$ on a period of time $T$. 
Therefore, the \textbf{time average} operator is defined such as (see \citet{morel2015mathematical,drew1983mathematical,ishii2010thermo}), 
\begin{equation*}
    \left<f\right>(\textbf{x},t) = \frac{1}{T}\int_{t-T}^t f(\textbf{x},t')dt',
\end{equation*}
with $t$ the time and $t'$ the variable of integration.
Notice that the location vector \textbf{x} is the same after and before averaged $f$ since we average over the time dependency of $f$ therefore we use a unique position vector.  

Next, we introduce the ensemble average. 
This averaging technics has been introduced in the context of averaged dispersed 
two-phase flows by \citet{batchelor1972sedimentation,hinch1977averaged} and latter by \citet{zhang1994averaged}.
It is based on statistical approach and defined as follows. 
Let $\mathscr{C}_N$ be a set of parameters describing a configuration of a dispersed two phase flow. 
We then define $P(\mathscr{C}_N,t)$ as the probability density function of being in the configuration $\mathscr{C}_N$ at the time $t$. 
Then for any quantity $f(\mathscr{C}_N,\textbf{x},t)$ of the flow, we can define the \textbf{ensemble average} operator as
\begin{equation}
    \label{eq:enselblea}
    \avg{f}(\textbf{x},t) 
    = \int f(\mathscr{C}_N,\textbf{x},t) P(\mathscr{C}_N)d\mathscr{C}_N,
\end{equation}
where the integration is on all the realization $\mathscr{C}_N$. 
This method is somewhat more general because it allows averaging over an entire 
phase-space of parameters, unlike the two previous methods. 
This averaging technics is widely used in kinetic theory or in turbulence while deriving the Maxwell equation  for example, see \citet[Chapter 7]{rao2008introduction}.  

In short, the first method focus on averaging over the timescale, the second over a volume and the third method over all or a set of configuration of the flow.
While, the latter method and the Volume averaged method seem different in a lot of ways it has been shown that there are strictly equivalent \citet{jackson1997locally,zhang1994ensemble}.
In what follow we apply the volume average method to derive the averaged equation of motion, as we think it is more instinctive.
It is important to notice that in any averaging process we must respect the separation of 
scale. 
Consequently, we must respect the mathematical constraints, $D\ll R\ll \mathcal{L}$ with $\mathcal{L}$ the size of the domain, i.e. the scale of the vessel in the industrial context of the process. 
Moreover, it is interesting to mention some mathematical properties shared by all the averaging operators. 
For two arbitrary Eulerian fields $f$ and $h$ we have,
\begin{align}
    &\avg{f+h} = \avg{f}+\avg{h}, 
    &\avg{\avg{f}h} = \avg{f}\avg{h}, \nonumber \\
    &\avg{\frac{\partial f}{\partial t}} 
    = \pddt\avg{f}, 
    &\avg{\frac{\partial f}{\partial x_i}}
    = \frac{\partial}{\partial x_i}\avg{f}. 
    \label{eq:avg_properties}
\end{align}
The two first relations are called the Reynolds' rules, the $3^{th}$ one is the Leibniz' 
rule and the last one, the Gauss' rule \citep{drew1983mathematical}.
Besides, it is important to notice that the derivative operator of the last equation change of differentiation variable, from \textbf{y} to \textbf{x}, in the context of volume average.
In the following, we use the notation $\nablab$ to represent the global gradient operator, $\frac{\partial}{\partial \textbf{x}}$.
Thus, this last property can be re-written as, $\avg{\nablabh f}= \nablab \avg{f}$ in the volume average context \citep{jackson1997locally}.

Now we introduce the subclass operators, or conditional average operators.
The average of the quantity $f$ considering only the volume of the $k^{th}$ phase, will be defined, such that, 
\begin{equation}
    \phi_k \kavg{f} 
    = \avg{\chi_k f_k},
    \label{eq:avg_k_phase}
\end{equation}
where $\phi_k$ is the volume fraction of the phase $k$ at \textbf{x} and $\kavg{\ldots}$ is the conditional average operator on the phase $k$.
It can be obtained by substituting $f$ by $1$ in \ref{eq:avg_k_phase}, it reads,
\begin{equation*}
    \phi_k  
    = \avg{\chi_k}
\end{equation*}
\tb{We recall here that as all the global average operators are all equivalent, thus we adopt here the general notation $\avg{\ldots}$ to refer to a kind of global average.  }
Additionally, from the previous averaging operators, we can define the mass-weighted average operator by
\begin{equation}
    \avg{\rho} \mavg{f} 
    = \avg{\rho f}
    = \sum_k \phi_k \rho_k\kavg{f}.
\end{equation}

Having established averages operators over the volume of both phases independently, we now define one last conditional averaging operator.
Namely, the surface phase average operator,  
\begin{equation}
    a_I\Iavg{f} 
    = \avg{\delta_I f},
    \label{eq:avg_I_phase}
\end{equation}
where $a_I$ is the interfacial area concentration.  
$a_I$ can be thought of the ratio between the surface of the interfaces over the volume where the interface are included. 
It is defined by, 
\begin{equation}
    a_I
    = \avg{\delta_I}.
\end{equation}

Now, we would like to emphasize that the quantities inside phase average, are implicitly defined inside the $k^{th}$ phase so that it avoid redundancies. 
Therefore, $\kavg{f} = \kavg{f_k}= \avg{f_k\chi_k}$.
At the interface however it is not the case, indeed by definition, at the interface the quantities are defined in both phases.
Thus, we adopt the following convention, when averaging an equation on the volume of phase $k$, $\Iavg{f_k}$ will be the interfacial average of $f_k$, with $f_k$ being the quantity $f$ defined on the phase $k$. 
However, $\Iavg{f}$ without the subscript $_k$, will refer to the interface average of the quantity $f$ defined on the neighboring phase of $k$. 

From the \textit{single-fluid} and \textit{two-fluid} equations exposed previously, one can  notice that to obtain an averaged equation on the volume of the $k$ phase two choices are available. 
The first one and more logical choice is to apply the global average on the \textit{two-fluid} formulation equations.
Indeed, half the work is already done since our quantities are already product of $\chi_k$ and the average operators commute with the derivatives.
The other way is to take the conditional average on the volume of the $k$ phase on the \textit{single-fluid} formulation. 
Which implies re-doing some algebra which were already done to get this form of the 
equation.
Anyhow, those two methods are of course strictly equivalent.
In the next few sections we follow the same routine as the previous one and expose the conservation equations in both form but under the averaged form. 


\subsection{The averaged topological equations}

We start by averaging the PIF transport equation, or \ref{eq:phaseindicator_transport}. 
Using the phase average operator \ref{eq:avg_k_phase}, we can easily show that
\begin{equation}
    \pddt \phi_k 
    + \nablab \cdot \left(
        \kavg{\textbf{u}}\phi_k 
    \right) 
    = a_I\Iavg{(\textbf{u}_k-\textbf{u}_I) \cdot \textbf{n}_k}, 
\end{equation}
which correspond to the transport equation of the volume fraction $\phi_k$ advected by the mean velocity fields $\kavg{\textbf{u}}$.
The term on the RHS is the interface average of the volume exchange term. 
We will see in the next subsection that this equation is strictly equivalent to the averaged mass balance divided by the density.

Regarding the averaged interfacial transport equation, we can obtain it by averaging \ref{eq:interface_transport}, yielding
\begin{equation}
    \pddt a_I
    + \nablab \cdot \left(
        a_I
        \Iavg{\textbf{u}}
    \right)
    = - a_I \Iavg{\kappa \textbf{u}_I \cdot \textbf{n}}
    \label{eq:avg_I_a}
\end{equation}
where this equation correspond to the transport equation of the interfacial concentration $a_I$, along the interfacial averaged velocity field $\Iavg{\textbf{u}}$.

\subsection{Average of an arbitrary conservation equation}

From the \ref{eq:two-fluid_global} and the phase average operator definition (\ref{eq:avg_k_phase}) we can show that the phase averaged conservation equation of an arbitrary quantity $f_k$ reads as, 
\begin{equation}
    \pddt (\phi_k\kavg{f})
    = \nablab \cdot \left(
        \phi_k \kavg{\bm{\Phi} - f \textbf{u}}
    \right)
    + \phi_k \kavg{\textbf{S}}
    + a_I \Iavg{
        \bm{\Phi}_k \cdot \textbf{n}_k
        + f_k 
        \left(
            \textbf{u}_I
            - \textbf{u}_k
        \right) \cdot \textbf{n}_k
    },
    \label{eq:avg_k_global}
\end{equation}
where we have used the average operator properties \ref{eq:avg_properties}.
Likewise, the jump condition can be averaged too, using \ref{eq:avg}, yielding,
\begin{equation}
    \sum_k 
    \Iavg{
        \bm{\Phi}_k 
        \cdot \textbf{n}_k
        + f_k 
        \left(
            \textbf{u}_I
            - \textbf{u}_k
        \right) 
        \cdot \textbf{n}_k
    }
    = \Iavg{\textbf{J}_I}.
    \label{eq:avg_general_jump}
\end{equation}
So as the microscopic jump condition this equation maintains the consistency between the different averaged quantity in presence. 
Similarly, the bulk or global averaged conservation equation can be obtained averaging \ref{eq:single-fluid_global} yielding,
\begin{equation*}
    \pddt \avg{f}
    = \nablab \cdot \avg{\bm{\Phi} - f \textbf{u}}
    + \avg{\textbf{S}}
    + a_I\avg{\textbf{J}_I}.
    \label{eq:avg_global}
\end{equation*}


\subsection{The averaged mass balance equations}

We first consider the mass conservation. 
By applying the phase average operator (\ref{eq:avg}) to the mass conservation (\ref{eq:two-fluid_mass}) one obtain the mass balance of the $k^{th}$ phases, 
\begin{equation}
    \pddt \phi_k 
    + \nablab \cdot \left(
        \kavg{\textbf{u}}\phi_k 
    \right) 
    = \frac{a_I}{\rho_k}\Iavg{M_k},
    \label{eq:avg_k_mass}
\end{equation}
Notice that it correspond to the transport equation of the mass fraction of the 
$k^{th}$ phase $\rho_k \phi_k$. 
The source term due to mass transfer can also be expressed though the jump condition, 
\ref{eq:mass_jump} as, $a_I\Iavg{M_k} = - a_I\Iavg{M}$, where we recall that $M$ is the mass transfer term of the neighboring phase such as defined by our notation convention adopted in the previous section. 
We can notice that taking the sum of \ref{eq:avg_k_mass} for every phase $k^{th}$ and using the jump condition \ref{eq:mass_jump}, gives us and equation for the bulk density,
\begin{equation}
    \pddt \avg{\rho}
    + \nablab \cdot 
        \avg{\rho\textbf{u}} 
    = 0,
    \label{eq:avg_mass}
\end{equation}
which is also the global averaged equation of the \textit{single-fluid} formulation of the mass balance. 
Notice that it is possible to uncorrelated $\rho$ and $\textbf{u}$ under the global average operator by using the mass weighted average operator, indeed $\avg{\rho\textbf{u}} = \avg{\rho}\mavg{\textbf{u}}$.
Therefore, the mean density is transported by the mass weighted average velocity. 

\subsection{The averaged momentum balance equations}

The averaged momentum conservation equation on the $k$ phase is obtain by applying 
\ref{eq:avg} on \ref{eq:one-fuild_momentum}. 
It directly gives, 
\begin{equation}
    \pddt (\phi_k\kavg{\rho\textbf{u}}) 
    % + \nablab\cdot(\phi_k\kavg{\textbf{uu}})
    = \nablab\cdot\left(
        \phi_k \kavg{\textbf{T}
        - \rho\textbf{uu}}
    \right)
    +\phi_k\kavg{\textbf{b}}
    + a_I\Iavg{M_k \textbf{u}_k +\textbf{n}_k\cdot\textbf{T}_k},
\end{equation}
Notice that the term including the density, are both average on the volume of the phase $k$.
Thus, the density can be taken out the average such as, $\kavg{\rho \textbf{u}} = \rho_k\kavg{\textbf{u}}$ since $\rho$ is constant in each phase. 
Also, the stress and mass transfer terms inside the interfacial term at the RHS of the equation, are both expressed in their own phase as a consequence of the subscript $_k$.
Usually the drag force term is expressed as the force applied on the interface due to the stress from the neighboring phase. 
Similar thinking goes for the mass transfer term. 
Then to introduce such quantities we make use of the jump condition \ref{eq:stressjump}. 
Applying the preceding remarks on the previous equation yields,
\begin{equation}
    \pddt \left(
        \phi_k\kavg{\rho\textbf{u}}
    \right)
    = \nablab\cdot\left(
        \phi_k \kavg{
            \textbf{T}
            -\rho \textbf{uu}}
    \right)
    +\phi_k\kavg{\textbf{b}}
    + a_I\Iavg{
        \textbf{f}_I 
        - M \textbf{u} 
        - \textbf{n}\cdot\textbf{T}
    }.
    \label{eq:avg_k_momentum}
\end{equation}
We recall that by dropping the indices on the interfacial terms $M$, $\textbf{n}$ and $\textbf{T}$ refer to the mass transfer stress tensor defined in the neighboring phase. 
Also, the advantage of \ref{eq:avg_k_momentum} is that we clearly see the action of the surface tension force $\textbf{f}_I$ explicitly.

Applying \ref{eq:avg} on \ref{eq:two-fuild_momentum} gives the bulk average of the momentum equation, namely,
\begin{equation}
    \pddt 
        \avg{\rho\textbf{u}}
    % +  \nablabh \cdot \left(
    %     \avg{\rho \textbf{u} \textbf{u}}
    % \right)
    = \nablab \cdot \avg{
        \textbf{T}
        -  \rho \textbf{u} \textbf{u}} 
    + \avg{\textbf{b}}
    + \Iavg{\textbf{f}_I}.
    \label{eq:avg_momentum}
\end{equation}
In this equation it is interesting to notice that the surface jump force appear as a source term. 
On closed surfaces, i.e. droplets or bubbles surface, the resultant of the force $\textbf{f}_I$ cancel out as a result of the divergence theorem \citep{tryggvason2011direct}. 
Nonetheless, the averaged quantity $\Iavg{\textbf{f}_I}$ is not null, despite the fact that we consider solely closed surfaces.
Indeed, it can be shown that this force has the form 
\begin{equation*}
    \avg{\textbf{f}_I}
    = \nabla \cdot \textbf{M}_I + \mathcal{O}(D^2/R^2),
\end{equation*} 
with $\textbf{M}_I$ being a second order tensor that has the form of a stress. 
This point will be discussed in \ref{sec:hybrid_model} were we demonstrate that any averaged quantity can be express as a Taylor expansion series of higher order moments. 

\subsection{The averaged energy balance equations}

Likewise, we apply \ref{eq:avg} to \ref{eq:two-fuild_totenergy} to derive the phase averaged total energy equation.
Also, by making use of the jump condition (\ref{eq:total_energy_jump}), it is trivial to show, 
\begin{multline}
    \pddt \left(
        \phi_k\kavg{\rho E}
    \right)
    % + \rho_k \nablab \cdot \left(
    %     \phi_k\kavg{\textbf{u} E} 
    % \right)
    % \\
    = \nablab \cdot \left(
        \phi_k\kavg{\textbf{T}\cdot\textbf{u}
        - \textbf{q}
        - \rho E \textbf{u} }
    \right)
    + \phi_k\kavg{\textbf{b}\cdot\textbf{u}} \\
    +a_I \Iavg{
        \textbf{f}_I\cdot\textbf{u}_I
        + \left(
        \textbf{q}
        - \textbf{T}\cdot\textbf{u}
        \right) 
        \cdot\textbf{n}
        -M E}.
    \label{eq:avg_k_total_energy}
\end{multline}
The interfacial on the RHS represent, the heat flux directed inside the phase,
the work of the interfacial force from the external phase, the energy transfer due 
to mass transfer, and the work of the interfacial forces.
The other terms have been described in the previous section. 
Again we can derive an equation for the total energy conservation averaged on the whole bulk. 
From \ref{eq:one-fuild_totenergy}, it yields
\begin{equation}
    \pddt \avg{\rho E}
    = \nablab \cdot 
        \avg{\textbf{T}\cdot\textbf{u}
        - \textbf{q}
        - \rho E \textbf{u}}
    + \avg{\textbf{b}\cdot\textbf{u}}
    + \Iavg{\textbf{f}_I\cdot\textbf{u}_I}.
    \label{eq:avg_totenergy}
\end{equation}
Again, as previously it is possible to separate, mechanical and internal energy from this general equation. 
The mechanical energy equation is obtained by averaging \ref{eq:one-fuild_meca}, namely,
\begin{equation}
    \frac{1}{2}\frac{\partial }{\partial t}
            \avg{\rho u^2}
    + \frac{1}{2}\nablab \cdot 
            \avg{\rho u^2 \textbf{u}}
    = \nablab\cdot\avg{\textbf{u}\cdot \textbf{T}}
    +\avg{\textbf{u}\cdot\textbf{b} - \textbf{T}: \nablabh\textbf{u}}
    +\Iavg{\textbf{u}\cdot\textbf{f}_I}
\end{equation}
Since we considered the \textit{energy jump} as being solely due to the work of the \textit{force jump}, it must appear entirely in this equation. 
Likewise, the internal averaged energy equation is obtained averaging \ref{eq:one-fuild_internal_energy},
\begin{equation}
    \pddt \avg{\rho e}
    + \nablab \cdot \avg{\rho  e\textbf{u}}
    = \avg{\textbf{T}:\nablabh\textbf{u}}
    + \avg{\nablabh\cdot\textbf{q}}.
\end{equation}
Again, since the \textit{energy jump} is purely mechanical the surface source term do nor appear in this equation. 
Additional, work on the averaged equations of energy is presented in \ref{ap:average}.


\section{Mixture momentum equaiton}

Averaging the above equations gives directly, 
\paragraph{Dispersed phase equations}
\begin{align}
    % m_b (\pddt + \textbf{u}_b \cdot \grad )n_b &= - n_b \div \textbf{u}_b\\
    m_p (\pddt + \textbf{u}_p \cdot \grad )n_p &= - n_p \div \textbf{u}_p\\
    n_p m_p (\pddt + \textbf{u}_p\cdot  \grad) \textbf{u}_p
    + \div \pavg{ m_p \textbf{u}_p' \textbf{u}_p'}
    &= 
    m_p n_p \textbf{g}
    + \pSavg{\bm\sigma_f^0 \cdot \textbf{n}_p}
\end{align}
For the higher moments balance we may neglect inertia as the relative velocity is negligible. 
Additionally we introduce the notation,

\paragraph{Momentum conservation of the mixture}

Local
\begin{align}
     \div \textbf{u}^0 &= 0, \\
    \rho_f(\pddt 
    + \textbf{u}^0 \cdot \grad) \textbf{u}^0
    &= 
    \div \bm\sigma_*^0
    +\rho_f \textbf{g}
    +\kappa(\bm\sigma_f^0\cdot \textbf{n})\delta_\Gamma,
\end{align}
with 
\begin{equation}
    \bm\sigma_*^0 = 
    \chi_f \bm\sigma_f^0
    + \zeta^{-1}(\chi_d \bm\sigma_d^0
    + \delta_\Gamma \bm\sigma_\Gamma^0)
\end{equation}

\begin{align}
    \label{eq:NS_not_dispersed_mass}
    \div\textbf{u}&=0, \\
    \rho_f (\pddt 
    + \textbf{u}\cdot \grad)
    \textbf{u}
    +\div\avg{\textbf{u}'\textbf{u}'}
    &= 
    (1 +\kappa\phi)\div \bm\Sigma
    +\div  \bm{\sigma}_\text{eff} 
    + \kappa \avg{\delta_\Gamma \bm{\sigma}_f' \cdot \textbf{n}} 
    + \rho_f \textbf{g} 
    \label{eq:NS_not_dispersed}
\end{align}
with $\kappa = (1/\zeta - 1)$. 
with the effective stress, 
\begin{equation}
    \bm\sigma_\text{eff}  = 
    \zeta^{-1} \avg{\chi_d  \bm\sigma_d' + \chi_\Gamma  \bm\sigma_\Gamma^0}
    -  2\mu_f\avg{\chi_d  \textbf{e}_d'}
\end{equation}
\paragraph{Momentum conservation of the mixture: hybrid}
\begin{align}
    \label{eq:NS_not_dispersed_mass}
    \div\textbf{u}&=0, \\
    \rho_f (\pddt 
    + \textbf{u}\cdot \grad)
    \textbf{u}
    +\div\avg{\textbf{u}'\textbf{u}'}
    &= 
    (1+\kappa \phi)\div \bm\Sigma
    +\div \bm{\sigma}_\text{eff} 
    + \kappa \pSavg{ \bm{\sigma}_f' \cdot \textbf{n}} 
    + \rho_f \textbf{g} 
\end{align}
with $\kappa = (1/\zeta - 1)$. 
with the effective stress, 
\begin{multline}
    \bm\sigma_\text{eff}  = 
    + \pSavg{\textbf{r}\bm\sigma_f' \cdot \textbf{n}}
    -2\mu_f  \pOavg{ \textbf{e}_d'}
    + \zeta^{-1} \pOavg{\textbf{r}\div \bm\Sigma}
    \\
    - \div \left[
        \frac{1}{2}
        \pSavg{\textbf{rr}\bm\sigma_f' \cdot \textbf{n}}
        -2\mu_f \pOavg{\textbf{r}\textbf{e}_d'}
        + \frac{1}{2}\zeta^{-1}\pOavg{\textbf{rr}(\div\bm\Sigma + \rho_d \textbf{g})} 
    \right] 
\end{multline}


Let us define,
\begin{align}
    n_p f^1 =\avg{\delta_1 f^0}
    && 
    f'' = f^0 -  f^1  
    && 
    \avg{\delta_1 f''} =\avg{\delta_1 f^0} -  \avg{\delta_1} f^1 = 0  \\
    n_p f^{1'}
    &=
    n_p (f^1 - f)
    &=
    \avg{(\delta_1 f^0 - n_p  f^0)}
    &=
    \avg{f^0(\delta_1 - n_p )}
    =
    \avg{f^0\delta_{1'}}
\end{align}
The single particle average gives directly,
\begin{equation}
    \pddt \delta_1  +  \div_r \textbf{w} \delta_1 = 0
 \end{equation}

\section{Conditional average}
 \paragraph*{Conditional eq:}
 \begin{align*}
    n_p \div \textbf{u}^1 &= 0, \\
   n_p \rho_f(\pddt 
   + \textbf{u}^1 \cdot \grad) \textbf{u}^1
   + \grad_r \avg{\delta_1 \textbf{u}^0 \textbf{w}}
   + \div \avg{\delta_1 \textbf{u}'' \textbf{u}''}
   &= 
   \div \bm\sigma_*^1
   +n_p \rho_f \textbf{g}
   +\kappa\avg{\delta_1\delta_\Gamma  \bm\sigma_f''\cdot \textbf{n}}
   +\kappa \phi^1 n_p\div \bm\Sigma^1,
\end{align*}
\begin{equation}
    \bm\sigma_*^1 = 
    n_p \bm\Sigma^1
    % + \kappa \avg{\chi_d \delta_1 \bm\Sigma^1}
    + \zeta^{-1}\avg{\delta_1(\chi_d \bm\sigma_d'' + \delta_\Gamma \bm\sigma_\Gamma^0)}
    - \avg{\delta_1 \chi_d \textbf{e}_d''}
\end{equation}
\paragraph*{Averaged equaiton times $n_p$}
\begin{align}
    n_p \div\textbf{u}&=0, \\
    n_p \rho_f (\pddt 
    + \textbf{u}\cdot \grad)
    \textbf{u}
    +\div_r\avg{n_p \textbf{u}^0\textbf{w}}
    +\div\avg{n_p \textbf{u}'\textbf{u}'}
    &= 
    \div \bm{\sigma}_*
    + \rho_f \textbf{g} 
    + \kappa \avg{\delta_\Gamma \bm{\sigma}_f' \cdot \textbf{n}} 
    + \kappa \phi n_p \div\bm\Sigma
\end{align}
\begin{equation}
    \bm\sigma_* = 
    n_p \bm\Sigma
    % + \kappa \avg{\chi_d \delta_1 \bm\Sigma^1}
    + n_p \zeta^{-1}\avg{\chi_d \bm\sigma_d' + \delta_\Gamma \bm\sigma_\Gamma^0}
    - n_p \avg{ \chi_d \textbf{e}_d'}
\end{equation}
\paragraph*{Disturbance fields equations}
\begin{equation}
    n_p \div \textbf{u}^{1'} = 0
\end{equation}
\begin{multline*}
   n_p \rho_f(\pddt \textbf{u}^{1'}
   + \textbf{u}^{1'} \cdot \grad \textbf{u}^{1'}
   + \textbf{u} \cdot \grad \textbf{u}^{1'}
   + \textbf{u}^{1'} \cdot \grad \textbf{u})
   +\textbf{w}\cdot  \grad_r n_p \textbf{u}^{1'}
   + \div \avg{\delta_1 \textbf{u}'' \textbf{u}''}
   - \div \avg{n_p \textbf{u}' \textbf{u}'}
   \\
   = 
   \div \bm\sigma_*^{1'}
   +\kappa\avg{\delta_\Gamma  (\delta_1\bm\sigma_f'' - n_p \bm\sigma_f')\cdot \textbf{n}}
   +\kappa \avg{\chi_d \div(\delta_1 \bm\Sigma^1 - n_p \bm\Sigma)},
\end{multline*}
\begin{equation}
    \bm\sigma_*^{1'} = 
    n_p \bm\Sigma^{1'}
    % + \kappa \avg{\chi_d \delta_1 \bm\Sigma^1}
    + \zeta^{-1}\avg{\chi_d (\delta_1\bm\sigma_d'' - n_p \bm\sigma_d') + (\delta_1-n_p)\delta_\Gamma \bm\sigma_\Gamma^0}
    - \avg{\chi_d (\delta_1 \textbf{e}_d''- n_p \textbf{e}_d')}
\end{equation}


\section{Zero-order hydrodynamic $\mathcal{O}( \phi^0)$}

This represents basically a single phase flow Newtonian turbulent flow, 
\begin{align}
    \div\textbf{u}&=0, \\
    \rho_f (\pddt 
    + \textbf{u}\cdot \grad)
    \textbf{u}
    &= 
    \div( \bm\Sigma
    + \avg{ \textbf{u}'\textbf{u}'} )
    + \rho_f \textbf{g} 
\end{align}


\section{First-order hydrodynamic $\mathcal{O}( \phi^1)$}

The first order hydrodynamic is given by solving the single particle conditioned eq at $\mathcal{O}(\phi)$, outside the particle it gives, 
\begin{equation}
    n_p \div \textbf{u}^{1'} = 0
\end{equation}
\begin{equation*}
    - n_p\div \bm\Sigma^{1'}
   = 
   0
\end{equation*}

\begin{align}
    \pSavg{\bm\sigma_f'\cdot \textbf{n}} 
    &=
    \phi
    \frac{\mu_f}{a^2}
    \frac{3(2+3\lambda)}{2(1+\lambda)}\textbf{u}_r
    + \phi\mu_f  \frac{3\lambda}{4(\lambda +1)} \grad^2 \textbf{u}
    \label{eq:drag_forces}
    \\
    \pSavg{\textbf{r}\bm\sigma_f'\cdot \textbf{n}} - \pOavg{2\mu_f\textbf{e}_d'} 
    &= \mu_f \phi 
    \frac{(5\lambda +2)}{(\lambda +1)}\textbf{E} 
    + \mathcal{O}(a^2/L^2)
    % + \phi a^2 \mu_f \frac{\lambda}{2(\lambda +1)}\grad^2 \textbf{E}
    \\
    \frac{1}{2}\pSavg{\textbf{rr}\bm\sigma_f'\cdot \textbf{n}} 
    - \pSavg{ 2\mu_f\textbf{r}\textbf{e}'_d} 
    &=
    - \mu_f \phi \frac{3\lambda}{4(\lambda +1)}[\bm\delta \textbf{u}_r + \frac{1}{2\lambda}\textbf{u}_r \bm\delta ]
    + \mathcal{O}(\phi a^2/L^2)
    % \\
    % \pOavg{\div\bm\Sigma} &=
    % \phi\div\bm\Sigma
    % + \mathcal{O}(\phi a^2/L^2)
    \\
    \pOavg{\textbf{r}\div\bm\Sigma} &=
    \mathcal{O}(\phi a^2/L^2)\\
    \pOavg{\textbf{rr}(\div\bm\Sigma+\rho_f \textbf{g})}&=
    \mathcal{O}(\phi a^2/L^2)
    \label{eq:second_moment_surf}
\end{align}

Because of the present symmetry the divergence of the last term might be given by,
\begin{equation}
    - \mu_f \frac{3\lambda}{4(\lambda +1)}[\grad (\phi \textbf{u}_r)+\grad (\phi\textbf{u}_r)] + \frac{3\lambda-2}{4(\lambda+1)}\div (\phi\textbf{u}_r) \bm\delta 
\end{equation} 



Thus, we end up with, 
\begin{align}
    \div\textbf{u}&=0, \\
    \rho_f (\pddt 
    + \textbf{u}\cdot \grad)
    \textbf{u}
    % +\div \avg{\textbf{u}'\textbf{u}'}
    &= 
    (1+\kappa \phi) \div \bm\Sigma
    +\div \bm{\sigma}_\text{eff} 
    % + \kappa \pOavg{\div\bm\Sigma} 
    + \rho_f \textbf{g} 
    +\phi \frac{\mu_f}{a^2} \frac{3(2+3\lambda)}{2(1+\lambda)}\textbf{u}_r
    + \phi\mu_f  \frac{3\lambda}{4(\lambda +1)} \grad^2 \textbf{u}
\end{align}
\begin{align*}
    \bm\sigma_\text{eff}  = 
    - \avg{ \textbf{u}'\textbf{u}'}
    + \mu_f \phi  \frac{(5\lambda +2)}{(\lambda +1)}\textbf{E} 
    + \mu_f \frac{3\lambda}{4(\lambda +1)}[\grad (\phi \textbf{u}_r)+\grad (\phi\textbf{u}_r)] 
    - \frac{3\lambda-2}{4(\lambda+1)}\div (\phi\textbf{u}_r) \bm\delta 
\end{align*}
\section{Second-order hydrodynamic $\mathcal{O}( \phi^2)$}

For the second order hydrodynamic problem one must solve the single particle conditioned problem but at $\phi^2$ instead.

The  one fluid equation at order phi and for force free particle s
\begin{align*}
    n_p \div \textbf{u}^{1'} &= 0\\
    -n_p\div\bm\Sigma^{1'} 
    &= 
   \div \bm\sigma_*^{1'}
   - (\rho_d-\rho_f)\zeta v_p\textbf{g}  \pavg{\delta_{1'}}
\end{align*}
\begin{multline}
    \bm\sigma_*^{1'}  = 
    + \pSavg{\textbf{r}(\delta_1\bm\sigma_f'' - n_p \bm\sigma_f') \cdot \textbf{n}}
    -2\mu_f  \pOavg{ (\delta_1 \textbf{e}_d''- n_p \textbf{e}_d')}\\
    +  \pOavg{\textbf{r}\div (\delta_1 \bm\Sigma^1-n_p \bm\Sigma)}
    \\
    - \div \left[
        \frac{1}{2}
        \pSavg{\textbf{rr}(\delta_1\bm\sigma_f'' - n_p \bm\sigma_f') \cdot \textbf{n}}
        -2\mu_f \pOavg{\textbf{r}(\delta_1 \textbf{e}_d''- n_p \textbf{e}_d')} \right.\\ \left.
        + \frac{1}{2}\pOavg{\textbf{rr}(\div(\delta_1 \bm\Sigma^1-n_p \bm\Sigma) + \delta_{1'} \rho_f \textbf{g})} 
    \right] 
\end{multline}
with 
\begin{equation}
    \textbf{u}^{1'}\cdot \textbf{n} =
     (\textbf{w} - \textbf{u})\cdot \textbf{n}
\end{equation}

Thus, one has to find these closure terms accurate at $\phi^2$.
Multiplying the `mean' or unconditioned closure above by $n_p$ one get directly these closure accurate at $\phi^2$. 
Because 
\begin{equation}
    \delta_1
    =
    \sum_i \delta(\textbf{y} - \textbf{x}_i)
\end{equation}
We have 
\begin{align}
    \pavg{\delta_1}
    =
    \avg{\sum_{i,j} \delta(\textbf{x}-\textbf{x}_i)\delta(\textbf{y}-\textbf{x}_j)}
    % &=
    % \delta(\textbf{x}-\textbf{y})\avg{\sum_{i} \delta(\textbf{x}-\textbf{x}_i)}
    % +\avg{\sum_{i,j\neq i} \delta(\textbf{x}-\textbf{x}_i)\delta(\textbf{x}+\textbf{r}-\textbf{x}_j)}\\
    =
    \delta(\textbf{x}-\textbf{y})n_p(\textbf{x})
    +n_p(\textbf{x},\textbf{y})\\
\end{align}
Hence, 
\begin{equation}
    \pavg{\delta_{1'}}
    =
    \delta(\textbf{x} - \textbf{y}) n_p(\textbf{x})
    + n_p(\textbf{x},\textbf{y})
    - n_p(\textbf{x}) n_p(\textbf{y}) 
\end{equation}
the first term represents the sphere at $\textbf{r} = \textbf{0}$ and the second and third the spatial correlation. 

For know the Stresslet term allow $i=j$, so first of all note that,
\begin{align}
    \pSavg{\textbf{r}\delta_1\bm\sigma_f'' \cdot \textbf{n}}
    &=
    % \avg{\sum_{i,j}\delta(\textbf{x}-\textbf{x}_i)\delta(\textbf{y}-\textbf{x}_j)\intS{\textbf{r}\bm\sigma_f'' \cdot \textbf{n}}}\\
    % &=
    % \delta(\textbf{x}-\textbf{y})\avg{\sum_{i}\delta(\textbf{x}-\textbf{x}_i)\intS{\textbf{r}\bm\sigma_f'' \cdot \textbf{n}}}\\
    % &+ \avg{\sum_{i,j\neq i}\delta(\textbf{x}-\textbf{x}_i)\delta(\textbf{y}-\textbf{x}_j)\intS{\textbf{r}\bm\sigma_f'' \cdot \textbf{n}}}\\
    % &=
    % \delta(\textbf{x}-\textbf{y})\avg{\sum_{i}\delta(\textbf{x}-\textbf{x}_i)\intS{\textbf{r}\bm\sigma_f'' \cdot \textbf{n}}}\\
     \avg{\sum_{i,j\neq i}\delta(\textbf{x}-\textbf{x}_i)\delta(\textbf{y}-\textbf{x}_j)\intS{\textbf{r}\bm\sigma_f'' \cdot \textbf{n}}}\\
\end{align}
By definition the first term cancel because $\delta(\textbf{x}-\textbf{y})\pavg{\bm\sigma_f''} = 0$, that proves the last equality. 
This is totally logical as when the two particle are at the same place this correspond to the single particle case which we have remove with the mean stress. 
Thus, this term corresponds to the stresslet on a particle centered at \textbf{x} knowing a particle is present at $\textbf{y}=\textbf{x}+\textbf{r}$, minus the mean stress conditioned on that a particle is at \textbf{y}. 
similar comments can be made for the higher moments. 


The mean stress reads,
\begin{align}
    \pOavg{\div (\delta_1 \bm\Sigma^1-n_p \bm\Sigma)}
    &=
    \pOavg{\div \delta_1 \bm\Sigma^1}
    - n_p\pOavg{\div \bm\Sigma}\\
    &=
    \delta(\textbf{x}- \textbf{y})\pOavg{\div \bm\Sigma^1}
\end{align}


Let consider $\bm\Sigma^{(0)}$ the stress at $\mathcal{O}(\phi^1)$ and $\bm\Sigma^{(1)}$ at $\mathcal{O}(\phi^2)$, then the reciprocal theorem gives, 

\begin{equation}
    \intO[out]{\textbf{n}\cdot \bm\Sigma^{(1)} \cdot \textbf{u}^{(0)}}
    + \intO[out]{\textbf{f} \cdot \textbf{u}^{(0)}}
    =
    \intS{\textbf{n}\cdot \bm\Sigma^{(0)} \cdot \textbf{u}^{(1)}}
\end{equation}

Let consider the test problem the one of a translating droplets $\textbf{u}^{(0)}= \mathcal{U}(\textbf{r})\cdot \textbf{U}_2$, then 

\subsection{two particles problem to interaction}
We are looking for an equation for 
\begin{equation}
    \pavg{\delta_1 (\textbf{u}^0  - \textbf{u}^1)}
    =
    \pavg{\delta_1 (\textbf{u}^0  - \textbf{u}^1)}
\end{equation}
Hence one need to conditional avg the local eq by $\avg{\delta_1\delta_2 \textbf{u}^0}$ and then remove the mean contribution etc. 
This reads, 
\begin{align*}
%     n_{2p} \div \textbf{u}^2 &= 0, \\
%    n_{2p} \rho_f(\pddt 
%    + \textbf{u}^2 \cdot \grad) \textbf{u}^2
%    + \grad_r \avg{\delta_{12} \textbf{u}^0 \textbf{w}}
%    + \div \avg{\delta_{12} \textbf{u}''' \textbf{u}'''}
0
   &= 
   \div \bm\sigma_*^2
   +n_{2p} \rho_f \textbf{g}
   +\kappa\avg{\delta_{12}\delta_\Gamma  \bm\sigma_f'''\cdot \textbf{n}}
   +\kappa \phi^2 n_{2p}\div \bm\Sigma^2,
\end{align*}
\begin{equation}
    \bm\sigma_*^2 = 
    n_{2p} \bm\Sigma^2
    % + \kappa \avg{\chi_d \delta_{12} \bm\Sigma^1}
    + \zeta^{-1}\avg{\delta_{12}(\chi_d \bm\sigma_d''' + \delta_\Gamma \bm\sigma_\Gamma^0)}
    - \avg{\delta_{12} \chi_d \textbf{e}_d'''}
\end{equation}

We deduce directly the disturbance field equation, 
\begin{equation}
    n_{2p}\div \textbf{u}^{2''}=0
\end{equation}
\begin{equation*}
    0
    = 
    \div \bm\sigma_*^{2''}
    +\kappa\avg{\delta_\Gamma  (\delta_{12}\bm\sigma_f''' - n_{2p} \bm\sigma_f'')\cdot \textbf{n}}
    +\kappa \avg{\chi_d \div(\delta_{12} \bm\Sigma^1 - n_{2p} \bm\Sigma)},
 \end{equation*}
 \begin{equation}
     \bm\sigma_*^{2''} = 
     n_{2p} \bm\Sigma^{2''}
     % + \kappa \avg{\chi_d \delta_{12} \bm\Sigma^1}
     + \zeta^{-1}\avg{\chi_d (\delta_{12}\bm\sigma_d''' - n_{2p} \bm\sigma_d'') + (\delta_{12}-n_{2p})\delta_\Gamma \bm\sigma_\Gamma^0}
     - \avg{\chi_d (\delta_{12} \textbf{e}_d'''- n_{2p} \textbf{e}_d'')}
 \end{equation}

 At the boundary of the droplet the velocity field is presecibed by,
\begin{equation}
   \textbf{u}^{2''} \cdot \textbf{n} 
   = (\textbf{w}_2 - \textbf{u}^1) \cdot \textbf{n}
\end{equation}
Note that $\textbf{u}^1$ is an unkown because at $\phi^2$ this field is still not knwon this  is the reason why the reflexion method existe?  

This equation is of course complicated however note that,
\begin{align*}
    \delta_p \delta_{12}
    &=
    \sum_{i,j,k\neq j} 
    \delta(\textbf{x}- \textbf{x}_i)
    \delta(\textbf{r}_1- \textbf{x}_j)
    \delta(\textbf{r}_2- \textbf{x}_k)\\
    &=
    \delta(\textbf{r}_2- \textbf{x})
    \sum_{i,j,k=i\neq j} 
    \delta(\textbf{x}- \textbf{x}_i)
    \delta(\textbf{r}_1- \textbf{x}_j)\\
    &+\delta(\textbf{r}_1- \textbf{x})
    \sum_{i,k\neq i} 
    \delta(\textbf{x}- \textbf{x}_i)
    \delta(\textbf{r}_2- \textbf{x}_k)\\
    &+\sum_{i,j\neq i,k\neq j,i} 
    \delta(\textbf{x}- \textbf{x}_i)
    \delta(\textbf{r}_1- \textbf{x}_j)
    \delta(\textbf{r}_2- \textbf{x}_k)\\
\end{align*}
Hence only pts forces at the particle location remain at $\phi^2$, likewise  


We deduce that,
\begin{align*}
    F_{ij}(\textbf{x},\textbf{x}+\textbf{r})
    &=
    -\left(\frac{3 r^{-1}}{4} + \frac{r^{-3} \beta^{2}}{4} + \frac{r^{-3}}{4}\right)\delta^{ij} + \left(- \frac{3 r^{-3}}{4} + \frac{3 r^{-5} \beta^{2}}{4} + \frac{3 r^{-5}}{4}\right)r^{i}r^{j}\\
    S_{ij}(\textbf{x},\textbf{x}+\textbf{r})
    &=
    -\left(\frac{5 r^{-3}}{6} - \frac{5 r^{-5} \beta^{2}}{6} - \frac{r^{-5}}{2} + \frac{35 r^{-7} \beta^{2}}{12}\right)\delta^{il}r^{j} 
    + \left(\frac{5 r^{-5} \beta^{2}}{6} + \frac{r^{-5}}{2} - \frac{5 r^{-7} \beta^{2}}{6}\right)\delta^{ij}r^{l} \\
    &+ \left(\frac{5 r^{-5} \beta^{2}}{6} + \frac{r^{-5}}{2} - \frac{5 r^{-7} \beta^{2}}{6}\right)\delta^{jl}r^{i} 
    + \left(\frac{5 r^{-5}}{2} - \frac{25 r^{-7} \beta^{2}}{6}\right)r^{i}r^{j}r^{l}
\end{align*}

\begin{align}
    \pOavg{\div (\delta_1 \bm\Sigma^1-n_p \bm\Sigma)}
    &=
    \pOavg{\div \delta_1 \bm\Sigma^1}
    - n_p\pOavg{\div \bm\Sigma}\\
\end{align}

\subsubsection*{Kim \& Karria the reflexion }

Let, 
\begin{align*}
    \mathcal{G} 
    &= \frac{\delta_{ij}}{r}
    + \frac{x_ix_j}{r^3}\\
    \mathcal{E}
    &= 
    \frac{\delta_{ik}x_j}{r^3}
    - 3\frac{x_ix_jx_k}{r^5}\\
\end{align*}
Let the particles \textbf{1} and \textbf{2} produce the disturbance field, 
\begin{align}
    \textbf{v}_1= - \textbf{F}_1^{(0)}\cdot \textbf{V}_1 = - \textbf{F}_1^{(0)}\left[1+\frac{1}{6}\grad^2 \right]\frac{\mathcal{G}(\textbf{x}-\textbf{x}_1)}{8}\\
    \textbf{v}_2= - \textbf{F}_2^{(0)}\cdot \textbf{V}_2 = - \textbf{F}_2^{(0)}\left[1+\frac{\beta}{6}\grad^2 \right]\frac{\mathcal{G}(\textbf{x}-\textbf{x}_2)}{8}
\end{align}
\begin{align}
    \textbf{e}_1= - \textbf{F}_1^{(0)}\cdot \textbf{E}_1 = - \textbf{F}_1^{(0)}\left[1+\frac{1}{6}\grad^2 \right]\frac{\mathcal{E}(\textbf{x}-\textbf{x}_1)}{8}\\
    \textbf{e}_2= - \textbf{F}_2^{(0)}\cdot \textbf{E}_2 = - \textbf{F}_2^{(0)}\left[1+\frac{\beta}{6}\grad^2 \right]\frac{\mathcal{E}(\textbf{x}-\textbf{x}_2)}{8}
\end{align}
with zeorth order forces being related to the droplet translation with 
\begin{align*}
    \textbf{F}^{(0)}_1 = 6 (\textbf{u} - \textbf{w}_1)\\
    \textbf{F}^{(0)}_2 = 6 (\textbf{u} - \textbf{w}_2)
\end{align*}


At the first order reflexion we may say that, 
\begin{align}
    \textbf{F}^{(1)}_1/(\pi\mu a) 
    = 6 (1 + \frac{1}{6}\grad^2) \textbf{v}_2|_{\textbf{x}=\textbf{x}_1}
    = - \textbf{F}_2^{(0)}\cdot 6 (1 + \frac{1}{6}\grad^2) \left[1+\frac{\beta}{6}\grad^2 \right]\frac{\mathcal{G}(\textbf{x}-\textbf{x}_2)}{8}\\
    \textbf{F}^{(1)}_2/(\pi\mu a) 
    = 6 \beta (1 + \frac{\beta^2}{6}\grad^2) \textbf{v}_1|_{\textbf{x}=\textbf{x}_2}
    = - \textbf{F}_1^{(0)}\cdot 6\beta (1 + \frac{\beta}{6}\grad^2) \left[1+\frac{1}{6}\grad^2 \right]\frac{\mathcal{G}(\textbf{x}-\textbf{x}_1)}{8}
\end{align}
\begin{align}
    \textbf{S}^{(1)}_1/(\pi\mu a^3) 
    = \frac{20}{3} (1 + \frac{1}{10}\grad^2) \textbf{e}_2|_{\textbf{x}=\textbf{x}_1}
    = - \textbf{F}_2^{(0)}\cdot\frac{20}{3} (1 + \frac{1}{10}\grad^2) \left[1+\frac{\beta}{6}\grad^2 \right]\frac{\mathcal{E}(\textbf{x}-\textbf{x}_2)}{8}\\
    \textbf{S}^{(1)}_2/(\pi\mu a^3) 
    = \frac{20}{3} \beta^3 (1 + \frac{\beta^2}{10}\grad^2) \textbf{e}_1|_{\textbf{x}=\textbf{x}_2}
    = - \textbf{F}_2^{(0)}\cdot\frac{20}{3} (1 + \frac{1}{10}\grad^2) \left[1+\frac{1}{6}\grad^2 \right]\frac{\mathcal{E}(\textbf{x}-\textbf{x}_1)}{8}
\end{align}


\begin{align*}
    \textbf{v}_{21}
    = - \textbf{F}_1^{(1)}\cdot \left[1+\frac{1}{6}\grad^2 \right]\frac{\mathcal{G}(\textbf{x}-\textbf{x}_1)}{8}
    + \textbf{S}_1^{(1)} :\grad  \frac{\mathcal{G}(\textbf{x}-\textbf{x}_1)}{8}\\
    \textbf{v}_{12}
    = - \textbf{F}_2^{(1)}\cdot \left[1+\frac{\beta^2}{6}\grad^2 \right]\frac{\mathcal{G}(\textbf{x}-\textbf{x}_1)}{8}
    + \textbf{S}_2^{(1)} :\grad  \frac{\mathcal{G}(\textbf{x}-\textbf{x}_1)}{8}\\
\end{align*}

Then the second reflexion can be obtained by computing the forces, 
\begin{align}
    \textbf{F}^{(2)}_1/(\pi\mu a) 
    &= 6 (1 + \frac{1}{6}\grad^2) \textbf{v}_{12}|_{\textbf{x}=\textbf{x}_1}\\
    &= 
    - \textbf{F}_2^{(1)}\cdot 6 (1 + \frac{1}{6}\grad^2) \left[1+\frac{\beta}{6}\grad^2 \right]\frac{\mathcal{G}(\textbf{x}-\textbf{x}_2)}{8}\\
    &+ \textbf{S}_2^{(1)}:  6 (1 + \frac{1}{6}\grad^2) \left[1+\frac{\beta}{6}\grad^2 \right]\frac{\grad\mathcal{G}(\textbf{x}-\textbf{x}_2)}{8}\\
    % \textbf{F}^{(1)}_2/(\pi\mu a) 
    % = 6 \beta (1 + \frac{\beta^2}{6}\grad^2) \textbf{v}_1|_{\textbf{x}=\textbf{x}_2}
    % = - \textbf{F}_1^{(0)}\cdot 6\beta (1 + \frac{\beta}{6}\grad^2) \left[1+\frac{1}{6}\grad^2 \right]\frac{\mathcal{G}(\textbf{x}-\textbf{x}_1)}{8}
\end{align}
% \begin{align}
%     \textbf{S}^{(1)}_1/(\pi\mu a^3) 
%     = \frac{20}{3} (1 + \frac{1}{10}\grad^2) \textbf{e}_2|_{\textbf{x}=\textbf{x}_1}
%     = - \textbf{F}_2^{(0)}\cdot\frac{20}{3} (1 + \frac{1}{10}\grad^2) \left[1+\frac{\beta}{6}\grad^2 \right]\frac{\mathcal{E}(\textbf{x}-\textbf{x}_2)}{8}\\
%     \textbf{S}^{(1)}_2/(\pi\mu a^3) 
%     = \frac{20}{3} \beta^3 (1 + \frac{\beta^2}{10}\grad^2) \textbf{e}_1|_{\textbf{x}=\textbf{x}_2}
%     = - \textbf{F}_2^{(0)}\cdot\frac{20}{3} (1 + \frac{1}{10}\grad^2) \left[1+\frac{1}{6}\grad^2 \right]\frac{\mathcal{E}(\textbf{x}-\textbf{x}_1)}{8}
% \end{align}

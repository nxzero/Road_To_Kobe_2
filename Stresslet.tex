\documentclass[12pt]{My_preprint}
\usetikzlibrary{arrows.meta,
                chains,
                positioning,
                shapes.geometric}

%%%%%%%%%%%%%%%%%%%%%%%%%%%%%%%%%%%%%%%%%%%%%%%%%%%%%%%%%%%%%%%%%%%%%%%%%%%%%%%

\renewcommand{\size}[1]{0.3\textwidth}

% \newcommand{\numref}[1]{\ref{#1}}
\renewcommand{\ref}[1]{\autoref{#1}}

%%%%%%%%%%%%%%%%%%%%%%%%%%%%%%% Title & Author %%%%%%%%%%%%%%%%%%%%%%%%%%%%%%%%


\title{Incoherence in the stresslet formulation for spherical drops.}

\author[1,2]{Nicolas Fintzi}
\affil[1]{IFP Energies Nouvelles, Rond-point de l’changeur de Solaize, 69360 Solaize}
\affil[2]{Sorbonne Université, Institut Jean le Rond ∂’Alembert, 4 place Jussieu, 75252 PARIS CEDEX 05, France}

\begin{document}
\maketitle
\begin{abstract}
    % In this note we base our reasoning on the book of \cite{pozrikidis1992boundary}. 
    In this note we study the stresslet produced by spherical drop of iso viscosity with the ambient fluid.
    The droplet is immersed in a purely extensional flow defined by the undisturbed field  $u^\infty(\textbf{x})$.  
    The objective it to study the value of the Stresslet coefficient in the case were the viscosity of the drop $\mu_d$ is the same as the viscosity of the continuous phase : $\mu_f$.
    We show that \ref{eq:stresslet} and \ref{eq:stress} are not coherent for $\lambda = \mu_f/\mu_d = 1$. 
\end{abstract}

\paragraph{Known formulas :}
In \citet[chapter 2]{pozrikidis1992boundary} formula (2.5.13) we have the following expression for the coefficient of the stresslet of an arbitrary particle, 
\begin{equation}
    \label{eq:stresslet}
    \mathscr{S}_{ki}
    = \frac{1}{2}
    \int_{S_p^+}
    \left[
        x_k f_i + x_i f_k 
        - \delta_{ik}
        \frac{2}{3}
        x_l f_l
        - 2 \mu_f (u_k n_i+u_i n_i)
    \right]
    dS
\end{equation}
where we integrate over the surface of the spherical drop $S_p$ where the superscript + indicates that the surface force $f$ and the velocity  is evaluated on the external surface of the particle.
Likewise, the superscript $-$ indicate that it is evaluated at the interior of the surface. 

Using the singularity solutions, we can demonstrate that the coefficient of the stresslet for spherical droplets of viscosity ratio $\lambda = \mu_f/\mu_d$ immersed in an extensional flow is, 
\begin{equation}
    \label{eq:stress}
    \mathscr{S}_{ij}
    = \frac{2}{3}\pi \mu_f a^3 \left(
        \frac{2+5\lambda}{1+\lambda}
    \right)
    \left[
        1+\frac{\lambda}{2(5\lambda +2)}\grad^2
    \right]
    \left(
        \frac{\partial u^\infty_j}{\partial x_i}
        + \frac{\partial u^\infty_i}{\partial x_j}
    \right)
\end{equation}
where $ \left(
    \frac{\partial u^\infty_j}{\partial x_i}
    + \frac{\partial u^\infty_i}{\partial x_j}
\right)$ is gradient of the undisturbed extensional flow at the location of the center of mass of the droplet. 
$a$ is the radius of the droplet. 
This formula has been first demonstrated by \citep{rallison1978note}. 

Now, the objective is to reformulate \ref{eq:stresslet} to show that the stresslet is zero for spherical droplet when $\lambda=0$, which is not the case according to \ref{eq:stress}.



\paragraph*{The stress inside and at the surface of the drop :}

We assume a Newtonian fluid for the droplets phase such that the stress in the interior volume of the droplets $V^-_p$ can be expressed as, $\sigma_{ij} = - p_d \delta_{ij} + \mu_d e_{ij}$ with the rate of strain $e_{ij} = \partial_i u_j + \partial_j u_i$. 
The surface stress yielding on $S_p$ can be written $\sigma^I_{ij} = \gamma (\delta_{ij} - n_in_j)$, for simplicity we take a constant surface tension. 


\paragraph*{Reformulation of the surface terms :}
Then, using Batchelor’s famous formula :
\begin{equation*}
    \int_{V^+_p} \sigma_{ij} dV 
    = \int_{S^+_p} x_i f_j dS.
    \label{eq:bachelor}
\end{equation*}
 we obtain, 
\begin{equation}
    \int_{S^+_p} \left[
        x_i f_j+ x_j f_i - \frac{1}{3}\delta_{ik}x_lf_l
    \right]  dS.
    = \int_{V^+_p} \left[
        \sigma_{ij} 
        + \sigma_{ji} 
        - \frac{1}{3}\delta_{ik}x_lf_l
        \right]
    dV.
\end{equation}
The surface and volume stress of the droplet are purely symmetric quantities.
Additionally, since we consider spherical drop $\sigma^I_{ji} + \sigma^I_{ij} -\frac{2}{3}\delta_{ij}\sigma_{ll}^I = 0$.
Therefore, we can rewrite \ref{eq:stresslet} such as :
\begin{equation}
    \label{eq:stresslet2}
    \mathscr{S}_{ki}
    = 
    \int_{V_p^-}
        \mu_d e_{ij} 
    dS
    - \int_{S_p^+}
    \left[
         \mu_f (u_k n_i+u_i n_i)
    \right]
    dS
\end{equation}
Then, making use of the surface divergence theorem on the first term on the right hands side we obtain :
\begin{equation*}
    \int_{V_p^+}
        \mu_d e_{ij}
    dV
    = \int_{S_p^-}
        \mu_d (u_k n_i+u_i n_k)
    dS
    = 
    \int_{S_p^+}
        \mu_d (u_k n_i+u_i n_k)
    dV
\end{equation*}
The passage from the second to the last integral is made possible since at the interface the velocity of the fluid and the one inside the drop is equal. 
Injecting this last integral into \ref{eq:stresslet2} gives this formula for the stresslet, 
\begin{equation}
    \mathscr{S}_{ki}
    = 
    \int_{S_p^+}
        \mu_d (u_k n_i+u_i n_k)
    dV
    - \int_{S_p^+}
    \mu_f (u_k n_i+u_i n_k)
    dV
    \label{eq:final_stress}
\end{equation}
If $\lambda =1$, then $\mu_d=\mu_f$ and both terms cancels out, and we are left with :  $\mathscr{S}_{ki} = 0$. 

\paragraph*{Problem :}
Why does the last expression \ref{eq:final_stress} is null for $\lambda=1$ and \ref{eq:stress} isn't ?
\begin{equation}
    \label{eq:stress}
    \mathscr{S}_{ij}(\lambda=1)
    =\pi \mu_f a^3 \left(
        \frac{7}{3}
    \right)
    \left[
        1+\frac{1}{14}\grad^2
    \right]
    \left(
        \frac{\partial u^\infty_j}{\partial x_i}
        + \frac{\partial u^\infty_i}{\partial x_j}
    \right)
\end{equation}
\bibliography{Bib/bib_bulles.bib}
\end{document}

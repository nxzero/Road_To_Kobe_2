\documentclass[12pt]{My_preprint}
\usetikzlibrary{arrows.meta,
                chains,
                positioning,
                shapes.geometric}

%%%%%%%%%%%%%%%%%%%%%%%%%%%%%%%%%%%%%%%%%%%%%%%%%%%%%%%%%%%%%%%%%%%%%%%%%%%%%%%
\newcommand{\size}{0.22\textwidth}
\newcommand{\avg}[1]{\left<#1\right>}
\renewcommand{\avg}[1]{\left<#1\right>}
\newcommand{\condavg}[1]{\left<#1 | \mathscr{C}_1\right>}
\newcommand{\Exp}[1]{\overline{\overline{#1}}}
\newcommand{\davg}[1]{\left<#1\right>_d}
\newcommand{\cavg}[1]{\left<#1\right>_c}
\newcommand{\pavg}[1]{\avg{\delta_\alpha #1}}
% \newcommand{\pnavg}[1]{n\left<#1\right>_p}

\newcommand{\avgcond}[1]{\left<#1\right>}
\renewcommand{\avgcond}[1]{\overline{#1}}
\newcommand{\kavg}[1]{\avgcond{#1}^k}
\newcommand{\Iavg}[1]{\avgcond{#1}^I}
\newcommand{\pnnavg}[1]{\avgcond{#1}^{p}}
\newcommand{\pnavg}[1]{n_p\pnnavg{#1}}
\newcommand{\oneavg}[1]{\avgcond{#1}^1}
\newcommand{\twoavg}[1]{\avgcond{#1}^2}
\newcommand{\smallavg}[2]{\avgcond{#1}^{#2}}
\newcommand{\sym}[1]{\text{Sym}\left[#1\right]}

\newcommand{\nstavg}[1]{\overline{#1}_{nst}}
\newcommand{\nstrelavg}[1]{\overline{#1}_{nst}^{rel}}
\newcommand{\mavg}[1]{\left<#1\right>_m}
\newcommand{\gavg}[2][\gamma]{\left<#2\right>_{#1}}
\newcommand{\partials}[1]{\partial_{i_1}\partial_{i_2}\ldots\partial{i_{#1}}}
\newcommand{\partialp}[2]{ \prod_{m=#1}^{#2} \partial_{i_m}}
\newcommand{\hatpartialp}[2]{ \prod_{m=#1}^{#2} \hat{\partial}_{j_m}}
\newcommand{\hatpartialpi}[2]{ \prod_{m=#1}^{#2} \hat{\partial}_{i_m}}
\newcommand{\pri}[2]{ \prod_{m=#1}^{#2} r_{i_m}}
\newcommand{\prj}[2]{ \prod_{m=#1}^{#2} r_{j_m}}

\newcommand{\grad}{\mathbf{\nabla}}
\renewcommand{\div}{\mathbf{\nabla}\cdot}
\newcommand{\gradI}{\mathbf{\nabla}_{||}}
\newcommand{\divI}{\mathbf{\nabla}_{||}\cdot}

\newcommand{\ddt}{\frac{d}{d t}}
\newcommand{\pddt}{\frac{\partial}{\partial t}}
\renewcommand{\pddt}{\partial_t}
\newcommand{\norm}[1]{\hat{#1}}
\newcommand{\Jump}[1]{\llbracket #1 \rrbracket \cdot \textbf{n} }

\newcommand{\CC}{\mathscr{C}}
\newcommand{\PP}{\mathscr{P}}

%%% Utiliser pour les commentaires
\newcommand{\JL}[1]{\color{red}#1\color{black}}
\newcommand{\DL}[1]{\color{green}#1\color{black}}
\newcommand{\tb}[1]{\color{blue}#1\color{black}}
% \renewcommand{\alpha}{}
\renewcommand{\JL}[1]{}
% \renewcommand{\tb}[1]{}

\renewcommand{\size}[1]{0.3\textwidth}
\newcommand{\expo}[2][n]{\frac{(-1)^#1}{#1!} \partialp{1}{#1} \pavg{\int_{\Omega_\alpha} \pri{1}{#1}#2 d\Omega}}
\newcommand{\expoU}[2][n]{\frac{(-1)^#1}{#1!} \partialp{1}{#1} \pavg{\textbf{u}_\alpha\int_{\Omega_\alpha} \pri{1}{#1}#2 d\Omega}}
\newcommand{\expoS}[2][n]{\frac{(-1)^#1}{#1!} \partialp{1}{#1} \pavg{\int_{\Sigma_\alpha} \pri{1}{#1}#2 d\Sigma}}

% \newcommand{\numref}[1]{\ref{#1}}
\renewcommand{\ref}[1]{\autoref{#1}}

%%%%%%%%%%%%%%%%%%%%%%%%%%%%%%% Title & Author %%%%%%%%%%%%%%%%%%%%%%%%%%%%%%%%


\title{Incoherence in the stresslet formulation for spherical drops.}

\author[1,2]{Nicolas Fintzi}
\affil[1]{IFP Energies Nouvelles, Rond-point de l’changeur de Solaize, 69360 Solaize}
\affil[2]{Sorbonne Université, Institut Jean le Rond ∂’Alembert, 4 place Jussieu, 75252 PARIS CEDEX 05, France}

\begin{document}
\maketitle
\begin{abstract}
    % In this note we base our reasoning on the book of \cite{pozrikidis1992boundary}. 
    In this note we study the stresslet produced by spherical drop of iso viscosity with the ambient fluid.
    The droplet is immersed in a purely extensional flow defined by the undisturbed field  $ \left(
        \frac{\partial u^\infty_j}{\partial x_i}
        + \frac{\partial u^\infty_i}{\partial x_j}
    \right)$ $x_i$.  
    The objective it to study the value of the Stresslet coefficient in the case were the viscosity of the drop $\mu_d$ is the same as the viscosity of the continuous phase : $\mu_f$.
    We show that \ref{eq:stresslet} and \ref{eq:stress} are not coherent for $\lambda = \mu_f/\mu_d = 1$. 
\end{abstract}

\paragraph{Known formulas :}
In \citet[chapter 2]{pozrikidis1992boundary} formula (2.5.13) we have the following expression for the coefficient of the stresslet of an arbitrary particle, 
\begin{equation}
    \label{eq:stresslet}
    \mathscr{S}_{ki}
    = \frac{1}{2}
    \int_{S_p^+}
    \left[
        x_k f_i + x_i f_k 
        - \delta_{ik}
        \frac{2}{3}
        x_l f_l
        - 2 \mu_f (u_k n_i+u_i n_i)
    \right]
    dS
\end{equation}
where we integrate over the surface of the spherical drop $S_p$ where the superscript + indicates that the surface force $f$ is evaluated on the external surface of the particle.
Likewise, the superscript $-$ indicate that it is evaluated at the interior of the surface. 

Using the singularity solutions, we can demonstrate that the coefficient of the stresslet for spherical droplets of viscosity ratio $\lambda = \mu_f/\mu_d$ immersed in an extensional flow is, 
\begin{equation}
    \label{eq:stress}
    \mathscr{S}_{ij}
    = \frac{2}{3}\pi \mu_f a^3 \left(
        \frac{2+5\lambda}{1+\lambda}
    \right)
    \left[
        1+\frac{\lambda}{2(5\lambda +2)}\grad^2
    \right]
    \left(
        \frac{\partial u^\infty_j}{\partial x_i}
        + \frac{\partial u^\infty_i}{\partial x_j}
    \right)
\end{equation}
where $ \left(
    \frac{\partial u^\infty_j}{\partial x_i}
    + \frac{\partial u^\infty_i}{\partial x_j}
\right)$ is the undisturbed extensional flow. 
$a$ is the radius of the droplet. 
This formula has been first demonstrated by \citep{rallison1978note}. 

Now, the objective is to reformulate \ref{eq:stresslet} to show that the stresslet is zero for spherical droplet when $\lambda=0$, which is not the case according to \ref{eq:stress}.



\paragraph*{The stress inside and at the surface of the drop :}

We assume a Newtonian fluid for the droplets phase such that the stress in the interior volume of the droplets $V^-_p$ can be expressed as, $\sigma_{ij} = - p_d \delta_{ij} + \mu_d e_{ij}$ with the rate of strain $e_{ij} = \partial_i u_j + \partial_j u_i$. 
The surface stress yielding on $S_p$ can be written $\sigma^I_{ij} = \gamma (\delta_{ij} - n_in_j)$, for simplicity we take a constant surface tension. 


\paragraph*{Reformulation of the surface terms :}
Then, using Batchelor’s famous formula :
\begin{equation*}
    \int_{V^+_p} \sigma_{ij} dV 
    = \int_{S^+_p} x_i f_j dS.
    \label{eq:bachelor}
\end{equation*}
 we obtain, 
\begin{equation}
    \int_{S^+_p} \left[
        x_i f_j+ x_j f_i - \frac{1}{3}\delta_{ik}x_lf_l
    \right]  dS.
    = \int_{V^+_p} \left[
        \sigma_{ij} 
        + \sigma_{ji} 
        - \frac{1}{3}\delta_{ik}x_lf_l
        \right]
    dV.
\end{equation}
The surface and viscous stress of the droplet are purely symmetric quantity.
Additionally, since we consider spherical drop $\sigma^I_{ji} + \sigma^I_{ij} -\frac{2}{3}\delta_{ij}\sigma_{ll}^I = 0$.
Therefore, we can rewrite \ref{eq:stresslet} such as :
\begin{equation}
    \label{eq:stresslet2}
    \mathscr{S}_{ki}
    = 
    \int_{V_p^-}
        \mu_d e_{ij} 
    dS
    - \int_{S_p^+}
    \left[
         \mu_f (u_k n_i+u_i n_i)
    \right]
    dS
\end{equation}
Then, making use of the surface divergence theorem on the second term and the fact that the velocity is continuous at the interface, we have i.e. :
\begin{equation*}
    \int_{S_p^+}
        \mu_f (u_k n_i+u_i n_i)
    dS
    = 
    \int_{V_p^+}
        \mu_f (\partial_i u_j + \partial_j u_i)
    dV
    = 
    \int_{V_p^+}
        \mu_f e_{ij}
    dV
\end{equation*}
Injecting this last integral into \ref{eq:stresslet2} gives this formula for the stresslet, 
\begin{equation}
    \mathscr{S}_{ki}
    = 
    \int_{V_p^-}
        \mu_d e_{ij} 
    dV
    - \int_{V_p^+}
    \mu_f e_{ij}
    dV.
    \label{eq:final_stress}
\end{equation}
If no strain is present at the interface which is assumed to be true, and $\lambda =1$, then both terms cancels out, and we are left with :  
$\mathscr{S}_{ki}
= 0$. 

\paragraph*{Problem : }
Why does the last expression \ref{eq:final_stress} is null for $\lambda=1$ and \ref{eq:stress} isn't ?
\begin{equation}
    \label{eq:stress}
    \mathscr{S}_{ij}(\lambda=1)
    =\pi \mu_f a^3 \left(
        \frac{7}{3}
    \right)
    \left[
        1+\frac{1}{14}\grad^2
    \right]
    \left(
        \frac{\partial u^\infty_j}{\partial x_i}
        + \frac{\partial u^\infty_i}{\partial x_j}
    \right)
\end{equation}
\bibliography{Bib/bib_bulles.bib}
\end{document}

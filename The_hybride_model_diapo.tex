\documentclass{sintefbeamer}

% packages, font, color, and newcommands
\usepackage{amsfonts, amsmath, oldgerm, lmodern, bm}
% \usepackage[font={footnotesize}]{caption}
\usepackage{natbib}
\usepackage{url}
\usepackage{tikz}
\usepackage{amssymb}
\usepackage{amsmath}
\usepackage{amsthm}
\usepackage{mathrsfs}
\usepackage{empheq}
\usepackage{mdframed}
\usepackage{bm}
\usepackage{animate}




\usepackage{graphicx}
\bibliographystyle{apalike}
\usefonttheme{serif}


% meta-data
\title{The hybrid model for arbitrary dispersed two-phase flows.}
\subtitle{Comparison between particular and \\continuous averaging approach.}
\author{\href{mailto:qilong-kirov.liu@connect.polyu.hk}{Fintzi Nicolas}}
\date{Created on May 22, 2022}

\titlebackground{image/3D/P_PHI_5.png}

% document body

% \newcommand{\size}{0.22\textwidth}
% \newcommand{\avg}[1]{\left<#1\right>}
% \newcommand{\davg}[1]{\left<#1\right>_d}
% \newcommand{\cavg}[1]{\left<#1\right>_c}
% \newcommand{\kavg}[1]{\left<#1\right>_k}
% \newcommand{\Iavg}[1]{\left<#1\right>_I}
% \newcommand{\pavg}[1]{n \left<#1\right>}
% \newcommand{\pnavg}[1]{\left<#1\right>}
% % \newcommand{\nstavg}[1]{\left<#1\right>_{nst}}
% \newcommand{\nstavg}[1]{\overline{#1}_{nst}}
% \newcommand{\nstrelavg}[1]{\overline{#1}_{nst}^{rel}}
% \newcommand{\mavg}[1]{\left<#1\right>_m}
\newcommand{\lavg}[1]{\theta_0\left<#1\right>^\lambda}
% \newcommand{\partials}[1]{\partial_{i_1}\partial_{i_2}\ldots\partial{i_{#1}}}
% \newcommand{\partialp}[2]{ \prod_{m=#1}^{#2} \partial_{i_m}}
% \newcommand{\hatpartialp}[2]{ \prod_{m=#1}^{#2} \hat{\partial}_{j_m}}
% \newcommand{\hatpartialpi}[2]{ \prod_{m=#1}^{#2} \hat{\partial}_{i_m}}
% \newcommand{\pri}[2]{ \prod_{m=#1}^{#2} r_{i_m}}
% \newcommand{\prj}[2]{ \prod_{m=#1}^{#2} r_{j_m}}
\newcommand{\nablab}{\bm{\nabla}}
\newcommand{\nablabh}{\hat{\bm{\nabla}}}
% \newcommand{\ddt}{\frac{d}{d t}}
% \newcommand{\pddt}{\frac{\partial}{\partial t}}


% document body
\begin{document}
\maketitle

\section{Local description of the two-phase flows}

\begin{frame}{Governing equations at the local scale.}
  The fluid in the phase $k$ of the flow follow :
  \begin{equation}
    \pddt f_k
    = \nablabh \cdot \left(
        \bm{\Phi}_k
        - f_k\textbf{u}_k
        \right)
    + \textbf{S}_k,
    \label{eq:general_conservation}
\end{equation}
where,
\begin{itemize}
  \item $f_k$ is an arbitrary quantity, (mass, momentum\ldots).
  \item $\textbf{u}_k$ is the velocity fields present in the phase $k$. 
  \item $\bm{\Phi}_k(f_k)$ is the non-convective term related to $f_k$.
  \item $\textbf{S}_k(f_k)$ is the source term related to $f_k$.
\end{itemize}
\end{frame}

\begin{frame}{The phase indicator function.}
  The phase indicator function (PIF) of the phase $k$ :
  \begin{equation}
    \chi_k(\textbf{y}) =  \left\{
      \begin{tabular}{cc}
        $1 \;\text{if} \;\textbf{y} \in V_k$\\
        $0 \;\text{if} \;\textbf{y} \notin V_k$
      \end{tabular}
      \right.,
      \label{eq:phase_indicator}
\end{equation}
Topological equation of the PIF :
\begin{equation}
  \pddt \chi_k
  + \textbf{u}_I  \nablabh \chi_k 
  = 0, \;\;\;\;\text{and}\;\;\;\;
    \nablabh \chi_k 
    = - \delta_I \textbf{n}_k.
  \label{eq:phaseindicator_transport}
\end{equation}

\begin{itemize}
  \item \textbf{y} local position vector,
  \item $\nablabh = \frac{\partial}{\partial \textbf{y}}$, is the local gradient operator.
  \item  $V_k$ is the volume occupied by the phase $k$.
  \item $\delta_I = \delta(\textbf{y} - \textbf{y}_I)$ with $\textbf{y}_I$ the position of the interface.
  \item $\textbf{n}_k$ normal exterior of phase $k$.
\end{itemize}
\end{frame}


\begin{frame}
  {The two fluid formulation}
  By multiplying the general conservation law with the PIF we get the \textit{two-fluid} formulation (valid over the whole domain) :
  \begin{equation}
    \pddt (\chi_k f_k)
    = \nablabh \cdot (\chi_k \bm{\Phi}_k - \chi_k f_k \textbf{u}_k)
    + \chi_k \textbf{S}_k
    + \underbrace{
    \left[
        \bm{\Phi}_k 
        + f_k 
        \left(
            \textbf{u}_I
            - \textbf{u}_k
        \right) 
    \right]
    \cdot \textbf{n}_k \delta_I }_{\text{Interfacial source term}},
    \label{eq:two-fluid_global}
\end{equation}
With the jump condition, 
\begin{equation}
  \pddt (f_I\delta_I)  
  = 
  + \nablabh \cdot (\delta_I \mathbf{\Phi}^I_{||} - \delta_I f_I \textbf{u}^I)
  +\textbf{S}_I \delta_I
  - \sum_k \left[
  f_k (\textbf{u}_I - \textbf{u}_k)
  + \mathbf{\Phi}_k
  \right] \cdot \textbf{n}_k \delta_I
  \label{eq:general_jump}
\end{equation}
\begin{itemize}
  \item The subscript $_I$ denote the surface quantities
\end{itemize}
\end{frame}

\begin{frame}{The single fluid formulation.}
  Adding (\ref{eq:two-fluid_global}) on each phase $k$ together with (\ref{eq:general_jump}) gives the single fluid formulation :
  \begin{equation}
    \pddt f
    = \nablabh \cdot (\bm{\Phi} - f \textbf{u})
    + \textbf{S}
    \label{eq:one-fluid_global}
\end{equation}
where,
\begin{itemize}
  \item $f = \sum_k f_k \chi_k + \delta_I f_I$ 
  \item $\textbf{S} = \sum_k \textbf{S}_k \chi_k + \delta_I \textbf{S}_I$
  \item $\bm{\Phi} = \sum_k \bm{\Phi}_k \chi_k+ \delta_I \mathbf{\Phi}_I$
\end{itemize}
\end{frame}


\section{Continuous phase averaged equations.}
\begin{frame}{Continuous volume average}
Definition of the volume average, 
  \begin{equation}
    \left<f\right>(\textbf{x},t) = \int g(\textbf{x},\textbf{y}) f(\textbf{y},t)dV,
    \label{eq:avg}
\end{equation}
where this operator follows those rules,
\begin{align}
  \left<f+g\right> = \left<f\right>+\left<g\right>, \;\;\;\;
  \left<\left<f\right>g\right> = \left<f\right>\left<g\right>, \\
  \left<\frac{\partial f}{\partial t}\right> 
  = \pddt\left<f\right>, \;\;\;\;
  \left<\nablabh f\right> 
  = \nablab\left<f\right>. 
  \label{eq:avg_properties}
\end{align}
\begin{itemize}
  \item \textbf{x} global position vector. 
  \item $g(\textbf{x},\textbf{y})$ the weighting function. 
  \item $\nablab = \frac{\partial}{\partial \textbf{x}}$, is the global gradient operator ($\nablabh \neq \nablab$).
\end{itemize}
\end{frame}

\begin{frame}{Continuous averaged equations}
  Continuous average on the phase $k$ : 
  \begin{equation*}
    \pddt (\phi_k\kavg{f})
    = \nablab \cdot \left(
        \phi_k \kavg{\bm{\Phi} - f \textbf{u}}
    \right)
    + \phi_k \kavg{\textbf{S}}
    + a_I \Iavg{
        \bm{\Phi}_k \cdot \textbf{n}_k
        + f_k 
        \left(
            \textbf{u}_I
            - \textbf{u}_k
        \right) \cdot \textbf{n}_k
    }.
    \label{eq:avg_k_global}
\end{equation*}
Averaged jump condition, 
\begin{equation}
  \pddt (a_I\Iavg{f_I})  
  = 
  \nablab \cdot \left(a_I \Iavg{\mathbf{\Phi}_{||I} - f_I \textbf{u}_I}\right)
  +a_I\Iavg{\textbf{S}_I} 
  - a_I \sum_k \Iavg{
  f_k (\textbf{u}_I - \textbf{u}_k)\cdot \textbf{n}_k
  + \mathbf{\Phi}_k\cdot \textbf{n}_k
  }
  \label{eq:avg_general_jump}
\end{equation}
Continuous average on the whole domain, 
\begin{equation*}
  \pddt \avg{f}
  = \nablab \cdot \avg{\bm{\Phi} - f \textbf{u}}
  + \avg{\textbf{S}}
  \label{eq:avg_global}
\end{equation*}
\begin{itemize}
  \item $\phi_k = \int g \chi_k dV$ is the volume fraction of phase $k$. 
  \item $a_I = \int g \delta_I dV$ is the interfacial concentration. 
\end{itemize}
\end{frame}

\begin{frame}
  {Application to mass and momentum equations}
  Mass conservation law for the phase $k$ ($f_k = \rho_k$) :
  \begin{equation}
    \pddt (\phi_k \rho_k)
    + \nablab \cdot \left(\phi_k \rho_k 
        \kavg{\textbf{u}}
    \right) 
    = a_I\Iavg{M_k},
    \label{eq:avg_k_mass}
\end{equation}
  Momentum conservation of the $k$ phase ($f_k = \rho_k \textbf{u}_k$): 
  \begin{equation}
    \pddt (\phi_k\kavg{\rho_k\textbf{u}}) 
    % + \nablab\cdot(\phi_k\kavg{\textbf{uu}})
    = \nablab\cdot\left[
        \phi_k \kavg{\textbf{T}
        - \rho_k \textbf{uu}}
    \right]
    +\phi_k\kavg{\textbf{b}}
    + a_I\Iavg{M_k \textbf{u}_k +\textbf{n}_k\cdot\textbf{T}_k},
\end{equation}
where, 
\begin{itemize}
  \item $M_k = \rho_k (\textbf{u}_k-\textbf{u}_I) \cdot \textbf{n}_k$ is the mass transfer term.
  \item $\textbf{T}_k$ is the stress tensor in phase $k$.
  \item $\textbf{b}_k$ are the body forces in phase $k$. 
\end{itemize}
\end{frame}

\section{A Lagrangian description of the dispersed phase.}

\begin{frame}{Evolution of a single particle's property.}
  For a Lagrangian property $q_\alpha$, where,
  \begin{equation}
    q_\alpha
    = \int_{\Omega_\alpha(t)} f_k(\textbf{y}) d\Omega,
    \label{eq:q_alpha}
\end{equation}
where $\Omega_\alpha$ is defined as, $\Omega_\alpha \subseteq  V_\alpha$.

With the Reynolds transport theorem, we can show that for $\Omega_\alpha = V_\alpha$ we have, 
\begin{align*}
  \ddt  q_\alpha 
  &= \int_{V_\alpha}\left[ \pddt f_k + \nablabh \cdot\left(f_k\textbf{u}_k\right) \right]dV 
    + \int_{S_\alpha} f_k (\textbf{u}_I-\textbf{u}_k)\cdot \textbf{n}_k d S,\\
  &= \int_{V_\alpha} \textbf{S}_k dV 
  + \int_{S_\alpha} \left[\bm{\Phi}_k + f_k (\textbf{u}_I-\textbf{u}_k) \right] \cdot \textbf{n}_k d S.
\end{align*}
\begin{itemize}
  \item $V_\alpha$ volume of the particle $\alpha$.
  \item $S_\alpha$ surface of the particle $\alpha$.
  \item $q_\alpha$ integrated property of the particle $\alpha$ (mass, momentum \ldots)
\end{itemize}
\end{frame}

\begin{frame}{From a Lagrangian to an Eulerian description of the dispersed phase.}
  Any Lagrangian quantity $q_\alpha(t)$ will be represented by the fields $q_\alpha(t)\delta(\textbf{y}-\textbf{y}_\alpha)$.

  Similarly, The Lagrangian quantity $\ddt q_\alpha$ will be represented by the fields $\delta_\alpha \ddt q_\alpha$.

  It can be shown that : 
  \begin{equation*}
    \pddt \delta_\alpha
    + \nablabh (\delta_\alpha \textbf{u}_\alpha)
    = 0,
    \label{eq:delta_q_alpha_dt}
\end{equation*}
\begin{equation*}
    \delta_\alpha \ddt q_\alpha
    = \pddt (\delta_\alpha q_\alpha)
    + \nablabh (\delta_\alpha q_\alpha \textbf{u}_\alpha),
    \label{eq:delta_q_alpha_dt}
\end{equation*}
Besides, using the Reynolds transport theorem we define the velocity of the particle $\alpha$ as,  
\begin{equation*}
  \textbf{u}_\alpha
  = \ddt \textbf{y}_\alpha
  = \frac{1}{m_\alpha} \left(
      \int_{V_\alpha} \rho_k \textbf{u}_k dV
      +  \int_{S_\alpha} \textbf{r} M_k dS
  \right)
  % = \frac{1}{m_\alpha}  \left(
  %     \textbf{p}_\alpha
  % - \int_{V_\alpha} \rho_k \textbf{w} dV
  % \right)
\end{equation*}

If there is more than one particle in the flow, we replace the fields $q_\alpha(t)\delta(\textbf{y}-\textbf{y}_\alpha)$ by its sum on every particle, i.e. : the field, $\sum_k q_\alpha(t)\delta(\textbf{y}-\textbf{y}_\alpha)$.
Notice that the above relations on the derivative still holds.
\end{frame}


\begin{frame}{Particular averaged equations}  
  The particular average is then the volume average of $\sum_k = \delta_\alpha q_\alpha$,
  \begin{equation*}
    \pavg{q}
    = \avg{\sum_\alpha \delta_\alpha q_\alpha} 
    = \int_V g(\textbf{x},\textbf{y}) \sum_\alpha \delta_\alpha(\textbf{y}- \textbf{y}_\alpha) q_\alpha(t) dV 
    =  \sum_\alpha g(\textbf{x},\textbf{y}_\alpha) q_\alpha(t).
\end{equation*}
The averaged Lagrangian balance for an arbitrary quantity $q_k$, yields, 
\begin{equation}
  \pddt   \left(\pavg{q_\alpha}\right)
  + \nablab \cdot \left(\pavg{q_\alpha \textbf{u}_\alpha}\right) 
  = \pavg{\int_{V_\alpha} \textbf{S}_k dV}
  + \pavg{\int_{S_\alpha} \left[\bm{\Phi} + f (\textbf{u}_I-\textbf{u}) \right] \cdot \textbf{n}_k d S}
  \label{eq:avg_p_global}
\end{equation}
\begin{itemize}
  \item $n(\textbf{x})$ is the number density of particle at \textbf{x}.  
  \item In (\ref{eq:avg_p_global}) we used the short hands $q_\alpha \sim \sum_\alpha \delta_\alpha q_\alpha$.
\end{itemize}
\end{frame}

\begin{frame}
  {Application to mass and momentum equations}
  Setting $f_k = \rho_k$ we obtain the mass conservation equation :
  \begin{equation}
    \pddt   \left(\pavg{m_\alpha}\right)
    + \nablab \cdot \left(\pavg{m_\alpha \textbf{u}_\alpha}\right) 
    = 
     \pavg{\int_{S_\alpha} M_k d S}
    \label{eq:avg_p_mass}
\end{equation}

Similarly, if $f_k = \rho_k \textbf{u}_k$ we obtain the momentum conservation equation :
\begin{equation}
    \pddt   \left(\pavg{\textbf{p}_\alpha}\right)
    + \nablab \cdot \left(\pavg{\textbf{p}_\alpha \textbf{u}_\alpha}\right) 
    = \pavg{\int_{V_\alpha} \textbf{b}_k dV}
    + \pavg{\int_{S_\alpha} \left[
      \textbf{T}_k + \rho_k \textbf{u}_k (\textbf{u}_I-\textbf{u}_k) 
      \right] \cdot \textbf{n}_k d S}
    \label{eq:avg_p_global}
\end{equation}
\begin{itemize}
  \item $m_\alpha = \int_{V_\alpha} \rho_k dV$ is the mass of the particle $\alpha$.
  \item $\textbf{p}_\alpha = \int_{V_\alpha} \rho_k \textbf{u}_k dV$ is the momentum of the particle $\alpha$.
  \item $\textbf{T}_k$ is the stress tensor of the fluid (Newtonian flows : $\textbf{T}_k = -p\textbf{I} + \mu (\nablab \textbf{u}+ \nablab \textbf{u}^T) $)
\end{itemize}
\end{frame}
  
\begin{frame}
  {Application to the dipole of mass and momentum}
  The transport of the dipole of mass is obtained by setting $f_k = \rho_k \textbf{rr}$ :
  \begin{equation*}
    \pddt   \left(\pavg{\mathcal{G}_\alpha}\right)
    + \nablab \cdot \left(\pavg{\mathcal{G}_\alpha \textbf{u}_\alpha}\right) 
    = 2 \pavg{\mathcal{S}_\alpha}
    + \pavg{\int_{S_\alpha} \textbf{rr} M_k dS},
\end{equation*}
  Lastly, for $f_k = \rho_k \textbf{r} \textbf{u}_k$ we get the moment of momentum balance :
  \begin{multline}
    \pddt   \left(\pavg{\mathcal{P}_\alpha}\right)
    + \nablab \cdot \left(\pavg{\mathcal{P}_\alpha \textbf{u}_\alpha}\right) 
    = \pavg{\int_{V_\alpha} \left( 
        \textbf{r} \textbf{b}_k 
        - \textbf{T}_k
        + \rho_k \textbf{w}_k  \textbf{w}_k 
    \right) dV}\\
    + \pavg{
    \int_{S_\alpha} \textbf{r} \left[
        \textbf{T}_k \cdot \textbf{n}_k
        + \textbf{w}_k M_k
    \right] dS}.
\end{multline}
\begin{itemize}
  \item $\mathcal{G}_\alpha = \int_{V_\alpha} \rho_k\textbf{rr} dV$ is the dipole of mass of the particle $\alpha$ 
  \begin{itemize}
    \item Equivalent to the classic Inertia tensor for solid particles. 
  \end{itemize}
  \item $\mathcal{P}_\alpha = \int_{V_\alpha} \rho_k \textbf{r}\textbf{u}_k dV$ is the moment of momentum of the particle $\alpha$.
  \item $2\mathcal{S}_\alpha = \int_{V_\alpha} \rho_k (\textbf{r}\textbf{u}_k+\textbf{u}_k\textbf{r}) dV$ is the symmetric part of $\mathcal{P}$. 
  % \item $2\mathcal{A}_\alpha = \int_{V_\alpha} \rho_k (\textbf{r}\textbf{u}_k-\textbf{u}_k\textbf{r}) dV$ is the antisymmetric part of $\mathcal{P}$, which is also the angular momentum. 
\end{itemize}
\end{frame}

\section{Equivalence between particular and continuous averaged equations.}


\begin{frame}{What is the link between both formalism ?}
  \textbf{Continuous average} :
  \begin{equation*}
      \pddt (\phi_k\kavg{f})
      + \nablab \cdot \left(
          \phi_k \kavg{f \textbf{u}}
      \right)
      = \nablab \cdot \left(
          \phi_k \kavg{\bm{\Phi}}
      \right)
      + \phi_k \kavg{\textbf{S}}
      + a_I \Iavg{
          \bm{\Phi}_k \cdot \textbf{n}_k
          + f_k 
          \left(
              \textbf{u}_I
              - \textbf{u}_k
          \right) \cdot \textbf{n}_k
      },
      \label{eq:avg_k_global}
  \end{equation*}
  \begin{center}
    \color{red}
    \textbf{Equal }
    or 
    \textbf{approximately equal with $\mathcal{O}(R/l)$} ?
  \end{center}
  \textbf{Particular average :}
  \begin{equation*}
    \pddt   \left(\pavg{q_\alpha}\right)
    + \nablab \cdot \left(\pavg{q_\alpha \textbf{u}_\alpha}\right) 
    = \pavg{\int_{V_\alpha} \textbf{S}_k dV}
    + \pavg{\int_{S_\alpha} \left[\bm{\Phi}_k + f (\textbf{u}_I-\textbf{u}_k) \right] \cdot \textbf{n}_k d S}
\end{equation*}
\begin{itemize}
  \item \citet{nott2011suspension} prove that those formulations were strictly equivalent the momentum equation, under the hypothesis of :
  \begin{itemize}
    \item Mono-disperse Solid spherical particle suspension.
  \end{itemize}
  \item This is the main argument for existence of the \textbf{particles-fluid-particles stress}.
\end{itemize}

Is this equivalence still true for an arbitrary dispersed phase and any conservation laws ?
\end{frame}


\begin{frame}{Taylor's expansion of the continuous averaged quantities}
Expansion of the weighed function around any center of mass $\textbf{y}_\alpha$,
\begin{equation*}
    g(\textbf{x},\textbf{y})
    = g_\alpha(\textbf{x},\textbf{y}_\alpha)
    - \textbf{r} \cdot \nablab g(\textbf{x},\textbf{y})|_{\textbf{y}_\alpha}
    + \frac{1}{2} \textbf{r}\textbf{r} : \nablab\nablab g(\textbf{x},\textbf{y})|_{\textbf{y}_\alpha}
    + \ldots
\end{equation*} 
Therefore,
\begin{align*}
  \phi_k \kavg{f_k} = \sum_\alpha \int_{V_\alpha} g f_k dV 
  &=  \pavg{q_\alpha}        
      - \nablab \cdot  \left
      (\pavg{Q_k}\right)        
      + \frac{1}{2} \nablab\nablab : \left(\pavg{\textbf{Q}_k^2}\right)
      + \ldots  \\
  a_I \Iavg{f_k} = \sum_\alpha \int_{S_\alpha} g f_k dS 
  &=  \pavg{q_\alpha^I}        
      - \nablab \cdot  \left(\pavg{Q_k^I}\right)        
      + \frac{1}{2} \nablab\nablab : \left(\pavg{\textbf{Q}_k^{I2}}\right)
      + \ldots,
\end{align*}  
where, $Q_k = \int_{V_\alpha} \textbf{r} f_k dV$, and $Q_k = \int_{V_\alpha} \textbf{r} \textbf{r} f_k dV$  are the higher moments of $f_k$.
\begin{itemize}
  \item To analyzes the similarities between both formalism we carry out the expansion of the continuous averaged equation into particular average quantities. 
\end{itemize}
\end{frame}

\begin{frame}  {First order expansion of the non-convective term}
  \begin{equation*}
    \pddt (\phi_k\kavg{f})
    +\nablab \cdot \left(
        \phi_k \kavg{f \textbf{u}}
    \right)
    = \textcolor{red}{\nablab \cdot \left(
        \phi_k \kavg{\bm{\Phi}}
    \right)}
    + \phi_k \kavg{\textbf{S}}
    + a_I \Iavg{
        \bm{\Phi}_k \cdot \textbf{n}_k
        + f_k 
        \left(
            \textbf{u}_I
            - \textbf{u}_k
        \right) \cdot \textbf{n}_k
    },
\end{equation*}
  The $1^{st}$ order expansion of the divergence of the non-convective term yields:
  \begin{equation*}
    \nablab\cdot
    (\phi_k \kavg{\bm{\Phi}})=
        - \nablab \cdot
        \pavg{
          \int_{V_\alpha}
          \bm{\Phi}_k dV,
        }
\end{equation*}
With,
\begin{align*}
    \int_{V_\alpha} \bm{\Phi}_k  dV
    &= \int_{V_\alpha}  \nablabh \cdot \left(\textbf{r} \bm{\Phi}_k\right) dV
    - \int_{V_\alpha} \textbf{r} \nablabh \cdot \mathbf{\Phi}_k dV\\
    &=\int_{S_\alpha} \textbf{r} \bm{\Phi}_k \cdot \textbf{n}_k dS
    + \int_{V_\alpha} \textbf{r} \left[
        \pddt f_k 
        + \nablabh \cdot (f_k \textbf{u}_k)
        - \textbf{S}_k
    \right] dV.\\
\end{align*}
Where we have used the local conservation law :
 $-\nablabh \cdot \bm{\Phi}_k
  = \pddt f_k  + \nablabh \cdot (f_k \textbf{u}_k)
  - \textbf{S}_k$. 
  In agreement with \citet{nott2011suspension} and \citet{prosperetti2004average}.
\end{frame}

\begin{frame}
  {First order expansion of the non-convective term}
From Reynolds transport theorem we deduce that :
\begin{equation*}
  \int_{V_\alpha} \textbf{r} \left[\pddt f_k + \nablabh \cdot (f_k \textbf{u}_k)\right]dV
  = \ddt \int_{V_\alpha} \textbf{r} f_k dV 
  - \int_{V_\alpha} \textbf{r} f_k (\textbf{u}_I   - \textbf{u}_k) \cdot \textbf{n}_k dV
  - \int_{V_\alpha} f_k \textbf{w}_k dV
\end{equation*}
\begin{itemize}
  \item where, $\textbf{u}_k = \textbf{u}_\alpha + \textbf{w}_k$.
\end{itemize}
Besides using the definition of the particular average derivative : 
\begin{equation*}
  \pavg{\ddt \int_{V_\alpha} \textbf{r} f_k dV}
  = \pddt \pavg{\int_{V_\alpha} \textbf{r} f_k dV}
  + \nablab\cdot\left(\pavg{\textbf{u}_\alpha \int_{V_\alpha} \textbf{r} f_k dV}\right)
\end{equation*}
\end{frame}

\begin{frame}
  {First order expansion of the non-convective term}

  \begin{align*}
    \nablab \cdot
    (\phi_k \kavg{\bm{\Phi}})
    & = \nablabh
    \cdot
    \left[
    \pnavg{\int_{S_\alpha}
    \textbf{r}
    (\bm{\Phi}_k \cdot \textbf{n}_k) dS}  \rightarrow \text{Dipole of $\mathbf{\Phi}_k$}
    \right. \\
      &+\pnavg{\int_{V_\alpha}
      \textbf{r}
      \textbf{S}_k dV} \rightarrow \text{Dipole of $\textbf{S}_k$}\\
      &
      + \pnavg{\int_{S_\alpha} 
      f_k\textbf{r}
      \left(\textbf{u}_I - \textbf{u}\right) \cdot \textbf{n}dS}
      \rightarrow \text{Dipole of the transfer term.}\\
      &- \pddt
      \pnavg{\int_{V_\alpha}
      \textbf{r}  f_k dV }\rightarrow \text{Time derivative of the Dipole of $f_k$.}\\
      &- \nablab \cdot\left(
        \pavg{\textbf{u}_\alpha 
        \int_{V_\alpha}
        \textbf{r}  f_k dV}
      \right)\rightarrow \text{Advection of the Dipole of $f_k$.}\\
      & \left. +\int_{V_\alpha}
      f_k
      \textbf{w}_k
      dV\right]\\
  \end{align*}
\end{frame}

\begin{frame}
  \frametitle{Expansion of the other terms}
  \begin{equation*}
    \textcolor{red}{\pddt (\phi_k\kavg{f})}
    +\nablab \cdot \left(
        \phi_k \kavg{f \textbf{u}}
    \right)
    = \nablab \cdot \left(
        \phi_k \kavg{\bm{\Phi}}
    \right)
    + \textcolor{red}{\phi_k \kavg{\textbf{S}}
    + a_I \Iavg{
        \bm{\Phi}_k \cdot \textbf{n}_k
        + f_k 
        \left(
            \textbf{u}_I
            - \textbf{u}_k
        \right) \cdot \textbf{n}_k
    }},
\end{equation*}
  
\begin{equation*}
  \pddt (\phi_k\kavg{f})
  = \pddt \left(\pavg{\int_{V_\alpha} f_k dV}\right)
  - \pddt\nablab \cdot \left(\pavg{\int_{V_\alpha} \textbf{r} f_k dV}\right)
\end{equation*}
\begin{equation*}
  \phi_k\kavg{\textbf{S}}
  =  \pavg{\int_{V_\alpha} \textbf{S}_k dV}
  - \nablab \cdot \left(\pavg{\int_{V_\alpha} \textbf{r} \textbf{S}_k dV}\right)
\end{equation*}
\begin{equation*}
  a_I\kavg{\bm{\Phi}_k \cdot \textbf{n}_k}
  =  \pavg{\int_{S_\alpha} \bm{\Phi}_k \cdot \textbf{n}_k dS}
  - \nablab \cdot \left(\pavg{\int_{S_\alpha} \textbf{r} \bm{\Phi}_k \cdot \textbf{n}_k dS}\right)
\end{equation*}
\begin{equation*}
  a_I\kavg{f_k (\textbf{u}_I - \textbf{u}_k) \cdot \textbf{n}_k}
  =  \pavg{\int_{S_\alpha} f_k (\textbf{u}_I - \textbf{u}_k) \cdot \textbf{n}_k dS}
  - \nablab \cdot \left(\pavg{\int_{S_\alpha} \textbf{r} f_k (\textbf{u}_I - \textbf{u}_k) \cdot \textbf{n}_k dS}\right)
\end{equation*}
\begin{itemize}
  \item Therefore, all the first order terms cancel out with the $\mathbf{\Phi}_k$'s zeroth order terms.
\end{itemize}
\end{frame}



\begin{frame}
  {First order expansion of the advection term}
  \begin{equation*}
    \pddt (\phi_k\kavg{f})
    +\textcolor{red}{\nablab \cdot \left(
        \phi_k \kavg{f \textbf{u}}
    \right)}
    = \nablab \cdot \left(
        \phi_k \kavg{\bm{\Phi}}
    \right)
    + \phi_k \kavg{\textbf{S}}
    + a_I \Iavg{
        \bm{\Phi}_k \cdot \textbf{n}_k
        + f_k 
        \left(
            \textbf{u}_I
            - \textbf{u}_k
        \right) \cdot \textbf{n}_k
    },
    \label{eq:avg_k_global}
\end{equation*}
  The expansion of the divergence of the advection term reads as, 
  \begin{align*}
    \nablab \cdot (\phi_k \kavg{\textbf{u} f})
    &= 
    \nablab \cdot \left(\pavg{
      \textbf{u}_\alpha  \int_{V_\alpha} f_k dV
      + \int_{V_\alpha} \textbf{w}_k f_k dV} \right)\\
    &- 
    \nablab\nablab : 
      \left(\pavg{\textbf{u}_\alpha  \int_{V_\alpha} \textbf{r} f_k dV} \right) +\ldots
\end{align*}
Where the last two terms on the RHS are equivalent to the last terms of the expansion of $\bm{\Phi}_k$.
\begin{itemize}
  \item Consequently, $\nablab \cdot \left(
    \phi_k \kavg{\bm{\Phi}}
\right)$ cancel the first moment of each term leaving with only the zeroth order moments. 
\item \textbf{This demonstration is valid for an arbitrary number of higher order moments.}
\end{itemize}
\end{frame}


\begin{frame}{Equivalence between both averaging methods}
  \textbf{Continuous average} :
  \begin{equation*}
      \pddt (\phi_k\kavg{f})
      = \nablab \cdot \left(
          \phi_k \kavg{\bm{\Phi} - f \textbf{u}}
      \right)
      + \phi_k \kavg{\textbf{S}}
      + a_I \Iavg{
          \bm{\Phi}_k \cdot \textbf{n}_k
          + f_k 
          \left(
              \textbf{u}_I
              - \textbf{u}_k
          \right) \cdot \textbf{n}_k
      },
      \label{eq:avg_k_global}
  \end{equation*}

  \begin{center}
    \begin{equation*}
      \Longleftrightarrow 
    \end{equation*}
  \end{center}
  \textbf{Particular average :}
  \begin{equation*}
    \pddt   \left(\pavg{q_\alpha}\right)
    + \nablab \cdot \left(\pavg{q_\alpha \textbf{u}_\alpha}\right) 
    = \pavg{\int_{V_\alpha} \textbf{S}_k dV}
    + \pavg{\int_{S_\alpha} \left[\bm{\Phi} + f (\textbf{u}_I-\textbf{u}) \right] \cdot \textbf{n}_k d S}
    \label{eq:avg_p_global}
\end{equation*}
\begin{itemize}
  \item Those two equations, are therefore the representation of the \textbf{same} physical problem if $\bm{\Phi}_k \neq 0$.
  \item The second equation provide us with particular quantities, where the first one provides us continuous phase averaged quantities. 
\end{itemize}
\end{frame}

\section{Discussion and conclusion}
\begin{frame}
  {The momentum equations}
  \textbf{Continuous average} :
  \begin{equation*}
      \pddt (\phi_k\kavg{\rho \textbf{u}})
      = \nablab \cdot \left(
          \phi_k \kavg{\textbf{T} - \rho \textbf{u} \textbf{u}}
      \right)
      + \phi_k \kavg{\textbf{b}}
      + a_I \Iavg{
          \textbf{T}_k \cdot \textbf{n}_k
          + \rho \textbf{u}
          \left(
              \textbf{u}_I
              - \textbf{u}_k
          \right) \cdot \textbf{n}_k
      },
  \end{equation*}

  \begin{center}
    \begin{equation*}
      \Longleftrightarrow 
    \end{equation*}
  \end{center}
  \textbf{Particular average :}
  \begin{equation*}
    \pddt   \left(\pavg{\textbf{p}_\alpha}\right)
    + \nablab \cdot \left(\pavg{\textbf{p}_\alpha \textbf{u}_\alpha}\right) 
    = \pavg{\int_{V_\alpha} \textbf{b}_k dV}
    + \pavg{\int_{S_\alpha} \left[
      \textbf{T}_k + \rho_k \textbf{u}_k (\textbf{u}_I-\textbf{u}_k) 
      \right] \cdot \textbf{n}_k d S}
\end{equation*}
\begin{itemize}
  \item Those two equations are therefore \textbf{rigorously equivalent} thus using the particular average makes no error. 
\end{itemize}
\end{frame}
\begin{frame}
  {Mass transport equations}
  \textbf{Continuous average} :
  \begin{equation}
    \pddt (\phi_k \rho_k)
    + \nablab \cdot \left(\phi_k \rho_k 
        \kavg{\textbf{u}}
    \right) 
    = a_I\Iavg{M_k},
\end{equation}

  \begin{center}
    \begin{equation*}
      \Longleftrightarrow 
    \end{equation*}
  \end{center}
  \textbf{Particular average :}
  \begin{equation}
    \pddt   \left(\pavg{m_\alpha}\right)
    + \nablab \cdot \left(\pavg{m_\alpha \textbf{u}_\alpha} 
    + \frac{1}{2}\nablab\nablab : 
    \textcolor{red}{\pavg{ \mathcal{G}_\alpha\textbf{u}_\alpha}}
    + \ldots
    \right) 
    = -\pavg{\int_{S_\alpha} M_c d S},
    \label{eq:massexp}
\end{equation}

\begin{itemize}
  \item Due to the absence of the non-convective term in the mass balance equation we see appear additional terms linked to the shape of the particle. 
  \item Besides, subtracting (\ref{eq:massexp}) with the original mass balance gives the following constrain, $\nablab\nablab\nablab \vdots \left(\pavg{ \mathcal{G}_\alpha\textbf{u}_\alpha}\right) = 0$
  \item It is more likely to be negligible except for highly inhomogeneous flows. 
\end{itemize}
\end{frame}

\begin{frame}
  {Area transport equations}
  \textbf{Continuous average} :
  \begin{equation}
    \pddt a_I
    + \nablab \cdot \left(a_I \textbf{u}_I
    \right) 
    = a_I\Iavg{\nablab \cdot \textbf{n} (\textbf{u}_I \cdot \textbf{n})},
\end{equation}

  \begin{center}
    \begin{equation*}
      \Longleftrightarrow 
    \end{equation*}
  \end{center}
  \textbf{Particular average :}
  \begin{equation}
    \pddt   \left(\pavg{A_\alpha}\right)
    + \nablab \cdot \left(\pavg{A_\alpha \textbf{u}_\alpha} 
    \textcolor{red}{+ \pavg{\int_{S_\alpha} \textbf{u}_I \cdot \textbf{nn} dS}}
    +\ldots
    \right) 
    = \pavg{\int_{S_\alpha} \nablab \cdot \textbf{n} (\textbf{u}_I \cdot \textbf{n}) dS}
\end{equation}

\begin{itemize}
  \item $A_\alpha = \int_{S_\alpha} dS$ is the surface of the particle $_\alpha$.  
  \item This additional term is due to the shift between mass center and surface center (cancel for spherical particles\ldots). 
\end{itemize}

\end{frame}


\begin{frame}[t]
  \frametitle{References}
  \bibliography{Bib/bib_bulles.bib}
\end{frame}
\backmatter


\begin{frame}  {Expansion of the non-convective term}
  The $n^{th}$ order expansion of the divergence of the non-convective term yields:
  \begin{equation}
    \nablab\cdot
    (\phi_k \kavg{\bm{\Phi}})=
    \sum_l^\infty
    \left[
        \frac{(-1)^{l}}{l!}
        \nablab^{n+1}
        \sum_{\alpha}
        g_{\alpha}
        \int_{V_\alpha}
        \prod^{l}_{m=1}
        r_{i_m} \bm{\Phi}dV
    \right],
\end{equation}
With,
\begin{equation}
    (l+1) \prod^{l}_{m=1} r_{i_m} \bm{\Phi}
    =\nablabh \cdot \left(\prod^{l+1}_{m=1} r_{i_m} \bm{\Phi}\right)
    - \prod^{l+1}_{m=1} r_{i_m} \left(
        \pddt f 
        + \nablabh \cdot (f \textbf{u})
        - \textbf{S}
    \right).
\end{equation}
In agreement with \citet{nott2011suspension} and \citet{prosperetti2004average}.
\end{frame}

\begin{frame}
  {Expansion of the non-convective term}
  After the use of the divergence theorem for the first term and the Reynolds transport theorem for the second term we get :
  \begin{align*}
    \nablab \cdot
    (\phi_k \kavg{\bm{\Phi}})
    & =\sum_l^\infty
    \frac{(-1)^{n}}{(n+1)!}
    \nablab^{n+1}
    \cdot
    \sum_{\alpha}
    g_{\alpha} \\
  &\left[
    \int_{S_\alpha}
    \textbf{r}^{n+1}
    (\bm{\Phi}_k \cdot \textbf{n}_k) dS
    \right.
      +\int_{V_\alpha}
      \textbf{r}^{n+1}
      \textbf{S}_k dV \rightarrow \text{    Moments of $\textbf{S}_k$ and $\bm{\Phi}_k$.}\\
      &- \pddt
      \int_{V_\alpha}
      \textbf{r}^{n+1}  f_k dV \rightarrow \text{Time derivative moments of $f_k$.}\\
      &
        + \int_{S_\alpha} 
          f_k\textbf{r}^{n+1} 
          \left(\textbf{u}_I - \textbf{u}\right) \cdot \textbf{n}dS
          \rightarrow \text{Moments of the transfer term.}\\
      &- \nablab \cdot\left(
        \textbf{u}_\alpha 
        \int_{V_\alpha}
        \textbf{r}^{n+1}  f_k dV
      \right)
      \left.+\frac{1}{l+1}\int_{V_\alpha}
      f\sum_{e=1}^{m=1} 
      \prod^{l+1}_{\substack{m=1\\ m\neq e}} 
      r_{i_m} 
      w_{i_e}
      dV\right]\\
  \end{align*}
\end{frame}

\begin{frame}
  {Expansion of the advection term}
  The expansion of the divergence of the advection term reads as, 
  \begin{align*}
    \nablab \cdot \phi_k \kavg{\textbf{u} f}
    &= \sum_{l=0}^\infty  
    \frac{(-1)^l}{l!} 
    \nablab^{n+1} \cdot
    \sum_\alpha  g_\alpha 
    \left[
      \textbf{u}_\alpha  \int_{V_\alpha} \textbf{r}^{n} f dV
    + \int_{V_\alpha} \prod^l_{m=1}r_{i_m} \textbf{w} f dV
    \right]
    \label{ap:eq:partial_uf}
\end{align*}
where, $\textbf{u} = \textbf{u}_\alpha + \textbf{w}$.
Where the first term on the RHS is equivalent to the last term of the expansion of $\bm{\Phi}_k$.
\end{frame}

\begin{frame}
  {Difference of the $\bm{\Phi}$ and $\textbf{u}f$ series}
If we carry out the differences of the remaining moments we are left with the series :
\begin{equation*}    
  \sum_{l=0}^\infty  
  \left[
      \frac{(-1)^l}{l!} \prod^{l+1}_{m=1}\partial_{i_m}
      \sum_\alpha  g_\alpha 
      \int_{V_\alpha} f
      \underbrace{\left(
          \prod^l_{m=1}r_{i_m} w_{i_{l+1}} 
          -
          \underbrace{\frac{1}{l+1}
          \sum_{e=1}^{m=1} 
          \prod^{l+1}_{\substack{m=1\\ m\neq e}} 
          r_{i_m} 
          w_{i_e}}_{\text{Symmetric part}}
      \right)}_{Anti symmetric tensor}
      dV
  \right] = 0
  \label{ap:eq:diff_rw_term}
\end{equation*}  
\begin{itemize}
  \item As this term is antisymmetric over the indices $i_1i_2\ldots i_{l+1}$ the tensor vanish under the operator $\partialp{1}{l+1}$
\end{itemize}
\end{frame}

\end{document}


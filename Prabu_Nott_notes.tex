% \documentclass[aip,pof,preprint]{revtex4-1}
\documentclass[aip,pof,preprint]{report}
%\documentclass[aps,showpacs,referee]{revtex4}
\usepackage[sort&compress]{natbib}
\bibliographystyle{aipnum4-1}
\newcommand{\ter}[1]{\mbox{$\bf #1$}}
\newcommand{\utheta}{u_{\raisebox{-0.15em}{\scriptsize \hspace*{-1ex} $\theta$}}}
%
%\documentclass[referee]{jfm}
%\documentclass{jfm}
%\documentclass[aps,showpacs,preprint]{revtex4}
\usepackage{amsmath}
\usepackage[dvips]{graphicx}
\usepackage[debug]{psfrag}
\usepackage{yfonts}
\usepackage{mathrsfs}
\usepackage{color}
\usepackage{cancel}
\renewcommand*{\CancelColor}{\color{red}}
%\bibliographystyle{unsrt}
%\bibliographystyle{jfm}
%\newcommand{\mcite}{\cite}
%\newcommand{\citep}{\cite}
%\newcommand{\citealp}{\cite}
%\newcommand{\citet}{\cite}
\newcommand{\margintext}[1]{\marginpar{\it \scriptsize #1}}
\newcommand{\etal}{\emph{et al.}}
\newcommand{\te}[1]{\mbox{\boldmath $#1$}}
\newcommand{\Frac}[2]{\frac{\displaystyle #1}{\displaystyle #2}}
\newcommand{\be}{\begin{equation}}
\newcommand{\ee}{\end{equation}}
\newcommand{\bea}{\begin{eqnarray}}
\newcommand{\eea}{\end{eqnarray}}
\newcommand{\Div}{\te{\nabla \cdot}}
\newcommand{\Divx}{\te{\nabla}_{\! x} \bdot}
\newcommand{\Grad}{\te{\nabla}}
\newcommand{\mysqrtp}[1]{{\left(#1\right)}^{1/2}}
\newcommand{\mysqrt}[1]{{#1}^{1/2}}
\newcommand{\der}[2]{\Frac{\mathrm{d}#1}{\mathrm{d}#2}}
\newcommand{\pder}[2]{\frac{\displaystyle\partial#1}{\displaystyle\partial#2}}
\newcommand{\textfrac}[2]{\ensuremath{#1/#2}}
\newcommand{\textfracp}[2]{\ensuremath{(#1/#2)}}
\newcommand{\Modulus}[1]{\ensuremath{\lvert#1\rvert}}
%\newcommand{\ol}{\overline}
\newcommand{\subfr}{_{\mbox{\scriptsize fr}}}
\newcommand{\ttimes}{\mbox{$\scriptstyle \times$}}
%\newcommand{\Dot}{\mbox{$\te{\cdot}$}}
%\newcommand{\eqref}[1]{(\ref{#1})}
\newcommand{\bdot}{\te{\cdot}}
\newcommand{\figref}[1]{Fig.~\ref{#1}}
\newcommand{\eqsref}[2]{(\ref{#1},\ref{#2})}
\newcommand{\eqssref}[2]{(\ref{#1}--\ref{#2})}
\newcommand{\eqrefp}[1]{eq.~\ref{#1}}
\newcommand{\eqsrefp}[2]{eqs~\ref{#1} and \ref{#2}}
\newcommand{\eqssefp}[2]{eqs~\ref{#1}--\ref{#2}}
\newcommand{\avg}[1]{\langle #1 \rangle}
\newcommand{\subtext}[1]{\mbox{\scriptsize #1}}
%\newcommand{\rhop}{\rho_{\subtext{p}}}
\newcommand{\varint}[2]{\int\limits_{\mbox{\scriptsize $#1$}}^{\mbox{\scriptsize $#2$}}\!}
\newcommand{\dup}{{\mathrm d}}
\newcommand{\matderiv}[1]{\Frac{{\mathrm D} #1}{{\mathrm D} t}}
\newcommand{\matderivhat}[1]{\Frac{\hat{\mathrm D} #1}{\hat{\mathrm D} t}}
\newcommand{\rhop}{\rho^{\subtext{p}}}
\newcommand{\vp}{v^{\subtext{p}}}
\newcommand{\rhof}{\rho^{\subtext{f}}}
\newcommand{\transpose}[1]{#1^{\mbox{\scriptsize T}}}
\newcommand{\cross}[2]{#1 \mbox{\small \te{\times}} #2}
\newcommand{\partialhat}{\hat{\partial}}
\newcommand{\ssum}[2]{\mbox{$\sum_{#1}^{#2}$}}
\newcommand{\vthinsp}{\mbox{\tiny $\,$}}
\newcommand{\vnthinsp}{\mbox{\tiny $\!$}}
\newcommand{\Gradhat}{\hat{\Grad}}
\newcommand{\Divhat}{\Gradhat \bdot}
\newcommand{\gamdot}{\dot{\gamma}}
\newcommand{\Sh}{\overline{\te{S}}^{\vthinsp \subtext{h}}}
\newcommand{\Sigmap}{\te{\Sigma}^{(\subtext{p})}}
\newcommand{\Sigmahp}{\te{\Sigma}^{\subtext{h(p})}}
\newcommand{\sigmas}{\avg{\te{\sigma}}^{\subtext{s}}}
\newcommand{\sigmaf}{\avg{\te{\sigma}}^{\subtext{f}}}
\newcommand{\sigmap}{\avg{\te{\sigma}}^{\subtext{p}}}
\newcommand{\sigmacp}{\avg{\te{\sigma^{\subtext{c}}}}^{\subtext{p}}}
\newcommand{\sigmahp}{\avg{\te{\sigma^{\subtext{h}}}}^{\subtext{p}}}
\newcommand{\fp}{\avg{\te{f}}^{\subtext{p}}}
\newcommand{\fhp}{\avg{\te{f}^{\subtext{h}}}^{\subtext{p}}}
\newcommand{\fcp}{\avg{\te{f}^{\subtext{c}}}^{\subtext{p}}}
\newcommand{\fipp}{\avg{\te{\sigma}^{\subtext{ip}}}^{\subtext{p}}}
\newcommand{\fdrag}{\avg{\te{f}^{\subtext{h}}}_{\subtext{drag}}}
\newcommand{\tehat}[1]{\hat{\te{#1}}}
\begin{document}


\appendix

\section{Volume-averaged momentum balances including inertial contributions}
\label{sec-appA}

	Here we derive the complete volume-averaged momentum balances, including the effects of particle and fluid inertia.  The suspension momentum balance is obtained by multiplying \eqref{eqns-point_mom} by $g$ and integrating over the entire volume,
\be
	\int g \rho \, \matderivhat{\te{u}} \, \dup V = {\color{red} \int g \Divhat \te{\sigma} \, \dup V} + \int g\, \te{b} \, \dup V.
\label{eqn-appA_suspn_mom_bal1}
\ee
The left hand side may be simplified using \eqref{eqns-point_contin}, to yield
\be
	\pder{}{t} \int \!  g \rho \te{u} \, \dup V + \int \! \Divhat ( g \rho \te{u} \te{u}) \, \dup V - \int \! \rho \te{u} \te{u} \bdot \Gradhat g \, \dup V,
\label{eqn-appA_suspn_mom_bal2}
\ee
The second term vanishes on application of the divergence theorem and using \eqref{eqn-g_bc}.  Using \eqref{eqn-reln_gradg}, \eqref{eqn-appA_suspn_mom_bal2} reduces to
\be
	\pder{}{t} \avg{\rho \te{u}} + \Div \avg{\rho \te{u u}}.
\label{eqn-appA_suspn_mom_bal3}
\ee
The integration of the right hand side proceeds in the same manner as in \S \ref{sec-vol_avg}; combining \eqref{eqn-appA_suspn_mom_bal3} with the right hand side of \eqref{eqn-suspn_mom_bal4}, the volume-averaged suspension momentum balance with allowance for inertia is
\be
	\pder{}{t} \avg{\rho \te{u}} + \Div \avg{\rho \te{u u}} = \Div \left[(1-\phi) \sigmaf + \phi \sigmas \right]  + \avg{\te{b}}.
\label{eqn-appA_suspn_mom_bal4}
\ee
The expressions for the fluid and solid phase stresses are given in \eqref{eqn-fluid_stress} and \eqref{eqn-suspn_mom_bal5} respectively, which are
\be
	(1-\phi) \sigmaf = - (1-\phi) \avg{p}^{\subtext{f}} \te{I} + 2 \eta \avg{\te{e}},
\label{eqn-appA_fluid_stress}
\ee
and
\be
	\phi \sigmas = \sum_{i} g_i \! \varint{V^i}{} \te{\sigma} \, \dup V -  \Grad \bdot \sum_{i} g_i \varint{V^i}{} \! \te{y}' \te{\sigma} \, \dup V + \cdots.
\label{eqn-appA_solid_stress1}
\ee
As in \S \ref{sec-vol_avg}, we transform every term in \eqref{eqn-appA_solid_stress1} using \eqref{eqn-transform} and \eqref{eqn-B5} to bring them to a more useful form.  Inertia has no direct effect on the the surface integrals in the transormed relation, but alters the volume integrals when we substitute $\Divhat \te{\sigma} = \rho \hat{\mathrm{D}} \te{u}/\hat{\mathrm{D}} t - \te{b}$.  The pointwise velocity in a rigid particle is $\te{u} = \te{u}_i + \cross{\te{\omega}_i}{\te{r}}$, where $\te{u}_i$ and $\te{\omega}_i$ are the translational and angular velocities, respectively, of its center.  As a result, $\hat{\mathrm{D}} \te{u}/\hat{\mathrm{D}} t = \dot{\te{u}}_i + \cross{\dot{\te{\omega}}_i}{\te{y}'} + \te{\omega}_i \,\te{\omega}_i \bdot \te{y}' - \te{\omega}_i\bdot \te{\omega}_i \, \te{y}'$, and hence the first term in \eqref{eqn-appA_solid_stress1} takes the form
\bea
	\sum_i g_i \! \varint{V^i}{} \te{\sigma} \, \dup V & = & \sum_i g_i \! \varint{S^i}{} \! \te{y}' \, \te{n} \bdot \te{\sigma} \, \dup S - \sum_i g_i \! \varint{V_i}{} \te{y}'  \left[ \rule{0em}{1em} (\rho \dot{\te{u}}_i - \te{b}) + \rho(\cross{\dot{\te{\omega}}_i}{\te{y}'}  \right. \nonumber \\
	& & \left. + \,\, \te{\omega}_i \,\te{\omega}_i \bdot \te{y}' - \te{\omega}_i \bdot \te{\omega}_i \, \te{y}') \rule{0em}{1em} \right] \, \dup V.
\label{eqn-appA_suspn_mom_bal5}
\eea
Substituting \eqref{eqn-appA_suspn_mom_bal5} and the similarly simplified higher order terms in \eqref{eqn-appA_solid_stress1}, we get
\bea
	\phi \sigmas & = & \sum_{i}  g_i \te{S}_i   - \Frac{1}{2} \Div \sum_{i}  g_i \te{Q}_i  + \, \cdots \nonumber \\
	& & + \, \sum_{i}  g_i \left( \, \varint{V^i}{} \te{y}' (\rho \dot{\te{u}}_i - \te{b}) \, \dup V + \frac{\mathcal{I}}{2} ( -\te{\epsilon} \bdot \dot{\te{\omega}}_i + \te{\omega}_i \bdot \te{\omega}_i \, \te{I} - \, \te{\omega}_i \te{\omega}_i ) \right)  \nonumber \\
	& & + \Frac{1}{2} \Div \left( \sum_{i}  g_i \varint{V_i}{} \te{y}' \te{y}' (\rho \dot{\te{u}}_i - \te{b}) \, \dup V \right) + \cdots,
\label{eqn-appA_solid_stress2}
\eea
where $\mathcal{I} \! \equiv \!  \frac{2}{5} \rhop  \vp a^2$ is the moment of inertia of the spherical particles, and we have followed the notation for the surface moments defined in \eqref{eqn-stress_moments}.  In \eqref{eqn-appA_solid_stress2}, the first line contains terms arising from the moments of the surface traction, namely the first term on the right-hand-side of \eqref{eqn-suspn_mom_bal6} and the first term in \eqref{eqn-B5}; the subsequent lines contains terms arising from the volume moments of the inertia and body forces, namely the second term on the right-hand-side of \eqref{eqn-suspn_mom_bal6} and the second term in \eqref{eqn-B5}.  Decomposing the body force density in particle $i$ as the sum of its mean $\te{b}_i$ and a deviation $\te{b}'$, and using of the equations of linear and angular momentum for the particle,
\be
	m \dot{\te{u}}_i  = \te{f}_i + \te{b}_i v^{\subtext{p}},  \quad \mathcal{I} \dot{\te{\omega}}_i  =  - \te{\epsilon} \, \te{:} \! \varint{S_i}{} \te{y}' (\te{n} \bdot \te{\sigma}) \, \dup S + \te{\tau}_i,
\label{eqn-appA_part_eqns_motion}
\ee
where $\te{\tau}_i$ is the net external torque on particle $i$, \eqref{eqn-appA_solid_stress2} may be simplified to
\bea
	\phi \sigmas & = & \sum_{i}  g_i \left( \overline{\te{S}}_i - \frac{1}{2} \te{\epsilon} \bdot \te{\tau}_i + \frac{1}{2} \mathcal{I} (\te{\omega}_i \bdot \te{\omega}_i  \te{I} - \te{\omega}_i \te{\omega}_i) + \varint{V^i}{} \te{y}' \te{b}' \, \dup V \right)  \nonumber \\
	& &  - \,  \Frac{1}{2} \Div \sum_{i}  g_i \left( \te{Q}_i  - \Frac{1}{5} \vp a^2 \te{I} \te{f}_i + \varint{V_i}{} \te{y}' \te{y}' \te{b}' \, \dup V \right) +  \cdots.
\label{eqn-appA_solid_stress}
\eea
In the derivation of \eqref{eqn-appA_solid_stress2} and \eqref{eqn-appA_solid_stress}, we have used the identities $\int \! \te{y}' \, \dup V \! = \! 0$ and$\varint{V_i}{} \! \te{y}' \te{y}' \, \rho \, \dup V \! = \! \frac{1}{5} \rhop  \vp a^2 \te{I}$.

	Substituting \eqref{eqn-appA_fluid_stress} and \eqref{eqn-appA_solid_stress} in \eqref{eqn-appA_suspn_mom_bal4}, the momentum balance for the suspension takes the form
\bea
	\pder{}{t} (\avg{\rho} \avg{\te{u}}_{\subtext{m}}) + \Div \left( \avg{\rho} \avg{\te{u}}_{\subtext{m}} \avg{\te{u}}_{\subtext{m}} \right) & = &   \avg{\te{b}} - \Grad \left[(1-\phi) \avg{p}^{\subtext{f}} \right] + 2 \eta \Div \avg{\te{e}} \nonumber \\
	& & + \, \Div \left[ \phi \sigmas + \avg{\rho \te{u}' \te{u}'} \right].
\label{eqn-appA_suspn_mom_bal}
\eea
where $\avg{\te{u}}_{\subtext{m}}$ is the mass-averaged suspension velocity, defined in \eqref{eqn-mass_wtd_vel}, and $\te{u}' \equiv \te{u} - \avg{\te{u}}_{\subtext{m}}$ is the deviation of the local velocity from $\avg{\te{u}}_{\subtext{m}}$.  The sum of $\phi \sigmas$ and the Reynold's stress $\avg{\rho \te{u}' \te{u}'}$ is the ``particle stress'' $\Sigmap$, introduced by \citet{batchelor70} as the particle contribution to the suspension stress.

	Next, we determine the momentum balance for the solid phase by multiplying \eqref{eqns-point_mom} by $\chi g$ and integrating over the entire volume.  Using \eqref{eqn-matder_chi}, \eqref{eqn-grad_chi} and \eqref{eqns-point_contin}, and employing the same manipulations as in \eqref{eqn-appA_suspn_mom_bal2}, we get 
\be
	\rhop \pder{}{t} (\phi \avg{\te{u}}^{\subtext{s}}) + \rhop \Div (\phi \avg{\te{u u}}^{\subtext{s}})  = \Div (\phi \sigmas) + \sum_{i} \varint{S_i}{} \! \te{n} \bdot \te{\sigma} g \, \dup S + \phi \avg{\te{b}}^{\subtext{s}}.
\label{eqn-appA_solid_mom_bal1}
\ee
To evaluate the integral in \eqref{eqn-appA_solid_mom_bal1}, we substitute the Taylor expansion of $g$ from \eqref{eqn-g_expn}; the result is
\be
	\sum_{i} \varint{S_i}{} \! \te{n} \bdot \te{\sigma} g \, \dup S = \sum_{i} g_i \te{f}_i - \Div \sum_{i} g_i \te{S}_i + \Frac{1}{2} \Grad \Grad \te{:} \sum_{i} g_i \te{Q}_i \cdots
\label{eqn-appA_ndotsigma}
\ee
%Substituting this expansion in \eqref{eqn-appA_solid_mom_bal1}, we see immediately that all the terms, except the first, are cancelled by the terms of $\Div (\phi \sigmas)$ coming from the first line in \eqref{eqn-appA_solid_stress2}.
On substituting in \eqref{eqn-appA_solid_mom_bal1} the expansions for  $\Div (\phi \sigmas)$ from \eqref{eqn-appA_solid_stress2} and $\sum_{i} \varint{S_i}{} \! \te{n} \bdot \te{\sigma} g \, \dup S$ from \eqref{eqn-appA_ndotsigma}, it is immeadiately clear that all the terms in the latter, except the first, are cancelled by the terms in the first line of the former.  With the cancellations, the solid phase momentum balance reduces to
\bea
	\rhop \pder{}{t} (\phi \avg{\te{u}}^{\subtext{s}}) + \rhop \Div (\phi \avg{\te{u u}}^{\subtext{s}})  & = &   \phi \avg{\te{b}}^{\subtext{s}} + \sum_{i} g_i \te{f}_i + \Div \left[ \sum_{i}  g_i \left( \, \varint{V^i}{} \te{y}' \te{b}' \, \dup V + \frac{\mathcal{I}}{2} ( -\te{\epsilon} \bdot \dot{\te{\omega}}_i + \, \te{\omega}_i \bdot \te{\omega}_i \, \te{I} \right. \right. \nonumber \\
	& & \left. \left. - \, \te{\omega}_i \te{\omega}_i ) \rule[-0.75em]{0em}{2.5em} \right) + \, \Frac{1}{2} \Div \sum_{i}  g_i \varint{V_i}{} \te{y}' \te{y}' (\rho \dot{\te{u}}_i - \te{b}) \, \dup V + \cdots \right],
\label{eqn-appA_solid_mom_bal}
\eea
Though \eqref{eqn-appA_solid_mom_bal} is adequate, it can be simplified considerably in the following way.  As noted earlier in this section, the terms within the square brackets are volume moments about the particle centers of $\rho D \te{u}/Dt - \te{b}$, starting from the first moment.  The left hand side of \eqref{eqn-appA_solid_mom_bal} and the body force $\phi \avg{\te{b}}^{\subtext{s}}$ may also be written as the sum of moments, by replacing $g$ with its Taylor expansion \eqref{eqn-g_expn}.  We then find that all moments, except the zeroth, of the left hand side and the body force cancel exactly with the terms within the square brackets of \eqref{eqn-appA_solid_mom_bal}, yielding
\be
	\rhop v^{\subtext{p}} \pder{}{t} \left( \sum_{i} g_i \te{u}_i \right) + \rhop v^{\subtext{p}} \, \Div \left( \sum_{i} g_i \te{u}_i \te{u}_i \right) = v^{\subtext{p}} \sum_{i} g_i \te{b}_i + \sum_{i} g_i \te{f}_i
\label{eqn-appA_part_mom_bal1}
\ee
{\color{red} from jackson 1997 : 
\begin{multline}
	\rhop (\phi \avg{u_i u_k}^{\subtext{s}})
	= 
	\rhop (\phi \avg{u_i u_k}^{\subtext{p}})
	+ \sum_i g_i\frac{\mathcal{I}}{2}\left(
		\omega_i\omega_k - \delta_{ik}\omega_l\omega_l
	\right)\\
	+ \epsilon_{ilj}\partial_j\sum_i g_i u_k  \omega_l
	+ \epsilon_{klj}\partial_j\sum_i g_i u_i  \omega_l
	+ O(a^2/L^2)	
	% + \mathcal{I} \partial_j\partial_j \sum_i g_i u_i u_k
\end{multline}
then if we go above the second order terms we get,
\begin{multline}
	\rhop (\phi \avg{u_i u_k}^{\subtext{s}})
	= 
	\rhop (\phi \avg{u_i u_k}^{\subtext{p}})
	+ \sum_i g_i\frac{\mathcal{I}}{2}\left(
		\omega_i\omega_k - \delta_{ik}\omega_l\omega_l
	\right)\\
	+ \epsilon_{ilj}\partial_j\sum_i g_i u_k  \omega_l
	+ \epsilon_{klj}\partial_j\sum_i g_i u_i  \omega_l
	+ \mathcal{G} \partial_j\partial_j \sum_i g_i u_i u_k
	+ \ldots
\end{multline}
and so on. 
Where, $\te{\mathcal{G}} = \int_V \te{y}'\te{y}'dV$. 
Then, from this development it is obvious that the terms $\omega_i\omega_k - \delta_{ik}\omega_l\omega_l$ cancel with the ones of the stress expansion derived above. 
The 3rd terms cancel out due to the operator $\partial_j\partial_k$ since $\epsilon_{ilj}\omega_l $ is antisymmetric on these indices. 
However, I cannot understand how the $u_i u_k$ terms at higher order and the $ u_i  \omega_l$ products cancel out with the expansion of the stress. 
}

Decomposing $\vp \te{b}_i$ into an external body force $\vp \te{b}_i^{\subtext{ext}}$ and an `action at a distance' inter-particle force $\te{f}_i^{\subtext{ip}}$, and recognizing that every term in this equation is a particle phase average (defined in \eqref{eqn-particle_phase_avg_defn}), we get
\be
	\rhop \vp \! \left[ \pder{}{t} (n \avg{\te{u}}^{\subtext{p}}) 
	+ \Div (n \avg{\te{u} \te{u}}^{\subtext{p}}  
	{\color{red}
	+\mathcal{G}\Div\avg{\te{u}\te{\omega}}
	+\mathcal{G}\Delta\avg{\te{u}\te{u}}...}
	) \right]  = n \vp \avg{\te{b}^{\subtext{ext}}}^{\subtext{p}} + n  \fp.
\label{eqn-appA_part_mom_bal}
\ee
{\color{red} where i just added the second order terms that i believe do not cancel}
This is clearly the momentum balance for the particle phase, as the left hand side is the average rate of change of momentum of the particle phase.  As we have shown above, it is simply another form of the momentum balance for the solid phase \eqref{eqn-appA_solid_mom_bal}.


\section{Transformation of volume integrals of the stress moments}
\label{appendix-B}
As we are interested in the divergence of $\phi \sigmas$ and not the stress itself, we consider the divergence of the $n^{\scriptsize \mbox{th}}$ term  ($n > 1$) in the right hand side of \eqref{eqn-suspn_mom_bal5}.  The manipulations are greatly eased by adopting the index notation, with the Einstein convention of  summing over repeated indices.  The $n^{\scriptsize \mbox{th}}$ term then is
\be
	\Frac{(-1)^{(n-1)}}{(n-1)!} \partial_{p_1} \partial_{p_2} \cdots \partial_{p_{n-1}} \partial_{p_{n}} \sum_{i} g_i \varint{V_i}{} y'_{p_1} y'_{p_2} \cdots y'_{p_{n-1}} \sigma_{{p_n} q} \, \dup V,
\label{eqn-B1}
\ee
where we have used the notation $\partial_k \equiv \partial/\partial x_k$.  To bring this to the desired form, we rewrite the integrand as
\be
	\prod_{r=1}^{n-1} y'_{p_r} \sigma_{p_n q} = \Frac{1}{n} \left[ \sum_{m=1}^{n} \sigma_{p_m q} \hspace*{-1.5ex} \prod_{\scriptsize \rule{1em}{0em}\parbox{3em}{\baselineskip=0pt $r \! = \!1$\\$r \! \ne \! m$}}^{n} \hspace*{-1.5ex}  y'_{p_r}  +  \sum_{m=1}^{n-1} (y'_{p_m} \sigma_{p_n q} - y'_{p_n} \sigma_{p_m q}) \hspace*{-1.5ex} \prod_{\scriptsize \rule{1em}{0em}\parbox{3em}{\baselineskip=0pt $r \! = \!1$\\$r \! \ne \! m$}}^{n-1} \hspace*{-1.5ex}  y'_{p_r} \right].
\label{eqn-B2}
\ee
Note that each term in the second sum on the right hand side of \eqref{eqn-B2} is anti-symmetric in the indices $p_m,p_n$.  Hence the integral too is anti-symmetric in these indices, and will vanish upon the operation $\partial_{p_m} \partial_{p_n}$.  As a result, only the terms in the first sum of \eqref{eqn-B2} are of consequence; this sum may be written as
\be
	\sum_{m=1}^{n} \sigma_{p_m q} \hspace*{-1.5ex} \prod_{\scriptsize \rule{1em}{0em}\parbox{3em}{\baselineskip=0pt $r \! = \!1$\\$r \! \ne \! m$}}^{n} \hspace*{-1.5ex} y'_{p_r} = \partial_m \left( \prod_{r = 1}^{n} y'_{p_r} \sigma_{m q} \right) - \prod_{r = 1}^{n} y'_{p_r} \partial_m\sigma_{m q}
\label{eqn-B3}
\ee
The expression in \eqref{eqn-B1} then takes the form
\be
	- \Frac{(-1)^{n}}{n!} \partial_{p_1} \partial_{p_2} \cdots \partial_{p_{n-1}} \partial_{p_{n}} \sum_{i} g_i \varint{V_i}{} \left[ \partial_m \left( \prod_{r = 1}^{n} y'_{p_r} \sigma_{m q} \right) - \prod_{r = 1}^{n} y'_{p_r} \partial_m\sigma_{m q} \right] \dup V
\label{eqn-B4}
\ee
The integral of the first term in the square brackets in \eqref{eqn-B4} reduces to a surface integral over $S_i$, by the divergence theorem.  For the second term, we substitute for $\partial_k\sigma_{k q}$ from \eqref{eqns-point_mom}, and write the integral as a volume moment of the force on the particle, including inertia.  The above expression thus reduces to
\be
	- \Frac{(-1)^{n}}{n!} \partial_{p_1} \cdots \partial_{p_{n}} \sum_{i} g_i \left[ \; \varint{S_i}{} y'_{p_1} \cdots y'_{p_n} n_k \sigma_{k q}  \, \dup S - \varint{V_i}{}  y'_{p_1} \cdots y'_{p_n} \! \left( \rho  \matderiv{u_q} - b_q \right) \! \dup V \right],  %\rule{2em}{0em}
\label{eqn-B5}
\ee
which is the desired transformation of \eqref{eqn-B1}.  \citet{prosperetti04}  derived it for $n\!=\!3$, and stated that the result could be generalized for larger $n$.  Our derivation is for arbitrary $n$. 


\end{document}


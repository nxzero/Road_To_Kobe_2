\documentclass[12pt]{My_preprint}

\usetikzlibrary{arrows.meta,
                chains,
                positioning,
                shapes.geometric}
%%%%%%%%%%%%%%%%%%%%%%%%%%%%%%%%%%%%%%%%%%%%%%%%%%%%%%%%%%%%%%%%%%%%%%%%%%%%%%%


% \newcommand{\numref}[1]{\ref{#1}}

%%%%%%%%%%%%%%%%%%%%%%%%%%%%%%% Title & Author %%%%%%%%%%%%%%%%%%%%%%%%%%%%%%%%


%\title{Reynolds stress and inertial interphase force in buoyant emulsion.}
%\title{Buoyancy-driven motion of inertial monodisperse suspension of drops}
%\title{Velocity and nearest particle statistics %Buoyancy-driven motion of inertial monodisperse suspension of drops}
\title{Yet another drag force correlation}

\author[1,2]{Nicolas Fintzi}
\author[2]{St\'ephane Popinet}
\author[1]{Jean-Lou Pierson}
\affil[1]{IFP Energies Nouvelles, Rond-point de l’changeur de Solaize, 69360 Solaize}
\affil[2]{Sorbonne Université, Institut Jean le Rond d’Alembert, 4 place Jussieu, 75252 PARIS CEDEX 05, France}
\normalmarginpar
\begin{document}

\maketitle

\begin{abstract}
    In this brief study, we propose a new model for the interphase drag force term to be used in the averaged Navier-Stokes equations. 
    This model is restricted to homogeneous and steady-state, mono-disperse multiphase flows. 
    Drawing upon previous numerical and experimental studies in the literature, we develop a drag force model that accounts for arbitrary viscosity ratios between the dispersed and continuous phase, noted $\lambda$. 
    Most prior models have focused on sedimentation involving solid particles or buoyant rising bubbles. 
    Following \citet{rivkind1976flow}, we interpolate between these two class of models to create a robust formulation that remains valid for intermediate values of $\lambda$, across all Reynolds numbers $Re$ and volume fractions $\phi$.
    Specifically, we employ the models of \citet{schiller1933} and \citet{mei1994} to describe the drag force on solid sphere and spherical bubbles, respectively.
    Then we incorporate a Richardson-Zaki law made for solid particles to account for the volume fraction ($\phi$), at arbitrary \textit{Reynolds} numbers ($Re$) \citep{kramer2019improvement}.
    To account for the variable $\lambda$, we adjust the Richardson-Zaki exponent based on experimental observations in the Stokes regime \citep{ishii1979drag}. 
    The \texttt{no-coalesence.h} algorithm presented in our last study enables us to perform statistically steady-state simulations of mono-disperse buoyant droplets for various \textit{Galileo} number $Ga = 5 \rightarrow 10$, volume fraction $\phi =1\% \rightarrow 20\%$ and viscosity ratio $\lambda = 0.1,1,10$. 
    From these DNS simulations, we could measure the drag coefficient and compare it to the semi-empirical prediction developed in this study. 
    By comparing our literature-based model with DNS results, we successfully validate this new drag force coefficient in the intermediate $\lambda$ regime.
\end{abstract}





\section{Introduction}

In the context of Euler-Euler models, this study focuses on the momentum exchange term in the averaged Navier-Stokes equations, or the drag force density term. 
The objective of this work is to quantify the drag force in a ``homogeneous'' and steady-state configuration.
This means that we consider only a constant and uniform relative motion between the dispersed and continuous phase and a constant mean volume fraction of droplets. 

In this configuration, a drag force correlation should predict the mean drag force density as a function of the local \textit{Reynolds} number based on the particle size, denoted as $Re$, and the volume fraction of the dispersed phase, $\phi$. Additionally, since we are considering droplets with a constant density ratio ($\zeta$) but arbitrary viscosity, the model must account for the viscosity ratio, $\lambda$. 
To the authors' knowledge, no existing model in the literature comprehensively describes the dependence of drag force density in terms of $Re$, $\phi$, and $\lambda$. 
Therefore, we aim to address this gap by proposing a new model incorporating these dependencies.
For isolated spherical droplets, \citet{magnaudet1997forces} proposed a semi-empirical formula based on the well-known models of \citet{schiller1933} for solid particles and \citet{mei1994} for spherical bubbles, which accurately predict the drag force as a function of both the \textit{Reynolds} number and the viscosity ratio.
Meanwhile, \citet{richardson1954} introduced robust empirical relations to predict the sedimentation velocity of solid particles in terms of volume fraction $\phi$ at low \textit{Reynolds} numbers. 
The Richardson-Zaki laws can be written as $u_p/u_0 = (1-\phi)^n$, where $u_p$ is the dispersed phase velocity and $u_0$ is the terminal velocity of an isolated particle. 
The exponent $n$ is an empirical constant known as the Richardson-Zaki exponent. 
\citet{ishii1979drag} extended the values of the Richardson-Zaki exponent for spherical bubbles in the Stokes regime. 
More recently, \citet{kramer2019improvement} proposed an accurate relation to model the Richardson-Zaki exponent for solid particles, for arbitrary \textit{Reynolds} numbers. 

In this work, we utilized the findings of these authors and combined their models to construct a robust model applicable at intermediate values of $\lambda$.
Since \citet{magnaudet1997forces} introduced a model valid for intermediate $\lambda$ at $\phi = 0$, and \citet{ishii1979drag} proposed corrections to the Richardson-Zaki exponent for droplets at $Re \approx 0$, the real challenge here is to accurately model the $\phi$ dependency of the drag force for finite Reynolds numbers, at intermediate $\lambda$.

The organization of this study is as follows:
In \ref{sec:methodology_drag} we present the theoretical and numerical methodology adopted to express the drag force in terms of the buoyancy force or the relative velocity of the dispersed and continuous phase within the averaged Navier-Stokes equations framework.
In \ref{sec:model_drag} we outline the theoretical development that leads to the creation of our new drag force model. 
Finally in \ref{sec:validation_drag} we demonstrate the validity of our model by comparing its predictions, for the drag force and sedimentation velocity, with the DNS results at intermediates values of $\lambda$.
\section{Methodology}

\subsection{The closure problem}
To clarify the role of the different forces and stresses in a multiphase flow model we start by listing the averaged momentum equations for dispersed two-phase flows obtained by an ensemble averaging procedure.
For mono-disperse emulsion the momentum equations of the continuous phase and dispersed phase can be written \citep{zhang1997momentum,jackson1997locally}
\begin{align}
    \pddt (\phi_f\rho_f \textbf{u}_f)
    + \div \left(\phi_f\rho_f \textbf{u}_f\textbf{u}_f + \bm{\sigma}_f^{\text{eff}}\right)
    &= \phi_f 
    \left(\div \bm{\sigma}_f
    + \rho_f \textbf{g}\right)
    - n_p \textbf{f}_p, 
    \label{eq:dt_uf}
    \\
    \pddt (n_m  m_p  \textbf{u}_p)
    + \div \left(n_p m_p  \textbf{u}_p\textbf{u}_p
    +  \bm{\sigma}_p^{\text{eff}}\right)
    &= 
    n_p v_p \left(\div \bm{\sigma}_f
    + \rho_d \textbf{g}\right)
    + n_p \textbf{f}_p, 
    \label{eq:dt_up}
\end{align}
respectively. 
The subscript $f$ and $p$ refer to continuous phase and particle phase averaged quantities, respectively.
$\phi_f$ is the volume fraction of the continuous phase, $n_p$ the particle number density, $\textbf{u}_f$ (resp. $\textbf{u}_p$) the averaged velocity of the fluid (resp. dispersed) phase, $\bm{\sigma}_f$ the averaged continuous phase stress tensor.
$\bm{\sigma}^{\text{eff}}_p$ and $\bm{\sigma}^{\text{eff}}_f$ are the effective stresses of the dispersed and continuous phase, respectively.  
Finally, $\textbf{f}_p$ represents the interphase momentum exchange, or drag force density term. 

Both momentum equations are completed by a transport equation for $\phi_f$ and a volume conservation laws, namely, 
\begin{align}
    % \phi_d \approx  \\
    \phi_f + n_pv_p  - \frac{v_\alpha d^2 }{20}\grad n_p \approx 1\\
    \pddt \phi_f + \div (\textbf{u}_f \phi_f)= 1
\end{align}
where we have introduced, $d$, as the diameter of the droplets. 
We recall that the first equation is only an approximation, that has been derived using the relation between volume fraction and number density\citep{zhang1997momentum}. 

The number density, continuous phase volume fraction, drag force density, effective stresses of the dispersed, and continuous phase can be expressed as ensemble average of local-non averaged quantities and read as, 
\begin{align}
    n_p &= \pavg{}
    \label{eq:n_p}\\
    \phi_f &= \avg{\chi_f}
    \label{eq:chi_f}\\
    \bm\sigma_f &= p_f \bm\delta + \mu_f (\grad \textbf{u}_f + \dagger \grad \textbf{u}_f) - \frac{\mu_f}{\phi_f} \avg{\delta_\Gamma (\textbf{u}_f' \textbf{n}+ \textbf{n}\textbf{u}_f')}
    \label{eq:sigma_f}\\
    n_p \textbf{f}_p  &= \pSavg{\bm\sigma'_f\cdot \textbf{n}}
    \label{eq:f_alpha}
    \\
    \bm{\sigma}_f^{\text{eff}} &= \pavg{\textbf{u}_\alpha'\textbf{u}_\alpha'}
    \label{eq:def_uup}
    \\
    \bm{\sigma}^{\text{eff}}_f &= \avg{\chi_f \textbf{u}_f'\textbf{u}_f'} - \pSavg{\textbf{r}\bm\sigma'_f\cdot \textbf{n}}
    + \frac{1}{2}\div\pSavg{\textbf{rr}\bm\sigma'_f\cdot \textbf{n}}
    \label{eq:def_sigma_eff_f}
\end{align}
respectively. 
Where we introduced: $\avg{\ldots}$ as an ensemble average procedure, 
$\textbf{u}_\alpha$ is the center of mass velocity of a particle labeled $\alpha$, $\chi_f$ is the phase indicator function of the continuous phase, and $\delta_p$ the Dirac delta function pointing on the particle center of masses, $\delta_\Gamma$ the interface indicator function, and \textbf{n} the normal of the surface pointing toward the continuous phase. 
The superscript $'$ indicates the relative values of a quantity with respect to its phasic mean values. 
Specifically $\bm{\sigma}_f' = \bm{\sigma}_f^0  - \bm{\sigma}_f$, 
$\textbf{u}_\alpha' = \textbf{u}_\alpha - \textbf{u}_p$ and $\textbf{u}_f' = \textbf{u}_f^0  -\textbf{u}_f$, with $\bm{\sigma}_f^0 $ and $\textbf{u}_f^0$ being the local stress of the continuous phase, the local velocity of the continuous phase, respectively. 

In the present situation, i.e. where the mixture is only governed by conservation of mass and momentum,  the ``closure problem'' consist in finding explicit expressions for the terms of the form $\avg{\ldots}$ in \ref{eq:sigma_f}, \ref{eq:f_alpha}, \ref{eq:def_uup} and \ref{{eq:def_sigma_eff_f}} in terms of the unknown of the problem, i.e. $n_p$, $\phi_f$, $\textbf{u}_p$ and $\textbf{u}_f$. 
Notice that since 


\subsection{ The momentum balance in homogeneous sedimentation}

In this work we restrict our attention to the homogeneous emulsion of buoyant droplets. 
Additionally, we will focus on the closure given by \ref{eq:f_alpha}. 

By statdy-state and ``homogeneous'', we imply that the ensemble averaged quantities, $\textbf{u}_p$, $\textbf{u}_f$, $n_p$ and $\phi_f$ are not function of space and time variables, i.e. $\textbf{x}$ and $t$. 
However, notice  that the mean continuous phase pressure, $p_f$, may be a function of $\textbf{x}$, even in the homogeneous case. 
Since every closure terms present in \ref{eq:dt_uf} and \ref{eq:dt_up} are ensemble average of fluctuating quantities they cannot be a function of the mean hydrostatic pressure, $p_f$.
Consequently, from \ref{eq:n_p} to \ref{eq:def_sigma_eff_f} the only space dependent quantity is $p_f$. 

In this situation, the averaged momentum conservation equations \ref{eq:dt_uf} and \ref{eq:dt_uf} can be re-written as, 
\begin{align}
    % \pddt (\phi_f\rho_f \textbf{u}_f)
    % + \div \left(\phi_f\rho_f \textbf{u}_f\textbf{u}_f + \phi_f  \bm{\sigma}_f^{\text{Re}} - n_p\textbf{M}_p \right)
    0 
    &= \phi_f 
    \left(\grad p_f
    + \rho_f \textbf{g}\right)
    - n_p \textbf{f}_p, 
    \label{eq:dt_uf_steady}
    \\
    % \pddt (\phi_d\rho_d \textbf{u}_p)
    % + \div \left(\phi_d\rho_d \textbf{u}_p\textbf{u}_p+ \phi_d \bm{\sigma}_p^{\text{Re}}\right)
    0
    &= 
    \phi_d \left(\grad p_f
    + \rho_d \textbf{g}\right)
    + n_p \textbf{f}_p. 
    % \label{eq:dy_up}
    \label{eq:dt_up_steady}
\end{align}
where we have used the relation $n_p v_p = \phi_d$, which is not an approximation anymore since $\grad n_p = 0$ in this specific case.
Our goal is to build a model for the force density closure $\textbf{f}_p$, then \ref{eq:dt_uf_steady} and \ref{eq:dt_up_steady} represent out starting point to accomplish this task. 



Multiplying \ref{eq:dt_uf_steady} by $\phi_d$ and \ref{eq:dt_up_steady} by $\phi_f$, and subtracting the resulting equations, gives directly the equilibrium between buoyancy and drag force density, namely, 
\begin{align}
     \textbf{f}_p
    &= 
    \frac{4}{3}\frac{d^3 \pi}{8}\ \phi_f (\rho_f -\rho_d ) \textbf{g}. 
    \label{eq:f_p_buoyant}
\end{align}
% In  mono-disperse suspension of droplets $n_p = \phi_d / v_p$ with $v_p =4/3\pi d^3/8$ the volume of a particle which yields the final results, 
% \begin{equation*}
%     \textbf{f}_p
%     = 
%     \frac{4}{3}\pi\frac{d^3}{8}\phi_f (\rho_f -\rho_d ) \textbf{g}
%     \label{eq:drag}
% \end{equation*}
% It is convinient to make dimensionless this force with Hadamard-Ribczynski formula, which is, 
% \begin{equation*}
%     \textbf{f}^0_p = \pi \mu_f d A \textbf{u}_{pf}
% \end{equation*}
% Dividing one by the other gives the dimensionless force
% \begin{equation*}
%     \textbf{f}^*_p 
%     = 
%     \frac{4}{3A}\frac{d^2 \phi_f (\rho_f -\rho_d ) \textbf{g}}{8 u_{pf}\mu_f}
% \end{equation*}
% This can be made dimensionless with $\phi_f$
Let us assume that the force density can be written in the form, 
\begin{equation*}
    \textbf{f}_p = C_d  \pi \rho_f \frac{d^2}{8} u_{pf}^2
    \label{eq:f_p_def}
\end{equation*}
where $C_d$ is a dimensionless coefficient, and $u_{pf}^2 = (\textbf{u}_p - \textbf{u}_f)\cdot (\textbf{u}_p - \textbf{u}_f)$, is the relative mean phase velocity squared. 
Using \ref{eq:f_p_buoyant}  and \ref{eq:f_p_def} we can show that $C_d$ is related to the mean phase relative velocity with, 
\begin{equation}
    C_d  
    = 
    \frac{4}{3}
    \frac{d \phi_f (\rho_f -\rho_d ) \textbf{g}}{\rho_f u_{pf}^2}
    \label{eq:C_d}
\end{equation}
The right-hand side of \ref{eq:C_d} can be reformulated according to three dimensionless groups, it yields, 
\begin{equation}
    C_d = 
    \frac{4\phi_f}{3} \left(\frac{Ga}{Re}\right)^2
    \label{eq:C_d_adim}
\end{equation}
where $Ga$ is the \textit{Galileo} number and $Re$ the \textit{Reynolds} number, namely, 
\begin{align}
    Ga^2 = \frac{
    d^3
    \rho_f
    (\rho_f -\rho_d ) g
}{\mu_f^2},
&& 
Re =   \frac{\rho_f d u_{pf}}{\mu_f}.
\end{align}

In summary, the force density term, $\textbf{f}_p$, can be written in terms of the dimensionless constant $C_d$, which can itself be written in terms of $Ga$, $\phi_f$ and $Re$ in the present context. 
The \textit{Galileo} number and the volume fraction $\phi_f$ are known parameters in our problem, indeed they are only function of the physical and geometrical properties of the mixture.
Consequently, the closure problem for $\textbf{f}_p$ consist in measuring the \textit{Reynolds} number or the mean relative motion between phases, knowing $\phi_f$ and $Ga$ a priori. 
The mean relative motion between phases, can be obtained either experimentally, theoretically or numerically. 
In this work, we combine experimental measures and theoretical results from the literature to provide the most complete model for $C_d$.
Then, based on the DNS presented in the following section we will be able to validate and see the limitation of our model. 

% \tb{
% This can be directly computed into our DNS. 
% \begin{equation*}
%     C_d  \phi_f^2 \frac{\rho_f^2 d^2 u_{pf}^2}{\mu_f^2}
%     = 
%     \frac{4}{3}
%     \phi_f^3 
%     \frac{
%         d^3
%         \rho_f
%         (\rho_f -\rho_d ) g
%     }{\mu_f^2}
% \end{equation*}
% Let us define the Galileo number as $Ga^2 = \frac{
%     d^3
%     \rho_f
%     (\rho_f -\rho_d ) g
% }{\mu_f^2}$ and the Reynolds number based on the drift velocity as, $Re =  \phi_f \frac{\rho_f d u_{pf}}{\mu_f}$,
% Then, the relation between the Reynolds and Galileo is given by, 
% \begin{equation*}
%     Re
%     = 
%     Ga
%     \sqrt{\frac{4\phi_f^3}{3 C_d}}
% \end{equation*}
% This the relative velocity is given by, 
% In stokes and dilute regime the $C_d$ noted $D_c^0$ is given by Hadamard-Ribczynski solution and reads, 
% \begin{equation*}
%     C_d^0 = \frac{8}{Re} \left(\frac{3\lambda +2 }{\lambda +1}\right)
%     = \frac{8}{Re}A
% \end{equation*}
% where we introduced the constant $A = \left(\frac{3\lambda +2 }{\lambda +1}\right)$. 
% Thus, the Reynolds number obtained for a given \textit{Galileo} number in stokes regime is, 
% \begin{equation*}
%     Re^0
%     = 
%     \frac{Ga^2}{6 A}
%     % \phi_f^3  
% \end{equation*}
% Since this is valid for an isolated particle we fixed $\phi_f=1$, this will be our renormalization constant. 
% \begin{equation*}
%     Re^*
%     = 
%     \frac{6A}{Ga}
%     \sqrt{\frac{4\phi_f^3}{3 C_d}}
% \end{equation*}
 
% \paragraph{Relation between Galileo and Reynolds numbers :}
% From the two previous expressions we can write the equality, 
% \begin{equation*}
%     \textbf{f}_p = C_d  \pi \rho_f \frac{d^2}{8} u_{pf}^2
% \end{equation*} 
% The hadamar ribinsky formula reads, 
% \begin{equation*}
%     \textbf{f}_p^0 =\pi \mu_f d A \textbf{u}_{pf}
% \end{equation*}
% dividing one by the other and by $\phi_f^2$ gives directly, 
% \begin{equation*}
%     \textbf{f}_p^* =   \frac{C_d  Re}{8 A} = \frac{C_d}{C_d^*}
% \end{equation*}
% }
\subsection{Computational methodology}


To represent a statistically steady-state and homogeneous buoyant emulsion we carry out DNS of tri-periodic rising droplets. 
Notice that for the statistics to converge, the domain have to be large enough, and the simulation time long enough. 

% We display in \ref{tab:dimensionless_numbers} the dimensionless numbers explored in this work. 
% \begin{table}[h!]
%     \centering
%     \caption{Dimensionless parameter range investigated in this work.}
%     \label{tab:dimensionless_numbers}
%     \begin{tabular}{|ccccccc|ccc|}
%         \hline
%         \multicolumn{7}{|c}{Primary parameters} & \multicolumn{3}{||c|}{Secondary parameters}\\ \hline
%         \multicolumn{1}{|c|}{$Ga$}                               & \multicolumn{1}{c|}{$Bo$}                   & \multicolumn{1}{c|}{$\phi$} & \multicolumn{1}{c|}{$\lambda$}                    & \multicolumn{1}{c|}{$\zeta$}                & \multicolumn{1}{c|}{$N_b$} & $t^*_\text{end}$ & \multicolumn{1}{||c|}{$\mathcal{L}/d$} & \multicolumn{1}{c|}{$Re$}  & $We$   \\ \hline
%         \multicolumn{1}{|c|}{\multirow{4}{*}{$5\rightarrow 80$}} & \multicolumn{1}{c|}{\multirow{4}{*}{$0.5$}} & \multicolumn{1}{c|}{$1\%$}  & \multicolumn{1}{c|}{\multirow{4}{*}{$10$ \& $1$\&$0.1$}} & \multicolumn{1}{c|}{\multirow{4}{*}{$0.9$}} & \multicolumn{1}{c|}{$160$} & $400$           & \multicolumn{1}{||c|}{$20$}            & \multicolumn{1}{c|}{$1.3\to 110$} & {$0.03\to 0.95$} \\ 
%         \multicolumn{1}{|c|}{}                                   & \multicolumn{1}{c|}{}                       & \multicolumn{1}{c|}{$5\%$}  & \multicolumn{1}{c|}{}                             & \multicolumn{1}{c|}{}                       & \multicolumn{1}{c|}{$800$} & $400$           & \multicolumn{1}{||c|}{$20$}            & \multicolumn{1}{c|}{$1.0\to 92$} &  {$0.02\to 0.67$}\\ 
%         \multicolumn{1}{|c|}{}                                   & \multicolumn{1}{c|}{}                       & \multicolumn{1}{c|}{$10\%$} & \multicolumn{1}{c|}{}                             & \multicolumn{1}{c|}{}                       & \multicolumn{1}{c|}{$200$} & $1000$           & \multicolumn{1}{||c|}{$10$}            & \multicolumn{1}{c|}{$1.9\to 77$}&  {$0.01\to 0.47$}\\ 
%         \multicolumn{1}{|c|}{}                                   & \multicolumn{1}{c|}{}                       & \multicolumn{1}{c|}{$20\%$} & \multicolumn{1}{c|}{}                             & \multicolumn{1}{c|}{}                       & \multicolumn{1}{c|}{$400$} & $1000$           & \multicolumn{1}{||c|}{$10$}            & \multicolumn{1}{c|}{$1.7\to 62$}&  {$9\cdot 10^{-3}\to 0.31$}\\ \hline
%         \end{tabular}
% \end{table}


The study's primary objective is to measure the mean relative velocity between the droplets and the continuous phase.
Thus, obtaining a sufficient number of DNS samples is crucial to ensure a good statistical convergence of these mean quantities. 
Also, the physical quantities measured in the simulations must remain independent of the domain size. 
Regarding the grid spacing $\Delta$, we show in \ref{ap:convergence}  that using a definition of $d/\Delta = 25$ is enough to obtain representative results for $Re$. 
We set $\mathcal{L}/d = 10$, which is roughly what \citet{hidman2023assessing} used for their DNS of fully-periodic buoyant rising bubbles.
Likewise, we use a number of particles per domain of at least $N_b = 160$ for all our cases, which introduces the need for a larger domain ($\mathcal{L}/d = 20$) for the dilute cases, so that the  $d/\Delta = 25$ is respected. 
Each DNS lasts for a time: $t^*_\text{end} = 400 \sqrt{d/g}$ for the larger domains ($\mathcal{L}/d=20$) and $t^*_\text{end} = 1000 \sqrt{d/g}$ for the smaller domain.
% It is shown in \ref{ap:validation} that these parameters are sufficient to obtain well converged statistics.  
It is shown in \citet{fintzi2024buoyancy} and \ref{ap:convergence} that these parameters are sufficient to obtain well converged statistics. 
\begin{table}[h!]
    \centering
    \caption{Dimensionless parameter range investigated in this work.}
    \begin{tabular}{|ccccccc|ccc|}
        \hline
        \multicolumn{7}{|c}{Primary parameters} & \multicolumn{3}{||c|}{Secondary parameters}\\ \hline
        \multicolumn{1}{|c|}{$Ga$}                               & \multicolumn{1}{c|}{$Bo$}                   & \multicolumn{1}{c|}{$\phi$} & \multicolumn{1}{c|}{$\lambda$}                    & \multicolumn{1}{c|}{$\zeta$}                & \multicolumn{1}{c|}{$N_b$} & $t^*_\text{end}$ & \multicolumn{1}{||c|}{$\mathcal{L}/d$} & \multicolumn{1}{c|}{$Re$}  & $We$   \\ \hline
        \multicolumn{1}{|c|}{\multirow{4}{*}{$5\rightarrow 80$}} & \multicolumn{1}{c|}{\multirow{4}{*}{$0.5$}} & \multicolumn{1}{c|}{$1\%$}  & \multicolumn{1}{c|}{\multirow{4}{*}{$10$ $\to$ $0.1$}} & \multicolumn{1}{c|}{\multirow{4}{*}{$0.9$}} & \multicolumn{1}{c|}{$160$} & $400$           & \multicolumn{1}{||c|}{$20$}            & \multicolumn{1}{c|}{$1.3\to 110$} & {$0.03\to 0.95$} \\ 
        \multicolumn{1}{|c|}{}                                   & \multicolumn{1}{c|}{}                       & \multicolumn{1}{c|}{$5\%$}  & \multicolumn{1}{c|}{}                             & \multicolumn{1}{c|}{}                       & \multicolumn{1}{c|}{$800$} & $400$           & \multicolumn{1}{||c|}{$20$}            & \multicolumn{1}{c|}{$1.0\to 92$} &  {$0.02\to 0.67$}\\ 
        \multicolumn{1}{|c|}{}                                   & \multicolumn{1}{c|}{}                       & \multicolumn{1}{c|}{$10\%$} & \multicolumn{1}{c|}{}                             & \multicolumn{1}{c|}{}                       & \multicolumn{1}{c|}{$200$} & $1000$           & \multicolumn{1}{||c|}{$10$}            & \multicolumn{1}{c|}{$1.9\to 77$}&  {$0.01\to 0.47$}\\ 
        \multicolumn{1}{|c|}{}                                   & \multicolumn{1}{c|}{}                       & \multicolumn{1}{c|}{$20\%$} & \multicolumn{1}{c|}{}                             & \multicolumn{1}{c|}{}                       & \multicolumn{1}{c|}{$400$} & $1000$           & \multicolumn{1}{||c|}{$10$}            & \multicolumn{1}{c|}{$1.7\to 62$}&  {$9\cdot 10^{-3}\to 0.31$}\\ \hline
        \end{tabular}
    \label{tab:simulations}
\end{table}
This study presents DNS results with dimensionless parameters in ranges outlined in \ref{tab:simulations}.
In summary, we investigated $5$ \textit{Galileo} number $Ga = 5,10,25,50,80$, $4$ different volume fractions $\phi = 0.01,0.05,0.1,0.2$, and three viscosity ratios $\lambda =0.1,1,10$ with $Bo = 0.5$ and $\zeta = 0.9$.
This makes a total of $60$ representative simulations.

Due to numerical constraint we used a slightly higher \textit{Bond} number ($Bo = 0.5$) in this study, compared to the previous set of DNS used in \ref{chap:microstructure}. 
\begin{figure}[h!]
    \centering
    \includegraphics[height = 0.3\textwidth]{image/HOMOGENEOUS_final/PA/chi.pdf}
    \caption{Mean aspect ratio of the droplets $\chi_p$, as a function of the \textit{Galileo} number, and the volume faction $\phi$,  for two different viscosity ratios.  
    The symbols correspond to different volume fraction ($\pmb\bigcirc$) $\phi = 0.01$; ($\pmb\triangle$) $ \phi = 0.05$; ($\pmb\square$) $\phi = 0.1$ ($\pmb\lozenge$) $\phi = 0.2$.
    The hollow symbols correspond to $\lambda = 1$, the filled symbols to $\lambda = 10$, and the small symbol to $\lambda = 0.1$.
    The nearly imperceptible vertical bars on each symbol, represent the standard deviation around the mean.  }
    \label{fig:chi2}
\end{figure}
To verify that the droplets remain in average approximately spherical we plotted in \ref{fig:chi2} the mean aspect ratio $\chi_p$ of the droplet for all of our simulation. 
A rigorous definition of this aspect ratio is given in \citet{fintzi2024buoyancy} or \citet{bunner2003effect}. 
Anyhow, at $Bo = à.5$ the maximum mean droplet deformation $\chi_p$ represents about $4\%$ of deformation, compared to the $2\%$ obtained for $Bo = 0.2$. 
This means that the droplet shape deviate approximately of $4\%$ from their original spherical shape. 


\subsection{Approximation of the ensemble average}

Following \citet{du2022analysis} we consider ergodicity at all time of the numerical experiment.
Thus, the ensemble average of a quantity $X$ can be approximated by a spatial average $\Xavg{X}$ and a time average $\Tavg{X}$ such that $\avg{X} \approx \Xavg{\Tavg{X}} = \Tavg{\Xavg{X}}$.
Consequently, the ensemble average of a numerical field, $X$, is taken through space and time such that,
\begin{equation}
    \avg{X}
    = \Tavg{\Xavg{X}}
    = \frac{1}{ t_{end} - t_0}\int_{t_0}^{t_{end}} 
    \Xavg{X}(t) dt
\end{equation}
where, 
\begin{equation}
    \Xavg{X}(t)
    = \frac{1}{L^3}\int 
    X(\textbf{x},t) d\textbf{x}
\end{equation}
where $L$ is the length of one side of the cubic numerical domain.
$t_0$ and $t_{end}$ are the starting time of sampling, and the ending time of sampling which is also the ending time of the simulation, respectively.
In practice, we take $t_0$ such that the simulation reach a statistically steady regime. 
It has been found that $t_0 < 50\sqrt{g/d}$.  
% $t_{end}$ is given in \ref{tab:simulations} and is shown to be sufficient as demonstrated in \ref{ap:convergence}. 

To compute the phase average of the local phase velocity $\textbf{u}_k^0$, we simply perform an integration over space and time, 
\begin{equation}
    \textbf{u}_k = \frac{1}{\phi_k} \Tavg{\Xavg{\chi_k \textbf{u}_k^0}}
\end{equation}
where the indicator function $\chi_f$ must be understood as its approximation in the DNS, which is the color function used by the code \url{http://basilisk.fr}. 

Since the homogeneous and statistically steady-state hypothesis are supposed to be true, the mean of the droplet center of mass velocity is equivalent to the dispersed phase phasic mean velocity, $\textbf{u}_d$.
Thus, we may compute the ensemble average relative velocity used in \ref{eq:C_d} with the operation, 
\begin{equation}
    \textbf{u}_{pf} = 
    \frac{1}{\phi_d} \Tavg{\Xavg{\chi_d \textbf{u}_d^0}}
    - \frac{1}{\phi_f} \Tavg{\Xavg{\chi_f \textbf{u}_f^0}}. 
\end{equation} 

Now that we have all the tools in hand, we present in the next section our generalized model for the drag force adapted to droplet of arbitrary viscosity. 
%Objectives : 
%\begin{itemize}
%    \item Present the rising velocity Vs. phi to demonstrate the relation with $\phi^{1/3}$ \citep{loisy2017buoyancy}
%    \item Discus the common points and differences with bubbles and solid particles. 
%    \item Present a proper definition of the drag force terms such as in \citet{wang2021numerical}. 
%    \item Discus the possible correlation between the shape /arrangement of the particles/flow lines with the rising velocity. \tb{Je ne sais pas trop quoi dire la dessus}
%    \item Show that \citet{rusche2000effect}'s fit for the drag force is not adapted for our case and propose a new one
%    \item All the references for teh Drag force terms are in \citet[chap 8]{morel2015mathematical} or in \citet{ishii2010thermo}
%\end{itemize}
%\todo[inline]{include fits of bubbly flow}

\subsection{Terminal velocity of an isolated spherical drop}
%First of all we want to investigate the dependency of the drift velocity with the volume fraction $\phi$. 
%It is known from several studies on the litterature, especially in \citep[chapter 8]{morel2015mathematical} and \citet[chapter 12]{ishii2010thermo} the the viscosity model for various system can be written generally, as,
%\begin{equation*}
%    \frac{\mu_m}{\mu_c}
%    = \left(
%        1 - \frac{\phi}{\phi_\text{max}}
%    \right)^{-2.5 \phi_\text{max}\mu_\text{eq}}
%\end{equation*}  
%with, $\mu_\text{eq} = \frac{\mu_d + 0.4 \mu_c}{\mu_d+\mu_c}$ and $\phi_\text{max}$ being the volume fraction corresponding to the \textit{maximum packing}. 
%\JL{la viscosite d'une suspension n'a rien a voir avec sa vitesse de chute meme si cela semble etre un argument donne dans la litterature... j'ai enleve tout cela pr l'instant}

In this part, we briefly review the various formulas used to calculate the drag force on a spherical droplet embedded in a steady uniform flow. As demonstrated in Appendix \ref{app:shape} the droplet remains approximatively spherical. This assumption remains valid for the whole range of parameters investigated even in high inertial regime where the maximum deviation from the spherical shape is around $10$ \%. Theoretical predictions for the force on a spherical droplet embedded in a steady uniform flow are limited to the limit of very small and very high Reynolds numbers. We define the drag coefficient, denoted as $C_D$ by the equation $F = \pi / 8 C_D \rho U_0^2 d^2$, where $F$ is the force on the drop, $U_0$ is the imposed velocity. The drag coefficient is a function of the Reynolds number $Re = \rho U_0 d /\mu $ and of is the viscosity ratio$\lambda = \mu _d /\mu _c$. % is the  as $F = C_D$ 
In the Stokes regime ($Re=0$)the drag coeficient is given by the Hadamard-Ribczynski formula


%In this section, we briefly survey the diverse formulas applied to calculate the drag force acting on a spherical droplet within a steady, uniform flow. As corroborated in Appendix \ref{app:shape}, the droplet maintains an approximate spherical shape, a validity sustained across the entire spectrum of investigated parameters, even in the high inertial regime. Theoretical predictions for the force acting on a spherical droplet in a steady uniform flow are confined to scenarios of extremely low and exceptionally high Reynolds numbers.

%We define the drag coefficient, denoted as $C_D$, by the equation F=π8CDρU02d2F = \frac{\pi}{8} C_D \rho U_0^2 d^2F=8π​CD​ρU02​d2, where FFF signifies the force on the droplet, and U0U_0U0​ is the imposed velocity. This coefficient varies with the Reynolds number Re=ρU0dμRe = \frac{\rho U_0 d}{\mu}Re=μρU0​d​ and the viscosity ratio λ=μdμc\lambda = \frac{\mu_d}{\mu_c}λ=μc​μd​​. In the Stokes regime, the drag coefficient adheres to the Hadamard-Ribczynski formula.

%In the Stokes flow regime, the drag coefficient defined as $F = $

%drag force on a spherical drop embedded in a steady uniform flow is given by the Hadamard-Ribczynski formula
%\JL{il faut choisir ton echelle caractersitique de longueur. Soit $a$ le rayon soit le diametre des particules.}
%\begin{equation}
%F_0 = -\pi \mu d U \frac{2+3\lambda}{1+\lambda}
%\end{equation}

\begin{equation}
C_D = \frac{8}{Re} \left( \frac{2+3\lambda}{1+\lambda} \right)
\end{equation}
%\ref{fig:U} shows the drift velocity $U$ divided by the stokes rising velocity of a spherical droplet $U_\text{stokes}$ defined in our notation as \citep{kim2013microhydrodynamics}, 
Balancing the drag force obtained using the previous formula with the buoyancy force one obtained the settling velocity in the Stokes regime
\begin{equation}
    U_0
    = (\rho_c - \rho_d)\frac{g d^2}{6\mu_c}\left(\frac{1+\lambda}{2 + 3\lambda}\right),
\end{equation}
or in dimensionless form 
\begin{equation}
    Re_0
    = \frac{Ar^2}{6}\left(\frac{1+\lambda}{2 + 3\lambda}\right).
\end{equation}
where $Re_0 = \rho_c U_0 d/\mu_c$ is the Reynolds number based on the terminal velocity.
In the opposite regime of very high Reynolds numbers ($Re\gg 1$), the flow outside the droplet can be considered as potential except in a thin boundary layer developing on the bubble surface. \citet{harper1968} have shown that the drag coefficient on a spherical drop is given by 

\begin{equation}
C_D = \frac{48}{Re}\left(1 + \frac{3\lambda}{2}\right).
\label{eq:harper}
\end{equation}
Equation \ref{eq:harper} is the leading order in the expansion performed by \citet{harper1968} in the limit $Re\gg 1$. This equation tends toward Levich formula for the drag on a clean bubble in the limit $\lambda \ll 1$. A detailed investigation perfomed by Dandy et Leal have shown that the oroginal formulation by Harper and mmore became accurate for $Re\geq 200$. The above review show that in the intermediate Reynolds number regimes of the pressent study $ 1 \leq ...$

\begin{equation}
C_D = \frac{24}{Re}(1+0.15Re^{0.687})
\end{equation}

\begin{equation}
C_D = \frac{16}{Re}\left(1+\left[\frac{8}{Re}+\frac{1}{2}\left(1+3.315Re^{-1/2}\right)\right]^{-1}\right)
\end{equation}


\begin{equation}
C_D(Re)Re^2=\frac{4}{3}Ga ^2
\label{}
\end{equation}


%where higher order terms can be found in the original publication of \citet{harper1968}. 

%to leading order as $Re = \rho U_0 d /\mu$ 


In practice the above formula are very limited randge of validity and empirical formulation have to be used for intermediate Reynolds numbers. Prendre la correlation de Rykind et Ryskin



\subsection{Hindered settling velocity}
As an exemple for a suspenison of homogeneous solid spherical particle $\phi_\text{max} = 0.62$ and for deformable particles system it can be approximated to $\phi_\text{max} = 1$. 
From this consideration we can deduce that the drift velocity $U$ is related to the volume fraction with, 
\begin{equation*}
    \frac{U(\phi)}{U_\text{stokes}} = \left\{\begin{tabular}{cc}
        $(1-\phi)^{1.5}$   & bubbles in liquid \\
        $(1-\phi)^{2.25}$   & drops in liquid \\
        $(1-\phi)^{3}$   & drops in gas \\
    \end{tabular}\right.
\end{equation*} 
Additionally, \citet{bunner2003effect} propose a scaling of $\sim (1 - \phi^{1/3})$ for spherical bubbles.
While \citet{ishii1979drag} proposed a $\sim (1 - \phi)^3$ for deformable bubbles. 

In our case, i.e. quasi spherical droplets, we reach a $\sim (1 - \phi^{1/3})$ scaling for $\lambda = 10$ and approximately a $\sim (1 - \phi^{1/2})$ scalings for $\lambda = 1$ .
\begin{figure}[h!]
    \centering
    \includegraphics[height = 0.35\textwidth]{image/HOMOGENEOUS/fCA/UstokesGa_mu_r_1-0.pdf}
    \includegraphics[height = 0.35\textwidth]{image/HOMOGENEOUS/fCA/UstokesGa_mu_r_0-1.pdf}
    \caption{Rising velocity divided by the rising velocity of an equivalent spherical drop in Stokes regime.($\bullet$) $Ga = 5$, ($\blacktriangle$) $Ga = 10$, ($\blacksquare$) $Ga = 25$ , ($\blacklozenge$) $Ga = 50$, ($\blacktriangleright$) $Ga = 75$ and ($\blacktriangleleft$) $Ga = 100$ . 
    The dashed lines are the empirical funtions (left)  
    $U/U_\text{stokes} = 2.72(Ga^{-2.77} - Ga\;10^{-3}) (1 - \phi^{0.45})$
    (right)  $U/U_\text{stokes} = 2.74(Ga^{-2.85} - Ga \;10^{-3}) (1 - \phi^{0.34})$ }
    \label{fig:U}
\end{figure}

The case for which $\lambda = 1$ have a tendency to processes more deformation, as caracterised by their averaged aspect ratio $\chi$ slightly higher than those for which $\lambda = 10$. 
The differences in the volume fraction dependency can be partially explain by this fact. 
Indeed, a  $\sim (1 - \phi^{2/3})$ power law were observed for slightly deformable bubbly flow \cite{zhang2021direct}. 
Thus it is not surprising that at low but finite deformation we found a scalings between $1/3$ and $2/3$. 

A last interesting fact is that at low $Ga$ and $\phi$ we observe a rising velocity $U / U_\text{stokes} > 1$ meaning that the rising velocity is higher than in the stokes isolated case. 
This phenomena has already been observe in \citet{loisy2017buoyancy}. 
This fact was explained to be due to the cumulative effect of the wakes in orderred array of bubbles  which tends to increases their colective velocity. 
Anyhow, since the the limit $Ga \rightarrow 0$ and $\phi \rightarrow 0$ we must recover $U/U_\text{stokes} = 1$ we can be sure that the $\phi$ scaling won't behaves like so.
Therefore it is crucial to point out that these scalings are surely not valid in the limit  $Ga \rightarrow 0$ and $\phi \rightarrow 0$ .

%\subsection{Interphase drag force}




Now we present our results for the drag force term in terms of the Reynolds number. 
\begin{figure}[h!]
    \centering
    
    \includegraphics[height = 0.35\textwidth]{image/HOMOGENEOUS/fCA/Re_mu_r_1-0.pdf}
    \includegraphics[height = 0.35\textwidth]{image/HOMOGENEOUS/fCA/Re_mu_r_0-1.pdf}
    
    \includegraphics[height = 0.35\textwidth]{image/HOMOGENEOUS/fCA/Fstokes_N_5_l_1.pdf}
    \includegraphics[height = 0.35\textwidth]{image/HOMOGENEOUS/fCA/Fstokes_N_5_l_10.pdf}
    \caption{
        (middle) Reynolds number based on the averaged rising velocity.
    (bottom) Ensemble averaged drag force divided by the stokes drag force on spherical droplet of equivalent size.
    The symbols correspond to different volume fraction ($\bullet$) $\phi = 1\%$, ($\blacktriangle$) $\phi = 5\%$, ($\blacksquare$) $\phi = 10\%$, ($\blacklozenge$) $\phi = 15\%$ and ($\blacktriangleright$) $\phi = 20\%$.
    (dashed lines) empirical formulas : extrapolation of  \citet{tenneti2011drag} for solid particles. }
    \label{fig:drag_force}
\end{figure}
\tb{As discussed in previous study that
for spherical bubbles, as the gas fraction increases and the in-
teractions become more important, the bubbles tend to align
themselves in horizontal pairs, whose average rising velocity
is lower than that of an isolated bubbl}

The interphase drag forces applied on the droplets is the buoyancy force $(\rho_d-\rho_c)v_\alpha \textbf{g}$.
The relevant quantity is therefore the dimensionless drag force, that is what is presented in the following \ref{fig:drag_force}. 


In the stokes regime the drag force on a spherical droplet is, 
\begin{equation*}
    \textbf{F}_s
    =\pi \mu_f d (\textbf{u}_p - \textbf{u}_c) \left(\frac{2+3\lambda}{1+\lambda}\right)
\end{equation*}
Let the dimensionless drag force be a function of $\lambda$ $Re$ and $\phi$, expressed such that, 
\begin{equation}
    \textbf{f}_p(Re,\phi,\lambda)
    = 
    f_1^*(Re)
    f_2^*(\phi)
    \textbf{f}_s(\lambda)
\end{equation}
where $f_{1,2}$ are coefficient which limit tends to $1$ at low $\phi$ and low $Re$. 
To determine the $\phi$ dependency we base our study on the following analysis. 
In \cite[chapter 4]{ashgriz2011handbook} they stipulate that the last droplets empirical fit was made in the study of \citet{rusche2000effect} where they performed empirical fits on experimental datas of emulsion. 
Some of which concerned droplets' sedimentation, were they stipulate that, 
\begin{align*}
    f_1^*(\phi) 
    &=1  + C_1 Re^{C_2}\\
    f_2^*(\phi) 
    &= e^{\phi K_1} + \phi K_2
    \label{eq:drag_fit}
\end{align*}
Nevertheless, the coefficients for the droplets doesn't agree with our numerical calculation. 
Instead we remark that the solid particles fit from \citet{rusche2000effect} well fits our results at $\lambda = 1$. 
Therefore, we propose to keep the shape of teh fits and adjuste our coefficient. 

\tb{Dans cette partie je ne sais pas trop quoi faire pour avoir un bon point de départ pour cree une formule empirique, je manque d'idée, notament pour ce qui est de la dépendence en $\lambda$ et pour l'explication physique des tendence observe. }

\section{Conclusion}

In this study, we have demonstrated the relationship between buoyancy forces and the mean drag force density on droplets, derived from the ensemble-averaged momentum equations. 
In the homogeneous regime, we linked the sedimentation velocity of the dispersed phase with volume fraction and Reynolds number through the Richardson-Zaki relation in Stokes flow, both for solid particle and bubbles.

We then extended the Richardson-Zaki framework to viscous droplets of arbitrary viscosity ratio, valid for arbitrary $Re$ and $\phi$. 
In the dilute limit, our correlation is validated by the literature, as it employs the well-known Schiller-Neuman, and Mei drag force coefficient, for solid particle and spherical bubbles, respectively. 
For solid particles ($\lambda \to \infty$), our model converges to the recent model of \citet{kramer2019improvement}, which is an improvement of the Richardson-Zaki relation valid for arbitrary $\lambda$ and $Re$.

To validate the intermediate $\lambda$ regime, we performed DNS of buoyant emulsions in a tri-periodic box. 
The results show good agreement, with our model accurately capturing the $\phi$-dependence for a given $\lambda$ and $Re$.

The main advantage of the formulation provided by \ref{eq:C_d_finalRe} is its robustness, since Richardson-Zaki relation have been shown to be valid at very high volume fraction  ($\phi \approx 0.5$) and Reynolds number, while in the dilute limit Schiller-Neuman, and Mei drag force coefficient are proven to be accurate up to $Re = 800$. 
Thus, we provided, a robust drag force coefficient that can directly be used in Euler-Euler framework for simulation of emulsion of arbitrary viscosity ratio. 


\appendix
\section{Mesh definition convergence}
\label{ap:convergence}

% \subsection{Statistical and mesh independence study}

In the aim of providing accurate closure terms it is of primary importance to verify the well convergence of the mean quantities, by varying the mesh definition. 
While, the domain size and duration of simulation have been validated in \citet{fintzi2024buoyancy}.
To tackle this problem we carried out four simulation with 160 rising droplets with different mesh definition. 
The flow parameters read as,  
\begin{align*}
    \lambda = 10,
    && \zeta = 1.11,
    && Bo = 0.5,
    && Ga = 80,
    && \phi = 0.05,
\end{align*}

\ref{fig:Re_and_Tc}(left) display the cumulative mean of the vertical Reynolds number based on the drift velocity, namely,
\begin{equation}
    \widetilde{Re}(t)
    = \frac{\rho_f d}{\mu_f t}\int_{t_0}^{t_0+t} \left(\Xavg{\textbf{u}^0_d} -  \Xavg{\textbf{u}_f^0}\right)dt'
\end{equation}
where $t_0$ is the starting sampling time. 
We reach mesh independent results for $d/\Delta \geq 25$ in agreements with the recent studies of \citet{hidman2023assessing} \citet{zhang2021direct} for low inertial bubbly flows.
Also, it is seen that $\widetilde{Re}$ reaches a constant values from $t^* = 50$ to the end of the simulation. 
\begin{figure}[h!]
    \centering
    \includegraphics[height = 0.35\textwidth]{image/HOMOGENEOUS_final/CA/Re.pdf}
    \caption{(left) Cumulative mean of the volume averaged Reynolds number along the simulation time based on the drift velocity $U = \textbf{u}_p - \textbf{u}_c$, with $\phi = 0.1$, $\rho_r = 1.11$, $ \mu_r =0.1$ and $Ga = 29.9$ and $N_b = 125$.
    (right) Cumulative mean of the fluid Reynolds stress tesor. }
    \label{fig:Re_and_Tc}
\end{figure}

% The well convergence of the rising velocity doesn't guarantee a statistical nor a mesh convergence for finer quantities such as the pseudo-turbulent kinetic energy. 
% Therefore, we provide on \ref{fig:UpUp} (left) the running average of the fluid phase pseudo-turbulent energy. 
% Similarly, \ref{fig:UpUp} (right) represent the particle center of mass pseudo-turbulent kinetic energy. 
% \begin{figure}[h!]
%     \centering
%     \includegraphics[height = 0.35\textwidth]{image/VALIDATION2.0/fCA/Tcum.pdf}
%     \includegraphics[height = 0.35\textwidth]{image/VALIDATION2.0/fPA/Tcum.pdf}
%     \caption{(left) Cumulative mean of the volume averaged granular temperature along the simulation time based on the drift velocity $U = \textbf{u}_p - \textbf{u}_c$, with $\phi = 0.1$, $\rho_r = 1.11$, $ \mu_r =0.1$ and $Ga = 29.9$ and $N_b = 125$.
%     (right) Cumulative mean of the dimensionless particle-fluid-particle stress horizontal component tensor. }
%     \label{fig:UpUp}
% \end{figure}
% Both figure exhibit well converged data. 
% Interestingly, $\widetilde{K}_c$ and $\widetilde{K}_\alpha$ reach a constant value at $t^* = 200$ which is four time greater than for $\widetilde{Re}$.


% \tb{Cite and compare to Berner and \citet{bunner2002dynamics} which found that Nb > 12 is sufficient \citet{roghair2011drag}}
% Now, let's investigate the required number of droplets per domain, $N_b$, and the minimum definition of cells per diameter of droplets $\delta$.  
% \tb{Include bibliography and expectation here \ldots}
% For this investigation we kept the physical parameters presented in the same section and made a double parametric analysis over $N$ and $\delta$. 
% We carried out simulations for $N = 2, 3, 4, 5, 6, 7$, and for a number of cells $10 <\delta < 40$. 
% In Basilisk the mesh definition is defined by a power of two, consequently depending on the size of the domain (which is fixed to keep a $\phi$ constant) the $\delta$ parameter is fixed at a power of 2 close. 
% \begin{figure}[h!]
%     \centering
%     \includegraphics[height= 0.3\textwidth]{image/VALIDATION/N_and_delta/DUd.pdf}
%     \includegraphics[height= 0.3\textwidth]{image/VALIDATION/N_and_delta/PHI.pdf}
%     \caption{(left) Averaged Reynolds number based on the drift velocity.
%             (right) Dispersed phase volume fraction at the end of each simulation.
%             The text on the side of the points is $\delta$.
%             N correspond to $N = N_b^3$. }
%     \label{fig:VALIDATION_Nd_1}
% \end{figure}
% \ref{fig:VALIDATION_Nd_1}(left), illustrate clearly that the drift velocity is independent of the parameters $N_b$ and $\delta$, for $N >4$. 
% On the other hand, \ref{fig:VALIDATION_Nd_1}(right), show that the volume fraction of the dispersed phase is lower for the low defined grid (red dots), due to a loss of volume during the simulation.
% This doesn't mean that the solver isn't volume conservative. 
% In fact, it is fund to be due to the \href{http://basilisk.fr/sandbox/fintzin/Rising-Suspension/no-coalescence.h}{no-coalescence.h} which generate fragment into the numerical domain, fragment which are deleted in the long run. 
% \begin{figure}[h!]
%     \centering
%     \includegraphics[height= 0.3\textwidth]{image/VALIDATION/N_and_delta/PA_UpUp.pdf}
%     \includegraphics[height= 0.3\textwidth]{image/VALIDATION/N_and_delta/Mh.pdf}
%     \caption{(left) Fluids phase averaged fluctuation tensor.
%             (right) Particular average of the first moment tensor, where $F_g$ is the buoyancy force applied on one droplet. 
%             The numerical values displayed alongside the dots are the number of cells per diameter.}
%     \label{fig:VALIDATION_Nd_2}
% \end{figure}
% Now, let's look at the behavior of more \textit{complicated} closure terms. 
% \ref{fig:VALIDATION_Nd_2}(left) demonstrate that the vertical component of the pseudo turbulent tensor is parameter independent rather early, independently of the grid definition. 
% This fact is rather surprising but notice that the standard deviation is quite high for small domain. 
% On \ref{fig:VALIDATION_Nd_2}(right), we can examine the vertical component of the first moment closure term. 
% It is found to be constant for all $N$, but rather inaccurate for coarse grids. 
% Which makes sens since the first moment results from a local calculation of the stress over a droplet volume, unlike the other quantities which results from the averaged center of mass velocity of a droplet. 

% As we have shown, the quantities presented converge for a number of droplets equivalent to $N = 4$ and $\delta = 25$. 
% Thus, we validate our simulation in space, i.e. we made sure that our domain were wide enough to minimize the influence of the periodicity on our results, and in mesh definition. 
% Nevertheless, at it is the number of realization that matter when carrying a particular average, it is interesting to look at the duration of the simulation.











\bibliography{Bib/bib_bulles.bib}



\end{document}

\documentclass[12pt]{My_preprint}

\usetikzlibrary{arrows.meta,
                chains,
                positioning,
                shapes.geometric}
%%%%%%%%%%%%%%%%%%%%%%%%%%%%%%%%%%%%%%%%%%%%%%%%%%%%%%%%%%%%%%%%%%%%%%%%%%%%%%%
\newcommand{\size}{0.22\textwidth}
\newcommand{\avg}[1]{\left<#1\right>}
\renewcommand{\avg}[1]{\left<#1\right>}
\newcommand{\condavg}[1]{\left<#1 | \mathscr{C}_1\right>}
\newcommand{\Exp}[1]{\overline{\overline{#1}}}
\newcommand{\davg}[1]{\left<#1\right>_d}
\newcommand{\Iavg}[1]{\left<#1\right>_I}
\newcommand{\pavg}[1]{\avg{\delta_\alpha #1}}
% \newcommand{\pnavg}[1]{n\left<#1\right>_p}

\newcommand{\avgcond}[1]{\overline{#1}}
% \renewcommand{\avgcond}[1]{\left{#1}}
\newcommand{\kavg}[1]{\avgcond{#1}^k}
\newcommand{\cavg}[1]{\avgcond{#1}^c}
\newcommand{\Tavg}[1]{\avgcond{#1}^T}
\newcommand{\Xavg}[1]{\avgcond{#1}^X}
\newcommand{\TXavg}[1]{\Tavg{\Xavg{#1}}}
\newcommand{\pnnavg}[1]{\avgcond{#1}^{p}}
\newcommand{\pnavg}[1]{n_p\pnnavg{#1}}
\newcommand{\oneavg}[1]{\avgcond{#1}^1}
\newcommand{\twoavg}[1]{\avgcond{#1}^2}
\newcommand{\smallavg}[2]{\avgcond{#1}^{#2}}
\newcommand{\sym}[1]{\left(#1\right)^{\text{Sym}}}
\newcommand{\CC}{\mathscr{C}}
\newcommand{\PP}{\mathscr{P}}
\newcommand{\nstavg}[1]{\overline{#1}^\text{nst}}
\newcommand{\nstrelavg}[1]{\nstavg{#1}_\text{rel}}
\newcommand{\mavg}[1]{\left<#1\right>_m}
\newcommand{\gavg}[2][\gamma]{\left<#2\right>_{#1}}
\newcommand{\partials}[1]{\partial_{i_1}\partial_{i_2}\ldots\partial{i_{#1}}}
\newcommand{\partialp}[2]{ \prod_{m=#1}^{#2} \partial_{i_m}}
\newcommand{\hatpartialp}[2]{ \prod_{m=#1}^{#2} \hat{\partial}_{j_m}}
\newcommand{\hatpartialpi}[2]{ \prod_{m=#1}^{#2} \hat{\partial}_{i_m}}
\newcommand{\pri}[2]{ \prod_{m=#1}^{#2} r_{i_m}}
\newcommand{\prj}[2]{ \prod_{m=#1}^{#2} r_{j_m}}
\newcommand{\nablab}{\mathbf{\nabla}}
\newcommand{\nablabh}{\nablab}
\newcommand{\nablabhI}{\nablab_{||}}
\newcommand{\ddt}{\frac{d}{d t}}
\newcommand{\pddt}{\frac{\partial}{\partial t}}
\renewcommand{\pddt}{\partial_t}
\newcommand{\norm}[1]{\hat{#1}}
\newcommand{\Jump}[1]{\llbracket #1 \rrbracket \cdot \textbf{n} }

%%% Utiliser pour les commentaires
\newcommand{\JL}[1]{\color{red}#1\color{black}}
\newcommand{\SP}[1]{\color{green}#1\color{black}}
\newcommand{\tb}[1]{\color{blue}#1\color{black}}
% \renewcommand{\alpha}{}
\renewcommand{\JL}[1]{}
% \renewcommand{\tb}[1]{}
% \renewcommand{\sigma}{\bm{\sigma}}

\renewcommand{\size}[1]{0.3\textwidth}
\newcommand{\expo}[2][n]{\frac{(-1)^#1}{#1!} \partialp{1}{#1} \pavg{\int_{\Omega_\alpha} \pri{1}{#1}#2 d\Omega}}
\newcommand{\expoU}[2][n]{\frac{(-1)^#1}{#1!} \partialp{1}{#1} \pavg{\textbf{u}_\alpha\int_{\Omega_\alpha} \pri{1}{#1}#2 d\Omega}}
\newcommand{\expoS}[2][n]{\frac{(-1)^#1}{#1!} \partialp{1}{#1} \pavg{\int_{\Sigma_\alpha} \pri{1}{#1}#2 d\Sigma}}


% \newcommand{\numref}[1]{\ref{#1}}
\renewcommand{\ref}[1]{\autoref{#1}}

%%%%%%%%%%%%%%%%%%%%%%%%%%%%%%% Title & Author %%%%%%%%%%%%%%%%%%%%%%%%%%%%%%%%


\title{Reynolds stress and inertial interphase force in buoyant emulsion.}


\author[1,2]{Nicolas Fintzi}
\author[1]{Jean-Lou Pierson}
\author[2]{Stephane Popinet}
\affil[1]{IFP Energies Nouvelles, Rond-point de l’changeur de Solaize, 69360 Solaize}
\affil[2]{Sorbonne Université, Institut Jean le Rond ∂’Alembert, 4 place Jussieu, 75252 PARIS CEDEX 05, France}
\normalmarginpar
\begin{document}

\maketitle

\begin{abstract}
    % We performed direct numerical simulation (DNS) of mono disperse rising droplets. 
    This article present a numerical analysis of the pseudo turbulence Reynolds stress and interphase drag force in inertial emulsion. 
    In the view of feeding averaged navier stokes model it is crucial to obtain those terms. 
    In this work new kind of algorithm which prevent coalescence between VOF tracer has been developed.
    This enables us to perform statistically steady state simulations of mono-disperse buoyant droplets for various \textit{Galileo} number $Ga = 5 \rightarrow 10$, volume fraction $\phi =1\% \rightarrow 10\%$ and viscosity ratio $\lambda = 0.1,1$. 
    By the mean of the recent concept of the nearest particle statistics \citep{zhang2021stress} we analyze meticulously the averaged flow flied around a particle and the averaged particles relative position.s 
    From this analysis we propose physical interpretation to explain the form of the closure terms.
    In this study we focus on the interphase drag and velocity fluctuation. 
\end{abstract}
\tableofcontents
\listoftodos
\begin{itemize}
    \item \tb{Kinetic energy spectra ?}
\end{itemize}



\section{Introduction}
\section{Introduction}

In the context of Euler-Euler models, this study focuses on the momentum exchange term in the averaged Navier-Stokes equations, or the drag force density term. 
The objective of this work is to quantify the drag force in a ``homogeneous'' and steady-state configuration.
This means that we consider only a constant and uniform relative motion between the dispersed and continuous phase and a constant mean volume fraction of droplets. 

In this configuration, a drag force correlation should predict the mean drag force density as a function of the local \textit{Reynolds} number based on the particle size, denoted as $Re$, and the volume fraction of the dispersed phase, $\phi$. Additionally, since we are considering droplets with a constant density ratio ($\zeta$) but arbitrary viscosity, the model must account for the viscosity ratio, $\lambda$. 
To the authors' knowledge, no existing model in the literature comprehensively describes the dependence of drag force density in terms of $Re$, $\phi$, and $\lambda$. 
Therefore, we aim to address this gap by proposing a new model incorporating these dependencies.
For isolated spherical droplets, \citet{magnaudet1997forces} proposed a semi-empirical formula based on the well-known models of \citet{schiller1933} for solid particles and \citet{mei1994} for spherical bubbles, which accurately predict the drag force as a function of both the \textit{Reynolds} number and the viscosity ratio.
Meanwhile, \citet{richardson1954} introduced robust empirical relations to predict the sedimentation velocity of solid particles in terms of volume fraction $\phi$ at low \textit{Reynolds} numbers. 
The Richardson-Zaki laws can be written as $u_p/u_0 = (1-\phi)^n$, where $u_p$ is the dispersed phase velocity and $u_0$ is the terminal velocity of an isolated particle. 
The exponent $n$ is an empirical constant known as the Richardson-Zaki exponent. 
\citet{ishii1979drag} extended the values of the Richardson-Zaki exponent for spherical bubbles in the Stokes regime. 
More recently, \citet{kramer2019improvement} proposed an accurate relation to model the Richardson-Zaki exponent for solid particles, for arbitrary \textit{Reynolds} numbers. 

In this work, we utilized the findings of these authors and combined their models to construct a robust model applicable at intermediate values of $\lambda$.
Since \citet{magnaudet1997forces} introduced a model valid for intermediate $\lambda$ at $\phi = 0$, and \citet{ishii1979drag} proposed corrections to the Richardson-Zaki exponent for droplets at $Re \approx 0$, the real challenge here is to accurately model the $\phi$ dependency of the drag force for finite Reynolds numbers, at intermediate $\lambda$.

The organization of this study is as follows:
In \ref{sec:methodology_drag} we present the theoretical and numerical methodology adopted to express the drag force in terms of the buoyancy force or the relative velocity of the dispersed and continuous phase within the averaged Navier-Stokes equations framework.
In \ref{sec:model_drag} we outline the theoretical development that leads to the creation of our new drag force model. 
Finally in \ref{sec:validation_drag} we demonstrate the validity of our model by comparing its predictions, for the drag force and sedimentation velocity, with the DNS results at intermediates values of $\lambda$.

%%%%%%%%%%%%%%%%%%%%%%%%%%%%%%%%%%%%%%%%%%%%%%%%
%%%%%%%%%%%%%%%%%%%%%%%%%%%%%%%%%%%%%%%%%%%%%%%%
\section{Theoretical context}


\subsection{Simulations setup}
Objective of this section :
\begin{itemize}
    \item Introduce the dimensionless parameters.
    \item Present the physical parameters of some industrial processes to locate our problematic. 
    \item Introduce the dimensionless parameters range investigated in this study.
    \item Present the tri-periodic box within which we add droplets in vof 
\end{itemize}
We investigate the dynamic of homogeneous mono-disperse emulsion subject to buoyancy forces. 
Both the dispersed (resp. continuous) phase is considered as Newtonian fluid defined by viscosity $\mu_d$ (resp. $\mu_c$), and density $\rho_d$ ($\mu_c$).
Throughout this work, the indices $d$ and $c$ indicate properties belonging to the dispersed and continuous phase, respectively. 
The interface between both fluid is considered as infinitely thin and deprived of any impurities so that it can only be described with the surface tension coefficient $\sigma$. 
In this work the density, viscosity of each phase, and surface tension coefficient, will be considered constant during  the time of a numerical experiment.
In dimensionless form the physics of the flow is described by $4$ dimensionless parameters: 
The viscosity and density ratio, $\mu_r = \mu_d / \mu_c$ and $\rho_r = \rho_d / \rho_c$, respectively. 
The \textit{Galileo} number, 
\begin{equation*}
    Ga =\sqrt{\rho_c(\rho_c - \rho_d) g d^3} / \mu_c
\end{equation*}
where $a$ is the equivalent radius of the droplets.
And the \textit{Bond} number, 
\begin{equation*}
    Bo =\frac{(\rho_c - \rho_d) g d^2}{\sigma}
\end{equation*}
with $g$ the gravity constant. 
The \textit{Galileo} number measure the influence of the buoyancy forces against the viscous forces.
Whereas the \textit{Bond} number evaluate the ratio between buoyancy and capillary forces. 
In addition to these $4$ parameters we introduce the number of particles, $N_b$, and the dispersed phase volume fraction $\phi_d$ which fully describe the topology of a finite domain of the flow. 


\begin{table}[h!]
    \centering
    \caption{Dimensionless parameter range investigated in this work.}
    \begin{tabular}{ccccccc}\hline
        $Ga$&$Bo$&$\phi$&$\mu_r$&$\rho_r$&$N_b$&$t^*_{end}$\\ \hline\hline
        $5\rightarrow 100$&$1$&$1\% \rightarrow 20\%$&$0.1 \& 1$&$1.111$&$125$&$500$\\ \hline
    \end{tabular}
    \label{tab:parameters}
\end{table}
\JL{il faut choisir entre $\mu_r$ et $\lambda$... Pr ailleurs je pense qu'l y a une coquille dans ta definition de $\mu_r$.}

We wish to investigate the moderate inertial emulsion regime with quasi spherical droplets. \todo{gives real parameters values compared to experiment ? Yes. tu peux le faire facilement en prenant par exmeple des gouttes allant de 500 microns a 2 mm. Cela ajoutera un peu de poids au choix des parametres. Par ailleurs se pose la question de lancer quelques cas à plus bas nombre de Bond (car experimentalement ils le seront) voir si cela change les resultats. De meme, pq ne pas lancer quelques cas ou les gouttes sont moins visqueuses que le fluide environnant ?}
Thus, the \textit{Bond} number must be low enough to obtain nearly spherical drops, and the viscosity and density ratio must approach the oil/water situation. 
It will be shown in the next few sections that a $Bo =1$ gives reasonable results. 
Additionally, for a statistical convergence reason explained in \ref{sec:preliminary} we choose $N_b = 125$. 
Therefore, in the following we will keep the dimensionless parameters within the ranges depicted in \ref{tab:parameters}.
In summary, we investigated $6$ \textit{Galileo} number $Ga = 5,10,25,50,75,100$, four volume fractions $\phi = 0.01,0.05,0.1,0.15,0.2$, and two viscosity ratios $\mu_r =0.1,1$. 
This makes a total of $60$ representative simulations of $125$ droplets which is expensive. 
Therefore, in the next sections we develop on our numerical strategy to make efficient multi VOF simulations. 

\tb{It is clear that at $\phi>0.1$ the mono disperse simulation aren't physicaly realistic since coalesence should arise at those volume fraction.}

To mimic infinitely large homogeneous emulsions we consider a tri periodic cubic domain of length $L$, within which, both phases are subject to the incompressible Navier Stokes equations with the corresponding boundaries conditions. 



\section{Numerical methodology}



%\subsection{Simulation set-up}
\subsection{The Basilisk flow solver}
Objective : 
\begin{itemize}
    %\item Present the tri-periodic box within which we add droplets in vof 
    \item Present : Governing equations under single fluid formulation 
    \item Present in details the numerical scheme. 
    \item Finally, present why the numerical coalescence is a problem
\end{itemize}


The one fluid formulation of the mass and momentum conservation equation reads as,
\begin{align}
    \label{eq:dt_rho}
    \pddt \rho
    + \nablabh\cdot(\rho\textbf{u})
    &= 0, \\
    \pddt (\rho \textbf{u})
    +\nablabh \cdot (\rho  \textbf{u} \textbf{u} - \bm{\sigma})
    &= 
    (\avg{\rho} - \rho)\textbf{g}
    + \textbf{f}_\sigma\delta(\textbf{x} -\textbf{x}_I)
    \label{eq:dt_urho}
\end{align}
respectively.  
\todo{mettre sous forme adim \citep{hidman2023assessing} ? pas besoin}
In \ref{eq:dt_urho} we introduced : the velocity of the fluid $\textbf{u}$,  the Newtonian stress  tensor $\bm{\sigma} = -p \textbf{I} + 2\mu \textbf{S}$ with $p$ the pressure fields and $\textbf{S}$ the symmetrical part of the velocity gradient.
The mean averaged density is defined as $\avg{\rho} = \rho_d\phi + (1-\phi) \rho_f$. 
Then, $\avg{\rho} \textbf{g}$ act as an additional body forces in \ref{eq:dt_urho}  to enforce a net-zero momentum source over the computational domain \citep{bunner2002dynamics} \JL{a presenter apres dans problem statement}. 
The last term of \ref{eq:dt_urho} account for surface tension force  times the Dirac delta function $\delta(\textbf{x}-\textbf{x}_I)$ since it acts at the interface position $\textbf{x}_I$. 
Additionally, we solve a transport equation for the approximation of the phase indicator function, i.e. the color function $\alpha_d$, it reads,
\begin{equation}
    \pddt \alpha_d + \textbf{u} \cdot \nablab \alpha_d =0.
    \label{eq:dt_alpha}
\end{equation}
The equations are discretized with a centered scheme on a Staggered multigrid solver from \url{http://basilisk.fr}. 
The two-phase flow solver use the volume of fluid method. 
The reconstruction of the interfaces is computed using Piecewise Linear Interface Calculation. 
For a more detailed description of the solver we refer the reader to \citet{popinet2018numerical}. \todo{Make a more precise description such as in \cite{hidman2023assessing} ? Oui et je ne suis pas sur que la reference de Stephane soit la bonne. Tu peux aussi t'inspirer des papiers de Luc Deike and Co}

In the next section we will see that this equation will be sightly be modified to avoid coalesce.

It is known that with the VOF method, we experience premature coalesce events between droplets \citet[Appendix B]{innocenti2020direct}.
Which can be problematic to gather datas,\citet{loisy2017buoyancy}
However, if we wish to reach a quasi steady state regime we must be able to conserve a specific population of droplets within time, in our case a mono-disperse configuration. 
To tackle this issue we used a specific algorithm which prevent coalescence. 



\subsection{The no-coalescence algorithm}
Objectives 
\begin{itemize}
    \item Presents the bibliography. 
    \item Introduce mani's algorithm
    \item explain step by step the algorithm
\end{itemize}


The key feature of numerical simulations is the use of a whole new algorithm which prevent numerical coalesce of droplets to occur.
First, the reader can find the described source code at \href{https://basilisk.fr/sandbox/fintzin/no-coalesce.h}{no-coalesce.h}. 
In the following we describe the global ideas and principles, then we dive into a step by step explanation of the algorithm. 
But first some worlds on the already existing algorithm is in order.

In previous studies several methods have been used to avoid coalescence. 
The first one is to increase artificially the surface tension coefficient locally such as it is done in \citet{hidman2023assessing}.
When using a level set method to track the  phase indicator function some authors developed a multiple marker level-set method to prevent coalesence, see \citet{balcazar2015multiple}. 
Similarly, for VOF tracer some author used a multi-vof approach. 
In a recent study \citet{zhang2021direct} used one VOF tracer per bubble in his simulation which prevent coalesence and allows to track bubbles independently. 
However, this approach is quite expensive as it requires solving \ref{eq:dt_alpha} for each drop. 
Instead, we adopt the methodology of \citet{karnakov2022computing} which consider a constant number of VOF tracer with respect to the number of dorplets. 
We then adoped another methodology to track bubbles independently that we adapted inside the \texttt{Basilisk} code. 

The adaptation of  \citet{karnakov2022computing} within the basilisk code has been carried out by \citet{mani2021numerical}, which developed an algorithm to prevents the adjacent droplets to have similar VOF tracers using the least VOF tracer as possible by allowing different drops to be included within the same VOF tracer.
Specifically we define $N(t)$ VOF tracer labeled as $\alpha_d^i$ for $i =0,\ldots,N$ where $N(t)$ is dependent on time since it is function of the particles positions.  
The only requirement is that the adjacent droplets at a given time $t$, have different VOF tracer to prevent coalescence. 


\begin{figure}[h!]
    \centering
    % \begin{tikzpicture}[scale=0.1,
    %     node distance = 4mm and 6mm,
    %   start chain = going below,
    %   base/.style = {draw, thick, fill=gray!10, align=center, 
    %                  inner xsep=2mm, inner ysep=2mm},
    %   rect/.style = {base},
    %   elli/.style = {ellipse, base},
    %   circ/.style = {circle, fill=graye!10, minimum size=12pt},
    %   diam/.style = {diamond, base, aspect=1.5},
    %   line/.style = {draw, rounded corners, -Stealth, semithick},
    % ]
    % % Place nodes
    % \begin{scope}[nodes = {on chain, join=by line}]
    % \node [rect, rounded corners=10pt] (step1) {start};
    % \node [rect] (step2) {(1) Apply tag function \\ on vof field $\alpha_d^i$};
    % \node [base] (step3) {(2) Check for any adjacent drops \\ that have the same $\alpha_d^i$};
    % \node [rect] (step5) {(3) Change drops vof tracers for all\\ adjacent drops.};
    % \node [diam] (step7) {$i < N(t)$};
    % \node [rect, rounded corners=10pt] (step8) {stop};
    % \end{scope}
    % % \node [rect, left=of step3] (step9) {$i = i+1$};
    % % Draw edges
    % % \path[line] (step7) -| (step9);
    % % \path[line] (step9) |- (step2);
    % %
    % \path       (step7) -- node [right,near start]{False}    (step8);
    % \node [right=of step5] {
        % };
        % \end{tikzpicture}
    \includegraphics[width = 0.6\textwidth]{image/VOF2.png}
    \caption{
    %     Simplified flowchar of the \texttt{no-coalescence.h} algorithm.
    % $\{\alpha_d^i;i\in\mathbb{N}^{*+}\}$ represent the list of VOF tracer currently used. 
    % $N(t)$ is the total number of tracer at time $t$. 
    (b) Interface of the droplets colored by the value of $\alpha_d^i$ at $t_g =100$.
    }
    \label{fig:diagram}
\end{figure}


The simplified workflow of the algorithm follows these three steps : 
\begin{enumerate}
    \item We first identify the different topologies, i.e. the droplets, within a single tracer $\alpha_d^i$. 
    This is done by using another Basilisk feature which assign to a scalar field a different value to each topological object such as a drop (see \href{http://basilisk.fr/src/tag.h}{tag.h})\todo{biblio ?}
    \item Then we identify the droplets/tag which are different and too close to one another.
    The distance criterion is fixed to a cube of 5 mesh cells length.  
    \item Lastly we assign a new VOF tracer for each required droplets among the VOF tracer that are not already adjacent to the droplets. 
    If no VOF tracer is available we create a new one for the droplets.
    \todo{Maybe more details ?}
\end{enumerate}
This algorithm is executed at each simulation time step. 
Besides having $N$ VOF tracer require some modifications to the previously mentioned governing equations. 
Especially, instead of solving \ref{eq:dt_alpha}  we solve $N$ transport equation for each $\alpha_d^i$\todo{Is it true ?}.
But also, we compute the surface tension force as the sum of the contribution from each VOF tracer, namely, 
\begin{equation}
    \textbf{f}_\sigma \delta(\textbf{x}-\textbf{x}_I)
    = \sum_{i=0}^{N(t)} \sigma \kappa \nablab \alpha_d^i
\end{equation} 
where $\kappa_i$ is the approximate curvature of $\alpha_d^i$. 
In the 2D simulations (not presented here) we do not used more than 4 VOF tracers for hundreds of droplets during a simulation. 
This is a consequence  of the four color map theorem derived from topological arguments.
In the three-dimensional were we simulated hundreds of droplets we observed the creation of $6$ VOF tracer in the long run of the simulations.
A picture of the colored VOF is shown \ref{fig:diagram} (b) where only 3 VOF tracer is needed.
Indeed, the four color map theorem isn't valid anymore therefore the increase of VOF tracer isn't surprising anymore. 

Overall, we used an optimized multi-VOF method which allows us to compute massive DNS with approximately $6$ VOF tracers. 

\subsection{Ensemble average approximation}

Objectives : 
\begin{itemize}
    \item Present How we compute the particle properties. 
    \item Explain How we approximate the ensemble average in the numerical calculations
    \item Focus on the drag force term and on the velocity fluctuation. 
\end{itemize}

Following \citet{du2022analysis} we consider ergodicity at all time of the numerical experiment.
Thus, the ensemble average of a quantity $X$ can be approximated by a spatial average $\Xavg{X}$ and a time average $\Tavg{X}$ such that $\avg{X} = \Xavg{\Tavg{X}}=\Tavg{\Xavg{X}}$.
Consequently, the ensemble average of a numerical field, $X$, is taken through space and time such that,
\begin{equation}
    \avg{X}
    = \Tavg{\Xavg{X}}
    = \frac{1}{ t_{end} - t_0}\int_{t_0}^{t_{end}} 
    \Xavg{X}(t) dt
\end{equation}
where, 
\begin{equation}
    \Xavg{X}(t)
    = \frac{1}{L^3}\int 
    X(\textbf{x},t) d\textbf{x}
\end{equation}
where $L$ is the length of the cubic domain.
$t_0$ and $t_{end}$ is the starting time of sampling and the time duration of the simulation, respectively.
In practice, we take $t_0$ such that the simulation reach a statistically steady regime for $t>t_0$.  
Both $t_{end} $ and $t_0$ are given in \ref{ap:A} after several validations studies. 

To compute the continuous phase averaged quantities such as \ref{eq:def_uuc} we proceed as such,
\begin{equation}
    \phi_c \bm{\sigma}^{\text{Re}}_c /\rho_c
    % = \Tavg{\Xavg{\chi_c \textbf{u}_c' \textbf{u}_c'}}
    = \Tavg{\Xavg{\chi_c (\textbf{u}_c^0 -\textbf{u}_c ) (\textbf{u}_c^0 -\textbf{u}_c)}}
    = \Tavg{\Xavg{\chi_c \textbf{u}_c^0 \textbf{u}_c^0}}
    -  \phi_c  \textbf{u}_c \textbf{u}_c.
    \label{eq:def_uuc_num} 
\end{equation}
where the indicator funciton $\chi_c$ must be understood as its approximation in the DNS, i.e the color function $1 - \alpha_d$. 
Consequently, \ref{eq:def_uuc_num} indicate that we must take the average of the product of the velocities, and then we retrieve the mean velocities' product. 
Additionally,  note that the Reynolds stress can be decomposed by such as : 
\begin{align*}
    \phi_c \bm{\sigma}^{\text{Re}}_c /\rho_c
    &= 
    \Tavg{\Xavg{\chi_c (\textbf{u}_c^0 -\Xavg{\chi_c\textbf{u}_c^0} / \Xavg{\chi_c} ) (\textbf{u}_c^0 -\Xavg{\chi_c\textbf{u}_c^0} / \Xavg{\chi_c} )}}\\
    &+ \Tavg{\Xavg{\chi_c} (\Xavg{\chi_c\textbf{u}_c^0} / \Xavg{\chi_c} - \textbf{u}_c ) (\Xavg{\chi_c\textbf{u}_c^0} / \Xavg{\chi_c} - \textbf{u}_c)}\\
\end{align*}
where the first term is the space fluctuation relative to the instantaneous mean velocity of the fluid $\Xavg{\chi_c\textbf{u}_c^0} / \Xavg{\chi_c}$, and the second is the time fluctuation of the instantaneous mean velocity of the fluid. 

Similar expression can be derived for the particular phase by integrating the property over the volume of the particle which is done throught the use of the \texttt{tag.h} function, then we average on each particle at all time.





\section{Preliminary tests}
\label{sec:preliminary}
\section{Time convergence and mesh independence study}

\begin{itemize}
    \item How do I compute the closure terms with the average methodo (Mehrabahdi)
    \item Mesh and size independence study 
    \begin{itemize}
        \item Rising velocity
        \item Velocity fluctuation / Reynolds stress convergence 
        \item PFP time convergence
    \end{itemize}
    \item Loisy single rising drop
    \item Show that eulerian two-point correlation does indeed decay in a tri periodic domain 
\end{itemize}

\subsubsection{Fixed array of bubbles}

\begin{figure}[h!]
    \centering
    \includegraphics[height = 0.3\textwidth]{image/VALIDATION2.0/Loisy/Re.pdf}
    \caption{Time evolution of the Reynolds number based on the drift velocity $U = \avg{\textbf{u}}_d - \avg{\textbf{u}}$ with $\phi = 0.1256$ ,$\rho_r =\mu_r =10$ and $Ga = 29.9$.}
\end{figure}


\subsubsection{Free array of droplets}



The first plot concerns the continuous averaged quantities
\begin{figure}[h!]
    \centering
    % \includegraphics[height = 0.3\textwidth]{image/VALIDATION2.0/fCA/Re.pdf}
    \includegraphics[height = 0.3\textwidth]{image/VALIDATION2.0/fCA/Recum.pdf}
    \includegraphics[height = 0.3\textwidth]{image/VALIDATION2.0/fCA/Tcum.pdf}
    \caption{(left) Cumulative mean of the volume averaged Reynolds number along the simulation time based on the drift velocity $U = \avg{\textbf{u}}_d - \avg{\textbf{u}}_c$, with $\phi = 0.1$, $\rho_r = 1.11$, $ \mu_r =0.1$ and $Ga = 29.9$ and $N_b = 125$.
    (right) Cumulative mean of the fluid Reynolds stress tesor. }
\end{figure}

This second plot focus on the particular averaged quantities : 
\begin{figure}[h!]
    \centering
    % \includegraphics[height = 0.3\textwidth]{image/VALIDATION2.0/fCA/Re.pdf}
    \includegraphics[height = 0.3\textwidth]{image/VALIDATION2.0/fPA/Tcum.pdf}
    \includegraphics[height = 0.3\textwidth]{image/VALIDATION2.0/fPA/Scum.pdf}
    \caption{(left) Cumulative mean of the volume averaged granular temperature along the simulation time based on the drift velocity $U = \avg{\textbf{u}}_d - \avg{\textbf{u}}_c$, with $\phi = 0.1$, $\rho_r = 1.11$, $ \mu_r =0.1$ and $Ga = 29.9$ and $N_b = 125$.
    (right) Cumulative mean of the dimensionless particle-fluid-particle stress horizontal component tensor. }
\end{figure}
\begin{figure}[h!]
    \centering
    \includegraphics[height = 0.3cm]{image/VALIDATION2.0/fPA/PFPxx_x3.pdf}
    \includegraphics[height = 0.3cm]{image/VALIDATION2.0/fPA/PFPyy_x3.pdf}
    \caption{
    (right) Cumulative mean of the dimensionless particle-fluid-particle stress horizontal component tensor decomposed into $\Sigma_{pfp}= \Sigma_1+\Sigma_2$ in terms of the number of realization $\omega$. }
\end{figure}

%%%%%%%%%%%%%%%%%%%%%%%%%%%%%%%%%%%%%%%%%%%%%%%%
%%%%%%%%%%%%%%%%%%%%%%%%%%%%%%%%%%%%%%%%%%%%%%%%
\section{Droplets configurations flowlines and deformations }







\subsection{Nearest particles arrangements}
\begin{itemize}
    \item Problematic : "How the particles are arranged relative to each other"
    \item Show : "How to compute the Radial and azimuthal probability density function : $P_{nst}(r)$  and $P_{nst}(\theta)$"
    \item  Conclusion on $P_{nst}(\theta)$ : "We observe that the particles pair becomes oriented with increasing $Ga$ and decreasing volume fraction.
    \item  Conclusion on $P_{nst}(r)$ : "We observe that the particles pair becomes randomly arranged for high $Ga$ but in average they are rather spaced from each other" 
\end{itemize}
\tb{Je me demande si cette section est vraiment utile .... car elle n'apport pas d'explication supplementaire a la drag force ni aux fluctuations, c'est peux être mieux de garder ca pour l'article qui traîte des interactions }

In this section we wish to investigate the particle arrangements and clustering effects. 
As in the previous section we treat this problem with the nearest particle statistics.
We introduce the probability density function defined such that $P_{nst}(\textbf{r})d\textbf{r}$ is the probable number of nearest neighboring particle at a disatnce $\textbf{r}$ from a test particle at $\textbf{r} = 0$. 
Let $\textbf{x}^i(t,\CC)$ and $\textbf{x}^j(\CC,t)$ be the position vector of the particle $i$ and $j$ function of the initial configuration of the flow $\CC$ and the time $t$. 
Then, the nearest pair probability density function is defined such as, 
\begin{equation}
    P_{nst}(\textbf{x},\textbf{r},t)= 
    \int \sum_{i}\delta(\textbf{x}-\textbf{x}^i(\CC,t))
    \sum_{j\neq i}\delta(\textbf{x}+\textbf{r}-\textbf{x}^j(\CC,t)) 
    % \delta(t+a-t_c^{ij}(\CC,t)) 
    h_{ij}(\CC,t) d\mathscr{P} 
    \label{eq:P_nstij}
\end{equation}
with $h_{ij} = 1$ if the particle $j$ is one of the nearest neighbor from the particle $i$, and $h_{ij} = 0$ if it is not. 
Since we model a statistically homogeneous configuration within space and time, the variable \textbf{x} and $t$ are of no interest, thus $P_{nst}(\textbf{x},\textbf{r},t) = P_{nst}(\textbf{r})$. 
\begin{figure}
    \centering
    \begin{tikzpicture}
        \node at (0,0){ \includegraphics[height=0.3\textwidth]{image/HOMOGENEOUS/fDrop/Pnst_theta_mu_r_1_0_Ga_10.pdf} };
        \node at (0.4\textwidth,0){ \includegraphics[height=0.3\textwidth]{image/HOMOGENEOUS/fDrop/Pnst_theta_mu_r_0_1_Ga_10.pdf} };
        \node at (0,-0.3\textwidth){ \includegraphics[height=0.3\textwidth]{image/HOMOGENEOUS/fDrop/Pnst_theta_mu_r_1_0_Ga_75.pdf} };
        \node at (0.4\textwidth,-0.3\textwidth){ \includegraphics[height=0.3\textwidth]{image/HOMOGENEOUS/fDrop/Pnst_theta_mu_r_0_1_Ga_100.pdf} };
        % \node at (0,-0.6\textwidth){ \includegraphics[height=0.3\textwidth]{image/HOMOGENEOUS/fDrop/Pnst_theta_mu_r_1_0_Ga_100.pdf} };
        % \node at (0.4\textwidth,-0.6\textwidth){ \includegraphics[height=0.3\textwidth]{image/HOMOGENEOUS/fDrop/Pnst_theta_mu_r_0_1_Ga_100.pdf} };
    \end{tikzpicture}
    \caption{Probability density function of the nearest particles : $P_{nst}(\theta)$ for different $Ga$ and $\lambda$. 
    Increasing $Ga$ from top to bottom, (left) $\lambda = 1$ (right) $\lambda = 10$. 
    The symbols correspond to different volume fraction ($\bullet$) $\phi = 1\%$, ($\blacktriangle$) $\phi = 5\%$, ($\blacksquare$) $\phi = 10\%$, ($\blacklozenge$) $\phi = 15\%$ and ($\blacktriangleright$) $\phi = 20\%$.
    (dashed lines) empirical formulas }
    \label{fig:P_nst_theta}
\end{figure}
By using polar coordinate such that $d \textbf{r} = r^2 \sin \phi dr d\phi d\theta$ we can further reduce the PDF to the only consideration of the angular dependency $\theta$ or the distance dependency $r$. 
These reduced p.d.f can be computed as follow, 
\begin{align*}
    P_{nst}(r) 
    &= \int_{-\pi/2}^{\pi/2}\int_{0}^{2\theta} P_{nst}(\textbf{x},\textbf{r},t) \sin \theta  d\phi d\theta\\
    P_{nst}(\theta)
    &= \int_{0}^{\infty}\int_{0}^{2\theta} P_{nst}(\textbf{x},\textbf{r},t) r^2  dr d\phi
\end{align*}
\todo{Check if those formulas are true}
Then $P_{nst}(\theta)$ is the probability that the nearest neighbor of a test particle is inclined at an angle $\theta$ relative to the flow direction. 
We observe that the particles pair becomes oriented with increasing $Ga$ and decreasing volume fraction

On \ref{fig:P_nst_theta} we observe that the particles pair becomes oriented with increasing $Ga$ and decreasing volume fraction.
Indeed, we observe a clear peak of $P_{nst}(\theta)$ at $\theta = \frac{\pi}{2}$. 
It seems that this tendency was also reported for spherical bubble in air-water system \citet{bunner2003effect}. 
Additionally, from \ref{fig:P_nst_theta} we can say that the viscosity ratio $\lambda$ seem to prevent the alignment/clustering of particles denoted by the slightly low peak for $\lambda =10$. 
\todo[inline]{Compart the Orientation with bubbly and solid flows \citet{roghair2011drag}}

\begin{figure}
    \centering
    \begin{tikzpicture}
        \node at (0,0){ \includegraphics[height=0.3\textwidth]{image/HOMOGENEOUS/fDrop/Pnst_r_mu_r_1_0_PHI_1.pdf} };
        \node at (0.4\textwidth,0){ \includegraphics[height=0.3\textwidth]{image/HOMOGENEOUS/fDrop/Pnst_r_mu_r_0_1_PHI_1.pdf} };
        \node at (0,-0.3\textwidth){ \includegraphics[height=0.3\textwidth]{image/HOMOGENEOUS/fDrop/Pnst_r_mu_r_1_0_PHI_10.pdf} };
        \node at (0.4\textwidth,-0.3\textwidth){ \includegraphics[height=0.3\textwidth]{image/HOMOGENEOUS/fDrop/Pnst_r_mu_r_0_1_PHI_10.pdf} };
        \node at (0,-0.6\textwidth){ \includegraphics[height=0.3\textwidth]{image/HOMOGENEOUS/fDrop/Pnst_r_mu_r_1_0_PHI_20.pdf} };
        \node at (0.4\textwidth,-0.6\textwidth){ \includegraphics[height=0.3\textwidth]{image/HOMOGENEOUS/fDrop/Pnst_r_mu_r_0_1_PHI_20.pdf} };
    \end{tikzpicture}
    \caption{Radial probability density function : $P_{nst}(r)$ for different $\phi$ and $\lambda$. 
    Increasing $\phi$ from top to bottom, (left) $\lambda = 1$ (right) $\lambda = 10$. 
    The symbols correspond to different Galileo number ($\bullet$) $Ga = 10$, ($\blacktriangle$) $Ga = 25$, ($\blacksquare$) $Ga = 50$, ($\blacklozenge$) $Ga = 75$ and ($\blacktriangleright$) $Ga = 100$.
    (dashed lines) Theoretical formula \ref{eq:P_nst_r}}
    \label{fig:P_nst_r}
\end{figure}
Note that for solid spherical particle in the dilute regime a theoretical formula for $P_{nst}(r)$ can be found assuming completely random distribution and no interactions nor overlap between particles \citep{zhang2021ensemble}, it reads, 
\begin{equation*}
    P_\text{nst}^\text{th}(r) = n_p e^{-4 \pi n_p (r^3 - d^3)/3}.
    \label{eq:P_nst_r}
\end{equation*}
It is evident that all the distribution presented \ref{fig:P_nst_r} have a peak at $r > 1$ where the theoretical formula  predict a peak at $r=1$. 
This is obviously due to the fact that we are in presence of particles interaction which tends to repulse the particles from each others and therefore to shift the distribution to the left. 
What is more interesting is that for $\lambda = 1$ at low volume fraction and high \textit{Galileo} we are able to recover approximately the random particle distribution $P_\text{nst}^\text{th}$ with our numerical results. 
Whereas for $\lambda = 10$ the particle are relatively maintained far from  each other as depicted \ref{fig:P_nst_r}(right). 
We can stipulate that for high viscosity ratio the particles have a tendency to generate more particle fluid mediated interaction as demonstrated by the flow lines \ref{fig:Stream}.


\subsection{Nearest-particle average fluid velocity}
Objectives : 
\begin{itemize}
    \item Problematic "How to analyse the flow around a particle in average"
    \item First : present the averaged the nearest particles' statistics method. And how to compute the nearest averaged velocity fields $\nstavg{\textbf{u}}$.
    \item Present the flowlines and show that for $\phi = 5 \rightarrow 20\%$ we observe that a vertical symmetry arise.
    \item Explain how this field it is related to the velocity fluctuation with \ref{eq:def_uu}
    \item Conclude that these velocity fields represent the PWFs since it represent the mean wakes \citep{du2022analysis}.  
    \item Additionally, some comment can be made regarding the shape of the particle thanks to the contour lines. 
    \item Approach these flow fields by analytical solution of potential flow to obtain an analytical solution for teh reyolds stress. 
\end{itemize}

Presently, we wish to investigate the  averaged flow structure around a fluid particle.
To obtain such a field we make use of the nearest particle statistics recently introduced by \citet{zhang2021stress}. 
We introduce $\nstavg{\textbf{u}}(\textbf{x},\textbf{r})$ as the velocity fields at \textbf{x} knowing there is a particles center of mass located at \textbf{r}.
Additionally, this particle is the nearest particle among all to the point \textbf{x}.  
Formally, this conditional average can be written as, 
\begin{equation}
    \nstavg{\textbf{u}}(\textbf{x},\textbf{r})=\frac{1}{P_{nst}(\textbf{x},\textbf{r})} 
    \int \textbf{u}(\textbf{x},\CC,t) 
    \sum_{\alpha}\delta(\textbf{x}+\textbf{r}-\textbf{x}^\alpha(\CC,t)) h_{\alpha}(\CC,\textbf{x},t) d\mathscr{P} 
    \label{eq:q_nst_avg}
\end{equation}
where $P_{nst}(\textbf{x},\textbf{r})$ is defined as,  
\begin{equation}
    P_{nst}(\textbf{x},\textbf{r})= 
    \int
    \sum_{\alpha}\delta(\textbf{x}+\textbf{r}-\textbf{x}^\alpha(\CC,t)) 
    h_\alpha(\CC,\textbf{x},t) d\mathscr{P}. 
    \label{eq:P_nsti}
\end{equation}
which is the probability of finding a particle center of mass at a distance \textbf{r} from the point \textbf{x} knowing that this particle is the nearest neighbor to the points \textbf{x}. 
The function $h_\alpha$ is defined such that, $h_\alpha = 1/N^p$ if $\alpha$ is the nearest particle to \textbf{x} and $0$ if not, where $N^p$ is the total number of nearest neighbor.
Indeed, the point \textbf{x} at mid-distance from two particles posses two nearest neighbors by definition, thus $N(\textbf{x},\CC,t) = 2$ in this case. 

\todo[inline]{Include the numerical computaiton of $\nstavg{\textbf{u}}$.  }

\ref{fig:Stream} shows the streamline of the field $\nstavg{\textbf{u}}(\textbf{x},\textbf{r})$ for three volume fractions. 
We clearly observe the induced wake of the particles centered at the origin. 

\begin{figure}
    \centering
    \begin{tikzpicture}
        \node (img) at (0,0)  {\includegraphics[height=0.4\textwidth]{image/VALIDATION2.0/Stream/Stream_PHI_20_Ga_10_l_1.pdf}};
        \node (img) at (0.4\textwidth,0)  {\includegraphics[height=0.4\textwidth]{image/VALIDATION2.0/Stream/Stream_PHI_20_Ga_10_l_10.pdf}};
        \node (img) at (0,-0.4\textwidth)  {\includegraphics[height=0.4\textwidth]{image/VALIDATION2.0/Stream/Stream_PHI_20_Ga_100_l_1.pdf}};
        \node (img) at (0.4\textwidth,-0.4\textwidth)  {\includegraphics[height=0.4\textwidth]{image/VALIDATION2.0/Stream/Stream_PHI_20_Ga_100_l_10.pdf}};
        \node (img) at (0,-0.8\textwidth)  {\includegraphics[height=0.4\textwidth]{image/VALIDATION2.0/Stream/Stream_PHI_5_Ga_25_l_10.pdf}};
        \node (img) at (0.4\textwidth,-0.8\textwidth)  {\includegraphics[height=0.4\textwidth]{image/VALIDATION2.0/Stream/Stream_PHI_20_Ga_25_l_10.pdf}};
    \end{tikzpicture}
    \caption{Nearest particle averaged velocity $\nstavg{\textbf{u}}(\textbf{r})$ for  $\phi = 5\%$ and $20\%$.
    Green lines : contour plots of the nearest averaged indicator function $\nstavg{\chi_d}(\textbf{r})$ (it represent the mean shape of the particles)}
    \label{fig:Stream}
\end{figure}
It is evident from these plots that the induced wake is the averaged wake resulting from the averaged translation of the particles. 
And this averaged wake has a tendency to be asymmetrical for low volume fraction and symmetrical for higher ones. 
Additionally, form basic mathematical consideration on the average operators we can demonstrate that :
\begin{multline*}
    \avg{\chi_k \textbf{u}'_k\textbf{u}'_k}(\textbf{x},t)
    + \phi_k \textbf{u}_k\textbf{u}_k
    = \\
    \underbrace{\int (\nstavg{\chi_k \textbf{u}^0_k}  \nstavg{\chi_k \textbf{u}^0_k} / (\nstavg{\chi_k})  P_{nst}(\textbf{x},t,\textbf{r}) d\textbf{r} }_\text{PWFs}
    +\underbrace{\int \nstavg{\chi_k \textbf{v}_k^0\textbf{v}_k^0}  P_{nst}(\textbf{x},t,\textbf{r}) d\textbf{r}}_\text{WIA}
    \label{eq:def_uu}
\end{multline*}
where, $\textbf{v}_k^0  = \textbf{u}_k^0 - \nstavg{\chi_k \textbf{u}^0_k} / \nstavg{\chi_k}$ is the fluctuation of the local velocity relative to the nearest averaged value. 
Consequently, we can decompose the ensemble averaged fluid velocity fluctuations with a first term representing the variance of $\nstavg{\textbf{u}}$ around the mean $\textbf{u}_k$, and a second term representing the variance of $\textbf{u}^0_k$ around the mean  $\nstavg{\textbf{u}}$. 

There are two phenomena causing velocity fluctuations in the liquid:
the agitation resulting from wakes and their collective interactions [wake-induced agitation (WIA)], and the non-turbulent fluctuations resulting from averaged wakes and potential flows around bubbles [potential flow and averaged wake fluctuations (PWFs)].
As a matter of fact in the phase space of $\nstavg{\textbf{u}}(\textbf{r})$ the bubble is fixed at the origin thus we recover the velocity fields representing what is called the PWFs. 
In their study \citet{du2022analysis} carry out transient simulation with fixed particles to recover the PWFs components here we show that a single simulation permit us to recover WIA and PWFs by the mean of the nearest particles' statistics. 

\todo[inline]{make the link with drag force/drag coef  \citet{dandy1989buoyancy}}
\todo[inline]{make the link with velocity fluctuation \citet{almeras2021statistics}}








\section{Interphase drag force}
%Objectives : 
%\begin{itemize}
%    \item Present the rising velocity Vs. phi to demonstrate the relation with $\phi^{1/3}$ \citep{loisy2017buoyancy}
%    \item Discus the common points and differences with bubbles and solid particles. 
%    \item Present a proper definition of the drag force terms such as in \citet{wang2021numerical}. 
%    \item Discus the possible correlation between the shape /arrangement of the particles/flow lines with the rising velocity. \tb{Je ne sais pas trop quoi dire la dessus}
%    \item Show that \citet{rusche2000effect}'s fit for the drag force is not adapted for our case and propose a new one
%    \item All the references for teh Drag force terms are in \citet[chap 8]{morel2015mathematical} or in \citet{ishii2010thermo}
%\end{itemize}
%\todo[inline]{include fits of bubbly flow}

\subsection{Terminal velocity of an isolated spherical drop}
%First of all we want to investigate the dependency of the drift velocity with the volume fraction $\phi$. 
%It is known from several studies on the litterature, especially in \citep[chapter 8]{morel2015mathematical} and \citet[chapter 12]{ishii2010thermo} the the viscosity model for various system can be written generally, as,
%\begin{equation*}
%    \frac{\mu_m}{\mu_c}
%    = \left(
%        1 - \frac{\phi}{\phi_\text{max}}
%    \right)^{-2.5 \phi_\text{max}\mu_\text{eq}}
%\end{equation*}  
%with, $\mu_\text{eq} = \frac{\mu_d + 0.4 \mu_c}{\mu_d+\mu_c}$ and $\phi_\text{max}$ being the volume fraction corresponding to the \textit{maximum packing}. 
%\JL{la viscosite d'une suspension n'a rien a voir avec sa vitesse de chute meme si cela semble etre un argument donne dans la litterature... j'ai enleve tout cela pr l'instant}

In this part, we briefly review the various formulas used to calculate the drag force on a spherical droplet embedded in a steady uniform flow. As demonstrated in Appendix \ref{app:shape} the droplet remains approximatively spherical. This assumption remains valid for the whole range of parameters investigated even in high inertial regime where the maximum deviation from the spherical shape is around $10$ \%. Theoretical predictions for the force on a spherical droplet embedded in a steady uniform flow are limited to the limit of very small and very high Reynolds numbers. We define the drag coefficient, denoted as $C_D$ by the equation $F = \pi / 8 C_D \rho U_0^2 d^2$, where $F$ is the force on the drop, $U_0$ is the imposed velocity. The drag coefficient is a function of the Reynolds number $Re = \rho U_0 d /\mu $ and of is the viscosity ratio$\lambda = \mu _d /\mu _c$. % is the  as $F = C_D$ 
In the Stokes regime ($Re=0$)the drag coeficient is given by the Hadamard-Ribczynski formula


%In this section, we briefly survey the diverse formulas applied to calculate the drag force acting on a spherical droplet within a steady, uniform flow. As corroborated in Appendix \ref{app:shape}, the droplet maintains an approximate spherical shape, a validity sustained across the entire spectrum of investigated parameters, even in the high inertial regime. Theoretical predictions for the force acting on a spherical droplet in a steady uniform flow are confined to scenarios of extremely low and exceptionally high Reynolds numbers.

%We define the drag coefficient, denoted as $C_D$, by the equation F=π8CDρU02d2F = \frac{\pi}{8} C_D \rho U_0^2 d^2F=8π​CD​ρU02​d2, where FFF signifies the force on the droplet, and U0U_0U0​ is the imposed velocity. This coefficient varies with the Reynolds number Re=ρU0dμRe = \frac{\rho U_0 d}{\mu}Re=μρU0​d​ and the viscosity ratio λ=μdμc\lambda = \frac{\mu_d}{\mu_c}λ=μc​μd​​. In the Stokes regime, the drag coefficient adheres to the Hadamard-Ribczynski formula.

%In the Stokes flow regime, the drag coefficient defined as $F = $

%drag force on a spherical drop embedded in a steady uniform flow is given by the Hadamard-Ribczynski formula
%\JL{il faut choisir ton echelle caractersitique de longueur. Soit $a$ le rayon soit le diametre des particules.}
%\begin{equation}
%F_0 = -\pi \mu d U \frac{2+3\lambda}{1+\lambda}
%\end{equation}

\begin{equation}
C_D = \frac{8}{Re} \left( \frac{2+3\lambda}{1+\lambda} \right)
\end{equation}
%\ref{fig:U} shows the drift velocity $U$ divided by the stokes rising velocity of a spherical droplet $U_\text{stokes}$ defined in our notation as \citep{kim2013microhydrodynamics}, 
Balancing the drag force obtained using the previous formula with the buoyancy force one obtained the settling velocity in the Stokes regime
\begin{equation}
    U_0
    = (\rho_c - \rho_d)\frac{g d^2}{6\mu_c}\left(\frac{1+\lambda}{2 + 3\lambda}\right),
\end{equation}
or in dimensionless form 
\begin{equation}
    Re_0
    = \frac{Ar^2}{6}\left(\frac{1+\lambda}{2 + 3\lambda}\right).
\end{equation}
where $Re_0 = \rho_c U_0 d/\mu_c$ is the Reynolds number based on the terminal velocity.
In the opposite regime of very high Reynolds numbers ($Re\gg 1$), the flow outside the droplet can be considered as potential except in a thin boundary layer developing on the bubble surface. \citet{harper1968} have shown that the drag coefficient on a spherical drop is given by 

\begin{equation}
C_D = \frac{48}{Re}\left(1 + \frac{3\lambda}{2}\right).
\label{eq:harper}
\end{equation}
Equation \ref{eq:harper} is the leading order in the expansion performed by \citet{harper1968} in the limit $Re\gg 1$. This equation tends toward Levich formula for the drag on a clean bubble in the limit $\lambda \ll 1$. A detailed investigation perfomed by Dandy et Leal have shown that the oroginal formulation by Harper and mmore became accurate for $Re\geq 200$. The above review show that in the intermediate Reynolds number regimes of the pressent study $ 1 \leq ...$

\begin{equation}
C_D = \frac{24}{Re}(1+0.15Re^{0.687})
\end{equation}

\begin{equation}
C_D = \frac{16}{Re}\left(1+\left[\frac{8}{Re}+\frac{1}{2}\left(1+3.315Re^{-1/2}\right)\right]^{-1}\right)
\end{equation}


\begin{equation}
C_D(Re)Re^2=\frac{4}{3}Ga ^2
\label{}
\end{equation}


%where higher order terms can be found in the original publication of \citet{harper1968}. 

%to leading order as $Re = \rho U_0 d /\mu$ 


In practice the above formula are very limited randge of validity and empirical formulation have to be used for intermediate Reynolds numbers. Prendre la correlation de Rykind et Ryskin



\subsection{Hindered settling velocity}
As an exemple for a suspenison of homogeneous solid spherical particle $\phi_\text{max} = 0.62$ and for deformable particles system it can be approximated to $\phi_\text{max} = 1$. 
From this consideration we can deduce that the drift velocity $U$ is related to the volume fraction with, 
\begin{equation*}
    \frac{U(\phi)}{U_\text{stokes}} = \left\{\begin{tabular}{cc}
        $(1-\phi)^{1.5}$   & bubbles in liquid \\
        $(1-\phi)^{2.25}$   & drops in liquid \\
        $(1-\phi)^{3}$   & drops in gas \\
    \end{tabular}\right.
\end{equation*} 
Additionally, \citet{bunner2003effect} propose a scaling of $\sim (1 - \phi^{1/3})$ for spherical bubbles.
While \citet{ishii1979drag} proposed a $\sim (1 - \phi)^3$ for deformable bubbles. 

In our case, i.e. quasi spherical droplets, we reach a $\sim (1 - \phi^{1/3})$ scaling for $\lambda = 10$ and approximately a $\sim (1 - \phi^{1/2})$ scalings for $\lambda = 1$ .
\begin{figure}[h!]
    \centering
    \includegraphics[height = 0.35\textwidth]{image/HOMOGENEOUS/fCA/UstokesGa_mu_r_1-0.pdf}
    \includegraphics[height = 0.35\textwidth]{image/HOMOGENEOUS/fCA/UstokesGa_mu_r_0-1.pdf}
    \caption{Rising velocity divided by the rising velocity of an equivalent spherical drop in Stokes regime.($\bullet$) $Ga = 5$, ($\blacktriangle$) $Ga = 10$, ($\blacksquare$) $Ga = 25$ , ($\blacklozenge$) $Ga = 50$, ($\blacktriangleright$) $Ga = 75$ and ($\blacktriangleleft$) $Ga = 100$ . 
    The dashed lines are the empirical funtions (left)  
    $U/U_\text{stokes} = 2.72(Ga^{-2.77} - Ga\;10^{-3}) (1 - \phi^{0.45})$
    (right)  $U/U_\text{stokes} = 2.74(Ga^{-2.85} - Ga \;10^{-3}) (1 - \phi^{0.34})$ }
    \label{fig:U}
\end{figure}

The case for which $\lambda = 1$ have a tendency to processes more deformation, as caracterised by their averaged aspect ratio $\chi$ slightly higher than those for which $\lambda = 10$. 
The differences in the volume fraction dependency can be partially explain by this fact. 
Indeed, a  $\sim (1 - \phi^{2/3})$ power law were observed for slightly deformable bubbly flow \cite{zhang2021direct}. 
Thus it is not surprising that at low but finite deformation we found a scalings between $1/3$ and $2/3$. 

A last interesting fact is that at low $Ga$ and $\phi$ we observe a rising velocity $U / U_\text{stokes} > 1$ meaning that the rising velocity is higher than in the stokes isolated case. 
This phenomena has already been observe in \citet{loisy2017buoyancy}. 
This fact was explained to be due to the cumulative effect of the wakes in orderred array of bubbles  which tends to increases their colective velocity. 
Anyhow, since the the limit $Ga \rightarrow 0$ and $\phi \rightarrow 0$ we must recover $U/U_\text{stokes} = 1$ we can be sure that the $\phi$ scaling won't behaves like so.
Therefore it is crucial to point out that these scalings are surely not valid in the limit  $Ga \rightarrow 0$ and $\phi \rightarrow 0$ .

%\subsection{Interphase drag force}




Now we present our results for the drag force term in terms of the Reynolds number. 
\begin{figure}[h!]
    \centering
    
    \includegraphics[height = 0.35\textwidth]{image/HOMOGENEOUS/fCA/Re_mu_r_1-0.pdf}
    \includegraphics[height = 0.35\textwidth]{image/HOMOGENEOUS/fCA/Re_mu_r_0-1.pdf}
    
    \includegraphics[height = 0.35\textwidth]{image/HOMOGENEOUS/fCA/Fstokes_N_5_l_1.pdf}
    \includegraphics[height = 0.35\textwidth]{image/HOMOGENEOUS/fCA/Fstokes_N_5_l_10.pdf}
    \caption{
        (middle) Reynolds number based on the averaged rising velocity.
    (bottom) Ensemble averaged drag force divided by the stokes drag force on spherical droplet of equivalent size.
    The symbols correspond to different volume fraction ($\bullet$) $\phi = 1\%$, ($\blacktriangle$) $\phi = 5\%$, ($\blacksquare$) $\phi = 10\%$, ($\blacklozenge$) $\phi = 15\%$ and ($\blacktriangleright$) $\phi = 20\%$.
    (dashed lines) empirical formulas : extrapolation of  \citet{tenneti2011drag} for solid particles. }
    \label{fig:drag_force}
\end{figure}
\tb{As discussed in previous study that
for spherical bubbles, as the gas fraction increases and the in-
teractions become more important, the bubbles tend to align
themselves in horizontal pairs, whose average rising velocity
is lower than that of an isolated bubbl}

The interphase drag forces applied on the droplets is the buoyancy force $(\rho_d-\rho_c)v_\alpha \textbf{g}$.
The relevant quantity is therefore the dimensionless drag force, that is what is presented in the following \ref{fig:drag_force}. 


In the stokes regime the drag force on a spherical droplet is, 
\begin{equation*}
    \textbf{F}_s
    =\pi \mu_f d (\textbf{u}_p - \textbf{u}_c) \left(\frac{2+3\lambda}{1+\lambda}\right)
\end{equation*}
Let the dimensionless drag force be a function of $\lambda$ $Re$ and $\phi$, expressed such that, 
\begin{equation}
    \textbf{f}_p(Re,\phi,\lambda)
    = 
    f_1^*(Re)
    f_2^*(\phi)
    \textbf{f}_s(\lambda)
\end{equation}
where $f_{1,2}$ are coefficient which limit tends to $1$ at low $\phi$ and low $Re$. 
To determine the $\phi$ dependency we base our study on the following analysis. 
In \cite[chapter 4]{ashgriz2011handbook} they stipulate that the last droplets empirical fit was made in the study of \citet{rusche2000effect} where they performed empirical fits on experimental datas of emulsion. 
Some of which concerned droplets' sedimentation, were they stipulate that, 
\begin{align*}
    f_1^*(\phi) 
    &=1  + C_1 Re^{C_2}\\
    f_2^*(\phi) 
    &= e^{\phi K_1} + \phi K_2
    \label{eq:drag_fit}
\end{align*}
Nevertheless, the coefficients for the droplets doesn't agree with our numerical calculation. 
Instead we remark that the solid particles fit from \citet{rusche2000effect} well fits our results at $\lambda = 1$. 
Therefore, we propose to keep the shape of teh fits and adjuste our coefficient. 

\tb{Dans cette partie je ne sais pas trop quoi faire pour avoir un bon point de départ pour cree une formule empirique, je manque d'idée, notament pour ce qui est de la dépendence en $\lambda$ et pour l'explication physique des tendence observe. }



\section{The Reynolds stress}

Objectives : 
\begin{itemize}
    \item Present the decomposition of the fluid reynolds stress according to isotropic and deviatoric part.
    \item Show the relation between the flowlines graphs and the actual values of the fluid velocity fluctuation.
    \item Compare our case with the fits of  \citet{almeras2019fluctuations} 
    \item Compare the fluctuation with theoretical dev \citet{lance1991turbulence,zhang1994ensemble}. 
    \item Find out that the potential flow solution 
    \item Conclusion : " In low inertia regime $B_{xx} = - 0.2$ and $B_{yy} = 0.4$ for all simulations, find out why" (This is also true for the theoretical solution\citep{lance1991turbulence} this is thus due to the form of teh wake)
\end{itemize}


In this section we analyze the velocity fluctuation within the suspension for both phases. 
We decompose both Reynolds stresses into an isotropic part and deviatoric part such that, 
\begin{align}
    \bm{\sigma}^{\text{Re}}_p &=  \rho_d \phi_d K^*_p(\textbf{u}_p - \textbf{u}_c)\cdot (\textbf{u}_p - \textbf{u}_c) 
    \left(
        \textbf{I}
        +\textbf{B}_p 
    \right)\\
    \bm{\sigma}^{\text{Re}}_c &=  \rho_c \phi_d K_c^*(\textbf{u}_p - \textbf{u}_c)\cdot (\textbf{u}_p - \textbf{u}_c) 
    \left(
        \textbf{I}
        +\textbf{B}_c
    \right)
\end{align}
where the $K^*$ is the dimensionless pseudo-turbulent  kinetic energy, $\textbf{I}$ a unit tensor and $\textbf{B}$ a tensor accounting for the deviation of the Reynolds stress left to determine. 
In the following subsection we present our numerical results for $K^*$ and $\textbf{B}$ for the dispersed and continuous phase. 
\tb{In this section the results are presenetd in terms of Ga because it is easier to show the tendency}

Alternatively we can use the decomposition,
\begin{align}
    \phi_c\bm{\sigma}^{\text{Re}}_c &=  \rho_c \phi_d 
    \left[
        B_1\textbf{I}(\textbf{u}_p - \textbf{u}_c)\cdot (\textbf{u}_p - \textbf{u}_c) 
        +B_2 (\textbf{u}_p - \textbf{u}_c) (\textbf{u}_p - \textbf{u}_c) 
    \right]\\
\end{align}
for which $B_1$ and $B_2$ are scalar coefficient. 
This decomposition has the advantage that in the potential flow limit theoretical solution are known such that  $B_1 = -0.15$ and  $B_2 = 0.05$ \citet{zhang1994averaged,wang2021numerical}. 
Lastly, \citet{lance1991turbulence} give also a theoretical solution for $\bm{\sigma}^{Re}_c$ based on the potential flow solution around a spherical particle in translation. 


\subsection{Continuous phase}
\todo{try \citet{almeras2021statistics} fits}
\todo{also check \citet{almeras2019fluctuations} results}
\tb{might be good to plot the velocity field to examine where does the fluctuation arise (pseudo-turbe or turbe)}
Look at \citep{wang2021numerical} and Mahra.. 2015 



The fluid averaged kinetic energy can be easily scaled on the numerical results shown \ref{fig:Tf_Bf}(left).
It is shown that for $\lambda= 1$ we observe non monotonic behavior while at $\lambda= 10$ it is decreasing.
Except for the cases $\phi = 0.01$ where we cannot be sure if it is due to statistical error or real physics. 
\begin{figure}[h!]
    \centering
    \includegraphics[height=0.3\textwidth]{image/HOMOGENEOUS/fCA/Tf_l_1.pdf}
    \includegraphics[height=0.3\textwidth]{image/HOMOGENEOUS/fCA/Bf_l_1.pdf}

    \includegraphics[height=0.3\textwidth]{image/HOMOGENEOUS/fCA/Tf_l_10.pdf}
    \includegraphics[height=0.3\textwidth]{image/HOMOGENEOUS/fCA/Bf_l_10.pdf}
    \caption{(left) Dimensionless turbulent kinetic energy in terms of the \textit{Galileo} number for different $\phi$. (dots) Numerical simulations, (dashed line) empirical formula \ref{eq:Tf_scaling}.
    The symbols correspond to different volume fraction ($\bullet$) $\phi = 1\%$, ($\blacktriangle$) $\phi = 5\%$, ($\blacksquare$) $\phi = 10\%$, ($\blacklozenge$) $\phi = 15\%$ and ($\blacktriangleright$) $\phi = 20\%$.
    (right) deviatoric part of the Reynolds stress, ($- \cdot -$)  vertical components, $B_{yy}$, ($- -$)  horizontal components, $B_{xx} = B_{zz}$.}
    \label{fig:Tf_Bf}
\end{figure}
\subsection{Dispersed phase}

\tb{Maybe include velocity fluctuation and compare to : Lingxin2021 }

\tb{Include Gaussian distribution of bubbles !!! ! ! ! }

Now let's focus on the particular averaged Reynolds stress tensor.
\ref{fig:Talpha_Balpha} shows that the granular temperature behavior is quite similar from the continuous averaged turbulent kinetic energy.
\begin{figure}[h!]
    \centering
    \includegraphics[height=0.3\textwidth]{image/HOMOGENEOUS/fPA/Talpha_l_1.pdf}
    \includegraphics[height=0.3\textwidth]{image/HOMOGENEOUS/fPA/Balpha_l_1.pdf}

    \includegraphics[height=0.3\textwidth]{image/HOMOGENEOUS/fPA/Talpha_l_10.pdf}
    \includegraphics[height=0.3\textwidth]{image/HOMOGENEOUS/fPA/Balpha_l_10.pdf}
    \caption{(left) Dimensionless turbulent kinetic energy $K_p$ in terms of the \textit{Galileo} number for different $\phi$. 
    The symbols correspond to different volume fraction ($\bullet$) $\phi = 1\%$, ($\blacktriangle$) $\phi = 5\%$, ($\blacksquare$) $\phi = 10\%$, ($\blacklozenge$) $\phi = 15\%$ and ($\blacktriangleright$) $\phi = 20\%$.
    (right) deviatoric part of the Reynolds stress, ($- \cdot -$)  vertical components, $B_{yy}$, ($- -$)  horizontal components, $B_{xx} = B_{zz}$.}
    \label{fig:Talpha_Balpha}
\end{figure}

% \subsection{The particle-fluid-particle Stress}
\begin{figure}
    \centering
    \includegraphics[height=0.3\textwidth]{image/HOMOGENEOUS/fPA/PFPxx.pdf}
    \includegraphics[height=0.3\textwidth]{image/HOMOGENEOUS/fPA/PFPyy.pdf}
    \caption{(left) Normalized PFP stress }
\end{figure}

Open the discussion on the fact that we yet still not have enought data for the pfp stress
% 
\subsection{Higher moments closure : Stresslet ? ? }


\begin{equation}
    \ddt \mathcal{P}_\alpha
    = \int_{\Omega_\alpha} \left(
        \rho_2  \textbf{w}_2 \textbf{w}_2 
        + \textbf{r} \div \mathbf{T}_2
    \right) d\Omega
\end{equation}
\begin{equation}
    \ddt \mathcal{P}_\alpha
    = \int_{\Omega_\alpha} \left(
        \rho_2  \textbf{w}_2 \textbf{w}_2 
        + \mathbf{T}_2
    \right) d\Omega
+ \int_{\Sigma}\textbf{rT}_2 \cdot \textbf{n}_2 d\Sigma
\end{equation}
surface jump condition : 
\begin{equation}
    - \int_{\Sigma_\alpha} 
    \sigma \textbf{I}_{||}
d\Sigma
= \int_{\Sigma}\textbf{rT}_1 \cdot \textbf{n}_1 d\Sigma
+ \int_{\Sigma}\textbf{rT}_2 \cdot \textbf{n}_2 d\Sigma
\end{equation}
\begin{equation}
    \ddt \mathcal{P}_\alpha
    = \int_{\Omega_\alpha} \left(
        \rho_2  \textbf{w}_2 \textbf{w}_2 
        - \mathbf{T}_2
    \right) d\Omega
    - \int_{\Sigma_\alpha} 
        \sigma \textbf{I}_{||}
    d\Sigma
    + \int_{\Sigma}\textbf{rT}_1 \cdot \textbf{n}_2 d\Sigma
\end{equation}
\begin{figure}[h!]
    \centering
    \includegraphics[height=0.3\textwidth]{image/HOMOGENEOUS/fPA/Sxx.pdf}
    \includegraphics[height=0.3\textwidth]{image/HOMOGENEOUS/fPA/Syy.pdf}
\end{figure}
\begin{equation}
    \int_{\Sigma}\textbf{r}_1 (\textbf{T}_1  -  \oneavg{\textbf{T}_1}) \cdot \textbf{n}_2 d\Sigma
    = \int_{\Omega_\alpha} \left(
        \mathbf{T}_2
        - \rho_2  \textbf{w}_2 \textbf{w}_2 
    \right) d\Omega
     + \int_{\Sigma_\alpha} 
     \sigma \textbf{I}_{||}
    d\Sigma
    + \ddt \mathcal{P}_\alpha
    -\int_{\Sigma}\textbf{r}_1   \oneavg{\textbf{T}_1} \cdot \textbf{n}_2 d\Sigma
\end{equation}
Assuming that $\textbf{T}_k = - p_k \textbf{I} + \mu_k \mathbb{S}_k  = -p_k \textbf{I} + \mu_k (\grad \textbf{u}_k + \grad \textbf{u}_k^T )$ 
\begin{multline}
    \int_{\Sigma}\textbf{r}_1 (\textbf{T}_1  -  \oneavg{\textbf{T}_1}) \cdot \textbf{n}_2 d\Sigma
    \\= \dot{ \mathcal{P}_\alpha}
    + \int_{\Omega_\alpha} \left(
        \mu_2\mathbb{S}_2
        - \rho_2  \textbf{w}_2 \textbf{w}_2 
    \right) d\Omega
     + \int_{\Sigma_\alpha} 
     \sigma \textbf{I}_{||}
    d\Sigma
    - \textbf{I} \int_{\Omega_\alpha} p_2 d\Omega
    + v_\alpha \oneavg{p}
    - \mu_1 v_\alpha \oneavg{\mathbb S}
\end{multline}
\begin{multline}
    \int_{\Sigma}\textbf{r}_1 (\textbf{T}_1  -  \oneavg{\textbf{T}_1}) \cdot \textbf{n}_2 d\Sigma
    \\= \dot{ \mathcal{P}_\alpha}
    + \int_{\Omega_\alpha} \left(
        \mu_2\mathbb{S}_2
        - \rho_2  \textbf{w}_2 \textbf{w}_2 
    \right) d\Omega
     + \int_{\Sigma_\alpha} 
     \sigma \textbf{I}_{||}
    d\Sigma
    - \textbf{I} \int_{\Omega_\alpha} p_2 d\Omega
    + v_\alpha \oneavg{p}
    - \mu_1 v_\alpha \oneavg{\mathbb S}
\end{multline}
Keeping only  the deviatoric part of the equations
\begin{equation}
    \textbf{S}_\alpha
    = \dot{ \mathcal{P}_\alpha}
    + \int_{\Omega_\alpha} \left(
        \mu_2\mathbb{S}_2
        - \rho_2  \textbf{w}_2 \textbf{w}_2 + w^2_2/3
    \right) d\Omega
     + \int_{\Sigma_\alpha} 
     \sigma \textbf{I}_{||}^{dev}
    d\Sigma
    - \mu_1 v_\alpha \oneavg{\mathbb S}
\end{equation}
For an almost spherical drop in a stokes regime: 
\begin{equation}
    \textbf{S}_\alpha
    = 
    \int_{\Omega_\alpha} \left(
        \mu_2\mathbb{S}_2
    \right) d\Omega
    - \mu_1 v_\alpha \oneavg{\mathbb S}
\end{equation}
 
The original formulas stipulate that :

\todo{This formulation must be chnaged look at \citet[chap 5]{pozrikidis1992boundary}}
\begin{align}
    \label{eq:M_decomposition}
    S^\alpha_{ij} 
    &= \frac{1}{2}  \int_{\Sigma_\alpha} \left[
        r_i(T_{jk}n_k)
        + (T_{ik}n_k)r_j
        \right]d\Sigma
        - \frac{\delta_{ij}}{3}\int_{\Sigma_\alpha} \left[
            r_l(T_{lk}n_k)
    \right]d\Sigma
    -\mu_1 \int_{\Sigma_{\alpha}}
    (u_in_j^2 + u_jn_i^2)d\Sigma
    \\
    T^\alpha_{ij}
    &= \frac{1}{2}  \int_{\Sigma_\alpha} \left[
        r_i(T_{jk}n_k)
        - (T_{ik}n_k)r_j
    \right]d\Sigma \nonumber
\end{align}

Inter changing the sens of the normal vector we can rewrite the last term of \ref{eq:M_decomposition} such as :
\begin{align}
    - \mu_1 \int_{\Sigma_{\alpha}}
    (u_in_j + u_jn_i)d\Sigma
    = 
    + \mu_1 \int_{\Sigma_{\alpha}}
    (u_i n^1_j + u_jn^1_i)d\Sigma
\end{align}
Using Gauss divergence theorem 
\begin{align}
    - \mu_1 \int_{\Sigma_{\alpha}}
    (u_in_j + u_jn_i)d\Sigma
    = 
    \mu_1 \int_{\Omega_1}
    (\grad u_i + \grad u_j )d\Omega
\end{align}
which correspond the average of the continuous phase stress in homogeneous medium.
Besides in Stokes flows :
\begin{equation}
    \frac{1}{2}  \int_{\Sigma_\alpha} \left[
        r_i(T_{jk}n_k)
        + (T_{ik}n_k)r_j
        \right]d\Sigma
    =
    \frac{1}{2}  \int_{\Sigma_\alpha} \left[
        T_{ij}
        + T_{ji}
        \right]d\Sigma
\end{equation}
And so on for the trace, Thus we can rewrite the stresslet as : 


%%%%%%%%%%%%%%%%%%%%%%%%%%%%%%%%%%%%%%%%%%%%%%%%
%%%%%%%%%%%%%%%%%%%%%%%%%%%%%%%%%%%%%%%%%%%%%%%%

\section{Conclusion}
\section{Conclusion}

In this study, we have demonstrated the relationship between buoyancy forces and the mean drag force density on droplets, derived from the ensemble-averaged momentum equations. 
In the homogeneous regime, we linked the sedimentation velocity of the dispersed phase with volume fraction and Reynolds number through the Richardson-Zaki relation in Stokes flow, both for solid particle and bubbles.

We then extended the Richardson-Zaki framework to viscous droplets of arbitrary viscosity ratio, valid for arbitrary $Re$ and $\phi$. 
In the dilute limit, our correlation is validated by the literature, as it employs the well-known Schiller-Neuman, and Mei drag force coefficient, for solid particle and spherical bubbles, respectively. 
For solid particles ($\lambda \to \infty$), our model converges to the recent model of \citet{kramer2019improvement}, which is an improvement of the Richardson-Zaki relation valid for arbitrary $\lambda$ and $Re$.

To validate the intermediate $\lambda$ regime, we performed DNS of buoyant emulsions in a tri-periodic box. 
The results show good agreement, with our model accurately capturing the $\phi$-dependence for a given $\lambda$ and $Re$.

The main advantage of the formulation provided by \ref{eq:C_d_finalRe} is its robustness, since Richardson-Zaki relation have been shown to be valid at very high volume fraction  ($\phi \approx 0.5$) and Reynolds number, while in the dilute limit Schiller-Neuman, and Mei drag force coefficient are proven to be accurate up to $Re = 800$. 
Thus, we provided, a robust drag force coefficient that can directly be used in Euler-Euler framework for simulation of emulsion of arbitrary viscosity ratio. 

\section*{Acknowledgement}

\appendix
\section{Statistical convergence and mesh independence studies}
\label{ap:A}
\section{Mesh definition convergence}
\label{ap:convergence}

% \subsection{Statistical and mesh independence study}

In the aim of providing accurate closure terms it is of primary importance to verify the well convergence of the mean quantities, by varying the mesh definition. 
While, the domain size and duration of simulation have been validated in \citet{fintzi2024buoyancy}.
To tackle this problem we carried out four simulation with 160 rising droplets with different mesh definition. 
The flow parameters read as,  
\begin{align*}
    \lambda = 10,
    && \zeta = 1.11,
    && Bo = 0.5,
    && Ga = 80,
    && \phi = 0.05,
\end{align*}

\ref{fig:Re_and_Tc}(left) display the cumulative mean of the vertical Reynolds number based on the drift velocity, namely,
\begin{equation}
    \widetilde{Re}(t)
    = \frac{\rho_f d}{\mu_f t}\int_{t_0}^{t_0+t} \left(\Xavg{\textbf{u}^0_d} -  \Xavg{\textbf{u}_f^0}\right)dt'
\end{equation}
where $t_0$ is the starting sampling time. 
We reach mesh independent results for $d/\Delta \geq 25$ in agreements with the recent studies of \citet{hidman2023assessing} \citet{zhang2021direct} for low inertial bubbly flows.
Also, it is seen that $\widetilde{Re}$ reaches a constant values from $t^* = 50$ to the end of the simulation. 
\begin{figure}[h!]
    \centering
    \includegraphics[height = 0.35\textwidth]{image/HOMOGENEOUS_final/CA/Re.pdf}
    \caption{(left) Cumulative mean of the volume averaged Reynolds number along the simulation time based on the drift velocity $U = \textbf{u}_p - \textbf{u}_c$, with $\phi = 0.1$, $\rho_r = 1.11$, $ \mu_r =0.1$ and $Ga = 29.9$ and $N_b = 125$.
    (right) Cumulative mean of the fluid Reynolds stress tesor. }
    \label{fig:Re_and_Tc}
\end{figure}

% The well convergence of the rising velocity doesn't guarantee a statistical nor a mesh convergence for finer quantities such as the pseudo-turbulent kinetic energy. 
% Therefore, we provide on \ref{fig:UpUp} (left) the running average of the fluid phase pseudo-turbulent energy. 
% Similarly, \ref{fig:UpUp} (right) represent the particle center of mass pseudo-turbulent kinetic energy. 
% \begin{figure}[h!]
%     \centering
%     \includegraphics[height = 0.35\textwidth]{image/VALIDATION2.0/fCA/Tcum.pdf}
%     \includegraphics[height = 0.35\textwidth]{image/VALIDATION2.0/fPA/Tcum.pdf}
%     \caption{(left) Cumulative mean of the volume averaged granular temperature along the simulation time based on the drift velocity $U = \textbf{u}_p - \textbf{u}_c$, with $\phi = 0.1$, $\rho_r = 1.11$, $ \mu_r =0.1$ and $Ga = 29.9$ and $N_b = 125$.
%     (right) Cumulative mean of the dimensionless particle-fluid-particle stress horizontal component tensor. }
%     \label{fig:UpUp}
% \end{figure}
% Both figure exhibit well converged data. 
% Interestingly, $\widetilde{K}_c$ and $\widetilde{K}_\alpha$ reach a constant value at $t^* = 200$ which is four time greater than for $\widetilde{Re}$.


% \tb{Cite and compare to Berner and \citet{bunner2002dynamics} which found that Nb > 12 is sufficient \citet{roghair2011drag}}
% Now, let's investigate the required number of droplets per domain, $N_b$, and the minimum definition of cells per diameter of droplets $\delta$.  
% \tb{Include bibliography and expectation here \ldots}
% For this investigation we kept the physical parameters presented in the same section and made a double parametric analysis over $N$ and $\delta$. 
% We carried out simulations for $N = 2, 3, 4, 5, 6, 7$, and for a number of cells $10 <\delta < 40$. 
% In Basilisk the mesh definition is defined by a power of two, consequently depending on the size of the domain (which is fixed to keep a $\phi$ constant) the $\delta$ parameter is fixed at a power of 2 close. 
% \begin{figure}[h!]
%     \centering
%     \includegraphics[height= 0.3\textwidth]{image/VALIDATION/N_and_delta/DUd.pdf}
%     \includegraphics[height= 0.3\textwidth]{image/VALIDATION/N_and_delta/PHI.pdf}
%     \caption{(left) Averaged Reynolds number based on the drift velocity.
%             (right) Dispersed phase volume fraction at the end of each simulation.
%             The text on the side of the points is $\delta$.
%             N correspond to $N = N_b^3$. }
%     \label{fig:VALIDATION_Nd_1}
% \end{figure}
% \ref{fig:VALIDATION_Nd_1}(left), illustrate clearly that the drift velocity is independent of the parameters $N_b$ and $\delta$, for $N >4$. 
% On the other hand, \ref{fig:VALIDATION_Nd_1}(right), show that the volume fraction of the dispersed phase is lower for the low defined grid (red dots), due to a loss of volume during the simulation.
% This doesn't mean that the solver isn't volume conservative. 
% In fact, it is fund to be due to the \href{http://basilisk.fr/sandbox/fintzin/Rising-Suspension/no-coalescence.h}{no-coalescence.h} which generate fragment into the numerical domain, fragment which are deleted in the long run. 
% \begin{figure}[h!]
%     \centering
%     \includegraphics[height= 0.3\textwidth]{image/VALIDATION/N_and_delta/PA_UpUp.pdf}
%     \includegraphics[height= 0.3\textwidth]{image/VALIDATION/N_and_delta/Mh.pdf}
%     \caption{(left) Fluids phase averaged fluctuation tensor.
%             (right) Particular average of the first moment tensor, where $F_g$ is the buoyancy force applied on one droplet. 
%             The numerical values displayed alongside the dots are the number of cells per diameter.}
%     \label{fig:VALIDATION_Nd_2}
% \end{figure}
% Now, let's look at the behavior of more \textit{complicated} closure terms. 
% \ref{fig:VALIDATION_Nd_2}(left) demonstrate that the vertical component of the pseudo turbulent tensor is parameter independent rather early, independently of the grid definition. 
% This fact is rather surprising but notice that the standard deviation is quite high for small domain. 
% On \ref{fig:VALIDATION_Nd_2}(right), we can examine the vertical component of the first moment closure term. 
% It is found to be constant for all $N$, but rather inaccurate for coarse grids. 
% Which makes sens since the first moment results from a local calculation of the stress over a droplet volume, unlike the other quantities which results from the averaged center of mass velocity of a droplet. 

% As we have shown, the quantities presented converge for a number of droplets equivalent to $N = 4$ and $\delta = 25$. 
% Thus, we validate our simulation in space, i.e. we made sure that our domain were wide enough to minimize the influence of the periodicity on our results, and in mesh definition. 
% Nevertheless, at it is the number of realization that matter when carrying a particular average, it is interesting to look at the duration of the simulation.



\bibliography{Bib/bib_bulles.bib}



\end{document}

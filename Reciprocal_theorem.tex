\documentclass[12pt]{My_preprint}
\title{
    General formulation of the statistical averaged momentum exchange term in emulsions, with the reciprocal theorem.
}

\author[1,2]{Nicolas Fintzi and  Jean-Lou Pierson}
\normalmarginpar


\begin{document}

\maketitle

\begin{abstract}
    Based upon the idea of \citet{hinch1977averaged}, .
\end{abstract}


\section{Introduction.}

Let us consider a dispersed flow made of two Newtonian incompressible fluids.
The local velocity field is ruled by the incompressible two-phases Navier-Stokes equations, the single-fluid formulation reads as,
\begin{align}
\div \textbf{u}^0 &= 0, \\
\rho_f (\pddt + \textbf{u}^0 \cdot \grad) \textbf{u}^0
&= 
\div [\chi_f \bm\sigma_f + (\chi_d \bm\sigma_d^0 + \delta_\Gamma \bm\sigma^0_\Gamma)/\zeta]
+ \textbf{g}
+ \kappa (\bm\sigma_f^0 \cdot \textbf{n}) \delta_\Gamma
\label{eq:local_NS}
\end{align}
With $\zeta = \rho_d/\rho_f$, $\kappa = (1 - \zeta)/\zeta$, $\bm\sigma_{f,d}^0 = -p_{f,d}^0 + \mu_{f,d}^0 (\grad \textbf{u}_{f,d}^0+ ^\dagger\grad \textbf{u}_{f,d}^0)$ and $\bm\sigma_\Gamma^0 =\gamma (\bm\delta - \textbf{nn})$. 

\section{The ensemble averaged Navier-Stokes equations and related closure problem.}

The ensemble averaged NS equations is obtained by taking the ensemble average $\avg{\ldots}$ on \ref{eq:local_NS}, namely if $\textbf{u} = \avg{\textbf{u}^0}$ we have, 
\begin{align}
    \div \textbf{u} &= 0, \\
    \rho_f (\pddt + \textbf{u} \cdot \grad) \textbf{u}
    &= 
    \div (\bm\Sigma + \bm\sigma_\text{eff})
    + \textbf{g}
    + \kappa \avg{\bm\sigma_f^0  \cdot \textbf{n} \delta_\Gamma}
    % - \kappa \bm\Sigma \cdot \grad \phi_d
\end{align}
with, 
\begin{align}
    \bm\sigma_\text{eff}
    &= -\avg{\textbf{u}'\textbf{u}'}
    - \phi_d (\bm\Sigma + 2\mu_f \textbf{e}_d^r)
    + \phi_d \bm\sigma_d /\zeta
    + \phi_\Gamma \bm\sigma_\Gamma /\zeta\\
    \bm\Sigma 
    &= -p_f \bm\delta
    + \mu_f (\grad \textbf{u} + ^\dagger \grad \textbf{u}). 
\end{align}
where $\textbf{e}_d^r = \textbf{e}_d^0 - (\grad \textbf{u} + ^\dagger \grad \textbf{u})/2$ and $\bm\sigma_f^r = \bm\sigma_f^0 - \bm\Sigma$. 
% We also used the exact relations, 
% \begin{equation}
%     \kappa \avg{\bm\sigma_f \cdot \textbf{n} \delta_\Gamma}
%     + \div \bm\Sigma
%     % = 
%     % \avg{\bm\sigma_f^r  \cdot \textbf{n} \delta_\Gamma}
%     % + \div (\bm\Sigma \phi_f)
%     % + \phi_d \div \Sigma
%     = 
%     \avg{\bm\sigma_f^r  \cdot \textbf{n} \delta_\Gamma}
%     - \bm\Sigma \cdot \grad \phi_d
%     + \div \bm\Sigma
% \end{equation}

Upon using a Taylor expansion on the dispersed phase -or interfacial phase- averaged quantities we arrive at the final form of the single phase flow equation, 
\begin{align}
    \div \textbf{u} &= 0, \\
    \rho_f (\pddt + \textbf{u} \cdot \grad) \textbf{u}
    &= 
    \div (\bm\Sigma
    + \bm\sigma_\text{eff})
    + \textbf{g}
    + \kappa \pSavg{\bm\sigma_f^0  \cdot \textbf{n}}
    % - \kappa \bm\Sigma \cdot \grad \phi_d
\end{align}
with the new definition of the effective stress being, 
\begin{multline}
    \bm\sigma_\text{eff}
    = -\avg{\textbf{u}'\textbf{u}'}
    - \phi_d (\bm\Sigma + 2\mu_f \textbf{e}_d^r)
    + (\phi_d \bm\sigma_d 
    + \phi_\Gamma \bm\sigma_\Gamma) /\zeta 
    % - \kappa \frac{1}{2}\pSavg{(\textbf{r}\bm\sigma_f^r+ \bm\sigma_f^r \textbf{r})  \cdot \textbf{n}}
    - \kappa \pSavg{\textbf{r}\bm\sigma_f^0  \cdot \textbf{n}}
    + \ldots\\
\end{multline}
Using 
\begin{multline}
    \frac{1}{\zeta}\intS{\bm{\sigma}_\Gamma^0}
    +\frac{1}{\zeta}\intO{\bm{\sigma}_d^0}
    % + \intO{(p_f \bm\delta - 2\mu_f \textbf{e}_d^0)}
    % = 
    % \rho_f \intO{(\textbf{w}_d^0\textbf{w}_d^0)}
    % \\
    % - \rho_f\left(\frac{d}{dt}\intO{ \textbf{rw}_d^0} \right)
    % +
    % \frac{1}{\zeta}\intS{ \textbf{r} \bm\sigma_f \cdot \textbf{n}}
    % + \intO{(p_f \bm\delta - 2\mu_f \textbf{e}_d^0)}
    % \\
    = \rho_f \intO{(\textbf{w}_d^0\textbf{w}_d^0)}
    - \rho_f\left(\frac{d}{dt}\intO{ \textbf{rw}_d^0} \right)
    + \frac{1}{\zeta}\intS{ \textbf{r} \bm\sigma_f^0 \cdot \textbf{n}}
    % + \frac{1}{\zeta}\intS{ \textbf{r} \bm\Sigma \cdot \textbf{n}}
    % + \intO{(p_f \bm\delta - 2\mu_f \textbf{e}_d^0)}
\end{multline}
Clearly, $- \kappa + 1 /\zeta = - (1- \zeta)/\zeta + 1/\zeta =1$ thus, 
\begin{multline}
    \bm\sigma_\text{eff}
    = -\avg{\textbf{u}'\textbf{u}'}
    % + \phi_d \bm\sigma_d /\zeta
    % + \phi_\Gamma \bm\sigma_\Gamma /\zeta
    + \rho_f \pOavg{(\textbf{w}_d^0\textbf{w}_d^0)}
    - \rho_f\pavg{\frac{1}{2}\frac{d^2}{dt^2}\intO{ \textbf{rr}}}\\
    + \frac{1}{2}\pSavg{(\textbf{r}\bm\sigma_f^0 + \bm\sigma_f^0\textbf{r} ) \cdot \textbf{n}}
    +\kappa \pSavg{\textbf{r}\times\bm\sigma_f^0  \cdot \textbf{n}}
    - \pOavg{(\bm\Sigma + 2\mu_f \textbf{e}_d^r)}
    + \ldots\\
\end{multline}
Without body toque the angular momentum balance reads, 
\begin{equation}
    \rho_f \ddt \intO{\textbf{r}\times \textbf{w}}
    = 
    \intS{\textbf{r}\times\bm\sigma_f^0 \cdot \textbf{r}}
    + \kappa \intS{\textbf{r}\times\bm\sigma_f^0 \cdot \textbf{r}}
\end{equation}

Conclusion is that: 
\begin{equation}
    \pSavg{\bm\sigma_f^0 \cdot \textbf{n}}
    = P_1 \intS[1]{\bm\sigma_f^{1r} \cdot \textbf{n}}
    + P_1 \intS[1]{\bm\Sigma \cdot \textbf{n}}
\end{equation}

\section{The conditional eq}

The state function $\delta_1$ evolve as, 
Let us consider the function,
\begin{equation}
    \delta_1[\textbf{y},\textbf{w},\FF,t]
    = \sum_i^N 
    \delta(\textbf{x}_i[\FF,t] - \textbf{y})
    \delta(\textbf{u}_i[\FF,t] - \textbf{w})
\end{equation}
which evolve as, 
\begin{equation}
    (\pddt + \textbf{w} \cdot \pddy + \textbf{a}_i \cdot \pddw)\delta_1 = 0 
\end{equation}
Averaging yields,
\begin{equation}
    \pddt P_1  +  \pddy\cdot  \textbf{w} P_1 + \pddw \cdot  \textbf{a}_p P_1 = 0 
\end{equation}
which gives 
\begin{equation}
    \pddt (\delta_1 - P_1)  +  \pddy \cdot \textbf{w} (\delta_1 - P_1) + \pddw\cdot  (\delta_1\textbf{a}_i - \textbf{a}_p P_1) = 0 
\end{equation}
Assuming that the particle does not accelerate, 
\begin{equation}
    \pddt (\delta_1 - P_1)  +  \pddy \cdot \textbf{w} (\delta_1 - P_1) = 0 
\end{equation}

it gives 
\begin{align}
    P_1 \div \textbf{u}^r &= 0, \\
    \rho_f (\pddt P_1 \textbf{u}^r 
    + P_1 \textbf{u}^r \cdot \grad \textbf{u}^r 
    + P_1 \textbf{u} \cdot \grad \textbf{u}^r 
    + P_1 \textbf{u}^r \cdot \grad \textbf{u}
    + \pddy \cdot P_1 \textbf{u}^r \textbf{w})
    = \div \bm\sigma_\text{eff}
    + \avg{(\delta_1 - P_1) \bm\sigma_f^0 \cdot \textbf{n}}
\end{align}

\begin{align}
    \bm\sigma_\text{eff}
    &= 
    \avg{\delta_1 \textbf{u}''\textbf{u}''}
    - P_1 \avg{\textbf{u}'\textbf{u}'}
    + \avg{
        (\delta_1 - P_1)[
            \chi_f \bm\sigma_f^0 
            + (\chi_d \bm\sigma_d^0
            + \delta_\Gamma \bm\sigma_\Gamma^0)/\zeta
        ]
    }\\
    &= 
    - (\rho_f \avg{\delta_1 \textbf{u}''\textbf{u}''}
    - P_1\rho_f \avg{\textbf{u}'\textbf{u}'})
    +P_1 \bm\Sigma^r
    + P_1(\phi_d^1 p_f^1 - \phi_d p_f) 
    + \avg{
        (\delta_1 - P_1)[
            -2 \mu_f \chi_d \textbf{e}_d^0
            + (\chi_d \bm\sigma_d^0
            + \delta_\Gamma \bm\sigma_\Gamma^0)/\zeta
        ]
    }
\end{align}

Anyhow, the conditionally averaged equation might be written without any approximation an for any point in the phase space for which $P_1 \neq 0$ as, 
\begin{align*}
    \div \textbf{u}^r = 0 \\
    \div \bm\Sigma^r
    = 
    \textbf{f}^r
    % - \avg{(\delta_1 - P_1) \bm\sigma_f^0 \cdot \textbf{n}}
    % + \div \bm\sigma^\text{eq}
\end{align*}

\begin{multline}
    P_1 \textbf{f}
    = 
    - \avg{(\delta_1 - P_1) \bm\sigma_f^0 \cdot \textbf{n}}
    + \div \bm\sigma^\text{eq} \\
    +    \rho_f (\pddt P_1 \textbf{u}^r 
    + P_1 \textbf{u}^r \cdot \grad \textbf{u}^r 
    + P_1 \textbf{u} \cdot \grad \textbf{u}^r 
    + P_1 \textbf{u}^r \cdot \grad \textbf{u}
    + \textbf{w} \cdot \pddy P_1 \textbf{u}^r )
\end{multline}
\begin{equation}
    \bm\sigma^\text{eq}
    =
    \rho_f \avg{\delta_1 \textbf{u}''\textbf{u}''}
    - P_1\rho_f \avg{\textbf{u}'\textbf{u}'}
    - P_1(\phi_d^1 p_f^1 - \phi_d p_f) 
    - \avg{
        (\delta_1 - P_1)[
            -2 \mu_f \chi_d \textbf{e}_d^0
            + (\chi_d \bm\sigma_d^0
            + \delta_\Gamma \bm\sigma_\Gamma^0)/\zeta
        ]
    }
\end{equation}

\section{Reciprocal th}
Let $\textbf{u}$, $\bm\sigma$ be a solution of the Stokes equations around an isolated sphere. 
Meaning that 
\begin{align}
    \div \textbf{u} = 0 \\
    \div \bm\sigma = 0 
\end{align}
Then using the reciprocal theorem one obtain, 

Multiplying both solution together yields, 
\begin{equation*}
    \textbf{u}\cdot \div\bm\Sigma^r
    =
    \textbf{u}^r \cdot \div \bm\sigma, 
    + \textbf{u} \cdot  \textbf{f}^r
\end{equation*}
Carrying an integration on the exterior domain of the droplet, and using $\hat{\textbf{u}}_f =\hat{\textbf{u}}_f  - \hat{\textbf{U}} + \hat{\textbf{U}}$ gives, 
\begin{equation}
    \intS{\textbf{U} \cdot  \bm\Sigma^r \cdot \textbf{n}}
    + \intS{(\textbf{u} - \textbf{U})\cdot  \bm\Sigma^r \cdot \textbf{n}}
    = 
    \intS{\textbf{U}^r \cdot  \bm\sigma \cdot \textbf{n}}
    + \intS{(\textbf{u}^r - \textbf{U}^r)\cdot  \bm\sigma \cdot \textbf{n}}
    % \intS{\textbf{u}_f^{(1)}\cdot  \hat{\bm\sigma}_f \cdot \textbf{n}}
    - \intOf{\hat{\textbf{u}}_f\cdot  \textbf{f}^r}
    % \intOf{\textbf{u}_f^{(0)}\cdot ( \div \hat{\bm\sigma}_f)}
    \label{eq:stepone}
\end{equation}

Likewise, Multiplying 
\begin{equation*}
    (\textbf{u} - \textbf{U})\cdot \div\bm\Sigma^r
    =
    (\textbf{u}^r - \textbf{U}^r) \cdot \div \bm\sigma
    + (\textbf{u} - \textbf{U}) \cdot \textbf{f}^r
\end{equation*}
which can be reformulated as, 
\begin{equation*}
    \div [\bm\Sigma^r \cdot (\textbf{u} - \textbf{U})]
    - \bm\Sigma^r : \grad (\textbf{u} - \textbf{U})
    =
    \div [\bm\sigma \cdot (\textbf{u}^r - \textbf{U}^r)], 
    - \bm\sigma : \grad (\textbf{u}^r - \textbf{U}^r)
    + (\textbf{u} - \textbf{U}) \cdot \textbf{f}^r
\end{equation*}
\begin{equation}
    \intS{(\textbf{u} - \textbf{U})\cdot \bm\Sigma^r \cdot \textbf{n}}
    + 2\intO{\textbf{e}^r : \grad \textbf{U}}
    =
    \intS{(\textbf{u}^r - \textbf{U}^r) \cdot  \bm\sigma\cdot \textbf{n} }
    + 2\intO{\textbf{e} : \grad \textbf{U}^r}. 
    - \intO{    (\textbf{u} - \textbf{U}) \cdot \textbf{f}^r}. 
    \label{eq:reciprocal_d}
\end{equation}

Using this expr we arrive at 
\begin{multline}
    \intS{\textbf{U}\cdot  \bm\Sigma^r \cdot \textbf{n}}
    - \lambda 2\intO{\textbf{e}^r : \grad \textbf{U}}
    = 
    \intS{\textbf{U}^r\cdot  \bm\sigma \cdot \textbf{n}}
    -\lambda  2\intO{ \textbf{e} : \grad \textbf{U}^r}. \\
    - \intO{    (\textbf{u} - \textbf{U}) \cdot \textbf{f}^r}. 
    - \intOf{\hat{\textbf{u}}_f\cdot  \textbf{f}^r}
    \label{eq:reciprocal_all}
\end{multline}

\section{The exact expression. }

Since moment expansion still lakes of accuracy we prefer to use instead the method of \citet{pahtz2023general}.
Note the relation 
\begin{equation}
    \int_0^1 
    \frac{\partial}{\partial \lambda}\delta(\textbf{x} - \textbf{x}_i - \lambda \textbf{r})
    d\lambda
    \overset{\text {I.P.P.}}{=} 
    % [\delta(\textbf{x} - \textbf{x}_i - \lambda \textbf{r})]_0^1
    % = 
    \delta(\textbf{x} - \textbf{x}_i - \textbf{r})
    - \delta(\textbf{x} - \textbf{x}_i)
\end{equation}
Additionally, using the relation, 
\begin{equation}
    \frac{\partial}{\partial \lambda}\delta(\textbf{x} - \textbf{x}_i - \lambda \textbf{r})
    % =
    % \frac{\partial (\textbf{r}\lambda)}{\partial \lambda}
    % \cdot 
    % \frac{\partial}{\partial (\textbf{r}\lambda)}\delta(\textbf{x} - \textbf{x}_i + \lambda \textbf{r})
    =
    - \textbf{r}
    \cdot \grad
    \delta(\textbf{x} - \textbf{x}_i - \lambda \textbf{r})
\end{equation}
We then condlude that,
\begin{equation}
    \delta(\textbf{x}  - \textbf{x}_i - \textbf{r})
    =
    \delta(\textbf{x} - \textbf{x}_i)
    -
    \div \textbf{r}  
    \int_0^1 
    \delta(\textbf{x} - \textbf{x}_i - \lambda \textbf{r})
    d\lambda
\end{equation}
Meaning that any surface or volume int may be written, 
\begin{equation}
    \chi_d f[\textbf{x},t,\FF]
    = 
    \delta(\textbf{x} - \textbf{x}_i)
    \intO[i]{f[\textbf{x}_i+\textbf{r},t,\FF]}
    +
    \div 
    \intS[i]{
    \int_0^1 
    f[\textbf{x}_i+\textbf{r},t,\FF]\textbf{r}  
    \delta(\textbf{x} - \textbf{x}_i - \lambda \textbf{r})
    d\lambda}
\end{equation}
\begin{equation}
    \chi_d 
    = 
    \delta(\textbf{x} - \textbf{x}_i)
    \intO[i]{}
    +
    \div 
    \intS[i]{
    \int_0^1 
    \textbf{r}  
    \delta(\textbf{x} - \textbf{x}_i - \lambda \textbf{r})
    d\lambda}
\end{equation}
The same way, 
\begin{equation}
    <\delta(\textbf{x}- \textbf{x}_i),\varphi(\textbf{x})>
    = \varphi(\textbf{x}_i)
\end{equation}
we have 
\begin{equation}
    <\delta(\textbf{x}- \textbf{x}_i - \textbf{r} -(\lambda - 1)\textbf{r}),\varphi(\textbf{x}_i + \textbf{r})>
    = \varphi(\textbf{x} + (1 - \lambda)\textbf{r})
\end{equation}
we deduce taht
\begin{equation}
    \chi_d f[\textbf{x},t,\FF]
    = 
    \delta(\textbf{x} - \textbf{x}_i)
    \intO[i]{f[\textbf{x}_i+\textbf{r},t,\FF]}
    +
    \div 
    \intS[i]{
    \int_0^1 
    f[\textbf{x}_i+\textbf{r},t,\FF]\textbf{r}  
    \delta(\textbf{x} - \textbf{x}_i - \lambda \textbf{r})
    d\lambda}
\end{equation}

\begin{align}
    \div \textbf{u} &= 0, \\
    \rho_f (\pddt + \textbf{u} \cdot \grad) \textbf{u}
    &= 
    \div (\bm\Sigma
    + \bm\sigma_\text{eff})
    + \textbf{g}
    + \kappa \pSavg{\bm\sigma_f^0  \cdot \textbf{n}}
    % - \kappa \bm\Sigma \cdot \grad \phi_d
\end{align}
with the new definition of the effective stress being, 
\begin{multline}
    \bm\sigma_\text{eff}
    = -\avg{\textbf{u}'\textbf{u}'}
    - \phi_d (\bm\Sigma + 2\mu_f \textbf{e}_d^r)
    + (\phi_d \bm\sigma_d 
    + \phi_\Gamma \bm\sigma_\Gamma) /\zeta 
    % - \kappa \frac{1}{2}\pSavg{(\textbf{r}\bm\sigma_f^r+ \bm\sigma_f^r \textbf{r})  \cdot \textbf{n}}
    - \kappa \pSMavg{\textbf{r}\bm\sigma_f^0  \cdot \textbf{n}}
    \\
\end{multline}

more generally we may write, 
\begin{equation}
    \intS[1]{
    \bm\Sigma[\textbf{x}]\cdot \textbf{n}
    }
    - \intS[1]{
    \bm\Sigma[\textbf{x}_1]\cdot \textbf{n}
    }
    = 
    + 
    \intS[1]{
    \grad \bm\Sigma[\textbf{x}_1] \textbf{r}\cdot \textbf{n}
    }
    + \ldots
\end{equation}

Moment of momentum equaiton with 
\begin{multline}
    \ddt \intO[i]{\intL{\textbf{ru}_d^0}}
    = \intO[i]{\intL{
        (\pddt \textbf{ru}_d^0
        + \div \textbf{u}_d^0\textbf{ru}_d^0)
    }}\\
    + \intO[i]{\int_0^1
        \textbf{ru}_d^0(\pddt 
        +  \textbf{u}_d^0 \cdot \grad)
        \delta(\textbf{x} - \textbf{x}_i - \lambda\textbf{r})
    d\lambda}
\end{multline}

the first int is really the moment of momentum but with the moments the seocnd may be written, 
\begin{equation}
    + \intO[i]{\int_0^1
    \textbf{ru}_d^0(\pddt 
    +  \textbf{u}_d^0 \cdot \grad)
    \delta(\textbf{x} - \textbf{x}_i - \lambda\textbf{r})
d\lambda}
=
\end{equation}

\begin{align}
    \pddt \delta(\textbf{x}-\textbf{x}_i - \textbf{r}\lambda)
    + \textbf{u}_i \cdot\grad \delta(\textbf{x}-\textbf{x}_i - \textbf{r}\lambda)
    = 0\\
    \pddt \delta(\textbf{x}-\textbf{x}_i - \textbf{r}\lambda)
    =
    \frac{\partial \textbf{x}_i}{\partial \pddt}
    \frac{\partial \lambda}{\partial \textbf{x}_i}
    \frac{\partial}{\partial\lambda } 
    \delta(\textbf{x}-\textbf{x}_i - \textbf{r}\lambda)\\
    \grad \delta(\textbf{x}-\textbf{x}_i - \textbf{r}\lambda)
    = 
    -  \grad\delta(\textbf{x}-\textbf{x}_i - \textbf{r}\lambda)
\end{align}




\begin{equation}
    \avg{\chi_f \textbf{u}_f^0 }
    = 
    \int \avg{\chi_f \textbf{u}_f^0 \sum_i \delta(\textbf{x}_i - \textbf{x} - \textbf{r}) h_i[\textbf{x},t,\FF]}
    d\textbf{r}
\end{equation}
but also, 
\begin{align}
    \avg{\delta_\Gamma \bm\sigma_f^0\cdot \textbf{n} }
    &= 
    \int \avg{\delta_\Gamma \bm\sigma_f^0\cdot \textbf{n} \sum_i \delta(\textbf{x}_i - \textbf{x} - \textbf{r}) h_i[\textbf{x},t,\FF]}
    d\textbf{r}\\
    &= 
    \int \avg{\sum_i \delta(\textbf{x}_i - \textbf{x} - \textbf{r}) h_i[\textbf{x},t,\FF] \sum_j \delta(|\textbf{x} - \textbf{x}_j|-a)\bm\sigma_f^0\cdot \textbf{n}}
    d\textbf{r}
\end{align}
Which is the stress at the point \textbf{x} knowing we are at a point on the surface and knowing thereis a NNP at \textbf{y}. 
If we consider that this sum is only non zero for $\textbf{x}_i = \textbf{x}_j$ we obtain, 
\begin{align}
    \avg{\delta_\Gamma \bm\sigma_f^0\cdot \textbf{n} }
    &= 
    \int \avg{\sum_i \delta(\textbf{x}_i - \textbf{x} - \textbf{r}) h_i[\textbf{x},t,\FF]  \delta(|\textbf{x} - \textbf{x}_i|)\bm\sigma_f^0\cdot \textbf{n}}
    d\textbf{r}\\
    &= 
    \int \avg{\sum_i \delta(\textbf{x}_i - \textbf{x} - \textbf{r}) h_i[\textbf{x},t,\FF]  \delta(|\textbf{r}|-a)\bm\sigma_f^0\cdot \textbf{n}}
    d\textbf{r}\\
    &= 
    \int_{|\textbf{r}|<a}  
    \avg{\bm\sigma_f^0 \sum_i \delta(\textbf{x}_i - \textbf{x} - \textbf{r}) h_i[\textbf{x},t,\FF] }
    \cdot \textbf{n} d\textbf{r}\\
    &= 
    \int_{|\textbf{r}|<a}  
    \avg{\bm\sigma_f^0 \sum_i \delta(\textbf{x}_i - \textbf{x} - \textbf{r}) h_i[\textbf{x},t,\FF] }
    \cdot \textbf{n} d\textbf{r}\\
    &= 
    \int_{|\textbf{r}|<a}  
    \avg{\bm\sigma_f^0 \sum_i \delta(\textbf{x}_i - \textbf{x} - \textbf{r}) h_i[\textbf{x},t,\FF] }
    \cdot \textbf{n} d\textbf{r}\\
    &= 
    \int_{|\textbf{r}|<a}  
    [- P_\text{nst} p^\text{nst}_f \bm\delta 
    +2\mu_f  (\grad (P_\text{nst}\textbf{u}_f^\text{nst})+^\dagger \grad (P_\text{nst}\textbf{u}_f^\text{nst})) 
    ]\cdot \textbf{n} d\textbf{r}
\end{align}

\bibliography{Bib/bib_bulles.bib}
\appendix

\end{document}

